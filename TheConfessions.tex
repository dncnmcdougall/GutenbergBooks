\documentclass[b5paper,openright,12pt,twoside]{book}

\usepackage{amsmath}
\usepackage{graphicx}
\usepackage[T1]{fontenc}
\usepackage{titlesec}
\usepackage{titling}
\usepackage{lettrine}
\usepackage{yfonts}
\usepackage{expl3}
\usepackage{xparse}

\title{The Confessions of \\Saint Augustine}

\author{Saint Augustine\\\small{Bishop of Hippo}\\\\\tiny{Translated by E. B. Pusey (Edward Bouverie)}}

\date{AD 401}

% \renewcommand{\thechapter}{\Roman{chapter}}
\usepackage{fancyhdr}
\setlength{\headheight}{15pt}

\pagestyle{fancy}
  \renewcommand{\headrulewidth}{0pt} % remove lines as well
  \renewcommand{\footrulewidth}{0pt}


\fancyhf{}
\fancyhf[FLE,FRO]{\thepage}
\fancyhf[FRE]{\textit{\theauthor} }
\fancyhf[FLO]{\textit{ CHAPTER \thechapter.} }


\titlespacing*{\chapter}{0pt}{-50pt}{15pt}
\titleformat{\chapter}[display]{\normalfont\huge\bfseries}{\chaptertitlename\ \thechapter}{20pt}{\Huge}

\setcounter{DefaultLines}{2}
\renewcommand{\DefaultLoversize}{0.15}
\setlength{\DefaultFindent}{0.1em}

%\renewcommand*{\LettrineFont}{\rmdefault}

%\NewDocumentCommand\start{mg}{%
%    \yinipar{#1} \IfNoValueTF{#2}{}{\uppercase{#2}}
%}
\NewDocumentCommand\start{mg}{%
    \IfNoValueTF{#2}{\lettrine{#1}}{\lettrine{#1}{#2}}
}


\begin{document}
\pagestyle{empty}
\frontmatter
\maketitle
\pagenumbering{gobble}
\noindent
Project Gutenberg's The Confessions of Saint Augustine, by Saint Augustine
\\
\noindent
This eBook is for the use of anyone anywhere at no cost and with
almost no restrictions whatsoever.  You may copy it, give it away or
re-use it under the terms of the Project Gutenberg License included
with this eBook or online at www.gutenberg.org
\\
\\
\\
\\
\noindent
Title: The Confessions of Saint Augustine
\\
\\
\noindent
Author: Saint Augustine
\\
\\
\noindent
Translator: E. B. Pusey (Edward Bouverie)
\\
\\
\noindent
AD 401

\tableofcontents

\chapter{BOOK I}
\setcounter{page}{1}
\pagenumbering{arabic}

\start{G}{reat} art Thou, O Lord, and greatly to be praised; great is Thy power,
and Thy wisdom infinite. And Thee would man praise; man, but a particle
of Thy creation; man, that bears about him his mortality, the witness of
his sin, the witness that Thou resistest the proud: yet would man praise
Thee; he, but a particle of Thy creation. Thou awakest us to delight in
Thy praise; for Thou madest us for Thyself, and our heart is restless,
until it repose in Thee. Grant me, Lord, to know and understand which is
first, to call on Thee or to praise Thee? and, again, to know Thee or
to call on Thee? for who can call on Thee, not knowing Thee? for he that
knoweth Thee not, may call on Thee as other than Thou art. Or, is it
rather, that we call on Thee that we may know Thee? but how shall they
call on Him in whom they have not believed? or how shall they believe
without a preacher? and they that seek the Lord shall praise Him: for
they that seek shall find Him, and they that find shall praise Him.
I will seek Thee, Lord, by calling on Thee; and will call on Thee,
believing in Thee; for to us hast Thou been preached. My faith, Lord,
shall call on Thee, which Thou hast given me, wherewith Thou hast
inspired me, through the Incarnation of Thy Son, through the ministry of
the Preacher.

And how shall I call upon my God, my God and Lord, since, when I call
for Him, I shall be calling Him to myself? and what room is there within
me, whither my God can come into me? whither can God come into me, God
who made heaven and earth? is there, indeed, O Lord my God, aught in me
that can contain Thee? do then heaven and earth, which Thou hast made,
and wherein Thou hast made me, contain Thee? or, because nothing which
exists could exist without Thee, doth therefore whatever exists contain
Thee? Since, then, I too exist, why do I seek that Thou shouldest enter
into me, who were not, wert Thou not in me? Why? because I am not gone
down in hell, and yet Thou art there also. For if I go down into hell,
Thou art there. I could not be then, O my God, could not be at all,
wert Thou not in me; or, rather, unless I were in Thee, of whom are all
things, by whom are all things, in whom are all things? Even so, Lord,
even so. Whither do I call Thee, since I am in Thee? or whence canst
Thou enter into me? for whither can I go beyond heaven and earth, that
thence my God should come into me, who hath said, I fill the heaven and
the earth.

Do the heaven and earth then contain Thee, since Thou fillest them? or
dost Thou fill them and yet overflow, since they do not contain Thee?
And whither, when the heaven and the earth are filled, pourest Thou
forth the remainder of Thyself? or hast Thou no need that aught contain
Thee, who containest all things, since what Thou fillest Thou fillest
by containing it? for the vessels which Thou fillest uphold Thee not,
since, though they were broken, Thou wert not poured out. And when Thou
art poured out on us, Thou art not cast down, but Thou upliftest us;
Thou art not dissipated, but Thou gatherest us. But Thou who fillest
all things, fillest Thou them with Thy whole self? or, since all things
cannot contain Thee wholly, do they contain part of Thee? and all at
once the same part? or each its own part, the greater more, the smaller
less? And is, then one part of Thee greater, another less? or, art Thou
wholly every where, while nothing contains Thee wholly?

What art Thou then, my God? what, but the Lord God? For who is Lord
but the Lord? or who is God save our God? Most highest, most good, most
potent, most omnipotent; most merciful, yet most just; most hidden,
yet most present; most beautiful, yet most strong, stable, yet
incomprehensible; unchangeable, yet all-changing; never new, never old;
all-renewing, and bringing age upon the proud, and they know it not;
ever working, ever at rest; still gathering, yet nothing lacking;
supporting, filling, and overspreading; creating, nourishing, and
maturing; seeking, yet having all things. Thou lovest, without passion;
art jealous, without anxiety; repentest, yet grievest not; art angry,
yet serene; changest Thy works, Thy purpose unchanged; receivest again
what Thou findest, yet didst never lose; never in need, yet rejoicing
in gains; never covetous, yet exacting usury. Thou receivest over and
above, that Thou mayest owe; and who hath aught that is not Thine? Thou
payest debts, owing nothing; remittest debts, losing nothing. And what
had I now said, my God, my life, my holy joy? or what saith any man when
he speaks of Thee? Yet woe to him that speaketh not, since mute are even
the most eloquent.

Oh! that I might repose on Thee! Oh! that Thou wouldest enter into my
heart, and inebriate it, that I may forget my ills, and embrace Thee,
my sole good! What art Thou to me? In Thy pity, teach me to utter it.
Or what am I to Thee that Thou demandest my love, and, if I give it not,
art wroth with me, and threatenest me with grievous woes? Is it then a
slight woe to love Thee not? Oh! for Thy mercies' sake, tell me, O Lord
my God, what Thou art unto me. Say unto my soul, I am thy salvation. So
speak, that I may hear. Behold, Lord, my heart is before Thee; open Thou
the ears thereof, and say unto my soul, I am thy salvation. After this
voice let me haste, and take hold on Thee. Hide not Thy face from me.
Let me die--lest I die--only let me see Thy face.

Narrow is the mansion of my soul; enlarge Thou it, that Thou mayest
enter in. It is ruinous; repair Thou it. It has that within which must
offend Thine eyes; I confess and know it. But who shall cleanse it? or
to whom should I cry, save Thee? Lord, cleanse me from my secret faults,
and spare Thy servant from the power of the enemy. I believe, and
therefore do I speak. Lord, Thou knowest. Have I not confessed against
myself my transgressions unto Thee, and Thou, my God, hast forgiven the
iniquity of my heart? I contend not in judgment with Thee, who art the
truth; I fear to deceive myself; lest mine iniquity lie unto itself.
Therefore I contend not in judgment with Thee; for if Thou, Lord,
shouldest mark iniquities, O Lord, who shall abide it?

Yet suffer me to speak unto Thy mercy, me, dust and ashes. Yet suffer me
to speak, since I speak to Thy mercy, and not to scornful man. Thou too,
perhaps, despisest me, yet wilt Thou return and have compassion upon me.
For what would I say, O Lord my God, but that I know not whence I
came into this dying life (shall I call it?) or living death. Then
immediately did the comforts of Thy compassion take me up, as I heard
(for I remember it not) from the parents of my flesh, out of whose
substance Thou didst sometime fashion me. Thus there received me the
comforts of woman's milk. For neither my mother nor my nurses stored
their own breasts for me; but Thou didst bestow the food of my infancy
through them, according to Thine ordinance, whereby Thou distributest
Thy riches through the hidden springs of all things. Thou also gavest me
to desire no more than Thou gavest; and to my nurses willingly to give
me what Thou gavest them. For they, with a heaven-taught affection,
willingly gave me what they abounded with from Thee. For this my good
from them, was good for them. Nor, indeed, from them was it, but through
them; for from Thee, O God, are all good things, and from my God is all
my health. This I since learned, Thou, through these Thy gifts, within
me and without, proclaiming Thyself unto me. For then I knew but to
suck; to repose in what pleased, and cry at what offended my flesh;
nothing more.

Afterwards I began to smile; first in sleep, then waking: for so it
was told me of myself, and I believed it; for we see the like in other
infants, though of myself I remember it not. Thus, little by little, I
became conscious where I was; and to have a wish to express my wishes
to those who could content them, and I could not; for the wishes were
within me, and they without; nor could they by any sense of theirs enter
within my spirit. So I flung about at random limbs and voice, making
the few signs I could, and such as I could, like, though in truth very
little like, what I wished. And when I was not presently obeyed (my
wishes being hurtful or unintelligible), then I was indignant with my
elders for not submitting to me, with those owing me no service, for
not serving me; and avenged myself on them by tears. Such have I learnt
infants to be from observing them; and that I was myself such, they, all
unconscious, have shown me better than my nurses who knew it.

And, lo! my infancy died long since, and I live. But Thou, Lord, who for
ever livest, and in whom nothing dies: for before the foundation of the
worlds, and before all that can be called "before," Thou art, and art
God and Lord of all which Thou hast created: in Thee abide, fixed
for ever, the first causes of all things unabiding; and of all things
changeable, the springs abide in Thee unchangeable: and in Thee live the
eternal reasons of all things unreasoning and temporal. Say, Lord, to
me, Thy suppliant; say, all-pitying, to me, Thy pitiable one; say, did
my infancy succeed another age of mine that died before it? was it
that which I spent within my mother's womb? for of that I have heard
somewhat, and have myself seen women with child? and what before that
life again, O God my joy, was I any where or any body? For this have I
none to tell me, neither father nor mother, nor experience of others,
nor mine own memory. Dost Thou mock me for asking this, and bid me
praise Thee and acknowledge Thee, for that I do know?

I acknowledge Thee, Lord of heaven and earth, and praise Thee for my
first rudiments of being, and my infancy, whereof I remember nothing;
for Thou hast appointed that man should from others guess much as to
himself; and believe much on the strength of weak females. Even then I
had being and life, and (at my infancy's close) I could seek for signs
whereby to make known to others my sensations. Whence could such a being
be, save from Thee, Lord? Shall any be his own artificer? or can there
elsewhere be derived any vein, which may stream essence and life into
us, save from thee, O Lord, in whom essence and life are one? for Thou
Thyself art supremely Essence and Life. For Thou art most high, and art
not changed, neither in Thee doth to-day come to a close; yet in Thee
doth it come to a close; because all such things also are in Thee. For
they had no way to pass away, unless Thou upheldest them. And since
Thy years fail not, Thy years are one to-day. How many of ours and
our fathers' years have flowed away through Thy "to-day," and from it
received the measure and the mould of such being as they had; and still
others shall flow away, and so receive the mould of their degree of
being. But Thou art still the same, and all things of tomorrow, and all
beyond, and all of yesterday, and all behind it, Thou hast done to-day.
What is it to me, though any comprehend not this? Let him also rejoice
and say, What thing is this? Let him rejoice even thus! and be content
rather by not discovering to discover Thee, than by discovering not to
discover Thee.

Hear, O God. Alas, for man's sin! So saith man, and Thou pitiest him;
for Thou madest him, but sin in him Thou madest not. Who remindeth me of
the sins of my infancy? for in Thy sight none is pure from sin, not even
the infant whose life is but a day upon the earth. Who remindeth me?
doth not each little infant, in whom I see what of myself I remember
not? What then was my sin? was it that I hung upon the breast and cried?
for should I now so do for food suitable to my age, justly should I be
laughed at and reproved. What I then did was worthy reproof; but since
I could not understand reproof, custom and reason forbade me to be
reproved. For those habits, when grown, we root out and cast away. Now
no man, though he prunes, wittingly casts away what is good. Or was
it then good, even for a while, to cry for what, if given, would hurt?
bitterly to resent, that persons free, and its own elders, yea, the very
authors of its birth, served it not? that many besides, wiser than it,
obeyed not the nod of its good pleasure? to do its best to strike and
hurt, because commands were not obeyed, which had been obeyed to its
hurt? The weakness then of infant limbs, not its will, is its innocence.
Myself have seen and known even a baby envious; it could not speak, yet
it turned pale and looked bitterly on its foster-brother. Who knows not
this? Mothers and nurses tell you that they allay these things by I know
not what remedies. Is that too innocence, when the fountain of milk
is flowing in rich abundance, not to endure one to share it, though
in extremest need, and whose very life as yet depends thereon? We bear
gently with all this, not as being no or slight evils, but because they
will disappear as years increase; for, though tolerated now, the very
same tempers are utterly intolerable when found in riper years.

Thou, then, O Lord my God, who gavest life to this my infancy,
furnishing thus with senses (as we see) the frame Thou gavest,
compacting its limbs, ornamenting its proportions, and, for its general
good and safety, implanting in it all vital functions, Thou commandest
me to praise Thee in these things, to confess unto Thee, and sing unto
Thy name, Thou most Highest. For Thou art God, Almighty and Good, even
hadst Thou done nought but only this, which none could do but Thou:
whose Unity is the mould of all things; who out of Thy own fairness
makest all things fair; and orderest all things by Thy law. This age
then, Lord, whereof I have no remembrance, which I take on others' word,
and guess from other infants that I have passed, true though the guess
be, I am yet loth to count in this life of mine which I live in this
world. For no less than that which I spent in my mother's womb, is it
hid from me in the shadows of forgetfulness. But if I was shapen in
iniquity, and in sin did my mother conceive me, where, I beseech Thee, O
my God, where, Lord, or when, was I Thy servant guiltless? But, lo! that
period I pass by; and what have I now to do with that, of which I can
recall no vestige?

Passing hence from infancy, I came to boyhood, or rather it came to me,
displacing infancy. Nor did that depart,--(for whither went it?)--and
yet it was no more. For I was no longer a speechless infant, but a
speaking boy. This I remember; and have since observed how I learned to
speak. It was not that my elders taught me words (as, soon after, other
learning) in any set method; but I, longing by cries and broken accents
and various motions of my limbs to express my thoughts, that so I might
have my will, and yet unable to express all I willed, or to whom I
willed, did myself, by the understanding which Thou, my God, gavest me,
practise the sounds in my memory. When they named any thing, and as they
spoke turned towards it, I saw and remembered that they called what they
would point out by the name they uttered. And that they meant this
thing and no other was plain from the motion of their body, the natural
language, as it were, of all nations, expressed by the countenance,
glances of the eye, gestures of the limbs, and tones of the voice,
indicating the affections of the mind, as it pursues, possesses,
rejects, or shuns. And thus by constantly hearing words, as they
occurred in various sentences, I collected gradually for what they
stood; and having broken in my mouth to these signs, I thereby gave
utterance to my will. Thus I exchanged with those about me these current
signs of our wills, and so launched deeper into the stormy intercourse
of human life, yet depending on parental authority and the beck of
elders.

O God my God, what miseries and mockeries did I now experience, when
obedience to my teachers was proposed to me, as proper in a boy, in
order that in this world I might prosper, and excel in tongue-science,
which should serve to the "praise of men," and to deceitful riches. Next
I was put to school to get learning, in which I (poor wretch) knew not
what use there was; and yet, if idle in learning, I was beaten. For this
was judged right by our forefathers; and many, passing the same course
before us, framed for us weary paths, through which we were fain to
pass; multiplying toil and grief upon the sons of Adam. But, Lord, we
found that men called upon Thee, and we learnt from them to think of
Thee (according to our powers) as of some great One, who, though hidden
from our senses, couldest hear and help us. For so I began, as a boy, to
pray to Thee, my aid and refuge; and broke the fetters of my tongue to
call on Thee, praying Thee, though small, yet with no small earnestness,
that I might not be beaten at school. And when Thou heardest me not (not
thereby giving me over to folly), my elders, yea my very parents, who
yet wished me no ill, mocked my stripes, my then great and grievous ill.

Is there, Lord, any of soul so great, and cleaving to Thee with so
intense affection (for a sort of stupidity will in a way do it); but
is there any one who, from cleaving devoutly to Thee, is endued with so
great a spirit, that he can think as lightly of the racks and hooks and
other torments (against which, throughout all lands, men call on Thee
with extreme dread), mocking at those by whom they are feared most
bitterly, as our parents mocked the torments which we suffered in
boyhood from our masters? For we feared not our torments less; nor
prayed we less to Thee to escape them. And yet we sinned, in writing or
reading or studying less than was exacted of us. For we wanted not, O
Lord, memory or capacity, whereof Thy will gave enough for our age; but
our sole delight was play; and for this we were punished by those who
yet themselves were doing the like. But elder folks' idleness is called
"business"; that of boys, being really the same, is punished by those
elders; and none commiserates either boys or men. For will any of sound
discretion approve of my being beaten as a boy, because, by playing a
ball, I made less progress in studies which I was to learn, only that,
as a man, I might play more unbeseemingly? and what else did he who beat
me? who, if worsted in some trifling discussion with his fellow-tutor,
was more embittered and jealous than I when beaten at ball by a
play-fellow?

And yet, I sinned herein, O Lord God, the Creator and Disposer of all
things in nature, of sin the Disposer only, O Lord my God, I sinned in
transgressing the commands of my parents and those of my masters. For
what they, with whatever motive, would have me learn, I might afterwards
have put to good use. For I disobeyed, not from a better choice, but
from love of play, loving the pride of victory in my contests, and to
have my ears tickled with lying fables, that they might itch the more;
the same curiosity flashing from my eyes more and more, for the shows
and games of my elders. Yet those who give these shows are in such
esteem, that almost all wish the same for their children, and yet are
very willing that they should be beaten, if those very games detain them
from the studies, whereby they would have them attain to be the givers
of them. Look with pity, Lord, on these things, and deliver us who call
upon Thee now; deliver those too who call not on Thee yet, that they may
call on Thee, and Thou mayest deliver them.

  As a boy, then, I had already heard of an eternal life, promised
us through the humility of the Lord our God stooping to our pride; and
even from the womb of my mother, who greatly hoped in Thee, I was sealed
with the mark of His cross and salted with His salt. Thou sawest, Lord,
how while yet a boy, being seized on a time with sudden oppression of
the stomach, and like near to death--Thou sawest, my God (for Thou wert
my keeper), with what eagerness and what faith I sought, from the pious
care of my mother and Thy Church, the mother of us all, the baptism of
Thy Christ, my God and Lord. Whereupon the mother of my flesh, being
much troubled (since, with a heart pure in Thy faith, she even more
lovingly travailed in birth of my salvation), would in eager haste
have provided for my consecration and cleansing by the health-giving
sacraments, confessing Thee, Lord Jesus, for the remission of sins,
unless I had suddenly recovered. And so, as if I must needs be
again polluted should I live, my cleansing was deferred, because the
defilements of sin would, after that washing, bring greater and more
perilous guilt. I then already believed: and my mother, and the whole
household, except my father: yet did not he prevail over the power of my
mother's piety in me, that as he did not yet believe, so neither
should I. For it was her earnest care that Thou my God, rather than he,
shouldest be my father; and in this Thou didst aid her to prevail over
her husband, whom she, the better, obeyed, therein also obeying Thee,
who hast so commanded.

  I beseech Thee, my God, I would fain know, if so Thou willest, for
what purpose my baptism was then deferred? was it for my good that the
rein was laid loose, as it were, upon me, for me to sin? or was it not
laid loose? If not, why does it still echo in our ears on all sides,
"Let him alone, let him do as he will, for he is not yet baptised?" but
as to bodily health, no one says, "Let him be worse wounded, for he is
not yet healed." How much better then, had I been at once healed; and
then, by my friends' and my own, my soul's recovered health had been
kept safe in Thy keeping who gavest it. Better truly. But how many and
great waves of temptation seemed to hang over me after my boyhood! These
my mother foresaw; and preferred to expose to them the clay whence I
might afterwards be moulded, than the very cast, when made.

  In boyhood itself, however (so much less dreaded for me than youth),
I loved not study, and hated to be forced to it. Yet I was forced; and
this was well done towards me, but I did not well; for, unless forced, I
had not learnt. But no one doth well against his will, even though what
he doth, be well. Yet neither did they well who forced me, but what was
well came to me from Thee, my God. For they were regardless how I should
employ what they forced me to learn, except to satiate the insatiate
desires of a wealthy beggary, and a shameful glory. But Thou, by whom
the very hairs of our head are numbered, didst use for my good the error
of all who urged me to learn; and my own, who would not learn, Thou
didst use for my punishment--a fit penalty for one, so small a boy and
so great a sinner. So by those who did not well, Thou didst well for me;
and by my own sin Thou didst justly punish me. For Thou hast commanded,
and so it is, that every inordinate affection should be its own
punishment.

But why did I so much hate the Greek, which I studied as a boy? I do not
yet fully know. For the Latin I loved; not what my first masters, but
what the so-called grammarians taught me. For those first lessons,
reading, writing and arithmetic, I thought as great a burden and penalty
as any Greek. And yet whence was this too, but from the sin and vanity
of this life, because I was flesh, and a breath that passeth away and
cometh not again? For those first lessons were better certainly, because
more certain; by them I obtained, and still retain, the power of reading
what I find written, and myself writing what I will; whereas in the
others, I was forced to learn the wanderings of one Aeneas, forgetful of
my own, and to weep for dead Dido, because she killed herself for love;
the while, with dry eyes, I endured my miserable self dying among these
things, far from Thee, O God my life.

For what more miserable than a miserable being who commiserates not
himself; weeping the death of Dido for love to Aeneas, but weeping not
his own death for want of love to Thee, O God. Thou light of my heart,
Thou bread of my inmost soul, Thou Power who givest vigour to my mind,
who quickenest my thoughts, I loved Thee not. I committed fornication
against Thee, and all around me thus fornicating there echoed "Well
done! well done!" for the friendship of this world is fornication
against Thee; and "Well done! well done!" echoes on till one is ashamed
not to be thus a man. And for all this I wept not, I who wept for Dido
slain, and "seeking by the sword a stroke and wound extreme," myself
seeking the while a worse extreme, the extremest and lowest of Thy
creatures, having forsaken Thee, earth passing into the earth. And
if forbid to read all this, I was grieved that I might not read what
grieved me. Madness like this is thought a higher and a richer learning,
than that by which I learned to read and write.

But now, my God, cry Thou aloud in my soul; and let Thy truth tell me,
"Not so, not so. Far better was that first study." For, lo, I would
readily forget the wanderings of Aeneas and all the rest, rather than
how to read and write. But over the entrance of the Grammar School is a
vail drawn! true; yet is this not so much an emblem of aught recondite,
as a cloak of error. Let not those, whom I no longer fear, cry out
against me, while I confess to Thee, my God, whatever my soul will, and
acquiesce in the condemnation of my evil ways, that I may love Thy
good ways. Let not either buyers or sellers of grammar-learning cry out
against me. For if I question them whether it be true that Aeneas came
on a time to Carthage, as the poet tells, the less learned will reply
that they know not, the more learned that he never did. But should I ask
with what letters the name "Aeneas" is written, every one who has
learnt this will answer me aright, as to the signs which men have
conventionally settled. If, again, I should ask which might be forgotten
with least detriment to the concerns of life, reading and writing or
these poetic fictions? who does not foresee what all must answer who
have not wholly forgotten themselves? I sinned, then, when as a boy I
preferred those empty to those more profitable studies, or rather loved
the one and hated the other. "One and one, two"; "two and two, four";
this was to me a hateful singsong: "the wooden horse lined with
armed men," and "the burning of Troy," and "Creusa's shade and sad
similitude," were the choice spectacle of my vanity.

Why then did I hate the Greek classics, which have the like tales? For
Homer also curiously wove the like fictions, and is most sweetly vain,
yet was he bitter to my boyish taste. And so I suppose would Virgil
be to Grecian children, when forced to learn him as I was Homer.
Difficulty, in truth, the difficulty of a foreign tongue, dashed, as it
were, with gall all the sweetness of Grecian fable. For not one word of
it did I understand, and to make me understand I was urged vehemently
with cruel threats and punishments. Time was also (as an infant) I
knew no Latin; but this I learned without fear or suffering, by mere
observation, amid the caresses of my nursery and jests of friends,
smiling and sportively encouraging me. This I learned without any
pressure of punishment to urge me on, for my heart urged me to give
birth to its conceptions, which I could only do by learning words not of
those who taught, but of those who talked with me; in whose ears also I
gave birth to the thoughts, whatever I conceived. No doubt, then, that
a free curiosity has more force in our learning these things, than a
frightful enforcement. Only this enforcement restrains the rovings of
that freedom, through Thy laws, O my God, Thy laws, from the master's
cane to the martyr's trials, being able to temper for us a wholesome
bitter, recalling us to Thyself from that deadly pleasure which lures us
from Thee.

Hear, Lord, my prayer; let not my soul faint under Thy discipline, nor
let me faint in confessing unto Thee all Thy mercies, whereby Thou
hast drawn me out of all my most evil ways, that Thou mightest become a
delight to me above all the allurements which I once pursued; that I may
most entirely love Thee, and clasp Thy hand with all my affections, and
Thou mayest yet rescue me from every temptation, even unto the end. For
lo, O Lord, my King and my God, for Thy service be whatever useful
thing my childhood learned; for Thy service, that I speak, write, read,
reckon. For Thou didst grant me Thy discipline, while I was learning
vanities; and my sin of delighting in those vanities Thou hast forgiven.
In them, indeed, I learnt many a useful word, but these may as well be
learned in things not vain; and that is the safe path for the steps of
youth.

But woe is thee, thou torrent of human custom! Who shall stand against
thee? how long shalt thou not be dried up? how long roll the sons of Eve
into that huge and hideous ocean, which even they scarcely overpass who
climb the cross? Did not I read in thee of Jove the thunderer and the
adulterer? both, doubtless, he could not be; but so the feigned thunder
might countenance and pander to real adultery. And now which of our
gowned masters lends a sober ear to one who from their own school cries
out, "These were Homer's fictions, transferring things human to the
gods; would he had brought down things divine to us!" Yet more truly had
he said, "These are indeed his fictions; but attributing a divine nature
to wicked men, that crimes might be no longer crimes, and whoso commits
them might seem to imitate not abandoned men, but the celestial gods."

And yet, thou hellish torrent, into thee are cast the sons of men with
rich rewards, for compassing such learning; and a great solemnity is
made of it, when this is going on in the forum, within sight of laws
appointing a salary beside the scholar's payments; and thou lashest
thy rocks and roarest, "Hence words are learnt; hence eloquence; most
necessary to gain your ends, or maintain opinions." As if we should have
never known such words as "golden shower," "lap," "beguile," "temples
of the heavens," or others in that passage, unless Terence had brought
a lewd youth upon the stage, setting up Jupiter as his example of
seduction.

         "Viewing a picture, where the tale was drawn,
         Of Jove's descending in a golden shower
         To Danae's lap a woman to beguile."

And then mark how he excites himself to lust as by celestial authority:

         "And what God?  Great Jove,
         Who shakes heaven's highest temples with his thunder,

         And I, poor mortal man, not do the same!
         I did it, and with all my heart I did it."

Not one whit more easily are the words learnt for all this vileness;
but by their means the vileness is committed with less shame. Not that
I blame the words, being, as it were, choice and precious vessels; but
that wine of error which is drunk to us in them by intoxicated teachers;
and if we, too, drink not, we are beaten, and have no sober judge to
whom we may appeal. Yet, O my God (in whose presence I now without hurt
may remember this), all this unhappily I learnt willingly with great
delight, and for this was pronounced a hopeful boy.

Bear with me, my God, while I say somewhat of my wit, Thy gift, and on
what dotages I wasted it. For a task was set me, troublesome enough to
my soul, upon terms of praise or shame, and fear of stripes, to speak
the words of Juno, as she raged and mourned that she could not

        "This Trojan prince from Latinum turn."

Which words I had heard that Juno never uttered; but we were forced to
go astray in the footsteps of these poetic fictions, and to say in prose
much what he expressed in verse. And his speaking was most applauded, in
whom the passions of rage and grief were most preeminent, and clothed
in the most fitting language, maintaining the dignity of the character.
What is it to me, O my true life, my God, that my declamation was
applauded above so many of my own age and class? is not all this smoke
and wind? and was there nothing else whereon to exercise my wit and
tongue? Thy praises, Lord, Thy praises might have stayed the yet tender
shoot of my heart by the prop of Thy Scriptures; so had it not trailed
away amid these empty trifles, a defiled prey for the fowls of the air.
For in more ways than one do men sacrifice to the rebellious angels.

But what marvel that I was thus carried away to vanities, and went out
from Thy presence, O my God, when men were set before me as models, who,
if in relating some action of theirs, in itself not ill, they committed
some barbarism or solecism, being censured, were abashed; but when
in rich and adorned and well-ordered discourse they related their own
disordered life, being bepraised, they gloried? These things Thou seest,
Lord, and holdest Thy peace; long-suffering, and plenteous in mercy and
truth. Wilt Thou hold Thy peace for ever? and even now Thou drawest out
of this horrible gulf the soul that seeketh Thee, that thirsteth for
Thy pleasures, whose heart saith unto Thee, I have sought Thy face; Thy
face, Lord, will I seek. For darkened affections is removal from Thee.
For it is not by our feet, or change of place, that men leave Thee, or
return unto Thee. Or did that Thy younger son look out for horses or
chariots, or ships, fly with visible wings, or journey by the motion of
his limbs, that he might in a far country waste in riotous living all
Thou gavest at his departure? a loving Father, when Thou gavest, and
more loving unto him, when he returned empty. So then in lustful, that
is, in darkened affections, is the true distance from Thy face.

Behold, O Lord God, yea, behold patiently as Thou art wont how carefully
the sons of men observe the covenanted rules of letters and syllables
received from those who spake before them, neglecting the eternal
covenant of everlasting salvation received from Thee. Insomuch, that
a teacher or learner of the hereditary laws of pronunciation will more
offend men by speaking without the aspirate, of a "uman being," in
despite of the laws of grammar, than if he, a "human being," hate a
"human being" in despite of Thine. As if any enemy could be more hurtful
than the hatred with which he is incensed against him; or could wound
more deeply him whom he persecutes, than he wounds his own soul by his
enmity. Assuredly no science of letters can be so innate as the record
of conscience, "that he is doing to another what from another he would
be loth to suffer." How deep are Thy ways, O God, Thou only great,
that sittest silent on high and by an unwearied law dispensing penal
blindness to lawless desires. In quest of the fame of eloquence, a man
standing before a human judge, surrounded by a human throng, declaiming
against his enemy with fiercest hatred, will take heed most watchfully,
lest, by an error of the tongue, he murder the word "human being"; but
takes no heed, lest, through the fury of his spirit, he murder the real
human being.

This was the world at whose gate unhappy I lay in my boyhood; this
the stage where I had feared more to commit a barbarism, than having
committed one, to envy those who had not. These things I speak and
confess to Thee, my God; for which I had praise from them, whom I then
thought it all virtue to please. For I saw not the abyss of vileness,
wherein I was cast away from Thine eyes. Before them what more foul than
I was already, displeasing even such as myself? with innumerable lies
deceiving my tutor, my masters, my parents, from love of play, eagerness
to see vain shows and restlessness to imitate them! Thefts also I
committed, from my parents' cellar and table, enslaved by greediness, or
that I might have to give to boys, who sold me their play, which all
the while they liked no less than I. In this play, too, I often
sought unfair conquests, conquered myself meanwhile by vain desire of
preeminence. And what could I so ill endure, or, when I detected it,
upbraided I so fiercely, as that I was doing to others? and for which
if, detected, I was upbraided, I chose rather to quarrel than to yield.
And is this the innocence of boyhood? Not so, Lord, not so; I cry Thy
mercy, my God. For these very sins, as riper years succeed, these very
sins are transferred from tutors and masters, from nuts and balls and
sparrows, to magistrates and kings, to gold and manors and slaves, just
as severer punishments displace the cane. It was the low stature then of
childhood which Thou our King didst commend as an emblem of lowliness,
when Thou saidst, Of such is the kingdom of heaven.

Yet, Lord, to Thee, the Creator and Governor of the universe, most
excellent and most good, thanks were due to Thee our God, even hadst
Thou destined for me boyhood only. For even then I was, I lived, and
felt; and had an implanted providence over my well-being--a trace of
that mysterious Unity whence I was derived; I guarded by the inward
sense the entireness of my senses, and in these minute pursuits, and in
my thoughts on things minute, I learnt to delight in truth, I hated to
be deceived, had a vigorous memory, was gifted with speech, was
soothed by friendship, avoided pain, baseness, ignorance. In so small a
creature, what was not wonderful, not admirable? But all are gifts of
my God: it was not I who gave them me; and good these are, and these
together are myself. Good, then, is He that made me, and He is my good;
and before Him will I exult for every good which of a boy I had. For it
was my sin, that not in Him, but in His creatures-myself and others--I
sought for pleasures, sublimities, truths, and so fell headlong into
sorrows, confusions, errors. Thanks be to Thee, my joy and my glory
and my confidence, my God, thanks be to Thee for Thy gifts; but do Thou
preserve them to me. For so wilt Thou preserve me, and those things
shall be enlarged and perfected which Thou hast given me, and I myself
shall be with Thee, since even to be Thou hast given me.




\chapter{BOOK II}


\start{I}{ will} now call to mind my past foulness, and the carnal corruptions of
my soul; not because I love them, but that I may love Thee, O my God.
For love of Thy love I do it; reviewing my most wicked ways in the very
bitterness of my remembrance, that Thou mayest grow sweet unto me (Thou
sweetness never failing, Thou blissful and assured sweetness); and
gathering me again out of that my dissipation, wherein I was torn
piecemeal, while turned from Thee, the One Good, I lost myself among a
multiplicity of things. For I even burnt in my youth heretofore, to be
satiated in things below; and I dared to grow wild again, with these
various and shadowy loves: my beauty consumed away, and I stank in Thine
eyes; pleasing myself, and desirous to please in the eyes of men.

And what was it that I delighted in, but to love, and be loved? but
I kept not the measure of love, of mind to mind, friendship's bright
boundary: but out of the muddy concupiscence of the flesh, and the
bubblings of youth, mists fumed up which beclouded and overcast my
heart, that I could not discern the clear brightness of love from the
fog of lustfulness. Both did confusedly boil in me, and hurried my
unstayed youth over the precipice of unholy desires, and sunk me in a
gulf of flagitiousnesses. Thy wrath had gathered over me, and I knew it
not. I was grown deaf by the clanking of the chain of my mortality, the
punishment of the pride of my soul, and I strayed further from Thee,
and Thou lettest me alone, and I was tossed about, and wasted, and
dissipated, and I boiled over in my fornications, and Thou heldest Thy
peace, O Thou my tardy joy! Thou then heldest Thy peace, and I wandered
further and further from Thee, into more and more fruitless seed-plots
of sorrows, with a proud dejectedness, and a restless weariness.

Oh! that some one had then attempered my disorder, and turned to account
the fleeting beauties of these, the extreme points of Thy creation! had
put a bound to their pleasureableness, that so the tides of my youth
might have cast themselves upon the marriage shore, if they could not be
calmed, and kept within the object of a family, as Thy law prescribes,
O Lord: who this way formest the offspring of this our death, being
able with a gentle hand to blunt the thorns which were excluded from Thy
paradise? For Thy omnipotency is not far from us, even when we be far
from Thee. Else ought I more watchfully to have heeded the voice from
the clouds: Nevertheless such shall have trouble in the flesh, but I
spare you. And it is good for a man not to touch a woman. And, he that
is unmarried thinketh of the things of the Lord, how he may please the
Lord; but he that is married careth for the things of this world, how he
may please his wife.

To these words I should have listened more attentively, and being
severed for the kingdom of heaven's sake, had more happily awaited Thy
embraces; but I, poor wretch, foamed like a troubled sea, following the
rushing of my own tide, forsaking Thee, and exceeded all Thy limits; yet
I escaped not Thy scourges. For what mortal can? For Thou wert ever with
me mercifully rigorous, and besprinkling with most bitter alloy all my
unlawful pleasures: that I might seek pleasures without alloy. But where
to find such, I could not discover, save in Thee, O Lord, who teachest
by sorrow, and woundest us, to heal; and killest us, lest we die from
Thee. Where was I, and how far was I exiled from the delights of Thy
house, in that sixteenth year of the age of my flesh, when the madness
of lust (to which human shamelessness giveth free licence, though
unlicensed by Thy laws) took the rule over me, and I resigned myself
wholly to it? My friends meanwhile took no care by marriage to save my
fall; their only care was that I should learn to speak excellently, and
be a persuasive orator.

For that year were my studies intermitted: whilst after my return from
Madaura (a neighbour city, whither I had journeyed to learn grammar and
rhetoric), the expenses for a further journey to Carthage were being
provided for me; and that rather by the resolution than the means of my
father, who was but a poor freeman of Thagaste. To whom tell I this? not
to Thee, my God; but before Thee to mine own kind, even to that small
portion of mankind as may light upon these writings of mine. And to what
purpose? that whosoever reads this, may think out of what depths we are
to cry unto Thee. For what is nearer to Thine ears than a confessing
heart, and a life of faith? Who did not extol my father, for that beyond
the ability of his means, he would furnish his son with all necessaries
for a far journey for his studies' sake? For many far abler citizens
did no such thing for their children. But yet this same father had no
concern how I grew towards Thee, or how chaste I were; so that I were
but copious in speech, however barren I were to Thy culture, O God, who
art the only true and good Lord of Thy field, my heart.

But while in that my sixteenth year I lived with my parents, leaving all
school for a while (a season of idleness being interposed through the
narrowness of my parents' fortunes), the briers of unclean desires grew
rank over my head, and there was no hand to root them out. When that my
father saw me at the baths, now growing towards manhood, and endued
with a restless youthfulness, he, as already hence anticipating his
descendants, gladly told it to my mother; rejoicing in that tumult of
the senses wherein the world forgetteth Thee its Creator, and becometh
enamoured of Thy creature, instead of Thyself, through the fumes of that
invisible wine of its self-will, turning aside and bowing down to the
very basest things. But in my mother's breast Thou hadst already begun
Thy temple, and the foundation of Thy holy habitation, whereas my
father was as yet but a Catechumen, and that but recently. She then was
startled with a holy fear and trembling; and though I was not as yet
baptised, feared for me those crooked ways in which they walk who turn
their back to Thee, and not their face.

Woe is me! and dare I say that Thou heldest Thy peace, O my God, while I
wandered further from Thee? Didst Thou then indeed hold Thy peace to me?
And whose but Thine were these words which by my mother, Thy faithful
one, Thou sangest in my ears? Nothing whereof sunk into my heart, so as
to do it. For she wished, and I remember in private with great anxiety
warned me, "not to commit fornication; but especially never to defile
another man's wife." These seemed to me womanish advices, which I should
blush to obey. But they were Thine, and I knew it not: and I thought
Thou wert silent and that it was she who spake; by whom Thou wert not
silent unto me; and in her wast despised by me, her son, the son of Thy
handmaid, Thy servant. But I knew it not; and ran headlong with such
blindness, that amongst my equals I was ashamed of a less shamelessness,
when I heard them boast of their flagitiousness, yea, and the more
boasting, the more they were degraded: and I took pleasure, not only in
the pleasure of the deed, but in the praise. What is worthy of dispraise
but vice? But I made myself worse than I was, that I might not be
dispraised; and when in any thing I had not sinned as the abandoned
ones, I would say that I had done what I had not done, that I might not
seem contemptible in proportion as I was innocent; or of less account,
the more chaste.

Behold with what companions I walked the streets of Babylon, and
wallowed in the mire thereof, as if in a bed of spices and precious
ointments. And that I might cleave the faster to its very centre, the
invisible enemy trod me down, and seduced me, for that I was easy to be
seduced. Neither did the mother of my flesh (who had now fled out of
the centre of Babylon, yet went more slowly in the skirts thereof as
she advised me to chastity, so heed what she had heard of me from her
husband, as to restrain within the bounds of conjugal affection, if it
could not be pared away to the quick) what she felt to be pestilent
at present and for the future dangerous. She heeded not this, for she
feared lest a wife should prove a clog and hindrance to my hopes. Not
those hopes of the world to come, which my mother reposed in Thee; but
the hope of learning, which both my parents were too desirous I should
attain; my father, because he had next to no thought of Thee, and of
me but vain conceits; my mother, because she accounted that those
usual courses of learning would not only be no hindrance, but even some
furtherance towards attaining Thee. For thus I conjecture, recalling, as
well as I may, the disposition of my parents. The reins, meantime, were
slackened to me, beyond all temper of due severity, to spend my time in
sport, yea, even unto dissoluteness in whatsoever I affected. And in all
was a mist, intercepting from me, O my God, the brightness of Thy truth;
and mine iniquity burst out as from very fatness.

Theft is punished by Thy law, O Lord, and the law written in the hearts
of men, which iniquity itself effaces not. For what thief will abide a
thief? not even a rich thief, one stealing through want. Yet I lusted to
thieve, and did it, compelled by no hunger, nor poverty, but through a
cloyedness of well-doing, and a pamperedness of iniquity. For I stole
that, of which I had enough, and much better. Nor cared I to enjoy what
I stole, but joyed in the theft and sin itself. A pear tree there was
near our vineyard, laden with fruit, tempting neither for colour nor
taste. To shake and rob this, some lewd young fellows of us went, late
one night (having according to our pestilent custom prolonged our sports
in the streets till then), and took huge loads, not for our eating, but
to fling to the very hogs, having only tasted them. And this, but to
do what we liked only, because it was misliked. Behold my heart, O
God, behold my heart, which Thou hadst pity upon in the bottom of the
bottomless pit. Now, behold, let my heart tell Thee what it sought
there, that I should be gratuitously evil, having no temptation to ill,
but the ill itself. It was foul, and I loved it; I loved to perish,
I loved mine own fault, not that for which I was faulty, but my fault
itself. Foul soul, falling from Thy firmament to utter destruction; not
seeking aught through the shame, but the shame itself!

For there is an attractiveness in beautiful bodies, in gold and silver,
and all things; and in bodily touch, sympathy hath much influence, and
each other sense hath his proper object answerably tempered. Worldy
honour hath also its grace, and the power of overcoming, and of mastery;
whence springs also the thirst of revenge. But yet, to obtain all these,
we may not depart from Thee, O Lord, nor decline from Thy law. The life
also which here we live hath its own enchantment, through a certain
proportion of its own, and a correspondence with all things beautiful
here below. Human friendship also is endeared with a sweet tie, by
reason of the unity formed of many souls. Upon occasion of all these,
and the like, is sin committed, while through an immoderate inclination
towards these goods of the lowest order, the better and higher are
forsaken,--Thou, our Lord God, Thy truth, and Thy law. For these lower
things have their delights, but not like my God, who made all things;
for in Him doth the righteous delight, and He is the joy of the upright
in heart.

When, then, we ask why a crime was done, we believe it not, unless it
appear that there might have been some desire of obtaining some of those
which we called lower goods, or a fear of losing them. For they are
beautiful and comely; although compared with those higher and beatific
goods, they be abject and low. A man hath murdered another; why? he
loved his wife or his estate; or would rob for his own livelihood; or
feared to lose some such things by him; or, wronged, was on fire to be
revenged. Would any commit murder upon no cause, delighted simply in
murdering? who would believe it? for as for that furious and savage man,
of whom it is said that he was gratuitously evil and cruel, yet is the
cause assigned; "lest" (saith he) "through idleness hand or heart should
grow inactive." And to what end? that, through that practice of guilt,
he might, having taken the city, attain to honours, empire, riches, and
be freed from fear of the laws, and his embarrassments from domestic
needs, and consciousness of villainies. So then, not even Catiline
himself loved his own villainies, but something else, for whose sake he
did them.

What then did wretched I so love in thee, thou theft of mine, thou deed
of darkness, in that sixteenth year of my age? Lovely thou wert not,
because thou wert theft. But art thou any thing, that thus I speak to
thee? Fair were the pears we stole, because they were Thy creation, Thou
fairest of all, Creator of all, Thou good God; God, the sovereign good
and my true good. Fair were those pears, but not them did my wretched
soul desire; for I had store of better, and those I gathered, only that
I might steal. For, when gathered, I flung them away, my only feast
therein being my own sin, which I was pleased to enjoy. For if aught
of those pears came within my mouth, what sweetened it was the sin.
And now, O Lord my God, I enquire what in that theft delighted me; and
behold it hath no loveliness; I mean not such loveliness as in justice
and wisdom; nor such as is in the mind and memory, and senses, and
animal life of man; nor yet as the stars are glorious and beautiful in
their orbs; or the earth, or sea, full of embryo-life, replacing by its
birth that which decayeth; nay, nor even that false and shadowy beauty
which belongeth to deceiving vices.

For so doth pride imitate exaltedness; whereas Thou alone art God
exalted over all. Ambition, what seeks it, but honours and glory?
whereas Thou alone art to be honoured above all, and glorious for
evermore. The cruelty of the great would fain be feared; but who is
to be feared but God alone, out of whose power what can be wrested or
withdrawn? when, or where, or whither, or by whom? The tendernesses of
the wanton would fain be counted love: yet is nothing more tender than
Thy charity; nor is aught loved more healthfully than that Thy truth,
bright and beautiful above all. Curiosity makes semblance of a desire
of knowledge; whereas Thou supremely knowest all. Yea, ignorance
and foolishness itself is cloaked under the name of simplicity and
uninjuriousness; because nothing is found more single than Thee: and
what less injurious, since they are his own works which injure the
sinner? Yea, sloth would fain be at rest; but what stable rest besides
the Lord? Luxury affects to be called plenty and abundance; but Thou art
the fulness and never-failing plenteousness of incorruptible pleasures.
Prodigality presents a shadow of liberality: but Thou art the most
overflowing Giver of all good. Covetousness would possess many things;
and Thou possessest all things. Envy disputes for excellency: what more
excellent than Thou? Anger seeks revenge: who revenges more justly
than Thou? Fear startles at things unwonted and sudden, which endangers
things beloved, and takes forethought for their safety; but to Thee what
unwonted or sudden, or who separateth from Thee what Thou lovest? Or
where but with Thee is unshaken safety? Grief pines away for things
lost, the delight of its desires; because it would have nothing taken
from it, as nothing can from Thee.

Thus doth the soul commit fornication, when she turns from Thee, seeking
without Thee, what she findeth not pure and untainted, till she returns
to Thee. Thus all pervertedly imitate Thee, who remove far from Thee,
and lift themselves up against Thee. But even by thus imitating Thee,
they imply Thee to be the Creator of all nature; whence there is no
place whither altogether to retire from Thee. What then did I love in
that theft? and wherein did I even corruptly and pervertedly imitate my
Lord? Did I wish even by stealth to do contrary to Thy law, because
by power I could not, so that being a prisoner, I might mimic a maimed
liberty by doing with impunity things unpermitted me, a darkened
likeness of Thy Omnipotency? Behold, Thy servant, fleeing from his Lord,
and obtaining a shadow. O rottenness, O monstrousness of life, and depth
of death! could I like what I might not, only because I might not?

What shall I render unto the Lord, that, whilst my memory recalls these
things, my soul is not affrighted at them? I will love Thee, O Lord,
and thank Thee, and confess unto Thy name; because Thou hast forgiven me
these so great and heinous deeds of mine. To Thy grace I ascribe it, and
to Thy mercy, that Thou hast melted away my sins as it were ice. To Thy
grace I ascribe also whatsoever I have not done of evil; for what might
I not have done, who even loved a sin for its own sake? Yea, all I
confess to have been forgiven me; both what evils I committed by my own
wilfulness, and what by Thy guidance I committed not. What man is
he, who, weighing his own infirmity, dares to ascribe his purity and
innocency to his own strength; that so he should love Thee the less, as
if he had less needed Thy mercy, whereby Thou remittest sins to those
that turn to Thee? For whosoever, called by Thee, followed Thy voice,
and avoided those things which he reads me recalling and confessing
of myself, let him not scorn me, who being sick, was cured by that
Physician, through whose aid it was that he was not, or rather was less,
sick: and for this let him love Thee as much, yea and more; since by
whom he sees me to have been recovered from such deep consumption of
sin, by Him he sees himself to have been from the like consumption of
sin preserved.

What fruit had I then (wretched man!) in those things, of the
remembrance whereof I am now ashamed? Especially, in that theft which
I loved for the theft's sake; and it too was nothing, and therefore the
more miserable I, who loved it. Yet alone I had not done it: such was I
then, I remember, alone I had never done it. I loved then in it also
the company of the accomplices, with whom I did it? I did not then love
nothing else but the theft, yea rather I did love nothing else; for that
circumstance of the company was also nothing. What is, in truth? who can
teach me, save He that enlighteneth my heart, and discovereth its
dark corners? What is it which hath come into my mind to enquire, and
discuss, and consider? For had I then loved the pears I stole,
and wished to enjoy them, I might have done it alone, had the bare
commission of the theft sufficed to attain my pleasure; nor needed
I have inflamed the itching of my desires by the excitement of
accomplices. But since my pleasure was not in those pears, it was in the
offence itself, which the company of fellow-sinners occasioned.

What then was this feeling? For of a truth it was too foul: and woe was
me, who had it. But yet what was it? Who can understand his errors? It
was the sport, which as it were tickled our hearts, that we beguiled
those who little thought what we were doing, and much disliked it. Why
then was my delight of such sort that I did it not alone? Because none
doth ordinarily laugh alone? ordinarily no one; yet laughter sometimes
masters men alone and singly when no one whatever is with them, if
anything very ludicrous presents itself to their senses or mind. Yet
I had not done this alone; alone I had never done it. Behold my God,
before Thee, the vivid remembrance of my soul; alone, I had never
committed that theft wherein what I stole pleased me not, but that
I stole; nor had it alone liked me to do it, nor had I done it. O
friendship too unfriendly! thou incomprehensible inveigler of the soul,
thou greediness to do mischief out of mirth and wantonness, thou thirst
of others' loss, without lust of my own gain or revenge: but when it is
said, "Let's go, let's do it," we are ashamed not to be shameless.

Who can disentangle that twisted and intricate knottiness? Foul is it: I
hate to think on it, to look on it. But Thee I long for, O Righteousness
and Innocency, beautiful and comely to all pure eyes, and of a
satisfaction unsating. With Thee is rest entire, and life imperturbable.
Whoso enters into Thee, enters into the joy of his Lord: and shall not
fear, and shall do excellently in the All-Excellent. I sank away from
Thee, and I wandered, O my God, too much astray from Thee my stay, in
these days of my youth, and I became to myself a barren land.




\chapter{BOOK III}


\start{T}{o} Carthage I came, where there sang all around me in my ears a cauldron
of unholy loves. I loved not yet, yet I loved to love, and out of a
deep-seated want, I hated myself for wanting not. I sought what I might
love, in love with loving, and safety I hated, and a way without snares.
For within me was a famine of that inward food, Thyself, my God; yet,
through that famine I was not hungered; but was without all longing for
incorruptible sustenance, not because filled therewith, but the more
empty, the more I loathed it. For this cause my soul was sickly and full
of sores, it miserably cast itself forth, desiring to be scraped by the
touch of objects of sense. Yet if these had not a soul, they would not
be objects of love. To love then, and to be beloved, was sweet to
me; but more, when I obtained to enjoy the person I loved, I defiled,
therefore, the spring of friendship with the filth of concupiscence, and
I beclouded its brightness with the hell of lustfulness; and thus
foul and unseemly, I would fain, through exceeding vanity, be fine
and courtly. I fell headlong then into the love wherein I longed to be
ensnared. My God, my Mercy, with how much gall didst Thou out of Thy
great goodness besprinkle for me that sweetness? For I was both beloved,
and secretly arrived at the bond of enjoying; and was with joy fettered
with sorrow-bringing bonds, that I might be scourged with the iron
burning rods of jealousy, and suspicions, and fears, and angers, and
quarrels.

Stage-plays also carried me away, full of images of my miseries, and of
fuel to my fire. Why is it, that man desires to be made sad, beholding
doleful and tragical things, which yet himself would no means suffer?
yet he desires as a spectator to feel sorrow at them, and this very
sorrow is his pleasure. What is this but a miserable madness? for a man
is the more affected with these actions, the less free he is from such
affections. Howsoever, when he suffers in his own person, it uses to be
styled misery: when he compassionates others, then it is mercy. But what
sort of compassion is this for feigned and scenical passions? for the
auditor is not called on to relieve, but only to grieve: and he applauds
the actor of these fictions the more, the more he grieves. And if the
calamities of those persons (whether of old times, or mere fiction)
be so acted, that the spectator is not moved to tears, he goes away
disgusted and criticising; but if he be moved to passion, he stays
intent, and weeps for joy.

Are griefs then too loved? Verily all desire joy. Or whereas no man
likes to be miserable, is he yet pleased to be merciful? which because
it cannot be without passion, for this reason alone are passions loved?
This also springs from that vein of friendship. But whither goes that
vein? whither flows it? wherefore runs it into that torrent of pitch
bubbling forth those monstrous tides of foul lustfulness, into which it
is wilfully changed and transformed, being of its own will precipitated
and corrupted from its heavenly clearness? Shall compassion then be
put away? by no means. Be griefs then sometimes loved. But beware of
uncleanness, O my soul, under the guardianship of my God, the God of our
fathers, who is to be praised and exalted above all for ever, beware of
uncleanness. For I have not now ceased to pity; but then in the theatres
I rejoiced with lovers when they wickedly enjoyed one another, although
this was imaginary only in the play. And when they lost one another, as
if very compassionate, I sorrowed with them, yet had my delight in both.
But now I much more pity him that rejoiceth in his wickedness, than him
who is thought to suffer hardship, by missing some pernicious pleasure,
and the loss of some miserable felicity. This certainly is the truer
mercy, but in it grief delights not. For though he that grieves for the
miserable, be commended for his office of charity; yet had he, who is
genuinely compassionate, rather there were nothing for him to grieve
for. For if good will be ill willed (which can never be), then may
he, who truly and sincerely commiserates, wish there might be some
miserable, that he might commiserate. Some sorrow may then be allowed,
none loved. For thus dost Thou, O Lord God, who lovest souls far more
purely than we, and hast more incorruptibly pity on them, yet are
wounded with no sorrowfulness. And who is sufficient for these things?

But I, miserable, then loved to grieve, and sought out what to grieve
at, when in another's and that feigned and personated misery, that
acting best pleased me, and attracted me the most vehemently, which
drew tears from me. What marvel that an unhappy sheep, straying from
Thy flock, and impatient of Thy keeping, I became infected with a foul
disease? And hence the love of griefs; not such as should sink deep into
me; for I loved not to suffer, what I loved to look on; but such as upon
hearing their fictions should lightly scratch the surface; upon which,
as on envenomed nails, followed inflamed swelling, impostumes, and a
putrefied sore. My life being such, was it life, O my God?

And Thy faithful mercy hovered over me afar. Upon how grievous
iniquities consumed I myself, pursuing a sacrilegious curiosity, that
having forsaken Thee, it might bring me to the treacherous abyss, and
the beguiling service of devils, to whom I sacrificed my evil actions,
and in all these things Thou didst scourge me! I dared even, while Thy
solemnities were celebrated within the walls of Thy Church, to desire,
and to compass a business deserving death for its fruits, for which Thou
scourgedst me with grievous punishments, though nothing to my fault,
O Thou my exceeding mercy, my God, my refuge from those terrible
destroyers, among whom I wandered with a stiff neck, withdrawing
further from Thee, loving mine own ways, and not Thine; loving a vagrant
liberty.

Those studies also, which were accounted commendable, had a view to
excelling in the courts of litigation; the more bepraised, the craftier.
Such is men's blindness, glorying even in their blindness. And now I
was chief in the rhetoric school, whereat I joyed proudly, and I swelled
with arrogancy, though (Lord, Thou knowest) far quieter and altogether
removed from the subvertings of those "Subverters" (for this ill-omened
and devilish name was the very badge of gallantry) among whom I lived,
with a shameless shame that I was not even as they. With them I lived,
and was sometimes delighted with their friendship, whose doings I ever
did abhor--i.e., their "subvertings," wherewith they wantonly persecuted
the modesty of strangers, which they disturbed by a gratuitous jeering,
feeding thereon their malicious birth. Nothing can be liker the very
actions of devils than these. What then could they be more truly called
than "Subverters"? themselves subverted and altogether perverted first,
the deceiving spirits secretly deriding and seducing them, wherein
themselves delight to jeer at and deceive others.

Among such as these, in that unsettled age of mine, learned I books
of eloquence, wherein I desired to be eminent, out of a damnable and
vainglorious end, a joy in human vanity. In the ordinary course of
study, I fell upon a certain book of Cicero, whose speech almost all
admire, not so his heart. This book of his contains an exhortation
to philosophy, and is called "Hortensius." But this book altered my
affections, and turned my prayers to Thyself O Lord; and made me have
other purposes and desires. Every vain hope at once became worthless to
me; and I longed with an incredibly burning desire for an immortality of
wisdom, and began now to arise, that I might return to Thee. For not
to sharpen my tongue (which thing I seemed to be purchasing with my
mother's allowances, in that my nineteenth year, my father being dead
two years before), not to sharpen my tongue did I employ that book; nor
did it infuse into me its style, but its matter.

How did I burn then, my God, how did I burn to re-mount from earthly
things to Thee, nor knew I what Thou wouldest do with me? For with Thee
is wisdom. But the love of wisdom is in Greek called "philosophy,"
with which that book inflamed me. Some there be that seduce through
philosophy, under a great, and smooth, and honourable name colouring and
disguising their own errors: and almost all who in that and former ages
were such, are in that book censured and set forth: there also is
made plain that wholesome advice of Thy Spirit, by Thy good and devout
servant: Beware lest any man spoil you through philosophy and vain
deceit, after the tradition of men, after the rudiments of the world,
and not after Christ. For in Him dwelleth all the fulness of the Godhead
bodily. And since at that time (Thou, O light of my heart, knowest)
Apostolic Scripture was not known to me, I was delighted with that
exhortation, so far only, that I was thereby strongly roused, and
kindled, and inflamed to love, and seek, and obtain, and hold, and
embrace not this or that sect, but wisdom itself whatever it were; and
this alone checked me thus unkindled, that the name of Christ was not
in it. For this name, according to Thy mercy, O Lord, this name of
my Saviour Thy Son, had my tender heart, even with my mother's milk,
devoutly drunk in and deeply treasured; and whatsoever was without that
name, though never so learned, polished, or true, took not entire hold
of me.

I resolved then to bend my mind to the holy Scriptures, that I might see
what they were. But behold, I see a thing not understood by the proud,
nor laid open to children, lowly in access, in its recesses lofty, and
veiled with mysteries; and I was not such as could enter into it, or
stoop my neck to follow its steps. For not as I now speak, did I feel
when I turned to those Scriptures; but they seemed to me unworthy to be
compared to the stateliness of Tully: for my swelling pride shrunk from
their lowliness, nor could my sharp wit pierce the interior thereof. Yet
were they such as would grow up in a little one. But I disdained to be a
little one; and, swollen with pride, took myself to be a great one.

Therefore I fell among men proudly doting, exceeding carnal and prating,
in whose mouths were the snares of the Devil, limed with the mixture of
the syllables of Thy name, and of our Lord Jesus Christ, and of the Holy
Ghost, the Paraclete, our Comforter. These names departed not out of
their mouth, but so far forth as the sound only and the noise of the
tongue, for the heart was void of truth. Yet they cried out "Truth,
Truth," and spake much thereof to me, yet it was not in them: but they
spake falsehood, not of Thee only (who truly art Truth), but even of
those elements of this world, Thy creatures. And I indeed ought to have
passed by even philosophers who spake truth concerning them, for love
of Thee, my Father, supremely good, Beauty of all things beautiful.
O Truth, Truth, how inwardly did even then the marrow of my soul pant
after Thee, when they often and diversely, and in many and huge books,
echoed of Thee to me, though it was but an echo? And these were the
dishes wherein to me, hungering after Thee, they, instead of Thee,
served up the Sun and Moon, beautiful works of Thine, but yet Thy works,
not Thyself, no nor Thy first works. For Thy spiritual works are before
these corporeal works, celestial though they be, and shining. But I
hungered and thirsted not even after those first works of Thine, but
after Thee Thyself, the Truth, in whom is no variableness, neither
shadow of turning: yet they still set before me in those dishes,
glittering fantasies, than which better were it to love this very sun
(which is real to our sight at least), than those fantasies which by
our eyes deceive our mind. Yet because I thought them to be Thee, I fed
thereon; not eagerly, for Thou didst not in them taste to me as Thou
art; for Thou wast not these emptinesses, nor was I nourished by them,
but exhausted rather. Food in sleep shows very like our food awake; yet
are not those asleep nourished by it, for they are asleep. But those
were not even any way like to Thee, as Thou hast now spoken to me; for
those were corporeal fantasies, false bodies, than which these true
bodies, celestial or terrestrial, which with our fleshly sight we
behold, are far more certain: these things the beasts and birds discern
as well as we, and they are more certain than when we fancy them. And
again, we do with more certainty fancy them, than by them conjecture
other vaster and infinite bodies which have no being. Such empty husks
was I then fed on; and was not fed. But Thou, my soul's Love, in looking
for whom I fail, that I may become strong, art neither those bodies
which we see, though in heaven; nor those which we see not there; for
Thou hast created them, nor dost Thou account them among the chiefest of
Thy works. How far then art Thou from those fantasies of mine, fantasies
of bodies which altogether are not, than which the images of those
bodies, which are, are far more certain, and more certain still the
bodies themselves, which yet Thou art not; no, nor yet the soul, which
is the life of the bodies. So then, better and more certain is the life
of the bodies than the bodies. But Thou art the life of souls, the life
of lives, having life in Thyself; and changest not, life of my soul.

Where then wert Thou then to me, and how far from me? Far verily was I
straying from Thee, barred from the very husks of the swine, whom with
husks I fed. For how much better are the fables of poets and grammarians
than these snares? For verses, and poems, and "Medea flying," are more
profitable truly than these men's five elements, variously disguised,
answering to five dens of darkness, which have no being, yet slay the
believer. For verses and poems I can turn to true food, and "Medea
flying," though I did sing, I maintained not; though I heard it sung,
I believed not: but those things I did believe. Woe, woe, by what steps
was I brought down to the depths of hell! toiling and turmoiling through
want of Truth, since I sought after Thee, my God (to Thee I confess
it, who hadst mercy on me, not as yet confessing), not according to the
understanding of the mind, wherein Thou willedst that I should excel
the beasts, but according to the sense of the flesh. But Thou wert more
inward to me than my most inward part; and higher than my highest. I
lighted upon that bold woman, simple and knoweth nothing, shadowed out
in Solomon, sitting at the door, and saying, Eat ye bread of secrecies
willingly, and drink ye stolen waters which are sweet: she seduced me,
because she found my soul dwelling abroad in the eye of my flesh, and
ruminating on such food as through it I had devoured.

For other than this, that which really is I knew not; and was, as it
were through sharpness of wit, persuaded to assent to foolish deceivers,
when they asked me, "whence is evil?" "is God bounded by a bodily shape,
and has hairs and nails?" "are they to be esteemed righteous who had
many wives at once, and did kill men, and sacrifice living creatures?"
At which I, in my ignorance, was much troubled, and departing from the
truth, seemed to myself to be making towards it; because as yet I knew
not that evil was nothing but a privation of good, until at last a thing
ceases altogether to be; which how should I see, the sight of whose eyes
reached only to bodies, and of my mind to a phantasm? And I knew not God
to be a Spirit, not one who hath parts extended in length and breadth,
or whose being was bulk; for every bulk is less in a part than in the
whole: and if it be infinite, it must be less in such part as is defined
by a certain space, than in its infinitude; and so is not wholly every
where, as Spirit, as God. And what that should be in us, by which we
were like to God, and might be rightly said to be after the image of
God, I was altogether ignorant.

Nor knew I that true inward righteousness which judgeth not according
to custom, but out of the most rightful law of God Almighty, whereby
the ways of places and times were disposed according to those times and
places; itself meantime being the same always and every where, not one
thing in one place, and another in another; according to which Abraham,
and Isaac, and Jacob, and Moses, and David, were righteous, and all
those commended by the mouth of God; but were judged unrighteous by
silly men, judging out of man's judgment, and measuring by their own
petty habits, the moral habits of the whole human race. As if in an
armory, one ignorant what were adapted to each part should cover his
head with greaves, or seek to be shod with a helmet, and complain that
they fitted not: or as if on a day when business is publicly stopped in
the afternoon, one were angered at not being allowed to keep open shop,
because he had been in the forenoon; or when in one house he observeth
some servant take a thing in his hand, which the butler is not suffered
to meddle with; or something permitted out of doors, which is forbidden
in the dining-room; and should be angry, that in one house, and one
family, the same thing is not allotted every where, and to all. Even
such are they who are fretted to hear something to have been lawful
for righteous men formerly, which now is not; or that God, for certain
temporal respects, commanded them one thing, and these another, obeying
both the same righteousness: whereas they see, in one man, and one day,
and one house, different things to be fit for different members, and
a thing formerly lawful, after a certain time not so; in one corner
permitted or commanded, but in another rightly forbidden and punished.
Is justice therefore various or mutable? No, but the times, over which
it presides, flow not evenly, because they are times. But men whose days
are few upon the earth, for that by their senses they cannot harmonise
the causes of things in former ages and other nations, which they had
not experience of, with these which they have experience of, whereas in
one and the same body, day, or family, they easily see what is fitting
for each member, and season, part, and person; to the one they take
exceptions, to the other they submit.

These things I then knew not, nor observed; they struck my sight on all
sides, and I saw them not. I indited verses, in which I might not place
every foot every where, but differently in different metres; nor even in
any one metre the self-same foot in all places. Yet the art itself, by
which I indited, had not different principles for these different cases,
but comprised all in one. Still I saw not how that righteousness, which
good and holy men obeyed, did far more excellently and sublimely contain
in one all those things which God commanded, and in no part varied;
although in varying times it prescribed not every thing at once, but
apportioned and enjoined what was fit for each. And I in my blindness,
censured the holy Fathers, not only wherein they made use of things
present as God commanded and inspired them, but also wherein they were
foretelling things to come, as God was revealing in them.

Can it at any time or place be unjust to love God with all his heart,
with all his soul, and with all his mind; and his neighbour as himself?
Therefore are those foul offences which be against nature, to be every
where and at all times detested and punished; such as were those of the
men of Sodom: which should all nations commit, they should all stand
guilty of the same crime, by the law of God, which hath not so made men
that they should so abuse one another. For even that intercourse which
should be between God and us is violated, when that same nature, of
which He is Author, is polluted by perversity of lust. But those
actions which are offences against the customs of men, are to be avoided
according to the customs severally prevailing; so that a thing agreed
upon, and confirmed, by custom or law of any city or nation, may not be
violated at the lawless pleasure of any, whether native or foreigner.
For any part which harmoniseth not with its whole, is offensive. But
when God commands a thing to be done, against the customs or compact of
any people, though it were never by them done heretofore, it is to be
done; and if intermitted, it is to be restored; and if never ordained,
is now to be ordained. For lawful if it be for a king, in the state
which he reigns over, to command that which no one before him, nor he
himself heretofore, had commanded, and to obey him cannot be against the
common weal of the state (nay, it were against it if he were not obeyed,
for to obey princes is a general compact of human society); how much
more unhesitatingly ought we to obey God, in all which He commands, the
Ruler of all His creatures! For as among the powers in man's society,
the greater authority is obeyed in preference to the lesser, so must God
above all.

So in acts of violence, where there is a wish to hurt, whether by
reproach or injury; and these either for revenge, as one enemy against
another; or for some profit belonging to another, as the robber to
the traveller; or to avoid some evil, as towards one who is feared; or
through envy, as one less fortunate to one more so, or one well thriven
in any thing, to him whose being on a par with himself he fears, or
grieves at, or for the mere pleasure at another's pain, as spectators
of gladiators, or deriders and mockers of others. These be the heads
of iniquity which spring from the lust of the flesh, of the eye, or of
rule, either singly, or two combined, or all together; and so do men
live ill against the three, and seven, that psaltery of of ten strings,
Thy Ten Commandments, O God, most high, and most sweet. But what foul
offences can there be against Thee, who canst not be defiled? or
what acts of violence against Thee, who canst not be harmed? But Thou
avengest what men commit against themselves, seeing also when they sin
against Thee, they do wickedly against their own souls, and iniquity
gives itself the lie, by corrupting and perverting their nature, which
Thou hast created and ordained, or by an immoderate use of things
allowed, or in burning in things unallowed, to that use which is against
nature; or are found guilty, raging with heart and tongue against Thee,
kicking against the pricks; or when, bursting the pale of human society,
they boldly joy in self-willed combinations or divisions, according as
they have any object to gain or subject of offence. And these things are
done when Thou art forsaken, O Fountain of Life, who art the only and
true Creator and Governor of the Universe, and by a self-willed pride,
any one false thing is selected therefrom and loved. So then by a
humble devoutness we return to Thee; and Thou cleansest us from our
evil habits, and art merciful to their sins who confess, and hearest the
groaning of the prisoner, and loosest us from the chains which we made
for ourselves, if we lift not up against Thee the horns of an unreal
liberty, suffering the loss of all, through covetousness of more, by
loving more our own private good than Thee, the Good of all.

Amidst these offences of foulness and violence, and so many iniquities,
are sins of men, who are on the whole making proficiency; which by those
that judge rightly, are, after the rule of perfection, discommended, yet
the persons commended, upon hope of future fruit, as in the green blade
of growing corn. And there are some, resembling offences of foulness or
violence, which yet are no sins; because they offend neither Thee, our
Lord God, nor human society; when, namely, things fitting for a given
period are obtained for the service of life, and we know not whether out
of a lust of having; or when things are, for the sake of correction, by
constituted authority punished, and we know not whether out of a lust of
hurting. Many an action then which in men's sight is disapproved, is
by Thy testimony approved; and many, by men praised, are (Thou being
witness) condemned: because the show of the action, and the mind of the
doer, and the unknown exigency of the period, severally vary. But when
Thou on a sudden commandest an unwonted and unthought of thing, yea,
although Thou hast sometime forbidden it, and still for the time hidest
the reason of Thy command, and it be against the ordinance of some
society of men, who doubts but it is to be done, seeing that society
of men is just which serves Thee? But blessed are they who know Thy
commands! For all things were done by Thy servants; either to show forth
something needful for the present, or to foreshow things to come.

These things I being ignorant of, scoffed at those Thy holy servants and
prophets. And what gained I by scoffing at them, but to be scoffed at by
Thee, being insensibly and step by step drawn on to those follies, as
to believe that a fig-tree wept when it was plucked, and the tree, its
mother, shed milky tears? Which fig notwithstanding (plucked by some
other's, not his own, guilt) had some Manichaean saint eaten, and
mingled with his bowels, he should breathe out of it angels, yea, there
shall burst forth particles of divinity, at every moan or groan in his
prayer, which particles of the most high and true God had remained bound
in that fig, unless they had been set at liberty by the teeth or belly
of some "Elect" saint! And I, miserable, believed that more mercy was
to be shown to the fruits of the earth than men, for whom they were
created. For if any one an hungered, not a Manichaean, should ask for
any, that morsel would seem as it were condemned to capital punishment,
which should be given him.

And Thou sentest Thine hand from above, and drewest my soul out of that
profound darkness, my mother, Thy faithful one, weeping to Thee for me,
more than mothers weep the bodily deaths of their children. For she,
by that faith and spirit which she had from Thee, discerned the death
wherein I lay, and Thou heardest her, O Lord; Thou heardest her, and
despisedst not her tears, when streaming down, they watered the ground
under her eyes in every place where she prayed; yea Thou heardest her.
For whence was that dream whereby Thou comfortedst her; so that she
allowed me to live with her, and to eat at the same table in the
house, which she had begun to shrink from, abhorring and detesting
the blasphemies of my error? For she saw herself standing on a certain
wooden rule, and a shining youth coming towards her, cheerful and
smiling upon her, herself grieving, and overwhelmed with grief. But he
having (in order to instruct, as is their wont not to be instructed)
enquired of her the causes of her grief and daily tears, and she
answering that she was bewailing my perdition, he bade her rest
contented, and told her to look and observe, "That where she was, there
was I also." And when she looked, she saw me standing by her in the same
rule. Whence was this, but that Thine ears were towards her heart? O
Thou Good omnipotent, who so carest for every one of us, as if Thou
caredst for him only; and so for all, as if they were but one!

Whence was this also, that when she had told me this vision, and I would
fain bend it to mean, "That she rather should not despair of being one
day what I was"; she presently, without any hesitation, replies: "No;
for it was not told me that, 'where he, there thou also'; but 'where
thou, there he also'?" I confess to Thee, O Lord, that to the best of my
remembrance (and I have oft spoken of this), that Thy answer, through
my waking mother,--that she was not perplexed by the plausibility of my
false interpretation, and so quickly saw what was to be seen, and which
I certainly had not perceived before she spake,--even then moved me more
than the dream itself, by which a joy to the holy woman, to be fulfilled
so long after, was, for the consolation of her present anguish, so long
before foresignified. For almost nine years passed, in which I wallowed
in the mire of that deep pit, and the darkness of falsehood, often
assaying to rise, but dashed down the more grievously. All which time
that chaste, godly, and sober widow (such as Thou lovest), now more
cheered with hope, yet no whit relaxing in her weeping and mourning,
ceased not at all hours of her devotions to bewail my case unto Thee.
And her prayers entered into Thy presence; and yet Thou sufferedst me to
be yet involved and reinvolved in that darkness.

Thou gavest her meantime another answer, which I call to mind; for much
I pass by, hasting to those things which more press me to confess unto
Thee, and much I do not remember. Thou gavest her then another answer,
by a Priest of Thine, a certain Bishop brought up in Thy Church,
and well studied in Thy books. Whom when this woman had entreated to
vouchsafe to converse with me, refute my errors, unteach me ill things,
and teach me good things (for this he was wont to do, when he found
persons fitted to receive it), he refused, wisely, as I afterwards
perceived. For he answered, that I was yet unteachable, being puffed
up with the novelty of that heresy, and had already perplexed divers
unskilful persons with captious questions, as she had told him: "but
let him alone a while" (saith he), "only pray God for him, he will of
himself by reading find what that error is, and how great its impiety."
At the same time he told her, how himself, when a little one, had by his
seduced mother been consigned over to the Manichees, and had not
only read, but frequently copied out almost all, their books, and had
(without any argument or proof from any one) seen how much that sect was
to be avoided; and had avoided it. Which when he had said, and she would
not be satisfied, but urged him more, with entreaties and many tears,
that he would see me and discourse with me; he, a little displeased at
her importunity, saith, "Go thy ways and God bless thee, for it is not
possible that the son of these tears should perish." Which answer she
took (as she often mentioned in her conversations with me) as if it had
sounded from heaven.




\chapter{BOOK IV}


\start{F}{or} this space of nine years (from my nineteenth year to my
eight-and-twentieth) we lived seduced and seducing, deceived and
deceiving, in divers lusts; openly, by sciences which they call liberal;
secretly, with a false-named religion; here proud, there superstitious,
every where vain. Here, hunting after the emptiness of popular praise,
down even to theatrical applauses, and poetic prizes, and strifes for
grassy garlands, and the follies of shows, and the intemperance of
desires. There, desiring to be cleansed from these defilements, by
carrying food to those who were called "elect" and "holy," out of which,
in the workhouse of their stomachs, they should forge for us Angels
and Gods, by whom we might be cleansed. These things did I follow, and
practise with my friends, deceived by me, and with me. Let the arrogant
mock me, and such as have not been, to their soul's health, stricken and
cast down by Thee, O my God; but I would still confess to Thee mine own
shame in Thy praise. Suffer me, I beseech Thee, and give me grace to go
over in my present remembrance the wanderings of my forepassed time,
and to offer unto Thee the sacrifice of thanksgiving. For what am I to
myself without Thee, but a guide to mine own downfall? or what am I even
at the best, but an infant sucking the milk Thou givest, and feeding
upon Thee, the food that perisheth not? But what sort of man is any man,
seeing he is but a man? Let now the strong and the mighty laugh at us,
but let us poor and needy confess unto Thee.

In those years I taught rhetoric, and, overcome by cupidity, made sale
of a loquacity to overcome by. Yet I preferred (Lord, Thou knowest)
honest scholars (as they are accounted), and these I, without artifice,
taught artifices, not to be practised against the life of the guiltless,
though sometimes for the life of the guilty. And Thou, O God, from afar
perceivedst me stumbling in that slippery course, and amid much smoke
sending out some sparks of faithfulness, which I showed in that my
guidance of such as loved vanity, and sought after leasing, myself their
companion. In those years I had one,--not in that which is called
lawful marriage, but whom I had found out in a wayward passion, void of
understanding; yet but one, remaining faithful even to her; in whom I
in my own case experienced what difference there is betwixt the
self-restraint of the marriage-covenant, for the sake of issue, and
the bargain of a lustful love, where children are born against their
parents' will, although, once born, they constrain love.

I remember also, that when I had settled to enter the lists for a
theatrical prize, some wizard asked me what I would give him to win; but
I, detesting and abhorring such foul mysteries, answered, "Though the
garland were of imperishable gold, I would not suffer a fly to be
killed to gain me it." For he was to kill some living creatures in his
sacrifices, and by those honours to invite the devils to favour me. But
this ill also I rejected, not out of a pure love for Thee, O God of my
heart; for I knew not how to love Thee, who knew not how to conceive
aught beyond a material brightness. And doth not a soul, sighing after
such fictions, commit fornication against Thee, trust in things unreal,
and feed the wind? Still I would not forsooth have sacrifices offered
to devils for me, to whom I was sacrificing myself by that superstition.
For what else is it to feed the wind, but to feed them, that is by going
astray to become their pleasure and derision?

Those impostors then, whom they style Mathematicians, I consulted
without scruple; because they seemed to use no sacrifice, nor to pray to
any spirit for their divinations: which art, however, Christian and
true piety consistently rejects and condemns. For, it is a good thing to
confess unto Thee, and to say, Have mercy upon me, heal my soul, for I
have sinned against Thee; and not to abuse Thy mercy for a licence to
sin, but to remember the Lord's words, Behold, thou art made whole, sin
no more, lest a worse thing come unto thee. All which wholesome advice
they labour to destroy, saying, "The cause of thy sin is inevitably
determined in heaven"; and "This did Venus, or Saturn, or Mars":
that man, forsooth, flesh and blood, and proud corruption, might be
blameless; while the Creator and Ordainer of heaven and the stars is
to bear the blame. And who is He but our God? the very sweetness and
well-spring of righteousness, who renderest to every man according to
his works: and a broken and contrite heart wilt Thou not despise.

There was in those days a wise man, very skilful in physic, and renowned
therein, who had with his own proconsular hand put the Agonistic garland
upon my distempered head, but not as a physician: for this disease Thou
only curest, who resistest the proud, and givest grace to the humble.
But didst Thou fail me even by that old man, or forbear to heal my soul?
For having become more acquainted with him, and hanging assiduously and
fixedly on his speech (for though in simple terms, it was vivid, lively,
and earnest), when he had gathered by my discourse that I was given to
the books of nativity-casters, he kindly and fatherly advised me to cast
them away, and not fruitlessly bestow a care and diligence, necessary
for useful things, upon these vanities; saying, that he had in his
earliest years studied that art, so as to make it the profession whereby
he should live, and that, understanding Hippocrates, he could soon have
understood such a study as this; and yet he had given it over, and taken
to physic, for no other reason but that he found it utterly false;
and he, a grave man, would not get his living by deluding people. "But
thou," saith he, "hast rhetoric to maintain thyself by, so that thou
followest this of free choice, not of necessity: the more then oughtest
thou to give me credit herein, who laboured to acquire it so perfectly
as to get my living by it alone." Of whom when I had demanded, how then
could many true things be foretold by it, he answered me (as he could)
"that the force of chance, diffused throughout the whole order of
things, brought this about. For if when a man by haphazard opens the
pages of some poet, who sang and thought of something wholly different,
a verse oftentimes fell out, wondrously agreeable to the present
business: it were not to be wondered at, if out of the soul of man,
unconscious what takes place in it, by some higher instinct an answer
should be given, by hap, not by art, corresponding to the business and
actions of the demander."

And thus much, either from or through him, Thou conveyedst to me, and
tracedst in my memory, what I might hereafter examine for myself. But at
that time neither he, nor my dearest Nebridius, a youth singularly good
and of a holy fear, who derided the whole body of divination, could
persuade me to cast it aside, the authority of the authors swaying me
yet more, and as yet I had found no certain proof (such as I sought)
whereby it might without all doubt appear, that what had been truly
foretold by those consulted was the result of haphazard, not of the art
of the star-gazers.

In those years when I first began to teach rhetoric in my native town,
I had made one my friend, but too dear to me, from a community of
pursuits, of mine own age, and, as myself, in the first opening flower
of youth. He had grown up of a child with me, and we had been both
school-fellows and play-fellows. But he was not yet my friend as
afterwards, nor even then, as true friendship is; for true it cannot be,
unless in such as Thou cementest together, cleaving unto Thee, by that
love which is shed abroad in our hearts by the Holy Ghost, which is
given unto us. Yet was it but too sweet, ripened by the warmth of
kindred studies: for, from the true faith (which he as a youth had
not soundly and thoroughly imbibed), I had warped him also to those
superstitious and pernicious fables, for which my mother bewailed me.
With me he now erred in mind, nor could my soul be without him. But
behold Thou wert close on the steps of Thy fugitives, at once God of
vengeance, and Fountain of mercies, turning us to Thyself by wonderful
means; Thou tookest that man out of this life, when he had scarce filled
up one whole year of my friendship, sweet to me above all sweetness of
that my life.

Who can recount all Thy praises, which he hath felt in his one self?
What diddest Thou then, my God, and how unsearchable is the abyss of
Thy judgments? For long, sore sick of a fever, he lay senseless in
a death-sweat; and his recovery being despaired of, he was baptised,
unknowing; myself meanwhile little regarding, and presuming that his
soul would retain rather what it had received of me, not what was
wrought on his unconscious body. But it proved far otherwise: for he was
refreshed, and restored. Forthwith, as soon as I could speak with him
(and I could, so soon as he was able, for I never left him, and we hung
but too much upon each other), I essayed to jest with him, as though he
would jest with me at that baptism which he had received, when utterly
absent in mind and feeling, but had now understood that he had received.
But he so shrunk from me, as from an enemy; and with a wonderful and
sudden freedom bade me, as I would continue his friend, forbear such
language to him. I, all astonished and amazed, suppressed all my
emotions till he should grow well, and his health were strong enough for
me to deal with him as I would. But he was taken away from my frenzy,
that with Thee he might be preserved for my comfort; a few days after in
my absence, he was attacked again by the fever, and so departed.

At this grief my heart was utterly darkened; and whatever I beheld was
death. My native country was a torment to me, and my father's house a
strange unhappiness; and whatever I had shared with him, wanting him,
became a distracting torture. Mine eyes sought him every where, but he
was not granted them; and I hated all places, for that they had not him;
nor could they now tell me, "he is coming," as when he was alive and
absent. I became a great riddle to myself, and I asked my soul, why she
was so sad, and why she disquieted me sorely: but she knew not what to
answer me. And if I said, Trust in God, she very rightly obeyed me not;
because that most dear friend, whom she had lost, was, being man, both
truer and better than that phantasm she was bid to trust in. Only tears
were sweet to me, for they succeeded my friend, in the dearest of my
affections.

And now, Lord, these things are passed by, and time hath assuaged my
wound. May I learn from Thee, who art Truth, and approach the ear of my
heart unto Thy mouth, that Thou mayest tell me why weeping is sweet to
the miserable? Hast Thou, although present every where, cast away our
misery far from Thee? And Thou abidest in Thyself, but we are tossed
about in divers trials. And yet unless we mourned in Thine ears, we
should have no hope left. Whence then is sweet fruit gathered from the
bitterness of life, from groaning, tears, sighs, and complaints? Doth
this sweeten it, that we hope Thou hearest? This is true of prayer, for
therein is a longing to approach unto Thee. But is it also in grief for
a thing lost, and the sorrow wherewith I was then overwhelmed? For I
neither hoped he should return to life nor did I desire this with my
tears; but I wept only and grieved. For I was miserable, and had lost my
joy. Or is weeping indeed a bitter thing, and for very loathing of the
things which we before enjoyed, does it then, when we shrink from them,
please us?

But what speak I of these things? for now is no time to question, but to
confess unto Thee. Wretched I was; and wretched is every soul bound by
the friendship of perishable things; he is torn asunder when he loses
them, and then he feels the wretchedness which he had ere yet he lost
them. So was it then with me; I wept most bitterly, and found my repose
in bitterness. Thus was I wretched, and that wretched life I held dearer
than my friend. For though I would willingly have changed it, yet was I
more unwilling to part with it than with him; yea, I know not whether I
would have parted with it even for him, as is related (if not feigned)
of Pylades and Orestes, that they would gladly have died for each other
or together, not to live together being to them worse than death. But in
me there had arisen some unexplained feeling, too contrary to this, for
at once I loathed exceedingly to live and feared to die. I suppose, the
more I loved him, the more did I hate, and fear (as a most cruel enemy)
death, which had bereaved me of him: and I imagined it would speedily
make an end of all men, since it had power over him. Thus was it with
me, I remember. Behold my heart, O my God, behold and see into me; for
well I remember it, O my Hope, who cleansest me from the impurity of
such affections, directing mine eyes towards Thee, and plucking my feet
out of the snare. For I wondered that others, subject to death, did
live, since he whom I loved, as if he should never die, was dead; and I
wondered yet more that myself, who was to him a second self, could live,
he being dead. Well said one of his friend, "Thou half of my soul";
for I felt that my soul and his soul were "one soul in two bodies": and
therefore was my life a horror to me, because I would not live halved.
And therefore perchance I feared to die, lest he whom I had much loved
should die wholly.

O madness, which knowest not how to love men, like men! O foolish man
that I then was, enduring impatiently the lot of man! I fretted then,
sighed, wept, was distracted; had neither rest nor counsel. For I bore
about a shattered and bleeding soul, impatient of being borne by me, yet
where to repose it, I found not. Not in calm groves, not in games and
music, nor in fragrant spots, nor in curious banquetings, nor in the
pleasures of the bed and the couch; nor (finally) in books or poesy,
found it repose. All things looked ghastly, yea, the very light;
whatsoever was not what he was, was revolting and hateful, except
groaning and tears. For in those alone found I a little refreshment. But
when my soul was withdrawn from them a huge load of misery weighed
me down. To Thee, O Lord, it ought to have been raised, for Thee to
lighten; I knew it; but neither could nor would; the more, since, when I
thought of Thee, Thou wert not to me any solid or substantial thing. For
Thou wert not Thyself, but a mere phantom, and my error was my God. If
I offered to discharge my load thereon, that it might rest, it glided
through the void, and came rushing down again on me; and I had remained
to myself a hapless spot, where I could neither be, nor be from thence.
For whither should my heart flee from my heart? Whither should I
flee from myself? Whither not follow myself? And yet I fled out of my
country; for so should mine eyes less look for him, where they were not
wont to see him. And thus from Thagaste, I came to Carthage.

Times lose no time; nor do they roll idly by; through our senses they
work strange operations on the mind. Behold, they went and came day by
day, and by coming and going, introduced into my mind other imaginations
and other remembrances; and little by little patched me up again with my
old kind of delights, unto which that my sorrow gave way. And yet there
succeeded, not indeed other griefs, yet the causes of other griefs. For
whence had that former grief so easily reached my very inmost soul, but
that I had poured out my soul upon the dust, in loving one that must
die, as if he would never die? For what restored and refreshed me
chiefly was the solaces of other friends, with whom I did love, what
instead of Thee I loved; and this was a great fable, and protracted lie,
by whose adulterous stimulus, our soul, which lay itching in our ears,
was being defiled. But that fable would not die to me, so oft as any of
my friends died. There were other things which in them did more take my
mind; to talk and jest together, to do kind offices by turns; to read
together honied books; to play the fool or be earnest together; to
dissent at times without discontent, as a man might with his own self;
and even with the seldomness of these dissentings, to season our more
frequent consentings; sometimes to teach, and sometimes learn; long for
the absent with impatience; and welcome the coming with joy. These and
the like expressions, proceeding out of the hearts of those that loved
and were loved again, by the countenance, the tongue, the eyes, and
a thousand pleasing gestures, were so much fuel to melt our souls
together, and out of many make but one.

This is it that is loved in friends; and so loved, that a man's
conscience condemns itself, if he love not him that loves him again, or
love not again him that loves him, looking for nothing from his person
but indications of his love. Hence that mourning, if one die, and
darkenings of sorrows, that steeping of the heart in tears, all
sweetness turned to bitterness; and upon the loss of life of the dying,
the death of the living. Blessed whoso loveth Thee, and his friend in
Thee, and his enemy for Thee. For he alone loses none dear to him, to
whom all are dear in Him who cannot be lost. And who is this but our
God, the God that made heaven and earth, and filleth them, because by
filling them He created them? Thee none loseth, but who leaveth. And
who leaveth Thee, whither goeth or whither fleeth he, but from Thee
well-pleased, to Thee displeased? For where doth he not find Thy law in
his own punishment? And Thy law is truth, and truth Thou.

Turn us, O God of Hosts, show us Thy countenance, and we shall be whole.
For whithersoever the soul of man turns itself, unless toward Thee, it
is riveted upon sorrows, yea though it is riveted on things beautiful.
And yet they, out of Thee, and out of the soul, were not, unless they
were from Thee. They rise, and set; and by rising, they begin as it were
to be; they grow, that they may be perfected; and perfected, they wax
old and wither; and all grow not old, but all wither. So then when they
rise and tend to be, the more quickly they grow that they may be, so
much the more they haste not to be. This is the law of them. Thus much
has Thou allotted them, because they are portions of things, which
exist not all at once, but by passing away and succeeding, they together
complete that universe, whereof they are portions. And even thus is our
speech completed by signs giving forth a sound: but this again is not
perfected unless one word pass away when it hath sounded its part, that
another may succeed. Out of all these things let my soul praise Thee,
O God, Creator of all; yet let not my soul be riveted unto these things
with the glue of love, through the senses of the body. For they go
whither they were to go, that they might not be; and they rend her with
pestilent longings, because she longs to be, yet loves to repose in what
she loves. But in these things is no place of repose; they abide not,
they flee; and who can follow them with the senses of the flesh? yea,
who can grasp them, when they are hard by? For the sense of the flesh is
slow, because it is the sense of the flesh; and thereby is it bounded.
It sufficeth for that it was made for; but it sufficeth not to stay
things running their course from their appointed starting-place to the
end appointed. For in Thy Word, by which they are created, they hear
their decree, "hence and hitherto."

Be not foolish, O my soul, nor become deaf in the ear of thine heart
with the tumult of thy folly. Hearken thou too.

The Word itself calleth thee to return: and there is the place of rest
imperturbable, where love is not forsaken, if itself forsaketh not.
Behold, these things pass away, that others may replace them, and so
this lower universe be completed by all his parts. But do I depart any
whither? saith the Word of God. There fix thy dwelling, trust there
whatsoever thou hast thence, O my soul, at least now thou art tired out
with vanities. Entrust Truth, whatsoever thou hast from the Truth, and
thou shalt lose nothing; and thy decay shall bloom again, and all thy
diseases be healed, and thy mortal parts be reformed and renewed, and
bound around thee: nor shall they lay thee whither themselves descend;
but they shall stand fast with thee, and abide for ever before God, Who
abideth and standeth fast for ever.

Why then be perverted and follow thy flesh? Be it converted and follow
thee. Whatever by her thou hast sense of, is in part; and the whole,
whereof these are parts, thou knowest not; and yet they delight thee.
But had the sense of thy flesh a capacity for comprehending the whole,
and not itself also, for thy punishment, been justly restricted to
a part of the whole, thou wouldest, that whatsoever existeth at this
present, should pass away, that so the whole might better please thee.
For what we speak also, by the same sense of the flesh thou hearest; yet
wouldest not thou have the syllables stay, but fly away, that others may
come, and thou hear the whole. And so ever, when any one thing is made
up of many, all of which do not exist together, all collectively would
please more than they do severally, could all be perceived collectively.
But far better than these is He who made all; and He is our God, nor
doth He pass away, for neither doth aught succeed Him.

If bodies please thee, praise God on occasion of them, and turn back
thy love upon their Maker; lest in these things which please thee, thou
displease. If souls please thee, be they loved in God: for they too are
mutable, but in Him are they firmly stablished; else would they pass,
and pass away. In Him then be they beloved; and carry unto Him along
with thee what souls thou canst, and say to them, "Him let us love, Him
let us love: He made these, nor is He far off. For He did not make them,
and so depart, but they are of Him, and in Him. See there He is, where
truth is loved. He is within the very heart, yet hath the heart strayed
from Him. Go back into your heart, ye transgressors, and cleave fast to
Him that made you. Stand with Him, and ye shall stand fast. Rest in Him,
and ye shall be at rest. Whither go ye in rough ways? Whither go ye?
The good that you love is from Him; but it is good and pleasant through
reference to Him, and justly shall it be embittered, because unjustly is
any thing loved which is from Him, if He be forsaken for it. To what end
then would ye still and still walk these difficult and toilsome ways?
There is no rest, where ye seek it. Seek what ye seek; but it is not
there where ye seek. Ye seek a blessed life in the land of death; it is
not there. For how should there be a blessed life where life itself is
not?

"But our true Life came down hither, and bore our death, and slew him,
out of the abundance of His own life: and He thundered, calling aloud to
us to return hence to Him into that secret place, whence He came forth
to us, first into the Virgin's womb, wherein He espoused the human
creation, our mortal flesh, that it might not be for ever mortal, and
thence like a bridegroom coming out of his chamber, rejoicing as a giant
to run his course. For He lingered not, but ran, calling aloud by words,
deeds, death, life, descent, ascension; crying aloud to us to return
unto Him. And He departed from our eyes, that we might return into our
heart, and there find Him. For He departed, and lo, He is here. He would
not be long with us, yet left us not; for He departed thither, whence
He never parted, because the world was made by Him. And in this world
He was, and into this world He came to save sinners, unto whom my soul
confesseth, and He healeth it, for it hath sinned against Him. O ye sons
of men, how long so slow of heart? Even now, after the descent of Life
to you, will ye not ascend and live? But whither ascend ye, when ye are
on high, and set your mouth against the heavens? Descend, that ye may
ascend, and ascend to God. For ye have fallen, by ascending against
Him." Tell them this, that they may weep in the valley of tears, and
so carry them up with thee unto God; because out of His spirit thou
speakest thus unto them, if thou speakest, burning with the fire of
charity.

These things I then knew not, and I loved these lower beauties, and I
was sinking to the very depths, and to my friends I said, "Do we love
any thing but the beautiful? What then is the beautiful? and what is
beauty? What is it that attracts and wins us to the things we love? for
unless there were in them a grace and beauty, they could by no
means draw us unto them." And I marked and perceived that in bodies
themselves, there was a beauty, from their forming a sort of whole, and
again, another from apt and mutual correspondence, as of a part of
the body with its whole, or a shoe with a foot, and the like. And this
consideration sprang up in my mind, out of my inmost heart, and I wrote
"on the fair and fit," I think, two or three books. Thou knowest, O
Lord, for it is gone from me; for I have them not, but they are strayed
from me, I know not how.

But what moved me, O Lord my God, to dedicate these books unto Hierius,
an orator of Rome, whom I knew not by face, but loved for the fame
of his learning which was eminent in him, and some words of his I had
heard, which pleased me? But more did he please me, for that he pleased
others, who highly extolled him, amazed that out of a Syrian, first
instructed in Greek eloquence, should afterwards be formed a wonderful
Latin orator, and one most learned in things pertaining unto philosophy.
One is commended, and, unseen, he is loved: doth this love enter the
heart of the hearer from the mouth of the commender? Not so. But by one
who loveth is another kindled. For hence he is loved who is commended,
when the commender is believed to extol him with an unfeigned heart;
that is, when one that loves him, praises him.

For so did I then love men, upon the judgment of men, not Thine, O my
God, in Whom no man is deceived. But yet why not for qualities, like
those of a famous charioteer, or fighter with beasts in the theatre,
known far and wide by a vulgar popularity, but far otherwise, and
earnestly, and so as I would be myself commended? For I would not be
commended or loved, as actors are (though I myself did commend and love
them), but had rather be unknown, than so known; and even hated, than
so loved. Where now are the impulses to such various and divers kinds of
loves laid up in one soul? Why, since we are equally men, do I love
in another what, if I did not hate, I should not spurn and cast from
myself? For it holds not, that as a good horse is loved by him, who
would not, though he might, be that horse, therefore the same may be
said of an actor, who shares our nature. Do I then love in a man, what I
hate to be, who am a man? Man himself is a great deep, whose very hairs
Thou numberest, O Lord, and they fall not to the ground without Thee.
And yet are the hairs of his head easier to be numbered than his
feelings, and the beatings of his heart.

But that orator was of that sort whom I loved, as wishing to be myself
such; and I erred through a swelling pride, and was tossed about with
every wind, but yet was steered by Thee, though very secretly. And
whence do I know, and whence do I confidently confess unto Thee, that
I had loved him more for the love of his commenders, than for the very
things for which he was commended? Because, had he been unpraised, and
these self-same men had dispraised him, and with dispraise and contempt
told the very same things of him, I had never been so kindled and
excited to love him. And yet the things had not been other, nor he
himself other; but only the feelings of the relators. See where the
impotent soul lies along, that is not yet stayed up by the solidity
of truth! Just as the gales of tongues blow from the breast of the
opinionative, so is it carried this way and that, driven forward and
backward, and the light is overclouded to it, and the truth unseen. And
lo, it is before us. And it was to me a great matter, that my discourse
and labours should be known to that man: which should he approve, I were
the more kindled; but if he disapproved, my empty heart, void of Thy
solidity, had been wounded. And yet the "fair and fit," whereon I wrote
to him, I dwelt on with pleasure, and surveyed it, and admired it,
though none joined therein.

But I saw not yet, whereon this weighty matter turned in Thy wisdom,
O Thou Omnipotent, who only doest wonders; and my mind ranged through
corporeal forms; and "fair," I defined and distinguished what is so
in itself, and "fit," whose beauty is in correspondence to some other
thing: and this I supported by corporeal examples. And I turned to
the nature of the mind, but the false notion which I had of spiritual
things, let me not see the truth. Yet the force of truth did of itself
flash into mine eyes, and I turned away my panting soul from incorporeal
substance to lineaments, and colours, and bulky magnitudes. And not
being able to see these in the mind, I thought I could not see my mind.
And whereas in virtue I loved peace, and in viciousness I abhorred
discord; in the first I observed a unity, but in the other, a sort
of division. And in that unity I conceived the rational soul, and the
nature of truth and of the chief good to consist; but in this division
I miserably imagined there to be some unknown substance of irrational
life, and the nature of the chief evil, which should not only be a
substance, but real life also, and yet not derived from Thee, O my God,
of whom are all things. And yet that first I called a Monad, as it had
been a soul without sex; but the latter a Duad;--anger, in deeds of
violence, and in flagitiousness, lust; not knowing whereof I spake. For
I had not known or learned that neither was evil a substance, nor our
soul that chief and unchangeable good.

For as deeds of violence arise, if that emotion of the soul be
corrupted, whence vehement action springs, stirring itself insolently
and unrulily; and lusts, when that affection of the soul is ungoverned,
whereby carnal pleasures are drunk in, so do errors and false opinions
defile the conversation, if the reasonable soul itself be corrupted; as
it was then in me, who knew not that it must be enlightened by another
light, that it may be partaker of truth, seeing itself is not that
nature of truth. For Thou shalt light my candle, O Lord my God, Thou
shalt enlighten my darkness: and of Thy fulness have we all received,
for Thou art the true light that lighteth every man that cometh into the
world; for in Thee there is no variableness, neither shadow of change.

But I pressed towards Thee, and was thrust from Thee, that I might taste
of death: for thou resistest the proud. But what prouder, than for me
with a strange madness to maintain myself to be that by nature which
Thou art? For whereas I was subject to change (so much being manifest
to me, my very desire to become wise, being the wish, of worse to become
better), yet chose I rather to imagine Thee subject to change, and
myself not to be that which Thou art. Therefore I was repelled by Thee,
and Thou resistedst my vain stiffneckedness, and I imagined corporeal
forms, and, myself flesh, I accused flesh; and, a wind that passeth
away, I returned not to Thee, but I passed on and on to things which
have no being, neither in Thee, nor in me, nor in the body. Neither were
they created for me by Thy truth, but by my vanity devised out of
things corporeal. And I was wont to ask Thy faithful little ones, my
fellow-citizens (from whom, unknown to myself, I stood exiled), I was
wont, prating and foolishly, to ask them, "Why then doth the soul err
which God created?" But I would not be asked, "Why then doth God
err?" And I maintained that Thy unchangeable substance did err upon
constraint, rather than confess that my changeable substance had gone
astray voluntarily, and now, in punishment, lay in error.

I was then some six or seven and twenty years old when I wrote those
volumes; revolving within me corporeal fictions, buzzing in the ears
of my heart, which I turned, O sweet truth, to thy inward melody,
meditating on the "fair and fit," and longing to stand and hearken to
Thee, and to rejoice greatly at the Bridegroom's voice, but could not;
for by the voices of mine own errors, I was hurried abroad, and through
the weight of my own pride, I was sinking into the lowest pit. For Thou
didst not make me to hear joy and gladness, nor did the bones exult
which were not yet humbled.

And what did it profit me, that scarce twenty years old, a book of
Aristotle, which they call the ten Predicaments, falling into my hands
(on whose very name I hung, as on something great and divine, so often
as my rhetoric master of Carthage, and others, accounted learned,
mouthed it with cheeks bursting with pride), I read and understood it
unaided? And on my conferring with others, who said that they scarcely
understood it with very able tutors, not only orally explaining it, but
drawing many things in sand, they could tell me no more of it than I had
learned, reading it by myself. And the book appeared to me to speak very
clearly of substances, such as "man," and of their qualities, as the
figure of a man, of what sort it is; and stature, how many feet high;
and his relationship, whose brother he is; or where placed; or when
born; or whether he stands or sits; or be shod or armed; or does, or
suffers anything; and all the innumerable things which might be ranged
under these nine Predicaments, of which I have given some specimens, or
under that chief Predicament of Substance.

What did all this further me, seeing it even hindered me? when,
imagining whatever was, was comprehended under those ten Predicaments,
I essayed in such wise to understand, O my God, Thy wonderful and
unchangeable Unity also, as if Thou also hadst been subjected to Thine
own greatness or beauty; so that (as in bodies) they should exist in
Thee, as their subject: whereas Thou Thyself art Thy greatness and
beauty; but a body is not great or fair in that it is a body, seeing
that, though it were less great or fair, it should notwithstanding be
a body. But it was falsehood which of Thee I conceived, not truth,
fictions of my misery, not the realities of Thy blessedness. For Thou
hadst commanded, and it was done in me, that the earth should bring
forth briars and thorns to me, and that in the sweat of my brows I
should eat my bread.

And what did it profit me, that all the books I could procure of the
so-called liberal arts, I, the vile slave of vile affections, read by
myself, and understood? And I delighted in them, but knew not whence
came all, that therein was true or certain. For I had my back to the
light, and my face to the things enlightened; whence my face, with which
I discerned the things enlightened, itself was not enlightened.
Whatever was written, either on rhetoric, or logic, geometry, music,
and arithmetic, by myself without much difficulty or any instructor,
I understood, Thou knowest, O Lord my God; because both quickness of
understanding, and acuteness in discerning, is Thy gift: yet did I not
thence sacrifice to Thee. So then it served not to my use, but rather
to my perdition, since I went about to get so good a portion of my
substance into my own keeping; and I kept not my strength for Thee, but
wandered from Thee into a far country, to spend it upon harlotries. For
what profited me good abilities, not employed to good uses? For I felt
not that those arts were attained with great difficulty, even by the
studious and talented, until I attempted to explain them to such; when
he most excelled in them who followed me not altogether slowly.

But what did this further me, imagining that Thou, O Lord God, the
Truth, wert a vast and bright body, and I a fragment of that body?
Perverseness too great! But such was I. Nor do I blush, O my God, to
confess to Thee Thy mercies towards me, and to call upon Thee, who
blushed not then to profess to men my blasphemies, and to bark against
Thee. What profited me then my nimble wit in those sciences and all
those most knotty volumes, unravelled by me, without aid from human
instruction; seeing I erred so foully, and with such sacrilegious
shamefulness, in the doctrine of piety? Or what hindrance was a far
slower wit to Thy little ones, since they departed not far from Thee,
that in the nest of Thy Church they might securely be fledged, and
nourish the wings of charity, by the food of a sound faith. O Lord our
God, under the shadow of Thy wings let us hope; protect us, and carry
us. Thou wilt carry us both when little, and even to hoar hairs wilt
Thou carry us; for our firmness, when it is Thou, then is it firmness;
but when our own, it is infirmity. Our good ever lives with Thee;
from which when we turn away, we are turned aside. Let us now, O Lord,
return, that we may not be overturned, because with Thee our good lives
without any decay, which good art Thou; nor need we fear, lest there
be no place whither to return, because we fell from it: for through our
absence, our mansion fell not--Thy eternity.




\chapter{BOOK V}


\start{A}{ccept} the sacrifice of my confessions from the ministry of my tongue,
which Thou hast formed and stirred up to confess unto Thy name. Heal
Thou all my bones, and let them say, O Lord, who is like unto Thee? For
he who confesses to Thee doth not teach Thee what takes place within
him; seeing a closed heart closes not out Thy eye, nor can man's
hard-heartedness thrust back Thy hand: for Thou dissolvest it at Thy
will in pity or in vengeance, and nothing can hide itself from Thy heat.
But let my soul praise Thee, that it may love Thee; and let it confess
Thy own mercies to Thee, that it may praise Thee. Thy whole creation
ceaseth not, nor is silent in Thy praises; neither the spirit of man
with voice directed unto Thee, nor creation animate or inanimate, by the
voice of those who meditate thereon: that so our souls may from their
weariness arise towards Thee, leaning on those things which Thou hast
created, and passing on to Thyself, who madest them wonderfully; and
there is refreshment and true strength.

Let the restless, the godless, depart and flee from Thee; yet Thou seest
them, and dividest the darkness. And behold, the universe with them is
fair, though they are foul. And how have they injured Thee? or how have
they disgraced Thy government, which, from the heaven to this lowest
earth, is just and perfect? For whither fled they, when they fled from
Thy presence? or where dost not Thou find them? But they fled, that they
might not see Thee seeing them, and, blinded, might stumble against Thee
(because Thou forsakest nothing Thou hast made); that the unjust, I say,
might stumble upon Thee, and justly be hurt; withdrawing themselves from
thy gentleness, and stumbling at Thy uprightness, and falling upon their
own ruggedness. Ignorant, in truth, that Thou art every where, Whom no
place encompasseth! and Thou alone art near, even to those that remove
far from Thee. Let them then be turned, and seek Thee; because not as
they have forsaken their Creator, hast Thou forsaken Thy creation. Let
them be turned and seek Thee; and behold, Thou art there in their heart,
in the heart of those that confess to Thee, and cast themselves upon
Thee, and weep in Thy bosom, after all their rugged ways. Then dost
Thou gently wipe away their tears, and they weep the more, and joy
in weeping; even for that Thou, Lord,--not man of flesh and blood,
but--Thou, Lord, who madest them, re-makest and comfortest them. But
where was I, when I was seeking Thee? And Thou wert before me, but I had
gone away from Thee; nor did I find myself, how much less Thee!

I would lay open before my God that nine-and-twentieth year of mine
age. There had then come to Carthage a certain Bishop of the Manichees,
Faustus by name, a great snare of the Devil, and many were entangled
by him through that lure of his smooth language: which though I did
commend, yet could I separate from the truth of the things which I was
earnest to learn: nor did I so much regard the service of oratory as the
science which this Faustus, so praised among them, set before me to
feed upon. Fame had before bespoken him most knowing in all valuable
learning, and exquisitely skilled in the liberal sciences. And since I
had read and well remembered much of the philosophers, I compared some
things of theirs with those long fables of the Manichees, and found the
former the more probable; even although they could only prevail so far
as to make judgment of this lower world, the Lord of it they could by
no means find out. For Thou art great, O Lord, and hast respect unto the
humble, but the proud Thou beholdest afar off. Nor dost Thou draw near,
but to the contrite in heart, nor art found by the proud, no, not though
by curious skill they could number the stars and the sand, and measure
the starry heavens, and track the courses of the planets.

For with their understanding and wit, which Thou bestowedst on them,
they search out these things; and much have they found out; and
foretold, many years before, eclipses of those luminaries, the sun
and moon,--what day and hour, and how many digits,--nor did their
calculation fail; and it came to pass as they foretold; and they wrote
down the rules they had found out, and these are read at this day, and
out of them do others foretell in what year and month of the year, and
what day of the month, and what hour of the day, and what part of its
light, moon or sun is to be eclipsed, and so it shall be, as it is
foreshowed. At these things men, that know not this art, marvel and are
astonished, and they that know it, exult, and are puffed up; and by
an ungodly pride departing from Thee, and failing of Thy light, they
foresee a failure of the sun's light, which shall be, so long before,
but see not their own, which is. For they search not religiously whence
they have the wit, wherewith they search out this. And finding that Thou
madest them, they give not themselves up to Thee, to preserve what Thou
madest, nor sacrifice to Thee what they have made themselves; nor slay
their own soaring imaginations, as fowls of the air, nor their own
diving curiosities (wherewith, like the fishes of the sea, they wander
over the unknown paths of the abyss), nor their own luxuriousness, as
beasts of the field, that Thou, Lord, a consuming fire, mayest burn up
those dead cares of theirs, and re-create themselves immortally.

But they knew not the way, Thy Word, by Whom Thou madest these things
which they number, and themselves who number, and the sense whereby
they perceive what they number, and the understanding, out of which they
number; or that of Thy wisdom there is no number. But the Only Begotten
is Himself made unto us wisdom, and righteousness, and sanctification,
and was numbered among us, and paid tribute unto Caesar. They knew not
this way whereby to descend to Him from themselves, and by Him ascend
unto Him. They knew not this way, and deemed themselves exalted amongst
the stars and shining; and behold, they fell upon the earth, and their
foolish heart was darkened. They discourse many things truly concerning
the creature; but Truth, Artificer of the creature, they seek not
piously, and therefore find Him not; or if they find Him, knowing Him
to be God, they glorify Him not as God, neither are thankful, but
become vain in their imaginations, and profess themselves to be wise,
attributing to themselves what is Thine; and thereby with most perverse
blindness, study to impute to Thee what is their own, forging lies of
Thee who art the Truth, and changing the glory of uncorruptible God
into an image made like corruptible man, and to birds, and four-footed
beasts, and creeping things, changing Thy truth into a lie, and
worshipping and serving the creature more than the Creator.

Yet many truths concerning the creature retained I from these men, and
saw the reason thereof from calculations, the succession of times, and
the visible testimonies of the stars; and compared them with the saying
of Manichaeus, which in his frenzy he had written most largely on these
subjects; but discovered not any account of the solstices, or equinoxes,
or the eclipses of the greater lights, nor whatever of this sort I
had learned in the books of secular philosophy. But I was commanded to
believe; and yet it corresponded not with what had been established by
calculations and my own sight, but was quite contrary.

Doth then, O Lord God of truth, whoso knoweth these things, therefore
please Thee? Surely unhappy is he who knoweth all these, and knoweth not
Thee: but happy whoso knoweth Thee, though he know not these. And whoso
knoweth both Thee and them is not the happier for them, but for Thee
only, if, knowing Thee, he glorifies Thee as God, and is thankful, and
becomes not vain in his imaginations. For as he is better off who knows
how to possess a tree, and return thanks to Thee for the use thereof,
although he know not how many cubits high it is, or how wide it spreads,
than he that can measure it, and count all its boughs, and neither owns
it, nor knows or loves its Creator: so a believer, whose all this world
of wealth is, and who having nothing, yet possesseth all things, by
cleaving unto Thee, whom all things serve, though he know not even
the circles of the Great Bear, yet is it folly to doubt but he is in a
better state than one who can measure the heavens, and number the stars,
and poise the elements, yet neglecteth Thee who hast made all things in
number, weight, and measure.

But yet who bade that Manichaeus write on these things also, skill in
which was no element of piety? For Thou hast said to man, Behold
piety and wisdom; of which he might be ignorant, though he had perfect
knowledge of these things; but these things, since, knowing not, he most
impudently dared to teach, he plainly could have no knowledge of piety.
For it is vanity to make profession of these worldly things even when
known; but confession to Thee is piety. Wherefore this wanderer to this
end spake much of these things, that convicted by those who had truly
learned them, it might be manifest what understanding he had in the
other abstruser things. For he would not have himself meanly thought of,
but went about to persuade men, "That the Holy Ghost, the Comforter and
Enricher of Thy faithful ones, was with plenary authority personally
within him." When then he was found out to have taught falsely of the
heaven and stars, and of the motions of the sun and moon (although these
things pertain not to the doctrine of religion), yet his sacrilegious
presumption would become evident enough, seeing he delivered things
which not only he knew not, but which were falsified, with so mad a
vanity of pride, that he sought to ascribe them to himself, as to a
divine person.

For when I hear any Christian brother ignorant of these things, and
mistaken on them, I can patiently behold such a man holding his opinion;
nor do I see that any ignorance as to the position or character of the
corporeal creation can injure him, so long as he doth not believe any
thing unworthy of Thee, O Lord, the Creator of all. But it doth injure
him, if he imagine it to pertain to the form of the doctrine of piety,
and will yet affirm that too stiffly whereof he is ignorant. And yet
is even such an infirmity, in the infancy of faith, borne by our mother
Charity, till the new-born may grow up unto a perfect man, so as not
to be carried about with every wind of doctrine. But in him who in such
wise presumed to be the teacher, source, guide, chief of all whom he
could so persuade, that whoso followed him thought that he followed,
not a mere man, but Thy Holy Spirit; who would not judge that so great
madness, when once convicted of having taught any thing false, were
to be detested and utterly rejected? But I had not as yet clearly
ascertained whether the vicissitudes of longer and shorter days and
nights, and of day and night itself, with the eclipses of the greater
lights, and whatever else of the kind I had read of in other books,
might be explained consistently with his sayings; so that, if they by
any means might, it should still remain a question to me whether it
were so or no; but I might, on account of his reputed sanctity, rest my
credence upon his authority.

And for almost all those nine years, wherein with unsettled mind I had
been their disciple, I had longed but too intensely for the coming of
this Faustus. For the rest of the sect, whom by chance I had lighted
upon, when unable to solve my objections about these things, still held
out to me the coming of this Faustus, by conference with whom these
and greater difficulties, if I had them, were to be most readily and
abundantly cleared. When then he came, I found him a man of pleasing
discourse, and who could speak fluently and in better terms, yet still
but the self-same things which they were wont to say. But what availed
the utmost neatness of the cup-bearer to my thirst for a more precious
draught? Mine ears were already cloyed with the like, nor did they seem
to me therefore better, because better said; nor therefore true, because
eloquent; nor the soul therefore wise, because the face was comely,
and the language graceful. But they who held him out to me were no good
judges of things; and therefore to them he appeared understanding and
wise, because in words pleasing. I felt however that another sort of
people were suspicious even of truth, and refused to assent to it, if
delivered in a smooth and copious discourse. But Thou, O my God, hadst
already taught me by wonderful and secret ways, and therefore I believe
that Thou taughtest me, because it is truth, nor is there besides Thee
any teacher of truth, where or whencesoever it may shine upon us. Of
Thyself therefore had I now learned, that neither ought any thing to
seem to be spoken truly, because eloquently; nor therefore falsely,
because the utterance of the lips is inharmonious; nor, again, therefore
true, because rudely delivered; nor therefore false, because the
language is rich; but that wisdom and folly are as wholesome and
unwholesome food; and adorned or unadorned phrases as courtly or country
vessels; either kind of meats may be served up in either kind of dishes.

That greediness then, wherewith I had of so long time expected that man,
was delighted verily with his action and feeling when disputing, and his
choice and readiness of words to clothe his ideas. I was then delighted,
and, with many others and more than they, did I praise and extol him.
It troubled me, however, that in the assembly of his auditors, I was not
allowed to put in and communicate those questions that troubled me,
in familiar converse with him. Which when I might, and with my friends
began to engage his ears at such times as it was not unbecoming for him
to discuss with me, and had brought forward such things as moved me; I
found him first utterly ignorant of liberal sciences, save grammar, and
that but in an ordinary way. But because he had read some of Tully's
Orations, a very few books of Seneca, some things of the poets, and such
few volumes of his own sect as were written in Latin and neatly, and
was daily practised in speaking, he acquired a certain eloquence, which
proved the more pleasing and seductive because under the guidance of a
good wit, and with a kind of natural gracefulness. Is it not thus, as I
recall it, O Lord my God, Thou judge of my conscience? before Thee is
my heart, and my remembrance, Who didst at that time direct me by the
hidden mystery of Thy providence, and didst set those shameful errors of
mine before my face, that I might see and hate them.

For after it was clear that he was ignorant of those arts in which I
thought he excelled, I began to despair of his opening and solving the
difficulties which perplexed me (of which indeed however ignorant, he
might have held the truths of piety, had he not been a Manichee). For
their books are fraught with prolix fables, of the heaven, and stars,
sun, and moon, and I now no longer thought him able satisfactorily to
decide what I much desired, whether, on comparison of these things with
the calculations I had elsewhere read, the account given in the books of
Manichaeus were preferable, or at least as good. Which when I proposed
to be considered and discussed, he, so far modestly, shrunk from the
burthen. For he knew that he knew not these things, and was not ashamed
to confess it. For he was not one of those talking persons, many of whom
I had endured, who undertook to teach me these things, and said nothing.
But this man had a heart, though not right towards Thee, yet neither
altogether treacherous to himself. For he was not altogether ignorant of
his own ignorance, nor would he rashly be entangled in a dispute, whence
he could neither retreat nor extricate himself fairly. Even for this I
liked him the better. For fairer is the modesty of a candid mind, than
the knowledge of those things which I desired; and such I found him, in
all the more difficult and subtile questions.

My zeal for the writings of Manichaeus being thus blunted, and
despairing yet more of their other teachers, seeing that in divers
things which perplexed me, he, so renowned among them, had so turned
out; I began to engage with him in the study of that literature, on
which he also was much set (and which as rhetoric-reader I was at that
time teaching young students at Carthage), and to read with him, either
what himself desired to hear, or such as I judged fit for his genius.
But all my efforts whereby I had purposed to advance in that sect,
upon knowledge of that man, came utterly to an end; not that I detached
myself from them altogether, but as one finding nothing better, I had
settled to be content meanwhile with what I had in whatever way fallen
upon, unless by chance something more eligible should dawn upon me.
Thus, that Faustus, to so many a snare of death, had now neither willing
nor witting it, begun to loosen that wherein I was taken. For Thy hands,
O my God, in the secret purpose of Thy providence, did not forsake my
soul; and out of my mother's heart's blood, through her tears night and
day poured out, was a sacrifice offered for me unto Thee; and Thou didst
deal with me by wondrous ways. Thou didst it, O my God: for the steps
of a man are ordered by the Lord, and He shall dispose his way. Or how
shall we obtain salvation, but from Thy hand, re-making what it made?

Thou didst deal with me, that I should be persuaded to go to Rome, and
to teach there rather, what I was teaching at Carthage. And how I was
persuaded to this, I will not neglect to confess to Thee; because herein
also the deepest recesses of Thy wisdom, and Thy most present mercy to
us, must be considered and confessed. I did not wish therefore to go to
Rome, because higher gains and higher dignities were warranted me by my
friends who persuaded me to this (though even these things had at that
time an influence over my mind), but my chief and almost only reason
was, that I heard that young men studied there more peacefully, and were
kept quiet under a restraint of more regular discipline; so that they
did not, at their pleasures, petulantly rush into the school of
one whose pupils they were not, nor were even admitted without his
permission. Whereas at Carthage there reigns among the scholars a most
disgraceful and unruly licence. They burst in audaciously, and
with gestures almost frantic, disturb all order which any one hath
established for the good of his scholars. Divers outrages they commit,
with a wonderful stolidity, punishable by law, did not custom uphold
them; that custom evincing them to be the more miserable, in that they
now do as lawful what by Thy eternal law shall never be lawful; and they
think they do it unpunished, whereas they are punished with the very
blindness whereby they do it, and suffer incomparably worse than what
they do. The manners then which, when a student, I would not make my
own, I was fain as a teacher to endure in others: and so I was well
pleased to go where, all that knew it, assured me that the like was not
done. But Thou, my refuge and my portion in the land of the living;
that I might change my earthly dwelling for the salvation of my soul,
at Carthage didst goad me, that I might thereby be torn from it; and at
Rome didst proffer me allurements, whereby I might be drawn thither,
by men in love with a dying life, the one doing frantic, the other
promising vain, things; and, to correct my steps, didst secretly use
their and my own perverseness. For both they who disturbed my quiet were
blinded with a disgraceful frenzy, and they who invited me elsewhere
savoured of earth. And I, who here detested real misery, was there
seeking unreal happiness.

But why I went hence, and went thither, Thou knewest, O God, yet
showedst it neither to me, nor to my mother, who grievously bewailed my
journey, and followed me as far as the sea. But I deceived her, holding
me by force, that either she might keep me back or go with me, and I
feigned that I had a friend whom I could not leave, till he had a fair
wind to sail. And I lied to my mother, and such a mother, and escaped:
for this also hast Thou mercifully forgiven me, preserving me, thus full
of execrable defilements, from the waters of the sea, for the water of
Thy Grace; whereby when I was cleansed, the streams of my mother's eyes
should be dried, with which for me she daily watered the ground under
her face. And yet refusing to return without me, I scarcely persuaded
her to stay that night in a place hard by our ship, where was an Oratory
in memory of the blessed Cyprian. That night I privily departed, but she
was not behind in weeping and prayer. And what, O Lord, was she with so
many tears asking of Thee, but that Thou wouldest not suffer me to sail?
But Thou, in the depth of Thy counsels and hearing the main point of her
desire, regardest not what she then asked, that Thou mightest make me
what she ever asked. The wind blew and swelled our sails, and withdrew
the shore from our sight; and she on the morrow was there, frantic with
sorrow, and with complaints and groans filled Thine ears, Who didst then
disregard them; whilst through my desires, Thou wert hurrying me to end
all desire, and the earthly part of her affection to me was chastened
by the allotted scourge of sorrows. For she loved my being with her, as
mothers do, but much more than many; and she knew not how great joy Thou
wert about to work for her out of my absence. She knew not; therefore
did she weep and wail, and by this agony there appeared in her the
inheritance of Eve, with sorrow seeking what in sorrow she had brought
forth. And yet, after accusing my treachery and hardheartedness, she
betook herself again to intercede to Thee for me, went to her wonted
place, and I to Rome.

And lo, there was I received by the scourge of bodily sickness, and I
was going down to hell, carrying all the sins which I had committed,
both against Thee, and myself, and others, many and grievous, over and
above that bond of original sin, whereby we all die in Adam. For
Thou hadst not forgiven me any of these things in Christ, nor had He
abolished by His Cross the enmity which by my sins I had incurred with
Thee. For how should He, by the crucifixion of a phantasm, which I
believed Him to be? So true, then, was the death of my soul, as that
of His flesh seemed to me false; and how true the death of His body,
so false was the life of my soul, which did not believe it. And now the
fever heightening, I was parting and departing for ever. For had I then
parted hence, whither had I departed, but into fire and torments, such
as my misdeeds deserved in the truth of Thy appointment? And this she
knew not, yet in absence prayed for me. But Thou, everywhere present,
heardest her where she was, and, where I was, hadst compassion upon me;
that I should recover the health of my body, though frenzied as yet
in my sacrilegious heart. For I did not in all that danger desire Thy
baptism; and I was better as a boy, when I begged it of my mother's
piety, as I have before recited and confessed. But I had grown up to my
own shame, and I madly scoffed at the prescripts of Thy medicine, who
wouldest not suffer me, being such, to die a double death. With which
wound had my mother's heart been pierced, it could never be healed. For
I cannot express the affection she bore to me, and with how much more
vehement anguish she was now in labour of me in the spirit, than at her
childbearing in the flesh.

I see not then how she should have been healed, had such a death of mine
stricken through the bowels of her love. And where would have been those
her so strong and unceasing prayers, unintermitting to Thee alone? But
wouldest Thou, God of mercies, despise the contrite and humbled heart of
that chaste and sober widow, so frequent in almsdeeds, so full of duty
and service to Thy saints, no day intermitting the oblation at Thine
altar, twice a day, morning and evening, without any intermission,
coming to Thy church, not for idle tattlings and old wives' fables; but
that she might hear Thee in Thy discourses, and Thou her in her prayers.
Couldest Thou despise and reject from Thy aid the tears of such an one,
wherewith she begged of Thee not gold or silver, nor any mutable or
passing good, but the salvation of her son's soul? Thou, by whose gift
she was such? Never, Lord. Yea, Thou wert at hand, and wert hearing and
doing, in that order wherein Thou hadst determined before that it should
be done. Far be it that Thou shouldest deceive her in Thy visions and
answers, some whereof I have, some I have not mentioned, which she laid
up in her faithful heart, and ever praying, urged upon Thee, as
Thine own handwriting. For Thou, because Thy mercy endureth for ever,
vouchsafest to those to whom Thou forgivest all of their debts, to
become also a debtor by Thy promises.

Thou recoveredst me then of that sickness, and healedst the son of Thy
handmaid, for the time in body, that he might live, for Thee to bestow
upon him a better and more abiding health. And even then, at Rome, I
joined myself to those deceiving and deceived "holy ones"; not with
their disciples only (of which number was he, in whose house I had
fallen sick and recovered); but also with those whom they call "The
Elect." For I still thought "that it was not we that sin, but that I
know not what other nature sinned in us"; and it delighted my pride, to
be free from blame; and when I had done any evil, not to confess I had
done any, that Thou mightest heal my soul because it had sinned against
Thee: but I loved to excuse it, and to accuse I know not what other
thing, which was with me, but which I was not. But in truth it was
wholly I, and mine impiety had divided me against myself: and that sin
was the more incurable, whereby I did not judge myself a sinner; and
execrable iniquity it was, that I had rather have Thee, Thee, O God
Almighty, to be overcome in me to my destruction, than myself of Thee to
salvation. Not as yet then hadst Thou set a watch before my mouth, and a
door of safe keeping around my lips, that my heart might not turn
aside to wicked speeches, to make excuses of sins, with men that work
iniquity; and, therefore, was I still united with their Elect.

But now despairing to make proficiency in that false doctrine, even
those things (with which if I should find no better, I had resolved to
rest contented) I now held more laxly and carelessly. For there half
arose a thought in me that those philosophers, whom they call Academics,
were wiser than the rest, for that they held men ought to doubt
everything, and laid down that no truth can be comprehended by man:
for so, not then understanding even their meaning, I also was clearly
convinced that they thought, as they are commonly reported. Yet did I
freely and openly discourage that host of mine from that over-confidence
which I perceived him to have in those fables, which the books of
Manichaeus are full of. Yet I lived in more familiar friendship with
them, than with others who were not of this heresy. Nor did I maintain
it with my ancient eagerness; still my intimacy with that sect (Rome
secretly harbouring many of them) made me slower to seek any other way:
especially since I despaired of finding the truth, from which they had
turned me aside, in Thy Church, O Lord of heaven and earth, Creator of
all things visible and invisible: and it seemed to me very unseemly to
believe Thee to have the shape of human flesh, and to be bounded by the
bodily lineaments of our members. And because, when I wished to think on
my God, I knew not what to think of, but a mass of bodies (for what was
not such did not seem to me to be anything), this was the greatest, and
almost only cause of my inevitable error.

For hence I believed Evil also to be some such kind of substance, and
to have its own foul and hideous bulk; whether gross, which they called
earth, or thin and subtile (like the body of the air), which they
imagine to be some malignant mind, creeping through that earth. And
because a piety, such as it was, constrained me to believe that the good
God never created any evil nature, I conceived two masses, contrary
to one another, both unbounded, but the evil narrower, the good more
expansive. And from this pestilent beginning, the other sacrilegious
conceits followed on me. For when my mind endeavoured to recur to the
Catholic faith, I was driven back, since that was not the Catholic faith
which I thought to be so. And I seemed to myself more reverential, if I
believed of Thee, my God (to whom Thy mercies confess out of my mouth),
as unbounded, at least on other sides, although on that one where the
mass of evil was opposed to Thee, I was constrained to confess Thee
bounded; than if on all sides I should imagine Thee to be bounded by the
form of a human body. And it seemed to me better to believe Thee to have
created no evil (which to me ignorant seemed not some only, but a bodily
substance, because I could not conceive of mind unless as a subtile
body, and that diffused in definite spaces), than to believe the nature
of evil, such as I conceived it, could come from Thee. Yea, and our
Saviour Himself, Thy Only Begotten, I believed to have been reached
forth (as it were) for our salvation, out of the mass of Thy most lucid
substance, so as to believe nothing of Him, but what I could imagine in
my vanity. His Nature then, being such, I thought could not be born
of the Virgin Mary, without being mingled with the flesh: and how that
which I had so figured to myself could be mingled, and not defiled, I
saw not. I feared therefore to believe Him born in the flesh, lest
I should be forced to believe Him defiled by the flesh. Now will Thy
spiritual ones mildly and lovingly smile upon me, if they shall read
these my confessions. Yet such was I.

Furthermore, what the Manichees had criticised in Thy Scriptures, I
thought could not be defended; yet at times verily I had a wish to
confer upon these several points with some one very well skilled in
those books, and to make trial what he thought thereon; for the words
of one Helpidius, as he spoke and disputed face to face against the
said Manichees, had begun to stir me even at Carthage: in that he
had produced things out of the Scriptures, not easily withstood, the
Manichees' answer whereto seemed to me weak. And this answer they
liked not to give publicly, but only to us in private. It was, that the
Scriptures of the New Testament had been corrupted by I know not whom,
who wished to engraff the law of the Jews upon the Christian faith: yet
themselves produced not any uncorrupted copies. But I, conceiving of
things corporeal only, was mainly held down, vehemently oppressed and
in a manner suffocated by those "masses"; panting under which after the
breath of Thy truth, I could not breathe it pure and untainted.

I began then diligently to practise that for which I came to Rome, to
teach rhetoric; and first, to gather some to my house, to whom, and
through whom, I had begun to be known; when lo, I found other offences
committed in Rome, to which I was not exposed in Africa. True, those
"subvertings" by profligate young men were not here practised, as was
told me: but on a sudden, said they, to avoid paying their
master's stipend, a number of youths plot together, and remove to
another;--breakers of faith, who for love of money hold justice cheap.
These also my heart hated, though not with a perfect hatred: for
perchance I hated them more because I was to suffer by them, than
because they did things utterly unlawful. Of a truth such are base
persons, and they go a whoring from Thee, loving these fleeting
mockeries of things temporal, and filthy lucre, which fouls the hand
that grasps it; hugging the fleeting world, and despising Thee, Who
abidest, and recallest, and forgivest the adulteress soul of man, when
she returns to Thee. And now I hate such depraved and crooked persons,
though I love them if corrigible, so as to prefer to money the learning
which they acquire, and to learning, Thee, O God, the truth and fulness
of assured good, and most pure peace. But then I rather for my own sake
misliked them evil, than liked and wished them good for Thine.

When therefore they of Milan had sent to Rome to the prefect of the
city, to furnish them with a rhetoric reader for their city, and sent
him at the public expense, I made application (through those very
persons, intoxicated with Manichaean vanities, to be freed wherefrom
I was to go, neither of us however knowing it) that Symmachus, then
prefect of the city, would try me by setting me some subject, and so
send me. To Milan I came, to Ambrose the Bishop, known to the whole
world as among the best of men, Thy devout servant; whose eloquent
discourse did then plentifully dispense unto Thy people the flour of Thy
wheat, the gladness of Thy oil, and the sober inebriation of Thy wine.
To him was I unknowing led by Thee, that by him I might knowingly be
led to Thee. That man of God received me as a father, and showed me an
Episcopal kindness on my coming. Thenceforth I began to love him, at
first indeed not as a teacher of the truth (which I utterly despaired
of in Thy Church), but as a person kind towards myself. And I listened
diligently to him preaching to the people, not with that intent I ought,
but, as it were, trying his eloquence, whether it answered the fame
thereof, or flowed fuller or lower than was reported; and I hung on his
words attentively; but of the matter I was as a careless and scornful
looker-on; and I was delighted with the sweetness of his discourse,
more recondite, yet in manner less winning and harmonious, than that of
Faustus. Of the matter, however, there was no comparison; for the one
was wandering amid Manichaean delusions, the other teaching salvation
most soundly. But salvation is far from sinners, such as I then stood
before him; and yet was I drawing nearer by little and little, and
unconsciously.

For though I took no pains to learn what he spake, but only to hear how
he spake (for that empty care alone was left me, despairing of a way,
open for man, to Thee), yet together with the words which I would
choose, came also into my mind the things which I would refuse; for
I could not separate them. And while I opened my heart to admit "how
eloquently he spake," there also entered "how truly he spake"; but this
by degrees. For first, these things also had now begun to appear to
me capable of defence; and the Catholic faith, for which I had thought
nothing could be said against the Manichees' objections, I now thought
might be maintained without shamelessness; especially after I had heard
one or two places of the Old Testament resolved, and ofttimes "in a
figure," which when I understood literally, I was slain spiritually.
Very many places then of those books having been explained, I now blamed
my despair, in believing that no answer could be given to such as hated
and scoffed at the Law and the Prophets. Yet did I not therefore then
see that the Catholic way was to be held, because it also could find
learned maintainers, who could at large and with some show of reason
answer objections; nor that what I held was therefore to be condemned,
because both sides could be maintained. For the Catholic cause seemed to
me in such sort not vanquished, as still not as yet to be victorious.

Hereupon I earnestly bent my mind, to see if in any way I could by any
certain proof convict the Manichees of falsehood. Could I once have
conceived a spiritual substance, all their strongholds had been beaten
down, and cast utterly out of my mind; but I could not. Notwithstanding,
concerning the frame of this world, and the whole of nature, which the
senses of the flesh can reach to, as I more and more considered and
compared things, I judged the tenets of most of the philosophers to have
been much more probable. So then after the manner of the Academics (as
they are supposed) doubting of every thing, and wavering between all, I
settled so far, that the Manichees were to be abandoned; judging that,
even while doubting, I might not continue in that sect, to which I
already preferred some of the philosophers; to which philosophers
notwithstanding, for that they were without the saving Name of Christ,
I utterly refused to commit the cure of my sick soul. I determined
therefore so long to be a Catechumen in the Catholic Church, to which
I had been commended by my parents, till something certain should dawn
upon me, whither I might steer my course.




\chapter{BOOK VI}


\start{O}{ Thou}, my hope from my youth, where wert Thou to me, and whither wert
Thou gone? Hadst not Thou created me, and separated me from the beasts
of the field, and fowls of the air? Thou hadst made me wiser, yet did I
walk in darkness, and in slippery places, and sought Thee abroad out of
myself, and found not the God of my heart; and had come into the depths
of the sea, and distrusted and despaired of ever finding truth. My
mother had now come to me, resolute through piety, following me over sea
and land, in all perils confiding in Thee. For in perils of the sea, she
comforted the very mariners (by whom passengers unacquainted with the
deep, use rather to be comforted when troubled), assuring them of a safe
arrival, because Thou hadst by a vision assured her thereof. She found
me in grievous peril, through despair of ever finding truth. But when
I had discovered to her that I was now no longer a Manichee, though
not yet a Catholic Christian, she was not overjoyed, as at something
unexpected; although she was now assured concerning that part of my
misery, for which she bewailed me as one dead, though to be reawakened
by Thee, carrying me forth upon the bier of her thoughts, that Thou
mightest say to the son of the widow, Young man, I say unto thee, Arise;
and he should revive, and begin to speak, and Thou shouldest deliver him
to his mother. Her heart then was shaken with no tumultuous exultation,
when she heard that what she daily with tears desired of Thee was
already in so great part realised; in that, though I had not yet
attained the truth, I was rescued from falsehood; but, as being assured,
that Thou, Who hadst promised the whole, wouldest one day give the rest,
most calmly, and with a heart full of confidence, she replied to me,
"She believed in Christ, that before she departed this life, she should
see me a Catholic believer." Thus much to me. But to Thee, Fountain
of mercies, poured she forth more copious prayers and tears, that Thou
wouldest hasten Thy help, and enlighten my darkness; and she hastened
the more eagerly to the Church, and hung upon the lips of Ambrose,
praying for the fountain of that water, which springeth up unto life
everlasting. But that man she loved as an angel of God, because she knew
that by him I had been brought for the present to that doubtful state of
faith I now was in, through which she anticipated most confidently that
I should pass from sickness unto health, after the access, as it were,
of a sharper fit, which physicians call "the crisis."

When then my mother had once, as she was wont in Afric, brought to the
Churches built in memory of the Saints, certain cakes, and bread and
wine, and was forbidden by the door-keeper; so soon as she knew that the
Bishop had forbidden this, she so piously and obediently embraced
his wishes, that I myself wondered how readily she censured her own
practice, rather than discuss his prohibition. For wine-bibbing did not
lay siege to her spirit, nor did love of wine provoke her to hatred of
the truth, as it doth too many (both men and women), who revolt at a
lesson of sobriety, as men well-drunk at a draught mingled with
water. But she, when she had brought her basket with the accustomed
festival-food, to be but tasted by herself, and then given away, never
joined therewith more than one small cup of wine, diluted according to
her own abstemious habits, which for courtesy she would taste. And if
there were many churches of the departed saints that were to be honoured
in that manner, she still carried round that same one cup, to be
used every where; and this, though not only made very watery, but
unpleasantly heated with carrying about, she would distribute to those
about her by small sips; for she sought there devotion, not pleasure.
So soon, then, as she found this custom to be forbidden by that famous
preacher and most pious prelate, even to those that would use it
soberly, lest so an occasion of excess might be given to the drunken;
and for these, as it were, anniversary funeral solemnities did much
resemble the superstition of the Gentiles, she most willingly forbare
it: and for a basket filled with fruits of the earth, she had learned to
bring to the Churches of the martyrs a breast filled with more
purified petitions, and to give what she could to the poor; that so
the communication of the Lord's Body might be there rightly celebrated,
where, after the example of His Passion, the martyrs had been sacrificed
and crowned. But yet it seems to me, O Lord my God, and thus thinks my
heart of it in Thy sight, that perhaps she would not so readily have
yielded to the cutting off of this custom, had it been forbidden by
another, whom she loved not as Ambrose, whom, for my salvation,
she loved most entirely; and he her again, for her most religious
conversation, whereby in good works, so fervent in spirit, she was
constant at church; so that, when he saw me, he often burst forth into
her praises; congratulating me that I had such a mother; not knowing
what a son she had in me, who doubted of all these things, and imagined
the way to life could not be found out.

Nor did I yet groan in my prayers, that Thou wouldest help me; but
my spirit was wholly intent on learning, and restless to dispute. And
Ambrose himself, as the world counts happy, I esteemed a happy man, whom
personages so great held in such honour; only his celibacy seemed to me
a painful course. But what hope he bore within him, what struggles he
had against the temptations which beset his very excellencies, or what
comfort in adversities, and what sweet joys Thy Bread had for the hidden
mouth of his spirit, when chewing the cud thereof, I neither could
conjecture, nor had experienced. Nor did he know the tides of my
feelings, or the abyss of my danger. For I could not ask of him, what
I would as I would, being shut out both from his ear and speech by
multitudes of busy people, whose weaknesses he served. With whom when he
was not taken up (which was but a little time), he was either refreshing
his body with the sustenance absolutely necessary, or his mind with
reading. But when he was reading, his eye glided over the pages, and
his heart searched out the sense, but his voice and tongue were at rest.
Ofttimes when we had come (for no man was forbidden to enter, nor was it
his wont that any who came should be announced to him), we saw him thus
reading to himself, and never otherwise; and having long sat silent
(for who durst intrude on one so intent?) we were fain to depart,
conjecturing that in the small interval which he obtained, free from the
din of others' business, for the recruiting of his mind, he was loth to
be taken off; and perchance he dreaded lest if the author he read should
deliver any thing obscurely, some attentive or perplexed hearer should
desire him to expound it, or to discuss some of the harder questions; so
that his time being thus spent, he could not turn over so many volumes
as he desired; although the preserving of his voice (which a very little
speaking would weaken) might be the truer reason for his reading to
himself. But with what intent soever he did it, certainly in such a man
it was good.

I however certainly had no opportunity of enquiring what I wished of
that so holy oracle of Thine, his breast, unless the thing might be
answered briefly. But those tides in me, to be poured out to him,
required his full leisure, and never found it. I heard him indeed every
Lord's day, rightly expounding the Word of truth among the people; and
I was more and more convinced that all the knots of those crafty
calumnies, which those our deceivers had knit against the Divine Books,
could be unravelled. But when I understood withal, that "man created
by Thee, after Thine own image," was not so understood by Thy spiritual
sons, whom of the Catholic Mother Thou hast born again through grace,
as though they believed and conceived of Thee as bounded by human shape
(although what a spiritual substance should be I had not even a faint or
shadowy notion); yet, with joy I blushed at having so many years barked
not against the Catholic faith, but against the fictions of carnal
imaginations. For so rash and impious had I been, that what I ought by
enquiring to have learned, I had pronounced on, condemning. For Thou,
Most High, and most near; most secret, and most present; Who hast not
limbs some larger, some smaller, but art wholly every where, and no
where in space, art not of such corporeal shape, yet hast Thou made man
after Thine own image; and behold, from head to foot is he contained in
space.

Ignorant then how this Thy image should subsist, I should have knocked
and proposed the doubt, how it was to be believed, not insultingly
opposed it, as if believed. Doubt, then, what to hold for certain,
the more sharply gnawed my heart, the more ashamed I was, that so long
deluded and deceived by the promise of certainties, I had with childish
error and vehemence, prated of so many uncertainties. For that they were
falsehoods became clear to me later. However I was certain that they
were uncertain, and that I had formerly accounted them certain, when
with a blind contentiousness, I accused Thy Catholic Church, whom I now
discovered, not indeed as yet to teach truly, but at least not to teach
that for which I had grievously censured her. So I was confounded, and
converted: and I joyed, O my God, that the One Only Church, the body of
Thine Only Son (wherein the name of Christ had been put upon me as an
infant), had no taste for infantine conceits; nor in her sound doctrine
maintained any tenet which should confine Thee, the Creator of all, in
space, however great and large, yet bounded every where by the limits of
a human form.

I joyed also that the old Scriptures of the law and the Prophets were
laid before me, not now to be perused with that eye to which before they
seemed absurd, when I reviled Thy holy ones for so thinking, whereas
indeed they thought not so: and with joy I heard Ambrose in his sermons
to the people, oftentimes most diligently recommend this text for a
rule, The letter killeth, but the Spirit giveth life; whilst he drew
aside the mystic veil, laying open spiritually what, according to the
letter, seemed to teach something unsound; teaching herein nothing that
offended me, though he taught what I knew not as yet, whether it were
true. For I kept my heart from assenting to any thing, fearing to fall
headlong; but by hanging in suspense I was the worse killed. For I
wished to be as assured of the things I saw not, as I was that seven and
three are ten. For I was not so mad as to think that even this could not
be comprehended; but I desired to have other things as clear as this,
whether things corporeal, which were not present to my senses, or
spiritual, whereof I knew not how to conceive, except corporeally. And
by believing might I have been cured, that so the eyesight of my soul
being cleared, might in some way be directed to Thy truth, which abideth
always, and in no part faileth. But as it happens that one who has tried
a bad physician, fears to trust himself with a good one, so was it with
the health of my soul, which could not be healed but by believing, and
lest it should believe falsehoods, refused to be cured; resisting Thy
hands, Who hast prepared the medicines of faith, and hast applied
them to the diseases of the whole world, and given unto them so great
authority.

Being led, however, from this to prefer the Catholic doctrine, I felt
that her proceeding was more unassuming and honest, in that she required
to be believed things not demonstrated (whether it was that they could
in themselves be demonstrated but not to certain persons, or could not
at all be), whereas among the Manichees our credulity was mocked by a
promise of certain knowledge, and then so many most fabulous and
absurd things were imposed to be believed, because they could not be
demonstrated. Then Thou, O Lord, little by little with most tender and
most merciful hand, touching and composing my heart, didst persuade
me--considering what innumerable things I believed, which I saw not, nor
was present while they were done, as so many things in secular history,
so many reports of places and of cities, which I had not seen; so many
of friends, so many of physicians, so many continually of other men,
which unless we should believe, we should do nothing at all in this
life; lastly, with how unshaken an assurance I believed of what
parents I was born, which I could not know, had I not believed upon
hearsay--considering all this, Thou didst persuade me, that not they who
believed Thy Books (which Thou hast established in so great authority
among almost all nations), but they who believed them not, were to be
blamed; and that they were not to be heard, who should say to me, "How
knowest thou those Scriptures to have been imparted unto mankind by the
Spirit of the one true and most true God?" For this very thing was
of all most to be believed, since no contentiousness of blasphemous
questionings, of all that multitude which I had read in the
self-contradicting philosophers, could wring this belief from me,
"That Thou art" whatsoever Thou wert (what I knew not), and "That the
government of human things belongs to Thee."

This I believed, sometimes more strongly, more weakly otherwhiles; yet I
ever believed both that Thou wert, and hadst a care of us; though I was
ignorant, both what was to be thought of Thy substance, and what way led
or led back to Thee. Since then we were too weak by abstract reasonings
to find out truth: and for this very cause needed the authority of Holy
Writ; I had now begun to believe that Thou wouldest never have given
such excellency of authority to that Writ in all lands, hadst Thou not
willed thereby to be believed in, thereby sought. For now what things,
sounding strangely in the Scripture, were wont to offend me, having
heard divers of them expounded satisfactorily, I referred to the depth
of the mysteries, and its authority appeared to me the more venerable,
and more worthy of religious credence, in that, while it lay open to all
to read, it reserved the majesty of its mysteries within its profounder
meaning, stooping to all in the great plainness of its words and
lowliness of its style, yet calling forth the intensest application of
such as are not light of heart; that so it might receive all in its open
bosom, and through narrow passages waft over towards Thee some few, yet
many more than if it stood not aloft on such a height of authority, nor
drew multitudes within its bosom by its holy lowliness. These things
I thought on, and Thou wert with me; I sighed, and Thou heardest me; I
wavered, and Thou didst guide me; I wandered through the broad way of
the world, and Thou didst not forsake me.

I panted after honours, gains, marriage; and thou deridedst me. In these
desires I underwent most bitter crosses, Thou being the more gracious,
the less Thou sufferedst aught to grow sweet to me, which was not Thou.
Behold my heart, O Lord, who wouldest I should remember all this, and
confess to Thee. Let my soul cleave unto Thee, now that Thou hast freed
it from that fast-holding birdlime of death. How wretched was it! and
Thou didst irritate the feeling of its wound, that forsaking all else,
it might be converted unto Thee, who art above all, and without whom all
things would be nothing; be converted, and be healed. How miserable was
I then, and how didst Thou deal with me, to make me feel my misery on
that day, when I was preparing to recite a panegyric of the Emperor,
wherein I was to utter many a lie, and lying, was to be applauded by
those who knew I lied, and my heart was panting with these anxieties,
and boiling with the feverishness of consuming thoughts. For, passing
through one of the streets of Milan, I observed a poor beggar, then, I
suppose, with a full belly, joking and joyous: and I sighed, and spoke
to the friends around me, of the many sorrows of our frenzies; for that
by all such efforts of ours, as those wherein I then toiled dragging
along, under the goading of desire, the burthen of my own wretchedness,
and, by dragging, augmenting it, we yet looked to arrive only at that
very joyousness whither that beggar-man had arrived before us, who
should never perchance attain it. For what he had obtained by means of a
few begged pence, the same was I plotting for by many a toilsome turning
and winding; the joy of a temporary felicity. For he verily had not the
true joy; but yet I with those my ambitious designs was seeking one much
less true. And certainly he was joyous, I anxious; he void of care, I
full of fears. But should any ask me, had I rather be merry or fearful?
I would answer merry. Again, if he asked had I rather be such as he was,
or what I then was? I should choose to be myself, though worn with cares
and fears; but out of wrong judgment; for, was it the truth? For I ought
not to prefer myself to him, because more learned than he, seeing I
had no joy therein, but sought to please men by it; and that not to
instruct, but simply to please. Wherefore also Thou didst break my bones
with the staff of Thy correction.

Away with those then from my soul who say to her, "It makes a difference
whence a man's joy is. That beggar-man joyed in drunkenness; Thou
desiredst to joy in glory." What glory, Lord? That which is not in
Thee. For even as his was no true joy, so was that no true glory: and
it overthrew my soul more. He that very night should digest his
drunkenness; but I had slept and risen again with mine, and was to sleep
again, and again to rise with it, how many days, Thou, God, knowest. But
"it doth make a difference whence a man's joy is." I know it, and the
joy of a faithful hope lieth incomparably beyond such vanity. Yea, and
so was he then beyond me: for he verily was the happier; not only for
that he was thoroughly drenched in mirth, I disembowelled with cares:
but he, by fair wishes, had gotten wine; I, by lying, was seeking for
empty, swelling praise. Much to this purpose said I then to my friends:
and I often marked in them how it fared with me; and I found it went ill
with me, and grieved, and doubled that very ill; and if any prosperity
smiled on me, I was loth to catch at it, for almost before I could grasp
it, it flew away.

These things we, who were living as friends together, bemoaned together,
but chiefly and most familiarly did I speak thereof with Alypius and
Nebridius, of whom Alypius was born in the same town with me, of persons
of chief rank there, but younger than I. For he had studied under me,
both when I first lectured in our town, and afterwards at Carthage, and
he loved me much, because I seemed to him kind, and learned; and I him,
for his great towardliness to virtue, which was eminent enough in one of
no greater years. Yet the whirlpool of Carthaginian habits (amongst whom
those idle spectacles are hotly followed) had drawn him into the
madness of the Circus. But while he was miserably tossed therein, and
I, professing rhetoric there, had a public school, as yet he used not my
teaching, by reason of some unkindness risen betwixt his father and me.
I had found then how deadly he doted upon the Circus, and was deeply
grieved that he seemed likely, nay, or had thrown away so great promise:
yet had I no means of advising or with a sort of constraint reclaiming
him, either by the kindness of a friend, or the authority of a master.
For I supposed that he thought of me as did his father; but he was not
such; laying aside then his father's mind in that matter, he began to
greet me, come sometimes into my lecture room, hear a little, and be
gone.

I however had forgotten to deal with him, that he should not, through
a blind and headlong desire of vain pastimes, undo so good a wit. But
Thou, O Lord, who guidest the course of all Thou hast created, hadst
not forgotten him, who was one day to be among Thy children, Priest
and Dispenser of Thy Sacrament; and that his amendment might plainly be
attributed to Thyself, Thou effectedst it through me, unknowingly. For
as one day I sat in my accustomed place, with my scholars before me,
he entered, greeted me, sat down, and applied his mind to what I
then handled. I had by chance a passage in hand, which while I was
explaining, a likeness from the Circensian races occurred to me, as
likely to make what I would convey pleasanter and plainer, seasoned
with biting mockery of those whom that madness had enthralled; God, Thou
knowest that I then thought not of curing Alypius of that infection. But
he took it wholly to himself, and thought that I said it simply for his
sake. And whence another would have taken occasion of offence with me,
that right-minded youth took as a ground of being offended at himself,
and loving me more fervently. For Thou hadst said it long ago, and put
it into Thy book, Rebuke a wise man and he will love Thee. But I had
not rebuked him, but Thou, who employest all, knowing or not knowing, in
that order which Thyself knowest (and that order is just), didst of my
heart and tongue make burning coals, by which to set on fire the
hopeful mind, thus languishing, and so cure it. Let him be silent in Thy
praises, who considers not Thy mercies, which confess unto Thee out of
my inmost soul. For he upon that speech burst out of that pit so deep,
wherein he was wilfully plunged, and was blinded with its wretched
pastimes; and he shook his mind with a strong self-command; whereupon
all the filths of the Circensian pastimes flew off from him, nor came he
again thither. Upon this, he prevailed with his unwilling father that he
might be my scholar. He gave way, and gave in. And Alypius beginning
to be my hearer again, was involved in the same superstition with me,
loving in the Manichees that show of continency which he supposed true
and unfeigned. Whereas it was a senseless and seducing continency,
ensnaring precious souls, unable as yet to reach the depth of virtue,
yet readily beguiled with the surface of what was but a shadowy and
counterfeit virtue.

He, not forsaking that secular course which his parents had charmed him
to pursue, had gone before me to Rome, to study law, and there he was
carried away incredibly with an incredible eagerness after the shows of
gladiators. For being utterly averse to and detesting spectacles, he was
one day by chance met by divers of his acquaintance and fellow-students
coming from dinner, and they with a familiar violence haled him,
vehemently refusing and resisting, into the Amphitheatre, during these
cruel and deadly shows, he thus protesting: "Though you hale my body to
that place, and there set me, can you force me also to turn my mind or
my eyes to those shows? I shall then be absent while present, and
so shall overcome both you and them." They, hearing this, led him on
nevertheless, desirous perchance to try that very thing, whether he
could do as he said. When they were come thither, and had taken their
places as they could, the whole place kindled with that savage pastime.
But he, closing the passage of his eyes, forbade his mind to range
abroad after such evil; and would he had stopped his ears also! For in
the fight, when one fell, a mighty cry of the whole people striking him
strongly, overcome by curiosity, and as if prepared to despise and be
superior to it whatsoever it were, even when seen, he opened his eyes,
and was stricken with a deeper wound in his soul than the other, whom he
desired to behold, was in his body; and he fell more miserably than he
upon whose fall that mighty noise was raised, which entered through his
ears, and unlocked his eyes, to make way for the striking and beating
down of a soul, bold rather than resolute, and the weaker, in that it
had presumed on itself, which ought to have relied on Thee. For so soon
as he saw that blood, he therewith drunk down savageness; nor turned
away, but fixed his eye, drinking in frenzy, unawares, and was delighted
with that guilty fight, and intoxicated with the bloody pastime. Nor was
he now the man he came, but one of the throng he came unto, yea, a true
associate of theirs that brought him thither. Why say more? He beheld,
shouted, kindled, carried thence with him the madness which should goad
him to return not only with them who first drew him thither, but also
before them, yea and to draw in others. Yet thence didst Thou with a
most strong and most merciful hand pluck him, and taughtest him to have
confidence not in himself, but in Thee. But this was after.

But this was already being laid up in his memory to be a medicine
hereafter. So was that also, that when he was yet studying under me at
Carthage, and was thinking over at mid-day in the market-place what he
was to say by heart (as scholars use to practise), Thou sufferedst him
to be apprehended by the officers of the market-place for a thief. For
no other cause, I deem, didst Thou, our God, suffer it, but that he who
was hereafter to prove so great a man, should already begin to learn
that in judging of causes, man was not readily to be condemned by man
out of a rash credulity. For as he was walking up and down by himself
before the judgment-seat, with his note-book and pen, lo, a young man, a
lawyer, the real thief, privily bringing a hatchet, got in, unperceived
by Alypius, as far as the leaden gratings which fence in the
silversmiths' shops, and began to cut away the lead. But the noise of
the hatchet being heard, the silversmiths beneath began to make a stir,
and sent to apprehend whomever they should find. But he, hearing their
voices, ran away, leaving his hatchet, fearing to be taken with it.
Alypius now, who had not seen him enter, was aware of his going, and
saw with what speed he made away. And being desirous to know the matter,
entered the place; where finding the hatchet, he was standing, wondering
and considering it, when behold, those that had been sent, find him
alone with the hatchet in his hand, the noise whereof had startled and
brought them thither. They seize him, hale him away, and gathering the
dwellers in the market-place together, boast of having taken a notorious
thief, and so he was being led away to be taken before the judge.

But thus far was Alypius to be instructed. For forthwith, O Lord, Thou
succouredst his innocency, whereof Thou alone wert witness. For as he
was being led either to prison or to punishment, a certain architect met
them, who had the chief charge of the public buildings. Glad they
were to meet him especially, by whom they were wont to be suspected of
stealing the goods lost out of the marketplace, as though to show him at
last by whom these thefts were committed. He, however, had divers times
seen Alypius at a certain senator's house, to whom he often went to pay
his respects; and recognising him immediately, took him aside by the
hand, and enquiring the occasion of so great a calamity, heard the whole
matter, and bade all present, amid much uproar and threats, to go with
him. So they came to the house of the young man who had done the
deed. There, before the door, was a boy so young as to be likely, not
apprehending any harm to his master, to disclose the whole. For he
had attended his master to the market-place. Whom so soon as Alypius
remembered, he told the architect: and he showing the hatchet to the
boy, asked him "Whose that was?" "Ours," quoth he presently: and being
further questioned, he discovered every thing. Thus the crime being
transferred to that house, and the multitude ashamed, which had begun
to insult over Alypius, he who was to be a dispenser of Thy Word, and an
examiner of many causes in Thy Church, went away better experienced and
instructed.

Him then I had found at Rome, and he clave to me by a most strong tie,
and went with me to Milan, both that he might not leave me, and might
practise something of the law he had studied, more to please his parents
than himself. There he had thrice sat as Assessor, with an uncorruptness
much wondered at by others, he wondering at others rather who could
prefer gold to honesty. His character was tried besides, not only with
the bait of covetousness, but with the goad of fear. At Rome he was
Assessor to the count of the Italian Treasury. There was at that time a
very powerful senator, to whose favours many stood indebted, many much
feared. He would needs, by his usual power, have a thing allowed him
which by the laws was unallowed. Alypius resisted it: a bribe was
promised; with all his heart he scorned it: threats were held out; he
trampled upon them: all wondering at so unwonted a spirit, which neither
desired the friendship, nor feared the enmity of one so great and so
mightily renowned for innumerable means of doing good or evil. And the
very judge, whose councillor Alypius was, although also unwilling
it should be, yet did not openly refuse, but put the matter off upon
Alypius, alleging that he would not allow him to do it: for in truth had
the judge done it, Alypius would have decided otherwise. With this one
thing in the way of learning was he well-nigh seduced, that he might
have books copied for him at Praetorian prices, but consulting justice,
he altered his deliberation for the better; esteeming equity whereby he
was hindered more gainful than the power whereby he were allowed. These
are slight things, but he that is faithful in little, is faithful also
in much. Nor can that any how be void, which proceeded out of the mouth
of Thy Truth: If ye have not been faithful in the unrighteous Mammon,
who will commit to your trust true riches? And if ye have not been
faithful in that which is another man's, who shall give you that which
is your own? He being such, did at that time cleave to me, and with me
wavered in purpose, what course of life was to be taken.

Nebridius also, who having left his native country near Carthage, yea
and Carthage itself, where he had much lived, leaving his excellent
family-estate and house, and a mother behind, who was not to follow him,
had come to Milan, for no other reason but that with me he might live in
a most ardent search after truth and wisdom. Like me he sighed, like
me he wavered, an ardent searcher after true life, and a most acute
examiner of the most difficult questions. Thus were there the mouths
of three indigent persons, sighing out their wants one to another, and
waiting upon Thee that Thou mightest give them their meat in due season.
And in all the bitterness which by Thy mercy followed our worldly
affairs, as we looked towards the end, why we should suffer all this,
darkness met us; and we turned away groaning, and saying, How long shall
these things be? This too we often said; and so saying forsook them not,
for as yet there dawned nothing certain, which these forsaken, we might
embrace.

And I, viewing and reviewing things, most wondered at the length of time
from that my nineteenth year, wherein I had begun to kindle with the
desire of wisdom, settling when I had found her, to abandon all the
empty hopes and lying frenzies of vain desires. And lo, I was now in
my thirtieth year, sticking in the same mire, greedy of enjoying things
present, which passed away and wasted my soul; while I said to myself,
"Tomorrow I shall find it; it will appear manifestly and I shall grasp
it; lo, Faustus the Manichee will come, and clear every thing! O you
great men, ye Academicians, it is true then, that no certainty can
be attained for the ordering of life! Nay, let us search the more
diligently, and despair not. Lo, things in the ecclesiastical books
are not absurd to us now, which sometimes seemed absurd, and may be
otherwise taken, and in a good sense. I will take my stand, where, as
a child, my parents placed me, until the clear truth be found out. But
where shall it be sought or when? Ambrose has no leisure; we have no
leisure to read; where shall we find even the books? Whence, or when
procure them? from whom borrow them? Let set times be appointed, and
certain hours be ordered for the health of our soul. Great hope has
dawned; the Catholic Faith teaches not what we thought, and vainly
accused it of; her instructed members hold it profane to believe God to
be bounded by the figure of a human body: and do we doubt to 'knock,'
that the rest 'may be opened'? The forenoons our scholars take up; what
do we during the rest? Why not this? But when then pay we court to our
great friends, whose favour we need? When compose what we may sell
to scholars? When refresh ourselves, unbending our minds from this
intenseness of care?

"Perish every thing, dismiss we these empty vanities, and betake
ourselves to the one search for truth! Life is vain, death uncertain; if
it steals upon us on a sudden, in what state shall we depart hence?
and where shall we learn what here we have neglected? and shall we not
rather suffer the punishment of this negligence? What, if death itself
cut off and end all care and feeling? Then must this be ascertained.
But God forbid this! It is no vain and empty thing, that the excellent
dignity of the authority of the Christian Faith hath overspread the
whole world. Never would such and so great things be by God wrought for
us, if with the death of the body the life of the soul came to an end.
Wherefore delay then to abandon worldly hopes, and give ourselves wholly
to seek after God and the blessed life? But wait! Even those things are
pleasant; they have some, and no small sweetness. We must not lightly
abandon them, for it were a shame to return again to them. See, it is
no great matter now to obtain some station, and then what should we more
wish for? We have store of powerful friends; if nothing else offer, and
we be in much haste, at least a presidentship may be given us: and a
wife with some money, that she increase not our charges: and this shall
be the bound of desire. Many great men, and most worthy of imitation,
have given themselves to the study of wisdom in the state of marriage."

While I went over these things, and these winds shifted and drove my
heart this way and that, time passed on, but I delayed to turn to the
Lord; and from day to day deferred to live in Thee, and deferred not
daily to die in myself. Loving a happy life, I feared it in its own
abode, and sought it, by fleeing from it. I thought I should be too
miserable, unless folded in female arms; and of the medicine of Thy
mercy to cure that infirmity I thought not, not having tried it. As for
continency, I supposed it to be in our own power (though in myself I did
not find that power), being so foolish as not to know what is written,
None can be continent unless Thou give it; and that Thou wouldest give
it, if with inward groanings I did knock at Thine ears, and with a
settled faith did cast my care on Thee.

Alypius indeed kept me from marrying; alleging that so could we by no
means with undistracted leisure live together in the love of wisdom, as
we had long desired. For himself was even then most pure in this point,
so that it was wonderful; and that the more, since in the outset of his
youth he had entered into that course, but had not stuck fast therein;
rather had he felt remorse and revolting at it, living thenceforth until
now most continently. But I opposed him with the examples of those who
as married men had cherished wisdom, and served God acceptably, and
retained their friends, and loved them faithfully. Of whose greatness of
spirit I was far short; and bound with the disease of the flesh, and its
deadly sweetness, drew along my chain, dreading to be loosed, and as if
my wound had been fretted, put back his good persuasions, as it were the
hand of one that would unchain me. Moreover, by me did the serpent
speak unto Alypius himself, by my tongue weaving and laying in his
path pleasurable snares, wherein his virtuous and free feet might be
entangled.

For when he wondered that I, whom he esteemed not slightly, should stick
so fast in the birdlime of that pleasure, as to protest (so oft as we
discussed it) that I could never lead a single life; and urged in my
defence when I saw him wonder, that there was great difference between
his momentary and scarce-remembered knowledge of that life, which so
he might easily despise, and my continued acquaintance whereto if the
honourable name of marriage were added, he ought not to wonder why I
could not contemn that course; he began also to desire to be married;
not as overcome with desire of such pleasure, but out of curiosity. For
he would fain know, he said, what that should be, without which my life,
to him so pleasing, would to me seem not life but a punishment. For his
mind, free from that chain, was amazed at my thraldom; and through that
amazement was going on to a desire of trying it, thence to the trial
itself, and thence perhaps to sink into that bondage whereat he
wondered, seeing he was willing to make a covenant with death; and he
that loves danger, shall fall into it. For whatever honour there be in
the office of well-ordering a married life, and a family, moved us but
slightly. But me for the most part the habit of satisfying an insatiable
appetite tormented, while it held me captive; him, an admiring wonder
was leading captive. So were we, until Thou, O Most High, not forsaking
our dust, commiserating us miserable, didst come to our help, by
wondrous and secret ways.

Continual effort was made to have me married. I wooed, I was
promised, chiefly through my mother's pains, that so once married, the
health-giving baptism might cleanse me, towards which she rejoiced
that I was being daily fitted, and observed that her prayers, and Thy
promises, were being fulfilled in my faith. At which time verily, both
at my request and her own longing, with strong cries of heart she
daily begged of Thee, that Thou wouldest by a vision discover unto her
something concerning my future marriage; Thou never wouldest. She saw
indeed certain vain and fantastic things, such as the energy of the
human spirit, busied thereon, brought together; and these she told me
of, not with that confidence she was wont, when Thou showedst her any
thing, but slighting them. For she could, she said, through a certain
feeling, which in words she could not express, discern betwixt Thy
revelations, and the dreams of her own soul. Yet the matter was pressed
on, and a maiden asked in marriage, two years under the fit age; and, as
pleasing, was waited for.

And many of us friends conferring about, and detesting the turbulent
turmoils of human life, had debated and now almost resolved on living
apart from business and the bustle of men; and this was to be thus
obtained; we were to bring whatever we might severally procure, and
make one household of all; so that through the truth of our friendship
nothing should belong especially to any; but the whole thus derived from
all, should as a whole belong to each, and all to all. We thought there
might be some often persons in this society; some of whom were very
rich, especially Romanianus our townsman, from childhood a very familiar
friend of mine, whom the grievous perplexities of his affairs had
brought up to court; who was the most earnest for this project; and
therein was his voice of great weight, because his ample estate far
exceeded any of the rest. We had settled also that two annual officers,
as it were, should provide all things necessary, the rest being
undisturbed. But when we began to consider whether the wives, which
some of us already had, others hoped to have, would allow this, all that
plan, which was being so well moulded, fell to pieces in our hands, was
utterly dashed and cast aside. Thence we betook us to sighs, and groans,
and our steps to follow the broad and beaten ways of the world; for many
thoughts were in our heart, but Thy counsel standeth for ever. Out
of which counsel Thou didst deride ours, and preparedst Thine own;
purposing to give us meat in due season, and to fill our souls with
blessing.

Meanwhile my sins were being multiplied, and my concubine being torn
from my side as a hindrance to my marriage, my heart which clave unto
her was torn and wounded and bleeding. And she returned to Afric, vowing
unto Thee never to know any other man, leaving with me my son by her.
But unhappy I, who could not imitate a very woman, impatient of delay,
inasmuch as not till after two years was I to obtain her I sought not
being so much a lover of marriage as a slave to lust, procured another,
though no wife, that so by the servitude of an enduring custom, the
disease of my soul might be kept up and carried on in its vigour, or
even augmented, into the dominion of marriage. Nor was that my wound
cured, which had been made by the cutting away of the former, but after
inflammation and most acute pain, it mortified, and my pains became less
acute, but more desperate.

To Thee be praise, glory to Thee, Fountain of mercies. I was becoming
more miserable, and Thou nearer. Thy right hand was continually ready to
pluck me out of the mire, and to wash me thoroughly, and I knew it
not; nor did anything call me back from a yet deeper gulf of carnal
pleasures, but the fear of death, and of Thy judgment to come; which
amid all my changes, never departed from my breast. And in my disputes
with my friends Alypius and Nebridius of the nature of good and evil, I
held that Epicurus had in my mind won the palm, had I not believed that
after death there remained a life for the soul, and places of requital
according to men's deserts, which Epicurus would not believe. And I
asked, "were we immortal, and to live in perpetual bodily pleasure,
without fear of losing it, why should we not be happy, or what else
should we seek?" not knowing that great misery was involved in this very
thing, that, being thus sunk and blinded, I could not discern that light
of excellence and beauty, to be embraced for its own sake, which the eye
of flesh cannot see, and is seen by the inner man. Nor did I, unhappy,
consider from what source it sprung, that even on these things, foul as
they were, I with pleasure discoursed with my friends, nor could I,
even according to the notions I then had of happiness, be happy without
friends, amid what abundance soever of carnal pleasures. And yet these
friends I loved for themselves only, and I felt that I was beloved of
them again for myself only.

O crooked paths! Woe to the audacious soul, which hoped, by forsaking
Thee, to gain some better thing! Turned it hath, and turned again, upon
back, sides, and belly, yet all was painful; and Thou alone rest.
And behold, Thou art at hand, and deliverest us from our wretched
wanderings, and placest us in Thy way, and dost comfort us, and say,
"Run; I will carry you; yea I will bring you through; there also will I
carry you."




\chapter{BOOK VII}


\start{D}{eceased} was now that my evil and abominable youth, and I was passing
into early manhood; the more defiled by vain things as I grew in years,
who could not imagine any substance, but such as is wont to be seen with
these eyes. I thought not of Thee, O God, under the figure of a human
body; since I began to hear aught of wisdom, I always avoided this; and
rejoiced to have found the same in the faith of our spiritual mother,
Thy Catholic Church. But what else to conceive of Thee I knew not. And
I, a man, and such a man, sought to conceive of Thee the sovereign,
only, true God; and I did in my inmost soul believe that Thou wert
incorruptible, and uninjurable, and unchangeable; because though not
knowing whence or how, yet I saw plainly, and was sure, that that which
may be corrupted must be inferior to that which cannot; what could not
be injured I preferred unhesitatingly to what could receive injury; the
unchangeable to things subject to change. My heart passionately cried
out against all my phantoms, and with this one blow I sought to beat
away from the eye of my mind all that unclean troop which buzzed around
it. And lo, being scarce put off, in the twinkling of an eye they
gathered again thick about me, flew against my face, and beclouded
it; so that though not under the form of the human body, yet was I
constrained to conceive of Thee (that incorruptible, uninjurable, and
unchangeable, which I preferred before the corruptible, and injurable,
and changeable) as being in space, whether infused into the world, or
diffused infinitely without it. Because whatsoever I conceived, deprived
of this space, seemed to me nothing, yea altogether nothing, not even
a void, as if a body were taken out of its place, and the place should
remain empty of any body at all, of earth and water, air and heaven, yet
would it remain a void place, as it were a spacious nothing.

I then being thus gross-hearted, nor clear even to myself, whatsoever
was not extended over certain spaces, nor diffused, nor condensed, nor
swelled out, or did not or could not receive some of these dimensions,
I thought to be altogether nothing. For over such forms as my eyes are
wont to range, did my heart then range: nor yet did I see that this same
notion of the mind, whereby I formed those very images, was not of this
sort, and yet it could not have formed them, had not itself been some
great thing. So also did I endeavour to conceive of Thee, Life of my
life, as vast, through infinite spaces on every side penetrating
the whole mass of the universe, and beyond it, every way, through
unmeasurable boundless spaces; so that the earth should have Thee, the
heaven have Thee, all things have Thee, and they be bounded in Thee, and
Thou bounded nowhere. For that as the body of this air which is above
the earth, hindereth not the light of the sun from passing through it,
penetrating it, not by bursting or by cutting, but by filling it wholly:
so I thought the body not of heaven, air, and sea only, but of the earth
too, pervious to Thee, so that in all its parts, the greatest as the
smallest, it should admit Thy presence, by a secret inspiration, within
and without, directing all things which Thou hast created. So I guessed,
only as unable to conceive aught else, for it was false. For thus should
a greater part of the earth contain a greater portion of Thee, and a
less, a lesser: and all things should in such sort be full of Thee,
that the body of an elephant should contain more of Thee, than that of
a sparrow, by how much larger it is, and takes up more room; and thus
shouldest Thou make the several portions of Thyself present unto the
several portions of the world, in fragments, large to the large,
petty to the petty. But such art not Thou. But not as yet hadst Thou
enlightened my darkness.

It was enough for me, Lord, to oppose to those deceived deceivers, and
dumb praters, since Thy word sounded not out of them;--that was enough
which long ago, while we were yet at Carthage, Nebridius used to
propound, at which all we that heard it were staggered: "That said
nation of darkness, which the Manichees are wont to set as an opposing
mass over against Thee, what could it have done unto Thee, hadst Thou
refused to fight with it? For, if they answered, 'it would have
done Thee some hurt,' then shouldest Thou be subject to injury and
corruption: but it could do Thee no hurt,' then was no reason brought
for Thy fighting with it; and fighting in such wise, as that a certain
portion or member of Thee, or offspring of Thy very Substance, should be
mingled with opposed powers, and natures not created by Thee, and be
by them so far corrupted and changed to the worse, as to be turned
from happiness into misery, and need assistance, whereby it might be
extricated and purified; and that this offspring of Thy Substance was
the soul, which being enthralled, defiled, corrupted, Thy Word, free,
pure, and whole, might relieve; that Word itself being still corruptible
because it was of one and the same Substance. So then, should they
affirm Thee, whatsoever Thou art, that is, Thy Substance whereby
Thou art, to be incorruptible, then were all these sayings false and
execrable; but if corruptible, the very statement showed it to be false
and revolting." This argument then of Nebridius sufficed against those
who deserved wholly to be vomited out of the overcharged stomach; for
they had no escape, without horrible blasphemy of heart and tongue, thus
thinking and speaking of Thee.

But I also as yet, although I held and was firmly persuaded that Thou
our Lord the true God, who madest not only our souls, but our bodies,
and not only our souls and bodies, but all beings, and all things, wert
undefilable and unalterable, and in no degree mutable; yet understood I
not, clearly and without difficulty, the cause of evil. And yet whatever
it were, I perceived it was in such wise to be sought out, as should not
constrain me to believe the immutable God to be mutable, lest I should
become that evil I was seeking out. I sought it out then, thus far free
from anxiety, certain of the untruth of what these held, from whom I
shrunk with my whole heart: for I saw, that through enquiring the origin
of evil, they were filled with evil, in that they preferred to think
that Thy substance did suffer ill than their own did commit it.

And I strained to perceive what I now heard, that free-will was the
cause of our doing ill, and Thy just judgment of our suffering ill. But
I was not able clearly to discern it. So then endeavouring to draw my
soul's vision out of that deep pit, I was again plunged therein, and
endeavouring often, I was plunged back as often. But this raised me a
little into Thy light, that I knew as well that I had a will, as that I
lived: when then I did will or nill any thing, I was most sure that no
other than myself did will and nill: and I all but saw that there
was the cause of my sin. But what I did against my will, I saw that
I suffered rather than did, and I judged not to be my fault, but my
punishment; whereby, however, holding Thee to be just, I speedily
confessed myself to be not unjustly punished. But again I said, Who made
me? Did not my God, Who is not only good, but goodness itself? Whence
then came I to will evil and nill good, so that I am thus justly
punished? who set this in me, and ingrafted into me this plant of
bitterness, seeing I was wholly formed by my most sweet God? If the
devil were the author, whence is that same devil? And if he also by his
own perverse will, of a good angel became a devil, whence, again, came
in him that evil will whereby he became a devil, seeing the whole nature
of angels was made by that most good Creator? By these thoughts I was
again sunk down and choked; yet not brought down to that hell of error
(where no man confesseth unto Thee), to think rather that Thou dost
suffer ill, than that man doth it.

For I was in such wise striving to find out the rest, as one who had
already found that the incorruptible must needs be better than the
corruptible: and Thee therefore, whatsoever Thou wert, I confessed to
be incorruptible. For never soul was, nor shall be, able to conceive any
thing which may be better than Thou, who art the sovereign and the
best good. But since most truly and certainly, the incorruptible is
preferable to the corruptible (as I did now prefer it), then, wert Thou
not incorruptible, I could in thought have arrived at something better
than my God. Where then I saw the incorruptible to be preferable to the
corruptible, there ought I to seek for Thee, and there observe "wherein
evil itself was"; that is, whence corruption comes, by which Thy
substance can by no means be impaired. For corruption does no ways
impair our God; by no will, by no necessity, by no unlooked-for chance:
because He is God, and what He wills is good, and Himself is that
good; but to be corrupted is not good. Nor art Thou against Thy will
constrained to any thing, since Thy will is not greater than Thy power.
But greater should it be, were Thyself greater than Thyself. For the
will and power of God is God Himself. And what can be unlooked-for by
Thee, Who knowest all things? Nor is there any nature in things, but
Thou knowest it. And what should we more say, "why that substance which
God is should not be corruptible," seeing if it were so, it should not
be God?

And I sought "whence is evil," and sought in an evil way; and saw not
the evil in my very search. I set now before the sight of my spirit
the whole creation, whatsoever we can see therein (as sea, earth, air,
stars, trees, mortal creatures); yea, and whatever in it we do not see,
as the firmament of heaven, all angels moreover, and all the spiritual
inhabitants thereof. But these very beings, as though they were
bodies, did my fancy dispose in place, and I made one great mass of Thy
creation, distinguished as to the kinds of bodies; some, real bodies,
some, what myself had feigned for spirits. And this mass I made huge,
not as it was (which I could not know), but as I thought convenient, yet
every way finite. But Thee, O Lord, I imagined on every part environing
and penetrating it, though every way infinite: as if there were a sea,
every where, and on every side, through unmeasured space, one only
boundless sea, and it contained within it some sponge, huge, but
bounded; that sponge must needs, in all its parts, be filled from that
unmeasurable sea: so conceived I Thy creation, itself finite, full of
Thee, the Infinite; and I said, Behold God, and behold what God hath
created; and God is good, yea, most mightily and incomparably better
than all these: but yet He, the Good, created them good; and see how
He environeth and fulfils them. Where is evil then, and whence, and how
crept it in hither? What is its root, and what its seed? Or hath it no
being? Why then fear we and avoid what is not? Or if we fear it idly,
then is that very fear evil, whereby the soul is thus idly goaded and
racked. Yea, and so much a greater evil, as we have nothing to fear, and
yet do fear. Therefore either is that evil which we fear, or else evil
is, that we fear. Whence is it then? seeing God, the Good, hath created
all these things good. He indeed, the greater and chiefest Good, hath
created these lesser goods; still both Creator and created, all are
good. Whence is evil? Or, was there some evil matter of which He made,
and formed, and ordered it, yet left something in it which He did not
convert into good? Why so then? Had He no might to turn and change the
whole, so that no evil should remain in it, seeing He is All-mighty?
Lastly, why would He make any thing at all of it, and not rather by
the same All-mightiness cause it not to be at all? Or, could it then be
against His will? Or if it were from eternity, why suffered He it so to
be for infinite spaces of times past, and was pleased so long after to
make something out of it? Or if He were suddenly pleased now to effect
somewhat, this rather should the All-mighty have effected, that this
evil matter should not be, and He alone be, the whole, true, sovereign,
and infinite Good. Or if it was not good that He who was good should not
also frame and create something that were good, then, that evil matter
being taken away and brought to nothing, He might form good matter,
whereof to create all things. For He should not be All-mighty, if He
might not create something good without the aid of that matter which
Himself had not created. These thoughts I revolved in my miserable
heart, overcharged with most gnawing cares, lest I should die ere I had
found the truth; yet was the faith of Thy Christ, our Lord and Saviour,
professed in the Church Catholic, firmly fixed in my heart, in many
points, indeed, as yet unformed, and fluctuating from the rule of
doctrine; yet did not my mind utterly leave it, but rather daily took in
more and more of it.

By this time also had I rejected the lying divinations and impious
dotages of the astrologers. Let Thine own mercies, out of my very
inmost soul, confess unto Thee for this also, O my God. For Thou, Thou
altogether (for who else calls us back from the death of all errors,
save the Life which cannot die, and the Wisdom which needing no light
enlightens the minds that need it, whereby the universe is directed,
down to the whirling leaves of trees?)--Thou madest provision for my
obstinacy wherewith I struggled against Vindicianus, an acute old man,
and Nebridius, a young man of admirable talents; the first vehemently
affirming, and the latter often (though with some doubtfulness) saying,
"That there was no such art whereby to foresee things to come, but that
men's conjectures were a sort of lottery, and that out of many things
which they said should come to pass, some actually did, unawares to them
who spake it, who stumbled upon it, through their oft speaking."
Thou providedst then a friend for me, no negligent consulter of the
astrologers; nor yet well skilled in those arts, but (as I said) a
curious consulter with them, and yet knowing something, which he said
he had heard of his father, which how far it went to overthrow the
estimation of that art, he knew not. This man then, Firminus by name,
having had a liberal education, and well taught in Rhetoric, consulted
me, as one very dear to him, what, according to his so-called
constellations, I thought on certain affairs of his, wherein his worldly
hopes had risen, and I, who had herein now begun to incline towards
Nebridius' opinion, did not altogether refuse to conjecture, and tell
him what came into my unresolved mind; but added, that I was now almost
persuaded that these were but empty and ridiculous follies. Thereupon he
told me that his father had been very curious in such books, and had
a friend as earnest in them as himself, who with joint study and
conference fanned the flame of their affections to these toys, so that
they would observe the moments whereat the very dumb animals, which bred
about their houses, gave birth, and then observed the relative position
of the heavens, thereby to make fresh experiments in this so-called art.
He said then that he had heard of his father, that what time his mother
was about to give birth to him, Firminus, a woman-servant of that friend
of his father's was also with child, which could not escape her master,
who took care with most exact diligence to know the births of his very
puppies. And so it was that (the one for his wife, and the other for his
servant, with the most careful observation, reckoning days, hours,
nay, the lesser divisions of the hours) both were delivered at the same
instant; so that both were constrained to allow the same constellations,
even to the minutest points, the one for his son, the other for his
new-born slave. For so soon as the women began to be in labour, they
each gave notice to the other what was fallen out in their houses, and
had messengers ready to send to one another so soon as they had notice
of the actual birth, of which they had easily provided, each in his own
province, to give instant intelligence. Thus then the messengers of
the respective parties met, he averred, at such an equal distance from
either house that neither of them could make out any difference in the
position of the stars, or any other minutest points; and yet Firminus,
born in a high estate in his parents' house, ran his course through
the gilded paths of life, was increased in riches, raised to honours;
whereas that slave continued to serve his masters, without any
relaxation of his yoke, as Firminus, who knew him, told me.

Upon hearing and believing these things, told by one of such
credibility, all that my resistance gave way; and first I endeavoured to
reclaim Firminus himself from that curiosity, by telling him that upon
inspecting his constellations, I ought if I were to predict truly, to
have seen in them parents eminent among their neighbours, a noble family
in its own city, high birth, good education, liberal learning. But if
that servant had consulted me upon the same constellations, since they
were his also, I ought again (to tell him too truly) to see in them
a lineage the most abject, a slavish condition, and every thing else
utterly at variance with the former. Whence then, if I spake the truth,
I should, from the same constellations, speak diversely, or if I
spake the same, speak falsely: thence it followed most certainly that
whatever, upon consideration of the constellations, was spoken truly,
was spoken not out of art, but chance; and whatever spoken falsely, was
not out of ignorance in the art, but the failure of the chance.

An opening thus made, ruminating with myself on the like things, that
no one of those dotards (who lived by such a trade, and whom I longed
to attack, and with derision to confute) might urge against me that
Firminus had informed me falsely, or his father him; I bent my thoughts
on those that are born twins, who for the most part come out of the womb
so near one to other, that the small interval (how much force soever
in the nature of things folk may pretend it to have) cannot be noted
by human observation, or be at all expressed in those figures which the
astrologer is to inspect, that he may pronounce truly. Yet they cannot
be true: for looking into the same figures, he must have predicted the
same of Esau and Jacob, whereas the same happened not to them. Therefore
he must speak falsely; or if truly, then, looking into the same figures,
he must not give the same answer. Not by art, then, but by chance, would
he speak truly. For Thou, O Lord, most righteous Ruler of the Universe,
while consulters and consulted know it not, dost by Thy hidden
inspiration effect that the consulter should hear what, according to the
hidden deservings of souls, he ought to hear, out of the unsearchable
depth of Thy just judgment, to Whom let no man say, What is this? Why
that? Let him not so say, for he is man.

Now then, O my Helper, hadst Thou loosed me from those fetters: and I
sought "whence is evil," and found no way. But Thou sufferedst me not by
any fluctuations of thought to be carried away from the Faith whereby I
believed Thee both to be, and Thy substance to be unchangeable, and that
Thou hast a care of, and wouldest judge men, and that in Christ, Thy
Son, Our Lord, and the holy Scriptures, which the authority of Thy
Catholic Church pressed upon me, Thou hadst set the way of man's
salvation, to that life which is to be after this death. These things
being safe and immovably settled in my mind, I sought anxiously "whence
was evil?" What were the pangs of my teeming heart, what groans, O my
God! yet even there were Thine ears open, and I knew it not; and when
in silence I vehemently sought, those silent contritions of my soul were
strong cries unto Thy mercy. Thou knewest what I suffered, and no man.
For, what was that which was thence through my tongue distilled into the
ears of my most familiar friends? Did the whole tumult of my soul, for
which neither time nor utterance sufficed, reach them? Yet went up the
whole to Thy hearing, all which I roared out from the groanings of my
heart; and my desire was before Thee, and the light of mine eyes was not
with me: for that was within, I without: nor was that confined to place,
but I was intent on things contained in place, but there found I no
resting-place, nor did they so receive me, that I could say, "It is
enough," "it is well": nor did they yet suffer me to turn back, where
it might be well enough with me. For to these things was I superior, but
inferior to Thee; and Thou art my true joy when subjected to Thee, and
Thou hadst subjected to me what Thou createdst below me. And this was
the true temperament, and middle region of my safety, to remain in
Thy Image, and by serving Thee, rule the body. But when I rose proudly
against Thee, and ran against the Lord with my neck, with the thick
bosses of my buckler, even these inferior things were set above me, and
pressed me down, and no where was there respite or space of breathing.
They met my sight on all sides by heaps and troops, and in thought the
images thereof presented themselves unsought, as I would return to
Thee, as if they would say unto me, "Whither goest thou, unworthy
and defiled?" And these things had grown out of my wound; for Thou
"humbledst the proud like one that is wounded," and through my own
swelling was I separated from Thee; yea, my pride-swollen face closed up
mine eyes.

But Thou, Lord, abidest for ever, yet not for ever art Thou angry with
us; because Thou pitiest our dust and ashes, and it was pleasing in Thy
sight to reform my deformities; and by inward goads didst Thou rouse me,
that I should be ill at ease, until Thou wert manifested to my inward
sight. Thus, by the secret hand of Thy medicining was my swelling
abated, and the troubled and bedimmed eyesight of my mind, by the
smarting anointings of healthful sorrows, was from day to day healed.

And Thou, willing first to show me how Thou resistest the proud, but
givest grace unto the humble, and by how great an act of Thy mercy Thou
hadst traced out to men the way of humility, in that Thy Word was made
flesh, and dwelt among men:--Thou procuredst for me, by means of one
puffed up with most unnatural pride, certain books of the Platonists,
translated from Greek into Latin. And therein I read, not indeed in the
very words, but to the very same purpose, enforced by many and divers
reasons, that In the beginning was the Word, and the Word was with God,
and the Word was God: the Same was in the beginning with God: all things
were made by Him, and without Him was nothing made: that which was made
by Him is life, and the life was the light of men, and the light shineth
in the darkness, and the darkness comprehended it not. And that the soul
of man, though it bears witness to the light, yet itself is not that
light; but the Word of God, being God, is that true light that lighteth
every man that cometh into the world. And that He was in the world, and
the world was made by Him, and the world knew Him not. But, that He came
unto His own, and His own received Him not; but as many as received Him,
to them gave He power to become the sons of God, as many as believed in
His name; this I read not there.

Again I read there, that God the Word was born not of flesh nor of
blood, nor of the will of man, nor of the will of the flesh, but of God.
But that the Word was made flesh, and dwelt among us, I read not there.
For I traced in those books that it was many and divers ways said, that
the Son was in the form of the Father, and thought it not robbery to be
equal with God, for that naturally He was the Same Substance. But that
He emptied Himself, taking the form of a servant, being made in the
likeness of men, and found in fashion as a man, humbled Himself, and
became obedient unto death, and that the death of the cross: wherefore
God exalted Him from the dead, and gave Him a name above every name,
that at the name of Jesus every knee should how, of things in heaven,
and things in earth, and things under the earth; and that every tongue
should confess that the Lord Jesus Christ is in the glory of God the
Father; those books have not. For that before all times and above all
times Thy Only-Begotten Son remaineth unchangeable, co-eternal with
Thee, and that of His fulness souls receive, that they may be blessed;
and that by participation of wisdom abiding in them, they are renewed,
so as to be wise, is there. But that in due time He died for the
ungodly; and that Thou sparedst not Thine Only Son, but deliveredst Him
for us all, is not there. For Thou hiddest these things from the wise,
and revealedst them to babes; that they that labour and are heavy laden
might come unto Him, and He refresh them, because He is meek and lowly
in heart; and the meek He directeth in judgment, and the gentle He
teacheth His ways, beholding our lowliness and trouble, and forgiving
all our sins. But such as are lifted up in the lofty walk of some
would-be sublimer learning, hear not Him, saying, Learn of Me, for I am
meek and lowly in heart, and ye shall find rest to your souls. Although
they knew God, yet they glorify Him not as God, nor are thankful,
but wax vain in their thoughts; and their foolish heart is darkened;
professing that they were wise, they became fools.

And therefore did I read there also, that they had changed the glory of
Thy incorruptible nature into idols and divers shapes, into the likeness
of the image of corruptible man, and birds, and beasts, and creeping
things; namely, into that Egyptian food for which Esau lost his
birthright, for that Thy first-born people worshipped the head of a
four-footed beast instead of Thee; turning in heart back towards Egypt;
and bowing Thy image, their own soul, before the image of a calf that
eateth hay. These things found I here, but I fed not on them. For it
pleased Thee, O Lord, to take away the reproach of diminution from
Jacob, that the elder should serve the younger: and Thou calledst the
Gentiles into Thine inheritance. And I had come to Thee from among the
Gentiles; and I set my mind upon the gold which Thou willedst Thy people
to take from Egypt, seeing Thine it was, wheresoever it were. And to the
Athenians Thou saidst by Thy Apostle, that in Thee we live, move, and
have our being, as one of their own poets had said. And verily these
books came from thence. But I set not my mind on the idols of Egypt,
whom they served with Thy gold, who changed the truth of God into a lie,
and worshipped and served the creature more than the Creator.

And being thence admonished to return to myself, I entered even into my
inward self, Thou being my Guide: and able I was, for Thou wert become
my Helper. And I entered and beheld with the eye of my soul (such as
it was), above the same eye of my soul, above my mind, the Light
Unchangeable. Not this ordinary light, which all flesh may look upon,
nor as it were a greater of the same kind, as though the brightness of
this should be manifold brighter, and with its greatness take up all
space. Not such was this light, but other, yea, far other from these.
Nor was it above my soul, as oil is above water, nor yet as heaven above
earth: but above to my soul, because It made me; and I below It, because
I was made by It. He that knows the Truth, knows what that Light is;
and he that knows It, knows eternity. Love knoweth it. O Truth Who art
Eternity! and Love Who art Truth! and Eternity Who art Love! Thou art
my God, to Thee do I sigh night and day. Thee when I first knew, Thou
liftedst me up, that I might see there was what I might see, and that I
was not yet such as to see. And Thou didst beat back the weakness of my
sight, streaming forth Thy beams of light upon me most strongly, and I
trembled with love and awe: and I perceived myself to be far off from
Thee, in the region of unlikeness, as if I heard this Thy voice from on
high: "I am the food of grown men, grow, and thou shalt feed upon Me;
nor shalt thou convert Me, like the food of thy flesh into thee, but
thou shalt be converted into Me." And I learned, that Thou for iniquity
chastenest man, and Thou madest my soul to consume away like a spider.
And I said, "Is Truth therefore nothing because it is not diffused
through space finite or infinite?" And Thou criedst to me from afar:
"Yet verily, I AM that I AM." And I heard, as the heart heareth, nor had
I room to doubt, and I should sooner doubt that I live than that Truth
is not, which is clearly seen, being understood by those things which
are made. And I beheld the other things below Thee, and I perceived that
they neither altogether are, nor altogether are not, for they are, since
they are from Thee, but are not, because they are not what Thou art. For
that truly is which remains unchangeably. It is good then for me to hold
fast unto God; for if I remain not in Him, I cannot in myself; but He
remaining in Himself, reneweth all things. And Thou art the Lord my God,
since Thou standest not in need of my goodness.

And it was manifested unto me, that those things be good which yet are
corrupted; which neither were they sovereignly good, nor unless they
were good could be corrupted: for if sovereignly good, they were
incorruptible, if not good at all, there were nothing in them to be
corrupted. For corruption injures, but unless it diminished goodness, it
could not injure. Either then corruption injures not, which cannot be;
or which is most certain, all which is corrupted is deprived of good.
But if they be deprived of all good, they shall cease to be. For if they
shall be, and can now no longer be corrupted, they shall be better than
before, because they shall abide incorruptibly. And what more monstrous
than to affirm things to become better by losing all their good?
Therefore, if they shall be deprived of all good, they shall no longer
be. So long therefore as they are, they are good: therefore whatsoever
is, is good. That evil then which I sought, whence it is, is not any
substance: for were it a substance, it should be good. For either
it should be an incorruptible substance, and so a chief good: or
a corruptible substance; which unless it were good, could not be
corrupted. I perceived therefore, and it was manifested to me that Thou
madest all things good, nor is there any substance at all, which Thou
madest not; and for that Thou madest not all things equal, therefore are
all things; because each is good, and altogether very good, because our
God made all things very good.

And to Thee is nothing whatsoever evil: yea, not only to Thee, but also
to Thy creation as a whole, because there is nothing without, which may
break in, and corrupt that order which Thou hast appointed it. But in
the parts thereof some things, because unharmonising with other some,
are accounted evil: whereas those very things harmonise with others,
and are good; and in themselves are good. And all these things which
harmonise not together, do yet with the inferior part, which we call
Earth, having its own cloudy and windy sky harmonising with it. Far be
it then that I should say, "These things should not be": for should I
see nought but these, I should indeed long for the better; but still
must even for these alone praise Thee; for that Thou art to be praised,
do show from the earth, dragons, and all deeps, fire, hail, snow,
ice, and stormy wind, which fulfil Thy word; mountains, and all hills,
fruitful trees, and all cedars; beasts, and all cattle, creeping things,
and flying fowls; kings of the earth, and all people, princes, and all
judges of the earth; young men and maidens, old men and young, praise
Thy Name. But when, from heaven, these praise Thee, praise Thee, our
God, in the heights all Thy angels, all Thy hosts, sun and moon, all the
stars and light, the Heaven of heavens, and the waters that be above the
heavens, praise Thy Name; I did not now long for things better, because
I conceived of all: and with a sounder judgment I apprehended that the
things above were better than these below, but altogether better than
those above by themselves.

There is no soundness in them, whom aught of Thy creation displeaseth:
as neither in me, when much which Thou hast made, displeased me. And
because my soul durst not be displeased at my God, it would fain not
account that Thine, which displeased it. Hence it had gone into the
opinion of two substances, and had no rest, but talked idly. And
returning thence, it had made to itself a God, through infinite measures
of all space; and thought it to be Thee, and placed it in its heart;
and had again become the temple of its own idol, to Thee abominable. But
after Thou hadst soothed my head, unknown to me, and closed mine eyes
that they should not behold vanity, I ceased somewhat of my former self,
and my frenzy was lulled to sleep; and I awoke in Thee, and saw Thee
infinite, but in another way, and this sight was not derived from the
flesh.

And I looked back on other things; and I saw that they owed their being
to Thee; and were all bounded in Thee: but in a different way; not as
being in space; but because Thou containest all things in Thine hand
in Thy Truth; and all things are true so far as they nor is there any
falsehood, unless when that is thought to be, which is not. And I saw
that all things did harmonise, not with their places only, but with
their seasons. And that Thou, who only art Eternal, didst not begin to
work after innumerable spaces of times spent; for that all spaces of
times, both which have passed, and which shall pass, neither go nor
come, but through Thee, working and abiding.

And I perceived and found it nothing strange, that bread which is
pleasant to a healthy palate is loathsome to one distempered: and to
sore eyes light is offensive, which to the sound is delightful. And Thy
righteousness displeaseth the wicked; much more the viper and reptiles,
which Thou hast created good, fitting in with the inferior portions of
Thy Creation, with which the very wicked also fit in; and that the more,
by how much they be unlike Thee; but with the superior creatures, by how
much they become more like to Thee. And I enquired what iniquity was,
and found it to be no substance, but the perversion of the will, turned
aside from Thee, O God, the Supreme, towards these lower things, and
casting out its bowels, and puffed up outwardly.

And I wondered that I now loved Thee, and no phantasm for Thee. And
yet did I not press on to enjoy my God; but was borne up to Thee by Thy
beauty, and soon borne down from Thee by mine own weight, sinking with
sorrow into these inferior things. This weight was carnal custom. Yet
dwelt there with me a remembrance of Thee; nor did I any way doubt that
there was One to whom I might cleave, but that I was not yet such as to
cleave to Thee: for that the body which is corrupted presseth down the
soul, and the earthly tabernacle weigheth down the mind that museth upon
many things. And most certain I was, that Thy invisible works from the
creation of the world are clearly seen, being understood by the things
that are made, even Thy eternal power and Godhead. For examining whence
it was that I admired the beauty of bodies celestial or terrestrial;
and what aided me in judging soundly on things mutable, and pronouncing,
"This ought to be thus, this not"; examining, I say, whence it was that
I so judged, seeing I did so judge, I had found the unchangeable and
true Eternity of Truth above my changeable mind. And thus by degrees
I passed from bodies to the soul, which through the bodily senses
perceives; and thence to its inward faculty, to which the bodily senses
represent things external, whitherto reach the faculties of beasts; and
thence again to the reasoning faculty, to which what is received from
the senses of the body, is referred to be judged. Which finding
itself also to be in me a thing variable, raised itself up to its
own understanding, and drew away my thoughts from the power of habit,
withdrawing itself from those troops of contradictory phantasms; that so
it might find what that light was whereby it was bedewed, when, without
all doubting, it cried out, "That the unchangeable was to be preferred
to the changeable"; whence also it knew That Unchangeable, which, unless
it had in some way known, it had had no sure ground to prefer it to the
changeable. And thus with the flash of one trembling glance it arrived
at THAT WHICH IS. And then I saw Thy invisible things understood by
the things which are made. But I could not fix my gaze thereon; and my
infirmity being struck back, I was thrown again on my wonted habits,
carrying along with me only a loving memory thereof, and a longing for
what I had, as it were, perceived the odour of, but was not yet able to
feed on.

Then I sought a way of obtaining strength sufficient to enjoy Thee; and
found it not, until I embraced that Mediator betwixt God and men, the
Man Christ Jesus, who is over all, God blessed for evermore, calling
unto me, and saying, I am the way, the truth, and the life, and mingling
that food which I was unable to receive, with our flesh. For, the Word
was made flesh, that Thy wisdom, whereby Thou createdst all things,
might provide milk for our infant state. For I did not hold to my Lord
Jesus Christ, I, humbled, to the Humble; nor knew I yet whereto His
infirmity would guide us. For Thy Word, the Eternal Truth, far above the
higher parts of Thy Creation, raises up the subdued unto Itself: but
in this lower world built for Itself a lowly habitation of our clay,
whereby to abase from themselves such as would be subdued, and bring
them over to Himself; allaying their swelling, and fomenting their love;
to the end they might go on no further in self-confidence, but rather
consent to become weak, seeing before their feet the Divinity weak by
taking our coats of skin; and wearied, might cast themselves down upon
It, and It rising, might lift them up.

But I thought otherwise; conceiving only of my Lord Christ as of a man
of excellent wisdom, whom no one could be equalled unto; especially,
for that being wonderfully born of a Virgin, He seemed, in conformity
therewith, through the Divine care for us, to have attained that great
eminence of authority, for an ensample of despising things temporal for
the obtaining of immortality. But what mystery there lay in "The Word
was made flesh," I could not even imagine. Only I had learnt out of what
is delivered to us in writing of Him that He did eat, and drink, sleep,
walk, rejoiced in spirit, was sorrowful, discoursed; that flesh did not
cleave by itself unto Thy Word, but with the human soul and mind. All
know this who know the unchangeableness of Thy Word, which I now knew,
as far as I could, nor did I at all doubt thereof. For, now to move the
limbs of the body by will, now not, now to be moved by some affection,
now not, now to deliver wise sayings through human signs, now to keep
silence, belong to soul and mind subject to variation. And should these
things be falsely written of Him, all the rest also would risk the
charge, nor would there remain in those books any saving faith for
mankind. Since then they were written truly, I acknowledged a perfect
man to be in Christ; not the body of a man only, nor, with the body, a
sensitive soul without a rational, but very man; whom, not only as being
a form of Truth, but for a certain great excellence of human nature and
a more perfect participation of wisdom, I judged to be preferred before
others. But Alypius imagined the Catholics to believe God to be so
clothed with flesh, that besides God and flesh, there was no soul at all
in Christ, and did not think that a human mind was ascribed to Him. And
because he was well persuaded that the actions recorded of Him could
only be performed by a vital and a rational creature, he moved the more
slowly towards the Christian Faith. But understanding afterwards that
this was the error of the Apollinarian heretics, he joyed in and was
conformed to the Catholic Faith. But somewhat later, I confess, did I
learn how in that saying, The Word was made flesh, the Catholic truth
is distinguished from the falsehood of Photinus. For the rejection of
heretics makes the tenets of Thy Church and sound doctrine to stand out
more clearly. For there must also be heresies, that the approved may be
made manifest among the weak.

But having then read those books of the Platonists, and thence been
taught to search for incorporeal truth, I saw Thy invisible things,
understood by those things which are made; and though cast back, I
perceived what that was which through the darkness of my mind I was
hindered from contemplating, being assured "That Thou wert, and wert
infinite, and yet not diffused in space, finite or infinite; and that
Thou truly art Who art the same ever, in no part nor motion varying;
and that all other things are from Thee, on this most sure ground alone,
that they are." Of these things I was assured, yet too unsure to enjoy
Thee. I prated as one well skilled; but had I not sought Thy way in
Christ our Saviour, I had proved to be, not skilled, but killed. For now
I had begun to wish to seem wise, being filled with mine own punishment,
yet I did not mourn, but rather scorn, puffed up with knowledge. For
where was that charity building upon the foundation of humility, which
is Christ Jesus? or when should these books teach me it? Upon these, I
believe, Thou therefore willedst that I should fall, before I studied
Thy Scriptures, that it might be imprinted on my memory how I was
affected by them; and that afterwards when my spirits were tamed through
Thy books, and my wounds touched by Thy healing fingers, I might discern
and distinguish between presumption and confession; between those who
saw whither they were to go, yet saw not the way, and the way that
leadeth not to behold only but to dwell in the beatific country. For
had I first been formed in Thy Holy Scriptures, and hadst Thou in the
familiar use of them grown sweet unto me, and had I then fallen upon
those other volumes, they might perhaps have withdrawn me from the solid
ground of piety, or, had I continued in that healthful frame which I had
thence imbibed, I might have thought that it might have been obtained by
the study of those books alone.

Most eagerly then did I seize that venerable writing of Thy Spirit; and
chiefly the Apostle Paul. Whereupon those difficulties vanished away,
wherein he once seemed to me to contradict himself, and the text of his
discourse not to agree with the testimonies of the Law and the Prophets.
And the face of that pure word appeared to me one and the same; and I
learned to rejoice with trembling. So I began; and whatsoever truth
I had read in those other books, I found here amid the praise of Thy
Grace; that whoso sees, may not so glory as if he had not received, not
only what he sees, but also that he sees (for what hath he, which he
hath not received?), and that he may be not only admonished to behold
Thee, who art ever the same, but also healed, to hold Thee; and that he
who cannot see afar off, may yet walk on the way, whereby he may arrive,
and behold, and hold Thee. For, though a man be delighted with the law
of God after the inner man, what shall he do with that other law in his
members which warreth against the law of his mind, and bringeth him
into captivity to the law of sin which is in his members? For, Thou art
righteous, O Lord, but we have sinned and committed iniquity, and have
done wickedly, and Thy hand is grown heavy upon us, and we are justly
delivered over unto that ancient sinner, the king of death; because
he persuaded our will to be like his will whereby he abode not in Thy
truth. What shall wretched man do? who shall deliver him from the body
of his death, but only Thy Grace, through Jesus Christ our Lord, whom
Thou hast begotten co-eternal, and formedst in the beginning of Thy
ways, in whom the prince of this world found nothing worthy of death,
yet killed he Him; and the handwriting, which was contrary to us, was
blotted out? This those writings contain not. Those pages present not
the image of this piety, the tears of confession, Thy sacrifice, a
troubled spirit, a broken and a contrite heart, the salvation of the
people, the Bridal City, the earnest of the Holy Ghost, the Cup of our
Redemption. No man sings there, Shall not my soul be submitted unto God?
for of Him cometh my salvation. For He is my God and my salvation, my
guardian, I shall no more be moved. No one there hears Him call, Come
unto Me, all ye that labour. They scorn to learn of Him, because He is
meek and lowly in heart; for these things hast Thou hid from the wise
and prudent, and hast revealed them unto babes. For it is one thing,
from the mountain's shaggy top to see the land of peace, and to find no
way thither; and in vain to essay through ways unpassable, opposed and
beset by fugitives and deserters, under their captain the lion and the
dragon: and another to keep on the way that leads thither, guarded by
the host of the heavenly General; where they spoil not who have deserted
the heavenly army; for they avoid it, as very torment. These things did
wonderfully sink into my bowels, when I read that least of Thy Apostles,
and had meditated upon Thy works, and trembled exceedingly.




\chapter{BOOK VIII}


\start{O}{ my} God, let me, with thanksgiving, remember, and confess unto Thee Thy
mercies on me. Let my bones be bedewed with Thy love, and let them say
unto Thee, Who is like unto Thee, O Lord? Thou hast broken my bonds in
sunder, I will offer unto Thee the sacrifice of thanksgiving. And how
Thou hast broken them, I will declare; and all who worship Thee, when
they hear this, shall say, "Blessed be the Lord, in heaven and in earth,
great and wonderful is his name." Thy words had stuck fast in my heart,
and I was hedged round about on all sides by Thee. Of Thy eternal life I
was now certain, though I saw it in a figure and as through a glass. Yet
I had ceased to doubt that there was an incorruptible substance, whence
was all other substance; nor did I now desire to be more certain of
Thee, but more steadfast in Thee. But for my temporal life, all was
wavering, and my heart had to be purged from the old leaven. The Way,
the Saviour Himself, well pleased me, but as yet I shrunk from going
through its straitness. And Thou didst put into my mind, and it seemed
good in my eyes, to go to Simplicianus, who seemed to me a good servant
of Thine; and Thy grace shone in him. I had heard also that from his
very youth he had lived most devoted unto Thee. Now he was grown into
years; and by reason of so great age spent in such zealous following of
Thy ways, he seemed to me likely to have learned much experience; and
so he had. Out of which store I wished that he would tell me (setting
before him my anxieties) which were the fittest way for one in my case
to walk in Thy paths.

For, I saw the church full; and one went this way, and another that way.
But I was displeased that I led a secular life; yea now that my desires
no longer inflamed me, as of old, with hopes of honour and profit,
a very grievous burden it was to undergo so heavy a bondage. For, in
comparison of Thy sweetness, and the beauty of Thy house which I loved,
those things delighted me no longer. But still I was enthralled with
the love of woman; nor did the Apostle forbid me to marry, although he
advised me to something better, chiefly wishing that all men were as
himself was. But I being weak, chose the more indulgent place; and
because of this alone, was tossed up and down in all beside, faint and
wasted with withering cares, because in other matters I was constrained
against my will to conform myself to a married life, to which I was
given up and enthralled. I had heard from the mouth of the Truth,
that there were some eunuchs which had made themselves eunuchs for the
kingdom of heaven's sake: but, saith He, let him who can receive it,
receive it. Surely vain are all men who are ignorant of God, and could
not out of the good things which are seen, find out Him who is good. But
I was no longer in that vanity; I had surmounted it; and by the common
witness of all Thy creatures had found Thee our Creator, and Thy Word,
God with Thee, and together with Thee one God, by whom Thou createdst
all things. There is yet another kind of ungodly, who knowing God,
glorified Him not as God, neither were thankful. Into this also had
I fallen, but Thy right hand upheld me, and took me thence, and Thou
placedst me where I might recover. For Thou hast said unto man, Behold,
the fear of the Lord is wisdom, and, Desire not to seem wise; because
they who affirmed themselves to be wise, became fools. But I had now
found the goodly pearl, which, selling all that I had, I ought to have
bought, and I hesitated.

To Simplicianus then I went, the father of Ambrose (a Bishop now) in
receiving Thy grace, and whom Ambrose truly loved as a father. To him I
related the mazes of my wanderings. But when I mentioned that I had read
certain books of the Platonists, which Victorinus, sometime Rhetoric
Professor of Rome (who had died a Christian, as I had heard), had
translated into Latin, he testified his joy that I had not fallen upon
the writings of other philosophers, full of fallacies and deceits, after
the rudiments of this world, whereas the Platonists many ways led to the
belief in God and His Word. Then to exhort me to the humility of
Christ, hidden from the wise, and revealed to little ones, he spoke of
Victorinus himself, whom while at Rome he had most intimately known: and
of him he related what I will not conceal. For it contains great praise
of Thy grace, to be confessed unto Thee, how that aged man, most learned
and skilled in the liberal sciences, and who had read, and weighed
so many works of the philosophers; the instructor of so many noble
Senators, who also, as a monument of his excellent discharge of his
office, had (which men of this world esteem a high honour) both deserved
and obtained a statue in the Roman Forum; he, to that age a worshipper
of idols, and a partaker of the sacrilegious rites, to which almost all
the nobility of Rome were given up, and had inspired the people with the
love of

         Anubis, barking Deity, and all
         The monster Gods of every kind, who fought
         'Gainst Neptune, Venus, and Minerva:

whom Rome once conquered, now adored, all which the aged Victorinus had
with thundering eloquence so many years defended;--he now blushed not
to be the child of Thy Christ, and the new-born babe of Thy fountain;
submitting his neck to the yoke of humility, and subduing his forehead
to the reproach of the Cross.

O Lord, Lord, Which hast bowed the heavens and come down, touched the
mountains and they did smoke, by what means didst Thou convey Thyself
into that breast? He used to read (as Simplicianus said) the holy
Scripture, most studiously sought and searched into all the Christian
writings, and said to Simplicianus (not openly, but privately and as
a friend), "Understand that I am already a Christian." Whereto he
answered, "I will not believe it, nor will I rank you among Christians,
unless I see you in the Church of Christ." The other, in banter,
replied, "Do walls then make Christians?" And this he often said, that
he was already a Christian; and Simplicianus as often made the same
answer, and the conceit of the "walls" was by the other as often
renewed. For he feared to offend his friends, proud daemon-worshippers,
from the height of whose Babylonian dignity, as from cedars of Libanus,
which the Lord had not yet broken down, he supposed the weight of enmity
would fall upon him. But after that by reading and earnest thought he
had gathered firmness, and feared to be denied by Christ before the holy
angels, should he now be afraid to confess Him before men, and appeared
to himself guilty of a heavy offence, in being ashamed of the Sacraments
of the humility of Thy Word, and not being ashamed of the sacrilegious
rites of those proud daemons, whose pride he had imitated and their
rites adopted, he became bold-faced against vanity, and shame-faced
towards the truth, and suddenly and unexpectedly said to Simplicianus
(as himself told me), "Go we to the Church; I wish to be made a
Christian." But he, not containing himself for joy, went with him. And
having been admitted to the first Sacrament and become a Catechumen, not
long after he further gave in his name, that he might be regenerated by
baptism, Rome wondering, the Church rejoicing. The proud saw, and were
wroth; they gnashed with their teeth, and melted away. But the Lord
God was the hope of Thy servant, and he regarded not vanities and lying
madness.

To conclude, when the hour was come for making profession of his faith
(which at Rome they, who are about to approach to Thy grace, deliver,
from an elevated place, in the sight of all the faithful, in a set
form of words committed to memory), the presbyters, he said, offered
Victorinus (as was done to such as seemed likely through bashfulness to
be alarmed) to make his profession more privately: but he chose rather
to profess his salvation in the presence of the holy multitude. "For
it was not salvation that he taught in rhetoric, and yet that he had
publicly professed: how much less then ought he, when pronouncing Thy
word, to dread Thy meek flock, who, when delivering his own words,
had not feared a mad multitude!" When, then, he went up to make his
profession, all, as they knew him, whispered his name one to another
with the voice of congratulation. And who there knew him not? and there
ran a low murmur through all the mouths of the rejoicing multitude,
Victorinus! Victorinus! Sudden was the burst of rapture, that they saw
him; suddenly were they hushed that they might hear him. He pronounced
the true faith with an excellent boldness, and all wished to draw him
into their very heart; yea by their love and joy they drew him thither,
such were the hands wherewith they drew him.

Good God! what takes place in man, that he should more rejoice at the
salvation of a soul despaired of, and freed from greater peril, than if
there had always been hope of him, or the danger had been less? For so
Thou also, merciful Father, dost more rejoice over one penitent than
over ninety-nine just persons that need no repentance. And with much
joyfulness do we hear, so often as we hear with what joy the sheep which
had strayed is brought back upon the shepherd's shoulder, and the groat
is restored to Thy treasury, the neighbours rejoicing with the woman
who found it; and the joy of the solemn service of Thy house forceth
to tears, when in Thy house it is read of Thy younger son, that he was
dead, and liveth again; had been lost, and is found. For Thou rejoicest
in us, and in Thy holy angels, holy through holy charity. For Thou art
ever the same; for all things which abide not the same nor for ever,
Thou for ever knowest in the same way.

What then takes place in the soul, when it is more delighted at finding
or recovering the things it loves, than if it had ever had them? yea,
and other things witness hereunto; and all things are full of witnesses,
crying out, "So is it." The conquering commander triumpheth; yet had he
not conquered unless he had fought; and the more peril there was in the
battle, so much the more joy is there in the triumph. The storm tosses
the sailors, threatens shipwreck; all wax pale at approaching death;
sky and sea are calmed, and they are exceeding joyed, as having been
exceeding afraid. A friend is sick, and his pulse threatens danger; all
who long for his recovery are sick in mind with him. He is restored,
though as yet he walks not with his former strength; yet there is such
joy, as was not, when before he walked sound and strong. Yea, the very
pleasures of human life men acquire by difficulties, not those only
which fall upon us unlooked for, and against our wills, but even by
self-chosen, and pleasure-seeking trouble. Eating and drinking have no
pleasure, unless there precede the pinching of hunger and thirst. Men,
given to drink, eat certain salt meats, to procure a troublesome heat,
which the drink allaying, causes pleasure. It is also ordered that the
affianced bride should not at once be given, lest as a husband he should
hold cheap whom, as betrothed, he sighed not after.

This law holds in foul and accursed joy; this in permitted and lawful
joy; this in the very purest perfection of friendship; this, in him who
was dead, and lived again; had been lost and was found. Every where the
greater joy is ushered in by the greater pain. What means this, O Lord
my God, whereas Thou art everlastingly joy to Thyself, and some things
around Thee evermore rejoice in Thee? What means this, that this portion
of things thus ebbs and flows alternately displeased and reconciled?
Is this their allotted measure? Is this all Thou hast assigned to them,
whereas from the highest heavens to the lowest earth, from the beginning
of the world to the end of ages, from the angel to the worm, from the
first motion to the last, Thou settest each in its place, and realisest
each in their season, every thing good after its kind? Woe is me! how
high art Thou in the highest, and how deep in the deepest! and Thou
never departest, and we scarcely return to Thee.

Up, Lord, and do; stir us up, and recall us; kindle and draw us;
inflame, grow sweet unto us, let us now love, let us run. Do not many,
out of a deeper hell of blindness than Victorinus, return to Thee,
approach, and are enlightened, receiving that Light, which they who
receive, receive power from Thee to become Thy sons? But if they be less
known to the nations, even they that know them, joy less for them. For
when many joy together, each also has more exuberant joy for that they
are kindled and inflamed one by the other. Again, because those known to
many, influence the more towards salvation, and lead the way with many
to follow. And therefore do they also who preceded them much rejoice in
them, because they rejoice not in them alone. For far be it, that in Thy
tabernacle the persons of the rich should be accepted before the poor,
or the noble before the ignoble; seeing rather Thou hast chosen the weak
things of the world to confound the strong; and the base things of this
world, and the things despised hast Thou chosen, and those things which
are not, that Thou mightest bring to nought things that are. And yet
even that least of Thy apostles, by whose tongue Thou soundedst forth
these words, when through his warfare, Paulus the Proconsul, his pride
conquered, was made to pass under the easy yoke of Thy Christ, and
became a provincial of the great King; he also for his former name Saul,
was pleased to be called Paul, in testimony of so great a victory. For
the enemy is more overcome in one, of whom he hath more hold; by whom
he hath hold of more. But the proud he hath more hold of, through their
nobility; and by them, of more through their authority. By how much the
more welcome then the heart of Victorinus was esteemed, which the devil
had held as an impregnable possession, the tongue of Victorinus,
with which mighty and keen weapon he had slain many; so much the more
abundantly ought Thy sons to rejoice, for that our King hath bound the
strong man, and they saw his vessels taken from him and cleansed, and
made meet for Thy honour; and become serviceable for the Lord, unto
every good work.

But when that man of Thine, Simplicianus, related to me this of
Victorinus, I was on fire to imitate him; for for this very end had
he related it. But when he had subjoined also, how in the days of the
Emperor Julian a law was made, whereby Christians were forbidden to
teach the liberal sciences or oratory; and how he, obeying this law,
chose rather to give over the wordy school than Thy Word, by which
Thou makest eloquent the tongues of the dumb; he seemed to me not more
resolute than blessed, in having thus found opportunity to wait on Thee
only. Which thing I was sighing for, bound as I was, not with another's
irons, but by my own iron will. My will the enemy held, and thence had
made a chain for me, and bound me. For of a forward will, was a lust
made; and a lust served, became custom; and custom not resisted, became
necessity. By which links, as it were, joined together (whence I called
it a chain) a hard bondage held me enthralled. But that new will which
had begun to be in me, freely to serve Thee, and to wish to enjoy Thee,
O God, the only assured pleasantness, was not yet able to overcome my
former wilfulness, strengthened by age. Thus did my two wills, one new,
and the other old, one carnal, the other spiritual, struggle within me;
and by their discord, undid my soul.

Thus, I understood, by my own experience, what I had read, how the flesh
lusteth against the spirit and the spirit against the flesh. Myself
verily either way; yet more myself, in that which I approved in myself,
than in that which in myself I disapproved. For in this last, it was now
for the more part not myself, because in much I rather endured against
my will, than acted willingly. And yet it was through me that custom had
obtained this power of warring against me, because I had come willingly,
whither I willed not. And who has any right to speak against it, if just
punishment follow the sinner? Nor had I now any longer my former plea,
that I therefore as yet hesitated to be above the world and serve Thee,
for that the truth was not altogether ascertained to me; for now it too
was. But I still under service to the earth, refused to fight under Thy
banner, and feared as much to be freed of all incumbrances, as we should
fear to be encumbered with it. Thus with the baggage of this present
world was I held down pleasantly, as in sleep: and the thoughts wherein
I meditated on Thee were like the efforts of such as would awake, who
yet overcome with a heavy drowsiness, are again drenched therein. And as
no one would sleep for ever, and in all men's sober judgment waking is
better, yet a man for the most part, feeling a heavy lethargy in all his
limbs, defers to shake off sleep, and though half displeased, yet, even
after it is time to rise, with pleasure yields to it, so was I assured
that much better were it for me to give myself up to Thy charity, than
to give myself over to mine own cupidity; but though the former course
satisfied me and gained the mastery, the latter pleased me and held me
mastered. Nor had I any thing to answer Thee calling to me, Awake,
thou that sleepest, and arise from the dead, and Christ shall give thee
light. And when Thou didst on all sides show me that what Thou saidst
was true, I, convicted by the truth, had nothing at all to answer, but
only those dull and drowsy words, "Anon, anon," "presently," "leave
me but a little." But "presently, presently," had no present, and my
"little while" went on for a long while; in vain I delighted in Thy
law according to the inner man, when another law in my members rebelled
against the law of my mind, and led me captive under the law of sin
which was in my members. For the law of sin is the violence of custom,
whereby the mind is drawn and holden, even against its will; but
deservedly, for that it willingly fell into it. Who then should deliver
me thus wretched from the body of this death, but Thy grace only,
through Jesus Christ our Lord?

And how Thou didst deliver me out of the bonds of desire, wherewith I
was bound most straitly to carnal concupiscence, and out of the drudgery
of worldly things, I will now declare, and confess unto Thy name, O
Lord, my helper and my redeemer. Amid increasing anxiety, I was doing
my wonted business, and daily sighing unto Thee. I attended Thy Church,
whenever free from the business under the burden of which I groaned.
Alypius was with me, now after the third sitting released from his law
business, and awaiting to whom to sell his counsel, as I sold the skill
of speaking, if indeed teaching can impart it. Nebridius had now, in
consideration of our friendship, consented to teach under Verecundus, a
citizen and a grammarian of Milan, and a very intimate friend of us all;
who urgently desired, and by the right of friendship challenged from our
company, such faithful aid as he greatly needed. Nebridius then was not
drawn to this by any desire of advantage (for he might have made much
more of his learning had he so willed), but as a most kind and gentle
friend, he would not be wanting to a good office, and slight our
request. But he acted herein very discreetly, shunning to become known
to personages great according to this world, avoiding the distraction
of mind thence ensuing, and desiring to have it free and at leisure, as
many hours as might be, to seek, or read, or hear something concerning
wisdom.

Upon a day then, Nebridius being absent (I recollect not why), lo, there
came to see me and Alypius, one Pontitianus, our countryman so far as
being an African, in high office in the Emperor's court. What he would
with us, I know not, but we sat down to converse, and it happened that
upon a table for some game, before us, he observed a book, took, opened
it, and contrary to his expectation, found it the Apostle Paul; for he
thought it some of those books which I was wearing myself in teaching.
Whereat smiling, and looking at me, he expressed his joy and wonder that
he had on a sudden found this book, and this only before my eyes. For he
was a Christian, and baptised, and often bowed himself before Thee our
God in the Church, in frequent and continued prayers. When then I had
told him that I bestowed very great pains upon those Scriptures, a
conversation arose (suggested by his account) on Antony the Egyptian
monk: whose name was in high reputation among Thy servants, though to
that hour unknown to us. Which when he discovered, he dwelt the more
upon that subject, informing and wondering at our ignorance of one so
eminent. But we stood amazed, hearing Thy wonderful works most fully
attested, in times so recent, and almost in our own, wrought in the true
Faith and Church Catholic. We all wondered; we, that they were so great,
and he, that they had not reached us.

Thence his discourse turned to the flocks in the monasteries, and their
holy ways, a sweet-smelling savour unto Thee, and the fruitful deserts
of the wilderness, whereof we knew nothing. And there was a monastery
at Milan, full of good brethren, without the city walls, under the
fostering care of Ambrose, and we knew it not. He went on with his
discourse, and we listened in intent silence. He told us then how one
afternoon at Triers, when the Emperor was taken up with the Circensian
games, he and three others, his companions, went out to walk in gardens
near the city walls, and there as they happened to walk in pairs, one
went apart with him, and the other two wandered by themselves; and
these, in their wanderings, lighted upon a certain cottage, inhabited
by certain of Thy servants, poor in spirit, of whom is the kingdom
of heaven, and there they found a little book containing the life of
Antony. This one of them began to read, admire, and kindle at it; and
as he read, to meditate on taking up such a life, and giving over his
secular service to serve Thee. And these two were of those whom they
style agents for the public affairs. Then suddenly, filled with a holy
love, and a sober shame, in anger with himself cast his eyes upon his
friend, saying, "Tell me, I pray thee, what would we attain by all these
labours of ours? what aim we at? what serve we for? Can our hopes in
court rise higher than to be the Emperor's favourites? and in this, what
is there not brittle, and full of perils? and by how many perils arrive
we at a greater peril? and when arrive we thither? But a friend of God,
if I wish it, I become now at once." So spake he. And in pain with the
travail of a new life, he turned his eyes again upon the book, and
read on, and was changed inwardly, where Thou sawest, and his mind was
stripped of the world, as soon appeared. For as he read, and rolled up
and down the waves of his heart, he stormed at himself a while, then
discerned, and determined on a better course; and now being Thine, said
to his friend, "Now have I broken loose from those our hopes, and am
resolved to serve God; and this, from this hour, in this place, I begin
upon. If thou likest not to imitate me, oppose not." The other answered,
he would cleave to him, to partake so glorious a reward, so glorious
a service. Thus both being now Thine, were building the tower at the
necessary cost, the forsaking all that they had, and following Thee.
Then Pontitianus and the other with him, that had walked in other parts
of the garden, came in search of them to the same place; and finding
them, reminded them to return, for the day was now far spent. But they
relating their resolution and purpose, and how that will was begun and
settled in them, begged them, if they would not join, not to molest
them. But the others, though nothing altered from their former selves,
did yet bewail themselves (as he affirmed), and piously congratulated
them, recommending themselves to their prayers; and so, with hearts
lingering on the earth, went away to the palace. But the other two,
fixing their heart on heaven, remained in the cottage. And both had
affianced brides, who when they heard hereof, also dedicated their
virginity unto God.

Such was the story of Pontitianus; but Thou, O Lord, while he was
speaking, didst turn me round towards myself, taking me from behind my
back where I had placed me, unwilling to observe myself; and setting
me before my face, that I might see how foul I was, how crooked and
defiled, bespotted and ulcerous. And I beheld and stood aghast; and
whither to flee from myself I found not. And if I sought to turn mine
eye from off myself, he went on with his relation, and Thou again didst
set me over against myself, and thrustedst me before my eyes, that I
might find out mine iniquity, and hate it. I had known it, but made as
though I saw it not, winked at it, and forgot it.

But now, the more ardently I loved those whose healthful affections I
heard of, that they had resigned themselves wholly to Thee to be cured,
the more did I abhor myself, when compared with them. For many of my
years (some twelve) had now run out with me since my nineteenth, when,
upon the reading of Cicero's Hortensius, I was stirred to an earnest
love of wisdom; and still I was deferring to reject mere earthly
felicity, and give myself to search out that, whereof not the finding
only, but the very search, was to be preferred to the treasures and
kingdoms of the world, though already found, and to the pleasures of the
body, though spread around me at my will. But I wretched, most wretched,
in the very commencement of my early youth, had begged chastity of Thee,
and said, "Give me chastity and continency, only not yet." For I feared
lest Thou shouldest hear me soon, and soon cure me of the disease
of concupiscence, which I wished to have satisfied, rather than
extinguished. And I had wandered through crooked ways in a sacrilegious
superstition, not indeed assured thereof, but as preferring it to the
others which I did not seek religiously, but opposed maliciously.

And I had thought that I therefore deferred from day to day to reject
the hopes of this world, and follow Thee only, because there did not
appear aught certain, whither to direct my course. And now was the day
come wherein I was to be laid bare to myself, and my conscience was
to upbraid me. "Where art thou now, my tongue? Thou saidst that for an
uncertain truth thou likedst not to cast off the baggage of vanity; now,
it is certain, and yet that burden still oppresseth thee, while they who
neither have so worn themselves out with seeking it, nor for often years
and more have been thinking thereon, have had their shoulders lightened,
and received wings to fly away." Thus was I gnawed within, and
exceedingly confounded with a horrible shame, while Pontitianus was so
speaking. And he having brought to a close his tale and the business
he came for, went his way; and I into myself. What said I not against
myself? with what scourges of condemnation lashed I not my soul, that it
might follow me, striving to go after Thee! Yet it drew back; refused,
but excused not itself. All arguments were spent and confuted; there
remained a mute shrinking; and she feared, as she would death, to be
restrained from the flux of that custom, whereby she was wasting to
death.

Then in this great contention of my inward dwelling, which I had
strongly raised against my soul, in the chamber of my heart, troubled in
mind and countenance, I turned upon Alypius. "What ails us?" I exclaim:
"what is it? what heardest thou? The unlearned start up and take heaven
by force, and we with our learning, and without heart, lo, where we
wallow in flesh and blood! Are we ashamed to follow, because others
are gone before, and not ashamed not even to follow?" Some such words I
uttered, and my fever of mind tore me away from him, while he, gazing on
me in astonishment, kept silence. For it was not my wonted tone; and my
forehead, cheeks, eyes, colour, tone of voice, spake my mind more than
the words I uttered. A little garden there was to our lodging, which we
had the use of, as of the whole house; for the master of the house, our
host, was not living there. Thither had the tumult of my breast hurried
me, where no man might hinder the hot contention wherein I had engaged
with myself, until it should end as Thou knewest, I knew not. Only I
was healthfully distracted and dying, to live; knowing what evil thing I
was, and not knowing what good thing I was shortly to become. I retired
then into the garden, and Alypius, on my steps. For his presence did not
lessen my privacy; or how could he forsake me so disturbed? We sate down
as far removed as might be from the house. I was troubled in spirit,
most vehemently indignant that I entered not into Thy will and covenant,
O my God, which all my bones cried out unto me to enter, and praised it
to the skies. And therein we enter not by ships, or chariots, or feet,
no, move not so far as I had come from the house to that place where we
were sitting. For, not to go only, but to go in thither was nothing else
but to will to go, but to will resolutely and thoroughly; not to turn
and toss, this way and that, a maimed and half-divided will, struggling,
with one part sinking as another rose.

Lastly, in the very fever of my irresoluteness, I made with my body many
such motions as men sometimes would, but cannot, if either they have not
the limbs, or these be bound with bands, weakened with infirmity, or
any other way hindered. Thus, if I tore my hair, beat my forehead, if
locking my fingers I clasped my knee; I willed, I did it. But I might
have willed, and not done it; if the power of motion in my limbs had not
obeyed. So many things then I did, when "to will" was not in itself "to
be able"; and I did not what both I longed incomparably more to do, and
which soon after, when I should will, I should be able to do; because
soon after, when I should will, I should will thoroughly. For in these
things the ability was one with the will, and to will was to do; and yet
was it not done: and more easily did my body obey the weakest willing of
my soul, in moving its limbs at its nod, than the soul obeyed itself to
accomplish in the will alone this its momentous will.

Whence is this monstrousness? and to what end? Let Thy mercy gleam that
I may ask, if so be the secret penalties of men, and those darkest
pangs of the sons of Adam, may perhaps answer me. Whence is this
monstrousness? and to what end? The mind commands the body, and it obeys
instantly; the mind commands itself, and is resisted. The mind commands
the hand to be moved; and such readiness is there, that command is
scarce distinct from obedience. Yet the mind is mind, the hand is body.
The mind commands the mind, its own self, to will, and yet it doth not.
Whence this monstrousness? and to what end? It commands itself, I say,
to will, and would not command, unless it willed, and what it commands
is not done. But it willeth not entirely: therefore doth it not command
entirely. For so far forth it commandeth, as it willeth: and, so far
forth is the thing commanded, not done, as it willeth not. For the will
commandeth that there be a will; not another, but itself. But it doth
not command entirely, therefore what it commandeth, is not. For were
the will entire, it would not even command it to be, because it would
already be. It is therefore no monstrousness partly to will, partly to
nill, but a disease of the mind, that it doth not wholly rise, by truth
upborne, borne down by custom. And therefore are there two wills, for
that one of them is not entire: and what the one lacketh, the other
hath.

Let them perish from Thy presence, O God, as perish vain talkers and
seducers of the soul: who observing that in deliberating there were two
wills, affirm that there are two minds in us of two kinds, one good, the
other evil. Themselves are truly evil, when they hold these evil things;
and themselves shall become good when they hold the truth and assent
unto the truth, that Thy Apostle may say to them, Ye were sometimes
darkness, but now light in the Lord. But they, wishing to be light, not
in the Lord, but in themselves, imagining the nature of the soul to
be that which God is, are made more gross darkness through a dreadful
arrogancy; for that they went back farther from Thee, the true Light
that enlightened every man that cometh into the world. Take heed what
you say, and blush for shame: draw near unto Him and be enlightened,
and your faces shall not be ashamed. Myself when I was deliberating
upon serving the Lord my God now, as I had long purposed, it was I who
willed, I who nilled, I, I myself. I neither willed entirely, nor nilled
entirely. Therefore was I at strife with myself, and rent asunder by
myself. And this rent befell me against my will, and yet indicated, not
the presence of another mind, but the punishment of my own. Therefore it
was no more I that wrought it, but sin that dwelt in me; the punishment
of a sin more freely committed, in that I was a son of Adam.

For if there be so many contrary natures as there be conflicting wills,
there shall now be not two only, but many. If a man deliberate whether
he should go to their conventicle or to the theatre, these Manichees
cry out, Behold, here are two natures: one good, draws this way; another
bad, draws back that way. For whence else is this hesitation between
conflicting wills? But I say that both be bad: that which draws to them,
as that which draws back to the theatre. But they believe not that
will to be other than good, which draws to them. What then if one of us
should deliberate, and amid the strife of his two wills be in a strait,
whether he should go to the theatre or to our church? would not these
Manichees also be in a strait what to answer? For either they must
confess (which they fain would not) that the will which leads to our
church is good, as well as theirs, who have received and are held by the
mysteries of theirs: or they must suppose two evil natures, and two evil
souls conflicting in one man, and it will not be true, which they say,
that there is one good and another bad; or they must be converted to the
truth, and no more deny that where one deliberates, one soul fluctuates
between contrary wills.

Let them no more say then, when they perceive two conflicting wills
in one man, that the conflict is between two contrary souls, of two
contrary substances, from two contrary principles, one good, and the
other bad. For Thou, O true God, dost disprove, check, and convict them;
as when, both wills being bad, one deliberates whether he should kill
a man by poison or by the sword; whether he should seize this or that
estate of another's, when he cannot both; whether he should purchase
pleasure by luxury, or keep his money by covetousness; whether he go to
the circus or the theatre, if both be open on one day; or thirdly, to
rob another's house, if he have the opportunity; or, fourthly, to commit
adultery, if at the same time he have the means thereof also; all these
meeting together in the same juncture of time, and all being equally
desired, which cannot at one time be acted: for they rend the mind
amid four, or even (amid the vast variety of things desired) more,
conflicting wills, nor do they yet allege that there are so many divers
substances. So also in wills which are good. For I ask them, is it good
to take pleasure in reading the Apostle? or good to take pleasure in
a sober Psalm? or good to discourse on the Gospel? They will answer to
each, "it is good." What then if all give equal pleasure, and all at
once? Do not divers wills distract the mind, while he deliberates which
he should rather choose? yet are they all good, and are at variance till
one be chosen, whither the one entire will may be borne, which before
was divided into many. Thus also, when, above, eternity delights us, and
the pleasure of temporal good holds us down below, it is the same soul
which willeth not this or that with an entire will; and therefore is
rent asunder with grievous perplexities, while out of truth it sets this
first, but out of habit sets not that aside.

Thus soul-sick was I, and tormented, accusing myself much more severely
than my wont, rolling and turning me in my chain, till that were wholly
broken, whereby I now was but just, but still was, held. And Thou, O
Lord, pressedst upon me in my inward parts by a severe mercy, redoubling
the lashes of fear and shame, lest I should again give way, and not
bursting that same slight remaining tie, it should recover strength, and
bind me the faster. For I said with myself, "Be it done now, be it done
now." And as I spake, I all but enacted it: I all but did it, and did
it not: yet sunk not back to my former state, but kept my stand hard by,
and took breath. And I essayed again, and wanted somewhat less of it,
and somewhat less, and all but touched, and laid hold of it; and yet
came not at it, nor touched nor laid hold of it; hesitating to die to
death and to live to life: and the worse whereto I was inured, prevailed
more with me than the better whereto I was unused: and the very moment
wherein I was to become other than I was, the nearer it approached me,
the greater horror did it strike into me; yet did it not strike me back,
nor turned me away, but held me in suspense.

The very toys of toys, and vanities of vanities, my ancient mistresses,
still held me; they plucked my fleshy garment, and whispered softly,
"Dost thou cast us off? and from that moment shall we no more be with
thee for ever? and from that moment shall not this or that be lawful
for thee for ever?" And what was it which they suggested in that I said,
"this or that," what did they suggest, O my God? Let Thy mercy turn it
away from the soul of Thy servant. What defilements did they suggest!
what shame! And now I much less than half heard them, and not openly
showing themselves and contradicting me, but muttering as it were behind
my back, and privily plucking me, as I was departing, but to look back
on them. Yet they did retard me, so that I hesitated to burst and
shake myself free from them, and to spring over whither I was called;
a violent habit saying to me, "Thinkest thou, thou canst live without
them?"

But now it spake very faintly. For on that side whither I had set my
face, and whither I trembled to go, there appeared unto me the chaste
dignity of Continency, serene, yet not relaxedly, gay, honestly alluring
me to come and doubt not; and stretching forth to receive and embrace
me, her holy hands full of multitudes of good examples: there were so
many young men and maidens here, a multitude of youth and every age,
grave widows and aged virgins; and Continence herself in all, not
barren, but a fruitful mother of children of joys, by Thee her Husband,
O Lord. And she smiled on me with a persuasive mockery, as would she
say, "Canst not thou what these youths, what these maidens can? or can
they either in themselves, and not rather in the Lord their God? The
Lord their God gave me unto them. Why standest thou in thyself, and
so standest not? cast thyself upon Him, fear not He will not withdraw
Himself that thou shouldest fall; cast thyself fearlessly upon Him, He
will receive, and will heal thee." And I blushed exceedingly, for that
I yet heard the muttering of those toys, and hung in suspense. And she
again seemed to say, "Stop thine ears against those thy unclean members
on the earth, that they may be mortified. They tell thee of delights,
but not as doth the law of the Lord thy God." This controversy in my
heart was self against self only. But Alypius sitting close by my side,
in silence waited the issue of my unwonted emotion.

But when a deep consideration had from the secret bottom of my soul
drawn together and heaped up all my misery in the sight of my heart;
there arose a mighty storm, bringing a mighty shower of tears. Which
that I might pour forth wholly, in its natural expressions, I rose from
Alypius: solitude was suggested to me as fitter for the business of
weeping; so I retired so far that even his presence could not be a
burden to me. Thus was it then with me, and he perceived something of
it; for something I suppose I had spoken, wherein the tones of my voice
appeared choked with weeping, and so had risen up. He then remained
where we were sitting, most extremely astonished. I cast myself down I
know not how, under a certain fig-tree, giving full vent to my tears;
and the floods of mine eyes gushed out an acceptable sacrifice to Thee.
And, not indeed in these words, yet to this purpose, spake I much unto
Thee: and Thou, O Lord, how long? how long, Lord, wilt Thou be angry for
ever? Remember not our former iniquities, for I felt that I was held by
them. I sent up these sorrowful words: How long, how long, "to-morrow,
and tomorrow?" Why not now? why not is there this hour an end to my
uncleanness?

So was I speaking and weeping in the most bitter contrition of my heart,
when, lo! I heard from a neighbouring house a voice, as of boy or girl,
I know not, chanting, and oft repeating, "Take up and read; Take up and
read." Instantly, my countenance altered, I began to think most intently
whether children were wont in any kind of play to sing such words: nor
could I remember ever to have heard the like. So checking the torrent
of my tears, I arose; interpreting it to be no other than a command from
God to open the book, and read the first chapter I should find. For I
had heard of Antony, that coming in during the reading of the Gospel,
he received the admonition, as if what was being read was spoken to him:
Go, sell all that thou hast, and give to the poor, and thou shalt have
treasure in heaven, and come and follow me: and by such oracle he was
forthwith converted unto Thee. Eagerly then I returned to the place
where Alypius was sitting; for there had I laid the volume of the
Apostle when I arose thence. I seized, opened, and in silence read that
section on which my eyes first fell: Not in rioting and drunkenness, not
in chambering and wantonness, not in strife and envying; but put ye
on the Lord Jesus Christ, and make not provision for the flesh, in
concupiscence. No further would I read; nor needed I: for instantly at
the end of this sentence, by a light as it were of serenity infused into
my heart, all the darkness of doubt vanished away.

Then putting my finger between, or some other mark, I shut the volume,
and with a calmed countenance made it known to Alypius. And what was
wrought in him, which I knew not, he thus showed me. He asked to see
what I had read: I showed him; and he looked even further than I had
read, and I knew not what followed. This followed, him that is weak in
the faith, receive; which he applied to himself, and disclosed to me.
And by this admonition was he strengthened; and by a good resolution and
purpose, and most corresponding to his character, wherein he did always
very far differ from me, for the better, without any turbulent delay he
joined me. Thence we go in to my mother; we tell her; she rejoiceth: we
relate in order how it took place; she leaps for joy, and triumpheth,
and blesseth Thee, Who are able to do above that which we ask or think;
for she perceived that Thou hadst given her more for me, than she
was wont to beg by her pitiful and most sorrowful groanings. For thou
convertedst me unto Thyself, so that I sought neither wife, nor any hope
of this world, standing in that rule of faith, where Thou hadst showed
me unto her in a vision, so many years before. And Thou didst convert
her mourning into joy, much more plentiful than she had desired, and
in a much more precious and purer way than she erst required, by having
grandchildren of my body.




\chapter{BOOK IX}


\start{O}{ Lord}, I am Thy servant; I am Thy servant, and the son of Thy handmaid:
Thou hast broken my bonds in sunder. I will offer to Thee the sacrifice
of praise. Let my heart and my tongue praise Thee; yea, let all my bones
say, O Lord, who is like unto Thee? Let them say, and answer Thou me,
and say unto my soul, I am thy salvation. Who am I, and what am I? What
evil have not been either my deeds, or if not my deeds, my words, or if
not my words, my will? But Thou, O Lord, are good and merciful, and Thy
right hand had respect unto the depth of my death, and from the bottom
of my heart emptied that abyss of corruption. And this Thy whole gift
was, to nill what I willed, and to will what Thou willedst. But where
through all those years, and out of what low and deep recess was my
free-will called forth in a moment, whereby to submit my neck to Thy
easy yoke, and my shoulders unto Thy light burden, O Christ Jesus, my
Helper and my Redeemer? How sweet did it at once become to me, to want
the sweetnesses of those toys! and what I feared to be parted from, was
now a joy to part with. For Thou didst cast them forth from me, Thou
true and highest sweetness. Thou castest them forth, and for them
enteredst in Thyself, sweeter than all pleasure, though not to flesh and
blood; brighter than all light, but more hidden than all depths, higher
than all honour, but not to the high in their own conceits. Now was my
soul free from the biting cares of canvassing and getting, and weltering
in filth, and scratching off the itch of lust. And my infant tongue
spake freely to Thee, my brightness, and my riches, and my health, the
Lord my God.

And I resolved in Thy sight, not tumultuously to tear, but gently to
withdraw, the service of my tongue from the marts of lip-labour: that
the young, no students in Thy law, nor in Thy peace, but in lying
dotages and law-skirmishes, should no longer buy at my mouth arms for
their madness. And very seasonably, it now wanted but very few days unto
the Vacation of the Vintage, and I resolved to endure them, then in a
regular way to take my leave, and having been purchased by Thee, no
more to return for sale. Our purpose then was known to Thee; but to men,
other than our own friends, was it not known. For we had agreed among
ourselves not to let it out abroad to any: although to us, now ascending
from the valley of tears, and singing that song of degrees, Thou hadst
given sharp arrows, and destroying coals against the subtle tongue,
which as though advising for us, would thwart, and would out of love
devour us, as it doth its meat.

Thou hadst pierced our hearts with Thy charity, and we carried Thy words
as it were fixed in our entrails: and the examples of Thy servants,
whom for black Thou hadst made bright, and for dead, alive, being piled
together in the receptacle of our thoughts, kindled and burned up that
our heavy torpor, that we should not sink down to the abyss; and they
fired us so vehemently, that all the blasts of subtle tongues from
gainsayers might only inflame us the more fiercely, not extinguish
us. Nevertheless, because for Thy Name's sake which Thou hast hallowed
throughout the earth, this our vow and purpose might also find some to
commend it, it seemed like ostentation not to wait for the vacation now
so near, but to quit beforehand a public profession, which was before
the eyes of all; so that all looking on this act of mine, and observing
how near was the time of vintage which I wished to anticipate, would
talk much of me, as if I had desired to appear some great one. And what
end had it served me, that people should repute and dispute upon my
purpose, and that our good should be evil spoken of.

Moreover, it had at first troubled me that in this very summer my lungs
began to give way, amid too great literary labour, and to breathe deeply
with difficulty, and by the pain in my chest to show that they were
injured, and to refuse any full or lengthened speaking; this had
troubled me, for it almost constrained me of necessity to lay down that
burden of teaching, or, if I could be cured and recover, at least to
intermit it. But when the full wish for leisure, that I might see
how that Thou art the Lord, arose, and was fixed, in me; my God, Thou
knowest, I began even to rejoice that I had this secondary, and that
no feigned, excuse, which might something moderate the offence taken by
those who, for their sons' sake, wished me never to have the freedom of
Thy sons. Full then of such joy, I endured till that interval of time
were run; it may have been some twenty days, yet they were endured
manfully; endured, for the covetousness which aforetime bore a part of
this heavy business, had left me, and I remained alone, and had been
overwhelmed, had not patience taken its place. Perchance, some of Thy
servants, my brethren, may say that I sinned in this, that with a heart
fully set on Thy service, I suffered myself to sit even one hour in the
chair of lies. Nor would I be contentious. But hast not Thou, O most
merciful Lord, pardoned and remitted this sin also, with my other most
horrible and deadly sins, in the holy water?

Verecundus was worn down with care about this our blessedness, for that
being held back by bonds, whereby he was most straitly bound, he saw
that he should be severed from us. For himself was not yet a Christian,
his wife one of the faithful; and yet hereby, more rigidly than by any
other chain, was he let and hindered from the journey which we had now
essayed. For he would not, he said, be a Christian on any other terms
than on those he could not. However, he offered us courteously to remain
at his country-house so long as we should stay there. Thou, O Lord,
shalt reward him in the resurrection of the just, seeing Thou hast
already given him the lot of the righteous. For although, in our
absence, being now at Rome, he was seized with bodily sickness, and
therein being made a Christian, and one of the faithful, he departed
this life; yet hadst Thou mercy not on him only, but on us also: lest
remembering the exceeding kindness of our friend towards us, yet unable
to number him among Thy flock, we should be agonised with intolerable
sorrow. Thanks unto Thee, our God, we are Thine: Thy suggestions
and consolations tell us, Faithful in promises, Thou now requitest
Verecundus for his country-house of Cassiacum, where from the fever
of the world we reposed in Thee, with the eternal freshness of Thy
Paradise: for that Thou hast forgiven him his sins upon earth, in that
rich mountain, that mountain which yieldeth milk, Thine own mountain.

He then had at that time sorrow, but Nebridius joy. For although he
also, not being yet a Christian, had fallen into the pit of that most
pernicious error, believing the flesh of Thy Son to be a phantom: yet
emerging thence, he believed as we did; not as yet endued with any
Sacraments of Thy Church, but a most ardent searcher out of truth. Whom,
not long after our conversion and regeneration by Thy Baptism, being
also a faithful member of the Church Catholic, and serving Thee in
perfect chastity and continence amongst his people in Africa, his whole
house having through him first been made Christian, didst Thou release
from the flesh; and now he lives in Abraham's bosom. Whatever that be,
which is signified by that bosom, there lives my Nebridius, my sweet
friend, and Thy child, O Lord, adopted of a freed man: there he liveth.
For what other place is there for such a soul? There he liveth, whereof
he asked much of me, a poor inexperienced man. Now lays he not his ear
to my mouth, but his spiritual mouth unto Thy fountain, and drinketh as
much as he can receive, wisdom in proportion to his thirst, endlessly
happy. Nor do I think that he is so inebriated therewith, as to forget
me; seeing Thou, Lord, Whom he drinketh, art mindful of us. So were
we then, comforting Verecundus, who sorrowed, as far as friendship
permitted, that our conversion was of such sort; and exhorting him to
become faithful, according to his measure, namely, of a married estate;
and awaiting Nebridius to follow us, which, being so near, he was all
but doing: and so, lo! those days rolled by at length; for long and many
they seemed, for the love I bare to the easeful liberty, that I might
sing to Thee, from my inmost marrow, My heart hath said unto Thee, I
have sought Thy face: Thy face, Lord, will I seek.

Now was the day come wherein I was in deed to be freed of my Rhetoric
Professorship, whereof in thought I was already freed. And it was done.
Thou didst rescue my tongue, whence Thou hadst before rescued my heart.
And I blessed Thee, rejoicing; retiring with all mine to the villa. What
I there did in writing, which was now enlisted in Thy service, though
still, in this breathing-time as it were, panting from the school of
pride, my books may witness, as well what I debated with others, as what
with myself alone, before Thee: what with Nebridius, who was absent, my
Epistles bear witness. And when shall I have time to rehearse all Thy
great benefits towards us at that time, especially when hasting on to
yet greater mercies? For my remembrance recalls me, and pleasant is it
to me, O Lord, to confess to Thee, by what inward goads Thou tamedst me;
and how Thou hast evened me, lowering the mountains and hills of my high
imaginations, straightening my crookedness, and smoothing my rough ways;
and how Thou also subduedst the brother of my heart, Alypius, unto the
name of Thy Only Begotten, our Lord and Saviour Jesus Christ, which
he would not at first vouchsafe to have inserted in our writings. For
rather would he have them savour of the lofty cedars of the Schools,
which the Lord hath now broken down, than of the wholesome herbs of the
Church, the antidote against serpents.

Oh, in what accents spake I unto Thee, my God, when I read the Psalms of
David, those faithful songs, and sounds of devotion, which allow of no
swelling spirit, as yet a Catechumen, and a novice in Thy real love,
resting in that villa, with Alypius a Catechumen, my mother cleaving to
us, in female garb with masculine faith, with the tranquillity of age,
motherly love, Christian piety! Oh, what accents did I utter unto Thee
in those Psalms, and how was I by them kindled towards Thee, and on
fire to rehearse them, if possible, through the whole world, against the
pride of mankind! And yet they are sung through the whole world, nor can
any hide himself from Thy heat. With what vehement and bitter sorrow was
I angered at the Manichees! and again I pitied them, for they knew not
those Sacraments, those medicines, and were mad against the antidote
which might have recovered them of their madness. How I would they
had then been somewhere near me, and without my knowing that they were
there, could have beheld my countenance, and heard my words, when I read
the fourth Psalm in that time of my rest, and how that Psalm wrought
upon me: When I called, the God of my righteousness heard me; in
tribulation Thou enlargedst me. Have mercy upon me, O Lord, and hear
my prayer. Would that what I uttered on these words, they could hear,
without my knowing whether they heard, lest they should think I spake
it for their sakes! Because in truth neither should I speak the same
things, nor in the same way, if I perceived that they heard and saw me;
nor if I spake them would they so receive them, as when I spake by and
for myself before Thee, out of the natural feelings of my soul.

I trembled for fear, and again kindled with hope, and with rejoicing in
Thy mercy, O Father; and all issued forth both by mine eyes and voice,
when Thy good Spirit turning unto us, said, O ye sons of men, how long
slow of heart? why do ye love vanity, and seek after leasing? For I had
loved vanity, and sought after leasing. And Thou, O Lord, hadst already
magnified Thy Holy One, raising Him from the dead, and setting Him at
Thy right hand, whence from on high He should send His promise, the
Comforter, the Spirit of truth. And He had already sent Him, but I knew
it not; He had sent Him, because He was now magnified, rising again from
the dead, and ascending into heaven. For till then, the Spirit was not
yet given, because Jesus was not yet glorified. And the prophet cries
out, How long, slow of heart? why do ye love vanity, and seek after
leasing? Know this, that the Lord hath magnified His Holy One. He cries
out, How long? He cries out, Know this: and I so long, not knowing,
loved vanity, and sought after leasing: and therefore I heard and
trembled, because it was spoken unto such as I remembered myself to
have been. For in those phantoms which I had held for truths, was
there vanity and leasing; and I spake aloud many things earnestly and
forcibly, in the bitterness of my remembrance. Which would they had
heard, who yet love vanity and seek after leasing! They would perchance
have been troubled, and have vomited it up; and Thou wouldest hear them
when they cried unto Thee; for by a true death in the flesh did He die
for us, who now intercedeth unto Thee for us.

I further read, Be angry, and sin not. And how was I moved, O my God,
who had now learned to be angry at myself for things past, that I might
not sin in time to come! Yea, to be justly angry; for that it was not
another nature of a people of darkness which sinned for me, as they say
who are not angry at themselves, and treasure up wrath against the day
of wrath, and of the revelation of Thy just judgment. Nor were my good
things now without, nor sought with the eyes of flesh in that earthly
sun; for they that would have joy from without soon become vain, and
waste themselves on the things seen and temporal, and in their famished
thoughts do lick their very shadows. Oh that they were wearied out with
their famine, and said, Who will show us good things? And we would say,
and they hear, The light of Thy countenance is sealed upon us. For we
are not that light which enlighteneth every man, but we are enlightened
by Thee; that having been sometimes darkness, we may be light in Thee.
Oh that they could see the eternal Internal, which having tasted, I was
grieved that I could not show It them, so long as they brought me their
heart in their eyes roving abroad from Thee, while they said, Who will
show us good things? For there, where I was angry within myself in my
chamber, where I was inwardly pricked, where I had sacrificed, slaying
my old man and commencing the purpose of a new life, putting my trust
in Thee,--there hadst Thou begun to grow sweet unto me, and hadst put
gladness in my heart. And I cried out, as I read this outwardly, finding
it inwardly. Nor would I be multiplied with worldly goods; wasting away
time, and wasted by time; whereas I had in Thy eternal Simple Essence
other corn, and wine, and oil.

And with a loud cry of my heart I cried out in the next verse, O in
peace, O for The Self-same! O what said he, I will lay me down and
sleep, for who shall hinder us, when cometh to pass that saying which is
written, Death is swallowed up in victory? And Thou surpassingly art the
Self-same, Who art not changed; and in Thee is rest which forgetteth all
toil, for there is none other with Thee, nor are we to seek those many
other things, which are not what Thou art: but Thou, Lord, alone hast
made me dwell in hope. I read, and kindled; nor found I what to do to
those deaf and dead, of whom myself had been, a pestilent person, a
bitter and a blind bawler against those writings, which are honied
with the honey of heaven, and lightsome with Thine own light: and I was
consumed with zeal at the enemies of this Scripture.

When shall I recall all which passed in those holy-days? Yet neither
have I forgotten, nor will I pass over the severity of Thy scourge, and
the wonderful swiftness of Thy mercy. Thou didst then torment me with
pain in my teeth; which when it had come to such height that I could not
speak, it came into my heart to desire all my friends present to pray
for me to Thee, the God of all manner of health. And this I wrote on
wax, and gave it them to read. Presently so soon as with humble devotion
we had bowed our knees, that pain went away. But what pain? or how went
it away? I was affrighted, O my Lord, my God; for from infancy I had
never experienced the like. And the power of Thy Nod was deeply conveyed
to me, and rejoicing in faith, I praised Thy Name. And that faith
suffered me not to be at ease about my past sins, which were not yet
forgiven me by Thy baptism.

The vintage-vacation ended, I gave notice to the Milanese to provide
their scholars with another master to sell words to them; for that I had
both made choice to serve Thee, and through my difficulty of breathing
and pain in my chest was not equal to the Professorship. And by letters
I signified to Thy Prelate, the holy man Ambrose, my former errors and
present desires, begging his advice what of Thy Scriptures I had best
read, to become readier and fitter for receiving so great grace. He
recommended Isaiah the Prophet: I believe, because he above the rest
is a more clear foreshower of the Gospel and of the calling of the
Gentiles. But I, not understanding the first lesson in him, and
imagining the whole to be like it, laid it by, to be resumed when better
practised in our Lord's own words.

Thence, when the time was come wherein I was to give in my name, we left
the country and returned to Milan. It pleased Alypius also to be with
me born again in Thee, being already clothed with the humility befitting
Thy Sacraments; and a most valiant tamer of the body, so as, with
unwonted venture, to wear the frozen ground of Italy with his bare feet.
We joined with us the boy Adeodatus, born after the flesh, of my sin.
Excellently hadst Thou made him. He was not quite fifteen, and in wit
surpassed many grave and learned men. I confess unto Thee Thy gifts,
O Lord my God, Creator of all, and abundantly able to reform our
deformities: for I had no part in that boy, but the sin. For that we
brought him up in Thy discipline, it was Thou, none else, had inspired
us with it. I confess unto Thee Thy gifts. There is a book of ours
entitled The Master; it is a dialogue between him and me. Thou knowest
that all there ascribed to the person conversing with me were his ideas,
in his sixteenth year. Much besides, and yet more admirable, I found
in him. That talent struck awe into me. And who but Thou could be the
workmaster of such wonders? Soon didst Thou take his life from the
earth: and I now remember him without anxiety, fearing nothing for
his childhood or youth, or his whole self. Him we joined with us, our
contemporary in grace, to be brought up in Thy discipline: and we were
baptised, and anxiety for our past life vanished from us. Nor was I
sated in those days with the wondrous sweetness of considering the depth
of Thy counsels concerning the salvation of mankind. How did I weep,
in Thy Hymns and Canticles, touched to the quick by the voices of Thy
sweet-attuned Church! The voices flowed into mine ears, and the
Truth distilled into my heart, whence the affections of my devotion
overflowed, and tears ran down, and happy was I therein.

Not long had the Church of Milan begun to use this kind of consolation
and exhortation, the brethren zealously joining with harmony of voice
and hearts. For it was a year, or not much more, that Justina, mother
to the Emperor Valentinian, a child, persecuted Thy servant Ambrose, in
favour of her heresy, to which she was seduced by the Arians. The devout
people kept watch in the Church, ready to die with their Bishop Thy
servant. There my mother Thy handmaid, bearing a chief part of those
anxieties and watchings, lived for prayer. We, yet unwarmed by the heat
of Thy Spirit, still were stirred up by the sight of the amazed and
disquieted city. Then it was first instituted that after the manner of
the Eastern Churches, Hymns and Psalms should be sung, lest the people
should wax faint through the tediousness of sorrow: and from that day to
this the custom is retained, divers (yea, almost all) Thy congregations,
throughout other parts of the world following herein.

Then didst Thou by a vision discover to Thy forenamed Bishop where the
bodies of Gervasius and Protasius the martyrs lay hid (whom Thou hadst
in Thy secret treasury stored uncorrupted so many years), whence Thou
mightest seasonably produce them to repress the fury of a woman, but an
Empress. For when they were discovered and dug up, and with due honour
translated to the Ambrosian Basilica, not only they who were vexed with
unclean spirits (the devils confessing themselves) were cured, but a
certain man who had for many years been blind, a citizen, and well known
to the city, asking and hearing the reason of the people's confused joy,
sprang forth desiring his guide to lead him thither. Led thither, he
begged to be allowed to touch with his handkerchief the bier of Thy
saints, whose death is precious in Thy sight. Which when he had done,
and put to his eyes, they were forthwith opened. Thence did the fame
spread, thence Thy praises glowed, shone; thence the mind of that enemy,
though not turned to the soundness of believing, was yet turned back
from her fury of persecuting. Thanks to Thee, O my God. Whence and
whither hast Thou thus led my remembrance, that I should confess these
things also unto Thee? which great though they be, I had passed by in
forgetfulness. And yet then, when the odour of Thy ointments was so
fragrant, did we not run after Thee. Therefore did I more weep among
the singing of Thy Hymns, formerly sighing after Thee, and at length
breathing in Thee, as far as the breath may enter into this our house of
grass.

Thou that makest men to dwell of one mind in one house, didst join with
us Euodius also, a young man of our own city. Who being an officer of
Court, was before us converted to Thee and baptised: and quitting his
secular warfare, girded himself to Thine. We were together, about to
dwell together in our devout purpose. We sought where we might serve
Thee most usefully, and were together returning to Africa: whitherward
being as far as Ostia, my mother departed this life. Much I omit, as
hastening much. Receive my confessions and thanksgivings, O my God, for
innumerable things whereof I am silent. But I will not omit whatsoever
my soul would bring forth concerning that Thy handmaid, who brought me
forth, both in the flesh, that I might be born to this temporal light,
and in heart, that I might be born to Light eternal. Not her gifts, but
Thine in her, would I speak of; for neither did she make nor educate
herself. Thou createdst her; nor did her father and mother know what a
one should come from them. And the sceptre of Thy Christ, the discipline
of Thine only Son, in a Christian house, a good member of Thy Church,
educated her in Thy fear. Yet for her good discipline was she wont
to commend not so much her mother's diligence, as that of a certain
decrepit maid-servant, who had carried her father when a child, as
little ones used to be carried at the backs of elder girls. For which
reason, and for her great age, and excellent conversation, was she,
in that Christian family, well respected by its heads. Whence also the
charge of her master's daughters was entrusted to her, to which she gave
diligent heed, restraining them earnestly, when necessary, with a holy
severity, and teaching them with a grave discretion. For, except at
those hours wherein they were most temporately fed at their parents'
table, she would not suffer them, though parched with thirst, to drink
even water; preventing an evil custom, and adding this wholesome advice:
"Ye drink water now, because you have not wine in your power; but when
you come to be married, and be made mistresses of cellars and cupboards,
you will scorn water, but the custom of drinking will abide." By this
method of instruction, and the authority she had, she refrained the
greediness of childhood, and moulded their very thirst to such an
excellent moderation that what they should not, that they would not.

And yet (as Thy handmaid told me her son) there had crept upon her
a love of wine. For when (as the manner was) she, as though a sober
maiden, was bidden by her parents to draw wine out of the hogshed,
holding the vessel under the opening, before she poured the wine into
the flagon, she sipped a little with the tip of her lips; for more her
instinctive feelings refused. For this she did, not out of any desire
of drink, but out of the exuberance of youth, whereby it boils over in
mirthful freaks, which in youthful spirits are wont to be kept under by
the gravity of their elders. And thus by adding to that little, daily
littles (for whoso despiseth little things shall fall by little and
little), she had fallen into such a habit as greedily to drink off her
little cup brim-full almost of wine. Where was then that discreet old
woman, and that her earnest countermanding? Would aught avail against
a secret disease, if Thy healing hand, O Lord, watched not over us?
Father, mother, and governors absent, Thou present, who createdst, who
callest, who also by those set over us, workest something towards the
salvation of our souls, what didst Thou then, O my God? how didst Thou
cure her? how heal her? didst Thou not out of another soul bring forth a
hard and a sharp taunt, like a lancet out of Thy secret store, and with
one touch remove all that foul stuff? For a maid-servant with whom she
used to go to the cellar, falling to words (as it happens) with her
little mistress, when alone with her, taunted her with this fault, with
most bitter insult, calling her wine-bibber. With which taunt she, stung
to the quick, saw the foulness of her fault, and instantly condemned and
forsook it. As flattering friends pervert, so reproachful enemies mostly
correct. Yet not what by them Thou doest, but what themselves purposed,
dost Thou repay them. For she in her anger sought to vex her young
mistress, not to amend her; and did it in private, either for that the
time and place of the quarrel so found them; or lest herself also should
have anger, for discovering it thus late. But Thou, Lord, Governor
of all in heaven and earth, who turnest to Thy purposes the deepest
currents, and the ruled turbulence of the tide of times, didst by the
very unhealthiness of one soul heal another; lest any, when he observes
this, should ascribe it to his own power, even when another, whom he
wished to be reformed, is reformed through words of his.

Brought up thus modestly and soberly, and made subject rather by Thee
to her parents, than by her parents to Thee, so soon as she was of
marriageable age, being bestowed upon a husband, she served him as her
lord; and did her diligence to win him unto Thee, preaching Thee unto
him by her conversation; by which Thou ornamentedst her, making her
reverently amiable, and admirable unto her husband. And she so endured
the wronging of her bed as never to have any quarrel with her husband
thereon. For she looked for Thy mercy upon him, that believing in Thee,
he might be made chaste. But besides this, he was fervid, as in his
affections, so in anger: but she had learnt not to resist an angry
husband, not in deed only, but not even in word. Only when he was
smoothed and tranquil, and in a temper to receive it, she would give an
account of her actions, if haply he had overhastily taken offence. In
a word, while many matrons, who had milder husbands, yet bore even in
their faces marks of shame, would in familiar talk blame their husbands'
lives, she would blame their tongues, giving them, as in jest, earnest
advice: "That from the time they heard the marriage writings read to
them, they should account them as indentures, whereby they were
made servants; and so, remembering their condition, ought not to set
themselves up against their lords." And when they, knowing what a
choleric husband she endured, marvelled that it had never been heard,
nor by any token perceived, that Patricius had beaten his wife, or that
there had been any domestic difference between them, even for one day,
and confidentially asking the reason, she taught them her practice above
mentioned. Those wives who observed it found the good, and returned
thanks; those who observed it not, found no relief, and suffered.

Her mother-in-law also, at first by whisperings of evil servants
incensed against her, she so overcame by observance and persevering
endurance and meekness, that she of her own accord discovered to her
son the meddling tongues whereby the domestic peace betwixt her and her
daughter-in-law had been disturbed, asking him to correct them. Then,
when in compliance with his mother, and for the well-ordering of the
family, he had with stripes corrected those discovered, at her will who
had discovered them, she promised the like reward to any who, to please
her, should speak ill of her daughter-in-law to her: and none now
venturing, they lived together with a remarkable sweetness of mutual
kindness.

This great gift also thou bestowedst, O my God, my mercy, upon that good
handmaid of Thine, in whose womb Thou createdst me, that between any
disagreeing and discordant parties where she was able, she showed
herself such a peacemaker, that hearing on both sides most bitter
things, such as swelling and indigested choler uses to break out into,
when the crudities of enmities are breathed out in sour discourses to a
present friend against an absent enemy, she never would disclose aught
of the one unto the other, but what might tend to their reconcilement.
A small good this might appear to me, did I not to my grief know
numberless persons, who through some horrible and wide-spreading
contagion of sin, not only disclose to persons mutually angered things
said in anger, but add withal things never spoken, whereas to humane
humanity, it ought to seem a light thing not to torment or increase ill
will by ill words, unless one study withal by good words to quench it.
Such was she, Thyself, her most inward Instructor, teaching her in the
school of the heart.

Finally, her own husband, towards the very end of his earthly life,
did she gain unto Thee; nor had she to complain of that in him as a
believer, which before he was a believer she had borne from him. She was
also the servant of Thy servants; whosoever of them knew her, did in her
much praise and honour and love Thee; for that through the witness of
the fruits of a holy conversation they perceived Thy presence in her
heart. For she had been the wife of one man, had requited her parents,
had governed her house piously, was well reported of for good works, had
brought up children, so often travailing in birth of them, as she saw
them swerving from Thee. Lastly, of all of us Thy servants, O Lord (whom
on occasion of Thy own gift Thou sufferest to speak), us, who before her
sleeping in Thee lived united together, having received the grace of Thy
baptism, did she so take care of, as though she had been mother of us
all; so served us, as though she had been child to us all.

The day now approaching whereon she was to depart this life (which day
Thou well knewest, we knew not), it came to pass, Thyself, as I believe,
by Thy secret ways so ordering it, that she and I stood alone, leaning
in a certain window, which looked into the garden of the house where we
now lay, at Ostia; where removed from the din of men, we were recruiting
from the fatigues of a long journey, for the voyage. We were discoursing
then together, alone, very sweetly; and forgetting those things which
are behind, and reaching forth unto those things which are before, we
were enquiring between ourselves in the presence of the Truth, which
Thou art, of what sort the eternal life of the saints was to be, which
eye hath not seen, nor ear heard, nor hath it entered into the heart of
man. But yet we gasped with the mouth of our heart, after those heavenly
streams of Thy fountain, the fountain of life, which is with Thee; that
being bedewed thence according to our capacity, we might in some sort
meditate upon so high a mystery.

And when our discourse was brought to that point, that the very highest
delight of the earthly senses, in the very purest material light,
was, in respect of the sweetness of that life, not only not worthy of
comparison, but not even of mention; we raising up ourselves with a more
glowing affection towards the "Self-same," did by degrees pass through
all things bodily, even the very heaven whence sun and moon and stars
shine upon the earth; yea, we were soaring higher yet, by inward musing,
and discourse, and admiring of Thy works; and we came to our own
minds, and went beyond them, that we might arrive at that region of
never-failing plenty, where Thou feedest Israel for ever with the food
of truth, and where life is the Wisdom by whom all these things are
made, and what have been, and what shall be, and she is not made, but
is, as she hath been, and so shall she be ever; yea rather, to "have
been," and "hereafter to be," are not in her, but only "to be," seeing
she is eternal. For to "have been," and to "be hereafter," are not
eternal. And while we were discoursing and panting after her, we
slightly touched on her with the whole effort of our heart; and we
sighed, and there we leave bound the first fruits of the Spirit; and
returned to vocal expressions of our mouth, where the word spoken
has beginning and end. And what is like unto Thy Word, our Lord, who
endureth in Himself without becoming old, and maketh all things new?

We were saying then: If to any the tumult of the flesh were hushed,
hushed the images of earth, and waters, and air, hushed also the pole of
heaven, yea the very soul be hushed to herself, and by not thinking on
self surmount self, hushed all dreams and imaginary revelations, every
tongue and every sign, and whatsoever exists only in transition, since
if any could hear, all these say, We made not ourselves, but He made us
that abideth for ever--If then having uttered this, they too should be
hushed, having roused only our ears to Him who made them, and He alone
speak, not by them but by Himself, that we may hear His Word, not
through any tongue of flesh, nor Angel's voice, nor sound of thunder,
nor in the dark riddle of a similitude, but might hear Whom in these
things we love, might hear His Very Self without these (as we two now
strained ourselves, and in swift thought touched on that Eternal Wisdom
which abideth over all);--could this be continued on, and other visions
of kind far unlike be withdrawn, and this one ravish, and absorb, and
wrap up its beholder amid these inward joys, so that life might be for
ever like that one moment of understanding which now we sighed after;
were not this, Enter into thy Master's joy? And when shall that be? When
we shall all rise again, though we shall not all be changed?

Such things was I speaking, and even if not in this very manner, and
these same words, yet, Lord, Thou knowest that in that day when we were
speaking of these things, and this world with all its delights became,
as we spake, contemptible to us, my mother said, "Son, for mine own part
I have no further delight in any thing in this life. What I do here any
longer, and to what I am here, I know not, now that my hopes in this
world are accomplished. One thing there was for which I desired to
linger for a while in this life, that I might see thee a Catholic
Christian before I died. My God hath done this for me more abundantly,
that I should now see thee withal, despising earthly happiness, become
His servant: what do I here?"

What answer I made her unto these things, I remember not. For scarce
five days after, or not much more, she fell sick of a fever; and in that
sickness one day she fell into a swoon, and was for a while withdrawn
from these visible things. We hastened round her; but she was soon
brought back to her senses; and looking on me and my brother standing by
her, said to us enquiringly, "Where was I?" And then looking fixedly on
us, with grief amazed: "Here," saith she, "shall you bury your mother."
I held my peace and refrained weeping; but my brother spake something,
wishing for her, as the happier lot, that she might die, not in a
strange place, but in her own land. Whereat, she with anxious look,
checking him with her eyes, for that he still savoured such things, and
then looking upon me: "Behold," saith she, "what he saith": and soon
after to us both, "Lay," she saith, "this body any where; let not the
care for that any way disquiet you: this only I request, that you would
remember me at the Lord's altar, wherever you be." And having delivered
this sentiment in what words she could, she held her peace, being
exercised by her growing sickness.

But I, considering Thy gifts, Thou unseen God, which Thou instillest
into the hearts of Thy faithful ones, whence wondrous fruits do spring,
did rejoice and give thanks to Thee, recalling what I before knew, how
careful and anxious she had ever been as to her place of burial, which
she had provided and prepared for herself by the body of her husband.
For because they had lived in great harmony together, she also wished
(so little can the human mind embrace things divine) to have this
addition to that happiness, and to have it remembered among men, that
after her pilgrimage beyond the seas, what was earthly of this united
pair had been permitted to be united beneath the same earth. But when
this emptiness had through the fulness of Thy goodness begun to cease in
her heart, I knew not, and rejoiced admiring what she had so disclosed
to me; though indeed in that our discourse also in the window, when she
said, "What do I here any longer?" there appeared no desire of dying
in her own country. I heard afterwards also, that when we were now
at Ostia, she with a mother's confidence, when I was absent, one day
discoursed with certain of my friends about the contempt of this life,
and the blessing of death: and when they were amazed at such courage
which Thou hadst given to a woman, and asked, "Whether she were not
afraid to leave her body so far from her own city?" she replied,
"Nothing is far to God; nor was it to be feared lest at the end of the
world, He should not recognise whence He were to raise me up." On the
ninth day then of her sickness, and the fifty-sixth year of her age, and
the three-and-thirtieth of mine, was that religious and holy soul freed
from the body.

I closed her eyes; and there flowed withal a mighty sorrow into my
heart, which was overflowing into tears; mine eyes at the same time, by
the violent command of my mind, drank up their fountain wholly dry; and
woe was me in such a strife! But when she breathed her last, the boy
Adeodatus burst out into a loud lament; then, checked by us all, held
his peace. In like manner also a childish feeling in me, which was,
through my heart's youthful voice, finding its vent in weeping, was
checked and silenced. For we thought it not fitting to solemnise that
funeral with tearful lament, and groanings; for thereby do they for
the most part express grief for the departed, as though unhappy, or
altogether dead; whereas she was neither unhappy in her death, nor
altogether dead. Of this we were assured on good grounds, the testimony
of her good conversation and her faith unfeigned.

What then was it which did grievously pain me within, but a fresh wound
wrought through the sudden wrench of that most sweet and dear custom of
living together? I joyed indeed in her testimony, when, in that her last
sickness, mingling her endearments with my acts of duty, she called me
"dutiful," and mentioned, with great affection of love, that she never
had heard any harsh or reproachful sound uttered by my mouth against
her. But yet, O my God, Who madest us, what comparison is there betwixt
that honour that I paid to her, and her slavery for me? Being then
forsaken of so great comfort in her, my soul was wounded, and that life
rent asunder as it were, which, of hers and mine together, had been made
but one.

The boy then being stilled from weeping, Euodius took up the Psalter,
and began to sing, our whole house answering him, the Psalm, I will sing
of mercy and judgments to Thee, O Lord. But hearing what we were doing,
many brethren and religious women came together; and whilst they (whose
office it was) made ready for the burial, as the manner is, I, in a part
of the house, where I might properly, together with those who thought
not fit to leave me, discoursed upon something fitting the time; and by
this balm of truth assuaged that torment, known to Thee, they unknowing
and listening intently, and conceiving me to be without all sense of
sorrow. But in Thy ears, where none of them heard, I blamed the weakness
of my feelings, and refrained my flood of grief, which gave way a little
unto me; but again came, as with a tide, yet not so as to burst out into
tears, nor to change of countenance; still I knew what I was keeping
down in my heart. And being very much displeased that these human things
had such power over me, which in the due order and appointment of our
natural condition must needs come to pass, with a new grief I grieved
for my grief, and was thus worn by a double sorrow.

And behold, the corpse was carried to the burial; we went and returned
without tears. For neither in those prayers which we poured forth unto
Thee, when the Sacrifice of our ransom was offered for her, when now the
corpse was by the grave's side, as the manner there is, previous to its
being laid therein, did I weep even during those prayers; yet was I the
whole day in secret heavily sad, and with troubled mind prayed Thee, as
I could, to heal my sorrow, yet Thou didst not; impressing, I believe,
upon my memory by this one instance, how strong is the bond of all
habit, even upon a soul, which now feeds upon no deceiving Word. It
seemed also good to me to go and bathe, having heard that the bath had
its name (balneum) from the Greek Balaneion for that it drives sadness
from the mind. And this also I confess unto Thy mercy, Father of the
fatherless, that I bathed, and was the same as before I bathed. For the
bitterness of sorrow could not exude out of my heart. Then I slept, and
woke up again, and found my grief not a little softened; and as I was
alone in my bed, I remembered those true verses of Thy Ambrose. For Thou
art the

         "Maker of all, the Lord,
           And Ruler of the height,
         Who, robing day in light, hast poured
           Soft slumbers o'er the night,
         That to our limbs the power
           Of toil may be renew'd,
         And hearts be rais'd that sink and cower,
           And sorrows be subdu'd."

And then by little and little I recovered my former thoughts of Thy
handmaid, her holy conversation towards Thee, her holy tenderness and
observance towards us, whereof I was suddenly deprived: and I was minded
to weep in Thy sight, for her and for myself, in her behalf and in my
own. And I gave way to the tears which I before restrained, to overflow
as much as they desired; reposing my heart upon them; and it found rest
in them, for it was in Thy ears, not in those of man, who would have
scornfully interpreted my weeping. And now, Lord, in writing I confess
it unto Thee. Read it, who will, and interpret it, how he will: and if
he finds sin therein, that I wept my mother for a small portion of an
hour (the mother who for the time was dead to mine eyes, who had for
many years wept for me that I might live in Thine eyes), let him not
deride me; but rather, if he be one of large charity, let him weep
himself for my sins unto Thee, the Father of all the brethren of Thy
Christ.

But now, with a heart cured of that wound, wherein it might seem
blameworthy for an earthly feeling, I pour out unto Thee, our God, in
behalf of that Thy handmaid, a far different kind of tears, flowing from
a spirit shaken by the thoughts of the dangers of every soul that dieth
in Adam. And although she having been quickened in Christ, even before
her release from the flesh, had lived to the praise of Thy name for
her faith and conversation; yet dare I not say that from what time Thou
regeneratedst her by baptism, no word issued from her mouth against Thy
Commandment. Thy Son, the Truth, hath said, Whosoever shall say unto
his brother, Thou fool, shall be in danger of hell fire. And woe be even
unto the commendable life of men, if, laying aside mercy, Thou shouldest
examine it. But because Thou art not extreme in enquiring after sins, we
confidently hope to find some place with Thee. But whosoever reckons up
his real merits to Thee, what reckons he up to Thee but Thine own gifts?
O that men would know themselves to be men; and that he that glorieth
would glory in the Lord.

I therefore, O my Praise and my Life, God of my heart, laying aside for
a while her good deeds, for which I give thanks to Thee with joy, do now
beseech Thee for the sins of my mother. Hearken unto me, I entreat Thee,
by the Medicine of our wounds, Who hung upon the tree, and now sitting
at Thy right hand maketh intercession to Thee for us. I know that she
dealt mercifully, and from her heart forgave her debtors their debts; do
Thou also forgive her debts, whatever she may have contracted in so
many years, since the water of salvation. Forgive her, Lord, forgive, I
beseech Thee; enter not into judgment with her. Let Thy mercy be exalted
above Thy justice, since Thy words are true, and Thou hast promised
mercy unto the merciful; which Thou gavest them to be, who wilt have
mercy on whom Thou wilt have mercy; and wilt have compassion on whom
Thou hast had compassion.

And, I believe, Thou hast already done what I ask; but accept, O Lord,
the free-will offerings of my mouth. For she, the day of her dissolution
now at hand, took no thought to have her body sumptuously wound up, or
embalmed with spices; nor desired she a choice monument, or to be buried
in her own land. These things she enjoined us not; but desired only to
have her name commemorated at Thy Altar, which she had served without
intermission of one day: whence she knew the holy Sacrifice to be
dispensed, by which the hand-writing that was against us is blotted out;
through which the enemy was triumphed over, who summing up our offences,
and seeking what to lay to our charge, found nothing in Him, in Whom we
conquer. Who shall restore to Him the innocent blood? Who repay Him
the price wherewith He bought us, and so take us from Him? Unto the
Sacrament of which our ransom, Thy handmaid bound her soul by the bond
of faith. Let none sever her from Thy protection: let neither the lion
nor the dragon interpose himself by force or fraud. For she will not
answer that she owes nothing, lest she be convicted and seized by the
crafty accuser: but she will answer that her sins are forgiven her by
Him, to Whom none can repay that price which He, Who owed nothing, paid
for us.

May she rest then in peace with the husband before and after whom she
had never any; whom she obeyed, with patience bringing forth fruit unto
Thee, that she might win him also unto Thee. And inspire, O Lord my God,
inspire Thy servants my brethren, Thy sons my masters, whom with
voice, and heart, and pen I serve, that so many as shall read these
Confessions, may at Thy Altar remember Monnica Thy handmaid, with
Patricius, her sometimes husband, by whose bodies Thou broughtest me
into this life, how I know not. May they with devout affection remember
my parents in this transitory light, my brethren under Thee our Father
in our Catholic Mother, and my fellow-citizens in that eternal Jerusalem
which Thy pilgrim people sigheth after from their Exodus, even unto
their return thither. That so my mother's last request of me, may
through my confessions, more than through my prayers, be, through the
prayers of many, more abundantly fulfilled to her.




\chapter{BOOK X}


\start{L}{et} me know Thee, O Lord, who knowest me: let me know Thee, as I am
known. Power of my soul, enter into it, and fit it for Thee, that
Thou mayest have and hold it without spot or wrinkle. This is my hope,
therefore do I speak; and in this hope do I rejoice, when I rejoice
healthfully. Other things of this life are the less to be sorrowed for,
the more they are sorrowed for; and the more to be sorrowed for, the
less men sorrow for them. For behold, Thou lovest the truth, and he that
doth it, cometh to the light. This would I do in my heart before Thee in
confession: and in my writing, before many witnesses.

And from Thee, O Lord, unto whose eyes the abyss of man's conscience is
naked, what could be hidden in me though I would not confess it? For
I should hide Thee from me, not me from Thee. But now, for that my
groaning is witness, that I am displeased with myself, Thou shinest out,
and art pleasing, and beloved, and longed for; that I may be ashamed of
myself, and renounce myself, and choose Thee, and neither please Thee
nor myself, but in Thee. To Thee therefore, O Lord, am I open, whatever
I am; and with what fruit I confess unto Thee, I have said. Nor do I it
with words and sounds of the flesh, but with the words of my soul, and
the cry of the thought which Thy ear knoweth. For when I am evil, then
to confess to Thee is nothing else than to be displeased with myself;
but when holy, nothing else than not to ascribe it to myself: because
Thou, O Lord, blessest the godly, but first Thou justifieth him when
ungodly. My confession then, O my God, in Thy sight, is made silently,
and not silently. For in sound, it is silent; in affection, it cries
aloud. For neither do I utter any thing right unto men, which Thou hast
not before heard from me; nor dost Thou hear any such thing from me,
which Thou hast not first said unto me.

What then have I to do with men, that they should hear my
confessions--as if they could heal all my infirmities--a race, curious
to know the lives of others, slothful to amend their own? Why seek they
to hear from me what I am; who will not hear from Thee what themselves
are? And how know they, when from myself they hear of myself, whether
I say true; seeing no man knows what is in man, but the spirit of man
which is in him? But if they hear from Thee of themselves, they cannot
say, "The Lord lieth." For what is it to hear from Thee of themselves,
but to know themselves? and who knoweth and saith, "It is false," unless
himself lieth? But because charity believeth all things (that is, among
those whom knitting unto itself it maketh one), I also, O Lord, will
in such wise confess unto Thee, that men may hear, to whom I cannot
demonstrate whether I confess truly; yet they believe me, whose ears
charity openeth unto me.

But do Thou, my inmost Physician, make plain unto me what fruit I may
reap by doing it. For the confessions of my past sins, which Thou hast
forgiven and covered, that Thou mightest bless me in Thee, changing my
soul by Faith and Thy Sacrament, when read and heard, stir up the heart,
that it sleep not in despair and say "I cannot," but awake in the love
of Thy mercy and the sweetness of Thy grace, whereby whoso is weak, is
strong, when by it he became conscious of his own weakness. And the good
delight to hear of the past evils of such as are now freed from them,
not because they are evils, but because they have been and are not. With
what fruit then, O Lord my God, to Whom my conscience daily confesseth,
trusting more in the hope of Thy mercy than in her own innocency,
with what fruit, I pray, do I by this book confess to men also in Thy
presence what I now am, not what I have been? For that other fruit I
have seen and spoken of. But what I now am, at the very time of making
these confessions, divers desire to know, who have or have not known me,
who have heard from me or of me; but their ear is not at my heart where
I am, whatever I am. They wish then to hear me confess what I am within;
whither neither their eye, nor ear, nor understanding can reach; they
wish it, as ready to believe--but will they know? For charity, whereby
they are good, telleth them that in my confessions I lie not; and she in
them, believeth me.

But for what fruit would they hear this? Do they desire to joy with me,
when they hear how near, by Thy gift, I approach unto Thee? and to pray
for me, when they shall hear how much I am held back by my own weight?
To such will I discover myself. For it is no mean fruit, O Lord my God,
that by many thanks should be given to Thee on our behalf, and Thou be
by many entreated for us. Let the brotherly mind love in me what Thou
teachest is to be loved, and lament in me what Thou teachest is to be
lamented. Let a brotherly, not a stranger, mind, not that of the strange
children, whose mouth talketh of vanity, and their right hand is a
right hand of iniquity, but that brotherly mind which when it approveth,
rejoiceth for me, and when it disapproveth me, is sorry for me; because
whether it approveth or disapproveth, it loveth me. To such will I
discover myself: they will breathe freely at my good deeds, sigh for my
ill. My good deeds are Thine appointments, and Thy gifts; my evil ones
are my offences, and Thy judgments. Let them breathe freely at the one,
sigh at the other; and let hymns and weeping go up into Thy sight,
out of the hearts of my brethren, Thy censers. And do Thou, O Lord,
be pleased with the incense of Thy holy temple, have mercy upon me
according to Thy great mercy for Thine own name's sake; and no ways
forsaking what Thou hast begun, perfect my imperfections.

This is the fruit of my confessions of what I am, not of what I have
been, to confess this, not before Thee only, in a secret exultation with
trembling, and a secret sorrow with hope; but in the ears also of the
believing sons of men, sharers of my joy, and partners in my mortality,
my fellow-citizens, and fellow-pilgrims, who are gone before, or are to
follow on, companions of my way. These are Thy servants, my brethren,
whom Thou willest to be Thy sons; my masters, whom Thou commandest me to
serve, if I would live with Thee, of Thee. But this Thy Word were little
did it only command by speaking, and not go before in performing. This
then I do in deed and word, this I do under Thy wings; in over great
peril, were not my soul subdued unto Thee under Thy wings, and my
infirmity known unto Thee. I am a little one, but my Father ever liveth,
and my Guardian is sufficient for me. For He is the same who begat me,
and defends me: and Thou Thyself art all my good; Thou, Almighty,
Who are with me, yea, before I am with Thee. To such then whom Thou
commandest me to serve will I discover, not what I have been, but what I
now am and what I yet am. But neither do I judge myself. Thus therefore
I would be heard.

For Thou, Lord, dost judge me: because, although no man knoweth the
things of a man, but the spirit of a man which is in him, yet is there
something of man, which neither the spirit of man that is in him, itself
knoweth. But Thou, Lord, knowest all of him, Who hast made him. Yet I,
though in Thy sight I despise myself, and account myself dust and ashes;
yet know I something of Thee, which I know not of myself. And truly,
now we see through a glass darkly, not face to face as yet. So long
therefore as I be absent from Thee, I am more present with myself than
with Thee; and yet know I Thee that Thou art in no ways passible; but I,
what temptations I can resist, what I cannot, I know not. And there is
hope, because Thou art faithful, Who wilt not suffer us to be tempted
above that we are able; but wilt with the temptation also make a way to
escape, that we may be able to bear it. I will confess then what I
know of myself, I will confess also what I know not of myself. And that
because what I do know of myself, I know by Thy shining upon me; and
what I know not of myself, so long know I not it, until my darkness be
made as the noon-day in Thy countenance.

Not with doubting, but with assured consciousness, do I love Thee, Lord.
Thou hast stricken my heart with Thy word, and I loved Thee. Yea also
heaven, and earth, and all that therein is, behold, on every side they
bid me love Thee; nor cease to say so unto all, that they may be without
excuse. But more deeply wilt Thou have mercy on whom Thou wilt have
mercy, and wilt have compassion on whom Thou hast had compassion: else
in deaf ears do the heaven and the earth speak Thy praises. But what do
I love, when I love Thee? not beauty of bodies, nor the fair harmony
of time, nor the brightness of the light, so gladsome to our eyes, nor
sweet melodies of varied songs, nor the fragrant smell of flowers, and
ointments, and spices, not manna and honey, not limbs acceptable to
embracements of flesh. None of these I love, when I love my God; and
yet I love a kind of light, and melody, and fragrance, and meat, and
embracement when I love my God, the light, melody, fragrance, meat,
embracement of my inner man: where there shineth unto my soul what space
cannot contain, and there soundeth what time beareth not away, and there
smelleth what breathing disperseth not, and there tasteth what eating
diminisheth not, and there clingeth what satiety divorceth not. This is
it which I love when I love my God.

And what is this? I asked the earth, and it answered me, "I am not He";
and whatsoever are in it confessed the same. I asked the sea and the
deeps, and the living creeping things, and they answered, "We are not
thy God, seek above us." I asked the moving air; and the whole air with
his inhabitants answered, "Anaximenes was deceived, I am not God." I
asked the heavens, sun, moon, stars, "Nor (say they) are we the God whom
thou seekest." And I replied unto all the things which encompass the
door of my flesh: "Ye have told me of my God, that ye are not He; tell
me something of Him." And they cried out with a loud voice, "He made
us." My questioning them, was my thoughts on them: and their form of
beauty gave the answer. And I turned myself unto myself, and said to
myself, "Who art thou?" And I answered, "A man." And behold, in me there
present themselves to me soul, and body, one without, the other within.
By which of these ought I to seek my God? I had sought Him in the body
from earth to heaven, so far as I could send messengers, the beams
of mine eyes. But the better is the inner, for to it as presiding and
judging, all the bodily messengers reported the answers of heaven and
earth, and all things therein, who said, "We are not God, but He made
us." These things did my inner man know by the ministry of the outer: I
the inner knew them; I, the mind, through the senses of my body. I asked
the whole frame of the world about my God; and it answered me, "I am not
He, but He made me."

Is not this corporeal figure apparent to all whose senses are perfect?
why then speaks it not the same to all? Animals small and great see it,
but they cannot ask it: because no reason is set over their senses to
judge on what they report. But men can ask, so that the invisible things
of God are clearly seen, being understood by the things that are made;
but by love of them, they are made subject unto them: and subjects
cannot judge. Nor yet do the creatures answer such as ask, unless they
can judge; nor yet do they change their voice (i.e., their appearance),
if one man only sees, another seeing asks, so as to appear one way to
this man, another way to that, but appearing the same way to both, it is
dumb to this, speaks to that; yea rather it speaks to all; but they only
understand, who compare its voice received from without, with the truth
within. For truth saith unto me, "Neither heaven, nor earth, nor any
other body is thy God." This, their very nature saith to him that seeth
them: "They are a mass; a mass is less in a part thereof than in the
whole." Now to thee I speak, O my soul, thou art my better part: for
thou quickenest the mass of my body, giving it life, which no body can
give to a body: but thy God is even unto thee the Life of thy life.

What then do I love, when I love my God? who is He above the head of
my soul? By my very soul will I ascend to Him. I will pass beyond that
power whereby I am united to my body, and fill its whole frame with
life. Nor can I by that power find my God; for so horse and mule that
have no understanding might find Him; seeing it is the same power,
whereby even their bodies live. But another power there is, not that
only whereby I animate, but that too whereby I imbue with sense my
flesh, which the Lord hath framed for me: commanding the eye not to
hear, and the ear not to see; but the eye, that through it I should
see, and the ear, that through it I should hear; and to the other senses
severally, what is to each their own peculiar seats and offices; which,
being divers, I the one mind, do through them enact. I will pass beyond
this power of mine also; for this also have the horse, and mule, for
they also perceive through the body.

I will pass then beyond this power of my nature also, rising by degrees
unto Him Who made me. And I come to the fields and spacious palaces of
my memory, where are the treasures of innumerable images, brought into
it from things of all sorts perceived by the senses. There is stored up,
whatsoever besides we think, either by enlarging or diminishing, or
any other way varying those things which the sense hath come to; and
whatever else hath been committed and laid up, which forgetfulness hath
not yet swallowed up and buried. When I enter there, I require what I
will to be brought forth, and something instantly comes; others must be
longer sought after, which are fetched, as it were, out of some inner
receptacle; others rush out in troops, and while one thing is desired
and required, they start forth, as who should say, "Is it perchance
I?" These I drive away with the hand of my heart, from the face of my
remembrance; until what I wish for be unveiled, and appear in sight, out
of its secret place. Other things come up readily, in unbroken order, as
they are called for; those in front making way for the following; and
as they make way, they are hidden from sight, ready to come when I will.
All which takes place when I repeat a thing by heart.

There are all things preserved distinctly and under general heads, each
having entered by its own avenue: as light, and all colours and forms of
bodies by the eyes; by the ears all sorts of sounds; all smells by the
avenue of the nostrils; all tastes by the mouth; and by the sensation of
the whole body, what is hard or soft; hot or cold; or rugged; heavy or
light; either outwardly or inwardly to the body. All these doth that
great harbour of the memory receive in her numberless secret and
inexpressible windings, to be forthcoming, and brought out at need; each
entering in by his own gate, and there laid up. Nor yet do the things
themselves enter in; only the images of the things perceived are there
in readiness, for thought to recall. Which images, how they are formed,
who can tell, though it doth plainly appear by which sense each hath
been brought in and stored up? For even while I dwell in darkness and
silence, in my memory I can produce colours, if I will, and discern
betwixt black and white, and what others I will: nor yet do sounds break
in and disturb the image drawn in by my eyes, which I am reviewing,
though they also are there, lying dormant, and laid up, as it were,
apart. For these too I call for, and forthwith they appear. And though
my tongue be still, and my throat mute, so can I sing as much as I will;
nor do those images of colours, which notwithstanding be there, intrude
themselves and interrupt, when another store is called for, which
flowed in by the ears. So the other things, piled in and up by the other
senses, I recall at my pleasure. Yea, I discern the breath of lilies
from violets, though smelling nothing; and I prefer honey to sweet wine,
smooth before rugged, at the time neither tasting nor handling, but
remembering only.

These things do I within, in that vast court of my memory. For there
are present with me, heaven, earth, sea, and whatever I could think on
therein, besides what I have forgotten. There also meet I with myself,
and recall myself, and when, where, and what I have done, and under what
feelings. There be all which I remember, either on my own experience,
or other's credit. Out of the same store do I myself with the past
continually combine fresh and fresh likenesses of things which I have
experienced, or, from what I have experienced, have believed: and thence
again infer future actions, events and hopes, and all these again I
reflect on, as present. "I will do this or that," say I to myself, in
that great receptacle of my mind, stored with the images of things so
many and so great, "and this or that will follow." "O that this or that
might be!" "God avert this or that!" So speak I to myself: and when
I speak, the images of all I speak of are present, out of the same
treasury of memory; nor would I speak of any thereof, were the images
wanting.

Great is this force of memory, excessive great, O my God; a large and
boundless chamber! who ever sounded the bottom thereof? yet is this a
power of mine, and belongs unto my nature; nor do I myself comprehend
all that I am. Therefore is the mind too strait to contain itself. And
where should that be, which it containeth not of itself? Is it without
it, and not within? how then doth it not comprehend itself? A wonderful
admiration surprises me, amazement seizes me upon this. And men go
abroad to admire the heights of mountains, the mighty billows of the
sea, the broad tides of rivers, the compass of the ocean, and the
circuits of the stars, and pass themselves by; nor wonder that when I
spake of all these things, I did not see them with mine eyes, yet could
not have spoken of them, unless I then actually saw the mountains,
billows, rivers, stars which I had seen, and that ocean which I believe
to be, inwardly in my memory, and that, with the same vast spaces
between, as if I saw them abroad. Yet did not I by seeing draw them into
myself, when with mine eyes I beheld them; nor are they themselves with
me, but their images only. And I know by what sense of the body each was
impressed upon me.

Yet not these alone does the unmeasurable capacity of my memory retain.
Here also is all, learnt of the liberal sciences and as yet unforgotten;
removed as it were to some inner place, which is yet no place: nor
are they the images thereof, but the things themselves. For, what is
literature, what the art of disputing, how many kinds of questions there
be, whatsoever of these I know, in such manner exists in my memory, as
that I have not taken in the image, and left out the thing, or that it
should have sounded and passed away like a voice fixed on the ear by
that impress, whereby it might be recalled, as if it sounded, when it
no longer sounded; or as a smell while it passes and evaporates into air
affects the sense of smell, whence it conveys into the memory an image
of itself, which remembering, we renew, or as meat, which verily in
the belly hath now no taste, and yet in the memory still in a manner
tasteth; or as any thing which the body by touch perceiveth, and which
when removed from us, the memory still conceives. For those things
are not transmitted into the memory, but their images only are with
an admirable swiftness caught up, and stored as it were in wondrous
cabinets, and thence wonderfully by the act of remembering, brought
forth.

But now when I hear that there be three kinds of questions, "Whether the
thing be? what it is? of what kind it is?" I do indeed hold the images
of the sounds of which those words be composed, and that those sounds,
with a noise passed through the air, and now are not. But the things
themselves which are signified by those sounds, I never reached with any
sense of my body, nor ever discerned them otherwise than in my mind; yet
in my memory have I laid up not their images, but themselves. Which how
they entered into me, let them say if they can; for I have gone over all
the avenues of my flesh, but cannot find by which they entered. For the
eyes say, "If those images were coloured, we reported of them." The ears
say, "If they sound, we gave knowledge of them." The nostrils say, "If
they smell, they passed by us." The taste says, "Unless they have a
savour, ask me not." The touch says, "If it have not size, I handled
it not; if I handled it not, I gave no notice of it." Whence and how
entered these things into my memory? I know not how. For when I learned
them, I gave not credit to another man's mind, but recognised them in
mine; and approving them for true, I commended them to it, laying them
up as it were, whence I might bring them forth when I willed. In my
heart then they were, even before I learned them, but in my memory
they were not. Where then? or wherefore, when they were spoken, did I
acknowledge them, and said, "So is it, it is true," unless that they
were already in the memory, but so thrown back and buried as it were in
deeper recesses, that had not the suggestion of another drawn them forth
I had perchance been unable to conceive of them?

Wherefore we find, that to learn these things whereof we imbibe not the
images by our senses, but perceive within by themselves, without images,
as they are, is nothing else, but by conception, to receive, and by
marking to take heed that those things which the memory did before
contain at random and unarranged, be laid up at hand as it were in that
same memory where before they lay unknown, scattered and neglected, and
so readily occur to the mind familiarised to them. And how many things
of this kind does my memory bear which have been already found out, and
as I said, placed as it were at hand, which we are said to have learned
and come to know which were I for some short space of time to cease to
call to mind, they are again so buried, and glide back, as it were, into
the deeper recesses, that they must again, as if new, be thought out
thence, for other abode they have none: but they must be drawn together
again, that they may be known; that is to say, they must as it were be
collected together from their dispersion: whence the word "cogitation"
is derived. For cogo (collect) and cogito (re-collect) have the same
relation to each other as ago and agito, facio and factito. But the mind
hath appropriated to itself this word (cogitation), so that, not what is
"collected" any how, but what is "recollected," i.e., brought together,
in the mind, is properly said to be cogitated, or thought upon.

The memory containeth also reasons and laws innumerable of numbers and
dimensions, none of which hath any bodily sense impressed; seeing they
have neither colour, nor sound, nor taste, nor smell, nor touch. I have
heard the sound of the words whereby when discussed they are denoted:
but the sounds are other than the things. For the sounds are other in
Greek than in Latin; but the things are neither Greek, nor Latin,
nor any other language. I have seen the lines of architects, the very
finest, like a spider's thread; but those are still different, they
are not the images of those lines which the eye of flesh showed me: he
knoweth them, whosoever without any conception whatsoever of a body,
recognises them within himself. I have perceived also the numbers of the
things with which we number all the senses of my body; but those numbers
wherewith we number are different, nor are they the images of these,
and therefore they indeed are. Let him who seeth them not, deride me for
saying these things, and I will pity him, while he derides me.

All these things I remember, and how I learnt them I remember. Many
things also most falsely objected against them have I heard, and
remember; which though they be false, yet is it not false that I
remember them; and I remember also that I have discerned betwixt those
truths and these falsehoods objected to them. And I perceive that the
present discerning of these things is different from remembering that
I oftentimes discerned them, when I often thought upon them. I both
remember then to have often understood these things; and what I now
discern and understand, I lay up in my memory, that hereafter I may
remember that I understand it now. So then I remember also to have
remembered; as if hereafter I shall call to remembrance, that I have now
been able to remember these things, by the force of memory shall I call
it to remembrance.

The same memory contains also the affections of my mind, not in the same
manner that my mind itself contains them, when it feels them; but far
otherwise, according to a power of its own. For without rejoicing I
remember myself to have joyed; and without sorrow do I recollect my
past sorrow. And that I once feared, I review without fear; and without
desire call to mind a past desire. Sometimes, on the contrary, with joy
do I remember my fore-past sorrow, and with sorrow, joy. Which is not
wonderful, as to the body; for mind is one thing, body another. If
I therefore with joy remember some past pain of body, it is not so
wonderful. But now seeing this very memory itself is mind (for when we
give a thing in charge, to be kept in memory, we say, "See that you keep
it in mind"; and when we forget, we say, "It did not come to my mind,"
and, "It slipped out of my mind," calling the memory itself the mind);
this being so, how is it that when with joy I remember my past sorrow,
the mind hath joy, the memory hath sorrow; the mind upon the joyfulness
which is in it, is joyful, yet the memory upon the sadness which is in
it, is not sad? Does the memory perchance not belong to the mind? Who
will say so? The memory then is, as it were, the belly of the mind, and
joy and sadness, like sweet and bitter food; which, when committed to
the memory, are as it were passed into the belly, where they may be
stowed, but cannot taste. Ridiculous it is to imagine these to be alike;
and yet are they not utterly unlike.

But, behold, out of my memory I bring it, when I say there be four
perturbations of the mind, desire, joy, fear, sorrow; and whatsoever I
can dispute thereon, by dividing each into its subordinate species, and
by defining it, in my memory find I what to say, and thence do I bring
it: yet am I not disturbed by any of these perturbations, when by
calling them to mind, I remember them; yea, and before I recalled
and brought them back, they were there; and therefore could they, by
recollection, thence be brought. Perchance, then, as meat is by chewing
the cud brought up out of the belly, so by recollection these out of the
memory. Why then does not the disputer, thus recollecting, taste in the
mouth of his musing the sweetness of joy, or the bitterness of sorrow?
Is the comparison unlike in this, because not in all respects like? For
who would willingly speak thereof, if so oft as we name grief or fear,
we should be compelled to be sad or fearful? And yet could we not speak
of them, did we not find in our memory, not only the sounds of the names
according to the images impressed by the senses of the body, but notions
of the very things themselves which we never received by any avenue of
the body, but which the mind itself perceiving by the experience of its
own passions, committed to the memory, or the memory of itself retained,
without being committed unto it.

But whether by images or no, who can readily say? Thus, I name a stone,
I name the sun, the things themselves not being present to my senses,
but their images to my memory. I name a bodily pain, yet it is not
present with me, when nothing aches: yet unless its image were present
to my memory, I should not know what to say thereof, nor in discoursing
discern pain from pleasure. I name bodily health; being sound in body,
the thing itself is present with me; yet, unless its image also were
present in my memory, I could by no means recall what the sound of
this name should signify. Nor would the sick, when health were named,
recognise what were spoken, unless the same image were by the force of
memory retained, although the thing itself were absent from the body. I
name numbers whereby we number; and not their images, but themselves
are present in my memory. I name the image of the sun, and that image is
present in my memory. For I recall not the image of its image, but the
image itself is present to me, calling it to mind. I name memory, and
I recognise what I name. And where do I recognise it, but in the memory
itself? Is it also present to itself by its image, and not by itself?

What, when I name forgetfulness, and withal recognise what I name?
whence should I recognise it, did I not remember it? I speak not of the
sound of the name, but of the thing which it signifies: which if I had
forgotten, I could not recognise what that sound signifies. When then I
remember memory, memory itself is, through itself, present with itself:
but when I remember forgetfulness, there are present both memory
and forgetfulness; memory whereby I remember, forgetfulness which I
remember. But what is forgetfulness, but the privation of memory? How
then is it present that I remember it, since when present I cannot
remember? But if what we remember we hold it in memory, yet, unless we
did remember forgetfulness, we could never at the hearing of the name
recognise the thing thereby signified, then forgetfulness is retained by
memory. Present then it is, that we forget not, and being so, we forget.
It is to be understood from this that forgetfulness when we remember it,
is not present to the memory by itself but by its image: because if
it were present by itself, it would not cause us to remember, but to
forget. Who now shall search out this? who shall comprehend how it is?

Lord, I, truly, toil therein, yea and toil in myself; I am become a
heavy soil requiring over much sweat of the brow. For we are not now
searching out the regions of heaven, or measuring the distances of the
stars, or enquiring the balancings of the earth. It is I myself who
remember, I the mind. It is not so wonderful, if what I myself am not,
be far from me. But what is nearer to me than myself? And lo, the force
of mine own memory is not understood by me; though I cannot so much as
name myself without it. For what shall I say, when it is clear to
me that I remember forgetfulness? Shall I say that that is not in my
memory, which I remember? or shall I say that forgetfulness is for this
purpose in my memory, that I might not forget? Both were most absurd.
What third way is there? How can I say that the image of forgetfulness
is retained by my memory, not forgetfulness itself, when I remember it?
How could I say this either, seeing that when the image of any thing is
impressed on the memory, the thing itself must needs be first present,
whence that image may be impressed? For thus do I remember Carthage,
thus all places where I have been, thus men's faces whom I have seen,
and things reported by the other senses; thus the health or sickness of
the body. For when these things were present, my memory received from
them images, which being present with me, I might look on and bring
back in my mind, when I remembered them in their absence. If then this
forgetfulness is retained in the memory through its image, not through
itself, then plainly itself was once present, that its image might
be taken. But when it was present, how did it write its image in the
memory, seeing that forgetfulness by its presence effaces even what it
finds already noted? And yet, in whatever way, although that way be
past conceiving and explaining, yet certain am I that I remember
forgetfulness itself also, whereby what we remember is effaced.

Great is the power of memory, a fearful thing, O my God, a deep and
boundless manifoldness; and this thing is the mind, and this am I
myself. What am I then, O my God? What nature am I? A life various and
manifold, and exceeding immense. Behold in the plains, and caves, and
caverns of my memory, innumerable and innumerably full of innumerable
kinds of things, either through images, as all bodies; or by actual
presence, as the arts; or by certain notions or impressions, as the
affections of the mind, which, even when the mind doth not feel, the
memory retaineth, while yet whatsoever is in the memory is also in the
mind--over all these do I run, I fly; I dive on this side and on that,
as far as I can, and there is no end. So great is the force of memory,
so great the force of life, even in the mortal life of man. What shall I
do then, O Thou my true life, my God? I will pass even beyond this power
of mine which is called memory: yea, I will pass beyond it, that I may
approach unto Thee, O sweet Light. What sayest Thou to me? See, I am
mounting up through my mind towards Thee who abidest above me. Yea, I
now will pass beyond this power of mine which is called memory, desirous
to arrive at Thee, whence Thou mayest be arrived at; and to cleave unto
Thee, whence one may cleave unto Thee. For even beasts and birds have
memory; else could they not return to their dens and nests, nor many
other things they are used unto: nor indeed could they be used to any
thing, but by memory. I will pass then beyond memory also, that I may
arrive at Him who hath separated me from the four-footed beasts and made
me wiser than the fowls of the air, I will pass beyond memory also,
and where shall I find Thee, Thou truly good and certain sweetness? And
where shall I find Thee? If I find Thee without my memory, then do I not
retain Thee in my memory. And how shall I find Thee, if I remember Thee
not?

For the woman that had lost her groat, and sought it with a light;
unless she had remembered it, she had never found it. For when it was
found, whence should she know whether it were the same, unless she
remembered it? I remember to have sought and found many a thing; and
this I thereby know, that when I was seeking any of them, and was asked,
"Is this it?" "Is that it?" so long said I "No," until that were offered
me which I sought. Which had I not remembered (whatever it were) though
it were offered me, yet should I not find it, because I could not
recognise it. And so it ever is, when we seek and find any lost thing.
Notwithstanding, when any thing is by chance lost from the sight, not
from the memory (as any visible body), yet its image is still retained
within, and it is sought until it be restored to sight; and when it is
found, it is recognised by the image which is within: nor do we say
that we have found what was lost, unless we recognise it; nor can we
recognise it, unless we remember it. But this was lost to the eyes, but
retained in the memory.

But what when the memory itself loses any thing, as falls out when we
forget and seek that we may recollect? Where in the end do we search,
but in the memory itself? and there, if one thing be perchance offered
instead of another, we reject it, until what we seek meets us; and when
it doth, we say, "This is it"; which we should not unless we recognised
it, nor recognise it unless we remembered it. Certainly then we had
forgotten it. Or, had not the whole escaped us, but by the part whereof
we had hold, was the lost part sought for; in that the memory felt that
it did not carry on together all which it was wont, and maimed, as it
were, by the curtailment of its ancient habit, demanded the restoration
of what it missed? For instance, if we see or think of some one known
to us, and having forgotten his name, try to recover it; whatever else
occurs, connects itself not therewith; because it was not wont to be
thought upon together with him, and therefore is rejected, until that
present itself, whereon the knowledge reposes equably as its wonted
object. And whence does that present itself, but out of the memory
itself? for even when we recognise it, on being reminded by another, it
is thence it comes. For we do not believe it as something new, but,
upon recollection, allow what was named to be right. But were it utterly
blotted out of the mind, we should not remember it, even when reminded.
For we have not as yet utterly forgotten that, which we remember
ourselves to have forgotten. What then we have utterly forgotten, though
lost, we cannot even seek after.

How then do I seek Thee, O Lord? For when I seek Thee, my God, I seek a
happy life. I will seek Thee, that my soul may live. For my body liveth
by my soul; and my soul by Thee. How then do I seek a happy life, seeing
I have it not, until I can say, where I ought to say it, "It is enough"?
How seek I it? By remembrance, as though I had forgotten it, remembering
that I had forgotten it? Or, desiring to learn it as a thing unknown,
either never having known, or so forgotten it, as not even to remember
that I had forgotten it? is not a happy life what all will, and no one
altogether wills it not? where have they known it, that they so will it?
where seen it, that they so love it? Truly we have it, how, I know not.
Yea, there is another way, wherein when one hath it, then is he happy;
and there are, who are blessed, in hope. These have it in a lower kind,
than they who have it in very deed; yet are they better off than such as
are happy neither in deed nor in hope. Yet even these, had they it not
in some sort, would not so will to be happy, which that they do will, is
most certain. They have known it then, I know not how, and so have it by
some sort of knowledge, what, I know not, and am perplexed whether it be
in the memory, which if it be, then we have been happy once; whether all
severally, or in that man who first sinned, in whom also we all died,
and from whom we are all born with misery, I now enquire not; but only,
whether the happy life be in the memory? For neither should we love it,
did we not know it. We hear the name, and we all confess that we desire
the thing; for we are not delighted with the mere sound. For when
a Greek hears it in Latin, he is not delighted, not knowing what is
spoken; but we Latins are delighted, as would he too, if he heard it in
Greek; because the thing itself is neither Greek nor Latin, which Greeks
and Latins, and men of all other tongues, long for so earnestly. Known
therefore it is to all, for they with one voice be asked, "would they
be happy?" they would answer without doubt, "they would." And this could
not be, unless the thing itself whereof it is the name were retained in
their memory.

But is it so, as one remembers Carthage who hath seen it? No. For a
happy life is not seen with the eye, because it is not a body. As we
remember numbers then? No. For these, he that hath in his knowledge,
seeks not further to attain unto; but a happy life we have in our
knowledge, and therefore love it, and yet still desire to attain it,
that we may be happy. As we remember eloquence then? No. For although
upon hearing this name also, some call to mind the thing, who still are
not yet eloquent, and many who desire to be so, whence it appears that
it is in their knowledge; yet these have by their bodily senses observed
others to be eloquent, and been delighted, and desire to be the like
(though indeed they would not be delighted but for some inward knowledge
thereof, nor wish to be the like, unless they were thus delighted);
whereas a happy life, we do by no bodily sense experience in others. As
then we remember joy? Perchance; for my joy I remember, even when sad,
as a happy life, when unhappy; nor did I ever with bodily sense see,
hear, smell, taste, or touch my joy; but I experienced it in my mind,
when I rejoiced; and the knowledge of it clave to my memory, so that I
can recall it with disgust sometimes, at others with longing, according
to the nature of the things, wherein I remember myself to have joyed.
For even from foul things have I been immersed in a sort of joy; which
now recalling, I detest and execrate; otherwhiles in good and honest
things, which I recall with longing, although perchance no longer
present; and therefore with sadness I recall former joy.

Where then and when did I experience my happy life, that I should
remember, and love, and long for it? Nor is it I alone, or some few
besides, but we all would fain be happy; which, unless by some certain
knowledge we knew, we should not with so certain a will desire. But how
is this, that if two men be asked whether they would go to the wars,
one, perchance, would answer that he would, the other, that he would
not; but if they were asked whether they would be happy, both would
instantly without any doubting say they would; and for no other reason
would the one go to the wars, and the other not, but to be happy. Is it
perchance that as one looks for his joy in this thing, another in that,
all agree in their desire of being happy, as they would (if they were
asked) that they wished to have joy, and this joy they call a happy
life? Although then one obtains this joy by one means, another by
another, all have one end, which they strive to attain, namely, joy.
Which being a thing which all must say they have experienced, it is
therefore found in the memory, and recognised whenever the name of a
happy life is mentioned.

Far be it, Lord, far be it from the heart of Thy servant who here
confesseth unto Thee, far be it, that, be the joy what it may, I should
therefore think myself happy. For there is a joy which is not given to
the ungodly, but to those who love Thee for Thine own sake, whose joy
Thou Thyself art. And this is the happy life, to rejoice to Thee, of
Thee, for Thee; this is it, and there is no other. For they who think
there is another, pursue some other and not the true joy. Yet is not
their will turned away from some semblance of joy.

It is not certain then that all wish to be happy, inasmuch as they who
wish not to joy in Thee, which is the only happy life, do not truly
desire the happy life. Or do all men desire this, but because the flesh
lusteth against the Spirit, and the Spirit against the flesh, that they
cannot do what they would, they fall upon that which they can, and are
content therewith; because, what they are not able to do, they do not
will so strongly as would suffice to make them able? For I ask any
one, had he rather joy in truth, or in falsehood? They will as little
hesitate to say "in the truth," as to say "that they desire to be
happy," for a happy life is joy in the truth: for this is a joying in
Thee, Who art the Truth, O God my light, health of my countenance, my
God. This is the happy life which all desire; this life which alone is
happy, all desire; to joy in the truth all desire. I have met with many
that would deceive; who would be deceived, no one. Where then did they
know this happy life, save where they know the truth also? For they love
it also, since they would not be deceived. And when they love a happy
life, which is no other than joying in the truth, then also do they love
the truth; which yet they would not love, were there not some notice of
it in their memory. Why then joy they not in it? why are they not happy?
because they are more strongly taken up with other things which have
more power to make them miserable, than that which they so faintly
remember to make them happy. For there is yet a little light in men; let
them walk, let them walk, that the darkness overtake them not.

But why doth "truth generate hatred," and the man of Thine, preaching
the truth, become an enemy to them? whereas a happy life is loved, which
is nothing else but joying in the truth; unless that truth is in that
kind loved, that they who love anything else would gladly have that
which they love to be the truth: and because they would not be deceived,
would not be convinced that they are so? Therefore do they hate the
truth for that thing's sake which they loved instead of the truth. They
love truth when she enlightens, they hate her when she reproves. For
since they would not be deceived, and would deceive, they love her when
she discovers herself unto them, and hate her when she discovers them.
Whence she shall so repay them, that they who would not be made manifest
by her, she both against their will makes manifest, and herself becometh
not manifest unto them. Thus, thus, yea thus doth the mind of man, thus
blind and sick, foul and ill-favoured, wish to be hidden, but that aught
should be hidden from it, it wills not. But the contrary is requited it,
that itself should not be hidden from the Truth; but the Truth is hid
from it. Yet even thus miserable, it had rather joy in truths than in
falsehoods. Happy then will it be, when, no distraction interposing, it
shall joy in that only Truth, by Whom all things are true.

See what a space I have gone over in my memory seeking Thee, O Lord; and
I have not found Thee, without it. Nor have I found any thing concerning
Thee, but what I have kept in memory, ever since I learnt Thee. For
since I learnt Thee, I have not forgotten Thee. For where I found Truth,
there found I my God, the Truth itself; which since I learnt, I have
not forgotten. Since then I learnt Thee, Thou residest in my memory; and
there do I find Thee, when I call Thee to remembrance, and delight in
Thee. These be my holy delights, which Thou hast given me in Thy mercy,
having regard to my poverty.

But where in my memory residest Thou, O Lord, where residest Thou
there? what manner of lodging hast Thou framed for Thee? what manner of
sanctuary hast Thou builded for Thee? Thou hast given this honour to my
memory, to reside in it; but in what quarter of it Thou residest, that
am I considering. For in thinking on Thee, I passed beyond such parts of
it as the beasts also have, for I found Thee not there among the images
of corporeal things: and I came to those parts to which I committed the
affections of my mind, nor found Thee there. And I entered into the
very seat of my mind (which it hath in my memory, inasmuch as the mind
remembers itself also), neither wert Thou there: for as Thou art not
a corporeal image, nor the affection of a living being (as when we
rejoice, condole, desire, fear, remember, forget, or the like); so
neither art Thou the mind itself; because Thou art the Lord God of the
mind; and all these are changed, but Thou remainest unchangeable over
all, and yet hast vouchsafed to dwell in my memory, since I learnt Thee.
And why seek I now in what place thereof Thou dwellest, as if there
were places therein? Sure I am, that in it Thou dwellest, since I have
remembered Thee ever since I learnt Thee, and there I find Thee, when I
call Thee to remembrance.

Where then did I find Thee, that I might learn Thee? For in my memory
Thou wert not, before I learned Thee. Where then did I find Thee, that
I might learn Thee, but in Thee above me? Place there is none; we go
backward and forward, and there is no place. Every where, O Truth, dost
Thou give audience to all who ask counsel of Thee, and at once answerest
all, though on manifold matters they ask Thy counsel. Clearly dost Thou
answer, though all do not clearly hear. All consult Thee on what they
will, though they hear not always what they will. He is Thy best servant
who looks not so much to hear that from Thee which himself willeth, as
rather to will that, which from Thee he heareth.

Too late loved I Thee, O Thou Beauty of ancient days, yet ever new! too
late I loved Thee! And behold, Thou wert within, and I abroad, and there
I searched for Thee; deformed I, plunging amid those fair forms which
Thou hadst made. Thou wert with me, but I was not with Thee. Things held
me far from Thee, which, unless they were in Thee, were not at all.
Thou calledst, and shoutedst, and burstest my deafness. Thou flashedst,
shonest, and scatteredst my blindness. Thou breathedst odours, and I
drew in breath and panted for Thee. I tasted, and hunger and thirst.
Thou touchedst me, and I burned for Thy peace.

When I shall with my whole self cleave to Thee, I shall no where have
sorrow or labour; and my life shall wholly live, as wholly full of Thee.
But now since whom Thou fillest, Thou liftest up, because I am not full
of Thee I am a burden to myself. Lamentable joys strive with joyous
sorrows: and on which side is the victory, I know not. Woe is me! Lord,
have pity on me. My evil sorrows strive with my good joys; and on which
side is the victory, I know not. Woe is me! Lord, have pity on me. Woe
is me! lo! I hide not my wounds; Thou art the Physician, I the sick;
Thou merciful, I miserable. Is not the life of man upon earth all trial?
Who wishes for troubles and difficulties? Thou commandest them to be
endured, not to be loved. No man loves what he endures, though he love
to endure. For though he rejoices that he endures, he had rather there
were nothing for him to endure. In adversity I long for prosperity, in
prosperity I fear adversity. What middle place is there betwixt these
two, where the life of man is not all trial? Woe to the prosperities of
the world, once and again, through fear of adversity, and corruption of
joy! Woe to the adversities of the world, once and again, and the third
time, from the longing for prosperity, and because adversity itself is
a hard thing, and lest it shatter endurance. Is not the life of man upon
earth all trial: without any interval?

And all my hope is no where but in Thy exceeding great mercy. Give what
Thou enjoinest, and enjoin what Thou wilt. Thou enjoinest us continency;
and when I knew, saith one, that no man can be continent, unless God
give it, this also was a part of wisdom to know whose gift she is. By
continency verily are we bound up and brought back into One, whence we
were dissipated into many. For too little doth he love Thee, who loves
any thing with Thee, which he loveth not for Thee. O love, who ever
burnest and never consumest! O charity, my God, kindle me. Thou
enjoinest continency: give me what Thou enjoinest, and enjoin what Thou
wilt.

Verily Thou enjoinest me continency from the lust of the flesh, the lust
of the eyes, and the ambition of the world. Thou enjoinest continency
from concubinage; and for wedlock itself, Thou hast counselled something
better than what Thou hast permitted. And since Thou gavest it, it was
done, even before I became a dispenser of Thy Sacrament. But there yet
live in my memory (whereof I have much spoken) the images of such things
as my ill custom there fixed; which haunt me, strengthless when I am
awake: but in sleep, not only so as to give pleasure, but even to obtain
assent, and what is very like reality. Yea, so far prevails the illusion
of the image, in my soul and in my flesh, that, when asleep, false
visions persuade to that which when waking, the true cannot. Am I not
then myself, O Lord my God? And yet there is so much difference betwixt
myself and myself, within that moment wherein I pass from waking to
sleeping, or return from sleeping to waking! Where is reason then,
which, awake, resisteth such suggestions? And should the things
themselves be urged on it, it remaineth unshaken. Is it clasped up with
the eyes? is it lulled asleep with the senses of the body? And whence is
it that often even in sleep we resist, and mindful of our purpose, and
abiding most chastely in it, yield no assent to such enticements? And
yet so much difference there is, that when it happeneth otherwise, upon
waking we return to peace of conscience: and by this very difference
discover that we did not, what yet we be sorry that in some way it was
done in us.

Art Thou not mighty, God Almighty, so as to heal all the diseases of my
soul, and by Thy more abundant grace to quench even the impure motions
of my sleep! Thou wilt increase, Lord, Thy gifts more and more in me,
that my soul may follow me to Thee, disentangled from the birdlime of
concupiscence; that it rebel not against itself, and even in dreams not
only not, through images of sense, commit those debasing corruptions,
even to pollution of the flesh, but not even to consent unto them. For
that nothing of this sort should have, over the pure affections even of
a sleeper, the very least influence, not even such as a thought would
restrain,--to work this, not only during life, but even at my present
age, is not hard for the Almighty, Who art able to do above all that
we ask or think. But what I yet am in this kind of my evil, have I
confessed unto my good Lord; rejoicing with trembling, in that which
Thou hast given me, and bemoaning that wherein I am still imperfect;
hoping that Thou wilt perfect Thy mercies in me, even to perfect peace,
which my outward and inward man shall have with Thee, when death shall
be swallowed up in victory.

There is another evil of the day, which I would were sufficient for it.
For by eating and drinking we repair the evil decays of our body, until
Thou destroy both belly and meat, when Thou shalt slay my emptiness
with a wonderful fulness, and clothe this incorruptible with an eternal
incorruption. But now the necessity is sweet unto me, against which
sweetness I fight, that I be not taken captive; and carry on a daily war
by fastings; often bringing my body into subjection; and my pains are
removed by pleasure. For hunger and thirst are in a manner pains; they
burn and kill like a fever, unless the medicine of nourishments come
to our aid. Which since it is at hand through the consolations of Thy
gifts, with which land, and water, and air serve our weakness, our
calamity is termed gratification.

This hast Thou taught me, that I should set myself to take food as
physic. But while I am passing from the discomfort of emptiness to the
content of replenishing, in the very passage the snare of concupiscence
besets me. For that passing, is pleasure, nor is there any other way to
pass thither, whither we needs must pass. And health being the cause of
eating and drinking, there joineth itself as an attendant a dangerous
pleasure, which mostly endeavours to go before it, so that I may for her
sake do what I say I do, or wish to do, for health's sake. Nor have
each the same measure; for what is enough for health, is too little for
pleasure. And oft it is uncertain, whether it be the necessary care of
the body which is yet asking for sustenance, or whether a voluptuous
deceivableness of greediness is proffering its services. In this
uncertainty the unhappy soul rejoiceth, and therein prepares an excuse
to shield itself, glad that it appeareth not what sufficeth for the
moderation of health, that under the cloak of health, it may disguise
the matter of gratification. These temptations I daily endeavour
to resist, and I call on Thy right hand, and to Thee do I refer my
perplexities; because I have as yet no settled counsel herein.

I hear the voice of my God commanding, Let not your hearts be
overcharged with surfeiting and drunkenness. Drunkenness is far from
me; Thou wilt have mercy, that it come not near me. But full feeding
sometimes creepeth upon Thy servant; Thou wilt have mercy, that it may
be far from me. For no one can be continent unless Thou give it. Many
things Thou givest us, praying for them; and what good soever we have
received before we prayed, from Thee we received it; yea to the end we
might afterwards know this, did we before receive it. Drunkard was I
never, but drunkards have I known made sober by Thee. From Thee then it
was, that they who never were such, should not so be, as from Thee it
was, that they who have been, should not ever so be; and from Thee it
was, that both might know from Whom it was. I heard another voice of
Thine, Go not after thy lusts, and from thy pleasure turn away. Yea by
Thy favour have I heard that which I have much loved; neither if we eat,
shall we abound; neither if we eat not, shall we lack; which is to say,
neither shall the one make me plenteous, nor the other miserable.
I heard also another, for I have learned in whatsoever state I am,
therewith to be content; I know how to abound, and how to suffer need. I
can do all things through Christ that strengtheneth me. Behold a soldier
of the heavenly camp, not the dust which we are. But remember, Lord,
that we are dust, and that of dust Thou hast made man; and he was lost
and is found. Nor could he of himself do this, because he whom I so
loved, saying this through the in-breathing of Thy inspiration, was
of the same dust. I can do all things (saith he) through Him that
strengtheneth me. Strengthen me, that I can. Give what Thou enjoinest,
and enjoin what Thou wilt. He confesses to have received, and when he
glorieth, in the Lord he glorieth. Another have I heard begging that he
might receive. Take from me (saith he) the desires of the belly; whence
it appeareth, O my holy God, that Thou givest, when that is done which
Thou commandest to be done.

Thou hast taught me, good Father, that to the pure, all things are pure;
but that it is evil unto the man that eateth with offence; and, that
every creature of Thine is good, and nothing to be refused, which is
received with thanksgiving; and that meat commendeth us not to God; and,
that no man should judge us in meat or drink; and, that he which eateth,
let him not despise him that eateth not; and let not him that eateth
not, judge him that eateth. These things have I learned, thanks be
to Thee, praise to Thee, my God, my Master, knocking at my ears,
enlightening my heart; deliver me out of all temptation. I fear not
uncleanness of meat, but the uncleanness of lusting. I know; that Noah
was permitted to eat all kind of flesh that was good for food; that
Elijah was fed with flesh; that endued with an admirable abstinence, was
not polluted by feeding on living creatures, locusts. I know also that
Esau was deceived by lusting for lentiles; and that David blamed himself
for desiring a draught of water; and that our King was tempted, not
concerning flesh, but bread. And therefore the people in the wilderness
also deserved to be reproved, not for desiring flesh, but because, in
the desire of food, they murmured against the Lord.

Placed then amid these temptations, I strive daily against concupiscence
in eating and drinking. For it is not of such nature that I can settle
on cutting it off once for all, and never touching it afterward, as
I could of concubinage. The bridle of the throat then is to be held
attempered between slackness and stiffness. And who is he, O Lord, who
is not some whit transported beyond the limits of necessity? whoever he
is, he is a great one; let him make Thy Name great. But I am not such,
for I am a sinful man. Yet do I too magnify Thy name; and He maketh
intercession to Thee for my sins who hath overcome the world; numbering
me among the weak members of His body; because Thine eyes have seen that
of Him which is imperfect, and in Thy book shall all be written.

With the allurements of smells, I am not much concerned. When absent, I
do not miss them; when present, I do not refuse them; yet ever ready to
be without them. So I seem to myself; perchance I am deceived. For that
also is a mournful darkness whereby my abilities within me are hidden
from me; so that my mind making enquiry into herself of her own powers,
ventures not readily to believe herself; because even what is in it
is mostly hidden, unless experience reveal it. And no one ought to be
secure in that life, the whole whereof is called a trial, that he who
hath been capable of worse to be made better, may not likewise of better
be made worse. Our only hope, only confidence, only assured promise is
Thy mercy.

The delights of the ear had more firmly entangled and subdued me; but
Thou didst loosen and free me. Now, in those melodies which Thy words
breathe soul into, when sung with a sweet and attuned voice, I do
a little repose; yet not so as to be held thereby, but that I can
disengage myself when I will. But with the words which are their
life and whereby they find admission into me, themselves seek in my
affections a place of some estimation, and I can scarcely assign them
one suitable. For at one time I seem to myself to give them more honour
than is seemly, feeling our minds to be more holily and fervently raised
unto a flame of devotion, by the holy words themselves when thus sung,
than when not; and that the several affections of our spirit, by a sweet
variety, have their own proper measures in the voice and singing, by
some hidden correspondence wherewith they are stirred up. But this
contentment of the flesh, to which the soul must not be given over to be
enervated, doth oft beguile me, the sense not so waiting upon reason as
patiently to follow her; but having been admitted merely for her sake,
it strives even to run before her, and lead her. Thus in these things I
unawares sin, but afterwards am aware of it.

At other times, shunning over-anxiously this very deception, I err in
too great strictness; and sometimes to that degree, as to wish the whole
melody of sweet music which is used to David's Psalter, banished from
my ears, and the Church's too; and that mode seems to me safer, which I
remember to have been often told me of Athanasius, Bishop of Alexandria,
who made the reader of the psalm utter it with so slight inflection
of voice, that it was nearer speaking than singing. Yet again, when
I remember the tears I shed at the Psalmody of Thy Church, in the
beginning of my recovered faith; and how at this time I am moved, not
with the singing, but with the things sung, when they are sung with a
clear voice and modulation most suitable, I acknowledge the great use
of this institution. Thus I fluctuate between peril of pleasure and
approved wholesomeness; inclined the rather (though not as pronouncing
an irrevocable opinion) to approve of the usage of singing in the
church; that so by the delight of the ears the weaker minds may rise to
the feeling of devotion. Yet when it befalls me to be more moved with
the voice than the words sung, I confess to have sinned penally, and
then had rather not hear music. See now my state; weep with me, and weep
for me, ye, whoso regulate your feelings within, as that good action
ensues. For you who do not act, these things touch not you. But Thou, O
Lord my God, hearken; behold, and see, and have mercy and heal me, Thou,
in whose presence I have become a problem to myself; and that is my
infirmity.

There remains the pleasure of these eyes of my flesh, on which to make
my confessions in the hearing of the ears of Thy temple, those brotherly
and devout ears; and so to conclude the temptations of the lust of
the flesh, which yet assail me, groaning earnestly, and desiring to be
clothed upon with my house from heaven. The eyes love fair and varied
forms, and bright and soft colours. Let not these occupy my soul; let
God rather occupy it, who made these things, very good indeed, yet is
He my good, not they. And these affect me, waking, the whole day, nor
is any rest given me from them, as there is from musical, sometimes in
silence, from all voices. For this queen of colours, the light, bathing
all which we behold, wherever I am through the day, gliding by me in
varied forms, soothes me when engaged on other things, and not observing
it. And so strongly doth it entwine itself, that if it be suddenly
withdrawn, it is with longing sought for, and if absent long, saddeneth
the mind.

O Thou Light, which Tobias saw, when, these eyes closed, he taught his
son the way of life; and himself went before with the feet of charity,
never swerving. Or which Isaac saw, when his fleshly eyes being heavy
and closed by old age, it was vouchsafed him, not knowingly, to bless
his sons, but by blessing to know them. Or which Jacob saw, when he
also, blind through great age, with illumined heart, in the persons of
his sons shed light on the different races of the future people, in
them foresignified; and laid his hands, mystically crossed, upon
his grandchildren by Joseph, not as their father by his outward eye
corrected them, but as himself inwardly discerned. This is the light, it
is one, and all are one, who see and love it. But that corporeal light
whereof I spake, it seasoneth the life of this world for her blind
lovers, with an enticing and dangerous sweetness. But they who know how
to praise Thee for it, "O all-creating Lord," take it up in Thy hymns,
and are not taken up with it in their sleep. Such would I be. These
seductions of the eyes I resist, lest my feet wherewith I walk upon Thy
way be ensnared; and I lift up mine invisible eyes to Thee, that Thou
wouldest pluck my feet out of the snare. Thou dost ever and anon pluck
them out, for they are ensnared. Thou ceasest not to pluck them out,
while I often entangle myself in the snares on all sides laid; because
Thou that keepest Israel shalt neither slumber nor sleep.

What innumerable toys, made by divers arts and manufactures, in our
apparel, shoes, utensils and all sorts of works, in pictures also and
divers images, and these far exceeding all necessary and moderate use
and all pious meaning, have men added to tempt their own eyes withal;
outwardly following what themselves make, inwardly forsaking Him by whom
themselves were made, and destroying that which themselves have been
made! But I, my God and my Glory, do hence also sing a hymn to Thee, and
do consecrate praise to Him who consecrateth me, because those beautiful
patterns which through men's souls are conveyed into their cunning
hands, come from that Beauty, which is above our souls, which my soul
day and night sigheth after. But the framers and followers of the
outward beauties derive thence the rule of judging of them, but not of
using them. And He is there, though they perceive Him not, that so they
might not wander, but keep their strength for Thee, and not scatter it
abroad upon pleasurable weariness. And I, though I speak and see this,
entangle my steps with these outward beauties; but Thou pluckest me out,
O Lord, Thou pluckest me out; because Thy loving-kindness is before my
eyes. For I am taken miserably, and Thou pluckest me out mercifully;
sometimes not perceiving it, when I had but lightly lighted upon them;
otherwhiles with pain, because I had stuck fast in them.

To this is added another form of temptation more manifoldly dangerous.
For besides that concupiscence of the flesh which consisteth in the
delight of all senses and pleasures, wherein its slaves, who go far from
Thee, waste and perish, the soul hath, through the same senses of the
body, a certain vain and curious desire, veiled under the title of
knowledge and learning, not of delighting in the flesh, but of making
experiments through the flesh. The seat whereof being in the appetite
of knowledge, and sight being the sense chiefly used for attaining
knowledge, it is in Divine language called The lust of the eyes. For, to
see, belongeth properly to the eyes; yet we use this word of the other
senses also, when we employ them in seeking knowledge. For we do not
say, hark how it flashes, or smell how it glows, or taste how it shines,
or feel how it gleams; for all these are said to be seen. And yet we
say not only, see how it shineth, which the eyes alone can perceive; but
also, see how it soundeth, see how it smelleth, see how it tasteth,
see how hard it is. And so the general experience of the senses, as
was said, is called The lust of the eyes, because the office of seeing,
wherein the eyes hold the prerogative, the other senses by way
of similitude take to themselves, when they make search after any
knowledge.

But by this may more evidently be discerned, wherein pleasure and
wherein curiosity is the object of the senses; for pleasure seeketh
objects beautiful, melodious, fragrant, savoury, soft; but curiosity,
for trial's sake, the contrary as well, not for the sake of suffering
annoyance, but out of the lust of making trial and knowing them. For
what pleasure hath it, to see in a mangled carcase what will make you
shudder? and yet if it be lying near, they flock thither, to be made
sad, and to turn pale. Even in sleep they are afraid to see it. As if
when awake, any one forced them to see it, or any report of its beauty
drew them thither! Thus also in the other senses, which it were long to
go through. From this disease of curiosity are all those strange sights
exhibited in the theatre. Hence men go on to search out the hidden
powers of nature (which is besides our end), which to know profits not,
and wherein men desire nothing but to know. Hence also, if with that
same end of perverted knowledge magical arts be enquired by. Hence also
in religion itself, is God tempted, when signs and wonders are demanded
of Him, not desired for any good end, but merely to make trial of.

In this so vast wilderness, full of snares and dangers, behold many of
them I have cut off, and thrust out of my heart, as Thou hast given me,
O God of my salvation. And yet when dare I say, since so many things of
this kind buzz on all sides about our daily life-when dare I say that
nothing of this sort engages my attention, or causes in me an idle
interest? True, the theatres do not now carry me away, nor care I to
know the courses of the stars, nor did my soul ever consult ghosts
departed; all sacrilegious mysteries I detest. From Thee, O Lord my God,
to whom I owe humble and single-hearted service, by what artifices
and suggestions doth the enemy deal with me to desire some sign! But I
beseech Thee by our King, and by our pure and holy country, Jerusalem,
that as any consenting thereto is far from me, so may it ever be further
and further. But when I pray Thee for the salvation of any, my end and
intention is far different. Thou givest and wilt give me to follow Thee
willingly, doing what Thou wilt.

Notwithstanding, in how many most petty and contemptible things is our
curiosity daily tempted, and how often we give way, who can recount? How
often do we begin as if we were tolerating people telling vain stories,
lest we offend the weak; then by degrees we take interest therein! I go
not now to the circus to see a dog coursing a hare; but in the field,
if passing, that coursing peradventure will distract me even from some
weighty thought, and draw me after it: not that I turn aside the body
of my beast, yet still incline my mind thither. And unless Thou, having
made me see my infirmity didst speedily admonish me either through the
sight itself by some contemplation to rise towards Thee, or altogether
to despise and pass it by, I dully stand fixed therein. What, when
sitting at home, a lizard catching flies, or a spider entangling them
rushing into her nets, oft-times takes my attention? Is the thing
different, because they are but small creatures? I go on from them to
praise Thee the wonderful Creator and Orderer of all, but this does not
first draw my attention. It is one thing to rise quickly, another not
to fall. And of such things is my life full; and my one hope is Thy
wonderful great mercy. For when our heart becomes the receptacle of such
things, and is overcharged with throngs of this abundant vanity, then
are our prayers also thereby often interrupted and distracted, and
whilst in Thy presence we direct the voice of our heart to Thine ears,
this so great concern is broken off by the rushing in of I know not what
idle thoughts. Shall we then account this also among things of slight
concernment, or shall aught bring us back to hope, save Thy complete
mercy, since Thou hast begun to change us?

And Thou knowest how far Thou hast already changed me, who first
healedst me of the lust of vindicating myself, that so Thou mightest
forgive all the rest of my iniquities, and heal all my infirmities,
and redeem life from corruption, and crown me with mercy and pity, and
satisfy my desire with good things: who didst curb my pride with Thy
fear, and tame my neck to Thy yoke. And now I bear it and it is light
unto me, because so hast Thou promised, and hast made it; and verily so
it was, and I knew it not, when I feared to take it.

But, O Lord, Thou alone Lord without pride, because Thou art the only
true Lord, who hast no lord; hath this third kind of temptation also
ceased from me, or can it cease through this whole life? To wish,
namely, to be feared and loved of men, for no other end, but that we
may have a joy therein which is no joy? A miserable life this and a foul
boastfulness! Hence especially it comes that men do neither purely love
nor fear Thee. And therefore dost Thou resist the proud, and givest
grace to the humble: yea, Thou thunderest down upon the ambitions of the
world, and the foundations of the mountains tremble. Because now certain
offices of human society make it necessary to be loved and feared of
men, the adversary of our true blessedness layeth hard at us, every
where spreading his snares of "well-done, well-done"; that greedily
catching at them, we may be taken unawares, and sever our joy from Thy
truth, and set it in the deceivingness of men; and be pleased at being
loved and feared, not for Thy sake, but in Thy stead: and thus having
been made like him, he may have them for his own, not in the bands of
charity, but in the bonds of punishment: who purposed to set his throne
in the north, that dark and chilled they might serve him, pervertedly
and crookedly imitating Thee. But we, O Lord, behold we are Thy little
flock; possess us as Thine, stretch Thy wings over us, and let us fly
under them. Be Thou our glory; let us be loved for Thee, and Thy word
feared in us. Who would be praised of men when Thou blamest, will not be
defended of men when Thou judgest; nor delivered when Thou condemnest.
But when--not the sinner is praised in the desires of his soul, nor he
blessed who doth ungodlily, but--a man is praised for some gift which
Thou hast given him, and he rejoices more at the praise for himself than
that he hath the gift for which he is praised, he also is praised, while
Thou dispraisest; better is he who praised than he who is praised. For
the one took pleasure in the gift of God in man; the other was better
pleased with the gift of man, than of God.

By these temptations we are assailed daily, O Lord; without ceasing are
we assailed. Our daily furnace is the tongue of men. And in this way
also Thou commandest us continence. Give what Thou enjoinest, and enjoin
what Thou wilt. Thou knowest on this matter the groans of my heart, and
the floods of mine eyes. For I cannot learn how far I am more cleansed
from this plague, and I much fear my secret sins, which Thine eyes know,
mine do not. For in other kinds of temptations I have some sort of means
of examining myself; in this, scarce any. For, in refraining my mind
from the pleasures of the flesh and idle curiosity, I see how much I
have attained lo, when I do without them; foregoing, or not having them.
For then I ask myself how much more or less troublesome it is to me not
to have them? Then, riches, which are desired, that they may serve to
some one or two or all of the three concupiscences, if the soul cannot
discern whether, when it hath them, it despiseth them, they may be
cast aside, that so it may prove itself. But to be without praise,
and therein essay our powers, must we live ill, yea so abandonedly and
atrociously, that no one should know without detesting us? What greater
madness can be said or thought of? But if praise useth and ought to
accompany a good life and good works, we ought as little to forego its
company, as good life itself. Yet I know not whether I can well or ill
be without anything, unless it be absent.

What then do I confess unto Thee in this kind of temptation, O Lord?
What, but that I am delighted with praise, but with truth itself, more
than with praise? For were it proposed to me, whether I would, being
frenzied in error on all things, be praised by all men, or being
consistent and most settled in the truth be blamed by all, I see which
I should choose. Yet fain would I that the approbation of another should
not even increase my joy for any good in me. Yet I own, it doth increase
it, and not so only, but dispraise doth diminish it. And when I am
troubled at this my misery, an excuse occurs to me, which of what value
it is, Thou God knowest, for it leaves me uncertain. For since Thou hast
commanded us not continency alone, that is, from what things to refrain
our love, but righteousness also, that is, whereon to bestow it, and
hast willed us to love not Thee only, but our neighbour also; often,
when pleased with intelligent praise, I seem to myself to be pleased
with the proficiency or towardliness of my neighbour, or to be grieved
for evil in him, when I hear him dispraise either what he understands
not, or is good. For sometimes I am grieved at my own praise, either
when those things be praised in me, in which I mislike myself, or even
lesser and slight goods are more esteemed than they ought. But again how
know I whether I am therefore thus affected, because I would not have
him who praiseth me differ from me about myself; not as being influenced
by concern for him, but because those same good things which please me
in myself, please me more when they please another also? For some how I
am not praised when my judgment of myself is not praised; forasmuch
as either those things are praised, which displease me; or those more,
which please me less. Am I then doubtful of myself in this matter?

Behold, in Thee, O Truth, I see that I ought not to be moved at my own
praises, for my own sake, but for the good of my neighbour. And whether
it be so with me, I know not. For herein I know less of myself than of
Thee. I beseech now, O my God, discover to me myself also, that I may
confess unto my brethren, who are to pray for me, wherein I find myself
maimed. Let me examine myself again more diligently. If in my praise I
am moved with the good of my neighbour, why am I less moved if another
be unjustly dispraised than if it be myself? Why am I more stung by
reproach cast upon myself, than at that cast upon another, with the
same injustice, before me? Know I not this also? or is it at last that I
deceive myself, and do not the truth before Thee in my heart and tongue?
This madness put far from me, O Lord, lest mine own mouth be to me the
sinner's oil to make fat my head. I am poor and needy; yet best, while
in hidden groanings I displease myself, and seek Thy mercy, until what
is lacking in my defective state be renewed and perfected, on to that
peace which the eye of the proud knoweth not.

Yet the word which cometh out of the mouth, and deeds known to men,
bring with them a most dangerous temptation through the love of praise:
which, to establish a certain excellency of our own, solicits and
collects men's suffrages. It tempts, even when it is reproved by myself
in myself, on the very ground that it is reproved; and often glories
more vainly of the very contempt of vain-glory; and so it is no longer
contempt of vain-glory, whereof it glories; for it doth not contemn when
it glorieth.

Within also, within is another evil, arising out of a like temptation;
whereby men become vain, pleasing themselves in themselves, though they
please not, or displease or care not to please others. But pleasing
themselves, they much displease Thee, not only taking pleasure in things
not good, as if good, but in Thy good things, as though their own; or
even if as Thine, yet as though for their own merits; or even if as
though from Thy grace, yet not with brotherly rejoicing, but envying
that grace to others. In all these and the like perils and travails,
Thou seest the trembling of my heart; and I rather feel my wounds to be
cured by Thee, than not inflicted by me.

Where hast Thou not walked with me, O Truth, teaching me what to beware,
and what to desire; when I referred to Thee what I could discover
here below, and consulted Thee? With my outward senses, as I might, I
surveyed the world, and observed the life, which my body hath from me,
and these my senses. Thence entered I the recesses of my memory, those
manifold and spacious chambers, wonderfully furnished with innumerable
stores; and I considered, and stood aghast; being able to discern
nothing of these things without Thee, and finding none of them to be
Thee. Nor was I myself, who found out these things, who went over them
all, and laboured to distinguish and to value every thing according
to its dignity, taking some things upon the report of my senses,
questioning about others which I felt to be mingled with myself,
numbering and distinguishing the reporters themselves, and in the large
treasure-house of my memory revolving some things, storing up others,
drawing out others. Nor yet was I myself when I did this, i.e., that my
power whereby I did it, neither was it Thou, for Thou art the abiding
light, which I consulted concerning all these, whether they were,
what they were, and how to be valued; and I heard Thee directing and
commanding me; and this I often do, this delights me, and as far as I
may be freed from necessary duties, unto this pleasure have I recourse.
Nor in all these which I run over consulting Thee can I find any safe
place for my soul, but in Thee; whither my scattered members may
be gathered, and nothing of me depart from Thee. And sometimes Thou
admittest me to an affection, very unusual, in my inmost soul; rising to
a strange sweetness, which if it were perfected in me, I know not what
in it would not belong to the life to come. But through my miserable
encumbrances I sink down again into these lower things, and am swept
back by former custom, and am held, and greatly weep, but am greatly
held. So much doth the burden of a bad custom weigh us down. Here I can
stay, but would not; there I would, but cannot; both ways, miserable.

Thus then have I considered the sicknesses of my sins in that threefold
concupiscence, and have called Thy right hand to my help. For with a
wounded heart have I beheld Thy brightness, and stricken back I said,
"Who can attain thither? I am cast away from the sight of Thine eyes."
Thou art the Truth who presidest over all, but I through my covetousness
would not indeed forego Thee, but would with Thee possess a lie; as no
man would in such wise speak falsely, as himself to be ignorant of the
truth. So then I lost Thee, because Thou vouchsafest not to be possessed
with a lie.

Whom could I find to reconcile me to Thee? was I to have recourse to
Angels? by what prayers? by what sacraments? Many endeavouring to return
unto Thee, and of themselves unable, have, as I hear, tried this, and
fallen into the desire of curious visions, and been accounted worthy
to be deluded. For they, being high minded, sought Thee by the pride of
learning, swelling out rather than smiting upon their breasts, and so
by the agreement of their heart, drew unto themselves the princes of the
air, the fellow-conspirators of their pride, by whom, through magical
influences, they were deceived, seeking a mediator, by whom they might
be purged, and there was none. For the devil it was, transforming
himself into an Angel of light. And it much enticed proud flesh, that he
had no body of flesh. For they were mortal, and sinners; but thou, Lord,
to whom they proudly sought to be reconciled, art immortal, and without
sin. But a mediator between God and man must have something like to God,
something like to men; lest being in both like to man, he should be
far from God: or if in both like God, too unlike man: and so not be a
mediator. That deceitful mediator then, by whom in Thy secret judgments
pride deserved to be deluded, hath one thing in common with man, that
is sin; another he would seem to have in common with God; and not being
clothed with the mortality of flesh, would vaunt himself to be immortal.
But since the wages of sin is death, this hath he in common with men,
that with them he should be condemned to death.

But the true Mediator, Whom in Thy secret mercy Thou hast showed to the
humble, and sentest, that by His example also they might learn that
same humility, that Mediator between God and man, the Man Christ Jesus,
appeared betwixt mortal sinners and the immortal just One; mortal with
men, just with God: that because the wages of righteousness is life and
peace, He might by a righteousness conjoined with God make void that
death of sinners, now made righteous, which He willed to have in common
with them. Hence He was showed forth to holy men of old; that so they,
through faith in His Passion to come, as we through faith of it passed,
might be saved. For as Man, He was a Mediator; but as the Word, not in
the middle between God and man, because equal to God, and God with God,
and together one God.

How hast Thou loved us, good Father, who sparedst not Thine only Son,
but deliveredst Him up for us ungodly! How hast Thou loved us, for whom
He that thought it no robbery to be equal with Thee, was made subject
even to the death of the cross, He alone, free among the dead, having
power to lay down His life, and power to take it again: for us to Thee
both Victor and Victim, and therefore Victor, because the Victim; for
us to Thee Priest and Sacrifice, and therefore Priest because the
Sacrifice; making us to Thee, of servants, sons by being born of Thee,
and serving us. Well then is my hope strong in Him, that Thou wilt heal
all my infirmities, by Him Who sitteth at Thy right hand and maketh
intercession for us; else should I despair. For many and great are my
infirmities, many they are, and great; but Thy medicine is mightier. We
might imagine that Thy Word was far from any union with man, and despair
of ourselves, unless He had been made flesh and dwelt among us.

Affrighted with my sins and the burden of my misery, I had cast in my
heart, and had purposed to flee to the wilderness: but Thou forbadest
me, and strengthenedst me, saying, Therefore Christ died for all, that
they which live may now no longer live unto themselves, but unto Him
that died for them. See, Lord, I cast my care upon Thee, that I may
live, and consider wondrous things out of Thy law. Thou knowest my
unskilfulness, and my infirmities; teach me, and heal me. He, Thine only
Son, in Whom are hid all the treasures of wisdom and knowledge, hath
redeemed me with His blood. Let not the proud speak evil of me; because
I meditate on my ransom, and eat and drink, and communicate it; and
poor, desired to be satisfied from Him, amongst those that eat and are
satisfied, and they shall praise the Lord who seek Him.




\chapter{BOOK XI}


\start{L}{ord}, since eternity is Thine, art Thou ignorant of what I say to Thee?
or dost Thou see in time, what passeth in time? Why then do I lay in
order before Thee so many relations? Not, of a truth, that Thou mightest
learn them through me, but to stir up mine own and my readers' devotions
towards Thee, that we may all say, Great is the Lord, and greatly to be
praised. I have said already; and again will say, for love of Thy
love do I this. For we pray also, and yet Truth hath said, Your Father
knoweth what you have need of, before you ask. It is then our affections
which we lay open unto Thee, confessing our own miseries, and Thy
mercies upon us, that Thou mayest free us wholly, since Thou hast begun,
that we may cease to be wretched in ourselves, and be blessed in Thee;
seeing Thou hast called us, to become poor in spirit, and meek, and
mourners, and hungering and athirst after righteousness, and merciful,
and pure in heart, and peace-makers. See, I have told Thee many things,
as I could and as I would, because Thou first wouldest that I should
confess unto Thee, my Lord God. For Thou art good, for Thy mercy
endureth for ever.

But how shall I suffice with the tongue of my pen to utter all Thy
exhortations, and all Thy terrors, and comforts, and guidances, whereby
Thou broughtest me to preach Thy Word, and dispense Thy Sacrament to Thy
people? And if I suffice to utter them in order, the drops of time are
precious with me; and long have I burned to meditate in Thy law, and
therein to confess to Thee my skill and unskilfulness, the daybreak of
Thy enlightening, and the remnants of my darkness, until infirmity be
swallowed up by strength. And I would not have aught besides steal away
those hours which I find free from the necessities of refreshing my body
and the powers of my mind, and of the service which we owe to men, or
which though we owe not, we yet pay.

O Lord my god, give ear unto my prayer, and let Thy mercy hearken unto
my desire: because it is anxious not for myself alone, but would serve
brotherly charity; and Thou seest my heart, that so it is. I would
sacrifice to Thee the service of my thought and tongue; do Thou give me,
what I may offer Thee. For I am poor and needy, Thou rich to all that
call upon Thee; Who, inaccessible to care, carest for us. Circumcise
from all rashness and all lying both my inward and outward lips: let
Thy Scriptures be my pure delights: let me not be deceived in them, nor
deceive out of them. Lord, hearken and pity, O Lord my God, Light of the
blind, and Strength of the weak; yea also Light of those that see, and
Strength of the strong; hearken unto my soul, and hear it crying out of
the depths. For if Thine ears be not with us in the depths also, whither
shall we go? whither cry? The day is Thine, and the night is Thine; at
Thy beck the moments flee by. Grant thereof a space for our meditations
in the hidden things of Thy law, and close it not against us who knock.
For not in vain wouldest Thou have the darksome secrets of so many pages
written; nor are those forests without their harts which retire therein
and range and walk; feed, lie down, and ruminate. Perfect me, O
Lord, and reveal them unto me. Behold, Thy voice is my joy; Thy voice
exceedeth the abundance of pleasures. Give what I love: for I do love;
and this hast Thou given: forsake not Thy own gifts, nor despise Thy
green herb that thirsteth. Let me confess unto Thee whatsoever I shall
find in Thy books, and hear the voice of praise, and drink in Thee,
and meditate on the wonderful things out of Thy law; even from the
beginning, wherein Thou madest the heaven and the earth, unto the
everlasting reigning of Thy holy city with Thee.

Lord, have mercy on me, and hear my desire. For it is not, I deem, of
the earth, not of gold and silver, and precious stones, or gorgeous
apparel, or honours and offices, or the pleasures of the flesh, or
necessaries for the body and for this life of our pilgrimage: all which
shall be added unto those that seek Thy kingdom and Thy righteousness.
Behold, O Lord my God, wherein is my desire. The wicked have told me of
delights, but not such as Thy law, O Lord. Behold, wherein is my desire.
Behold, Father, behold, and see and approve; and be it pleasing in the
sight of Thy mercy, that I may find grace before Thee, that the inward
parts of Thy words be opened to me knocking. I beseech by our Lord Jesus
Christ Thy Son, the Man of Thy right hand, the Son of man, whom Thou
hast established for Thyself, as Thy Mediator and ours, through Whom
Thou soughtest us, not seeking Thee, but soughtest us, that we might
seek Thee,--Thy Word, through Whom Thou madest all things, and among
them, me also;--Thy Only-Begotten, through Whom Thou calledst to
adoption the believing people, and therein me also;--I beseech Thee by
Him, who sitteth at Thy right hand, and intercedeth with Thee for us,
in Whom are hidden all the treasures of wisdom and knowledge. These do
I seek in Thy books. Of Him did Moses write; this saith Himself; this
saith the Truth.

I would hear and understand, how "In the Beginning Thou madest the
heaven and earth." Moses wrote this, wrote and departed, passed hence
from Thee to Thee; nor is he now before me. For if he were, I would hold
him and ask him, and beseech him by Thee to open these things unto me,
and would lay the ears of my body to the sounds bursting out of his
mouth. And should he speak Hebrew, in vain will it strike on my senses,
nor would aught of it touch my mind; but if Latin, I should know what
he said. But whence should I know, whether he spake truth? Yea, and if
I knew this also, should I know it from him? Truly within me, within, in
the chamber of my thoughts, Truth, neither Hebrew, nor Greek, nor Latin,
nor barbarian, without organs of voice or tongue, or sound of syllables,
would say, "It is truth," and I forthwith should say confidently to that
man of Thine, "thou sayest truly." Whereas then I cannot enquire of him,
Thee, Thee I beseech, O Truth, full of Whom he spake truth, Thee, my
God, I beseech, forgive my sins; and Thou, who gavest him Thy servant to
speak these things, give to me also to understand them.

Behold, the heavens and the earth are; they proclaim that they were
created; for they change and vary. Whereas whatsoever hath not been
made, and yet is, hath nothing in it, which before it had not; and
this it is, to change and vary. They proclaim also, that they made not
themselves; "therefore we are, because we have been made; we were not
therefore, before we were, so as to make ourselves." Now the evidence
of the thing, is the voice of the speakers. Thou therefore, Lord, madest
them; who art beautiful, for they are beautiful; who art good, for they
are good; who art, for they are; yet are they not beautiful nor good,
nor are they, as Thou their Creator art; compared with Whom, they are
neither beautiful, nor good, nor are. This we know, thanks be to Thee.
And our knowledge, compared with Thy knowledge, is ignorance.

But how didst Thou make the heaven and the earth? and what the engine of
Thy so mighty fabric? For it was not as a human artificer, forming one
body from another, according to the discretion of his mind, which can
in some way invest with such a form, as it seeth in itself by its inward
eye. And whence should he be able to do this, unless Thou hadst made
that mind? and he invests with a form what already existeth, and hath
a being, as clay, or stone, or wood, or gold, or the like. And whence
should they be, hadst not Thou appointed them? Thou madest the artificer
his body, Thou the mind commanding the limbs, Thou the matter whereof he
makes any thing; Thou the apprehension whereby to take in his art, and
see within what he doth without; Thou the sense of his body, whereby,
as by an interpreter, he may from mind to matter, convey that which he
doth, and report to his mind what is done; that it within may consult
the truth, which presideth over itself, whether it be well done or no.
All these praise Thee, the Creator of all. But how dost Thou make them?
how, O God, didst Thou make heaven and earth? Verily, neither in the
heaven, nor in the earth, didst Thou make heaven and earth; nor in the
air, or waters, seeing these also belong to the heaven and the earth;
nor in the whole world didst Thou make the whole world; because there
was no place where to make it, before it was made, that it might be. Nor
didst Thou hold any thing in Thy hand, whereof to make heaven and earth.
For whence shouldest Thou have this, which Thou hadst not made, thereof
to make any thing? For what is, but because Thou art? Therefore Thou
spokest, and they were made, and in Thy Word Thou madest them.

But how didst Thou speak? In the way that the voice came out of the
cloud, saying, This is my beloved Son? For that voice passed by and
passed away, began and ended; the syllables sounded and passed away,
the second after the first, the third after the second, and so forth in
order, until the last after the rest, and silence after the last. Whence
it is abundantly clear and plain that the motion of a creature expressed
it, itself temporal, serving Thy eternal will. And these Thy words,
created for a time, the outward ear reported to the intelligent soul,
whose inward ear lay listening to Thy Eternal Word. But she compared
these words sounding in time, with that Thy Eternal Word in silence, and
said "It is different, far different. These words are far beneath me,
nor are they, because they flee and pass away; but the Word of my Lord
abideth above me for ever." If then in sounding and passing words Thou
saidst that heaven and earth should be made, and so madest heaven and
earth, there was a corporeal creature before heaven and earth, by whose
motions in time that voice might take his course in time. But there was
nought corporeal before heaven and earth; or if there were, surely Thou
hadst, without such a passing voice, created that, whereof to make this
passing voice, by which to say, Let the heaven and the earth be made.
For whatsoever that were, whereof such a voice were made, unless by
Thee it were made, it could not be at all. By what Word then didst Thou
speak, that a body might be made, whereby these words again might be
made?

Thou callest us then to understand the Word, God, with Thee God, Which
is spoken eternally, and by It are all things spoken eternally. For what
was spoken was not spoken successively, one thing concluded that the
next might be spoken, but all things together and eternally. Else have
we time and change; and not a true eternity nor true immortality. This I
know, O my God, and give thanks. I know, I confess to Thee, O Lord, and
with me there knows and blesses Thee, whoso is not unthankful to assure
Truth. We know, Lord, we know; since inasmuch as anything is not which
was, and is, which was not, so far forth it dieth and ariseth. Nothing
then of Thy Word doth give place or replace, because It is truly
immortal and eternal. And therefore unto the Word coeternal with Thee
Thou dost at once and eternally say all that Thou dost say; and whatever
Thou sayest shall be made is made; nor dost Thou make, otherwise than by
saying; and yet are not all things made together, or everlasting, which
Thou makest by saying.

Why, I beseech Thee, O Lord my God? I see it in a way; but how to
express it, I know not, unless it be, that whatsoever begins to be, and
leaves off to be, begins then, and leaves off then, when in Thy eternal
Reason it is known, that it ought to begin or leave off; in which Reason
nothing beginneth or leaveth off. This is Thy Word, which is also "the
Beginning, because also It speaketh unto us." Thus in the Gospel He
speaketh through the flesh; and this sounded outwardly in the ears of
men; that it might be believed and sought inwardly, and found in
the eternal Verity; where the good and only Master teacheth all His
disciples. There, Lord, hear I Thy voice speaking unto me; because He
speaketh us, who teacheth us; but He that teacheth us not, though
He speaketh, to us He speaketh not. Who now teacheth us, but the
unchangeable Truth? for even when we are admonished through a changeable
creature; we are but led to the unchangeable Truth; where we learn
truly, while we stand and hear Him, and rejoice greatly because of the
Bridegroom's voice, restoring us to Him, from Whom we are. And therefore
the Beginning, because unless It abided, there should not, when we
went astray, be whither to return. But when we return from error, it is
through knowing; and that we may know, He teacheth us, because He is the
Beginning, and speaking unto us.

In this Beginning, O God, hast Thou made heaven and earth, in Thy
Word, in Thy Son, in Thy Power, in Thy Wisdom, in Thy Truth; wondrously
speaking, and wondrously making. Who shall comprehend? Who declare
it? What is that which gleams through me, and strikes my heart without
hurting it; and I shudder and kindle? I shudder, inasmuch as I am unlike
it; I kindle, inasmuch as I am like it. It is Wisdom, Wisdom's self
which gleameth through me; severing my cloudiness which yet again
mantles over me, fainting from it, through the darkness which for my
punishment gathers upon me. For my strength is brought down in need,
so that I cannot support my blessings, till Thou, Lord, Who hast been
gracious to all mine iniquities, shalt heal all my infirmities. For
Thou shalt also redeem my life from corruption, and crown me with loving
kindness and tender mercies, and shalt satisfy my desire with good
things, because my youth shall be renewed like an eagle's. For in hope
we are saved, wherefore we through patience wait for Thy promises. Let
him that is able, hear Thee inwardly discoursing out of Thy oracle: I
will boldly cry out, How wonderful are Thy works, O Lord, in Wisdom
hast Thou made them all; and this Wisdom is the Beginning, and in that
Beginning didst Thou make heaven and earth.

Lo, are they not full of their old leaven, who say to us, "What was
God doing before He made heaven and earth? For if (say they) He were
unemployed and wrought not, why does He not also henceforth, and for
ever, as He did heretofore? For did any new motion arise in God, and a
new will to make a creature, which He had never before made, how then
would that be a true eternity, where there ariseth a will, which was
not? For the will of God is not a creature, but before the creature;
seeing nothing could be created, unless the will of the Creator had
preceded. The will of God then belongeth to His very Substance. And
if aught have arisen in God's Substance, which before was not, that
Substance cannot be truly called eternal. But if the will of God has
been from eternity that the creature should be, why was not the creature
also from eternity?"

Who speak thus, do not yet understand Thee, O Wisdom of God, Light of
souls, understand not yet how the things be made, which by Thee, and
in Thee are made: yet they strive to comprehend things eternal, whilst
their heart fluttereth between the motions of things past and to come,
and is still unstable. Who shall hold it, and fix it, that it be settled
awhile, and awhile catch the glory of that ever-fixed Eternity, and
compare it with the times which are never fixed, and see that it cannot
be compared; and that a long time cannot become long, but out of many
motions passing by, which cannot be prolonged altogether; but that in
the Eternal nothing passeth, but the whole is present; whereas no time
is all at once present: and that all time past, is driven on by time to
come, and all to come followeth upon the past; and all past and to come,
is created, and flows out of that which is ever present? Who shall hold
the heart of man, that it may stand still, and see how eternity ever
still-standing, neither past nor to come, uttereth the times past and to
come? Can my hand do this, or the hand of my mouth by speech bring about
a thing so great?

See, I answer him that asketh, "What did God before He made heaven and
earth?" I answer not as one is said to have done merrily (eluding the
pressure of the question), "He was preparing hell (saith he) for pryers
into mysteries." It is one thing to answer enquiries, another to make
sport of enquirers. So I answer not; for rather had I answer, "I know
not," what I know not, than so as to raise a laugh at him who asketh
deep things and gain praise for one who answereth false things. But I
say that Thou, our God, art the Creator of every creature: and if by
the name "heaven and earth," every creature be understood; I boldly say,
"that before God made heaven and earth, He did not make any thing." For
if He made, what did He make but a creature? And would I knew whatsoever
I desire to know to my profit, as I know, that no creature was made,
before there was made any creature.

But if any excursive brain rove over the images of forepassed times, and
wonder that Thou the God Almighty and All-creating and All-supporting,
Maker of heaven and earth, didst for innumerable ages forbear from so
great a work, before Thou wouldest make it; let him awake and consider,
that he wonders at false conceits. For whence could innumerable ages
pass by, which Thou madest not, Thou the Author and Creator of all
ages? or what times should there be, which were not made by Thee? or
how should they pass by, if they never were? Seeing then Thou art the
Creator of all times, if any time was before Thou madest heaven and
earth, why say they that Thou didst forego working? For that very time
didst Thou make, nor could times pass by, before Thou madest those
times. But if before heaven and earth there was no time, why is it
demanded, what Thou then didst? For there was no "then," when there was
no time.

Nor dost Thou by time, precede time: else shouldest Thou not precede
all times. But Thou precedest all things past, by the sublimity of
an ever-present eternity; and surpassest all future because they are
future, and when they come, they shall be past; but Thou art the Same,
and Thy years fail not. Thy years neither come nor go; whereas ours both
come and go, that they all may come. Thy years stand together, because
they do stand; nor are departing thrust out by coming years, for they
pass not away; but ours shall all be, when they shall no more be. Thy
years are one day; and Thy day is not daily, but To-day, seeing Thy
To-day gives not place unto to-morrow, for neither doth it replace
yesterday. Thy To-day, is Eternity; therefore didst Thou beget The
Coeternal, to whom Thou saidst, This day have I begotten Thee. Thou hast
made all things; and before all times Thou art: neither in any time was
time not.

At no time then hadst Thou not made any thing, because time itself Thou
madest. And no times are coeternal with Thee, because Thou abidest;
but if they abode, they should not be times. For what is time? Who can
readily and briefly explain this? Who can even in thought comprehend it,
so as to utter a word about it? But what in discourse do we mention more
familiarly and knowingly, than time? And, we understand, when we speak
of it; we understand also, when we hear it spoken of by another. What
then is time? If no one asks me, I know: if I wish to explain it to one
that asketh, I know not: yet I say boldly that I know, that if nothing
passed away, time past were not; and if nothing were coming, a time to
come were not; and if nothing were, time present were not. Those two
times then, past and to come, how are they, seeing the past now is
not, and that to come is not yet? But the present, should it always be
present, and never pass into time past, verily it should not be time,
but eternity. If time present (if it is to be time) only cometh into
existence, because it passeth into time past, how can we say that either
this is, whose cause of being is, that it shall not be; so, namely, that
we cannot truly say that time is, but because it is tending not to be?

And yet we say, "a long time" and "a short time"; still, only of time
past or to come. A long time past (for example) we call an hundred years
since; and a long time to come, an hundred years hence. But a short
time past, we call (suppose) often days since; and a short time to come,
often days hence. But in what sense is that long or short, which is not?
For the past, is not now; and the future, is not yet. Let us not then
say, "it is long"; but of the past, "it hath been long"; and of the
future, "it will be long." O my Lord, my Light, shall not here also Thy
Truth mock at man? For that past time which was long, was it long when
it was now past, or when it was yet present? For then might it be long,
when there was, what could be long; but when past, it was no longer;
wherefore neither could that be long, which was not at all. Let us not
then say, "time past hath been long": for we shall not find, what hath
been long, seeing that since it was past, it is no more, but let us say,
"that present time was long"; because, when it was present, it was long.
For it had not yet passed away, so as not to be; and therefore there
was, what could be long; but after it was past, that ceased also to be
long, which ceased to be.

Let us see then, thou soul of man, whether present time can be long:
for to thee it is given to feel and to measure length of time. What wilt
thou answer me? Are an hundred years, when present, a long time? See
first, whether an hundred years can be present. For if the first of
these years be now current, it is present, but the other ninety and
nine are to come, and therefore are not yet, but if the second year be
current, one is now past, another present, the rest to come. And so if
we assume any middle year of this hundred to be present, all before it,
are past; all after it, to come; wherefore an hundred years cannot be
present. But see at least whether that one which is now current, itself
is present; for if the current month be its first, the rest are to come;
if the second, the first is already past, and the rest are not yet.
Therefore, neither is the year now current present; and if not present
as a whole, then is not the year present. For twelve months are a year;
of which whatever by the current month is present; the rest past, or to
come. Although neither is that current month present; but one day only;
the rest being to come, if it be the first; past, if the last; if any of
the middle, then amid past and to come.

See how the present time, which alone we found could be called long, is
abridged to the length scarce of one day. But let us examine that also;
because neither is one day present as a whole. For it is made up of four
and twenty hours of night and day: of which, the first hath the rest to
come; the last hath them past; and any of the middle hath those before
it past, those behind it to come. Yea, that one hour passeth away in
flying particles. Whatsoever of it hath flown away, is past; whatsoever
remaineth, is to come. If an instant of time be conceived, which cannot
be divided into the smallest particles of moments, that alone is it,
which may be called present. Which yet flies with such speed from future
to past, as not to be lengthened out with the least stay. For if it be,
it is divided into past and future. The present hath no space. Where
then is the time, which we may call long? Is it to come? Of it we do not
say, "it is long"; because it is not yet, so as to be long; but we say,
"it will be long." When therefore will it be? For if even then, when it
is yet to come, it shall not be long (because what can be long, as yet
is not), and so it shall then be long, when from future which as yet is
not, it shall begin now to be, and have become present, that so there
should exist what may be long; then does time present cry out in the
words above, that it cannot be long.

And yet, Lord, we perceive intervals of times, and compare them, and
say, some are shorter, and others longer. We measure also, how much
longer or shorter this time is than that; and we answer, "This is
double, or treble; and that, but once, or only just so much as that."
But we measure times as they are passing, by perceiving them; but past,
which now are not, or the future, which are not yet, who can measure?
unless a man shall presume to say, that can be measured, which is not.
When then time is passing, it may be perceived and measured; but when it
is past, it cannot, because it is not.

I ask, Father, I affirm not: O my God, rule and guide me. "Who will tell
me that there are not three times (as we learned when boys, and taught
boys), past, present, and future; but present only, because those two
are not? Or are they also; and when from future it becometh present,
doth it come out of some secret place; and so, when retiring, from
present it becometh past? For where did they, who foretold things to
come, see them, if as yet they be not? For that which is not, cannot be
seen. And they who relate things past, could not relate them, if in mind
they did not discern them, and if they were not, they could no way be
discerned. Things then past and to come, are."

Permit me, Lord, to seek further. O my hope, let not my purpose be
confounded. For if times past and to come be, I would know where they
be. Which yet if I cannot, yet I know, wherever they be, they are not
there as future, or past, but present. For if there also they be future,
they are not yet there; if there also they be past, they are no longer
there. Wheresoever then is whatsoever is, it is only as present.
Although when past facts are related, there are drawn out of the memory,
not the things themselves which are past, but words which, conceived by
the images of the things, they, in passing, have through the senses left
as traces in the mind. Thus my childhood, which now is not, is in time
past, which now is not: but now when I recall its image, and tell of it,
I behold it in the present, because it is still in my memory. Whether
there be a like cause of foretelling things to come also; that of
things which as yet are not, the images may be perceived before, already
existing, I confess, O my God, I know not. This indeed I know, that we
generally think before on our future actions, and that that forethinking
is present, but the action whereof we forethink is not yet, because it
is to come. Which, when we have set upon, and have begun to do what
we were forethinking, then shall that action be; because then it is no
longer future, but present.

Which way soever then this secret fore-perceiving of things to come be;
that only can be seen, which is. But what now is, is not future,
but present. When then things to come are said to be seen, it is not
themselves which as yet are not (that is, which are to be), but their
causes perchance or signs are seen, which already are. Therefore they
are not future but present to those who now see that, from which
the future, being foreconceived in the mind, is foretold. Which
fore-conceptions again now are; and those who foretell those things, do
behold the conceptions present before them. Let now the numerous variety
of things furnish me some example. I behold the day-break, I foreshow,
that the sun, is about to rise. What I behold, is present; what I
foresignify, to come; not the sun, which already is; but the sun-rising,
which is not yet. And yet did I not in my mind imagine the sun-rising
itself (as now while I speak of it), I could not foretell it. But
neither is that day-break which I discern in the sky, the sun-rising,
although it goes before it; nor that imagination of my mind; which two
are seen now present, that the other which is to be may be foretold.
Future things then are not yet: and if they be not yet, they are not:
and if they are not, they cannot be seen; yet foretold they may be from
things present, which are already, and are seen.

Thou then, Ruler of Thy creation, by what way dost Thou teach souls
things to come? For Thou didst teach Thy Prophets. By what way dost
Thou, to whom nothing is to come, teach things to come; or rather of the
future, dost teach things present? For, what is not, neither can it be
taught. Too far is this way of my ken: it is too mighty for me, I cannot
attain unto it; but from Thee I can, when Thou shalt vouchsafe it, O
sweet light of my hidden eyes.

What now is clear and plain is, that neither things to come nor past
are. Nor is it properly said, "there be three times, past, present,
and to come": yet perchance it might be properly said, "there be three
times; a present of things past, a present of things present, and a
present of things future." For these three do exist in some sort, in the
soul, but otherwhere do I not see them; present of things past, memory;
present of things present, sight; present of things future, expectation.
If thus we be permitted to speak, I see three times, and I confess there
are three. Let it be said too, "there be three times, past, present, and
to come": in our incorrect way. See, I object not, nor gainsay, nor find
fault, if what is so said be but understood, that neither what is to be,
now is, nor what is past. For but few things are there, which we
speak properly, most things improperly; still the things intended are
understood.

I said then even now, we measure times as they pass, in order to be able
to say, this time is twice so much as that one; or, this is just so
much as that; and so of any other parts of time, which be measurable.
Wherefore, as I said, we measure times as they pass. And if any should
ask me, "How knowest thou?" I might answer, "I know, that we do measure,
nor can we measure things that are not; and things past and to come, are
not." But time present how do we measure, seeing it hath no space? It
is measured while passing, but when it shall have passed, it is not
measured; for there will be nothing to be measured. But whence, by what
way, and whither passes it while it is a measuring? whence, but from the
future? Which way, but through the present? whither, but into the past?
From that therefore, which is not yet, through that, which hath no
space, into that, which now is not. Yet what do we measure, if not time
in some space? For we do not say, single, and double, and triple, and
equal, or any other like way that we speak of time, except of spaces
of times. In what space then do we measure time passing? In the future,
whence it passeth through? But what is not yet, we measure not. Or in
the present, by which it passes? but no space, we do not measure: or in
the past, to which it passes? But neither do we measure that, which now
is not.

My soul is on fire to know this most intricate enigma. Shut it not up, O
Lord my God, good Father; through Christ I beseech Thee, do not shut up
these usual, yet hidden things, from my desire, that it be hindered from
piercing into them; but let them dawn through Thy enlightening mercy, O
Lord. Whom shall I enquire of concerning these things? and to whom shall
I more fruitfully confess my ignorance, than to Thee, to Whom these
my studies, so vehemently kindled toward Thy Scriptures, are not
troublesome? Give what I love; for I do love, and this hast Thou
given me. Give, Father, Who truly knowest to give good gifts unto Thy
children. Give, because I have taken upon me to know, and trouble is
before me until Thou openest it. By Christ I beseech Thee, in His Name,
Holy of holies, let no man disturb me. For I believed, and therefore do
I speak. This is my hope, for this do I live, that I may contemplate the
delights of the Lord. Behold, Thou hast made my days old, and they pass
away, and how, I know not. And we talk of time, and time, and times, and
times, "How long time is it since he said this"; "how long time since he
did this"; and "how long time since I saw that"; and "this syllable hath
double time to that single short syllable." These words we speak, and
these we hear, and are understood, and understand. Most manifest and
ordinary they are, and the self-same things again are but too deeply
hidden, and the discovery of them were new.

I heard once from a learned man, that the motions of the sun, moon,
and stars, constituted time, and I assented not. For why should not
the motions of all bodies rather be times? Or, if the lights of heaven
should cease, and a potter's wheel run round, should there be no time
by which we might measure those whirlings, and say, that either it
moved with equal pauses, or if it turned sometimes slower, otherwhiles
quicker, that some rounds were longer, other shorter? Or, while we were
saying this, should we not also be speaking in time? Or, should there
in our words be some syllables short, others long, but because those
sounded in a shorter time, these in a longer? God, grant to men to see
in a small thing notices common to things great and small. The stars and
lights of heaven, are also for signs, and for seasons, and for years,
and for days; they are; yet neither should I say, that the going round
of that wooden wheel was a day, nor yet he, that it was therefore no
time.

I desire to know the force and nature of time, by which we measure the
motions of bodies, and say (for example) this motion is twice as long as
that. For I ask, Seeing "day" denotes not the stay only of the sun upon
the earth (according to which day is one thing, night another); but also
its whole circuit from east to east again; according to which we say,
"there passed so many days," the night being included when we say, "so
many days," and the nights not reckoned apart;--seeing then a day is
completed by the motion of the sun and by his circuit from east to east
again, I ask, does the motion alone make the day, or the stay in which
that motion is completed, or both? For if the first be the day; then
should we have a day, although the sun should finish that course in so
small a space of time, as one hour comes to. If the second, then should
not that make a day, if between one sun-rise and another there were
but so short a stay, as one hour comes to; but the sun must go four and
twenty times about, to complete one day. If both, then neither could
that be called a day; if the sun should run his whole round in the
space of one hour; nor that, if, while the sun stood still, so much
time should overpass, as the sun usually makes his whole course in, from
morning to morning. I will not therefore now ask, what that is which is
called day; but, what time is, whereby we, measuring the circuit of the
sun, should say that it was finished in half the time it was wont, if
so be it was finished in so small a space as twelve hours; and comparing
both times, should call this a single time, that a double time; even
supposing the sun to run his round from east to east, sometimes in that
single, sometimes in that double time. Let no man then tell me, that the
motions of the heavenly bodies constitute times, because, when at
the prayer of one, the sun had stood still, till he could achieve his
victorious battle, the sun stood still, but time went on. For in its own
allotted space of time was that battle waged and ended. I perceive
time then to be a certain extension. But do I perceive it, or seem to
perceive it? Thou, Light and Truth, wilt show me.

Dost Thou bid me assent, if any define time to be "motion of a body?"
Thou dost not bid me. For that no body is moved, but in time, I hear;
this Thou sayest; but that the motion of a body is time, I hear not;
Thou sayest it not. For when a body is moved, I by time measure, how
long it moveth, from the time it began to move until it left off? And if
I did not see whence it began; and it continue to move so that I see not
when it ends, I cannot measure, save perchance from the time I began,
until I cease to see. And if I look long, I can only pronounce it to be
a long time, but not how long; because when we say "how long," we do
it by comparison; as, "this is as long as that," or "twice so long as
that," or the like. But when we can mark the distances of the places,
whence and whither goeth the body moved, or his parts, if it moved as in
a lathe, then can we say precisely, in how much time the motion of
that body or his part, from this place unto that, was finished. Seeing
therefore the motion of a body is one thing, that by which we measure
how long it is, another; who sees not, which of the two is rather to
be called time? For and if a body be sometimes moved, sometimes stands
still, then we measure, not his motion only, but his standing still too
by time; and we say, "it stood still, as much as it moved"; or "it stood
still twice or thrice so long as it moved"; or any other space which our
measuring hath either ascertained, or guessed; more or less, as we use
to say. Time then is not the motion of a body.

And I confess to Thee, O Lord, that I yet know not what time is, and
again I confess unto Thee, O Lord, that I know that I speak this in
time, and that having long spoken of time, that very "long" is not long,
but by the pause of time. How then know I this, seeing I know not what
time is? or is it perchance that I know not how to express what I know?
Woe is me, that do not even know, what I know not. Behold, O my God,
before Thee I lie not; but as I speak, so is my heart. Thou shalt light
my candle; Thou, O Lord my God, wilt enlighten my darkness.

Does not my soul most truly confess unto Thee, that I do measure times?
Do I then measure, O my God, and know not what I measure? I measure the
motion of a body in time; and the time itself do I not measure? Or could
I indeed measure the motion of a body how long it were, and in how long
space it could come from this place to that, without measuring the time
in which it is moved? This same time then, how do I measure? do we by a
shorter time measure a longer, as by the space of a cubit, the space
of a rood? for so indeed we seem by the space of a short syllable, to
measure the space of a long syllable, and to say that this is double
the other. Thus measure we the spaces of stanzas, by the spaces of the
verses, and the spaces of the verses, by the spaces of the feet, and the
spaces of the feet, by the spaces of the syllables, and the spaces of
long, by the space of short syllables; not measuring by pages (for then
we measure spaces, not times); but when we utter the words and they pass
by, and we say "it is a long stanza," because composed of so many verses;
long verses, because consisting of so many feet; long feet, because
prolonged by so many syllables; a long syllable because double to a
short one. But neither do we this way obtain any certain measure of
time; because it may be, that a shorter verse, pronounced more fully,
may take up more time than a longer, pronounced hurriedly. And so for a
verse, a foot, a syllable. Whence it seemed to me, that time is nothing
else than protraction; but of what, I know not; and I marvel, if it
be not of the mind itself? For what, I beseech Thee, O my God, do I
measure, when I say, either indefinitely "this is a longer time than
that," or definitely "this is double that"? That I measure time, I know;
and yet I measure not time to come, for it is not yet; nor present,
because it is not protracted by any space; nor past, because it now is
not. What then do I measure? Times passing, not past? for so I said.

Courage, my mind, and press on mightily. God is our helper, He made us,
and not we ourselves. Press on where truth begins to dawn. Suppose, now,
the voice of a body begins to sound, and does sound, and sounds on, and
list, it ceases; it is silence now, and that voice is past, and is
no more a voice. Before it sounded, it was to come, and could not be
measured, because as yet it was not, and now it cannot, because it is
no longer. Then therefore while it sounded, it might; because there then
was what might be measured. But yet even then it was not at a stay; for
it was passing on, and passing away. Could it be measured the rather,
for that? For while passing, it was being extended into some space of
time, so that it might be measured, since the present hath no space. If
therefore then it might, then, lo, suppose another voice hath begun
to sound, and still soundeth in one continued tenor without any
interruption; let us measure it while it sounds; seeing when it hath
left sounding, it will then be past, and nothing left to be measured;
let us measure it verily, and tell how much it is. But it sounds still,
nor can it be measured but from the instant it began in, unto the end
it left in. For the very space between is the thing we measure, namely,
from some beginning unto some end. Wherefore, a voice that is not yet
ended, cannot be measured, so that it may be said how long, or short it
is; nor can it be called equal to another, or double to a single, or the
like. But when ended, it no longer is. How may it then be measured? And
yet we measure times; but yet neither those which are not yet, nor those
which no longer are, nor those which are not lengthened out by some
pause, nor those which have no bounds. We measure neither times to come,
nor past, nor present, nor passing; and yet we do measure times.

"Deus Creator omnium," this verse of eight syllables alternates between
short and long syllables. The four short then, the first, third, fifth,
and seventh, are but single, in respect of the four long, the second,
fourth, sixth, and eighth. Every one of these to every one of those,
hath a double time: I pronounce them, report on them, and find it so,
as one's plain sense perceives. By plain sense then, I measure a long
syllable by a short, and I sensibly find it to have twice so much; but
when one sounds after the other, if the former be short, the latter
long, how shall I detain the short one, and how, measuring, shall I
apply it to the long, that I may find this to have twice so much; seeing
the long does not begin to sound, unless the short leaves sounding? And
that very long one do I measure as present, seeing I measure it not
till it be ended? Now his ending is his passing away. What then is it I
measure? where is the short syllable by which I measure? where the long
which I measure? Both have sounded, have flown, passed away, are no
more; and yet I measure, and confidently answer (so far as is presumed
on a practised sense) that as to space of time this syllable is but
single, that double. And yet I could not do this, unless they were
already past and ended. It is not then themselves, which now are not,
that I measure, but something in my memory, which there remains fixed.

It is in thee, my mind, that I measure times. Interrupt me not, that
is, interrupt not thyself with the tumults of thy impressions. In thee
I measure times; the impression, which things as they pass by cause in
thee, remains even when they are gone; this it is which still present,
I measure, not the things which pass by to make this impression. This
I measure, when I measure times. Either then this is time, or I do not
measure times. What when we measure silence, and say that this silence
hath held as long time as did that voice? do we not stretch out our
thought to the measure of a voice, as if it sounded, that so we may be
able to report of the intervals of silence in a given space of time? For
though both voice and tongue be still, yet in thought we go over poems,
and verses, and any other discourse, or dimensions of motions, and
report as to the spaces of times, how much this is in respect of that,
no otherwise than if vocally we did pronounce them. If a man would utter
a lengthened sound, and had settled in thought how long it should be, he
hath in silence already gone through a space of time, and committing
it to memory, begins to utter that speech, which sounds on, until it be
brought unto the end proposed. Yea it hath sounded, and will sound; for
so much of it as is finished, hath sounded already, and the rest will
sound. And thus passeth it on, until the present intent conveys over
the future into the past; the past increasing by the diminution of the
future, until by the consumption of the future, all is past.

But how is that future diminished or consumed, which as yet is not? or
how that past increased, which is now no longer, save that in the mind
which enacteth this, there be three things done? For it expects, it
considers, it remembers; that so that which it expecteth, through
that which it considereth, passeth into that which it remembereth. Who
therefore denieth, that things to come are not as yet? and yet, there is
in the mind an expectation of things to come. And who denies past things
to be now no longer? and yet is there still in the mind a memory of
things past. And who denieth the present time hath no space, because it
passeth away in a moment? and yet our consideration continueth, through
which that which shall be present proceedeth to become absent. It is not
then future time, that is long, for as yet it is not: but a long future,
is "a long expectation of the future," nor is it time past, which now is
not, that is long; but a long past, is "a long memory of the past."

I am about to repeat a Psalm that I know. Before I begin, my expectation
is extended over the whole; but when I have begun, how much soever of
it I shall separate off into the past, is extended along my memory; thus
the life of this action of mine is divided between my memory as to what
I have repeated, and expectation as to what I am about to repeat; but
"consideration" is present with me, that through it what was future, may
be conveyed over, so as to become past. Which the more it is done again
and again, so much the more the expectation being shortened, is the
memory enlarged: till the whole expectation be at length exhausted, when
that whole action being ended, shall have passed into memory. And this
which takes place in the whole Psalm, the same takes place in each
several portion of it, and each several syllable; the same holds in that
longer action, whereof this Psalm may be part; the same holds in the
whole life of man, whereof all the actions of man are parts; the same
holds through the whole age of the sons of men, whereof all the lives of
men are parts.

But because Thy loving-kindness is better than all lives, behold, my
life is but a distraction, and Thy right hand upheld me, in my Lord the
Son of man, the Mediator betwixt Thee, The One, and us many, many also
through our manifold distractions amid many things, that by Him I may
apprehend in Whom I have been apprehended, and may be re-collected from
my old conversation, to follow The One, forgetting what is behind, and
not distended but extended, not to things which shall be and shall
pass away, but to those things which are before, not distractedly but
intently, I follow on for the prize of my heavenly calling, where I may
hear the voice of Thy praise, and contemplate Thy delights, neither
to come, nor to pass away. But now are my years spent in mourning. And
Thou, O Lord, art my comfort, my Father everlasting, but I have been
severed amid times, whose order I know not; and my thoughts, even
the inmost bowels of my soul, are rent and mangled with tumultuous
varieties, until I flow together into Thee, purified and molten by the
fire of Thy love.

And now will I stand, and become firm in Thee, in my mould, Thy truth;
nor will I endure the questions of men, who by a penal disease thirst
for more than they can contain, and say, "what did God before He made
heaven and earth?" Or, "How came it into His mind to make any thing,
having never before made any thing?" Give them, O Lord, well to
bethink themselves what they say, and to find, that "never" cannot be
predicated, when "time" is not. This then that He is said "never to have
made"; what else is it to say, than "in 'no time' to have made?" Let
them see therefore, that time cannot be without created being, and cease
to speak that vanity. May they also be extended towards those things
which are before; and understand Thee before all times, the eternal
Creator of all times, and that no times be coeternal with Thee, nor any
creature, even if there be any creature before all times.

O Lord my God, what a depth is that recess of Thy mysteries, and how far
from it have the consequences of my transgressions cast me! Heal mine
eyes, that I may share the joy of Thy light. Certainly, if there be mind
gifted with such vast knowledge and foreknowledge, as to know all things
past and to come, as I know one well-known Psalm, truly that mind is
passing wonderful, and fearfully amazing; in that nothing past, nothing
to come in after-ages, is any more hidden from him, than when I sung
that Psalm, was hidden from me what, and how much of it had passed away
from the beginning, what, and how much there remained unto the end. But
far be it that Thou the Creator of the Universe, the Creator of souls
and bodies, far be it, that Thou shouldest in such wise know all things
past and to come. Far, far more wonderfully, and far more mysteriously,
dost Thou know them. For not, as the feelings of one who singeth what he
knoweth, or heareth some well-known song, are through expectation of the
words to come, and the remembering of those that are past, varied,
and his senses divided,--not so doth any thing happen unto Thee,
unchangeably eternal, that is, the eternal Creator of minds. Like then
as Thou in the Beginning knewest the heaven and the earth, without any
variety of Thy knowledge, so madest Thou in the Beginning heaven and
earth, without any distraction of Thy action. Whoso understandeth, let
him confess unto Thee; and whoso understandeth not, let him confess
unto Thee. Oh how high art Thou, and yet the humble in heart are Thy
dwelling-place; for Thou raisest up those that are bowed down, and they
fall not, whose elevation Thou art.




\chapter{BOOK XII}


\start{M}{y} heart, O Lord, touched with the words of Thy Holy Scripture, is much
busied, amid this poverty of my life. And therefore most times, is the
poverty of human understanding copious in words, because enquiring hath
more to say than discovering, and demanding is longer than obtaining,
and our hand that knocks, hath more work to do, than our hand that
receives. We hold the promise, who shall make it null? If God be for us,
who can be against us? Ask, and ye shall have; seek, and ye shall find;
knock, and it shall be opened unto you. For every one that asketh,
receiveth; and he that seeketh, findeth; and to him that knocketh,
shall it be opened. These be Thine own promises: and who need fear to be
deceived, when the Truth promiseth?

The lowliness of my tongue confesseth unto Thy Highness, that Thou
madest heaven and earth; this heaven which I see, and this earth that I
tread upon, whence is this earth that I bear about me; Thou madest it.
But where is that heaven of heavens, O Lord, which we hear of in the
words of the Psalm. The heaven of heavens are the Lord's; but the earth
hath He given to the children of men? Where is that heaven which we see
not, to which all this which we see is earth? For this corporeal whole,
not being wholly every where, hath in such wise received its portion of
beauty in these lower parts, whereof the lowest is this our earth; but
to that heaven of heavens, even the heaven of our earth, is but earth:
yea both these great bodies, may not absurdly be called earth, to that
unknown heaven, which is the Lord's, not the sons' of men.

And now this earth was invisible and without form, and there was I know
not what depth of abyss, upon which there was no light, because it had
no shape. Therefore didst Thou command it to be written, that darkness
was upon the face of the deep; what else than the absence of light? For
had there been light, where should it have been but by being over all,
aloft, and enlightening? Where then light was not, what was the presence
of darkness, but the absence of light? Darkness therefore was upon it,
because light was not upon it; as where sound is not, there is silence.
And what is it to have silence there, but to have no sound there? Hast
not Thou, O Lord, taught his soul, which confesseth unto Thee? Hast not
Thou taught me, Lord, that before Thou formedst and diversifiedst this
formless matter, there was nothing, neither colour, nor figure, nor
body, nor spirit? and yet not altogether nothing; for there was a
certain formlessness, without any beauty.

How then should it be called, that it might be in some measure conveyed
to those of duller mind, but by some ordinary word? And what, among all
parts of the world can be found nearer to an absolute formlessness, than
earth and deep? For, occupying the lowest stage, they are less beautiful
than the other higher parts are, transparent all and shining. Wherefore
then may I not conceive the formlessness of matter (which Thou hadst
created without beauty, whereof to make this beautiful world) to be
suitably intimated unto men, by the name of earth invisible and without
form.

So that when thought seeketh what the sense may conceive under this,
and saith to itself, "It is no intellectual form, as life, or justice;
because it is the matter of bodies; nor object of sense, because being
invisible, and without form, there was in it no object of sight or
sense";--while man's thought thus saith to itself, it may endeavour
either to know it, by being ignorant of it; or to be ignorant, by
knowing it.

But I, Lord, if I would, by my tongue and my pen, confess unto Thee the
whole, whatever Thyself hath taught me of that matter,--the name whereof
hearing before, and not understanding, when they who understood it not,
told me of it, so I conceived of it as having innumerable forms and
diverse, and therefore did not conceive it at all, my mind tossed up and
down foul and horrible "forms" out of all order, but yet "forms" and I
called it without form not that it wanted all form, but because it had
such as my mind would, if presented to it, turn from, as unwonted and
jarring, and human frailness would be troubled at. And still that which
I conceived, was without form, not as being deprived of all form, but
in comparison of more beautiful forms; and true reason did persuade me,
that I must utterly uncase it of all remnants of form whatsoever, if
I would conceive matter absolutely without form; and I could not; for
sooner could I imagine that not to be at all, which should be deprived
of all form, than conceive a thing betwixt form and nothing, neither
formed, nor nothing, a formless almost nothing. So my mind gave over to
question thereupon with my spirit, it being filled with the images of
formed bodies, and changing and varying them, as it willed; and I bent
myself to the bodies themselves, and looked more deeply into their
changeableness, by which they cease to be what they have been, and begin
to be what they were not; and this same shifting from form to form, I
suspected to be through a certain formless state, not through a mere
nothing; yet this I longed to know, not to suspect only.-If then my
voice and pen would confess unto Thee the whole, whatsoever knots Thou
didst open for me in this question, what reader would hold out to take
in the whole? Nor shall my heart for all this cease to give Thee honour,
and a song of praise, for those things which it is not able to express.
For the changeableness of changeable things, is itself capable of all
those forms, into which these changeable things are changed. And this
changeableness, what is it? Is it soul? Is it body? Is it that which
constituteth soul or body? Might one say, "a nothing something", an
"is, is not," I would say, this were it: and yet in some way was it even
then, as being capable of receiving these visible and compound figures.

But whence had it this degree of being, but from Thee, from Whom are all
things, so far forth as they are? But so much the further from Thee, as
the unliker Thee; for it is not farness of place. Thou therefore,
Lord, Who art not one in one place, and otherwise in another, but the
Self-same, and the Self-same, and the Self-same, Holy, Holy, Holy, Lord
God Almighty, didst in the Beginning, which is of Thee, in Thy Wisdom,
which was born of Thine own Substance, create something, and that out of
nothing. For Thou createdst heaven and earth; not out of Thyself, for so
should they have been equal to Thine Only Begotten Son, and thereby to
Thee also; whereas no way were it right that aught should be equal to
Thee, which was not of Thee. And aught else besides Thee was there not,
whereof Thou mightest create them, O God, One Trinity, and Trine Unity;
and therefore out of nothing didst Thou create heaven and earth; a great
thing, and a small thing; for Thou art Almighty and Good, to make all
things good, even the great heaven, and the petty earth. Thou wert, and
nothing was there besides, out of which Thou createdst heaven and earth;
things of two sorts; one near Thee, the other near to nothing; one to
which Thou alone shouldest be superior; the other, to which nothing
should be inferior.

But that heaven of heavens was for Thyself, O Lord; but the earth which
Thou gavest to the sons of men, to be seen and felt, was not such as we
now see and feel. For it was invisible, without form, and there was a
deep, upon which there was no light; or, darkness was above the deep,
that is, more than in the deep. Because this deep of waters, visible
now, hath even in his depths, a light proper for its nature; perceivable
in whatever degree unto the fishes, and creeping things in the bottom
of it. But that whole deep was almost nothing, because hitherto it
was altogether without form; yet there was already that which could be
formed. For Thou, Lord, madest the world of a matter without form, which
out of nothing, Thou madest next to nothing, thereof to make those
great things, which we sons of men wonder at. For very wonderful is this
corporeal heaven; of which firmament between water and water, the second
day, after the creation of light, Thou saidst, Let it be made, and it
was made. Which firmament Thou calledst heaven; the heaven, that is, to
this earth and sea, which Thou madest the third day, by giving a visible
figure to the formless matter, which Thou madest before all days. For
already hadst Thou made both an heaven, before all days; but that was
the heaven of this heaven; because In the beginning Thou hadst made
heaven and earth. But this same earth which Thou madest was formless
matter, because it was invisible and without form, and darkness was
upon the deep, of which invisible earth and without form, of which
formlessness, of which almost nothing, Thou mightest make all these
things of which this changeable world consists, but subsists not; whose
very changeableness appears therein, that times can be observed and
numbered in it. For times are made by the alterations of things, while
the figures, the matter whereof is the invisible earth aforesaid, are
varied and turned.

And therefore the Spirit, the Teacher of Thy servant, when It recounts
Thee to have In the Beginning created heaven and earth, speaks nothing
of times, nothing of days. For verily that heaven of heavens which
Thou createdst in the Beginning, is some intellectual creature, which,
although no ways coeternal unto Thee, the Trinity, yet partaketh of
Thy eternity, and doth through the sweetness of that most happy
contemplation of Thyself, strongly restrain its own changeableness; and
without any fall since its first creation, cleaving close unto Thee, is
placed beyond all the rolling vicissitude of times. Yea, neither is this
very formlessness of the earth, invisible, and without form, numbered
among the days. For where no figure nor order is, there does nothing
come, or go; and where this is not, there plainly are no days, nor any
vicissitude of spaces of times.

O let the Light, the Truth, the Light of my heart, not mine own
darkness, speak unto me. I fell off into that, and became darkened; but
even thence, even thence I loved Thee. I went astray, and remembered
Thee. I heard Thy voice behind me, calling to me to return, and scarcely
heard it, through the tumultuousness of the enemies of peace. And now,
behold, I return in distress and panting after Thy fountain. Let no man
forbid me! of this will I drink, and so live. Let me not be mine own
life; from myself I lived ill, death was I to myself; and I revive in
Thee. Do Thou speak unto me, do Thou discourse unto me. I have believed
Thy Books, and their words be most full of mystery.

Already Thou hast told me with a strong voice, O Lord, in my inner ear,
that Thou art eternal, Who only hast immortality; since Thou canst not
be changed as to figure or motion, nor is Thy will altered by times:
seeing no will which varies is immortal. This is in Thy sight clear to
me, and let it be more and more cleared to me, I beseech Thee; and in
the manifestation thereof, let me with sobriety abide under Thy wings.
Thou hast told me also with a strong voice, O Lord, in my inner ear,
that Thou hast made all natures and substances, which are not what
Thyself is, and yet are; and that only is not from Thee, which is not,
and the motion of the will from Thee who art, unto that which in a less
degree is, because such motion is transgression and sin; and that no
man's sin doth either hurt Thee, or disturb the order of Thy government,
first or last. This is in Thy sight clear unto me, and let it be
more and more cleared to me, I beseech Thee: and in the manifestation
thereof, let me with sobriety abide under Thy wings.

Thou hast told me also with a strong voice, in my inner ear, that
neither is that creature coeternal unto Thyself, whose happiness
Thou only art, and which with a most persevering purity, drawing its
nourishment from Thee, doth in no place and at no time put forth its
natural mutability; and, Thyself being ever present with it, unto Whom
with its whole affection it keeps itself, having neither future to
expect, nor conveying into the past what it remembereth, is neither
altered by any change, nor distracted into any times. O blessed
creature, if such there be, for cleaving unto Thy Blessedness; blest in
Thee, its eternal Inhabitant and its Enlightener! Nor do I find by what
name I may the rather call the heaven of heavens which is the Lord's,
than Thine house, which contemplateth Thy delights without any defection
of going forth to another; one pure mind, most harmoniously one, by that
settled estate of peace of holy spirits, the citizens of Thy city in
heavenly places; far above those heavenly places that we see.

By this may the soul, whose pilgrimage is made long and far away, by
this may she understand, if she now thirsts for Thee, if her tears be
now become her bread, while they daily say unto her, Where is Thy God?
if she now seeks of Thee one thing, and desireth it, that she may dwell
in Thy house all the days of her life (and what is her life, but Thou?
and what Thy days, but Thy eternity, as Thy years which fail not,
because Thou art ever the same?); by this then may the soul that is
able, understand how far Thou art, above all times, eternal; seeing
Thy house which at no time went into a far country, although it be not
coeternal with Thee, yet by continually and unfailingly cleaving unto
Thee, suffers no changeableness of times. This is in Thy sight clear
unto me, and let it be more and more cleared unto me, I beseech Thee,
and in the manifestation thereof, let me with sobriety abide under Thy
wings.

There is, behold, I know not what formlessness in those changes of these
last and lowest creatures; and who shall tell me (unless such a one as
through the emptiness of his own heart, wonders and tosses himself up
and down amid his own fancies?), who but such a one would tell me, that
if all figure be so wasted and consumed away, that there should only
remain that formlessness, through which the thing was changed and turned
from one figure to another, that that could exhibit the vicissitudes
of times? For plainly it could not, because, without the variety of
motions, there are no times: and no variety, where there is no figure.

These things considered, as much as Thou givest, O my God, as much
as Thou stirrest me up to knock, and as much as Thou openest to me
knocking, two things I find that Thou hast made, not within the compass
of time, neither of which is coeternal with Thee. One, which is so
formed, that without any ceasing of contemplation, without any interval
of change, though changeable, yet not changed, it may thoroughly enjoy
Thy eternity and unchangeableness; the other which was so formless,
that it had not that, which could be changed from one form into another,
whether of motion, or of repose, so as to become subject unto time. But
this Thou didst not leave thus formless, because before all days, Thou
in the Beginning didst create Heaven and Earth; the two things that I
spake of. But the Earth was invisible and without form, and darkness
was upon the deep. In which words, is the formlessness conveyed unto us
(that such capacities may hereby be drawn on by degrees, as are not
able to conceive an utter privation of all form, without yet coming to
nothing), out of which another Heaven might be created, together with
a visible and well-formed earth: and the waters diversly ordered, and
whatsoever further is in the formation of the world, recorded to have
been, not without days, created; and that, as being of such nature, that
the successive changes of times may take place in them, as being subject
to appointed alterations of motions and of forms.

This then is what I conceive, O my God, when I hear Thy Scripture
saying, In the beginning God made Heaven and Earth: and the Earth was
invisible and without form, and darkness was upon the deep, and not
mentioning what day Thou createdst them; this is what I conceive, that
because of the Heaven of heavens,--that intellectual Heaven, whose
Intelligences know all at once, not in part, not darkly, not through a
glass, but as a whole, in manifestation, face to face; not, this thing
now, and that thing anon; but (as I said) know all at once, without any
succession of times;--and because of the earth invisible and without
form, without any succession of times, which succession presents "this
thing now, that thing anon"; because where is no form, there is
no distinction of things:--it is, then, on account of these two, a
primitive formed, and a primitive formless; the one, heaven but the
Heaven of heaven, the other earth but the earth invisible and without
form; because of these two do I conceive, did Thy Scripture say without
mention of days, In the Beginning God created Heaven and Earth. For
forthwith it subjoined what earth it spake of; and also, in that the
Firmament is recorded to be created the second day, and called Heaven,
it conveys to us of which Heaven He before spake, without mention of
days.

Wondrous depth of Thy words! whose surface, behold! is before us,
inviting to little ones; yet are they a wondrous depth. O my God, a
wondrous depth! It is awful to look therein; an awfulness of honour, and
a trembling of love. The enemies thereof I hate vehemently; oh that Thou
wouldest slay them with Thy two-edged sword, that they might no longer
be enemies unto it: for so do I love to have them slain unto themselves,
that they may live unto Thee. But behold others not faultfinders, but
extollers of the book of Genesis; "The Spirit of God," say they, "Who
by His servant Moses wrote these things, would not have those words
thus understood; He would not have it understood, as thou sayest, but
otherwise, as we say." Unto Whom Thyself, O Thou God all, being judge,
do I thus answer.

"Will you affirm that to be false, which with a strong voice Truth tells
me in my inner ear, concerning the Eternity of the Creator, that His
substance is no ways changed by time, nor His will separate from His
substance? Wherefore He willeth not one thing now, another anon, but
once, and at once, and always, He willeth all things that He willeth;
not again and again, nor now this, now that; nor willeth afterwards,
what before He willed not, nor willeth not, what before He willed;
because such a will is and no mutable thing is eternal: but our God is
eternal. Again, what He tells me in my inner ear, the expectation of
things to come becomes sight, when they are come, and this same sight
becomes memory, when they be past. Now all thought which thus varies is
mutable; and no mutable thing is eternal: but our God is eternal." These
things I infer, and put together, and find that my God, the eternal God,
hath not upon any new will made any creature, nor doth His knowledge
admit of any thing transitory. "What will ye say then, O ye gainsayers?
Are these things false?" "No," they say; "What then? Is it false, that
every nature already formed, or matter capable of form, is not, but from
Him Who is supremely good, because He is supremely?" "Neither do we deny
this," say they. "What then? do you deny this, that there is a certain
sublime creature, with so chaste a love cleaving unto the true and
truly eternal God, that although not coeternal with Him, yet is it not
detached from Him, nor dissolved into the variety and vicissitude of
times, but reposeth in the most true contemplation of Him only?" Because
Thou, O God, unto him that loveth Thee so much as Thou commandest, dost
show Thyself, and sufficest him; and therefore doth he not decline
from Thee, nor toward himself. This is the house of God, not of earthly
mould, nor of celestial bulk corporeal but spiritual, and partaker of
Thy eternity, because without defection for ever. For Thou hast made
it fast for ever and ever, Thou hast given it a law which it shall not
pass. Nor yet is it coeternal with Thee, O God, because not without
beginning; for it was made.

For although we find no time before it, for wisdom was created before
all things; not that Wisdom which is altogether equal and coeternal unto
Thee, our God, His Father, and by Whom all things were created, and in
Whom, as the Beginning, Thou createdst heaven and earth; but that
wisdom which is created, that is, the intellectual nature, which by
contemplating the light, is light. For this, though created, is also
called wisdom. But what difference there is betwixt the Light which
enlighteneth, and which is enlightened, so much is there betwixt the
Wisdom that createth, and that created; as betwixt the Righteousness
which justifieth, and the righteousness which is made by justification.
For we also are called Thy righteousness; for so saith a certain
servant of Thine, That we might be made the righteousness of God in Him.
Therefore since a certain created wisdom was created before all things,
the rational and intellectual mind of that chaste city of Thine, our
mother which is above, and is free and eternal in the heavens (in
what heavens, if not in those that praise Thee, the Heaven of heavens?
Because this is also the Heaven of heavens for the Lord);--though we
find no time before it (because that which hath been created before all
things, precedeth also the creature of time), yet is the Eternity of
the Creator Himself before it, from Whom, being created, it took the
beginning, not indeed of time (for time itself was not yet), but of its
creation.

Hence it is so of Thee, our God, as to be altogether other than Thou,
and not the Self-same: because though we find time neither before it,
nor even in it (it being meet ever to behold Thy face, nor is ever drawn
away from it, wherefore it is not varied by any change), yet is there in
it a liability to change, whence it would wax dark, and chill, but
that by a strong affection cleaving unto Thee, like perpetual noon, it
shineth and gloweth from Thee. O house most lightsome and delightsome!
I have loved thy beauty, and the place of the habitation of the glory
of my Lord, thy builder and possessor. Let my wayfaring sigh after thee,
and I say to Him that made thee, let Him take possession of me also in
thee, seeing He hath made me likewise. I have gone astray like a lost
sheep: yet upon the shoulders of my Shepherd, thy builder, hope I to be
brought back to thee.

"What say ye to me, O ye gainsayers that I was speaking unto, who yet
believe Moses to have been the holy servant of God, and his books the
oracles of the Holy Ghost? Is not this house of God, not coeternal
indeed with God, yet after its measure, eternal in the heavens, when you
seek for changes of times in vain, because you will not find them? For
that, to which it is ever good to cleave fast to God, surpasses all
extension, and all revolving periods of time." "It is," say they.
"What then of all that which my heart loudly uttered unto my God, when
inwardly it heard the voice of His praise, what part thereof do you
affirm to be false? Is it that the matter was without form, in which
because there was no form, there was no order? But where no order was,
there could be no vicissitude of times: and yet this almost nothing,
inasmuch as it was not altogether nothing, was from Him certainly, from
Whom is whatsoever is, in what degree soever it is." "This also," say
they, "do we not deny."

With these I now parley a little in Thy presence, O my God, who grant
all these things to be true, which Thy Truth whispers unto my soul. For
those who deny these things, let them bark and deafen themselves as much
as they please; I will essay to persuade them to quiet, and to open in
them a way for Thy word. But if they refuse, and repel me; I beseech, O
my God, be not Thou silent to me. Speak Thou truly in my heart; for
only Thou so speakest: and I will let them alone blowing upon the dust
without, and raising it up into their own eyes: and myself will enter
my chamber, and sing there a song of loves unto Thee; groaning with
groanings unutterable, in my wayfaring, and remembering Jerusalem, with
heart lifted up towards it, Jerusalem my country, Jerusalem my mother,
and Thyself that rulest over it, the Enlightener, Father, Guardian,
Husband, the pure and strong delight, and solid joy, and all good things
unspeakable, yea all at once, because the One Sovereign and true Good.
Nor will I be turned away, until Thou gather all that I am, from this
dispersed and disordered estate, into the peace of that our most dear
mother, where the first-fruits of my spirit be already (whence I am
ascertained of these things), and Thou conform and confirm it for ever,
O my God, my Mercy. But those who do not affirm all these truths to be
false, who honour Thy holy Scripture, set forth by holy Moses, placing
it, as we, on the summit of authority to be followed, and do yet
contradict me in some thing, I answer thus; By Thyself judge, O our God,
between my Confessions and these men's contradictions.

For they say, "Though these things be true, yet did not Moses intend
those two, when, by revelation of the Spirit, he said, In the beginning
God created heaven and earth. He did not under the name of heaven,
signify that spiritual or intellectual creature which always beholds the
face of God; nor under the name of earth, that formless matter." "What
then?" "That man of God," say they, "meant as we say, this declared he
by those words." "What?" "By the name of heaven and earth would he first
signify," say they, "universally and compendiously, all this visible
world; so as afterwards by the enumeration of the several days, to
arrange in detail, and, as it were, piece by piece, all those things,
which it pleased the Holy Ghost thus to enounce. For such were that
rude and carnal people to which he spake, that he thought them fit to be
entrusted with the knowledge of such works of God only as were visible."
They agree, however, that under the words earth invisible and without
form, and that darksome deep (out of which it is subsequently shown,
that all these visible things which we all know, were made and arranged
during those "days") may, not incongruously, be understood of this
formless first matter.

What now if another should say that "this same formlessness and
confusedness of matter, was for this reason first conveyed under the
name of heaven and earth, because out of it was this visible world with
all those natures which most manifestly appear in it, which is ofttimes
called by the name of heaven and earth, created and perfected?" What
again if another say that "invisible and visible nature is not indeed
inappropriately called heaven and earth; and so, that the universal
creation, which God made in His Wisdom, that is, in the Beginning, was
comprehended under those two words? Notwithstanding, since all things be
made not of the substance of God, but out of nothing (because they are
not the same that God is, and there is a mutable nature in them all,
whether they abide, as doth the eternal house of God, or be changed, as
the soul and body of man are): therefore the common matter of all things
visible and invisible (as yet unformed though capable of form), out of
which was to be created both heaven and earth (i.e., the invisible and
visible creature when formed), was entitled by the same names given to
the earth invisible and without form and the darkness upon the deep, but
with this distinction, that by the earth invisible and without form is
understood corporeal matter, antecedent to its being qualified by any
form; and by the darkness upon the deep, spiritual matter, before it
underwent any restraint of its unlimited fluidness, or received any
light from Wisdom?"

It yet remains for a man to say, if he will, that "the already perfected
and formed natures, visible and invisible, are not signified under the
name of heaven and earth, when we read, In the beginning God made heaven
and earth, but that the yet unformed commencement of things, the stuff
apt to receive form and making, was called by these names, because
therein were confusedly contained, not as yet distinguished by their
qualities and forms, all those things which being now digested into
order, are called Heaven and Earth, the one being the spiritual, the
other the corporeal, creation."

All which things being heard and well considered, I will not strive
about words: for that is profitable to nothing, but the subversion of
the hearers. But the law is good to edify, if a man use it lawfully: for
that the end of it is charity, out of a pure heart and good conscience,
and faith unfeigned. And well did our Master know, upon which two
commandments He hung all the Law and the Prophets. And what doth it
prejudice me, O my God, Thou light of my eyes in secret, zealously
confessing these things, since divers things may be understood under
these words which yet are all true,--what, I say, doth it prejudice me,
if I think otherwise than another thinketh the writer thought? All we
readers verily strive to trace out and to understand his meaning whom we
read; and seeing we believe him to speak truly, we dare not imagine
him to have said any thing, which ourselves either know or think to
be false. While every man endeavours then to understand in the Holy
Scriptures, the same as the writer understood, what hurt is it, if a man
understand what Thou, the light of all true-speaking minds, dost show
him to be true, although he whom he reads, understood not this, seeing
he also understood a Truth, though not this truth?

For true it is, O Lord, that Thou madest heaven and earth; and it is
true too, that the Beginning is Thy Wisdom, in Which Thou createst all:
and true again, that this visible world hath for its greater part
the heaven and the earth, which briefly comprise all made and created
natures. And true too, that whatsoever is mutable, gives us to
understand a certain want of form, whereby it receiveth a form, or is
changed, or turned. It is true, that that is subject to no times, which
so cleaveth to the unchangeable Form, as though subject to change,
never to be changed. It is true, that that formlessness which is almost
nothing, cannot be subject to the alteration of times. It is true, that
that whereof a thing is made, may by a certain mode of speech, be called
by the name of the thing made of it; whence that formlessness, whereof
heaven and earth were made, might be called heaven and earth. It is
true, that of things having form, there is not any nearer to having
no form, than the earth and the deep. It is true, that not only every
created and formed thing, but whatsoever is capable of being created and
formed, Thou madest, of Whom are all things. It is true, that whatsoever
is formed out of that which had no form, was unformed before it was
formed.

Out of these truths, of which they doubt not whose inward eye Thou hast
enabled to see such things, and who unshakenly believe Thy servant Moses
to have spoken in the Spirit of truth;--of all these then, he taketh
one, who saith, In the Beginning God made the heaven and the earth; that
is, "in His Word coeternal with Himself, God made the intelligible and
the sensible, or the spiritual and the corporeal creature." He another,
that saith, In the Beginning God made heaven and earth; that is, "in
His Word coeternal with Himself, did God make the universal bulk of this
corporeal world, together with all those apparent and known creatures,
which it containeth." He another, that saith, In the Beginning God made
heaven and earth; that is, "in His Word coeternal with Himself, did
God make the formless matter of creatures spiritual and corporeal." He
another, that saith, In the Beginning God created heaven and earth; that
is, "in His Word coeternal with Himself, did God create the formless
matter of the creature corporeal, wherein heaven and earth lay as yet
confused, which, being now distinguished and formed, we at this day see
in the bulk of this world." He another, who saith, In the Beginning God
made heaven and earth; that is, "in the very beginning of creating and
working, did God make that formless matter, confusedly containing in
itself both heaven and earth; out of which, being formed, do they now
stand out, and are apparent, with all that is in them."

And with regard to the understanding of the words following, out of all
those truths, he chooses one to himself, who saith, But the earth was
invisible, and without form, and darkness was upon the deep; that is,
"that corporeal thing that God made, was as yet a formless matter of
corporeal things, without order, without light." Another he who says,
The earth was invisible and without form, and darkness was upon the
deep; that is, "this all, which is called heaven and earth, was still
a formless and darksome matter, of which the corporeal heaven and the
corporeal earth were to be made, with all things in them, which are
known to our corporeal senses." Another he who says, The earth was
invisible and without form, and darkness was upon the deep; that is,
"this all, which is called heaven and earth, was still a formless and
a darksome matter; out of which was to be made, both that intelligible
heaven, otherwhere called the Heaven of heavens, and the earth, that is,
the whole corporeal nature, under which name is comprised this corporeal
heaven also; in a word, out of which every visible and invisible
creature was to be created." Another he who says, The earth was
invisible and without form, and darkness was upon the deep, "the
Scripture did not call that formlessness by the name of heaven and
earth; but that formlessness, saith he, already was, which he called the
earth invisible without form, and darkness upon the deep; of which
he had before said, that God had made heaven and earth, namely, the
spiritual and corporeal creature." Another he who says, The earth was
invisible and without form, and darkness was upon the deep; that is,
"there already was a certain formless matter, of which the Scripture
said before, that God made heaven and earth; namely, the whole corporeal
bulk of the world, divided into two great parts, upper and lower, with
all the common and known creatures in them."

For should any attempt to dispute against these two last opinions, thus,
"If you will not allow, that this formlessness of matter seems to be
called by the name of heaven and earth; Ergo, there was something which
God had not made, out of which to make heaven and earth; for neither
hath Scripture told us, that God made this matter, unless we understand
it to be signified by the name of heaven and earth, or of earth alone,
when it is said, In the Beginning God made the heaven and earth; that so
in what follows, and the earth was invisible and without form (although
it pleased Him so to call the formless matter), we are to understand
no other matter, but that which God made, whereof is written above, God
made heaven and earth." The maintainers of either of those two latter
opinions will, upon hearing this, return for answer, "we do not deny
this formless matter to be indeed created by God, that God of Whom are
all things, very good; for as we affirm that to be a greater good, which
is created and formed, so we confess that to be a lesser good which is
made capable of creation and form, yet still good. We say however that
Scripture hath not set down, that God made this formlessness, as also it
hath not many others; as the Cherubim, and Seraphim, and those which
the Apostle distinctly speaks of, Thrones, Dominions, Principalities,
Powers. All which that God made, is most apparent. Or if in that which
is said, He made heaven and earth, all things be comprehended, what
shall we say of the waters, upon which the Spirit of God moved? For if
they be comprised in this word earth; how then can formless matter be
meant in that name of earth, when we see the waters so beautiful? Or
if it be so taken; why then is it written, that out of the same
formlessness, the firmament was made, and called heaven; and that the
waters were made, is not written? For the waters remain not formless and
invisible, seeing we behold them flowing in so comely a manner. But if
they then received that beauty, when God said, Let the waters under the
firmament be gathered together, that so the gathering together be itself
the forming of them; what will be said as to those waters above the
firmament? Seeing neither if formless would they have been worthy of so
honourable a seat, nor is it written, by what word they were formed. If
then Genesis is silent as to God's making of any thing, which yet
that God did make neither sound faith nor well-grounded understanding
doubteth, nor again will any sober teaching dare to affirm these
waters to be coeternal with God, on the ground that we find them to be
mentioned in the hook of Genesis, but when they were created, we do
not find; why (seeing truth teaches us) should we not understand that
formless matter (which this Scripture calls the earth invisible and
without form, and darksome deep) to have been created of God out of
nothing, and therefore not to be coeternal to Him; notwithstanding this
history hath omitted to show when it was created?"

These things then being heard and perceived, according to the weakness
of my capacity (which I confess unto Thee, O Lord, that knowest it), two
sorts of disagreements I see may arise, when a thing is in words related
by true reporters; one, concerning the truth of the things, the other,
concerning the meaning of the relater. For we enquire one way about
the making of the creature, what is true; another way, what Moses,
that excellent minister of Thy Faith, would have his reader and hearer
understand by those words. For the first sort, away with all those
who imagine themselves to know as a truth, what is false; and for this
other, away with all them too, which imagine Moses to have written
things that be false. But let me be united in Thee, O Lord, with those
and delight myself in Thee, with them that feed on Thy truth, in the
largeness of charity, and let us approach together unto the words of
Thy book, and seek in them for Thy meaning, through the meaning of Thy
servant, by whose pen Thou hast dispensed them.

But which of us shall, among those so many truths, which occur to
enquirers in those words, as they are differently understood, so
discover that one meaning, as to affirm, "this Moses thought," and "this
would he have understood in that history"; with the same confidence
as he would, "this is true," whether Moses thought this or that? For
behold, O my God, I Thy servant, who have in this book vowed a sacrifice
of confession unto Thee, and pray, that by Thy mercy I may pay my vows
unto Thee, can I, with the same confidence wherewith I affirm, that in
Thy incommutable world Thou createdst all things visible and invisible,
affirm also, that Moses meant no other than this, when he wrote, In the
Beginning God made heaven and earth? No. Because I see not in his mind,
that he thought of this when he wrote these things, as I do see it
in Thy truth to be certain. For he might have his thoughts upon God's
commencement of creating, when he said In the beginning; and by heaven
and earth, in this place he might intend no formed and perfected nature
whether spiritual or corporeal, but both of them inchoate and as yet
formless. For I perceive, that whichsoever of the two had been said, it
might have been truly said; but which of the two he thought of in these
words, I do not so perceive. Although, whether it were either of these,
or any sense beside (that I have not here mentioned), which this so
great man saw in his mind, when he uttered these words, I doubt not but
that he saw it truly, and expressed it aptly.

Let no man harass me then, by saying, Moses thought not as you say, but
as I say: for if he should ask me, "How know you that Moses thought that
which you infer out of his words?" I ought to take it in good part, and
would answer perchance as I have above, or something more at large, if
he were unyielding. But when he saith, "Moses meant not what you say,
but what I say," yet denieth not that what each of us say, may both be
true, O my God, life of the poor, in Whose bosom is no contradiction,
pour down a softening dew into my heart, that I may patiently bear with
such as say this to me, not because they have a divine Spirit, and have
seen in the heart of Thy servant what they speak, but because they be
proud; not knowing Moses' opinion, but loving their own, not because it
is truth, but because it is theirs. Otherwise they would equally love
another true opinion, as I love what they say, when they say true: not
because it is theirs, but because it is true; and on that very ground
not theirs because it is true. But if they therefore love it, because
it is true, then is it both theirs, and mine; as being in common to all
lovers of truth. But whereas they contend that Moses did not mean what
I say, but what they say, this I like not, love not: for though it
were so, yet that their rashness belongs not to knowledge, but to
overboldness, and not insight but vanity was its parent. And therefore,
O Lord, are Thy judgements terrible; seeing Thy truth is neither mine,
nor his, nor another's; but belonging to us all, whom Thou callest
publicly to partake of it, warning us terribly, not to account
it private to ourselves, lest we be deprived of it. For whosoever
challenges that as proper to himself, which Thou propoundest to all to
enjoy, and would have that his own which belongs to all, is driven from
what is in common to his own; that is, from truth, to a lie. For he that
speaketh a lie, speaketh it of his own.

Hearken, O God, Thou best judge; Truth Itself, hearken to what I shall
say to this gainsayer, hearken, for before Thee do I speak, and before
my brethren, who employ Thy law lawfully, to the end of charity:
hearken and behold, if it please Thee, what I shall say to him. For this
brotherly and peaceful word do I return unto Him: "If we both see that
to be true that Thou sayest, and both see that to be true that I say,
where, I pray Thee, do we see it? Neither I in thee, nor thou in me; but
both in the unchangeable Truth itself, which is above our souls." Seeing
then we strive not about the very light of the Lord God, why strive
we about the thoughts of our neighbour which we cannot so see, as the
unchangeable Truth is seen: for that, if Moses himself had appeared to
us and said, "This I meant"; neither so should we see it, but should
believe it. Let us not then be puffed up for one against another, above
that which is written: let us love the Lord our God with all our heart,
with all our soul, and with all our mind: and our neighbour as ourself.
With a view to which two precepts of charity, unless we believe that
Moses meant, whatsoever in those books he did mean, we shall make God
a liar, imagining otherwise of our fellow servant's mind, than he hath
taught us. Behold now, how foolish it is, in such abundance of most
true meanings, as may be extracted out of those words, rashly to affirm,
which of them Moses principally meant; and with pernicious contentions
to offend charity itself, for whose sake he spake every thing, whose
words we go about to expound.

And yet I, O my God, Thou lifter up of my humility, and rest of my
labour, Who hearest my confessions, and forgivest my sins: seeing Thou
commandest me to love my neighbour as myself, I cannot believe that Thou
gavest a less gift unto Moses Thy faithful servant, than I would wish
or desire Thee to have given me, had I been born in the time he was, and
hadst Thou set me in that office, that by the service of my heart and
tongue those books might be dispensed, which for so long after were to
profit all nations, and through the whole world from such an eminence of
authority, were to surmount all sayings of false and proud teachings. I
should have desired verily, had I then been Moses (for we all come from
the same lump, and what is man, saving that Thou art mindful of him?),
I would then, had I been then what he was, and been enjoined by Thee to
write the book of Genesis, have desired such a power of expression and
such a style to be given me, that neither they who cannot yet understand
how God created, might reject the sayings, as beyond their capacity; and
they who had attained thereto, might find what true opinion soever they
had by thought arrived at, not passed over in those few words of
that Thy servant: and should another man by the light of truth have
discovered another, neither should that fail of being discoverable in
those same words.

For as a fountain within a narrow compass, is more plentiful, and
supplies a tide for more streams over larger spaces, than any one of
those streams, which, after a wide interval, is derived from the same
fountain; so the relation of that dispenser of Thine, which was to
benefit many who were to discourse thereon, does out of a narrow
scantling of language, overflow into streams of clearest truth, whence
every man may draw out for himself such truth as he can upon these
subjects, one, one truth, another, another, by larger circumlocutions of
discourse. For some, when they read, or hear these words, conceive that
God like a man or some mass endued with unbounded power, by some new
and sudden resolution, did, exterior to itself, as it were at a certain
distance, create heaven and earth, two great bodies above and below,
wherein all things were to be contained. And when they hear, God said,
Let it be made, and it was made; they conceive of words begun and ended,
sounding in time, and passing away; after whose departure, that came
into being, which was commanded so to do; and whatever of the like sort,
men's acquaintance with the material world would suggest. In whom, being
yet little ones and carnal, while their weakness is by this humble
kind of speech, carried on, as in a mother's bosom, their faith is
wholesomely built up, whereby they hold assured, that God made all
natures, which in admirable variety their eye beholdeth around. Which
words, if any despising, as too simple, with a proud weakness, shall
stretch himself beyond the guardian nest; he will, alas, fall miserably.
Have pity, O Lord God, lest they who go by the way trample on the
unfledged bird, and send Thine angel to replace it into the nest, that
it may live, till it can fly.

But others, unto whom these words are no longer a nest, but deep shady
fruit-bowers, see the fruits concealed therein, fly joyously around,
and with cheerful notes seek out, and pluck them. For reading or hearing
these words, they see that all times past and to come, are surpassed by
Thy eternal and stable abiding; and yet that there is no creature formed
in time, not of Thy making. Whose will, because it is the same that Thou
art, Thou madest all things, not by any change of will, nor by a will,
which before was not, and that these things were not out of Thyself, in
Thine own likeness, which is the form of all things; but out of nothing,
a formless unlikeness, which should be formed by Thy likeness (recurring
to Thy Unity, according to their appointed capacity, so far as is given
to each thing in his kind), and might all be made very good; whether
they abide around Thee, or being in gradation removed in time and place,
made or undergo the beautiful variations of the Universe. These things
they see, and rejoice, in the little degree they here may, in the light
of Thy truth.

Another bends his mind on that which is said, In the Beginning God made
heaven and earth; and beholdeth therein Wisdom, the Beginning because
It also speaketh unto us. Another likewise bends his mind on the same
words, and by Beginning understands the commencement of things created;
In the beginning He made, as if it were said, He at first made. And
among them that understand In the Beginning to mean, "In Thy Wisdom Thou
createdst heaven and earth," one believes the matter out of which the
heaven and earth were to be created, to be there called heaven and
earth; another, natures already formed and distinguished; another, one
formed nature, and that a spiritual, under the name Heaven, the other
formless, a corporeal matter, under the name Earth. They again who by
the names heaven and earth, understand matter as yet formless, out of
which heaven and earth were to be formed, neither do they understand it
in one way; but the one, that matter out of which both the intelligible
and the sensible creature were to be perfected; another, that only, out
of which this sensible corporeal mass was to be made, containing in
its vast bosom these visible and ordinary natures. Neither do they, who
believe the creatures already ordered and arranged, to be in this place
called heaven and earth, understand the same; but the one, both the
invisible and visible, the other, the visible only, in which we behold
this lightsome heaven, and darksome earth, with the things in them
contained.

But he that no otherwise understands In the Beginning He made, than if
it were said, At first He made, can only truly understand heaven and
earth of the matter of heaven and earth, that is, of the universal
intelligible and corporeal creation. For if he would understand thereby
the universe, as already formed, it may be rightly demanded of him, "If
God made this first, what made He afterwards?" and after the universe,
he will find nothing; whereupon must he against his will hear another
question; "How did God make this first, if nothing after?" But when
he says, God made matter first formless, then formed, there is no
absurdity, if he be but qualified to discern, what precedes by eternity,
what by time, what by choice, and what in original. By eternity, as
God is before all things; by time, as the flower before the fruit; by
choice, as the fruit before the flower; by original, as the sound before
the tune. Of these four, the first and last mentioned, are with extreme
difficulty understood, the two middle, easily. For a rare and too lofty
a vision is it, to behold Thy Eternity, O Lord, unchangeably making
things changeable; and thereby before them. And who, again, is of
so sharp-sighted understanding, as to be able without great pains to
discern, how the sound is therefore before the tune; because a tune is
a formed sound; and a thing not formed, may exist; whereas that which
existeth not, cannot be formed. Thus is the matter before the thing
made; not because it maketh it, seeing itself is rather made; nor is it
before by interval of time; for we do not first in time utter formless
sounds without singing, and subsequently adapt or fashion them into
the form of a chant, as wood or silver, whereof a chest or vessel is
fashioned. For such materials do by time also precede the forms of the
things made of them, but in singing it is not so; for when it is sung,
its sound is heard; for there is not first a formless sound, which is
afterwards formed into a chant. For each sound, so soon as made, passeth
away, nor canst thou find ought to recall and by art to compose. So
then the chant is concentrated in its sound, which sound of his is his
matter. And this indeed is formed, that it may be a tune; and therefore
(as I said) the matter of the sound is before the form of the tune; not
before, through any power it hath to make it a tune; for a sound is no
way the workmaster of the tune; but is something corporeal, subjected to
the soul which singeth, whereof to make a tune. Nor is it first in time;
for it is given forth together with the tune; nor first in choice, for
a sound is not better than a tune, a tune being not only a sound, but
a beautiful sound. But it is first in original, because a tune receives
not form to become a sound, but a sound receives a form to become a
tune. By this example, let him that is able, understand how the matter
of things was first made, and called heaven and earth, because heaven
and earth were made out of it. Yet was it not made first in time;
because the forms of things give rise to time; but that was without
form, but now is, in time, an object of sense together with its form.
And yet nothing can be related of that matter, but as though prior in
time, whereas in value it is last (because things formed are superior
to things without form) and is preceded by the Eternity of the Creator:
that so there might be out of nothing, whereof somewhat might be
created.

In this diversity of the true opinions, let Truth herself produce
concord. And our God have mercy upon us, that we may use the law
lawfully, the end of the commandment, pure charity. By this if man
demands of me, "which of these was the meaning of Thy servant Moses";
this were not the language of my Confessions, should I not confess unto
Thee, "I know not"; and yet I know that those senses are true, those
carnal ones excepted, of which I have spoken what seemed necessary. And
even those hopeful little ones who so think, have this benefit, that the
words of Thy Book affright them not, delivering high things lowlily,
and with few words a copious meaning. And all we who, I confess, see and
express the truth delivered in those words, let us love one another, and
jointly love Thee our God, the fountain of truth, if we are athirst for
it, and not for vanities; yea, let us so honour this Thy servant, the
dispenser of this Scripture, full of Thy Spirit, as to believe that,
when by Thy revelation he wrote these things, he intended that, which
among them chiefly excels both for light of truth, and fruitfulness of
profit.

So when one says, "Moses meant as I do"; and another, "Nay, but as I
do," I suppose that I speak more reverently, "Why not rather as both,
if both be true?" And if there be a third, or a fourth, yea if any other
seeth any other truth in those words, why may not he be believed to
have seen all these, through whom the One God hath tempered the holy
Scriptures to the senses of many, who should see therein things true but
divers? For I certainly (and fearlessly I speak it from my heart), that
were I to indite any thing to have supreme authority, I should prefer
so to write, that whatever truth any could apprehend on those matters,
might be conveyed in my words, rather than set down my own meaning so
clearly as to exclude the rest, which not being false, could not offend
me. I will not therefore, O my God, be so rash, as not to believe, that
Thou vouchsafedst as much to that great man. He without doubt, when he
wrote those words, perceived and thought on what truth soever we have
been able to find, yea and whatsoever we have not been able, nor yet
are, but which may be found in them.

Lastly, O Lord, who art God and not flesh and blood, if man did see
less, could any thing be concealed from Thy good Spirit (who shall lead
me into the land of uprightness), which Thou Thyself by those words wert
about to reveal to readers in times to come, though he through whom
they were spoken, perhaps among many true meanings, thought on some one?
which if so it be, let that which he thought on be of all the highest.
But to us, O Lord, do Thou, either reveal that same, or any other true
one which Thou pleasest; that so, whether Thou discoverest the same to
us, as to that Thy servant, or some other by occasion of those words,
yet Thou mayest feed us, not error deceive us. Behold, O Lord my God,
how much we have written upon a few words, how much I beseech Thee! What
strength of ours, yea what ages would suffice for all Thy books in this
manner? Permit me then in these more briefly to confess unto Thee, and
to choose some one true, certain, and good sense that Thou shalt inspire
me, although many should occur, where many may occur; this being the law
my confession, that if I should say that which Thy minister intended,
that is right and best; for this should I endeavour, which if I should
not attain, yet I should say that, which Thy Truth willed by his words
to tell me, which revealed also unto him, what It willed.




\chapter{BOOK XIII}


\start{I}{ call} upon Thee, O my God, my mercy, Who createdst me, and forgottest
not me, forgetting Thee. I call Thee into my soul which, by the longing
Thyself inspirest into her, Thou preparest for Thee. Forsake me not now
calling upon Thee, whom Thou preventedst before I called, and urgedst me
with much variety of repeated calls, that I would hear Thee from afar,
and be converted, and call upon Thee, that calledst after me; for Thou,
Lord, blottedst out all my evil deservings, so as not to repay into my
hands, wherewith I fell from Thee; and Thou hast prevented all my well
deservings, so as to repay the work of Thy hands wherewith Thou madest
me; because before I was, Thou wert; nor was I any thing, to which
Thou mightest grant to be; and yet behold, I am, out of Thy goodness,
preventing all this which Thou hast made me, and whereof Thou hast made
me. For neither hadst Thou need of me, nor am I any such good, as to be
helpful unto Thee, my Lord and God; not in serving Thee, as though Thou
wouldest tire in working; or lest Thy power might be less, if lacking
my service: nor cultivating Thy service, as a land, that must remain
uncultivated, unless I cultivated Thee: but serving and worshipping
Thee, that I might receive a well-being from Thee, from whom it comes,
that I have a being capable of well-being.

For of the fulness of Thy goodness, doth Thy creature subsist, that so
a good, which could no ways profit Thee, nor was of Thee (lest so it
should be equal to Thee), might yet be since it could be made of Thee.
For what did heaven and earth, which Thou madest in the Beginning,
deserve of Thee? Let those spiritual and corporeal natures which Thou
madest in Thy Wisdom, say wherein they deserved of Thee, to depend
thereon (even in that their several inchoate and formless state, whether
spiritual or corporeal, ready to fall away into an immoderate liberty
and far-distant unlikeliness unto Thee;--the spiritual, though without
form, superior to the corporeal though formed, and the corporeal though
without form, better than were it altogether nothing), and so to depend
upon Thy Word, as formless, unless by the same Word they were brought
back to Thy Unity, indued with form and from Thee the One Sovereign
Good were made all very good. How did they deserve of Thee, to be even
without form, since they had not been even this, but from Thee?

How did corporeal matter deserve of Thee, to be even invisible and
without form? seeing it were not even this, but that Thou madest it, and
therefore because it was not, could not deserve of Thee to be made. Or
how could the inchoate spiritual creature deserve of Thee, even to ebb
and flow darksomely like the deep,--unlike Thee, unless it had been
by the same Word turned to that, by Whom it was created, and by Him so
enlightened, become light; though not equally, yet conformably to that
Form which is equal unto Thee? For as in a body, to be, is not one with
being beautiful, else could it not be deformed; so likewise to a created
spirit to live, is not one with living wisely; else should it be wise
unchangeably. But good it is for it always to hold fast to Thee; lest
what light it hath obtained by turning to Thee, it lose by turning
from Thee, and relapse into life resembling the darksome deep. For we
ourselves also, who as to the soul are a spiritual creature, turned away
from Thee our light, were in that life sometimes darkness; and still
labour amidst the relics of our darkness, until in Thy Only One we
become Thy righteousness, like the mountains of God. For we have been
Thy judgments, which are like the great deep.

That which Thou saidst in the beginning of the creation, Let there be
light, and there was light; I do, not unsuitably, understand of the
spiritual creature: because there was already a sort of life, which Thou
mightest illuminate. But as it had no claim on Thee for a life, which
could be enlightened, so neither now that it was, had it any, to be
enlightened. For neither could its formless estate be pleasing unto
Thee, unless it became light, and that not by existing simply, but by
beholding the illuminating light, and cleaving to it; so that, that it
lived, and lived happily, it owes to nothing but Thy grace, being turned
by a better change unto That which cannot be changed into worse or
better; which Thou alone art, because Thou alone simply art; unto
Thee it being not one thing to live, another to live blessedly, seeing
Thyself art Thine own Blessedness.

What then could be wanting unto Thy good, which Thou Thyself art,
although these things had either never been, or remained without form;
which thou madest, not out of any want, but out of the fulness of Thy
goodness, restraining them and converting them to form, not as though
Thy joy were fulfilled by them? For to Thee being perfect, is their
imperfection displeasing, and hence were they perfected by Thee, and
please Thee; not as wert Thou imperfect, and by their perfecting wert
also to be perfected. For Thy good Spirit indeed was borne over the
waters, not borne up by them, as if He rested upon them. For those, on
whom Thy good Spirit is said to rest, He causes to rest in Himself. But
Thy incorruptible and unchangeable will, in itself all-sufficient for
itself, was borne upon that life which Thou hadst created; to which,
living is not one with happy living, seeing it liveth also, ebbing and
flowing in its own darkness: for which it remaineth to be converted unto
Him, by Whom it was made, and to live more and more by the fountain
of life, and in His light to see light, and to be perfected, and
enlightened, and beautified.

Lo, now the Trinity appears unto me in a glass darkly, which is Thou my
God, because Thou, O Father, in Him Who is the Beginning of our wisdom,
Which is Thy Wisdom, born of Thyself, equal unto Thee and coeternal,
that is, in Thy Son, createdst heaven and earth. Much now have we said
of the Heaven of heavens, and of the earth invisible and without form,
and of the darksome deep, in reference to the wandering instability of
its spiritual deformity, unless it had been converted unto Him, from
Whom it had its then degree of life, and by His enlightening became a
beauteous life, and the heaven of that heaven, which was afterwards
set between water and water. And under the name of God, I now held the
Father, who made these things, and under the name of Beginning, the Son,
in whom He made these things; and believing, as I did, my God as the
Trinity, I searched further in His holy words, and lo, Thy Spirit moved
upon the waters. Behold the Trinity, my God, Father, and Son, and Holy
Ghost, Creator of all creation.

But what was the cause, O true-speaking Light?--unto Thee lift I up my
heart, let it not teach me vanities, dispel its darkness; and tell me, I
beseech Thee, by our mother charity, tell me the reason, I beseech Thee,
why after the mention of heaven, and of the earth invisible and without
form, and darkness upon the deep, Thy Scripture should then at length
mention Thy Spirit? Was it because it was meet that the knowledge of Him
should be conveyed, as being "borne above"; and this could not be
said, unless that were first mentioned, over which Thy Spirit may
be understood to have been borne. For neither was He borne above the
Father, nor the Son, nor could He rightly be said to be borne above, if
He were borne over nothing. First then was that to be spoken of, over
which He might be borne; and then He, whom it was meet not otherwise to
be spoken of than as being borne. But wherefore was it not meet that
the knowledge of Him should be conveyed otherwise, than as being borne
above?

Hence let him that is able, follow with his understanding Thy Apostle,
where he thus speaks, Because Thy love is shed abroad in our hearts by
the Holy Ghost which is given unto us: and where concerning spiritual
gifts, he teacheth and showeth unto us a more excellent way of charity;
and where he bows his knee unto Thee for us, that we may know the
supereminent knowledge of the love of Christ. And therefore from the
beginning, was He borne supereminent above the waters. To whom shall I
speak this? how speak of the weight of evil desires, downwards to the
steep abyss; and how charity raises up again by Thy Spirit which was
borne above the waters? to whom shall I speak it? how speak it? For it
is not in space that we are merged and emerge. What can be more, and yet
what less like? They be affections, they be loves; the uncleanness
of our spirit flowing away downwards with the love of cares, and the
holiness of Thine raising us upward by love of unanxious repose; that
we may lift our hearts unto Thee, where Thy Spirit is borne above the
waters; and come to that supereminent repose, when our soul shall have
passed through the waters which yield no support.

Angels fell away, man's soul fell away, and thereby pointed the abyss in
that dark depth, ready for the whole spiritual creation, hadst not Thou
said from the beginning, Let there be light, and there had been light,
and every obedient intelligence of Thy heavenly City had cleaved to
Thee, and rested in Thy Spirit, Which is borne unchangeably over every
thing changeable. Otherwise, had even the heaven of heavens been in
itself a darksome deep; but now it is light in the Lord. For even in
that miserable restlessness of the spirits, who fell away and discovered
their own darkness, when bared of the clothing of Thy light, dost Thou
sufficiently reveal how noble Thou madest the reasonable creature; to
which nothing will suffice to yield a happy rest, less than Thee; and so
not even herself. For Thou, O our God, shalt lighten our darkness: from
Thee riseth our garment of light; and then shall our darkness be as
the noon day. Give Thyself unto me, O my God, restore Thyself unto me:
behold I love, and if it be too little, I would love more strongly. I
cannot measure so as to know, how much love there yet lacketh to me, ere
my life may run into Thy embracements, nor turn away, until it be hidden
in the hidden place of Thy Presence. This only I know, that woe is
me except in Thee: not only without but within myself also; and all
abundance, which is not my God, is emptiness to me.

But was not either the Father, or the Son, borne above the waters? if
this means, in space, like a body, then neither was the Holy Spirit;
but if the unchangeable supereminence of Divinity above all things
changeable, then were both Father, and Son, and Holy Ghost borne upon
the waters. Why then is this said of Thy Spirit only, why is it said
only of Him? As if He had been in place, Who is not in place, of Whom
only it is written, that He is Thy gift? In Thy Gift we rest; there we
enjoy Thee. Our rest is our place. Love lifts us up thither, and Thy
good Spirit lifts up our lowliness from the gates of death. In Thy good
pleasure is our peace. The body by its own weight strives towards its
own place. Weight makes not downward only, but to his own place. Fire
tends upward, a stone downward. They are urged by their own weight,
they seek their own places. Oil poured below water, is raised above the
water; water poured upon oil, sinks below the oil. They are urged by
their own weights to seek their own places. When out of their order,
they are restless; restored to order, they are at rest. My weight, is my
love; thereby am I borne, whithersoever I am borne. We are inflamed, by
Thy Gift we are kindled; and are carried upwards; we glow inwardly, and
go forwards. We ascend Thy ways that be in our heart, and sing a song of
degrees; we glow inwardly with Thy fire, with Thy good fire, and we go;
because we go upwards to the peace of Jerusalem: for gladdened was I in
those who said unto me, We will go up to the house of the Lord. There
hath Thy good pleasure placed us, that we may desire nothing else, but
to abide there for ever.

Blessed creature, which being itself other than Thou, has known no
other condition, than that, so soon as it was made, it was, without
any interval, by Thy Gift, Which is borne above every thing changeable,
borne aloft by that calling whereby Thou saidst, Let there be light, and
there was light. Whereas in us this took place at different times, in
that we were darkness, and are made light: but of that is only said,
what it would have been, had it not been enlightened. And, this is so
spoken, as if it had been unsettled and darksome before; that so the
cause whereby it was made otherwise, might appear, namely, that being
turned to the Light unfailing it became light. Whoso can, let him
understand this; let him ask of Thee. Why should he trouble me, as if I
could enlighten any man that cometh into this world?

Which of us comprehendeth the Almighty Trinity? and yet which speaks not
of It, if indeed it be It? Rare is the soul, which while it speaks of
It, knows what it speaks of. And they contend and strive, yet, without
peace, no man sees that vision. I would that men would consider these
three, that are in themselves. These three be indeed far other than the
Trinity: I do but tell, where they may practise themselves, and there
prove and feel how far they be. Now the three I spake of are, To Be,
to Know, and to Will. For I Am, and Know, and Will: I Am Knowing and
Willing: and I Know myself to Be, and to Will: and I Will to Be, and to
Know. In these three then, let him discern that can, how inseparable
a life there is, yea one life, mind, and one essence, yea lastly how
inseparable a distinction there is, and yet a distinction. Surely a man
hath it before him; let him look into himself, and see, and tell me. But
when he discovers and can say any thing of these, let him not therefore
think that he has found that which is above these Unchangeable, which
Is unchangeably, and Knows unchangeably, and Wills unchangeably; and
whether because of these three, there is in God also a Trinity, or
whether all three be in Each, so that the three belong to Each; or
whether both ways at once, wondrously, simply and yet manifoldly, Itself
a bound unto Itself within Itself, yet unbounded; whereby It is, and is
Known unto Itself and sufficeth to itself, unchangeably the Self-same,
by the abundant greatness of its Unity,--who can readily conceive this?
who could any ways express it? who would, any way, pronounce thereon
rashly?

Proceed in thy confession, say to the Lord thy God, O my faith, Holy,
Holy, Holy, O Lord my God, in Thy Name have we been baptised, Father,
Son, and Holy Ghost; in Thy Name do we baptise, Father, Son, and Holy
Ghost, because among us also, in His Christ did God make heaven and
earth, namely, the spiritual and carnal people of His Church. Yea and
our earth, before it received the form of doctrine, was invisible and
without form; and we were covered with the darkness of ignorance. For
Thou chastenedst man for iniquity, and Thy judgments were like the great
deep unto him. But because Thy Spirit was borne above the waters, Thy
mercy forsook not our misery, and Thou saidst, Let there be light,
Repent ye, for the kingdom of heaven is at hand. Repent ye, let there be
light. And because our soul was troubled within us, we remembered Thee,
O Lord, from the land of Jordan, and that mountain equal unto Thyself,
but little for our sakes: and our darkness displeased us, we turned unto
Thee and there was light. And, behold, we were sometimes darkness, but
now light in the Lord.

But as yet by faith and not by sight, for by hope we are saved; but hope
that is seen, is not hope. As yet doth deep call unto deep, but now in
the voice of Thy water-spouts. As yet doth he that saith, I could not
speak unto you as unto spiritual, but as unto carnal, even he as yet,
doth not think himself to have apprehended, and forgetteth those things
which are behind, and reacheth forth to those which are before, and
groaneth being burthened, and his soul thirsteth after the Living
God, as the hart after the water-brooks, and saith, When shall I come?
desiring to be clothed upon with his house which is from heaven, and
calleth upon this lower deep, saying, Be not conformed to this world,
but be ye transformed by the renewing of your mind. And, be not children
in understanding, but in malice, be ye children, that in understanding
ye may be perfect; and O foolish Galatians, who hath bewitched you? But
now no longer in his own voice; but in Thine who sentest Thy Spirit from
above; through Him who ascended up on high, and set open the flood-gates
of His gifts, that the force of His streams might make glad the city of
God. Him doth this friend of the Bridegroom sigh after, having now the
first-fruits of the Spirit laid up with Him, yet still groaning within
himself, waiting for the adoption, to wit, the redemption of his body;
to Him he sighs, a member of the Bride; for Him he is jealous, as being
a friend of the Bridegroom; for Him he is jealous, not for himself;
because in the voice of Thy water-spouts, not in his own voice, doth he
call to that other depth, over whom being jealous he feareth, lest as
the serpent beguiled Eve through his subtilty, so their minds should be
corrupted from the purity that is in our Bridegroom Thy only Son. O
what a light of beauty will that be, when we shall see Him as He is,
and those tears be passed away, which have been my meat day and night,
whilst they daily say unto me, Where is now Thy God?

Behold, I too say, O my God, Where art Thou? see, where Thou art! in
Thee I breathe a little, when I pour out my soul by myself in the voice
of joy and praise, the sound of him that keeps holy-day. And yet again
it is sad, because it relapseth, and becomes a deep, or rather perceives
itself still to be a deep. Unto it speaks my faith which Thou hast
kindled to enlighten my feet in the night, Why art thou sad, O my soul,
and why dost thou trouble me? Hope in the Lord; His word is a lanthorn
unto thy feet: hope and endure, until the night, the mother of the
wicked, until the wrath of the Lord, be overpast, whereof we also were
once children, who were sometimes darkness, relics whereof we bear
about us in our body, dead because of sin; until the day break, and the
shadows fly away. Hope thou in the Lord; in the morning I shall stand in
Thy presence, and contemplate Thee: I shall for ever confess unto Thee.
In the morning I shall stand in Thy presence, and shall see the health
of my countenance, my God, who also shall quicken our mortal bodies, by
the Spirit that dwelleth in us, because He hath in mercy been borne
over our inner darksome and floating deep: from Whom we have in this
pilgrimage received an earnest, that we should now be light: whilst we
are saved by hope, and are the children of light, and the children of
the day, not the children of the night, nor of the darkness, which yet
sometimes we were. Betwixt whom and us, in this uncertainty of human
knowledge, Thou only dividest; Thou, who provest our hearts, and callest
the light, day, and the darkness, night. For who discerneth us, but
Thou? And what have we, that we have not received of Thee? out of the
same lump vessels are made unto honour, whereof others also are made
unto dishonour.

Or who, except Thou, our God, made for us that firmament of authority
over us in Thy Divine Scripture? as it is said, For heaven shall be
folded up like a scroll; and now is it stretched over us like a skin.
For Thy Divine Scripture is of more eminent authority, since those
mortals by whom Thou dispensest it unto us, underwent mortality. And
Thou knowest, Lord, Thou knowest, how Thou with skins didst clothe men,
when they by sin became mortal. Whence Thou hast like a skin stretched
out the firmament of Thy book, that is, Thy harmonizing words, which
by the ministry of mortal men Thou spreadest over us. For by their very
death was that solid firmament of authority, in Thy discourses set forth
by them, more eminently extended over all that be under it; which whilst
they lived here, was not so eminently extended. Thou hadst not as yet
spread abroad the heaven like a skin; Thou hadst not as yet enlarged in
all directions the glory of their deaths.

Let us look, O Lord, upon the heavens, the work of Thy fingers; clear
from our eyes that cloud, which Thou hast spread under them. There is
Thy testimony, which giveth wisdom unto the little ones: perfect, O my
God, Thy praise out of the mouth of babes and sucklings. For we know no
other books, which so destroy pride, which so destroy the enemy and the
defender, who resisteth Thy reconciliation by defending his own sins. I
know not, Lord, I know not any other such pure words, which so persuade
me to confess, and make my neck pliant to Thy yoke, and invite me to
serve Thee for nought. Let me understand them, good Father: grant this
to me, who am placed under them: because for those placed under them,
hast Thou established them.

Other waters there be above this firmament, I believe immortal, and
separated from earthly corruption. Let them praise Thy Name, let them
praise Thee, the supercelestial people, Thine angels, who have no need
to gaze up at this firmament, or by reading to know of Thy Word. For
they always behold Thy face, and there read without any syllables in
time, what willeth Thy eternal will; they read, they choose, they love.
They are ever reading; and that never passes away which they read; for
by choosing, and by loving, they read the very unchangeableness of Thy
counsel. Their book is never closed, nor their scroll folded up; seeing
Thou Thyself art this to them, and art eternally; because Thou hast
ordained them above this firmament, which Thou hast firmly settled over
the infirmity of the lower people, where they might gaze up and learn
Thy mercy, announcing in time Thee Who madest times. For Thy mercy, O
Lord, is in the heavens, and Thy truth reacheth unto the clouds. The
clouds pass away, but the heaven abideth. The preachers of Thy word pass
out of this life into another; but Thy Scripture is spread abroad over
the people, even unto the end of the world. Yet heaven and earth also
shall pass away, but Thy words shall not pass away. Because the scroll
shall be rolled together: and the grass over which it was spread, shall
with the goodliness of it pass away; but Thy Word remaineth for ever,
which now appeareth unto us under the dark image of the clouds, and
through the glass of the heavens, not as it is: because we also, though
the well-beloved of Thy Son, yet it hath not yet appeared what we
shall be. He looketh through the lattice of our flesh, and He spake us
tenderly, and kindled us, and we ran after His odours. But when He shall
appear, then shall we be like Him, for we shall see Him as He is. As He
is, Lord, will our sight be.

For altogether, as Thou art, Thou only knowest; Who art unchangeably,
and knowest unchangeably, and willest unchangeably. And Thy Essence
Knoweth, and Willeth unchangeably; and Thy Knowledge Is, and Willeth
unchangeably; and Thy Will Is, and Knoweth unchangeably. Nor seemeth it
right in Thine eyes, that as the Unchangeable Light knoweth Itself, so
should it be known by the thing enlightened, and changeable. Therefore
is my soul like a land where no water is, because as it cannot of itself
enlighten itself, so can it not of itself satisfy itself. For so is the
fountain of life with Thee, like as in Thy light we shall see light.

Who gathered the embittered together into one society? For they have all
one end, a temporal and earthly felicity, for attaining whereof they do
all things, though they waver up and down with an innumerable variety of
cares. Who, Lord, but Thou, saidst, Let the waters be gathered together
into one place, and let the dry land appear, which thirsteth after Thee?
For the sea also is Thine, and Thou hast made it, and Thy hands prepared
the dry land. Nor is the bitterness of men's wills, but the gathering
together of the waters, called sea; for Thou restrainest the wicked
desires of men's souls, and settest them their bounds, how far they may
be allowed to pass, that their waves may break one against another: and
thus makest Thou it a sea, by the order of Thy dominion over all things.

But the souls that thirst after Thee, and that appear before Thee (being
by other bounds divided from the society of the sea), Thou waterest by
a sweet spring, that the earth may bring forth her fruit, and Thou, Lord
God, so commanding, our soul may bud forth works of mercy according
to their kind, loving our neighbour in the relief of his bodily
necessities, having seed in itself according to its likeness, when from
feeling of our infirmity, we compassionate so as to relieve the needy;
helping them, as we would be helped; if we were in like need; not only
in things easy, as in herb yielding seed, but also in the protection of
our assistance, with our best strength, like the tree yielding fruit:
that is, well-doing in rescuing him that suffers wrong, from the hand
of the powerful, and giving him the shelter of protection, by the mighty
strength of just judgment.

So, Lord, so, I beseech Thee, let there spring up, as Thou doest, as
Thou givest cheerfulness and ability, let truth spring out of the earth,
and righteousness look down from heaven, and let there be lights in the
firmament. Let us break our bread to the hungry, and bring the houseless
poor to our house. Let us clothe the naked, and despise not those of our
own flesh. Which fruits having sprung out of the earth, see it is good:
and let our temporary light break forth; and ourselves, from this lower
fruitfulness of action, arriving at the delightfulness of contemplation,
obtaining the Word of Life above, appear like lights in the world,
cleaving to the firmament of Thy Scripture. For there Thou instructest
us, to divide between the things intellectual, and things of sense, as
betwixt the day and the night; or between souls, given either to things
intellectual, or things of sense, so that now not Thou only in the
secret of Thy judgment, as before the firmament was made, dividest
between the light and the darkness, but Thy spiritual children also
set and ranked in the same firmament (now that Thy grace is laid open
throughout the world), may give light upon the earth, and divide betwixt
the day and the night, and be for signs of times, that old things
are passed away, and, behold, all things are become new; and that our
salvation is nearer than when we believed: and that the night is far
spent, and the day is at hand: and that Thou wilt crown Thy year with
blessing, sending the labourers of Thy goodness into Thy harvest, in
sowing whereof, others have laboured, sending also into another field,
whose harvest shall be in the end. Thus grantest Thou the prayers of him
that asketh, and blessest the years of the just; but Thou art the same,
and in Thy years which fail not, Thou preparest a garner for our passing
years. For Thou by an eternal counsel dost in their proper seasons
bestow heavenly blessings upon the earth. For to one is given by the
Spirit the word of wisdom, as it were the lesser light: to another
faith; to another the gift with the light of perspicuous truth, as it
were for the rule of the day. To another the word of knowledge by the
same Spirit, as it were the lesser light: to another faith; to another
the gift of healing; to another the working of miracles; to another
prophecy; to another discerning of spirits; to another divers kinds of
tongues. And all these as it were stars. For all these worketh the one
and self-same spirit, dividing to every man his own as He will; and
causing stars to appear manifestly, to profit withal. But the word of
knowledge, wherein are contained all Sacraments, which are varied in
their seasons as it were the moon, and those other notices of gifts,
which are reckoned up in order, as it were stars, inasmuch as they come
short of that brightness of wisdom, which gladdens the forementioned
day, are only for the rule of the night. For they are necessary to such,
as that Thy most prudent servant could not speak unto as unto spiritual,
but as unto carnal; even he, who speaketh wisdom among those that are
perfect. But the natural man, as it were a babe in Christ and fed on
milk, until he be strengthened for solid meat and his eye be enabled to
behold the Sun, let him not dwell in a night forsaken of all light, but
be content with the light of the moon and the stars. So dost Thou speak
to us, our All-wise God, in Thy Book, Thy firmament; that we may discern
all things, in an admirable contemplation; though as yet in signs and in
times, and in days, and in years.

But first, wash you, be clean; put away evil from your souls, and from
before mine eyes, that the dry land may appear. Learn to do good, judge
the fatherless, plead for the widow, that the earth may bring forth the
green herb for meat, and the tree bearing fruit; and come, let us reason
together, saith the Lord, that there may be lights in the firmament of
the heaven, and they may shine upon the earth. That rich man asked of
the good Master, what he should do to attain eternal life. Let the
good Master tell him (whom he thought no more than man; but He is good
because He is God), let Him tell him, if he would enter into life, he
must keep the commandments: let him put away from him the bitterness
of malice and wickedness; not kill, not commit adultery, not steal, not
bear false witness; that the dry land may appear, and bring forth the
honouring of father and mother, and the love of our neighbour. All these
(saith he) have I kept. Whence then so many thorns, if the earth be
fruitful? Go, root up the spreading thickets of covetousness; sell that
thou hast, and be filled with fruit, by giving to the poor, and thou
shalt have treasure in heaven; and follow the Lord if thou wilt be
perfect, associated with them, among whom He speaketh wisdom, Who
knoweth what to distribute to the day, and to the night, that thou also
mayest know it, and for thee there may be lights in the firmament of
heaven; which will not be, unless thy heart be there: nor will that
either be, unless there thy treasure be; as thou hast heard of the good
Master. But that barren earth was grieved; and the thorns choked the
word.

But you, chosen generation, you weak things of the world, who have
forsaken all, that ye may follow the Lord; go after Him, and confound
the mighty; go after Him, ye beautiful feet, and shine ye in the
firmament, that the heavens may declare His glory, dividing between the
light of the perfect, though not as the angels, and the darkness of the
little ones, though not despised. Shine over the earth; and let the
day, lightened by the sun, utter unto day, speech of wisdom; and night,
shining with the moon, show unto night, the word of knowledge. The moon
and stars shine for the night; yet doth not the night obscure them,
seeing they give it light in its degree. For behold God saying, as
it were, Let there be lights in the firmament of heaven; there came
suddenly a sound from heaven, as it had been the rushing of a mighty
wind, and there appeared cloven tongues like as of fire, and it sat upon
each of them. And there were made lights in the firmament of heaven,
having the word of life. Run ye to and fro every where, ye holy fires,
ye beauteous fires; for ye are the light of the world, nor are ye put
under a bushel; He whom you cleave unto, is exalted, and hath exalted
you. Run ye to and fro, and be known unto all nations.

Let the sea also conceive and bring forth your works; and let the waters
bring forth the moving creature that hath life. For ye, separating the
precious from the vile, are made the mouth of God, by whom He saith, Let
the waters bring forth, not the living creature which the earth brings
forth, but the moving creature having life, and the fowls that fly above
the earth. For Thy Sacraments, O God, by the ministry of Thy holy ones,
have moved amid the waves of temptations of the world, to hallow the
Gentiles in Thy Name, in Thy Baptism. And amid these things, many great
wonders were wrought, as it were great whales: and the voices of Thy
messengers flying above the earth, in the open firmament of Thy Book;
that being set over them, as their authority under which they were to
fly, whithersoever they went. For there is no speech nor language, where
their voice is not heard: seeing their sound is gone through all the
earth, and their words to the end of the world, because Thou, Lord,
multipliedst them by blessing.

Speak I untruly, or do I mingle and confound, and not distinguish
between the lucid knowledge of these things in the firmament of heaven,
and the material works in the wavy sea, and under the firmament of
heaven? For of those things whereof the knowledge is substantial and
defined, without any increase by generation, as it were lights of wisdom
and knowledge, yet even of them, the material operations are many and
divers; and one thing growing out of another, they are multiplied by Thy
blessing, O God, who hast refreshed the fastidiousness of mortal senses;
that so one thing in the understanding of our mind, may, by the motions
of the body, be many ways set out, and expressed. These Sacraments have
the waters brought forth; but in Thy word. The necessities of the people
estranged from the eternity of Thy truth, have brought them forth,
but in Thy Gospel; because the waters themselves cast them forth, the
diseased bitterness whereof was the cause, why they were sent forth in
Thy Word.

Now are all things fair that Thou hast made; but behold, Thyself art
unutterably fairer, that madest all; from whom had not Adam fallen, the
brackishness of the sea had never flowed out of him, that is, the human
race so profoundly curious, and tempestuously swelling, and restlessly
tumbling up and down; and then had there been no need of Thy dispensers
to work in many waters, after a corporeal and sensible manner,
mysterious doings and sayings. For such those moving and flying
creatures now seem to me to mean, whereby people being initiated and
consecrated by corporeal Sacraments, should not further profit, unless
their soul had a spiritual life, and unless after the word of admission,
it looked forwards to perfection.

And hereby, in Thy Word, not the deepness of the sea, but the earth
separated from the bitterness of the waters, brings forth, not the
moving creature that hath life, but the living soul. For now hath it no
more need of baptism, as the heathen have, and as itself had, when it
was covered with the waters; (for no other entrance is there into the
kingdom of heaven, since Thou hast appointed that this should be the
entrance): nor does it seek after wonderfulness of miracles to work
belief; for it is not such, that unless it sees signs and wonders, it
will not believe, now that the faithful earth is separated from the
waters that were bitter with infidelity; and tongues are for a sign, not
to them that believe, but to them that believe not. Neither then does
that earth which Thou hast founded upon the waters, need that flying
kind, which at Thy word the waters brought forth. Send Thou Thy word
into it by Thy messengers: for we speak of their working, yet it is Thou
that workest in them that they may work out a living soul in it. The
earth brings it forth, because the earth is the cause that they work
this in the soul; as the sea was the cause that they wrought upon
the moving creatures that have life, and the fowls that fly under the
firmament of heaven, of whom the earth hath no need; although it feeds
upon that fish which was taken out of the deep, upon that table which
Thou hast prepared in the presence of them that believe. For therefore
was He taken out of the deep, that He might feed the dry land; and the
fowl, though bred in the sea, is yet multiplied upon the earth. For of
the first preachings of the Evangelists, man's infidelity was the cause;
yet are the faithful also exhorted and blessed by them manifoldly, from
day to day. But the living soul takes his beginning from the earth: for
it profits only those already among the Faithful, to contain themselves
from the love of this world, that so their soul may live unto Thee,
which was dead while it lived in pleasures; in death-bringing pleasures,
Lord, for Thou, Lord, art the life-giving delight of the pure heart.

Now then let Thy ministers work upon the earth,--not as upon the waters
of infidelity, by preaching and speaking by miracles, and Sacraments,
and mystic words; wherein ignorance, the mother of admiration, might
be intent upon them, out of a reverence towards those secret signs. For
such is the entrance unto the Faith for the sons of Adam forgetful of
Thee, while they hide themselves from Thy face, and become a darksome
deep. But--let Thy ministers work now as on the dry land, separated from
the whirlpools of the great deep: and let them be a pattern unto the
Faithful, by living before them, and stirring them up to imitation. For
thus do men hear, so as not to hear only, but to do also. Seek the Lord,
and your soul shall live, that the earth may bring forth the living
soul. Be not conformed to the world. Contain yourselves from it: the
soul lives by avoiding what it dies by affecting. Contain yourselves
from the ungoverned wildness of pride, the sluggish voluptuousness of
luxury, and the false name of knowledge: that so the wild beasts may be
tamed, the cattle broken to the yoke, the serpents, harmless. For
these be the motions of our mind under an allegory; that is to say, the
haughtiness of pride, the delight of lust, and the poison of curiosity,
are the motions of a dead soul; for the soul dies not so as to lose all
motion; because it dies by forsaking the fountain of life, and so is
taken up by this transitory world, and is conformed unto it.

But Thy word, O God, is the fountain of life eternal; and passeth not
away: wherefore this departure of the soul is restrained by Thy word,
when it is said unto us, Be not conformed unto this world; that so the
earth may in the fountain of life bring forth a living soul; that is,
a soul made continent in Thy Word, by Thy Evangelists, by following the
followers of Thy Christ. For this is after his kind; because a man is
wont to imitate his friend. Be ye (saith he) as I am, for I also am
as you are. Thus in this living soul shall there be good beasts, in
meekness of action (for Thou hast commanded, Go on with thy business in
meekness, so shalt thou be beloved by all men); and good cattle, which
neither if they eat, shall they over-abound, nor, if they eat not, have
any lack; and good serpents, not dangerous, to do hurt, but wise to take
heed; and only making so much search into this temporal nature, as may
suffice that eternity be clearly seen, being understood by the things
that are made. For these creatures are obedient unto reason, when being
restrained from deadly prevailing upon us, they live, and are good.

For behold, O Lord, our God, our Creator, when our affections have
been restrained from the love of the world, by which we died through
evil-living; and begun to be a living soul, through good living; and
Thy word which Thou spokest by Thy apostle, is made good in us, Be not
conformed to this world: there follows that also, which Thou presently
subjoinedst, saying, But be ye transformed by the renewing of your mind;
not now after your kind, as though following your neighbour who went
before you, nor as living after the example of some better man (for Thou
saidst not, "Let man be made after his kind," but, Let us make man after
our own image and similitude), that we might prove what Thy will is. For
to this purpose said that dispenser of Thine (who begat children by the
Gospel), that he might not for ever have them babes, whom he must be
fain to feed with milk, and cherish as a nurse; be ye transformed (saith
he) by the renewing of your mind, that ye may prove what is that good
and acceptable and perfect will of God. Wherefore Thou sayest not, "Let
man be made," but Let us make man. Nor saidst Thou, "according to his
kind"; but, after our image and likeness. For man being renewed in his
mind, and beholding and understanding Thy truth, needs not man as his
director, so as to follow after his kind; but by Thy direction proveth
what is that good, that acceptable, and perfect will of Thine: yea, Thou
teachest him, now made capable, to discern the Trinity of the Unity, and
the Unity of the Trinity. Wherefore to that said in the plural, Let us
make man, is yet subjoined in the singular, And God made man: and
to that said in the plural, After our likeness, is subjoined in the
singular, After the image of God. Thus is man renewed in the knowledge
of God, after the image of Him that created him: and being made
spiritual, he judgeth all things (all things which are to be judged),
yet himself is judged of no man.

But that he judgeth all things, this answers to his having dominion over
the fish of the sea, and over the fowls of the air, and over all cattle
and wild beasts, and over all the earth, and over every creeping thing
that creepeth upon the earth. For this he doth by the understanding of
his mind, whereby he perceiveth the things of the Spirit of God; whereas
otherwise, man being placed in honour, had no understanding, and is
compared unto the brute beasts, and is become like unto them. In Thy
Church therefore, O our God, according to Thy grace which Thou hast
bestowed upon it (for we are Thy workmanship created unto good
works), not those only who are spiritually set over, but they also who
spiritually are subject to those that are set over them,--for in this
way didst Thou make man male and female, in Thy grace spiritual, where,
according to the sex of body, there is neither male nor female, because
neither Jew nor Grecian, neither bond nor free.--Spiritual persons
(whether such as are set over, or such as obey); do judge spiritually;
not of that spiritual knowledge which shines in the firmament (for they
ought not to judge as to so supreme authority), nor may they judge of
Thy Book itself, even though something there shineth not clearly; for we
submit our understanding unto it, and hold for certain, that even what
is closed to our sight, is yet rightly and truly spoken. For so man,
though now spiritual and renewed in the knowledge of God after His image
that created him, ought to be a doer of the law, not a judge. Neither
doth he judge of that distinction of spiritual and carnal men, who
are known unto Thine eyes, O our God, and have not as yet discovered
themselves unto us by works, that by their fruits we might know them:
but Thou, Lord, dost even now know them, and hast divided and called
them in secret, or ever the firmament was made. Nor doth he, though
spiritual, judge the unquiet people of this world; for what hath he
to do, to judge them that are without, knowing not which of them shall
hereafter come into the sweetness of Thy grace; and which continue in
the perpetual bitterness of ungodliness?

Man therefore, whom Thou hast made after Thine own image, received not
dominion over the lights of heaven, nor over that hidden heaven
itself, nor over the day and the night, which Thou calledst before the
foundation of the heaven, nor over the gathering together of the waters,
which is the sea; but He received dominion over the fishes of the sea,
and the fowls of the air, and over all cattle, and over all the earth,
and over all creeping things which creep upon the earth. For He judgeth
and approveth what He findeth right, and He disalloweth what He findeth
amiss, whether in the celebration of those Sacraments by which such
are initiated, as Thy mercy searches out in many waters: or in that, in
which that Fish is set forth, which, taken out of the deep, the devout
earth feedeth upon: or in the expressions and signs of words, subject to
the authority of Thy Book,--such signs, as proceed out of the mouth,
and sound forth, flying as it were under the firmament, by interpreting,
expounding, discoursing disputing, consecrating, or praying unto Thee,
so that the people may answer, Amen. The vocal pronouncing of all which
words, is occasioned by the deep of this world, and the blindness of the
flesh, which cannot see thoughts; So that there is need to speak aloud
into the ears; so that, although flying fowls be multiplied upon the
earth, yet they derive their beginning from the waters. The spiritual
man judgeth also by allowing of what is right, and disallowing what he
finds amiss, in the works and lives of the faithful; their alms, as it
were the earth bringing forth fruit, and of the living soul, living
by the taming of the affections, in chastity, in fasting, in holy
meditations; and of those things, which are perceived by the senses of
the body. Upon all these is he now said to judge, wherein he hath also
power of correction.

But what is this, and what kind of mystery? Behold, Thou blessest
mankind, O Lord, that they may increase and multiply, and replenish the
earth; dost Thou not thereby give us a hint to understand something? why
didst Thou not as well bless the light, which Thou calledst day; nor the
firmament of heaven, nor the lights, nor the stars, nor the earth, nor
the sea? I might say that Thou, O God, who created created us after
Thine Image, I might say, that it had been Thy good pleasure to bestow
this blessing peculiarly upon man; hadst Thou not in like manner blessed
the fishes and the whales, that they should increase and multiply, and
replenish the waters of the sea, and that the fowls should be multiplied
upon the earth. I might say likewise, that this blessing pertained
properly unto such creatures, as are bred of their own kind, had I found
it given to the fruit-trees, and plants, and beasts of the earth. But
now neither unto the herbs, nor the trees, nor the beasts, nor serpents
is it said, Increase and multiply; notwithstanding all these as well as
the fishes, fowls, or men, do by generation increase and continue their
kind.

What then shall I say, O Truth my Light? "that it was idly said, and
without meaning?" Not so, O Father of piety, far be it from a minister
of Thy word to say so. And if I understand not what Thou meanest by
that phrase, let my betters, that is, those of more understanding than
myself, make better use of it, according as Thou, my God, hast given to
each man to understand. But let my confession also be pleasing in Thine
eyes, wherein I confess unto Thee, that I believe, O Lord, that Thou
spokest not so in vain; nor will I suppress, what this lesson suggests
to me. For it is true, nor do I see what should hinder me from thus
understanding the figurative sayings of Thy Bible. For I know a thing
to be manifoldly signified by corporeal expressions, which is understood
one way by the mind; and that understood many ways in the mind, which
is signified one way by corporeal expression. Behold, the single love
of God and our neighbour, by what manifold sacraments, and innumerable
languages, and in each several language, in how innumerable modes of
speaking, it is corporeally expressed. Thus do the offspring of the
waters increase and multiply. Observe again, whosoever readest this;
behold, what Scripture delivers, and the voice pronounces one only way,
In the Beginning God created heaven and earth; is it not understood
manifoldly, not through any deceit of error, but by various kinds of
true senses? Thus do man's offspring increase and multiply.

If therefore we conceive of the natures of the things themselves, not
allegorically, but properly, then does the phrase increase and multiply,
agree unto all things, that come of seed. But if we treat of the words
as figuratively spoken (which I rather suppose to be the purpose of
the Scripture, which doth not, surely, superfluously ascribe this
benediction to the offspring of aquatic animals and man only); then
do we find "multitude" to belong to creatures spiritual as well as
corporeal, as in heaven and earth, and to righteous and unrighteous, as
in light and darkness; and to holy authors who have been the ministers
of the Law unto us, as in the firmament which is settled betwixt
the waters and the waters; and to the society of people yet in the
bitterness of infidelity, as in the sea; and to the zeal of holy souls,
as in the dry land; and to works of mercy belonging to this present
life, as in the herbs bearing seed, and in trees bearing fruit; and to
spiritual gifts set forth for edification, as in the lights of heaven;
and to affections formed unto temperance, as in the living soul. In all
these instances we meet with multitudes, abundance, and increase; but
what shall in such wise increase and multiply that one thing may be
expressed many ways, and one expression understood many ways; we find
not, except in signs corporeally expressed, and in things mentally
conceived. By signs corporeally pronounced we understand the generations
of the waters, necessarily occasioned by the depth of the flesh;
by things mentally conceived, human generations, on account of the
fruitfulness of reason. And for this end do we believe Thee, Lord, to
have said to these kinds, Increase and multiply. For in this blessing, I
conceive Thee to have granted us a power and a faculty, both to express
several ways what we understand but one; and to understand several ways,
what we read to be obscurely delivered but in one. Thus are the
waters of the sea replenished, which are not moved but by several
significations: thus with human increase is the earth also replenished,
whose dryness appeareth in its longing, and reason ruleth over it.

I would also say, O Lord my God, what the following Scripture minds me
of; yea, I will say, and not fear. For I will say the truth, Thyself
inspiring me with what Thou willedst me to deliver out of those words.
But by no other inspiration than Thine, do I believe myself to speak
truth, seeing Thou art the Truth, and every man a liar. He therefore
that speaketh a lie, speaketh of his own; that therefore I may speak
truth, I will speak of Thine. Behold, Thou hast given unto us for food
every herb bearing seed which is upon all the earth; and every tree,
in which is the fruit of a tree yielding seed. And not to us alone, but
also to all the fowls of the air, and to the beasts of the earth, and to
all creeping things; but unto the fishes and to the great whales, hast
Thou not given them. Now we said that by these fruits of the earth were
signified, and figured in an allegory, the works of mercy which are
provided for the necessities of this life out of the fruitful earth.
Such an earth was the devout Onesiphorus, unto whose house Thou gavest
mercy, because he often refreshed Thy Paul, and was not ashamed of his
chain. Thus did also the brethren, and such fruit did they bear, who out
of Macedonia supplied what was lacking to him. But how grieved he for
some trees, which did not afford him the fruit due unto him, where he
saith, At my first answer no man stood by me, but all men forsook me. I
pray God that it may not be laid to their charge. For these fruits are
due to such as minister the spiritual doctrine unto us out of their
understanding of the divine mysteries; and they are due to them, as men;
yea and due to them also, as the living soul, which giveth itself as an
example, in all continency; and due unto them also, as flying creatures,
for their blessings which are multiplied upon the earth, because their
sound went out into all lands.

But they are fed by these fruits, that are delighted with them; nor are
they delighted with them, whose God is their belly. For neither in them
that yield them, are the things yielded the fruit, but with what mind
they yield them. He therefore that served God, and not his own belly,
I plainly see why he rejoiced; I see it, and I rejoice with him. For he
had received from the Philippians, what they had sent by Epaphroditus
unto him: and yet I perceive why he rejoiced. For whereat he rejoiced
upon that he fed; for, speaking in truth, I rejoiced (saith he) greatly
in the Lord, that now at the last your care of me hath flourished again,
wherein ye were also careful, but it had become wearisome unto you.
These Philippians then had now dried up, with a long weariness, and
withered as it were as to bearing this fruit of a good work; and he
rejoiceth for them, that they flourished again, not for himself, that
they supplied his wants. Therefore subjoins he, not that I speak in
respect of want, for I have learned in whatsoever state I am, therewith
to be content. I know both how to be abased, and I know how to abound;
every where and in all things I am instructed both to be full, and to be
hungry; both to abound, and to suffer need. I can do all things through
Him which strengtheneth me.

Whereat then rejoicest thou, O great Paul? whereat rejoicest thou?
whereon feedest thou, O man, renewed in the knowledge of God, after the
image of Him that created thee, thou living soul, of so much continency,
thou tongue like flying fowls, speaking mysteries? (for to such
creatures, is this food due;) what is it that feeds thee? joy. Hear
we what follows: notwithstanding, ye have well done, that ye did
communicate with my affliction. Hereat he rejoiceth, hereon feedeth;
because they had well done, not because his strait was eased, who saith
unto Thee, Thou hast enlarged me when I was in distress; for that he
knew to abound, and to suffer want, in Thee Who strengthenest him.
For ye Philippians also know (saith he), that in the beginning of the
Gospel, when I departed from Macedonia, no Church communicated with
me as concerning giving and receiving, but ye only. For even in
Thessalonica ye sent once and again unto my necessity. Unto these good
works, he now rejoiceth that they are returned; and is gladdened that
they flourished again, as when a fruitful field resumes its green.

Was it for his own necessities, because he said, Ye sent unto my
necessity? Rejoiceth he for that? Verily not for that. But how know we
this? Because himself says immediately, not because I desire a gift, but
I desire fruit. I have learned of Thee, my God, to distinguish betwixt
a gift, and fruit. A gift, is the thing itself which he gives, that
imparts these necessaries unto us; as money, meat, drink, clothing,
shelter, help: but the fruit, is the good and right will of the giver.
For the Good Master said not only, He that receiveth a prophet, but
added, in the name of a prophet: nor did He only say, He that receiveth
a righteous man, but added, in the name of a righteous man. So verily
shall the one receive the reward of a prophet, the other, the reward of
a righteous man: nor saith He only, He that shall give to drink a cup
of cold water to one of my little ones; but added, in the name of a
disciple: and so concludeth, Verily I say unto you, he shall not lose
his reward. The gift is, to receive a prophet, to receive a righteous
man, to give a cup of cold water to a disciple: but the fruit, to do
this in the name of a prophet, in the name of a righteous man, in the
name of a disciple. With fruit was Elijah fed by the widow that knew
she fed a man of God, and therefore fed him: but by the raven was he fed
with a gift. Nor was the inner man of Elijah so fed, but the outer only;
which might also for want of that food have perished.

I will then speak what is true in Thy sight, O Lord, that when carnal
men and infidels (for the gaining and initiating whom, the initiatory
Sacraments and the mighty workings of miracles are necessary, which we
suppose to be signified by the name of fishes and whales) undertake
the bodily refreshment, or otherwise succour Thy servant with something
useful for this present life; whereas they be ignorant, why this is to
be done, and to what end; neither do they feed these, nor are these
fed by them; because neither do the one do it out of an holy and right
intent; nor do the other rejoice at their gifts, whose fruit they as
yet behold not. For upon that is the mind fed, of which it is glad.
And therefore do not the fishes and whales feed upon such meats, as the
earth brings not forth until after it was separated and divided from the
bitterness of the waves of the sea.

And Thou, O God, sawest every thing that Thou hadst made, and, behold,
it was very good. Yea we also see the same, and behold, all things are
very good. Of the several kinds of Thy works, when Thou hadst said "let
them be," and they were, Thou sawest each that it was good. Seven times
have I counted it to be written, that Thou sawest that that which Thou
madest was good: and this is the eighth, that Thou sawest every thing
that Thou hadst made, and, behold, it was not only good, but also very
good, as being now altogether. For severally, they were only good; but
altogether, both good, and very good. All beautiful bodies express the
same; by reason that a body consisting of members all beautiful, is
far more beautiful than the same members by themselves are, by whose
well-ordered blending the whole is perfected; notwithstanding that the
members severally be also beautiful.

And I looked narrowly to find, whether seven, or eight times Thou sawest
that Thy works were good, when they pleased Thee; but in Thy seeing I
found no times, whereby I might understand that Thou sawest so often,
what Thou madest. And I said, "Lord, is not this Thy Scripture true,
since Thou art true, and being Truth, hast set it forth? why then dost
Thou say unto me, 'that in Thy seeing there be no times'; whereas this
Thy Scripture tells me, that what Thou madest each day, Thou sawest that
it was good: and when I counted them, I found how often." Unto this Thou
answerest me, for Thou art my God, and with a strong voice tellest Thy
servant in his inner ear, breaking through my deafness and crying, "O
man, that which My Scripture saith, I say: and yet doth that speak in
time; but time has no relation to My Word; because My Word exists
in equal eternity with Myself. So the things which ye see through My
Spirit, I see; like as what ye speak by My Spirit, I speak. And so when
ye see those things in time, I see them not in time; as when ye speak in
time, I speak them not in time."

And I heard, O Lord my God, and drank up a drop of sweetness out of Thy
truth, and understood, that certain men there be who mislike Thy works;
and say, that many of them Thou madest, compelled by necessity; such
as the fabric of the heavens, and harmony of the stars; and that Thou
madest them not of what was Thine, but that they were otherwhere and
from other sources created, for Thee to bring together and compact and
combine, when out of Thy conquered enemies Thou raisedst up the walls of
the universe; that they, bound down by the structure, might not again
be able to rebel against Thee. For other things, they say Thou neither
madest them, nor even compactedst them, such as all flesh and all very
minute creatures, and whatsoever hath its root in the earth; but that
a mind at enmity with Thee, and another nature not created by Thee, and
contrary unto Thee, did, in these lower stages of the world, beget and
frame these things. Frenzied are they who say thus, because they see not
Thy works by Thy Spirit, nor recognise Thee in them.

But they who by Thy Spirit see these things, Thou seest in them.
Therefore when they see that these things are good, Thou seest that they
are good; and whatsoever things for Thy sake please, Thou pleasest in
them, and what through Thy Spirit please us, they please Thee in us. For
what man knoweth the things of a man, save the spirit of a man, which is
in him? even so the things of God knoweth no one, but the Spirit of God.
Now we (saith he) have received, not the spirit of this world, but the
Spirit which is of God, that we might know the things that are freely
given to us of God. And I am admonished, "Truly the things of God
knoweth no one, but the Spirit of God: how then do we also know, what
things are given us of God?" Answer is made me; "because the things
which we know by His Spirit, even these no one knoweth, but the Spirit
of God. For as it is rightly said unto those that were to speak by the
Spirit of God, it is not ye that speak: so is it rightly said to them
that know through the Spirit of God, 'It is not ye that know.' And no
less then is it rightly said to those that see through the Spirit of
God, 'It is not ye that see'; so whatsoever through the Spirit of God
they see to be good, it is not they, but God that sees that it is good."
It is one thing then for a man to think that to be ill which is good,
as the forenamed do; another, that that which is good, a man should see
that it is good (as Thy creatures be pleasing unto many, because they
be good, whom yet Thou pleasest not in them, when they prefer to enjoy
them, to Thee); and another, that when a man sees a thing that it is
good, God should in him see that it is good, so, namely, that He should
be loved in that which He made, Who cannot be loved, but by the Holy
Ghost which He hath given. Because the love of God is shed abroad in our
hearts by the Holy Ghost, Which is given unto us: by Whom we see that
whatsoever in any degree is, is good. For from Him it is, who Himself Is
not in degree, but what He Is, Is.

Thanks to Thee, O Lord. We behold the heaven and earth, whether the
corporeal part, superior and inferior, or the spiritual and corporeal
creature; and in the adorning of these parts, whereof the universal pile
of the world, or rather the universal creation, doth consist, we see
light made, and divided from the darkness. We see the firmament of
heaven, whether that primary body of the world, between the spiritual
upper waters and the inferior corporeal waters, or (since this also
is called heaven) this space of air through which wander the fowls of
heaven, betwixt those waters which are in vapours borne above them, and
in clear nights distill down in dew; and those heavier waters which flow
along the earth. We behold a face of waters gathered together in the
fields of the sea; and the dry land both void, and formed so as to be
visible and harmonized, yea and the matter of herbs and trees. We behold
the lights shining from above, the sun to suffice for the day, the moon
and the stars to cheer the night; and that by all these, times should
be marked and signified. We behold on all sides a moist element,
replenished with fishes, beasts, and birds; because the grossness of
the air, which bears up the flights of birds, thickeneth itself by the
exhalation of the waters. We behold the face of the earth decked out
with earthly creatures, and man, created after Thy image and likeness,
even through that Thy very image and likeness (that is the power of
reason and understanding), set over all irrational creatures. And as
in his soul there is one power which has dominion by directing, another
made subject, that it might obey; so was there for the man, corporeally
also, made a woman, who in the mind of her reasonable understanding
should have a parity of nature, but in the sex of her body, should be in
like manner subject to the sex of her husband, as the appetite of doing
is fain to conceive the skill of right-doing from the reason of
the mind. These things we behold, and they are severally good, and
altogether very good.

Let Thy works praise Thee, that we may love Thee; and let us love Thee,
that Thy works may praise Thee, which from time have beginning and
ending, rising and setting, growth and decay, form and privation. They
have then their succession of morning and evening, part secretly, part
apparently; for they were made of nothing, by Thee, not of Thee; not of
any matter not Thine, or that was before, but of matter concreated (that
is, at the same time created by Thee), because to its state without
form, Thou without any interval of time didst give form. For seeing
the matter of heaven and earth is one thing, and the form another, Thou
madest the matter of merely nothing, but the form of the world out of
the matter without form: yet both together, so that the form should
follow the matter, without any interval of delay.

We have also examined what Thou willedst to be shadowed forth, whether
by the creation, or the relation of things in such an order. And we have
seen, that things singly are good, and together very good, in Thy Word,
in Thy Only-Begotten, both heaven and earth, the Head and the body of
the Church, in Thy predestination before all times, without morning
and evening. But when Thou begannest to execute in time the things
predestinated, to the end Thou mightest reveal hidden things, and
rectify our disorders; for our sins hung over us, and we had sunk into
the dark deep; and Thy good Spirit was borne over us, to help us in due
season; and Thou didst justify the ungodly, and dividest them from the
wicked; and Thou madest the firmament of authority of Thy Book between
those placed above, who were to be docile unto Thee, and those under,
who were to be subject to them: and Thou gatheredst together the society
of unbelievers into one conspiracy, that the zeal of the faithful might
appear, and they might bring forth works of mercy, even distributing to
the poor their earthly riches, to obtain heavenly. And after this didst
Thou kindle certain lights in the firmament, Thy Holy ones, having the
word of life; and shining with an eminent authority set on high through
spiritual gifts; after that again, for the initiation of the unbelieving
Gentiles, didst Thou out of corporeal matter produce the Sacraments, and
visible miracles, and forms of words according to the firmament of Thy
Book, by which the faithful should be blessed and multiplied. Next
didst Thou form the living soul of the faithful, through affections well
ordered by the vigour of continency: and after that, the mind subjected
to Thee alone and needing to imitate no human authority, hast Thou
renewed after Thy image and likeness; and didst subject its rational
actions to the excellency of the understanding, as the woman to the man;
and to all Offices of Thy Ministry, necessary for the perfecting of the
faithful in this life, Thou willedst, that for their temporal uses, good
things, fruitful to themselves in time to come, be given by the same
faithful. All these we see, and they are very good, because Thou seest
them in us, Who hast given unto us Thy Spirit, by which we might see
them, and in them love Thee.

O Lord God, give peace unto us: (for Thou hast given us all things;) the
peace of rest, the peace of the Sabbath, which hath no evening. For
all this most goodly array of things very good, having finished their
courses, is to pass away, for in them there was morning and evening.

But the seventh day hath no evening, nor hath it setting; because Thou
hast sanctified it to an everlasting continuance; that that which Thou
didst after Thy works which were very good, resting the seventh day,
although Thou madest them in unbroken rest, that may the voice of
Thy Book announce beforehand unto us, that we also after our works
(therefore very good, because Thou hast given them us), shall rest in
Thee also in the Sabbath of eternal life.

For then shalt Thou rest in us, as now Thou workest in us; and so shall
that be Thy rest through us, as these are Thy works through us. But
Thou, Lord, ever workest, and art ever at rest. Nor dost Thou see in
time, nor art moved in time, nor restest in a time; and yet Thou makest
things seen in time, yea the times themselves, and the rest which
results from time.

We therefore see these things which Thou madest, because they are: but
they are, because Thou seest them. And we see without, that they are,
and within, that they are good, but Thou sawest them there, when made,
where Thou sawest them, yet to be made. And we were at a later time
moved to do well, after our hearts had conceived of Thy Spirit; but in
the former time we were moved to do evil, forsaking Thee; but Thou, the
One, the Good God, didst never cease doing good. And we also have some
good works, of Thy gift, but not eternal; after them we trust to rest in
Thy great hallowing. But Thou, being the Good which needeth no good, art
ever at rest, because Thy rest is Thou Thyself. And what man can teach
man to understand this? or what Angel, an Angel? or what Angel, a man?
Let it be asked of Thee, sought in Thee, knocked for at Thee; so, so
shall it be received, so shall it be found, so shall it be opened. Amen.

                         GRATIAS TIBI DOMINE


\end{document}

