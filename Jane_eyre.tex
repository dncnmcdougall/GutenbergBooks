\documentclass[b5paper,openright,12pt,twoside]{book}

\usepackage{amsmath}
\usepackage{graphicx}
\usepackage[T1]{fontenc}
\usepackage{titlesec}
\usepackage{titling}
\usepackage{lettrine}
\usepackage{yfonts}
\usepackage{expl3}
\usepackage{xparse}

\title{Jane Eyre\\an Autobiography}

\author{Charlotte Bront\"{e}}

\renewcommand{\thechapter}{\Roman{chapter}}
\usepackage{fancyhdr}
\setlength{\headheight}{15pt}

\pagestyle{fancy}
  \renewcommand{\headrulewidth}{0pt} % remove lines as well
  \renewcommand{\footrulewidth}{0pt}


\fancyhf{}
\fancyhf[FLE,FRO]{\thepage}
\fancyhf[FRE]{\textit{\theauthor} }
\fancyhf[FLO]{\textit{ CHAPTER \thechapter.} }


\titlespacing*{\chapter}{0pt}{-50pt}{15pt}
\titleformat{\chapter}[display]{\normalfont\huge\bfseries}{\chaptertitlename\ \thechapter}{20pt}{\Huge}

\setcounter{DefaultLines}{2}
\renewcommand{\DefaultLoversize}{0.15}
\setlength{\DefaultFindent}{0.1em}

%\renewcommand*{\LettrineFont}{\rmdefault}

%\NewDocumentCommand\start{mg}{%
%    \yinipar{#1} \IfNoValueTF{#2}{}{\uppercase{#2}}
%}
\NewDocumentCommand\start{mg}{%
    \IfNoValueTF{#2}{\lettrine{#1}}{\lettrine{#1}{#2}}
}


\begin{document}
\pagestyle{empty}
\frontmatter
\maketitle
\pagenumbering{gobble}
\noindent
Transcribed from the 1897 Service \& Paton edition by David Price, email
ccx074@pglaf.org
\\
\\
\noindent
Jane Eyre
an Autobiography
\\
\\
\noindent
BY\\
\noindent
Charlotte Bront\"{e}
\\
\\
\noindent
\emph{Illustrated By F. H. Townsend}
\\
\\
\noindent
London
Service \& Paton
5 Henrietta Street
1897
\\
\\
\noindent
\emph{The Illustrations}
\emph{in this Volume are the copyright of}
Service \& Patton, \emph{London}
\\
\\
\noindent
To\\
\noindent
W. M. Thackeray, Esq.,

\noindent
This work is respectfully inscribed

\noindent
By
The Author

\chapter*{Preface}
\pagestyle{fancy}
\pagenumbering{roman}

\start{A} preface to the first edition of "Jane Eyre" being unnecessary, I gave
none: this second edition demands a few words both of acknowledgment and
miscellaneous remark.

My thanks are due in three quarters.

To the Public, for the indulgent ear it has inclined to a plain tale with
few pretensions.

To the Press, for the fair field its honest suffrage has opened to an
obscure aspirant.

To my Publishers, for the aid their tact, their energy, their practical
sense and frank liberality have afforded an unknown and unrecommended
Author.

The Press and the Public are but vague personifications for me, and I
must thank them in vague terms; but my Publishers are definite: so are
certain generous critics who have encouraged me as only large-hearted and
high-minded men know how to encourage a struggling stranger; to them,
\emph{i.e.}, to my Publishers and the select Reviewers, I say cordially,
Gentlemen, I thank you from my heart.

Having thus acknowledged what I owe those who have aided and approved me,
I turn to another class; a small one, so far as I know, but not,
therefore, to be overlooked.  I mean the timorous or carping few who
doubt the tendency of such books as "Jane Eyre:" in whose eyes whatever
is unusual is wrong; whose ears detect in each protest against
bigotry--that parent of crime--an insult to piety, that regent of God on
earth.  I would suggest to such doubters certain obvious distinctions; I
would remind them of certain simple truths.

Conventionality is not morality.  Self-righteousness is not religion.  To
attack the first is not to assail the last.  To pluck the mask from the
face of the Pharisee, is not to lift an impious hand to the Crown of
Thorns.

These things and deeds are diametrically opposed: they are as distinct as
is vice from virtue.  Men too often confound them: they should not be
confounded: appearance should not be mistaken for truth; narrow human
doctrines, that only tend to elate and magnify a few, should not be
substituted for the world-redeeming creed of Christ.  There is--I repeat
it--a difference; and it is a good, and not a bad action to mark broadly
and clearly the line of separation between them.

The world may not like to see these ideas dissevered, for it has been
accustomed to blend them; finding it convenient to make external show
pass for sterling worth--to let white-washed walls vouch for clean
shrines.  It may hate him who dares to scrutinise and expose--to rase the
gilding, and show base metal under it--to penetrate the sepulchre, and
reveal charnel relics: but hate as it will, it is indebted to him.

Ahab did not like Micaiah, because he never prophesied good concerning
him, but evil; probably he liked the sycophant son of Chenaannah better;
yet might Ahab have escaped a bloody death, had he but stopped his ears
to flattery, and opened them to faithful counsel.

There is a man in our own days whose words are not framed to tickle
delicate ears: who, to my thinking, comes before the great ones of
society, much as the son of Imlah came before the throned Kings of Judah
and Israel; and who speaks truth as deep, with a power as prophet-like
and as vital--a mien as dauntless and as daring.  Is the satirist of
"Vanity Fair" admired in high places?  I cannot tell; but I think if some
of those amongst whom he hurls the Greek fire of his sarcasm, and over
whom he flashes the levin-brand of his denunciation, were to take his
warnings in time--they or their seed might yet escape a fatal
Rimoth-Gilead.

Why have I alluded to this man?  I have alluded to him, Reader, because I
think I see in him an intellect profounder and more unique than his
contemporaries have yet recognised; because I regard him as the first
social regenerator of the day--as the very master of that working corps
who would restore to rectitude the warped system of things; because I
think no commentator on his writings has yet found the comparison that
suits him, the terms which rightly characterise his talent.  They say he
is like Fielding: they talk of his wit, humour, comic powers.  He
resembles Fielding as an eagle does a vulture: Fielding could stoop on
carrion, but Thackeray never does.  His wit is bright, his humour
attractive, but both bear the same relation to his serious genius that
the mere lambent sheet-lightning playing under the edge of the summer-
cloud does to the electric death-spark hid in its womb.  Finally, I have
alluded to Mr. Thackeray, because to him--if he will accept the tribute
of a total stranger--I have dedicated this second edition of "JANE EYRE."

CURRER BELL.

\emph{December} 21\emph{st}, 1847.


\chapter*{Note to the Third Edition}


\start{I} avail myself of the opportunity which a third edition of "Jane Eyre"
affords me, of again addressing a word to the Public, to explain that my
claim to the title of novelist rests on this one work alone.  If,
therefore, the authorship of other works of fiction has been attributed
to me, an honour is awarded where it is not merited; and consequently,
denied where it is justly due.

This explanation will serve to rectify mistakes which may already have
been made, and to prevent future errors.

CURRER BELL.

\emph{April} 13\emph{th}, 1848.

\mainmatter
\pagebreak
\tableofcontents
\chapter{}
\setcounter{page}{1}
\pagenumbering{arabic}

\start{T}{here} was no possibility of taking a walk that day.  We had been
wandering, indeed, in the leafless shrubbery an hour in the morning; but
since dinner (Mrs. Reed, when there was no company, dined early) the cold
winter wind had brought with it clouds so sombre, and a rain so
penetrating, that further out-door exercise was now out of the question.

I was glad of it: I never liked long walks, especially on chilly
afternoons: dreadful to me was the coming home in the raw twilight, with
nipped fingers and toes, and a heart saddened by the chidings of Bessie,
the nurse, and humbled by the consciousness of my physical inferiority to
Eliza, John, and Georgiana Reed.

The said Eliza, John, and Georgiana were now clustered round their mama
in the drawing-room: she lay reclined on a sofa by the fireside, and with
her darlings about her (for the time neither quarrelling nor crying)
looked perfectly happy.  Me, she had dispensed from joining the group;
saying, "She regretted to be under the necessity of keeping me at a
distance; but that until she heard from Bessie, and could discover by her
own observation, that I was endeavouring in good earnest to acquire a
more sociable and childlike disposition, a more attractive and sprightly
manner--something lighter, franker, more natural, as it were--she really
must exclude me from privileges intended only for contented, happy,
little children."

"What does Bessie say I have done?" I asked.

"Jane, I don't like cavillers or questioners; besides, there is something
truly forbidding in a child taking up her elders in that manner.  Be
seated somewhere; and until you can speak pleasantly, remain silent."

A breakfast-room adjoined the drawing-room, I slipped in there.  It
contained a bookcase: I soon possessed myself of a volume, taking care
that it should be one stored with pictures.  I mounted into the window-
seat: gathering up my feet, I sat cross-legged, like a Turk; and, having
drawn the red moreen curtain nearly close, I was shrined in double
retirement.

Folds of scarlet drapery shut in my view to the right hand; to the left
were the clear panes of glass, protecting, but not separating me from the
drear November day.  At intervals, while turning over the leaves of my
book, I studied the aspect of that winter afternoon.  Afar, it offered a
pale blank of mist and cloud; near a scene of wet lawn and storm-beat
shrub, with ceaseless rain sweeping away wildly before a long and
lamentable blast.

I returned to my book--Bewick's History of British Birds: the letterpress
thereof I cared little for, generally speaking; and yet there were
certain introductory pages that, child as I was, I could not pass quite
as a blank.  They were those which treat of the haunts of sea-fowl; of
"the solitary rocks and promontories" by them only inhabited; of the
coast of Norway, studded with isles from its southern extremity, the
Lindeness, or Naze, to the North Cape--

   "Where the Northern Ocean, in vast whirls,
   Boils round the naked, melancholy isles
   Of farthest Thule; and the Atlantic surge
   Pours in among the stormy Hebrides."

Nor could I pass unnoticed the suggestion of the bleak shores of Lapland,
Siberia, Spitzbergen, Nova Zembla, Iceland, Greenland, with "the vast
sweep of the Arctic Zone, and those forlorn regions of dreary space,--that
reservoir of frost and snow, where firm fields of ice, the accumulation
of centuries of winters, glazed in Alpine heights above heights, surround
the pole, and concentre the multiplied rigours of extreme cold."  Of
these death-white realms I formed an idea of my own: shadowy, like all
the half-comprehended notions that float dim through children's brains,
but strangely impressive.  The words in these introductory pages
connected themselves with the succeeding vignettes, and gave significance
to the rock standing up alone in a sea of billow and spray; to the broken
boat stranded on a desolate coast; to the cold and ghastly moon glancing
through bars of cloud at a wreck just sinking.

I cannot tell what sentiment haunted the quite solitary churchyard, with
its inscribed headstone; its gate, its two trees, its low horizon,
girdled by a broken wall, and its newly-risen crescent, attesting the
hour of eventide.

The two ships becalmed on a torpid sea, I believed to be marine phantoms.

The fiend pinning down the thief's pack behind him, I passed over
quickly: it was an object of terror.

So was the black horned thing seated aloof on a rock, surveying a distant
crowd surrounding a gallows.

Each picture told a story; mysterious often to my undeveloped
understanding and imperfect feelings, yet ever profoundly interesting: as
interesting as the tales Bessie sometimes narrated on winter evenings,
when she chanced to be in good humour; and when, having brought her
ironing-table to the nursery hearth, she allowed us to sit about it, and
while she got up Mrs. Reed's lace frills, and crimped her nightcap
borders, fed our eager attention with passages of love and adventure
taken from old fairy tales and other ballads; or (as at a later period I
discovered) from the pages of Pamela, and Henry, Earl of Moreland.

With Bewick on my knee, I was then happy: happy at least in my way.  I
feared nothing but interruption, and that came too soon.  The breakfast-
room door opened.

"Boh!  Madam Mope!" cried the voice of John Reed; then he paused: he
found the room apparently empty.

"Where the dickens is she!" he continued.  "Lizzy!  Georgy! (calling to
his sisters) Joan is not here: tell mama she is run out into the rain--bad
animal!"

"It is well I drew the curtain," thought I; and I wished fervently he
might not discover my hiding-place: nor would John Reed have found it out
himself; he was not quick either of vision or conception; but Eliza just
put her head in at the door, and said at once--

"She is in the window-seat, to be sure, Jack."

And I came out immediately, for I trembled at the idea of being dragged
forth by the said Jack.

"What do you want?" I asked, with awkward diffidence.

"Say, 'What do you want, Master Reed?'" was the answer.  "I want you to
come here;" and seating himself in an arm-chair, he intimated by a
gesture that I was to approach and stand before him.

John Reed was a schoolboy of fourteen years old; four years older than I,
for I was but ten: large and stout for his age, with a dingy and
unwholesome skin; thick lineaments in a spacious visage, heavy limbs and
large extremities.  He gorged himself habitually at table, which made him
bilious, and gave him a dim and bleared eye and flabby cheeks.  He ought
now to have been at school; but his mama had taken him home for a month
or two, "on account of his delicate health."  Mr. Miles, the master,
affirmed that he would do very well if he had fewer cakes and sweetmeats
sent him from home; but the mother's heart turned from an opinion so
harsh, and inclined rather to the more refined idea that John's
sallowness was owing to over-application and, perhaps, to pining after
home.

John had not much affection for his mother and sisters, and an antipathy
to me.  He bullied and punished me; not two or three times in the week,
nor once or twice in the day, but continually: every nerve I had feared
him, and every morsel of flesh in my bones shrank when he came near.
There were moments when I was bewildered by the terror he inspired,
because I had no appeal whatever against either his menaces or his
inflictions; the servants did not like to offend their young master by
taking my part against him, and Mrs. Reed was blind and deaf on the
subject: she never saw him strike or heard him abuse me, though he did
both now and then in her very presence, more frequently, however, behind
her back.

Habitually obedient to John, I came up to his chair: he spent some three
minutes in thrusting out his tongue at me as far as he could without
damaging the roots: I knew he would soon strike, and while dreading the
blow, I mused on the disgusting and ugly appearance of him who would
presently deal it.  I wonder if he read that notion in my face; for, all
at once, without speaking, he struck suddenly and strongly.  I tottered,
and on regaining my equilibrium retired back a step or two from his
chair.

"That is for your impudence in answering mama awhile since," said he,
"and for your sneaking way of getting behind curtains, and for the look
you had in your eyes two minutes since, you rat!"

Accustomed to John Reed's abuse, I never had an idea of replying to it;
my care was how to endure the blow which would certainly follow the
insult.

"What were you doing behind the curtain?" he asked.

"I was reading."

"Show the book."

I returned to the window and fetched it thence.

"You have no business to take our books; you are a dependent, mama says;
you have no money; your father left you none; you ought to beg, and not
to live here with gentlemen's children like us, and eat the same meals we
do, and wear clothes at our mama's expense.  Now, I'll teach you to
rummage my bookshelves: for they \emph{are} mine; all the house belongs to me,
or will do in a few years.  Go and stand by the door, out of the way of
the mirror and the windows."

I did so, not at first aware what was his intention; but when I saw him
lift and poise the book and stand in act to hurl it, I instinctively
started aside with a cry of alarm: not soon enough, however; the volume
was flung, it hit me, and I fell, striking my head against the door and
cutting it.  The cut bled, the pain was sharp: my terror had passed its
climax; other feelings succeeded.

"Wicked and cruel boy!" I said.  "You are like a murderer--you are like a
slave-driver--you are like the Roman emperors!"

I had read Goldsmith's History of Rome, and had formed my opinion of
Nero, Caligula, etc.  Also I had drawn parallels in silence, which I never
thought thus to have declared aloud.

"What! what!" he cried.  "Did she say that to me?  Did you hear her,
Eliza and Georgiana?  Won't I tell mama? but first--"

He ran headlong at me: I felt him grasp my hair and my shoulder: he had
closed with a desperate thing.  I really saw in him a tyrant, a murderer.
I felt a drop or two of blood from my head trickle down my neck, and was
sensible of somewhat pungent suffering: these sensations for the time
predominated over fear, and I received him in frantic sort.  I don't very
well know what I did with my hands, but he called me "Rat!  Rat!" and
bellowed out aloud.  Aid was near him: Eliza and Georgiana had run for
Mrs. Reed, who was gone upstairs: she now came upon the scene, followed
by Bessie and her maid Abbot.  We were parted: I heard the words--

"Dear! dear!  What a fury to fly at Master John!"

"Did ever anybody see such a picture of passion!"

Then Mrs. Reed subjoined--

"Take her away to the red-room, and lock her in there."  Four hands were
immediately laid upon me, and I was borne upstairs.




\chapter{}


\lettrine{I} resisted all the way: a new thing for me, and a circumstance which
greatly strengthened the bad opinion Bessie and Miss Abbot were disposed
to entertain of me.  The fact is, I was a trifle beside myself; or rather
\emph{out} of myself, as the French would say: I was conscious that a moment's
mutiny had already rendered me liable to strange penalties, and, like any
other rebel slave, I felt resolved, in my desperation, to go all lengths.

"Hold her arms, Miss Abbot: she's like a mad cat."

"For shame! for shame!" cried the lady's-maid.  "What shocking conduct,
Miss Eyre, to strike a young gentleman, your benefactress's son!  Your
young master."

"Master!  How is he my master?  Am I a servant?"

"No; you are less than a servant, for you do nothing for your keep.
There, sit down, and think over your wickedness."

They had got me by this time into the apartment indicated by Mrs. Reed,
and had thrust me upon a stool: my impulse was to rise from it like a
spring; their two pair of hands arrested me instantly.

"If you don't sit still, you must be tied down," said Bessie.  "Miss
Abbot, lend me your garters; she would break mine directly."

Miss Abbot turned to divest a stout leg of the necessary ligature.  This
preparation for bonds, and the additional ignominy it inferred, took a
little of the excitement out of me.

"Don't take them off," I cried; "I will not stir."

In guarantee whereof, I attached myself to my seat by my hands.

"Mind you don't," said Bessie; and when she had ascertained that I was
really subsiding, she loosened her hold of me; then she and Miss Abbot
stood with folded arms, looking darkly and doubtfully on my face, as
incredulous of my sanity.

"She never did so before," at last said Bessie, turning to the Abigail.

"But it was always in her," was the reply.  "I've told Missis often my
opinion about the child, and Missis agreed with me.  She's an underhand
little thing: I never saw a girl of her age with so much cover."

Bessie answered not; but ere long, addressing me, she said--"You ought to
be aware, Miss, that you are under obligations to Mrs. Reed: she keeps
you: if she were to turn you off, you would have to go to the poorhouse."

I had nothing to say to these words: they were not new to me: my very
first recollections of existence included hints of the same kind.  This
reproach of my dependence had become a vague sing-song in my ear: very
painful and crushing, but only half intelligible.  Miss Abbot joined in--

"And you ought not to think yourself on an equality with the Misses Reed
and Master Reed, because Missis kindly allows you to be brought up with
them.  They will have a great deal of money, and you will have none: it
is your place to be humble, and to try to make yourself agreeable to
them."

"What we tell you is for your good," added Bessie, in no harsh voice,
"you should try to be useful and pleasant, then, perhaps, you would have
a home here; but if you become passionate and rude, Missis will send you
away, I am sure."

"Besides," said Miss Abbot, "God will punish her: He might strike her
dead in the midst of her tantrums, and then where would she go?  Come,
Bessie, we will leave her: I wouldn't have her heart for anything.  Say
your prayers, Miss Eyre, when you are by yourself; for if you don't
repent, something bad might be permitted to come down the chimney and
fetch you away."

They went, shutting the door, and locking it behind them.

The red-room was a square chamber, very seldom slept in, I might say
never, indeed, unless when a chance influx of visitors at Gateshead Hall
rendered it necessary to turn to account all the accommodation it
contained: yet it was one of the largest and stateliest chambers in the
mansion.  A bed supported on massive pillars of mahogany, hung with
curtains of deep red damask, stood out like a tabernacle in the centre;
the two large windows, with their blinds always drawn down, were half
shrouded in festoons and falls of similar drapery; the carpet was red;
the table at the foot of the bed was covered with a crimson cloth; the
walls were a soft fawn colour with a blush of pink in it; the wardrobe,
the toilet-table, the chairs were of darkly polished old mahogany.  Out
of these deep surrounding shades rose high, and glared white, the piled-
up mattresses and pillows of the bed, spread with a snowy Marseilles
counterpane.  Scarcely less prominent was an ample cushioned easy-chair
near the head of the bed, also white, with a footstool before it; and
looking, as I thought, like a pale throne.

This room was chill, because it seldom had a fire; it was silent, because
remote from the nursery and kitchen; solemn, because it was known to be
so seldom entered.  The house-maid alone came here on Saturdays, to wipe
from the mirrors and the furniture a week's quiet dust: and Mrs. Reed
herself, at far intervals, visited it to review the contents of a certain
secret drawer in the wardrobe, where were stored divers parchments, her
jewel-casket, and a miniature of her deceased husband; and in those last
words lies the secret of the red-room--the spell which kept it so lonely
in spite of its grandeur.

Mr. Reed had been dead nine years: it was in this chamber he breathed his
last; here he lay in state; hence his coffin was borne by the
undertaker's men; and, since that day, a sense of dreary consecration had
guarded it from frequent intrusion.

My seat, to which Bessie and the bitter Miss Abbot had left me riveted,
was a low ottoman near the marble chimney-piece; the bed rose before me;
to my right hand there was the high, dark wardrobe, with subdued, broken
reflections varying the gloss of its panels; to my left were the muffled
windows; a great looking-glass between them repeated the vacant majesty
of the bed and room.  I was not quite sure whether they had locked the
door; and when I dared move, I got up and went to see.  Alas! yes: no
jail was ever more secure.  Returning, I had to cross before the looking-
glass; my fascinated glance involuntarily explored the depth it revealed.
All looked colder and darker in that visionary hollow than in reality:
and the strange little figure there gazing at me, with a white face and
arms specking the gloom, and glittering eyes of fear moving where all
else was still, had the effect of a real spirit: I thought it like one of
the tiny phantoms, half fairy, half imp, Bessie's evening stories
represented as coming out of lone, ferny dells in moors, and appearing
before the eyes of belated travellers.  I returned to my stool.

Superstition was with me at that moment; but it was not yet her hour for
complete victory: my blood was still warm; the mood of the revolted slave
was still bracing me with its bitter vigour; I had to stem a rapid rush
of retrospective thought before I quailed to the dismal present.

All John Reed's violent tyrannies, all his sisters' proud indifference,
all his mother's aversion, all the servants' partiality, turned up in my
disturbed mind like a dark deposit in a turbid well.  Why was I always
suffering, always browbeaten, always accused, for ever condemned?  Why
could I never please?  Why was it useless to try to win any one's favour?
Eliza, who was headstrong and selfish, was respected.  Georgiana, who had
a spoiled temper, a very acrid spite, a captious and insolent carriage,
was universally indulged.  Her beauty, her pink cheeks and golden curls,
seemed to give delight to all who looked at her, and to purchase
indemnity for every fault.  John no one thwarted, much less punished;
though he twisted the necks of the pigeons, killed the little pea-chicks,
set the dogs at the sheep, stripped the hothouse vines of their fruit,
and broke the buds off the choicest plants in the conservatory: he called
his mother "old girl," too; sometimes reviled her for her dark skin,
similar to his own; bluntly disregarded her wishes; not unfrequently tore
and spoiled her silk attire; and he was still "her own darling."  I dared
commit no fault: I strove to fulfil every duty; and I was termed naughty
and tiresome, sullen and sneaking, from morning to noon, and from noon to
night.

My head still ached and bled with the blow and fall I had received: no
one had reproved John for wantonly striking me; and because I had turned
against him to avert farther irrational violence, I was loaded with
general opprobrium.

"Unjust!--unjust!" said my reason, forced by the agonising stimulus into
precocious though transitory power: and Resolve, equally wrought up,
instigated some strange expedient to achieve escape from insupportable
oppression--as running away, or, if that could not be effected, never
eating or drinking more, and letting myself die.

What a consternation of soul was mine that dreary afternoon!  How all my
brain was in tumult, and all my heart in insurrection!  Yet in what
darkness, what dense ignorance, was the mental battle fought!  I could
not answer the ceaseless inward question--\emph{why} I thus suffered; now, at
the distance of--I will not say how many years, I see it clearly.

I was a discord in Gateshead Hall: I was like nobody there; I had nothing
in harmony with Mrs. Reed or her children, or her chosen vassalage.  If
they did not love me, in fact, as little did I love them.  They were not
bound to regard with affection a thing that could not sympathise with one
amongst them; a heterogeneous thing, opposed to them in temperament, in
capacity, in propensities; a useless thing, incapable of serving their
interest, or adding to their pleasure; a noxious thing, cherishing the
germs of indignation at their treatment, of contempt of their judgment.  I
know that had I been a sanguine, brilliant, careless, exacting, handsome,
romping child--though equally dependent and friendless--Mrs. Reed would
have endured my presence more complacently; her children would have
entertained for me more of the cordiality of fellow-feeling; the servants
would have been less prone to make me the scapegoat of the nursery.

Daylight began to forsake the red-room; it was past four o'clock, and the
beclouded afternoon was tending to drear twilight.  I heard the rain
still beating continuously on the staircase window, and the wind howling
in the grove behind the hall; I grew by degrees cold as a stone, and then
my courage sank.  My habitual mood of humiliation, self-doubt, forlorn
depression, fell damp on the embers of my decaying ire.  All said I was
wicked, and perhaps I might be so; what thought had I been but just
conceiving of starving myself to death?  That certainly was a crime: and
was I fit to die?  Or was the vault under the chancel of Gateshead Church
an inviting bourne?  In such vault I had been told did Mr. Reed lie
buried; and led by this thought to recall his idea, I dwelt on it with
gathering dread.  I could not remember him; but I knew that he was my own
uncle--my mother's brother--that he had taken me when a parentless infant
to his house; and that in his last moments he had required a promise of
Mrs. Reed that she would rear and maintain me as one of her own children.
Mrs. Reed probably considered she had kept this promise; and so she had,
I dare say, as well as her nature would permit her; but how could she
really like an interloper not of her race, and unconnected with her,
after her husband's death, by any tie?  It must have been most irksome to
find herself bound by a hard-wrung pledge to stand in the stead of a
parent to a strange child she could not love, and to see an uncongenial
alien permanently intruded on her own family group.

A singular notion dawned upon me.  I doubted not--never doubted--that if
Mr. Reed had been alive he would have treated me kindly; and now, as I
sat looking at the white bed and overshadowed walls--occasionally also
turning a fascinated eye towards the dimly gleaning mirror--I began to
recall what I had heard of dead men, troubled in their graves by the
violation of their last wishes, revisiting the earth to punish the
perjured and avenge the oppressed; and I thought Mr. Reed's spirit,
harassed by the wrongs of his sister's child, might quit its
abode--whether in the church vault or in the unknown world of the
departed--and rise before me in this chamber.  I wiped my tears and
hushed my sobs, fearful lest any sign of violent grief might waken a
preternatural voice to comfort me, or elicit from the gloom some haloed
face, bending over me with strange pity.  This idea, consolatory in
theory, I felt would be terrible if realised: with all my might I
endeavoured to stifle it--I endeavoured to be firm.  Shaking my hair from
my eyes, I lifted my head and tried to look boldly round the dark room;
at this moment a light gleamed on the wall.  Was it, I asked myself, a
ray from the moon penetrating some aperture in the blind?  No; moonlight
was still, and this stirred; while I gazed, it glided up to the ceiling
and quivered over my head.  I can now conjecture readily that this streak
of light was, in all likelihood, a gleam from a lantern carried by some
one across the lawn: but then, prepared as my mind was for horror, shaken
as my nerves were by agitation, I thought the swift darting beam was a
herald of some coming vision from another world.  My heart beat thick, my
head grew hot; a sound filled my ears, which I deemed the rushing of
wings; something seemed near me; I was oppressed, suffocated: endurance
broke down; I rushed to the door and shook the lock in desperate effort.
Steps came running along the outer passage; the key turned, Bessie and
Abbot entered.

"Miss Eyre, are you ill?" said Bessie.

"What a dreadful noise! it went quite through me!" exclaimed Abbot.

"Take me out!  Let me go into the nursery!" was my cry.

"What for?  Are you hurt?  Have you seen something?" again demanded
Bessie.

"Oh!  I saw a light, and I thought a ghost would come."  I had now got
hold of Bessie's hand, and she did not snatch it from me.

"She has screamed out on purpose," declared Abbot, in some disgust.  "And
what a scream!  If she had been in great pain one would have excused it,
but she only wanted to bring us all here: I know her naughty tricks."

"What is all this?" demanded another voice peremptorily; and Mrs. Reed
came along the corridor, her cap flying wide, her gown rustling stormily.
"Abbot and Bessie, I believe I gave orders that Jane Eyre should be left
in the red-room till I came to her myself."

"Miss Jane screamed so loud, ma'am," pleaded Bessie.

"Let her go," was the only answer.  "Loose Bessie's hand, child: you
cannot succeed in getting out by these means, be assured.  I abhor
artifice, particularly in children; it is my duty to show you that tricks
will not answer: you will now stay here an hour longer, and it is only on
condition of perfect submission and stillness that I shall liberate you
then."

"O aunt! have pity!  Forgive me!  I cannot endure it--let me be punished
some other way!  I shall be killed if--"

"Silence!  This violence is all most repulsive:" and so, no doubt, she
felt it.  I was a precocious actress in her eyes; she sincerely looked on
me as a compound of virulent passions, mean spirit, and dangerous
duplicity.

Bessie and Abbot having retreated, Mrs. Reed, impatient of my now frantic
anguish and wild sobs, abruptly thrust me back and locked me in, without
farther parley.  I heard her sweeping away; and soon after she was gone,
I suppose I had a species of fit: unconsciousness closed the scene.




\chapter{}


\lettrine{T}{he} next thing I remember is, waking up with a feeling as if I had had a
frightful nightmare, and seeing before me a terrible red glare, crossed
with thick black bars.  I heard voices, too, speaking with a hollow
sound, and as if muffled by a rush of wind or water: agitation,
uncertainty, and an all-predominating sense of terror confused my
faculties.  Ere long, I became aware that some one was handling me;
lifting me up and supporting me in a sitting posture, and that more
tenderly than I had ever been raised or upheld before.  I rested my head
against a pillow or an arm, and felt easy.

In five minutes more the cloud of bewilderment dissolved: I knew quite
well that I was in my own bed, and that the red glare was the nursery
fire.  It was night: a candle burnt on the table; Bessie stood at the bed-
foot with a basin in her hand, and a gentleman sat in a chair near my
pillow, leaning over me.

I felt an inexpressible relief, a soothing conviction of protection and
security, when I knew that there was a stranger in the room, an
individual not belonging to Gateshead, and not related to Mrs. Reed.
Turning from Bessie (though her presence was far less obnoxious to me
than that of Abbot, for instance, would have been), I scrutinised the
face of the gentleman: I knew him; it was Mr. Lloyd, an apothecary,
sometimes called in by Mrs. Reed when the servants were ailing: for
herself and the children she employed a physician.

"Well, who am I?" he asked.

I pronounced his name, offering him at the same time my hand: he took it,
smiling and saying, "We shall do very well by-and-by."  Then he laid me
down, and addressing Bessie, charged her to be very careful that I was
not disturbed during the night.  Having given some further directions,
and intimates that he should call again the next day, he departed; to my
grief: I felt so sheltered and befriended while he sat in the chair near
my pillow; and as he closed the door after him, all the room darkened and
my heart again sank: inexpressible sadness weighed it down.

"Do you feel as if you should sleep, Miss?" asked Bessie, rather softly.

Scarcely dared I answer her; for I feared the next sentence might be
rough.  "I will try."

"Would you like to drink, or could you eat anything?"

"No, thank you, Bessie."

"Then I think I shall go to bed, for it is past twelve o'clock; but you
may call me if you want anything in the night."

Wonderful civility this!  It emboldened me to ask a question.

"Bessie, what is the matter with me?  Am I ill?"

"You fell sick, I suppose, in the red-room with crying; you'll be better
soon, no doubt."

Bessie went into the housemaid's apartment, which was near.  I heard her
say--

"Sarah, come and sleep with me in the nursery; I daren't for my life be
alone with that poor child to-night: she might die; it's such a strange
thing she should have that fit: I wonder if she saw anything.  Missis was
rather too hard."

Sarah came back with her; they both went to bed; they were whispering
together for half-an-hour before they fell asleep.  I caught scraps of
their conversation, from which I was able only too distinctly to infer
the main subject discussed.

"Something passed her, all dressed in white, and vanished"--"A great
black dog behind him"--"Three loud raps on the chamber door"--"A light in
the churchyard just over his grave," etc., etc.

At last both slept: the fire and the candle went out.  For me, the
watches of that long night passed in ghastly wakefulness; strained by
dread: such dread as children only can feel.

No severe or prolonged bodily illness followed this incident of the red-
room; it only gave my nerves a shock of which I feel the reverberation to
this day.  Yes, Mrs. Reed, to you I owe some fearful pangs of mental
suffering, but I ought to forgive you, for you knew not what you did:
while rending my heart-strings, you thought you were only uprooting my
bad propensities.

Next day, by noon, I was up and dressed, and sat wrapped in a shawl by
the nursery hearth.  I felt physically weak and broken down: but my worse
ailment was an unutterable wretchedness of mind: a wretchedness which
kept drawing from me silent tears; no sooner had I wiped one salt drop
from my cheek than another followed.  Yet, I thought, I ought to have
been happy, for none of the Reeds were there, they were all gone out in
the carriage with their mama.  Abbot, too, was sewing in another room,
and Bessie, as she moved hither and thither, putting away toys and
arranging drawers, addressed to me every now and then a word of unwonted
kindness.  This state of things should have been to me a paradise of
peace, accustomed as I was to a life of ceaseless reprimand and thankless
fagging; but, in fact, my racked nerves were now in such a state that no
calm could soothe, and no pleasure excite them agreeably.

Bessie had been down into the kitchen, and she brought up with her a tart
on a certain brightly painted china plate, whose bird of paradise,
nestling in a wreath of convolvuli and rosebuds, had been wont to stir in
me a most enthusiastic sense of admiration; and which plate I had often
petitioned to be allowed to take in my hand in order to examine it more
closely, but had always hitherto been deemed unworthy of such a
privilege.  This precious vessel was now placed on my knee, and I was
cordially invited to eat the circlet of delicate pastry upon it.  Vain
favour! coming, like most other favours long deferred and often wished
for, too late!  I could not eat the tart; and the plumage of the bird,
the tints of the flowers, seemed strangely faded: I put both plate and
tart away.  Bessie asked if I would have a book: the word \emph{book} acted as
a transient stimulus, and I begged her to fetch Gulliver's Travels from
the library.  This book I had again and again perused with delight.  I
considered it a narrative of facts, and discovered in it a vein of
interest deeper than what I found in fairy tales: for as to the elves,
having sought them in vain among foxglove leaves and bells, under
mushrooms and beneath the ground-ivy mantling old wall-nooks, I had at
length made up my mind to the sad truth, that they were all gone out of
England to some savage country where the woods were wilder and thicker,
and the population more scant; whereas, Lilliput and Brobdignag being, in
my creed, solid parts of the earth's surface, I doubted not that I might
one day, by taking a long voyage, see with my own eyes the little fields,
houses, and trees, the diminutive people, the tiny cows, sheep, and birds
of the one realm; and the corn-fields forest-high, the mighty mastiffs,
the monster cats, the tower-like men and women, of the other.  Yet, when
this cherished volume was now placed in my hand--when I turned over its
leaves, and sought in its marvellous pictures the charm I had, till now,
never failed to find--all was eerie and dreary; the giants were gaunt
goblins, the pigmies malevolent and fearful imps, Gulliver a most
desolate wanderer in most dread and dangerous regions.  I closed the
book, which I dared no longer peruse, and put it on the table, beside the
untasted tart.

Bessie had now finished dusting and tidying the room, and having washed
her hands, she opened a certain little drawer, full of splendid shreds of
silk and satin, and began making a new bonnet for Georgiana's doll.
Meantime she sang: her song was--

   "In the days when we went gipsying,
      A long time ago."

I had often heard the song before, and always with lively delight; for
Bessie had a sweet voice,--at least, I thought so.  But now, though her
voice was still sweet, I found in its melody an indescribable sadness.
Sometimes, preoccupied with her work, she sang the refrain very low, very
lingeringly; "A long time ago" came out like the saddest cadence of a
funeral hymn.  She passed into another ballad, this time a really doleful
one.

   "My feet they are sore, and my limbs they are weary;
      Long is the way, and the mountains are wild;
   Soon will the twilight close moonless and dreary
      Over the path of the poor orphan child.

   Why did they send me so far and so lonely,
      Up where the moors spread and grey rocks are piled?
   Men are hard-hearted, and kind angels only
      Watch o'er the steps of a poor orphan child.

   Yet distant and soft the night breeze is blowing,
      Clouds there are none, and clear stars beam mild,
   God, in His mercy, protection is showing,
      Comfort and hope to the poor orphan child.

   Ev'n should I fall o'er the broken bridge passing,
      Or stray in the marshes, by false lights beguiled,
   Still will my Father, with promise and blessing,
      Take to His bosom the poor orphan child.

   There is a thought that for strength should avail me,
      Though both of shelter and kindred despoiled;
   Heaven is a home, and a rest will not fail me;
      God is a friend to the poor orphan child."

"Come, Miss Jane, don't cry," said Bessie as she finished.  She might as
well have said to the fire, "don't burn!" but how could she divine the
morbid suffering to which I was a prey?  In the course of the morning Mr.
Lloyd came again.

"What, already up!" said he, as he entered the nursery.  "Well, nurse,
how is she?"

Bessie answered that I was doing very well.

"Then she ought to look more cheerful.  Come here, Miss Jane: your name
is Jane, is it not?"

"Yes, sir, Jane Eyre."

"Well, you have been crying, Miss Jane Eyre; can you tell me what about?
Have you any pain?"

"No, sir."

"Oh!  I daresay she is crying because she could not go out with Missis in
the carriage," interposed Bessie.

"Surely not! why, she is too old for such pettishness."

I thought so too; and my self-esteem being wounded by the false charge, I
answered promptly, "I never cried for such a thing in my life: I hate
going out in the carriage.  I cry because I am miserable."

"Oh fie, Miss!" said Bessie.

The good apothecary appeared a little puzzled.  I was standing before
him; he fixed his eyes on me very steadily: his eyes were small and grey;
not very bright, but I dare say I should think them shrewd now: he had a
hard-featured yet good-natured looking face.  Having considered me at
leisure, he said--

"What made you ill yesterday?"

"She had a fall," said Bessie, again putting in her word.

"Fall! why, that is like a baby again!  Can't she manage to walk at her
age?  She must be eight or nine years old."

"I was knocked down," was the blunt explanation, jerked out of me by
another pang of mortified pride; "but that did not make me ill," I added;
while Mr. Lloyd helped himself to a pinch of snuff.

As he was returning the box to his waistcoat pocket, a loud bell rang for
the servants' dinner; he knew what it was.  "That's for you, nurse," said
he; "you can go down; I'll give Miss Jane a lecture till you come back."

Bessie would rather have stayed, but she was obliged to go, because
punctuality at meals was rigidly enforced at Gateshead Hall.

"The fall did not make you ill; what did, then?" pursued Mr. Lloyd when
Bessie was gone.

"I was shut up in a room where there is a ghost till after dark."

I saw Mr. Lloyd smile and frown at the same time.

"Ghost!  What, you are a baby after all!  You are afraid of ghosts?"

"Of Mr. Reed's ghost I am: he died in that room, and was laid out there.
Neither Bessie nor any one else will go into it at night, if they can
help it; and it was cruel to shut me up alone without a candle,--so cruel
that I think I shall never forget it."

"Nonsense!  And is it that makes you so miserable?  Are you afraid now in
daylight?"

"No: but night will come again before long: and besides,--I am
unhappy,--very unhappy, for other things."

"What other things?  Can you tell me some of them?"

How much I wished to reply fully to this question!  How difficult it was
to frame any answer!  Children can feel, but they cannot analyse their
feelings; and if the analysis is partially effected in thought, they know
not how to express the result of the process in words.  Fearful, however,
of losing this first and only opportunity of relieving my grief by
imparting it, I, after a disturbed pause, contrived to frame a meagre,
though, as far as it went, true response.

"For one thing, I have no father or mother, brothers or sisters."

"You have a kind aunt and cousins."

Again I paused; then bunglingly enounced--

"But John Reed knocked me down, and my aunt shut me up in the red-room."

Mr. Lloyd a second time produced his snuff-box.

"Don't you think Gateshead Hall a very beautiful house?" asked he.  "Are
you not very thankful to have such a fine place to live at?"

"It is not my house, sir; and Abbot says I have less right to be here
than a servant."

"Pooh! you can't be silly enough to wish to leave such a splendid place?"

"If I had anywhere else to go, I should be glad to leave it; but I can
never get away from Gateshead till I am a woman."

"Perhaps you may--who knows?  Have you any relations besides Mrs. Reed?"

"I think not, sir."

"None belonging to your father?"

"I don't know.  I asked Aunt Reed once, and she said possibly I might
have some poor, low relations called Eyre, but she knew nothing about
them."

"If you had such, would you like to go to them?"

I reflected.  Poverty looks grim to grown people; still more so to
children: they have not much idea of industrious, working, respectable
poverty; they think of the word only as connected with ragged clothes,
scanty food, fireless grates, rude manners, and debasing vices: poverty
for me was synonymous with degradation.

"No; I should not like to belong to poor people," was my reply.

"Not even if they were kind to you?"

I shook my head: I could not see how poor people had the means of being
kind; and then to learn to speak like them, to adopt their manners, to be
uneducated, to grow up like one of the poor women I saw sometimes nursing
their children or washing their clothes at the cottage doors of the
village of Gateshead: no, I was not heroic enough to purchase liberty at
the price of caste.

"But are your relatives so very poor?  Are they working people?"

"I cannot tell; Aunt Reed says if I have any, they must be a beggarly
set: I should not like to go a begging."

"Would you like to go to school?"

Again I reflected: I scarcely knew what school was: Bessie sometimes
spoke of it as a place where young ladies sat in the stocks, wore
backboards, and were expected to be exceedingly genteel and precise: John
Reed hated his school, and abused his master; but John Reed's tastes were
no rule for mine, and if Bessie's accounts of school-discipline (gathered
from the young ladies of a family where she had lived before coming to
Gateshead) were somewhat appalling, her details of certain
accomplishments attained by these same young ladies were, I thought,
equally attractive.  She boasted of beautiful paintings of landscapes and
flowers by them executed; of songs they could sing and pieces they could
play, of purses they could net, of French books they could translate;
till my spirit was moved to emulation as I listened.  Besides, school
would be a complete change: it implied a long journey, an entire
separation from Gateshead, an entrance into a new life.

"I should indeed like to go to school," was the audible conclusion of my
musings.

"Well, well! who knows what may happen?" said Mr. Lloyd, as he got up.
"The child ought to have change of air and scene," he added, speaking to
himself; "nerves not in a good state."

Bessie now returned; at the same moment the carriage was heard rolling up
the gravel-walk.

"Is that your mistress, nurse?" asked Mr. Lloyd.  "I should like to speak
to her before I go."

Bessie invited him to walk into the breakfast-room, and led the way out.
In the interview which followed between him and Mrs. Reed, I presume,
from after-occurrences, that the apothecary ventured to recommend my
being sent to school; and the recommendation was no doubt readily enough
adopted; for as Abbot said, in discussing the subject with Bessie when
both sat sewing in the nursery one night, after I was in bed, and, as
they thought, asleep, "Missis was, she dared say, glad enough to get rid
of such a tiresome, ill-conditioned child, who always looked as if she
were watching everybody, and scheming plots underhand."  Abbot, I think,
gave me credit for being a sort of infantine Guy Fawkes.

On that same occasion I learned, for the first time, from Miss Abbot's
communications to Bessie, that my father had been a poor clergyman; that
my mother had married him against the wishes of her friends, who
considered the match beneath her; that my grandfather Reed was so
irritated at her disobedience, he cut her off without a shilling; that
after my mother and father had been married a year, the latter caught the
typhus fever while visiting among the poor of a large manufacturing town
where his curacy was situated, and where that disease was then prevalent:
that my mother took the infection from him, and both died within a month
of each other.

Bessie, when she heard this narrative, sighed and said, "Poor Miss Jane
is to be pitied, too, Abbot."

"Yes," responded Abbot; "if she were a nice, pretty child, one might
compassionate her forlornness; but one really cannot care for such a
little toad as that."

"Not a great deal, to be sure," agreed Bessie: "at any rate, a beauty
like Miss Georgiana would be more moving in the same condition."

"Yes, I doat on Miss Georgiana!" cried the fervent Abbot.  "Little
darling!--with her long curls and her blue eyes, and such a sweet colour
as she has; just as if she were painted!--Bessie, I could fancy a Welsh
rabbit for supper."

"So could I--with a roast onion.  Come, we'll go down."  They went.




\chapter{}


\lettrine{F}{rom} my discourse with Mr. Lloyd, and from the above reported conference
between Bessie and Abbot, I gathered enough of hope to suffice as a
motive for wishing to get well: a change seemed near,--I desired and
waited it in silence.  It tarried, however: days and weeks passed: I had
regained my normal state of health, but no new allusion was made to the
subject over which I brooded.  Mrs. Reed surveyed me at times with a
severe eye, but seldom addressed me: since my illness, she had drawn a
more marked line of separation than ever between me and her own children;
appointing me a small closet to sleep in by myself, condemning me to take
my meals alone, and pass all my time in the nursery, while my cousins
were constantly in the drawing-room.  Not a hint, however, did she drop
about sending me to school: still I felt an instinctive certainty that
she would not long endure me under the same roof with her; for her
glance, now more than ever, when turned on me, expressed an insuperable
and rooted aversion.

Eliza and Georgiana, evidently acting according to orders, spoke to me as
little as possible: John thrust his tongue in his cheek whenever he saw
me, and once attempted chastisement; but as I instantly turned against
him, roused by the same sentiment of deep ire and desperate revolt which
had stirred my corruption before, he thought it better to desist, and ran
from me tittering execrations, and vowing I had burst his nose.  I had
indeed levelled at that prominent feature as hard a blow as my knuckles
could inflict; and when I saw that either that or my look daunted him, I
had the greatest inclination to follow up my advantage to purpose; but he
was already with his mama.  I heard him in a blubbering tone commence the
tale of how "that nasty Jane Eyre" had flown at him like a mad cat: he
was stopped rather harshly--

"Don't talk to me about her, John: I told you not to go near her; she is
not worthy of notice; I do not choose that either you or your sisters
should associate with her."

Here, leaning over the banister, I cried out suddenly, and without at all
deliberating on my words--

"They are not fit to associate with me."

Mrs. Reed was rather a stout woman; but, on hearing this strange and
audacious declaration, she ran nimbly up the stair, swept me like a
whirlwind into the nursery, and crushing me down on the edge of my crib,
dared me in an emphatic voice to rise from that place, or utter one
syllable during the remainder of the day.

"What would Uncle Reed say to you, if he were alive?" was my scarcely
voluntary demand.  I say scarcely voluntary, for it seemed as if my
tongue pronounced words without my will consenting to their utterance:
something spoke out of me over which I had no control.

"What?" said Mrs. Reed under her breath: her usually cold composed grey
eye became troubled with a look like fear; she took her hand from my arm,
and gazed at me as if she really did not know whether I were child or
fiend.  I was now in for it.

"My Uncle Reed is in heaven, and can see all you do and think; and so can
papa and mama: they know how you shut me up all day long, and how you
wish me dead."

Mrs. Reed soon rallied her spirits: she shook me most soundly, she boxed
both my ears, and then left me without a word.  Bessie supplied the
hiatus by a homily of an hour's length, in which she proved beyond a
doubt that I was the most wicked and abandoned child ever reared under a
roof.  I half believed her; for I felt indeed only bad feelings surging
in my breast.

November, December, and half of January passed away.  Christmas and the
New Year had been celebrated at Gateshead with the usual festive cheer;
presents had been interchanged, dinners and evening parties given.  From
every enjoyment I was, of course, excluded: my share of the gaiety
consisted in witnessing the daily apparelling of Eliza and Georgiana, and
seeing them descend to the drawing-room, dressed out in thin muslin
frocks and scarlet sashes, with hair elaborately ringletted; and
afterwards, in listening to the sound of the piano or the harp played
below, to the passing to and fro of the butler and footman, to the
jingling of glass and china as refreshments were handed, to the broken
hum of conversation as the drawing-room door opened and closed.  When
tired of this occupation, I would retire from the stairhead to the
solitary and silent nursery: there, though somewhat sad, I was not
miserable.  To speak truth, I had not the least wish to go into company,
for in company I was very rarely noticed; and if Bessie had but been kind
and companionable, I should have deemed it a treat to spend the evenings
quietly with her, instead of passing them under the formidable eye of
Mrs. Reed, in a room full of ladies and gentlemen.  But Bessie, as soon
as she had dressed her young ladies, used to take herself off to the
lively regions of the kitchen and housekeeper's room, generally bearing
the candle along with her.  I then sat with my doll on my knee till the
fire got low, glancing round occasionally to make sure that nothing worse
than myself haunted the shadowy room; and when the embers sank to a dull
red, I undressed hastily, tugging at knots and strings as I best might,
and sought shelter from cold and darkness in my crib.  To this crib I
always took my doll; human beings must love something, and, in the dearth
of worthier objects of affection, I contrived to find a pleasure in
loving and cherishing a faded graven image, shabby as a miniature
scarecrow.  It puzzles me now to remember with what absurd sincerity I
doated on this little toy, half fancying it alive and capable of
sensation.  I could not sleep unless it was folded in my night-gown; and
when it lay there safe and warm, I was comparatively happy, believing it
to be happy likewise.

Long did the hours seem while I waited the departure of the company, and
listened for the sound of Bessie's step on the stairs: sometimes she
would come up in the interval to seek her thimble or her scissors, or
perhaps to bring me something by way of supper--a bun or a
cheese-cake--then she would sit on the bed while I ate it, and when I had
finished, she would tuck the clothes round me, and twice she kissed me,
and said, "Good night, Miss Jane."  When thus gentle, Bessie seemed to me
the best, prettiest, kindest being in the world; and I wished most
intensely that she would always be so pleasant and amiable, and never
push me about, or scold, or task me unreasonably, as she was too often
wont to do.  Bessie Lee must, I think, have been a girl of good natural
capacity, for she was smart in all she did, and had a remarkable knack of
narrative; so, at least, I judge from the impression made on me by her
nursery tales.  She was pretty too, if my recollections of her face and
person are correct.  I remember her as a slim young woman, with black
hair, dark eyes, very nice features, and good, clear complexion; but she
had a capricious and hasty temper, and indifferent ideas of principle or
justice: still, such as she was, I preferred her to any one else at
Gateshead Hall.

It was the fifteenth of January, about nine o'clock in the morning:
Bessie was gone down to breakfast; my cousins had not yet been summoned
to their mama; Eliza was putting on her bonnet and warm garden-coat to go
and feed her poultry, an occupation of which she was fond: and not less
so of selling the eggs to the housekeeper and hoarding up the money she
thus obtained.  She had a turn for traffic, and a marked propensity for
saving; shown not only in the vending of eggs and chickens, but also in
driving hard bargains with the gardener about flower-roots, seeds, and
slips of plants; that functionary having orders from Mrs. Reed to buy of
his young lady all the products of her parterre she wished to sell: and
Eliza would have sold the hair off her head if she could have made a
handsome profit thereby.  As to her money, she first secreted it in odd
corners, wrapped in a rag or an old curl-paper; but some of these hoards
having been discovered by the housemaid, Eliza, fearful of one day losing
her valued treasure, consented to intrust it to her mother, at a usurious
rate of interest--fifty or sixty per cent.; which interest she exacted
every quarter, keeping her accounts in a little book with anxious
accuracy.

Georgiana sat on a high stool, dressing her hair at the glass, and
interweaving her curls with artificial flowers and faded feathers, of
which she had found a store in a drawer in the attic.  I was making my
bed, having received strict orders from Bessie to get it arranged before
she returned (for Bessie now frequently employed me as a sort of under-
nurserymaid, to tidy the room, dust the chairs, \&c.).  Having spread the
quilt and folded my night-dress, I went to the window-seat to put in
order some picture-books and doll's house furniture scattered there; an
abrupt command from Georgiana to let her playthings alone (for the tiny
chairs and mirrors, the fairy plates and cups, were her property) stopped
my proceedings; and then, for lack of other occupation, I fell to
breathing on the frost-flowers with which the window was fretted, and
thus clearing a space in the glass through which I might look out on the
grounds, where all was still and petrified under the influence of a hard
frost.

From this window were visible the porter's lodge and the carriage-road,
and just as I had dissolved so much of the silver-white foliage veiling
the panes as left room to look out, I saw the gates thrown open and a
carriage roll through.  I watched it ascending the drive with
indifference; carriages often came to Gateshead, but none ever brought
visitors in whom I was interested; it stopped in front of the house, the
door-bell rang loudly, the new-comer was admitted.  All this being
nothing to me, my vacant attention soon found livelier attraction in the
spectacle of a little hungry robin, which came and chirruped on the twigs
of the leafless cherry-tree nailed against the wall near the casement.
The remains of my breakfast of bread and milk stood on the table, and
having crumbled a morsel of roll, I was tugging at the sash to put out
the crumbs on the window-sill, when Bessie came running upstairs into the
nursery.

"Miss Jane, take off your pinafore; what are you doing there?  Have you
washed your hands and face this morning?"  I gave another tug before I
answered, for I wanted the bird to be secure of its bread: the sash
yielded; I scattered the crumbs, some on the stone sill, some on the
cherry-tree bough, then, closing the window, I replied--

"No, Bessie; I have only just finished dusting."

"Troublesome, careless child! and what are you doing now?  You look quite
red, as if you had been about some mischief: what were you opening the
window for?"

I was spared the trouble of answering, for Bessie seemed in too great a
hurry to listen to explanations; she hauled me to the washstand,
inflicted a merciless, but happily brief scrub on my face and hands with
soap, water, and a coarse towel; disciplined my head with a bristly
brush, denuded me of my pinafore, and then hurrying me to the top of the
stairs, bid me go down directly, as I was wanted in the breakfast-room.

I would have asked who wanted me: I would have demanded if Mrs. Reed was
there; but Bessie was already gone, and had closed the nursery-door upon
me.  I slowly descended.  For nearly three months, I had never been
called to Mrs. Reed's presence; restricted so long to the nursery, the
breakfast, dining, and drawing-rooms were become for me awful regions, on
which it dismayed me to intrude.

I now stood in the empty hall; before me was the breakfast-room door, and
I stopped, intimidated and trembling.  What a miserable little poltroon
had fear, engendered of unjust punishment, made of me in those days!  I
feared to return to the nursery, and feared to go forward to the parlour;
ten minutes I stood in agitated hesitation; the vehement ringing of the
breakfast-room bell decided me; I \emph{must} enter.

"Who could want me?" I asked inwardly, as with both hands I turned the
stiff door-handle, which, for a second or two, resisted my efforts.  "What
should I see besides Aunt Reed in the apartment?--a man or a woman?"  The
handle turned, the door unclosed, and passing through and curtseying low,
I looked up at--a black pillar!--such, at least, appeared to me, at first
sight, the straight, narrow, sable-clad shape standing erect on the rug:
the grim face at the top was like a carved mask, placed above the shaft
by way of capital.

Mrs. Reed occupied her usual seat by the fireside; she made a signal to
me to approach; I did so, and she introduced me to the stony stranger
with the words: "This is the little girl respecting whom I applied to
you."

\emph{He}, for it was a man, turned his head slowly towards where I stood, and
having examined me with the two inquisitive-looking grey eyes which
twinkled under a pair of bushy brows, said solemnly, and in a bass voice,
"Her size is small: what is her age?"

"Ten years."

"So much?" was the doubtful answer; and he prolonged his scrutiny for
some minutes.  Presently he addressed me--"Your name, little girl?"

"Jane Eyre, sir."

In uttering these words I looked up: he seemed to me a tall gentleman;
but then I was very little; his features were large, and they and all the
lines of his frame were equally harsh and prim.

"Well, Jane Eyre, and are you a good child?"

Impossible to reply to this in the affirmative: my little world held a
contrary opinion: I was silent.  Mrs. Reed answered for me by an
expressive shake of the head, adding soon, "Perhaps the less said on that
subject the better, Mr. Brocklehurst."

"Sorry indeed to hear it! she and I must have some talk;" and bending
from the perpendicular, he installed his person in the arm-chair opposite
Mrs. Reed's.  "Come here," he said.

I stepped across the rug; he placed me square and straight before him.
What a face he had, now that it was almost on a level with mine! what a
great nose! and what a mouth! and what large prominent teeth!

"No sight so sad as that of a naughty child," he began, "especially a
naughty little girl.  Do you know where the wicked go after death?"

"They go to hell," was my ready and orthodox answer.

"And what is hell?  Can you tell me that?"

"A pit full of fire."

"And should you like to fall into that pit, and to be burning there for
ever?"

"No, sir."

"What must you do to avoid it?"

I deliberated a moment; my answer, when it did come, was objectionable:
"I must keep in good health, and not die."

"How can you keep in good health?  Children younger than you die daily.  I
buried a little child of five years old only a day or two since,--a good
little child, whose soul is now in heaven.  It is to be feared the same
could not be said of you were you to be called hence."

Not being in a condition to remove his doubt, I only cast my eyes down on
the two large feet planted on the rug, and sighed, wishing myself far
enough away.

"I hope that sigh is from the heart, and that you repent of ever having
been the occasion of discomfort to your excellent benefactress."

"Benefactress! benefactress!" said I inwardly: "they all call Mrs. Reed
my benefactress; if so, a benefactress is a disagreeable thing."

"Do you say your prayers night and morning?" continued my interrogator.

"Yes, sir."

"Do you read your Bible?"

"Sometimes."

"With pleasure?  Are you fond of it?"

"I like Revelations, and the book of Daniel, and Genesis and Samuel, and
a little bit of Exodus, and some parts of Kings and Chronicles, and Job
and Jonah."

"And the Psalms?  I hope you like them?"

"No, sir."

"No? oh, shocking!  I have a little boy, younger than you, who knows six
Psalms by heart: and when you ask him which he would rather have, a
gingerbread-nut to eat or a verse of a Psalm to learn, he says: 'Oh! the
verse of a Psalm! angels sing Psalms;' says he, 'I wish to be a little
angel here below;' he then gets two nuts in recompense for his infant
piety."

"Psalms are not interesting," I remarked.

"That proves you have a wicked heart; and you must pray to God to change
it: to give you a new and clean one: to take away your heart of stone and
give you a heart of flesh."

I was about to propound a question, touching the manner in which that
operation of changing my heart was to be performed, when Mrs. Reed
interposed, telling me to sit down; she then proceeded to carry on the
conversation herself.

"Mr. Brocklehurst, I believe I intimated in the letter which I wrote to
you three weeks ago, that this little girl has not quite the character
and disposition I could wish: should you admit her into Lowood school, I
should be glad if the superintendent and teachers were requested to keep
a strict eye on her, and, above all, to guard against her worst fault, a
tendency to deceit.  I mention this in your hearing, Jane, that you may
not attempt to impose on Mr. Brocklehurst."

Well might I dread, well might I dislike Mrs. Reed; for it was her nature
to wound me cruelly; never was I happy in her presence; however carefully
I obeyed, however strenuously I strove to please her, my efforts were
still repulsed and repaid by such sentences as the above.  Now, uttered
before a stranger, the accusation cut me to the heart; I dimly perceived
that she was already obliterating hope from the new phase of existence
which she destined me to enter; I felt, though I could not have expressed
the feeling, that she was sowing aversion and unkindness along my future
path; I saw myself transformed under Mr. Brocklehurst's eye into an
artful, noxious child, and what could I do to remedy the injury?

"Nothing, indeed," thought I, as I struggled to repress a sob, and
hastily wiped away some tears, the impotent evidences of my anguish.

"Deceit is, indeed, a sad fault in a child," said Mr. Brocklehurst; "it
is akin to falsehood, and all liars will have their portion in the lake
burning with fire and brimstone; she shall, however, be watched, Mrs.
Reed.  I will speak to Miss Temple and the teachers."

"I should wish her to be brought up in a manner suiting her prospects,"
continued my benefactress; "to be made useful, to be kept humble: as for
the vacations, she will, with your permission, spend them always at
Lowood."

"Your decisions are perfectly judicious, madam," returned Mr.
Brocklehurst.  "Humility is a Christian grace, and one peculiarly
appropriate to the pupils of Lowood; I, therefore, direct that especial
care shall be bestowed on its cultivation amongst them.  I have studied
how best to mortify in them the worldly sentiment of pride; and, only the
other day, I had a pleasing proof of my success.  My second daughter,
Augusta, went with her mama to visit the school, and on her return she
exclaimed: 'Oh, dear papa, how quiet and plain all the girls at Lowood
look, with their hair combed behind their ears, and their long pinafores,
and those little holland pockets outside their frocks--they are almost
like poor people's children! and,' said she, 'they looked at my dress and
mama's, as if they had never seen a silk gown before.'"

"This is the state of things I quite approve," returned Mrs. Reed; "had I
sought all England over, I could scarcely have found a system more
exactly fitting a child like Jane Eyre.  Consistency, my dear Mr.
Brocklehurst; I advocate consistency in all things."

"Consistency, madam, is the first of Christian duties; and it has been
observed in every arrangement connected with the establishment of Lowood:
plain fare, simple attire, unsophisticated accommodations, hardy and
active habits; such is the order of the day in the house and its
inhabitants."

"Quite right, sir.  I may then depend upon this child being received as a
pupil at Lowood, and there being trained in conformity to her position
and prospects?"

"Madam, you may: she shall be placed in that nursery of chosen plants,
and I trust she will show herself grateful for the inestimable privilege
of her election."

"I will send her, then, as soon as possible, Mr. Brocklehurst; for, I
assure you, I feel anxious to be relieved of a responsibility that was
becoming too irksome."

"No doubt, no doubt, madam; and now I wish you good morning.  I shall
return to Brocklehurst Hall in the course of a week or two: my good
friend, the Archdeacon, will not permit me to leave him sooner.  I shall
send Miss Temple notice that she is to expect a new girl, so that there
will be no difficulty about receiving her.  Good-bye."

"Good-bye, Mr. Brocklehurst; remember me to Mrs. and Miss Brocklehurst,
and to Augusta and Theodore, and Master Broughton Brocklehurst."

"I will, madam.  Little girl, here is a book entitled the 'Child's
Guide,' read it with prayer, especially that part containing 'An account
of the awfully sudden death of Martha G---, a naughty child addicted to
falsehood and deceit.'"

With these words Mr. Brocklehurst put into my hand a thin pamphlet sewn
in a cover, and having rung for his carriage, he departed.

Mrs. Reed and I were left alone: some minutes passed in silence; she was
sewing, I was watching her.  Mrs. Reed might be at that time some six or
seven and thirty; she was a woman of robust frame, square-shouldered and
strong-limbed, not tall, and, though stout, not obese: she had a somewhat
large face, the under jaw being much developed and very solid; her brow
was low, her chin large and prominent, mouth and nose sufficiently
regular; under her light eyebrows glimmered an eye devoid of ruth; her
skin was dark and opaque, her hair nearly flaxen; her constitution was
sound as a bell--illness never came near her; she was an exact, clever
manager; her household and tenantry were thoroughly under her control;
her children only at times defied her authority and laughed it to scorn;
she dressed well, and had a presence and port calculated to set off
handsome attire.

Sitting on a low stool, a few yards from her arm-chair, I examined her
figure; I perused her features.  In my hand I held the tract containing
the sudden death of the Liar, to which narrative my attention had been
pointed as to an appropriate warning.  What had just passed; what Mrs.
Reed had said concerning me to Mr. Brocklehurst; the whole tenor of their
conversation, was recent, raw, and stinging in my mind; I had felt every
word as acutely as I had heard it plainly, and a passion of resentment
fomented now within me.

Mrs. Reed looked up from her work; her eye settled on mine, her fingers
at the same time suspended their nimble movements.

"Go out of the room; return to the nursery," was her mandate.  My look or
something else must have struck her as offensive, for she spoke with
extreme though suppressed irritation.  I got up, I went to the door; I
came back again; I walked to the window, across the room, then close up
to her.

\emph{Speak} I must: I had been trodden on severely, and \emph{must} turn: but how?
What strength had I to dart retaliation at my antagonist?  I gathered my
energies and launched them in this blunt sentence--

"I am not deceitful: if I were, I should say I loved you; but I declare I
do not love you: I dislike you the worst of anybody in the world except
John Reed; and this book about the liar, you may give to your girl,
Georgiana, for it is she who tells lies, and not I."

Mrs. Reed's hands still lay on her work inactive: her eye of ice
continued to dwell freezingly on mine.

"What more have you to say?" she asked, rather in the tone in which a
person might address an opponent of adult age than such as is ordinarily
used to a child.

That eye of hers, that voice stirred every antipathy I had.  Shaking from
head to foot, thrilled with ungovernable excitement, I continued--

"I am glad you are no relation of mine: I will never call you aunt again
as long as I live.  I will never come to see you when I am grown up; and
if any one asks me how I liked you, and how you treated me, I will say
the very thought of you makes me sick, and that you treated me with
miserable cruelty."

"How dare you affirm that, Jane Eyre?"

"How dare I, Mrs. Reed?  How dare I?  Because it is the \emph{truth}.  You
think I have no feelings, and that I can do without one bit of love or
kindness; but I cannot live so: and you have no pity.  I shall remember
how you thrust me back--roughly and violently thrust me back--into the
red-room, and locked me up there, to my dying day; though I was in agony;
though I cried out, while suffocating with distress, 'Have mercy!  Have
mercy, Aunt Reed!'  And that punishment you made me suffer because your
wicked boy struck me--knocked me down for nothing.  I will tell anybody
who asks me questions, this exact tale.  People think you a good woman,
but you are bad, hard-hearted.  \emph{You} are deceitful!"

{How dare I, Mrs. Ried?  How dare I?  Because it is the truth: p30.jpg}

Ere I had finished this reply, my soul began to expand, to exult, with
the strangest sense of freedom, of triumph, I ever felt.  It seemed as if
an invisible bond had burst, and that I had struggled out into unhoped-
for liberty.  Not without cause was this sentiment: Mrs. Reed looked
frightened; her work had slipped from her knee; she was lifting up her
hands, rocking herself to and fro, and even twisting her face as if she
would cry.

"Jane, you are under a mistake: what is the matter with you?  Why do you
tremble so violently?  Would you like to drink some water?"

"No, Mrs. Reed."

"Is there anything else you wish for, Jane?  I assure you, I desire to be
your friend."

"Not you.  You told Mr. Brocklehurst I had a bad character, a deceitful
disposition; and I'll let everybody at Lowood know what you are, and what
you have done."

"Jane, you don't understand these things: children must be corrected for
their faults."

"Deceit is not my fault!" I cried out in a savage, high voice.

"But you are passionate, Jane, that you must allow: and now return to the
nursery--there's a dear--and lie down a little."

"I am not your dear; I cannot lie down: send me to school soon, Mrs.
Reed, for I hate to live here."

"I will indeed send her to school soon," murmured Mrs. Reed \emph{sotto voce};
and gathering up her work, she abruptly quitted the apartment.

I was left there alone--winner of the field.  It was the hardest battle I
had fought, and the first victory I had gained: I stood awhile on the
rug, where Mr. Brocklehurst had stood, and I enjoyed my conqueror's
solitude.  First, I smiled to myself and felt elate; but this fierce
pleasure subsided in me as fast as did the accelerated throb of my
pulses.  A child cannot quarrel with its elders, as I had done; cannot
give its furious feelings uncontrolled play, as I had given mine, without
experiencing afterwards the pang of remorse and the chill of reaction.  A
ridge of lighted heath, alive, glancing, devouring, would have been a
meet emblem of my mind when I accused and menaced Mrs. Reed: the same
ridge, black and blasted after the flames are dead, would have
represented as meetly my subsequent condition, when half-an-hour's
silence and reflection had shown me the madness of my conduct, and the
dreariness of my hated and hating position.

Something of vengeance I had tasted for the first time; as aromatic wine
it seemed, on swallowing, warm and racy: its after-flavour, metallic and
corroding, gave me a sensation as if I had been poisoned.  Willingly
would I now have gone and asked Mrs. Reed's pardon; but I knew, partly
from experience and partly from instinct, that was the way to make her
repulse me with double scorn, thereby re-exciting every turbulent impulse
of my nature.

I would fain exercise some better faculty than that of fierce speaking;
fain find nourishment for some less fiendish feeling than that of sombre
indignation.  I took a book--some Arabian tales; I sat down and
endeavoured to read.  I could make no sense of the subject; my own
thoughts swam always between me and the page I had usually found
fascinating.  I opened the glass-door in the breakfast-room: the
shrubbery was quite still: the black frost reigned, unbroken by sun or
breeze, through the grounds.  I covered my head and arms with the skirt
of my frock, and went out to walk in a part of the plantation which was
quite sequestrated; but I found no pleasure in the silent trees, the
falling fir-cones, the congealed relics of autumn, russet leaves, swept
by past winds in heaps, and now stiffened together.  I leaned against a
gate, and looked into an empty field where no sheep were feeding, where
the short grass was nipped and blanched.  It was a very grey day; a most
opaque sky, "onding on snaw," canopied all; thence flakes felt it
intervals, which settled on the hard path and on the hoary lea without
melting.  I stood, a wretched child enough, whispering to myself over and
over again, "What shall I do?--what shall I do?"

All at once I heard a clear voice call, "Miss Jane! where are you?  Come
to lunch!"

It was Bessie, I knew well enough; but I did not stir; her light step
came tripping down the path.

"You naughty little thing!" she said.  "Why don't you come when you are
called?"

Bessie's presence, compared with the thoughts over which I had been
brooding, seemed cheerful; even though, as usual, she was somewhat cross.
The fact is, after my conflict with and victory over Mrs. Reed, I was not
disposed to care much for the nursemaid's transitory anger; and I \emph{was}
disposed to bask in her youthful lightness of heart.  I just put my two
arms round her and said, "Come, Bessie! don't scold."

The action was more frank and fearless than any I was habituated to
indulge in: somehow it pleased her.

"You are a strange child, Miss Jane," she said, as she looked down at me;
"a little roving, solitary thing: and you are going to school, I
suppose?"

I nodded.

"And won't you be sorry to leave poor Bessie?"

"What does Bessie care for me?  She is always scolding me."

"Because you're such a queer, frightened, shy little thing.  You should
be bolder."

"What! to get more knocks?"

"Nonsense!  But you are rather put upon, that's certain.  My mother said,
when she came to see me last week, that she would not like a little one
of her own to be in your place.--Now, come in, and I've some good news
for you."

"I don't think you have, Bessie."

"Child! what do you mean?  What sorrowful eyes you fix on me!  Well, but
Missis and the young ladies and Master John are going out to tea this
afternoon, and you shall have tea with me.  I'll ask cook to bake you a
little cake, and then you shall help me to look over your drawers; for I
am soon to pack your trunk.  Missis intends you to leave Gateshead in a
day or two, and you shall choose what toys you like to take with you."

"Bessie, you must promise not to scold me any more till I go."

"Well, I will; but mind you are a very good girl, and don't be afraid of
me.  Don't start when I chance to speak rather sharply; it's so
provoking."

"I don't think I shall ever be afraid of you again, Bessie, because I
have got used to you, and I shall soon have another set of people to
dread."

"If you dread them they'll dislike you."

"As you do, Bessie?"

"I don't dislike you, Miss; I believe I am fonder of you than of all the
others."

"You don't show it."

"You little sharp thing! you've got quite a new way of talking.  What
makes you so venturesome and hardy?"

"Why, I shall soon be away from you, and besides"--I was going to say
something about what had passed between me and Mrs. Reed, but on second
thoughts I considered it better to remain silent on that head.

"And so you're glad to leave me?"

"Not at all, Bessie; indeed, just now I'm rather sorry."

"Just now! and rather!  How coolly my little lady says it!  I dare say
now if I were to ask you for a kiss you wouldn't give it me: you'd say
you'd \emph{rather} not."

"I'll kiss you and welcome: bend your head down."  Bessie stooped; we
mutually embraced, and I followed her into the house quite comforted.
That afternoon lapsed in peace and harmony; and in the evening Bessie
told me some of her most enchanting stories, and sang me some of her
sweetest songs.  Even for me life had its gleams of sunshine.




\chapter{}


\lettrine{F}{ive} o'clock had hardly struck on the morning of the 19th of January,
when Bessie brought a candle into my closet and found me already up and
nearly dressed.  I had risen half-an-hour before her entrance, and had
washed my face, and put on my clothes by the light of a half-moon just
setting, whose rays streamed through the narrow window near my crib.  I
was to leave Gateshead that day by a coach which passed the lodge gates
at six a.m.  Bessie was the only person yet risen; she had lit a fire in
the nursery, where she now proceeded to make my breakfast.  Few children
can eat when excited with the thoughts of a journey; nor could I.  Bessie,
having pressed me in vain to take a few spoonfuls of the boiled milk and
bread she had prepared for me, wrapped up some biscuits in a paper and
put them into my bag; then she helped me on with my pelisse and bonnet,
and wrapping herself in a shawl, she and I left the nursery.  As we
passed Mrs. Reed's bedroom, she said, "Will you go in and bid Missis good-
bye?"

"No, Bessie: she came to my crib last night when you were gone down to
supper, and said I need not disturb her in the morning, or my cousins
either; and she told me to remember that she had always been my best
friend, and to speak of her and be grateful to her accordingly."

"What did you say, Miss?"

"Nothing: I covered my face with the bedclothes, and turned from her to
the wall."

"That was wrong, Miss Jane."

"It was quite right, Bessie.  Your Missis has not been my friend: she has
been my foe."

"O Miss Jane! don't say so!"

"Good-bye to Gateshead!" cried I, as we passed through the hall and went
out at the front door.

The moon was set, and it was very dark; Bessie carried a lantern, whose
light glanced on wet steps and gravel road sodden by a recent thaw.  Raw
and chill was the winter morning: my teeth chattered as I hastened down
the drive.  There was a light in the porter's lodge: when we reached it,
we found the porter's wife just kindling her fire: my trunk, which had
been carried down the evening before, stood corded at the door.  It
wanted but a few minutes of six, and shortly after that hour had struck,
the distant roll of wheels announced the coming coach; I went to the door
and watched its lamps approach rapidly through the gloom.

"Is she going by herself?" asked the porter's wife.

"Yes."

"And how far is it?"

"Fifty miles."

"What a long way!  I wonder Mrs. Reed is not afraid to trust her so far
alone."

The coach drew up; there it was at the gates with its four horses and its
top laden with passengers: the guard and coachman loudly urged haste; my
trunk was hoisted up; I was taken from Bessie's neck, to which I clung
with kisses.

"Be sure and take good care of her," cried she to the guard, as he lifted
me into the inside.

"Ay, ay!" was the answer: the door was slapped to, a voice exclaimed "All
right," and on we drove.  Thus was I severed from Bessie and Gateshead;
thus whirled away to unknown, and, as I then deemed, remote and
mysterious regions.

I remember but little of the journey; I only know that the day seemed to
me of a preternatural length, and that we appeared to travel over
hundreds of miles of road.  We passed through several towns, and in one,
a very large one, the coach stopped; the horses were taken out, and the
passengers alighted to dine.  I was carried into an inn, where the guard
wanted me to have some dinner; but, as I had no appetite, he left me in
an immense room with a fireplace at each end, a chandelier pendent from
the ceiling, and a little red gallery high up against the wall filled
with musical instruments.  Here I walked about for a long time, feeling
very strange, and mortally apprehensive of some one coming in and
kidnapping me; for I believed in kidnappers, their exploits having
frequently figured in Bessie's fireside chronicles.  At last the guard
returned; once more I was stowed away in the coach, my protector mounted
his own seat, sounded his hollow horn, and away we rattled over the
"stony street" of L-.

The afternoon came on wet and somewhat misty: as it waned into dusk, I
began to feel that we were getting very far indeed from Gateshead: we
ceased to pass through towns; the country changed; great grey hills
heaved up round the horizon: as twilight deepened, we descended a valley,
dark with wood, and long after night had overclouded the prospect, I
heard a wild wind rushing amongst trees.

Lulled by the sound, I at last dropped asleep; I had not long slumbered
when the sudden cessation of motion awoke me; the coach-door was open,
and a person like a servant was standing at it: I saw her face and dress
by the light of the lamps.

"Is there a little girl called Jane Eyre here?" she asked.  I answered
"Yes," and was then lifted out; my trunk was handed down, and the coach
instantly drove away.

I was stiff with long sitting, and bewildered with the noise and motion
of the coach: Gathering my faculties, I looked about me.  Rain, wind, and
darkness filled the air; nevertheless, I dimly discerned a wall before me
and a door open in it; through this door I passed with my new guide: she
shut and locked it behind her.  There was now visible a house or
houses--for the building spread far--with many windows, and lights
burning in some; we went up a broad pebbly path, splashing wet, and were
admitted at a door; then the servant led me through a passage into a room
with a fire, where she left me alone.

I stood and warmed my numbed fingers over the blaze, then I looked round;
there was no candle, but the uncertain light from the hearth showed, by
intervals, papered walls, carpet, curtains, shining mahogany furniture:
it was a parlour, not so spacious or splendid as the drawing-room at
Gateshead, but comfortable enough.  I was puzzling to make out the
subject of a picture on the wall, when the door opened, and an individual
carrying a light entered; another followed close behind.

The first was a tall lady with dark hair, dark eyes, and a pale and large
forehead; her figure was partly enveloped in a shawl, her countenance was
grave, her bearing erect.

"The child is very young to be sent alone," said she, putting her candle
down on the table.  She considered me attentively for a minute or two,
then further added--

"She had better be put to bed soon; she looks tired: are you tired?" she
asked, placing her hand on my shoulder.

"A little, ma'am."

"And hungry too, no doubt: let her have some supper before she goes to
bed, Miss Miller.  Is this the first time you have left your parents to
come to school, my little girl?"

I explained to her that I had no parents.  She inquired how long they had
been dead: then how old I was, what was my name, whether I could read,
write, and sew a little: then she touched my cheek gently with her
forefinger, and saying, "She hoped I should be a good child," dismissed
me along with Miss Miller.

The lady I had left might be about twenty-nine; the one who went with me
appeared some years younger: the first impressed me by her voice, look,
and air.  Miss Miller was more ordinary; ruddy in complexion, though of a
careworn countenance; hurried in gait and action, like one who had always
a multiplicity of tasks on hand: she looked, indeed, what I afterwards
found she really was, an under-teacher.  Led by her, I passed from
compartment to compartment, from passage to passage, of a large and
irregular building; till, emerging from the total and somewhat dreary
silence pervading that portion of the house we had traversed, we came
upon the hum of many voices, and presently entered a wide, long room,
with great deal tables, two at each end, on each of which burnt a pair of
candles, and seated all round on benches, a congregation of girls of
every age, from nine or ten to twenty.  Seen by the dim light of the
dips, their number to me appeared countless, though not in reality
exceeding eighty; they were uniformly dressed in brown stuff frocks of
quaint fashion, and long holland pinafores.  It was the hour of study;
they were engaged in conning over their to-morrow's task, and the hum I
had heard was the combined result of their whispered repetitions.

Miss Miller signed to me to sit on a bench near the door, then walking up
to the top of the long room she cried out--

"Monitors, collect the lesson-books and put them away!"

Four tall girls arose from different tables, and going round, gathered
the books and removed them.  Miss Miller again gave the word of command--

"Monitors, fetch the supper-trays!"

The tall girls went out and returned presently, each bearing a tray, with
portions of something, I knew not what, arranged thereon, and a pitcher
of water and mug in the middle of each tray.  The portions were handed
round; those who liked took a draught of the water, the mug being common
to all.  When it came to my turn, I drank, for I was thirsty, but did not
touch the food, excitement and fatigue rendering me incapable of eating:
I now saw, however, that it was a thin oaten cake shared into fragments.

The meal over, prayers were read by Miss Miller, and the classes filed
off, two and two, upstairs.  Overpowered by this time with weariness, I
scarcely noticed what sort of a place the bedroom was, except that, like
the schoolroom, I saw it was very long.  To-night I was to be Miss
Miller's bed-fellow; she helped me to undress: when laid down I glanced
at the long rows of beds, each of which was quickly filled with two
occupants; in ten minutes the single light was extinguished, and amidst
silence and complete darkness I fell asleep.

The night passed rapidly.  I was too tired even to dream; I only once
awoke to hear the wind rave in furious gusts, and the rain fall in
torrents, and to be sensible that Miss Miller had taken her place by my
side.  When I again unclosed my eyes, a loud bell was ringing; the girls
were up and dressing; day had not yet begun to dawn, and a rushlight or
two burned in the room.  I too rose reluctantly; it was bitter cold, and
I dressed as well as I could for shivering, and washed when there was a
basin at liberty, which did not occur soon, as there was but one basin to
six girls, on the stands down the middle of the room.  Again the bell
rang: all formed in file, two and two, and in that order descended the
stairs and entered the cold and dimly lit schoolroom: here prayers were
read by Miss Miller; afterwards she called out--

"Form classes!"

A great tumult succeeded for some minutes, during which Miss Miller
repeatedly exclaimed, "Silence!" and "Order!"  When it subsided, I saw
them all drawn up in four semicircles, before four chairs, placed at the
four tables; all held books in their hands, and a great book, like a
Bible, lay on each table, before the vacant seat.  A pause of some
seconds succeeded, filled up by the low, vague hum of numbers; Miss
Miller walked from class to class, hushing this indefinite sound.

A distant bell tinkled: immediately three ladies entered the room, each
walked to a table and took her seat.  Miss Miller assumed the fourth
vacant chair, which was that nearest the door, and around which the
smallest of the children were assembled: to this inferior class I was
called, and placed at the bottom of it.

Business now began, the day's Collect was repeated, then certain texts of
Scripture were said, and to these succeeded a protracted reading of
chapters in the Bible, which lasted an hour.  By the time that exercise
was terminated, day had fully dawned.  The indefatigable bell now sounded
for the fourth time: the classes were marshalled and marched into another
room to breakfast: how glad I was to behold a prospect of getting
something to eat!  I was now nearly sick from inanition, having taken so
little the day before.

The refectory was a great, low-ceiled, gloomy room; on two long tables
smoked basins of something hot, which, however, to my dismay, sent forth
an odour far from inviting.  I saw a universal manifestation of
discontent when the fumes of the repast met the nostrils of those
destined to swallow it; from the van of the procession, the tall girls of
the first class, rose the whispered words--

"Disgusting!  The porridge is burnt again!"

"Silence!" ejaculated a voice; not that of Miss Miller, but one of the
upper teachers, a little and dark personage, smartly dressed, but of
somewhat morose aspect, who installed herself at the top of one table,
while a more buxom lady presided at the other.  I looked in vain for her
I had first seen the night before; she was not visible: Miss Miller
occupied the foot of the table where I sat, and a strange,
foreign-looking, elderly lady, the French teacher, as I afterwards found,
took the corresponding seat at the other board.  A long grace was said
and a hymn sung; then a servant brought in some tea for the teachers, and
the meal began.

Ravenous, and now very faint, I devoured a spoonful or two of my portion
without thinking of its taste; but the first edge of hunger blunted, I
perceived I had got in hand a nauseous mess; burnt porridge is almost as
bad as rotten potatoes; famine itself soon sickens over it.  The spoons
were moved slowly: I saw each girl taste her food and try to swallow it;
but in most cases the effort was soon relinquished.  Breakfast was over,
and none had breakfasted.  Thanks being returned for what we had not got,
and a second hymn chanted, the refectory was evacuated for the
schoolroom.  I was one of the last to go out, and in passing the tables,
I saw one teacher take a basin of the porridge and taste it; she looked
at the others; all their countenances expressed displeasure, and one of
them, the stout one, whispered--

"Abominable stuff!  How shameful!"

A quarter of an hour passed before lessons again began, during which the
schoolroom was in a glorious tumult; for that space of time it seemed to
be permitted to talk loud and more freely, and they used their privilege.
The whole conversation ran on the breakfast, which one and all abused
roundly.  Poor things! it was the sole consolation they had.  Miss Miller
was now the only teacher in the room: a group of great girls standing
about her spoke with serious and sullen gestures.  I heard the name of
Mr. Brocklehurst pronounced by some lips; at which Miss Miller shook her
head disapprovingly; but she made no great effort to check the general
wrath; doubtless she shared in it.

A clock in the schoolroom struck nine; Miss Miller left her circle, and
standing in the middle of the room, cried--

"Silence!  To your seats!"

Discipline prevailed: in five minutes the confused throng was resolved
into order, and comparative silence quelled the Babel clamour of tongues.
The upper teachers now punctually resumed their posts: but still, all
seemed to wait.  Ranged on benches down the sides of the room, the eighty
girls sat motionless and erect; a quaint assemblage they appeared, all
with plain locks combed from their faces, not a curl visible; in brown
dresses, made high and surrounded by a narrow tucker about the throat,
with little pockets of holland (shaped something like a Highlander's
purse) tied in front of their frocks, and destined to serve the purpose
of a work-bag: all, too, wearing woollen stockings and country-made
shoes, fastened with brass buckles.  Above twenty of those clad in this
costume were full-grown girls, or rather young women; it suited them ill,
and gave an air of oddity even to the prettiest.

I was still looking at them, and also at intervals examining the
teachers--none of whom precisely pleased me; for the stout one was a
little coarse, the dark one not a little fierce, the foreigner harsh and
grotesque, and Miss Miller, poor thing! looked purple, weather-beaten,
and over-worked--when, as my eye wandered from face to face, the whole
school rose simultaneously, as if moved by a common spring.

What was the matter?  I had heard no order given: I was puzzled.  Ere I
had gathered my wits, the classes were again seated: but as all eyes were
now turned to one point, mine followed the general direction, and
encountered the personage who had received me last night.  She stood at
the bottom of the long room, on the hearth; for there was a fire at each
end; she surveyed the two rows of girls silently and gravely.  Miss
Miller approaching, seemed to ask her a question, and having received her
answer, went back to her place, and said aloud--

"Monitor of the first class, fetch the globes!"

While the direction was being executed, the lady consulted moved slowly
up the room.  I suppose I have a considerable organ of veneration, for I
retain yet the sense of admiring awe with which my eyes traced her steps.
Seen now, in broad daylight, she looked tall, fair, and shapely; brown
eyes with a benignant light in their irids, and a fine pencilling of long
lashes round, relieved the whiteness of her large front; on each of her
temples her hair, of a very dark brown, was clustered in round curls,
according to the fashion of those times, when neither smooth bands nor
long ringlets were in vogue; her dress, also in the mode of the day, was
of purple cloth, relieved by a sort of Spanish trimming of black velvet;
a gold watch (watches were not so common then as now) shone at her
girdle.  Let the reader add, to complete the picture, refined features; a
complexion, if pale, clear; and a stately air and carriage, and he will
have, at least, as clearly as words can give it, a correct idea of the
exterior of Miss Temple--Maria Temple, as I afterwards saw the name
written in a prayer-book intrusted to me to carry to church.

The superintendent of Lowood (for such was this lady) having taken her
seat before a pair of globes placed on one of the tables, summoned the
first class round her, and commenced giving a lesson on geography; the
lower classes were called by the teachers: repetitions in history,
grammar, \&c., went on for an hour; writing and arithmetic succeeded, and
music lessons were given by Miss Temple to some of the elder girls.  The
duration of each lesson was measured by the clock, which at last struck
twelve.  The superintendent rose--

"I have a word to address to the pupils," said she.

The tumult of cessation from lessons was already breaking forth, but it
sank at her voice.  She went on--

"You had this morning a breakfast which you could not eat; you must be
hungry:--I have ordered that a lunch of bread and cheese shall be served
to all."

The teachers looked at her with a sort of surprise.

"It is to be done on my responsibility," she added, in an explanatory
tone to them, and immediately afterwards left the room.

The bread and cheese was presently brought in and distributed, to the
high delight and refreshment of the whole school.  The order was now
given "To the garden!"  Each put on a coarse straw bonnet, with strings
of coloured calico, and a cloak of grey frieze.  I was similarly
equipped, and, following the stream, I made my way into the open air.

The garden was a wide inclosure, surrounded with walls so high as to
exclude every glimpse of prospect; a covered verandah ran down one side,
and broad walks bordered a middle space divided into scores of little
beds: these beds were assigned as gardens for the pupils to cultivate,
and each bed had an owner.  When full of flowers they would doubtless
look pretty; but now, at the latter end of January, all was wintry blight
and brown decay.  I shuddered as I stood and looked round me: it was an
inclement day for outdoor exercise; not positively rainy, but darkened by
a drizzling yellow fog; all under foot was still soaking wet with the
floods of yesterday.  The stronger among the girls ran about and engaged
in active games, but sundry pale and thin ones herded together for
shelter and warmth in the verandah; and amongst these, as the dense mist
penetrated to their shivering frames, I heard frequently the sound of a
hollow cough.

As yet I had spoken to no one, nor did anybody seem to take notice of me;
I stood lonely enough: but to that feeling of isolation I was accustomed;
it did not oppress me much.  I leant against a pillar of the verandah,
drew my grey mantle close about me, and, trying to forget the cold which
nipped me without, and the unsatisfied hunger which gnawed me within,
delivered myself up to the employment of watching and thinking.  My
reflections were too undefined and fragmentary to merit record: I hardly
yet knew where I was; Gateshead and my past life seemed floated away to
an immeasurable distance; the present was vague and strange, and of the
future I could form no conjecture.  I looked round the convent-like
garden, and then up at the house--a large building, half of which seemed
grey and old, the other half quite new.  The new part, containing the
schoolroom and dormitory, was lit by mullioned and latticed windows,
which gave it a church-like aspect; a stone tablet over the door bore
this inscription:--

"Lowood Institution.--This portion was rebuilt A.D. ---, by Naomi
Brocklehurst, of Brocklehurst Hall, in this county."  "Let your light so
shine before men, that they may see your good works, and glorify your
Father which is in heaven."--St. Matt. v. 16.

I read these words over and over again: I felt that an explanation
belonged to them, and was unable fully to penetrate their import.  I was
still pondering the signification of "Institution," and endeavouring to
make out a connection between the first words and the verse of Scripture,
when the sound of a cough close behind me made me turn my head.  I saw a
girl sitting on a stone bench near; she was bent over a book, on the
perusal of which she seemed intent: from where I stood I could see the
title--it was "Rasselas;" a name that struck me as strange, and
consequently attractive.  In turning a leaf she happened to look up, and
I said to her directly--

"Is your book interesting?"  I had already formed the intention of asking
her to lend it to me some day.

"I like it," she answered, after a pause of a second or two, during which
she examined me.

"What is it about?" I continued.  I hardly know where I found the
hardihood thus to open a conversation with a stranger; the step was
contrary to my nature and habits: but I think her occupation touched a
chord of sympathy somewhere; for I too liked reading, though of a
frivolous and childish kind; I could not digest or comprehend the serious
or substantial.

"You may look at it," replied the girl, offering me the book.

I did so; a brief examination convinced me that the contents were less
taking than the title: "Rasselas" looked dull to my trifling taste; I saw
nothing about fairies, nothing about genii; no bright variety seemed
spread over the closely-printed pages.  I returned it to her; she
received it quietly, and without saying anything she was about to relapse
into her former studious mood: again I ventured to disturb her--

"Can you tell me what the writing on that stone over the door means?  What
is Lowood Institution?"

"This house where you are come to live."

"And why do they call it Institution?  Is it in any way different from
other schools?"

"It is partly a charity-school: you and I, and all the rest of us, are
charity-children.  I suppose you are an orphan: are not either your
father or your mother dead?"

"Both died before I can remember."

"Well, all the girls here have lost either one or both parents, and this
is called an institution for educating orphans."

"Do we pay no money?  Do they keep us for nothing?"

"We pay, or our friends pay, fifteen pounds a year for each."

"Then why do they call us charity-children?"

"Because fifteen pounds is not enough for board and teaching, and the
deficiency is supplied by subscription."

"Who subscribes?"

"Different benevolent-minded ladies and gentlemen in this neighbourhood
and in London."

"Who was Naomi Brocklehurst?"

"The lady who built the new part of this house as that tablet records,
and whose son overlooks and directs everything here."

"Why?"

"Because he is treasurer and manager of the establishment."

"Then this house does not belong to that tall lady who wears a watch, and
who said we were to have some bread and cheese?"

"To Miss Temple?  Oh, no!  I wish it did: she has to answer to Mr.
Brocklehurst for all she does.  Mr. Brocklehurst buys all our food and
all our clothes."

"Does he live here?"

"No--two miles off, at a large hall."

"Is he a good man?"

"He is a clergyman, and is said to do a great deal of good."

"Did you say that tall lady was called Miss Temple?"

"Yes."

"And what are the other teachers called?"

"The one with red cheeks is called Miss Smith; she attends to the work,
and cuts out--for we make our own clothes, our frocks, and pelisses, and
everything; the little one with black hair is Miss Scatcherd; she teaches
history and grammar, and hears the second class repetitions; and the one
who wears a shawl, and has a pocket-handkerchief tied to her side with a
yellow ribband, is Madame Pierrot: she comes from Lisle, in France, and
teaches French."

"Do you like the teachers?"

"Well enough."

"Do you like the little black one, and the Madame ---?--I cannot
pronounce her name as you do."

"Miss Scatcherd is hasty--you must take care not to offend her; Madame
Pierrot is not a bad sort of person."

"But Miss Temple is the best--isn't she?"

"Miss Temple is very good and very clever; she is above the rest, because
she knows far more than they do."

"Have you been long here?"

"Two years."

"Are you an orphan?"

"My mother is dead."

"Are you happy here?"

"You ask rather too many questions.  I have given you answers enough for
the present: now I want to read."

But at that moment the summons sounded for dinner; all re-entered the
house.  The odour which now filled the refectory was scarcely more
appetising than that which had regaled our nostrils at breakfast: the
dinner was served in two huge tin-plated vessels, whence rose a strong
steam redolent of rancid fat.  I found the mess to consist of indifferent
potatoes and strange shreds of rusty meat, mixed and cooked together.  Of
this preparation a tolerably abundant plateful was apportioned to each
pupil.  I ate what I could, and wondered within myself whether every
day's fare would be like this.

After dinner, we immediately adjourned to the schoolroom: lessons
recommenced, and were continued till five o'clock.

The only marked event of the afternoon was, that I saw the girl with whom
I had conversed in the verandah dismissed in disgrace by Miss Scatcherd
from a history class, and sent to stand in the middle of the large
schoolroom.  The punishment seemed to me in a high degree ignominious,
especially for so great a girl--she looked thirteen or upwards.  I
expected she would show signs of great distress and shame; but to my
surprise she neither wept nor blushed: composed, though grave, she stood,
the central mark of all eyes.  "How can she bear it so quietly--so
firmly?" I asked of myself.  "Were I in her place, it seems to me I
should wish the earth to open and swallow me up.  She looks as if she
were thinking of something beyond her punishment--beyond her situation:
of something not round her nor before her.  I have heard of day-dreams--is
she in a day-dream now?  Her eyes are fixed on the floor, but I am sure
they do not see it--her sight seems turned in, gone down into her heart:
she is looking at what she can remember, I believe; not at what is really
present.  I wonder what sort of a girl she is--whether good or naughty."

Soon after five p.m. we had another meal, consisting of a small mug of
coffee, and half-a-slice of brown bread.  I devoured my bread and drank
my coffee with relish; but I should have been glad of as much more--I was
still hungry.  Half-an-hour's recreation succeeded, then study; then the
glass of water and the piece of oat-cake, prayers, and bed.  Such was my
first day at Lowood.




\chapter{}


\lettrine{T}{he} next day commenced as before, getting up and dressing by rushlight;
but this morning we were obliged to dispense with the ceremony of
washing; the water in the pitchers was frozen.  A change had taken place
in the weather the preceding evening, and a keen north-east wind,
whistling through the crevices of our bedroom windows all night long, had
made us shiver in our beds, and turned the contents of the ewers to ice.

Before the long hour and a half of prayers and Bible-reading was over, I
felt ready to perish with cold.  Breakfast-time came at last, and this
morning the porridge was not burnt; the quality was eatable, the quantity
small.  How small my portion seemed!  I wished it had been doubled.

In the course of the day I was enrolled a member of the fourth class, and
regular tasks and occupations were assigned me: hitherto, I had only been
a spectator of the proceedings at Lowood; I was now to become an actor
therein.  At first, being little accustomed to learn by heart, the
lessons appeared to me both long and difficult; the frequent change from
task to task, too, bewildered me; and I was glad when, about three
o'clock in the afternoon, Miss Smith put into my hands a border of muslin
two yards long, together with needle, thimble, \&c., and sent me to sit in
a quiet corner of the schoolroom, with directions to hem the same.  At
that hour most of the others were sewing likewise; but one class still
stood round Miss Scatcherd's chair reading, and as all was quiet, the
subject of their lessons could be heard, together with the manner in
which each girl acquitted herself, and the animadversions or
commendations of Miss Scatcherd on the performance.  It was English
history: among the readers I observed my acquaintance of the verandah: at
the commencement of the lesson, her place had been at the top of the
class, but for some error of pronunciation, or some inattention to stops,
she was suddenly sent to the very bottom.  Even in that obscure position,
Miss Scatcherd continued to make her an object of constant notice: she
was continually addressing to her such phrases as the following:--

"Burns" (such it seems was her name: the girls here were all called by
their surnames, as boys are elsewhere), "Burns, you are standing on the
side of your shoe; turn your toes out immediately."  "Burns, you poke
your chin most unpleasantly; draw it in."  "Burns, I insist on your
holding your head up; I will not have you before me in that attitude,"
\&c. \&c.

A chapter having been read through twice, the books were closed and the
girls examined.  The lesson had comprised part of the reign of Charles
I., and there were sundry questions about tonnage and poundage and ship-
money, which most of them appeared unable to answer; still, every little
difficulty was solved instantly when it reached Burns: her memory seemed
to have retained the substance of the whole lesson, and she was ready
with answers on every point.  I kept expecting that Miss Scatcherd would
praise her attention; but, instead of that, she suddenly cried out--

"You dirty, disagreeable girl! you have never cleaned your nails this
morning!"

Burns made no answer: I wondered at her silence.  "Why," thought I, "does
she not explain that she could neither clean her nails nor wash her face,
as the water was frozen?"

My attention was now called off by Miss Smith desiring me to hold a skein
of thread: while she was winding it, she talked to me from time to time,
asking whether I had ever been at school before, whether I could mark,
stitch, knit, \&c.; till she dismissed me, I could not pursue my
observations on Miss Scatcherd's movements.  When I returned to my seat,
that lady was just delivering an order of which I did not catch the
import; but Burns immediately left the class, and going into the small
inner room where the books were kept, returned in half a minute, carrying
in her hand a bundle of twigs tied together at one end.  This ominous
tool she presented to Miss Scatcherd with a respectful curtesy; then she
quietly, and without being told, unloosed her pinafore, and the teacher
instantly and sharply inflicted on her neck a dozen strokes with the
bunch of twigs.  Not a tear rose to Burns' eye; and, while I paused from
my sewing, because my fingers quivered at this spectacle with a sentiment
of unavailing and impotent anger, not a feature of her pensive face
altered its ordinary expression.

"Hardened girl!" exclaimed Miss Scatcherd; "nothing can correct you of
your slatternly habits: carry the rod away."

Burns obeyed: I looked at her narrowly as she emerged from the
book-closet; she was just putting back her handkerchief into her pocket,
and the trace of a tear glistened on her thin cheek.

The play-hour in the evening I thought the pleasantest fraction of the
day at Lowood: the bit of bread, the draught of coffee swallowed at five
o'clock had revived vitality, if it had not satisfied hunger: the long
restraint of the day was slackened; the schoolroom felt warmer than in
the morning--its fires being allowed to burn a little more brightly, to
supply, in some measure, the place of candles, not yet introduced: the
ruddy gloaming, the licensed uproar, the confusion of many voices gave
one a welcome sense of liberty.

On the evening of the day on which I had seen Miss Scatcherd flog her
pupil, Burns, I wandered as usual among the forms and tables and laughing
groups without a companion, yet not feeling lonely: when I passed the
windows, I now and then lifted a blind, and looked out; it snowed fast, a
drift was already forming against the lower panes; putting my ear close
to the window, I could distinguish from the gleeful tumult within, the
disconsolate moan of the wind outside.

Probably, if I had lately left a good home and kind parents, this would
have been the hour when I should most keenly have regretted the
separation; that wind would then have saddened my heart; this obscure
chaos would have disturbed my peace! as it was, I derived from both a
strange excitement, and reckless and feverish, I wished the wind to howl
more wildly, the gloom to deepen to darkness, and the confusion to rise
to clamour.

Jumping over forms, and creeping under tables, I made my way to one of
the fire-places; there, kneeling by the high wire fender, I found Burns,
absorbed, silent, abstracted from all round her by the companionship of a
book, which she read by the dim glare of the embers.

"Is it still 'Rasselas'?" I asked, coming behind her.

"Yes," she said, "and I have just finished it."

And in five minutes more she shut it up.  I was glad of this.  "Now,"
thought I, "I can perhaps get her to talk."  I sat down by her on the
floor.

"What is your name besides Burns?"

"Helen."

"Do you come a long way from here?"

"I come from a place farther north, quite on the borders of Scotland."

"Will you ever go back?"

"I hope so; but nobody can be sure of the future."

"You must wish to leave Lowood?"

"No! why should I?  I was sent to Lowood to get an education; and it
would be of no use going away until I have attained that object."

"But that teacher, Miss Scatcherd, is so cruel to you?"

"Cruel?  Not at all!  She is severe: she dislikes my faults."

"And if I were in your place I should dislike her; I should resist her.
If she struck me with that rod, I should get it from her hand; I should
break it under her nose."

"Probably you would do nothing of the sort: but if you did, Mr.
Brocklehurst would expel you from the school; that would be a great grief
to your relations.  It is far better to endure patiently a smart which
nobody feels but yourself, than to commit a hasty action whose evil
consequences will extend to all connected with you; and besides, the
Bible bids us return good for evil."

"But then it seems disgraceful to be flogged, and to be sent to stand in
the middle of a room full of people; and you are such a great girl: I am
far younger than you, and I could not bear it."

"Yet it would be your duty to bear it, if you could not avoid it: it is
weak and silly to say you \emph{cannot bear} what it is your fate to be
required to bear."

I heard her with wonder: I could not comprehend this doctrine of
endurance; and still less could I understand or sympathise with the
forbearance she expressed for her chastiser.  Still I felt that Helen
Burns considered things by a light invisible to my eyes.  I suspected she
might be right and I wrong; but I would not ponder the matter deeply;
like Felix, I put it off to a more convenient season.

"You say you have faults, Helen: what are they?  To me you seem very
good."

"Then learn from me, not to judge by appearances: I am, as Miss Scatcherd
said, slatternly; I seldom put, and never keep, things, in order; I am
careless; I forget rules; I read when I should learn my lessons; I have
no method; and sometimes I say, like you, I cannot \emph{bear} to be subjected
to systematic arrangements.  This is all very provoking to Miss
Scatcherd, who is naturally neat, punctual, and particular."

"And cross and cruel," I added; but Helen Burns would not admit my
addition: she kept silence.

"Is Miss Temple as severe to you as Miss Scatcherd?"

At the utterance of Miss Temple's name, a soft smile flitted over her
grave face.

"Miss Temple is full of goodness; it pains her to be severe to any one,
even the worst in the school: she sees my errors, and tells me of them
gently; and, if I do anything worthy of praise, she gives me my meed
liberally.  One strong proof of my wretchedly defective nature is, that
even her expostulations, so mild, so rational, have not influence to cure
me of my faults; and even her praise, though I value it most highly,
cannot stimulate me to continued care and foresight."

"That is curious," said I, "it is so easy to be careful."

"For \emph{you} I have no doubt it is.  I observed you in your class this
morning, and saw you were closely attentive: your thoughts never seemed
to wander while Miss Miller explained the lesson and questioned you.  Now,
mine continually rove away; when I should be listening to Miss Scatcherd,
and collecting all she says with assiduity, often I lose the very sound
of her voice; I fall into a sort of dream.  Sometimes I think I am in
Northumberland, and that the noises I hear round me are the bubbling of a
little brook which runs through Deepden, near our house;--then, when it
comes to my turn to reply, I have to be awakened; and having heard
nothing of what was read for listening to the visionary brook, I have no
answer ready."

"Yet how well you replied this afternoon."

"It was mere chance; the subject on which we had been reading had
interested me.  This afternoon, instead of dreaming of Deepden, I was
wondering how a man who wished to do right could act so unjustly and
unwisely as Charles the First sometimes did; and I thought what a pity it
was that, with his integrity and conscientiousness, he could see no
farther than the prerogatives of the crown.  If he had but been able to
look to a distance, and see how what they call the spirit of the age was
tending!  Still, I like Charles--I respect him--I pity him, poor murdered
king!  Yes, his enemies were the worst: they shed blood they had no right
to shed.  How dared they kill him!"

Helen was talking to herself now: she had forgotten I could not very well
understand her--that I was ignorant, or nearly so, of the subject she
discussed.  I recalled her to my level.

"And when Miss Temple teaches you, do your thoughts wander then?"

"No, certainly, not often; because Miss Temple has generally something to
say which is newer than my own reflections; her language is singularly
agreeable to me, and the information she communicates is often just what
I wished to gain."

"Well, then, with Miss Temple you are good?"

"Yes, in a passive way: I make no effort; I follow as inclination guides
me.  There is no merit in such goodness."

"A great deal: you are good to those who are good to you.  It is all I
ever desire to be.  If people were always kind and obedient to those who
are cruel and unjust, the wicked people would have it all their own way:
they would never feel afraid, and so they would never alter, but would
grow worse and worse.  When we are struck at without a reason, we should
strike back again very hard; I am sure we should--so hard as to teach the
person who struck us never to do it again."

"You will change your mind, I hope, when you grow older: as yet you are
but a little untaught girl."

"But I feel this, Helen; I must dislike those who, whatever I do to
please them, persist in disliking me; I must resist those who punish me
unjustly.  It is as natural as that I should love those who show me
affection, or submit to punishment when I feel it is deserved."

"Heathens and savage tribes hold that doctrine, but Christians and
civilised nations disown it."

"How?  I don't understand."

"It is not violence that best overcomes hate--nor vengeance that most
certainly heals injury."

"What then?"

"Read the New Testament, and observe what Christ says, and how He acts;
make His word your rule, and His conduct your example."

"What does He say?"

"Love your enemies; bless them that curse you; do good to them that hate
you and despitefully use you."

"Then I should love Mrs. Reed, which I cannot do; I should bless her son
John, which is impossible."

In her turn, Helen Burns asked me to explain, and I proceeded forthwith
to pour out, in my own way, the tale of my sufferings and resentments.
Bitter and truculent when excited, I spoke as I felt, without reserve or
softening.

Helen heard me patiently to the end: I expected she would then make a
remark, but she said nothing.

"Well," I asked impatiently, "is not Mrs. Reed a hard-hearted, bad
woman?"

"She has been unkind to you, no doubt; because you see, she dislikes your
cast of character, as Miss Scatcherd does mine; but how minutely you
remember all she has done and said to you!  What a singularly deep
impression her injustice seems to have made on your heart!  No ill-usage
so brands its record on my feelings.  Would you not be happier if you
tried to forget her severity, together with the passionate emotions it
excited?  Life appears to me too short to be spent in nursing animosity
or registering wrongs.  We are, and must be, one and all, burdened with
faults in this world: but the time will soon come when, I trust, we shall
put them off in putting off our corruptible bodies; when debasement and
sin will fall from us with this cumbrous frame of flesh, and only the
spark of the spirit will remain,--the impalpable principle of light and
thought, pure as when it left the Creator to inspire the creature: whence
it came it will return; perhaps again to be communicated to some being
higher than man--perhaps to pass through gradations of glory, from the
pale human soul to brighten to the seraph!  Surely it will never, on the
contrary, be suffered to degenerate from man to fiend?  No; I cannot
believe that: I hold another creed: which no one ever taught me, and
which I seldom mention; but in which I delight, and to which I cling: for
it extends hope to all: it makes Eternity a rest--a mighty home, not a
terror and an abyss.  Besides, with this creed, I can so clearly
distinguish between the criminal and his crime; I can so sincerely
forgive the first while I abhor the last: with this creed revenge never
worries my heart, degradation never too deeply disgusts me, injustice
never crushes me too low: I live in calm, looking to the end."

Helen's head, always drooping, sank a little lower as she finished this
sentence.  I saw by her look she wished no longer to talk to me, but
rather to converse with her own thoughts.  She was not allowed much time
for meditation: a monitor, a great rough girl, presently came up,
exclaiming in a strong Cumberland accent--

"Helen Burns, if you don't go and put your drawer in order, and fold up
your work this minute, I'll tell Miss Scatcherd to come and look at it!"

Helen sighed as her reverie fled, and getting up, obeyed the monitor
without reply as without delay.




\chapter{}


\lettrine{M}{y} first quarter at Lowood seemed an age; and not the golden age either;
it comprised an irksome struggle with difficulties in habituating myself
to new rules and unwonted tasks.  The fear of failure in these points
harassed me worse than the physical hardships of my lot; though these
were no trifles.

During January, February, and part of March, the deep snows, and, after
their melting, the almost impassable roads, prevented our stirring beyond
the garden walls, except to go to church; but within these limits we had
to pass an hour every day in the open air.  Our clothing was insufficient
to protect us from the severe cold: we had no boots, the snow got into
our shoes and melted there: our ungloved hands became numbed and covered
with chilblains, as were our feet: I remember well the distracting
irritation I endured from this cause every evening, when my feet
inflamed; and the torture of thrusting the swelled, raw, and stiff toes
into my shoes in the morning.  Then the scanty supply of food was
distressing: with the keen appetites of growing children, we had scarcely
sufficient to keep alive a delicate invalid.  From this deficiency of
nourishment resulted an abuse, which pressed hardly on the younger
pupils: whenever the famished great girls had an opportunity, they would
coax or menace the little ones out of their portion.  Many a time I have
shared between two claimants the precious morsel of brown bread
distributed at tea-time; and after relinquishing to a third half the
contents of my mug of coffee, I have swallowed the remainder with an
accompaniment of secret tears, forced from me by the exigency of hunger.

Sundays were dreary days in that wintry season.  We had to walk two miles
to Brocklebridge Church, where our patron officiated.  We set out cold,
we arrived at church colder: during the morning service we became almost
paralysed.  It was too far to return to dinner, and an allowance of cold
meat and bread, in the same penurious proportion observed in our ordinary
meals, was served round between the services.

At the close of the afternoon service we returned by an exposed and hilly
road, where the bitter winter wind, blowing over a range of snowy summits
to the north, almost flayed the skin from our faces.

I can remember Miss Temple walking lightly and rapidly along our drooping
line, her plaid cloak, which the frosty wind fluttered, gathered close
about her, and encouraging us, by precept and example, to keep up our
spirits, and march forward, as she said, "like stalwart soldiers."  The
other teachers, poor things, were generally themselves too much dejected
to attempt the task of cheering others.

How we longed for the light and heat of a blazing fire when we got back!
But, to the little ones at least, this was denied: each hearth in the
schoolroom was immediately surrounded by a double row of great girls, and
behind them the younger children crouched in groups, wrapping their
starved arms in their pinafores.

A little solace came at tea-time, in the shape of a double ration of
bread--a whole, instead of a half, slice--with the delicious addition of
a thin scrape of butter: it was the hebdomadal treat to which we all
looked forward from Sabbath to Sabbath.  I generally contrived to reserve
a moiety of this bounteous repast for myself; but the remainder I was
invariably obliged to part with.

The Sunday evening was spent in repeating, by heart, the Church
Catechism, and the fifth, sixth, and seventh chapters of St. Matthew; and
in listening to a long sermon, read by Miss Miller, whose irrepressible
yawns attested her weariness.  A frequent interlude of these performances
was the enactment of the part of Eutychus by some half-dozen of little
girls, who, overpowered with sleep, would fall down, if not out of the
third loft, yet off the fourth form, and be taken up half dead.  The
remedy was, to thrust them forward into the centre of the schoolroom, and
oblige them to stand there till the sermon was finished.  Sometimes their
feet failed them, and they sank together in a heap; they were then
propped up with the monitors' high stools.

I have not yet alluded to the visits of Mr. Brocklehurst; and indeed that
gentleman was from home during the greater part of the first month after
my arrival; perhaps prolonging his stay with his friend the archdeacon:
his absence was a relief to me.  I need not say that I had my own reasons
for dreading his coming: but come he did at last.

One afternoon (I had then been three weeks at Lowood), as I was sitting
with a slate in my hand, puzzling over a sum in long division, my eyes,
raised in abstraction to the window, caught sight of a figure just
passing: I recognised almost instinctively that gaunt outline; and when,
two minutes after, all the school, teachers included, rose \emph{en masse}, it
was not necessary for me to look up in order to ascertain whose entrance
they thus greeted.  A long stride measured the schoolroom, and presently
beside Miss Temple, who herself had risen, stood the same black column
which had frowned on me so ominously from the hearthrug of Gateshead.  I
now glanced sideways at this piece of architecture.  Yes, I was right: it
was Mr. Brocklehurst, buttoned up in a surtout, and looking longer,
narrower, and more rigid than ever.

I had my own reasons for being dismayed at this apparition; too well I
remembered the perfidious hints given by Mrs. Reed about my disposition,
\&c.; the promise pledged by Mr. Brocklehurst to apprise Miss Temple and
the teachers of my vicious nature.  All along I had been dreading the
fulfilment of this promise,--I had been looking out daily for the "Coming
Man," whose information respecting my past life and conversation was to
brand me as a bad child for ever: now there he was.

He stood at Miss Temple's side; he was speaking low in her ear: I did not
doubt he was making disclosures of my villainy; and I watched her eye
with painful anxiety, expecting every moment to see its dark orb turn on
me a glance of repugnance and contempt.  I listened too; and as I
happened to be seated quite at the top of the room, I caught most of what
he said: its import relieved me from immediate apprehension.

"I suppose, Miss Temple, the thread I bought at Lowton will do; it struck
me that it would be just of the quality for the calico chemises, and I
sorted the needles to match.  You may tell Miss Smith that I forgot to
make a memorandum of the darning needles, but she shall have some papers
sent in next week; and she is not, on any account, to give out more than
one at a time to each pupil: if they have more, they are apt to be
careless and lose them.  And, O ma'am!  I wish the woollen stockings were
better looked to!--when I was here last, I went into the kitchen-garden
and examined the clothes drying on the line; there was a quantity of
black hose in a very bad state of repair: from the size of the holes in
them I was sure they had not been well mended from time to time."

He paused.

"Your directions shall be attended to, sir," said Miss Temple.

"And, ma'am," he continued, "the laundress tells me some of the girls
have two clean tuckers in the week: it is too much; the rules limit them
to one."

"I think I can explain that circumstance, sir.  Agnes and Catherine
Johnstone were invited to take tea with some friends at Lowton last
Thursday, and I gave them leave to put on clean tuckers for the
occasion."

Mr. Brocklehurst nodded.

"Well, for once it may pass; but please not to let the circumstance occur
too often.  And there is another thing which surprised me; I find, in
settling accounts with the housekeeper, that a lunch, consisting of bread
and cheese, has twice been served out to the girls during the past
fortnight.  How is this?  I looked over the regulations, and I find no
such meal as lunch mentioned.  Who introduced this innovation? and by
what authority?"

"I must be responsible for the circumstance, sir," replied Miss Temple:
"the breakfast was so ill prepared that the pupils could not possibly eat
it; and I dared not allow them to remain fasting till dinner-time."

"Madam, allow me an instant.  You are aware that my plan in bringing up
these girls is, not to accustom them to habits of luxury and indulgence,
but to render them hardy, patient, self-denying.  Should any little
accidental disappointment of the appetite occur, such as the spoiling of
a meal, the under or the over dressing of a dish, the incident ought not
to be neutralised by replacing with something more delicate the comfort
lost, thus pampering the body and obviating the aim of this institution;
it ought to be improved to the spiritual edification of the pupils, by
encouraging them to evince fortitude under temporary privation.  A brief
address on those occasions would not be mistimed, wherein a judicious
instructor would take the opportunity of referring to the sufferings of
the primitive Christians; to the torments of martyrs; to the exhortations
of our blessed Lord Himself, calling upon His disciples to take up their
cross and follow Him; to His warnings that man shall not live by bread
alone, but by every word that proceedeth out of the mouth of God; to His
divine consolations, "If ye suffer hunger or thirst for My sake, happy
are ye."  Oh, madam, when you put bread and cheese, instead of burnt
porridge, into these children's mouths, you may indeed feed their vile
bodies, but you little think how you starve their immortal souls!"

Mr. Brocklehurst again paused--perhaps overcome by his feelings.  Miss
Temple had looked down when he first began to speak to her; but she now
gazed straight before her, and her face, naturally pale as marble,
appeared to be assuming also the coldness and fixity of that material;
especially her mouth, closed as if it would have required a sculptor's
chisel to open it, and her brow settled gradually into petrified
severity.

Meantime, Mr. Brocklehurst, standing on the hearth with his hands behind
his back, majestically surveyed the whole school.  Suddenly his eye gave
a blink, as if it had met something that either dazzled or shocked its
pupil; turning, he said in more rapid accents than he had hitherto used--

"Miss Temple, Miss Temple, what--\emph{what} is that girl with curled hair?
Red hair, ma'am, curled--curled all over?"  And extending his cane he
pointed to the awful object, his hand shaking as he did so.

"It is Julia Severn," replied Miss Temple, very quietly.

"Julia Severn, ma'am!  And why has she, or any other, curled hair?  Why,
in defiance of every precept and principle of this house, does she
conform to the world so openly--here in an evangelical, charitable
establishment--as to wear her hair one mass of curls?"

"Julia's hair curls naturally," returned Miss Temple, still more quietly.

"Naturally!  Yes, but we are not to conform to nature; I wish these girls
to be the children of Grace: and why that abundance?  I have again and
again intimated that I desire the hair to be arranged closely, modestly,
plainly.  Miss Temple, that girl's hair must be cut off entirely; I will
send a barber to-morrow: and I see others who have far too much of the
excrescence--that tall girl, tell her to turn round.  Tell all the first
form to rise up and direct their faces to the wall."

Miss Temple passed her handkerchief over her lips, as if to smooth away
the involuntary smile that curled them; she gave the order, however, and
when the first class could take in what was required of them, they
obeyed.  Leaning a little back on my bench, I could see the looks and
grimaces with which they commented on this manoeuvre: it was a pity Mr.
Brocklehurst could not see them too; he would perhaps have felt that,
whatever he might do with the outside of the cup and platter, the inside
was further beyond his interference than he imagined.

He scrutinised the reverse of these living medals some five minutes, then
pronounced sentence.  These words fell like the knell of doom--

"All those top-knots must be cut off."

Miss Temple seemed to remonstrate.

"Madam," he pursued, "I have a Master to serve whose kingdom is not of
this world: my mission is to mortify in these girls the lusts of the
flesh; to teach them to clothe themselves with shame-facedness and
sobriety, not with braided hair and costly apparel; and each of the young
persons before us has a string of hair twisted in plaits which vanity
itself might have woven; these, I repeat, must be cut off; think of the
time wasted, of--"

Mr. Brocklehurst was here interrupted: three other visitors, ladies, now
entered the room.  They ought to have come a little sooner to have heard
his lecture on dress, for they were splendidly attired in velvet, silk,
and furs.  The two younger of the trio (fine girls of sixteen and
seventeen) had grey beaver hats, then in fashion, shaded with ostrich
plumes, and from under the brim of this graceful head-dress fell a
profusion of light tresses, elaborately curled; the elder lady was
enveloped in a costly velvet shawl, trimmed with ermine, and she wore a
false front of French curls.

These ladies were deferentially received by Miss Temple, as Mrs. and the
Misses Brocklehurst, and conducted to seats of honour at the top of the
room.  It seems they had come in the carriage with their reverend
relative, and had been conducting a rummaging scrutiny of the room
upstairs, while he transacted business with the housekeeper, questioned
the laundress, and lectured the superintendent.  They now proceeded to
address divers remarks and reproofs to Miss Smith, who was charged with
the care of the linen and the inspection of the dormitories: but I had no
time to listen to what they said; other matters called off and enchanted
my attention.

Hitherto, while gathering up the discourse of Mr. Brocklehurst and Miss
Temple, I had not, at the same time, neglected precautions to secure my
personal safety; which I thought would be effected, if I could only elude
observation.  To this end, I had sat well back on the form, and while
seeming to be busy with my sum, had held my slate in such a manner as to
conceal my face: I might have escaped notice, had not my treacherous
slate somehow happened to slip from my hand, and falling with an
obtrusive crash, directly drawn every eye upon me; I knew it was all over
now, and, as I stooped to pick up the two fragments of slate, I rallied
my forces for the worst.  It came.

"A careless girl!" said Mr. Brocklehurst, and immediately after--"It is
the new pupil, I perceive."  And before I could draw breath, "I must not
forget I have a word to say respecting her."  Then aloud: how loud it
seemed to me!  "Let the child who broke her slate come forward!"

Of my own accord I could not have stirred; I was paralysed: but the two
great girls who sit on each side of me, set me on my legs and pushed me
towards the dread judge, and then Miss Temple gently assisted me to his
very feet, and I caught her whispered counsel--

"Don't be afraid, Jane, I saw it was an accident; you shall not be
punished."

The kind whisper went to my heart like a dagger.

"Another minute, and she will despise me for a hypocrite," thought I; and
an impulse of fury against Reed, Brocklehurst, and Co. bounded in my
pulses at the conviction.  I was no Helen Burns.

"Fetch that stool," said Mr. Brocklehurst, pointing to a very high one
from which a monitor had just risen: it was brought.

"Place the child upon it."

And I was placed there, by whom I don't know: I was in no condition to
note particulars; I was only aware that they had hoisted me up to the
height of Mr. Brocklehurst's nose, that he was within a yard of me, and
that a spread of shot orange and purple silk pelisses and a cloud of
silvery plumage extended and waved below me.

Mr. Brocklehurst hemmed.

"Ladies," said he, turning to his family, "Miss Temple, teachers, and
children, you all see this girl?"

Of course they did; for I felt their eyes directed like burning-glasses
against my scorched skin.

"You see she is yet young; you observe she possesses the ordinary form of
childhood; God has graciously given her the shape that He has given to
all of us; no signal deformity points her out as a marked character.  Who
would think that the Evil One had already found a servant and agent in
her?  Yet such, I grieve to say, is the case."

A pause--in which I began to steady the palsy of my nerves, and to feel
that the Rubicon was passed; and that the trial, no longer to be shirked,
must be firmly sustained.

"My dear children," pursued the black marble clergyman, with pathos,
"this is a sad, a melancholy occasion; for it becomes my duty to warn
you, that this girl, who might be one of God's own lambs, is a little
castaway: not a member of the true flock, but evidently an interloper and
an alien.  You must be on your guard against her; you must shun her
example; if necessary, avoid her company, exclude her from your sports,
and shut her out from your converse.  Teachers, you must watch her: keep
your eyes on her movements, weigh well her words, scrutinise her actions,
punish her body to save her soul: if, indeed, such salvation be possible,
for (my tongue falters while I tell it) this girl, this child, the native
of a Christian land, worse than many a little heathen who says its
prayers to Brahma and kneels before Juggernaut--this girl is--a liar!"

Now came a pause of ten minutes, during which I, by this time in perfect
possession of my wits, observed all the female Brocklehursts produce
their pocket-handkerchiefs and apply them to their optics, while the
elderly lady swayed herself to and fro, and the two younger ones
whispered, "How shocking!"  Mr. Brocklehurst resumed.

"This I learned from her benefactress; from the pious and charitable lady
who adopted her in her orphan state, reared her as her own daughter, and
whose kindness, whose generosity the unhappy girl repaid by an
ingratitude so bad, so dreadful, that at last her excellent patroness was
obliged to separate her from her own young ones, fearful lest her vicious
example should contaminate their purity: she has sent her here to be
healed, even as the Jews of old sent their diseased to the troubled pool
of Bethesda; and, teachers, superintendent, I beg of you not to allow the
waters to stagnate round her."

With this sublime conclusion, Mr. Brocklehurst adjusted the top button of
his surtout, muttered something to his family, who rose, bowed to Miss
Temple, and then all the great people sailed in state from the room.
Turning at the door, my judge said--

"Let her stand half-an-hour longer on that stool, and let no one speak to
her during the remainder of the day."

There was I, then, mounted aloft; I, who had said I could not bear the
shame of standing on my natural feet in the middle of the room, was now
exposed to general view on a pedestal of infamy.  What my sensations were
no language can describe; but just as they all rose, stifling my breath
and constricting my throat, a girl came up and passed me: in passing, she
lifted her eyes.  What a strange light inspired them!  What an
extraordinary sensation that ray sent through me!  How the new feeling
bore me up!  It was as if a martyr, a hero, had passed a slave or victim,
and imparted strength in the transit.  I mastered the rising hysteria,
lifted up my head, and took a firm stand on the stool.  Helen Burns asked
some slight question about her work of Miss Smith, was chidden for the
triviality of the inquiry, returned to her place, and smiled at me as she
again went by.  What a smile!  I remember it now, and I know that it was
the effluence of fine intellect, of true courage; it lit up her marked
lineaments, her thin face, her sunken grey eye, like a reflection from
the aspect of an angel.  Yet at that moment Helen Burns wore on her arm
"the untidy badge;" scarcely an hour ago I had heard her condemned by
Miss Scatcherd to a dinner of bread and water on the morrow because she
had blotted an exercise in copying it out.  Such is the imperfect nature
of man! such spots are there on the disc of the clearest planet; and eyes
like Miss Scatcherd's can only see those minute defects, and are blind to
the full brightness of the orb.




\chapter{}


\lettrine{E}{re} the half-hour ended, five o'clock struck; school was dismissed, and
all were gone into the refectory to tea.  I now ventured to descend: it
was deep dusk; I retired into a corner and sat down on the floor.  The
spell by which I had been so far supported began to dissolve; reaction
took place, and soon, so overwhelming was the grief that seized me, I
sank prostrate with my face to the ground.  Now I wept: Helen Burns was
not here; nothing sustained me; left to myself I abandoned myself, and my
tears watered the boards.  I had meant to be so good, and to do so much
at Lowood: to make so many friends, to earn respect and win affection.
Already I had made visible progress: that very morning I had reached the
head of my class; Miss Miller had praised me warmly; Miss Temple had
smiled approbation; she had promised to teach me drawing, and to let me
learn French, if I continued to make similar improvement two months
longer: and then I was well received by my fellow-pupils; treated as an
equal by those of my own age, and not molested by any; now, here I lay
again crushed and trodden on; and could I ever rise more?

"Never," I thought; and ardently I wished to die.  While sobbing out this
wish in broken accents, some one approached: I started up--again Helen
Burns was near me; the fading fires just showed her coming up the long,
vacant room; she brought my coffee and bread.

"Come, eat something," she said; but I put both away from me, feeling as
if a drop or a crumb would have choked me in my present condition.  Helen
regarded me, probably with surprise: I could not now abate my agitation,
though I tried hard; I continued to weep aloud.  She sat down on the
ground near me, embraced her knees with her arms, and rested her head
upon them; in that attitude she remained silent as an Indian.  I was the
first who spoke--

"Helen, why do you stay with a girl whom everybody believes to be a
liar?"

"Everybody, Jane?  Why, there are only eighty people who have heard you
called so, and the world contains hundreds of millions."

"But what have I to do with millions?  The eighty, I know, despise me."

"Jane, you are mistaken: probably not one in the school either despises
or dislikes you: many, I am sure, pity you much."

"How can they pity me after what Mr. Brocklehurst has said?"

"Mr. Brocklehurst is not a god: nor is he even a great and admired man:
he is little liked here; he never took steps to make himself liked.  Had
he treated you as an especial favourite, you would have found enemies,
declared or covert, all around you; as it is, the greater number would
offer you sympathy if they dared.  Teachers and pupils may look coldly on
you for a day or two, but friendly feelings are concealed in their
hearts; and if you persevere in doing well, these feelings will ere long
appear so much the more evidently for their temporary suppression.
Besides, Jane"--she paused.

"Well, Helen?" said I, putting my hand into hers: she chafed my fingers
gently to warm them, and went on--

"If all the world hated you, and believed you wicked, while your own
conscience approved you, and absolved you from guilt, you would not be
without friends."

"No; I know I should think well of myself; but that is not enough: if
others don't love me I would rather die than live--I cannot bear to be
solitary and hated, Helen.  Look here; to gain some real affection from
you, or Miss Temple, or any other whom I truly love, I would willingly
submit to have the bone of my arm broken, or to let a bull toss me, or to
stand behind a kicking horse, and let it dash its hoof at my chest--"

"Hush, Jane! you think too much of the love of human beings; you are too
impulsive, too vehement; the sovereign hand that created your frame, and
put life into it, has provided you with other resources than your feeble
self, or than creatures feeble as you.  Besides this earth, and besides
the race of men, there is an invisible world and a kingdom of spirits:
that world is round us, for it is everywhere; and those spirits watch us,
for they are commissioned to guard us; and if we were dying in pain and
shame, if scorn smote us on all sides, and hatred crushed us, angels see
our tortures, recognise our innocence (if innocent we be: as I know you
are of this charge which Mr. Brocklehurst has weakly and pompously
repeated at second-hand from Mrs. Reed; for I read a sincere nature in
your ardent eyes and on your clear front), and God waits only the
separation of spirit from flesh to crown us with a full reward.  Why,
then, should we ever sink overwhelmed with distress, when life is so soon
over, and death is so certain an entrance to happiness--to glory?"

I was silent; Helen had calmed me; but in the tranquillity she imparted
there was an alloy of inexpressible sadness.  I felt the impression of
woe as she spoke, but I could not tell whence it came; and when, having
done speaking, she breathed a little fast and coughed a short cough, I
momentarily forgot my own sorrows to yield to a vague concern for her.

Resting my head on Helen's shoulder, I put my arms round her waist; she
drew me to her, and we reposed in silence.  We had not sat long thus,
when another person came in.  Some heavy clouds, swept from the sky by a
rising wind, had left the moon bare; and her light, streaming in through
a window near, shone full both on us and on the approaching figure, which
we at once recognised as Miss Temple.

"I came on purpose to find you, Jane Eyre," said she; "I want you in my
room; and as Helen Burns is with you, she may come too."

We went; following the superintendent's guidance, we had to thread some
intricate passages, and mount a staircase before we reached her
apartment; it contained a good fire, and looked cheerful.  Miss Temple
told Helen Burns to be seated in a low arm-chair on one side of the
hearth, and herself taking another, she called me to her side.

"Is it all over?" she asked, looking down at my face.  "Have you cried
your grief away?"

"I am afraid I never shall do that."

"Why?"

"Because I have been wrongly accused; and you, ma'am, and everybody else,
will now think me wicked."

"We shall think you what you prove yourself to be, my child.  Continue to
act as a good girl, and you will satisfy us."

"Shall I, Miss Temple?"

"You will," said she, passing her arm round me.  "And now tell me who is
the lady whom Mr. Brocklehurst called your benefactress?"

"Mrs. Reed, my uncle's wife.  My uncle is dead, and he left me to her
care."

"Did she not, then, adopt you of her own accord?"

"No, ma'am; she was sorry to have to do it: but my uncle, as I have often
heard the servants say, got her to promise before he died that she would
always keep me."

"Well now, Jane, you know, or at least I will tell you, that when a
criminal is accused, he is always allowed to speak in his own defence.
You have been charged with falsehood; defend yourself to me as well as
you can.  Say whatever your memory suggests is true; but add nothing and
exaggerate nothing."

I resolved, in the depth of my heart, that I would be most moderate--most
correct; and, having reflected a few minutes in order to arrange
coherently what I had to say, I told her all the story of my sad
childhood.  Exhausted by emotion, my language was more subdued than it
generally was when it developed that sad theme; and mindful of Helen's
warnings against the indulgence of resentment, I infused into the
narrative far less of gall and wormwood than ordinary.  Thus restrained
and simplified, it sounded more credible: I felt as I went on that Miss
Temple fully believed me.

In the course of the tale I had mentioned Mr. Lloyd as having come to see
me after the fit: for I never forgot the, to me, frightful episode of the
red-room: in detailing which, my excitement was sure, in some degree, to
break bounds; for nothing could soften in my recollection the spasm of
agony which clutched my heart when Mrs. Reed spurned my wild supplication
for pardon, and locked me a second time in the dark and haunted chamber.

I had finished: Miss Temple regarded me a few minutes in silence; she
then said--

"I know something of Mr. Lloyd; I shall write to him; if his reply agrees
with your statement, you shall be publicly cleared from every imputation;
to me, Jane, you are clear now."

She kissed me, and still keeping me at her side (where I was well
contented to stand, for I derived a child's pleasure from the
contemplation of her face, her dress, her one or two ornaments, her white
forehead, her clustered and shining curls, and beaming dark eyes), she
proceeded to address Helen Burns.

"How are you to-night, Helen?  Have you coughed much to-day?"

"Not quite so much, I think, ma'am."

"And the pain in your chest?"

"It is a little better."

Miss Temple got up, took her hand and examined her pulse; then she
returned to her own seat: as she resumed it, I heard her sigh low.  She
was pensive a few minutes, then rousing herself, she said cheerfully--

"But you two are my visitors to-night; I must treat you as such."  She
rang her bell.

"Barbara," she said to the servant who answered it, "I have not yet had
tea; bring the tray and place cups for these two young ladies."

And a tray was soon brought.  How pretty, to my eyes, did the china cups
and bright teapot look, placed on the little round table near the fire!
How fragrant was the steam of the beverage, and the scent of the toast!
of which, however, I, to my dismay (for I was beginning to be hungry)
discerned only a very small portion: Miss Temple discerned it too.

"Barbara," said she, "can you not bring a little more bread and butter?
There is not enough for three."

Barbara went out: she returned soon--

"Madam, Mrs. Harden says she has sent up the usual quantity."

Mrs. Harden, be it observed, was the housekeeper: a woman after Mr.
Brocklehurst's own heart, made up of equal parts of whalebone and iron.

"Oh, very well!" returned Miss Temple; "we must make it do, Barbara, I
suppose."  And as the girl withdrew she added, smiling, "Fortunately, I
have it in my power to supply deficiencies for this once."

Having invited Helen and me to approach the table, and placed before each
of us a cup of tea with one delicious but thin morsel of toast, she got
up, unlocked a drawer, and taking from it a parcel wrapped in paper,
disclosed presently to our eyes a good-sized seed-cake.

"I meant to give each of you some of this to take with you," said she,
"but as there is so little toast, you must have it now," and she
proceeded to cut slices with a generous hand.

We feasted that evening as on nectar and ambrosia; and not the least
delight of the entertainment was the smile of gratification with which
our hostess regarded us, as we satisfied our famished appetites on the
delicate fare she liberally supplied.

Tea over and the tray removed, she again summoned us to the fire; we sat
one on each side of her, and now a conversation followed between her and
Helen, which it was indeed a privilege to be admitted to hear.

Miss Temple had always something of serenity in her air, of state in her
mien, of refined propriety in her language, which precluded deviation
into the ardent, the excited, the eager: something which chastened the
pleasure of those who looked on her and listened to her, by a controlling
sense of awe; and such was my feeling now: but as to Helen Burns, I was
struck with wonder.

The refreshing meal, the brilliant fire, the presence and kindness of her
beloved instructress, or, perhaps, more than all these, something in her
own unique mind, had roused her powers within her.  They woke, they
kindled: first, they glowed in the bright tint of her cheek, which till
this hour I had never seen but pale and bloodless; then they shone in the
liquid lustre of her eyes, which had suddenly acquired a beauty more
singular than that of Miss Temple's--a beauty neither of fine colour nor
long eyelash, nor pencilled brow, but of meaning, of movement, of
radiance.  Then her soul sat on her lips, and language flowed, from what
source I cannot tell.  Has a girl of fourteen a heart large enough,
vigorous enough, to hold the swelling spring of pure, full, fervid
eloquence?  Such was the characteristic of Helen's discourse on that, to
me, memorable evening; her spirit seemed hastening to live within a very
brief span as much as many live during a protracted existence.

They conversed of things I had never heard of; of nations and times past;
of countries far away; of secrets of nature discovered or guessed at:
they spoke of books: how many they had read!  What stores of knowledge
they possessed!  Then they seemed so familiar with French names and
French authors: but my amazement reached its climax when Miss Temple
asked Helen if she sometimes snatched a moment to recall the Latin her
father had taught her, and taking a book from a shelf, bade her read and
construe a page of Virgil; and Helen obeyed, my organ of veneration
expanding at every sounding line.  She had scarcely finished ere the bell
announced bedtime! no delay could be admitted; Miss Temple embraced us
both, saying, as she drew us to her heart--

"God bless you, my children!"

Helen she held a little longer than me: she let her go more reluctantly;
it was Helen her eye followed to the door; it was for her she a second
time breathed a sad sigh; for her she wiped a tear from her cheek.

On reaching the bedroom, we heard the voice of Miss Scatcherd: she was
examining drawers; she had just pulled out Helen Burns's, and when we
entered Helen was greeted with a sharp reprimand, and told that to-morrow
she should have half-a-dozen of untidily folded articles pinned to her
shoulder.

"My things were indeed in shameful disorder," murmured Helen to me, in a
low voice: "I intended to have arranged them, but I forgot."

Next morning, Miss Scatcherd wrote in conspicuous characters on a piece
of pasteboard the word "Slattern," and bound it like a phylactery round
Helen's large, mild, intelligent, and benign-looking forehead.  She wore
it till evening, patient, unresentful, regarding it as a deserved
punishment.  The moment Miss Scatcherd withdrew after afternoon school, I
ran to Helen, tore it off, and thrust it into the fire: the fury of which
she was incapable had been burning in my soul all day, and tears, hot and
large, had continually been scalding my cheek; for the spectacle of her
sad resignation gave me an intolerable pain at the heart.

About a week subsequently to the incidents above narrated, Miss Temple,
who had written to Mr. Lloyd, received his answer: it appeared that what
he said went to corroborate my account.  Miss Temple, having assembled
the whole school, announced that inquiry had been made into the charges
alleged against Jane Eyre, and that she was most happy to be able to
pronounce her completely cleared from every imputation.  The teachers
then shook hands with me and kissed me, and a murmur of pleasure ran
through the ranks of my companions.

Thus relieved of a grievous load, I from that hour set to work afresh,
resolved to pioneer my way through every difficulty: I toiled hard, and
my success was proportionate to my efforts; my memory, not naturally
tenacious, improved with practice; exercise sharpened my wits; in a few
weeks I was promoted to a higher class; in less than two months I was
allowed to commence French and drawing.  I learned the first two tenses
of the verb \emph{Etre}, and sketched my first cottage (whose walls, by-the-
bye, outrivalled in slope those of the leaning tower of Pisa), on the
same day.  That night, on going to bed, I forgot to prepare in
imagination the Barmecide supper of hot roast potatoes, or white bread
and new milk, with which I was wont to amuse my inward cravings: I
feasted instead on the spectacle of ideal drawings, which I saw in the
dark; all the work of my own hands: freely pencilled houses and trees,
picturesque rocks and ruins, Cuyp-like groups of cattle, sweet paintings
of butterflies hovering over unblown roses, of birds picking at ripe
cherries, of wren's nests enclosing pearl-like eggs, wreathed about with
young ivy sprays.  I examined, too, in thought, the possibility of my
ever being able to translate currently a certain little French story
which Madame Pierrot had that day shown me; nor was that problem solved
to my satisfaction ere I fell sweetly asleep.

Well has Solomon said--"Better is a dinner of herbs where love is, than a
stalled ox and hatred therewith."

I would not now have exchanged Lowood with all its privations for
Gateshead and its daily luxuries.




\chapter{}


\lettrine{B}{ut} the privations, or rather the hardships, of Lowood lessened.  Spring
drew on: she was indeed already come; the frosts of winter had ceased;
its snows were melted, its cutting winds ameliorated.  My wretched feet,
flayed and swollen to lameness by the sharp air of January, began to heal
and subside under the gentler breathings of April; the nights and
mornings no longer by their Canadian temperature froze the very blood in
our veins; we could now endure the play-hour passed in the garden:
sometimes on a sunny day it began even to be pleasant and genial, and a
greenness grew over those brown beds, which, freshening daily, suggested
the thought that Hope traversed them at night, and left each morning
brighter traces of her steps.  Flowers peeped out amongst the leaves;
snow-drops, crocuses, purple auriculas, and golden-eyed pansies.  On
Thursday afternoons (half-holidays) we now took walks, and found still
sweeter flowers opening by the wayside, under the hedges.

I discovered, too, that a great pleasure, an enjoyment which the horizon
only bounded, lay all outside the high and spike-guarded walls of our
garden: this pleasure consisted in prospect of noble summits girdling a
great hill-hollow, rich in verdure and shadow; in a bright beck, full of
dark stones and sparkling eddies.  How different had this scene looked
when I viewed it laid out beneath the iron sky of winter, stiffened in
frost, shrouded with snow!--when mists as chill as death wandered to the
impulse of east winds along those purple peaks, and rolled down "ing" and
holm till they blended with the frozen fog of the beck!  That beck itself
was then a torrent, turbid and curbless: it tore asunder the wood, and
sent a raving sound through the air, often thickened with wild rain or
whirling sleet; and for the forest on its banks, \emph{that} showed only ranks
of skeletons.

April advanced to May: a bright serene May it was; days of blue sky,
placid sunshine, and soft western or southern gales filled up its
duration.  And now vegetation matured with vigour; Lowood shook loose its
tresses; it became all green, all flowery; its great elm, ash, and oak
skeletons were restored to majestic life; woodland plants sprang up
profusely in its recesses; unnumbered varieties of moss filled its
hollows, and it made a strange ground-sunshine out of the wealth of its
wild primrose plants: I have seen their pale gold gleam in overshadowed
spots like scatterings of the sweetest lustre.  All this I enjoyed often
and fully, free, unwatched, and almost alone: for this unwonted liberty
and pleasure there was a cause, to which it now becomes my task to
advert.

Have I not described a pleasant site for a dwelling, when I speak of it
as bosomed in hill and wood, and rising from the verge of a stream?
Assuredly, pleasant enough: but whether healthy or not is another
question.

That forest-dell, where Lowood lay, was the cradle of fog and fog-bred
pestilence; which, quickening with the quickening spring, crept into the
Orphan Asylum, breathed typhus through its crowded schoolroom and
dormitory, and, ere May arrived, transformed the seminary into an
hospital.

Semi-starvation and neglected colds had predisposed most of the pupils to
receive infection: forty-five out of the eighty girls lay ill at one
time.  Classes were broken up, rules relaxed.  The few who continued well
were allowed almost unlimited license; because the medical attendant
insisted on the necessity of frequent exercise to keep them in health:
and had it been otherwise, no one had leisure to watch or restrain them.
Miss Temple's whole attention was absorbed by the patients: she lived in
the sick-room, never quitting it except to snatch a few hours' rest at
night.  The teachers were fully occupied with packing up and making other
necessary preparations for the departure of those girls who were
fortunate enough to have friends and relations able and willing to remove
them from the seat of contagion.  Many, already smitten, went home only
to die: some died at the school, and were buried quietly and quickly, the
nature of the malady forbidding delay.

While disease had thus become an inhabitant of Lowood, and death its
frequent visitor; while there was gloom and fear within its walls; while
its rooms and passages steamed with hospital smells, the drug and the
pastille striving vainly to overcome the effluvia of mortality, that
bright May shone unclouded over the bold hills and beautiful woodland out
of doors.  Its garden, too, glowed with flowers: hollyhocks had sprung up
tall as trees, lilies had opened, tulips and roses were in bloom; the
borders of the little beds were gay with pink thrift and crimson double
daisies; the sweetbriars gave out, morning and evening, their scent of
spice and apples; and these fragrant treasures were all useless for most
of the inmates of Lowood, except to furnish now and then a handful of
herbs and blossoms to put in a coffin.

But I, and the rest who continued well, enjoyed fully the beauties of the
scene and season; they let us ramble in the wood, like gipsies, from
morning till night; we did what we liked, went where we liked: we lived
better too.  Mr. Brocklehurst and his family never came near Lowood now:
household matters were not scrutinised into; the cross housekeeper was
gone, driven away by the fear of infection; her successor, who had been
matron at the Lowton Dispensary, unused to the ways of her new abode,
provided with comparative liberality.  Besides, there were fewer to feed;
the sick could eat little; our breakfast-basins were better filled; when
there was no time to prepare a regular dinner, which often happened, she
would give us a large piece of cold pie, or a thick slice of bread and
cheese, and this we carried away with us to the wood, where we each chose
the spot we liked best, and dined sumptuously.

My favourite seat was a smooth and broad stone, rising white and dry from
the very middle of the beck, and only to be got at by wading through the
water; a feat I accomplished barefoot.  The stone was just broad enough
to accommodate, comfortably, another girl and me, at that time my chosen
comrade--one Mary Ann Wilson; a shrewd, observant personage, whose
society I took pleasure in, partly because she was witty and original,
and partly because she had a manner which set me at my ease.  Some years
older than I, she knew more of the world, and could tell me many things I
liked to hear: with her my curiosity found gratification: to my faults
also she gave ample indulgence, never imposing curb or rein on anything I
said.  She had a turn for narrative, I for analysis; she liked to inform,
I to question; so we got on swimmingly together, deriving much
entertainment, if not much improvement, from our mutual intercourse.

And where, meantime, was Helen Burns?  Why did I not spend these sweet
days of liberty with her?  Had I forgotten her? or was I so worthless as
to have grown tired of her pure society?  Surely the Mary Ann Wilson I
have mentioned was inferior to my first acquaintance: she could only tell
me amusing stories, and reciprocate any racy and pungent gossip I chose
to indulge in; while, if I have spoken truth of Helen, she was qualified
to give those who enjoyed the privilege of her converse a taste of far
higher things.

True, reader; and I knew and felt this: and though I am a defective
being, with many faults and few redeeming points, yet I never tired of
Helen Burns; nor ever ceased to cherish for her a sentiment of
attachment, as strong, tender, and respectful as any that ever animated
my heart.  How could it be otherwise, when Helen, at all times and under
all circumstances, evinced for me a quiet and faithful friendship, which
ill-humour never soured, nor irritation never troubled?  But Helen was
ill at present: for some weeks she had been removed from my sight to I
knew not what room upstairs.  She was not, I was told, in the hospital
portion of the house with the fever patients; for her complaint was
consumption, not typhus: and by consumption I, in my ignorance,
understood something mild, which time and care would be sure to
alleviate.

I was confirmed in this idea by the fact of her once or twice coming
downstairs on very warm sunny afternoons, and being taken by Miss Temple
into the garden; but, on these occasions, I was not allowed to go and
speak to her; I only saw her from the schoolroom window, and then not
distinctly; for she was much wrapped up, and sat at a distance under the
verandah.

One evening, in the beginning of June, I had stayed out very late with
Mary Ann in the wood; we had, as usual, separated ourselves from the
others, and had wandered far; so far that we lost our way, and had to ask
it at a lonely cottage, where a man and woman lived, who looked after a
herd of half-wild swine that fed on the mast in the wood.  When we got
back, it was after moonrise: a pony, which we knew to be the surgeon's,
was standing at the garden door.  Mary Ann remarked that she supposed
some one must be very ill, as Mr. Bates had been sent for at that time of
the evening.  She went into the house; I stayed behind a few minutes to
plant in my garden a handful of roots I had dug up in the forest, and
which I feared would wither if I left them till the morning.  This done,
I lingered yet a little longer: the flowers smelt so sweet as the dew
fell; it was such a pleasant evening, so serene, so warm; the still
glowing west promised so fairly another fine day on the morrow; the moon
rose with such majesty in the grave east.  I was noting these things and
enjoying them as a child might, when it entered my mind as it had never
done before:--

"How sad to be lying now on a sick bed, and to be in danger of dying!
This world is pleasant--it would be dreary to be called from it, and to
have to go who knows where?"

And then my mind made its first earnest effort to comprehend what had
been infused into it concerning heaven and hell; and for the first time
it recoiled, baffled; and for the first time glancing behind, on each
side, and before it, it saw all round an unfathomed gulf: it felt the one
point where it stood--the present; all the rest was formless cloud and
vacant depth; and it shuddered at the thought of tottering, and plunging
amid that chaos.  While pondering this new idea, I heard the front door
open; Mr. Bates came out, and with him was a nurse.  After she had seen
him mount his horse and depart, she was about to close the door, but I
ran up to her.

"How is Helen Burns?"

"Very poorly," was the answer.

"Is it her Mr. Bates has been to see?"

"Yes."

"And what does he say about her?"

"He says she'll not be here long."

This phrase, uttered in my hearing yesterday, would have only conveyed
the notion that she was about to be removed to Northumberland, to her own
home.  I should not have suspected that it meant she was dying; but I
knew instantly now!  It opened clear on my comprehension that Helen Burns
was numbering her last days in this world, and that she was going to be
taken to the region of spirits, if such region there were.  I experienced
a shock of horror, then a strong thrill of grief, then a desire--a
necessity to see her; and I asked in what room she lay.

"She is in Miss Temple's room," said the nurse.

"May I go up and speak to her?"

"Oh no, child!  It is not likely; and now it is time for you to come in;
you'll catch the fever if you stop out when the dew is falling."

The nurse closed the front door; I went in by the side entrance which led
to the schoolroom: I was just in time; it was nine o'clock, and Miss
Miller was calling the pupils to go to bed.

It might be two hours later, probably near eleven, when I--not having
been able to fall asleep, and deeming, from the perfect silence of the
dormitory, that my companions were all wrapt in profound repose--rose
softly, put on my frock over my night-dress, and, without shoes, crept
from the apartment, and set off in quest of Miss Temple's room.  It was
quite at the other end of the house; but I knew my way; and the light of
the unclouded summer moon, entering here and there at passage windows,
enabled me to find it without difficulty.  An odour of camphor and burnt
vinegar warned me when I came near the fever room: and I passed its door
quickly, fearful lest the nurse who sat up all night should hear me.  I
dreaded being discovered and sent back; for I \emph{must} see Helen,--I must
embrace her before she died,--I must give her one last kiss, exchange
with her one last word.

Having descended a staircase, traversed a portion of the house below, and
succeeded in opening and shutting, without noise, two doors, I reached
another flight of steps; these I mounted, and then just opposite to me
was Miss Temple's room.  A light shone through the keyhole and from under
the door; a profound stillness pervaded the vicinity.  Coming near, I
found the door slightly ajar; probably to admit some fresh air into the
close abode of sickness.  Indisposed to hesitate, and full of impatient
impulses--soul and senses quivering with keen throes--I put it back and
looked in.  My eye sought Helen, and feared to find death.

Close by Miss Temple's bed, and half covered with its white curtains,
there stood a little crib.  I saw the outline of a form under the
clothes, but the face was hid by the hangings: the nurse I had spoken to
in the garden sat in an easy-chair asleep; an unsnuffed candle burnt
dimly on the table.  Miss Temple was not to be seen: I knew afterwards
that she had been called to a delirious patient in the fever-room.  I
advanced; then paused by the crib side: my hand was on the curtain, but I
preferred speaking before I withdrew it.  I still recoiled at the dread
of seeing a corpse.

"Helen!" I whispered softly, "are you awake?"

She stirred herself, put back the curtain, and I saw her face, pale,
wasted, but quite composed: she looked so little changed that my fear was
instantly dissipated.

"Can it be you, Jane?" she asked, in her own gentle voice.

"Oh!" I thought, "she is not going to die; they are mistaken: she could
not speak and look so calmly if she were."

I got on to her crib and kissed her: her forehead was cold, and her cheek
both cold and thin, and so were her hand and wrist; but she smiled as of
old.

"Why are you come here, Jane?  It is past eleven o'clock: I heard it
strike some minutes since."

"I came to see you, Helen: I heard you were very ill, and I could not
sleep till I had spoken to you."

"You came to bid me good-bye, then: you are just in time probably."

"Are you going somewhere, Helen?  Are you going home?"

"Yes; to my long home--my last home."

"No, no, Helen!"  I stopped, distressed.  While I tried to devour my
tears, a fit of coughing seized Helen; it did not, however, wake the
nurse; when it was over, she lay some minutes exhausted; then she
whispered--

"Jane, your little feet are bare; lie down and cover yourself with my
quilt."

I did so: she put her arm over me, and I nestled close to her.  After a
long silence, she resumed, still whispering--

"I am very happy, Jane; and when you hear that I am dead, you must be
sure and not grieve: there is nothing to grieve about.  We all must die
one day, and the illness which is removing me is not painful; it is
gentle and gradual: my mind is at rest.  I leave no one to regret me
much: I have only a father; and he is lately married, and will not miss
me.  By dying young, I shall escape great sufferings.  I had not
qualities or talents to make my way very well in the world: I should have
been continually at fault."

"But where are you going to, Helen?  Can you see?  Do you know?"

"I believe; I have faith: I am going to God."

"Where is God?  What is God?"

"My Maker and yours, who will never destroy what He created.  I rely
implicitly on His power, and confide wholly in His goodness: I count the
hours till that eventful one arrives which shall restore me to Him,
reveal Him to me."

"You are sure, then, Helen, that there is such a place as heaven, and
that our souls can get to it when we die?"

"I am sure there is a future state; I believe God is good; I can resign
my immortal part to Him without any misgiving.  God is my father; God is
my friend: I love Him; I believe He loves me."

"And shall I see you again, Helen, when I die?"

"You will come to the same region of happiness: be received by the same
mighty, universal Parent, no doubt, dear Jane."

Again I questioned, but this time only in thought.  "Where is that
region?  Does it exist?"  And I clasped my arms closer round Helen; she
seemed dearer to me than ever; I felt as if I could not let her go; I lay
with my face hidden on her neck.  Presently she said, in the sweetest
tone--

"How comfortable I am!  That last fit of coughing has tired me a little;
I feel as if I could sleep: but don't leave me, Jane; I like to have you
near me."

"I'll stay with you, \emph{dear} Helen: no one shall take me away."

"Are you warm, darling?"

"Yes."

"Good-night, Jane."

"Good-night, Helen."

She kissed me, and I her, and we both soon slumbered.

When I awoke it was day: an unusual movement roused me; I looked up; I
was in somebody's arms; the nurse held me; she was carrying me through
the passage back to the dormitory.  I was not reprimanded for leaving my
bed; people had something else to think about; no explanation was
afforded then to my many questions; but a day or two afterwards I learned
that Miss Temple, on returning to her own room at dawn, had found me laid
in the little crib; my face against Helen Burns's shoulder, my arms round
her neck.  I was asleep, and Helen was--dead.

Her grave is in Brocklebridge churchyard: for fifteen years after her
death it was only covered by a grassy mound; but now a grey marble tablet
marks the spot, inscribed with her name, and the word "Resurgam."




\chapter{}


\lettrine{H}{itherto} I have recorded in detail the events of my insignificant
existence: to the first ten years of my life I have given almost as many
chapters.  But this is not to be a regular autobiography.  I am only
bound to invoke Memory where I know her responses will possess some
degree of interest; therefore I now pass a space of eight years almost in
silence: a few lines only are necessary to keep up the links of
connection.

When the typhus fever had fulfilled its mission of devastation at Lowood,
it gradually disappeared from thence; but not till its virulence and the
number of its victims had drawn public attention on the school.  Inquiry
was made into the origin of the scourge, and by degrees various facts
came out which excited public indignation in a high degree.  The
unhealthy nature of the site; the quantity and quality of the children's
food; the brackish, fetid water used in its preparation; the pupils'
wretched clothing and accommodations--all these things were discovered,
and the discovery produced a result mortifying to Mr. Brocklehurst, but
beneficial to the institution.

Several wealthy and benevolent individuals in the county subscribed
largely for the erection of a more convenient building in a better
situation; new regulations were made; improvements in diet and clothing
introduced; the funds of the school were intrusted to the management of a
committee.  Mr. Brocklehurst, who, from his wealth and family
connections, could not be overlooked, still retained the post of
treasurer; but he was aided in the discharge of his duties by gentlemen
of rather more enlarged and sympathising minds: his office of inspector,
too, was shared by those who knew how to combine reason with strictness,
comfort with economy, compassion with uprightness.  The school, thus
improved, became in time a truly useful and noble institution.  I
remained an inmate of its walls, after its regeneration, for eight years:
six as pupil, and two as teacher; and in both capacities I bear my
testimony to its value and importance.

During these eight years my life was uniform: but not unhappy, because it
was not inactive.  I had the means of an excellent education placed
within my reach; a fondness for some of my studies, and a desire to excel
in all, together with a great delight in pleasing my teachers, especially
such as I loved, urged me on: I availed myself fully of the advantages
offered me.  In time I rose to be the first girl of the first class; then
I was invested with the office of teacher; which I discharged with zeal
for two years: but at the end of that time I altered.

Miss Temple, through all changes, had thus far continued superintendent
of the seminary: to her instruction I owed the best part of my
acquirements; her friendship and society had been my continual solace;
she had stood me in the stead of mother, governess, and, latterly,
companion.  At this period she married, removed with her husband (a
clergyman, an excellent man, almost worthy of such a wife) to a distant
county, and consequently was lost to me.

From the day she left I was no longer the same: with her was gone every
settled feeling, every association that had made Lowood in some degree a
home to me.  I had imbibed from her something of her nature and much of
her habits: more harmonious thoughts: what seemed better regulated
feelings had become the inmates of my mind.  I had given in allegiance to
duty and order; I was quiet; I believed I was content: to the eyes of
others, usually even to my own, I appeared a disciplined and subdued
character.

But destiny, in the shape of the Rev. Mr. Nasmyth, came between me and
Miss Temple: I saw her in her travelling dress step into a post-chaise,
shortly after the marriage ceremony; I watched the chaise mount the hill
and disappear beyond its brow; and then retired to my own room, and there
spent in solitude the greatest part of the half-holiday granted in honour
of the occasion.

I walked about the chamber most of the time.  I imagined myself only to
be regretting my loss, and thinking how to repair it; but when my
reflections were concluded, and I looked up and found that the afternoon
was gone, and evening far advanced, another discovery dawned on me,
namely, that in the interval I had undergone a transforming process; that
my mind had put off all it had borrowed of Miss Temple--or rather that
she had taken with her the serene atmosphere I had been breathing in her
vicinity--and that now I was left in my natural element, and beginning to
feel the stirring of old emotions.  It did not seem as if a prop were
withdrawn, but rather as if a motive were gone: it was not the power to
be tranquil which had failed me, but the reason for tranquillity was no
more.  My world had for some years been in Lowood: my experience had been
of its rules and systems; now I remembered that the real world was wide,
and that a varied field of hopes and fears, of sensations and
excitements, awaited those who had courage to go forth into its expanse,
to seek real knowledge of life amidst its perils.

I went to my window, opened it, and looked out.  There were the two wings
of the building; there was the garden; there were the skirts of Lowood;
there was the hilly horizon.  My eye passed all other objects to rest on
those most remote, the blue peaks; it was those I longed to surmount; all
within their boundary of rock and heath seemed prison-ground, exile
limits.  I traced the white road winding round the base of one mountain,
and vanishing in a gorge between two; how I longed to follow it farther!
I recalled the time when I had travelled that very road in a coach; I
remembered descending that hill at twilight; an age seemed to have
elapsed since the day which brought me first to Lowood, and I had never
quitted it since.  My vacations had all been spent at school: Mrs. Reed
had never sent for me to Gateshead; neither she nor any of her family had
ever been to visit me.  I had had no communication by letter or message
with the outer world: school-rules, school-duties, school-habits and
notions, and voices, and faces, and phrases, and costumes, and
preferences, and antipathies--such was what I knew of existence.  And now
I felt that it was not enough; I tired of the routine of eight years in
one afternoon.  I desired liberty; for liberty I gasped; for liberty I
uttered a prayer; it seemed scattered on the wind then faintly blowing.  I
abandoned it and framed a humbler supplication; for change, stimulus:
that petition, too, seemed swept off into vague space: "Then," I cried,
half desperate, "grant me at least a new servitude!"

Here a bell, ringing the hour of supper, called me downstairs.

I was not free to resume the interrupted chain of my reflections till
bedtime: even then a teacher who occupied the same room with me kept me
from the subject to which I longed to recur, by a prolonged effusion of
small talk.  How I wished sleep would silence her.  It seemed as if,
could I but go back to the idea which had last entered my mind as I stood
at the window, some inventive suggestion would rise for my relief.

Miss Gryce snored at last; she was a heavy Welshwoman, and till now her
habitual nasal strains had never been regarded by me in any other light
than as a nuisance; to-night I hailed the first deep notes with
satisfaction; I was debarrassed of interruption; my half-effaced thought
instantly revived.

"A new servitude!  There is something in that," I soliloquised (mentally,
be it understood; I did not talk aloud), "I know there is, because it
does not sound too sweet; it is not like such words as Liberty,
Excitement, Enjoyment: delightful sounds truly; but no more than sounds
for me; and so hollow and fleeting that it is mere waste of time to
listen to them.  But Servitude!  That must be matter of fact.  Any one
may serve: I have served here eight years; now all I want is to serve
elsewhere.  Can I not get so much of my own will?  Is not the thing
feasible?  Yes--yes--the end is not so difficult; if I had only a brain
active enough to ferret out the means of attaining it."

I sat up in bed by way of arousing this said brain: it was a chilly
night; I covered my shoulders with a shawl, and then I proceeded \emph{to
think} again with all my might.

"What do I want?  A new place, in a new house, amongst new faces, under
new circumstances: I want this because it is of no use wanting anything
better.  How do people do to get a new place?  They apply to friends, I
suppose: I have no friends.  There are many others who have no friends,
who must look about for themselves and be their own helpers; and what is
their resource?"

I could not tell: nothing answered me; I then ordered my brain to find a
response, and quickly.  It worked and worked faster: I felt the pulses
throb in my head and temples; but for nearly an hour it worked in chaos;
and no result came of its efforts.  Feverish with vain labour, I got up
and took a turn in the room; undrew the curtain, noted a star or two,
shivered with cold, and again crept to bed.

A kind fairy, in my absence, had surely dropped the required suggestion
on my pillow; for as I lay down, it came quietly and naturally to my
mind.--"Those who want situations advertise; you must advertise in the
\emph{---shire Herald}."

"How?  I know nothing about advertising."

Replies rose smooth and prompt now:--

"You must enclose the advertisement and the money to pay for it under a
cover directed to the editor of the \emph{Herald}; you must put it, the first
opportunity you have, into the post at Lowton; answers must be addressed
to J.E., at the post-office there; you can go and inquire in about a week
after you send your letter, if any are come, and act accordingly."

This scheme I went over twice, thrice; it was then digested in my mind; I
had it in a clear practical form: I felt satisfied, and fell asleep.

With earliest day, I was up: I had my advertisement written, enclosed,
and directed before the bell rang to rouse the school; it ran thus:--

"A young lady accustomed to tuition" (had I not been a teacher two
years?) "is desirous of meeting with a situation in a private family
where the children are under fourteen" (I thought that as I was barely
eighteen, it would not do to undertake the guidance of pupils nearer my
own age).  "She is qualified to teach the usual branches of a good English
education, together with French, Drawing, and Music" (in those days,
reader, this now narrow catalogue of accomplishments, would have been
held tolerably comprehensive).  "Address, J.E., Post-office, Lowton, ---
shire."

This document remained locked in my drawer all day: after tea, I asked
leave of the new superintendent to go to Lowton, in order to perform some
small commissions for myself and one or two of my fellow-teachers;
permission was readily granted; I went.  It was a walk of two miles, and
the evening was wet, but the days were still long; I visited a shop or
two, slipped the letter into the post-office, and came back through heavy
rain, with streaming garments, but with a relieved heart.

The succeeding week seemed long: it came to an end at last, however, like
all sublunary things, and once more, towards the close of a pleasant
autumn day, I found myself afoot on the road to Lowton.  A picturesque
track it was, by the way; lying along the side of the beck and through
the sweetest curves of the dale: but that day I thought more of the
letters, that might or might not be awaiting me at the little burgh
whither I was bound, than of the charms of lea and water.

My ostensible errand on this occasion was to get measured for a pair of
shoes; so I discharged that business first, and when it was done, I
stepped across the clean and quiet little street from the shoemaker's to
the post-office: it was kept by an old dame, who wore horn spectacles on
her nose, and black mittens on her hands.

"Are there any letters for J.E.?" I asked.

She peered at me over her spectacles, and then she opened a drawer and
fumbled among its contents for a long time, so long that my hopes began
to falter.  At last, having held a document before her glasses for nearly
five minutes, she presented it across the counter, accompanying the act
by another inquisitive and mistrustful glance--it was for J.E.

"Is there only one?" I demanded.

"There are no more," said she; and I put it in my pocket and turned my
face homeward: I could not open it then; rules obliged me to be back by
eight, and it was already half-past seven.

Various duties awaited me on my arrival.  I had to sit with the girls
during their hour of study; then it was my turn to read prayers; to see
them to bed: afterwards I supped with the other teachers.  Even when we
finally retired for the night, the inevitable Miss Gryce was still my
companion: we had only a short end of candle in our candlestick, and I
dreaded lest she should talk till it was all burnt out; fortunately,
however, the heavy supper she had eaten produced a soporific effect: she
was already snoring before I had finished undressing.  There still
remained an inch of candle: I now took out my letter; the seal was an
initial F.; I broke it; the contents were brief.

"If J.E., who advertised in the \emph{---shire Herald} of last Thursday,
possesses the acquirements mentioned, and if she is in a position to give
satisfactory references as to character and competency, a situation can
be offered her where there is but one pupil, a little girl, under ten
years of age; and where the salary is thirty pounds per annum.  J.E. is
requested to send references, name, address, and all particulars to the
direction:--

"Mrs. Fairfax, Thornfield, near Millcote, ---shire."

I examined the document long: the writing was old-fashioned and rather
uncertain, like that of an elderly lady.  This circumstance was
satisfactory: a private fear had haunted me, that in thus acting for
myself, and by my own guidance, I ran the risk of getting into some
scrape; and, above all things, I wished the result of my endeavours to be
respectable, proper, \emph{en regle}.  I now felt that an elderly lady was no
bad ingredient in the business I had on hand.  Mrs. Fairfax!  I saw her
in a black gown and widow's cap; frigid, perhaps, but not uncivil: a
model of elderly English respectability.  Thornfield! that, doubtless,
was the name of her house: a neat orderly spot, I was sure; though I
failed in my efforts to conceive a correct plan of the premises.
Millcote, ---shire; I brushed up my recollections of the map of England,
yes, I saw it; both the shire and the town.  ---shire was seventy miles
nearer London than the remote county where I now resided: that was a
recommendation to me.  I longed to go where there was life and movement:
Millcote was a large manufacturing town on the banks of the A-; a busy
place enough, doubtless: so much the better; it would be a complete
change at least.  Not that my fancy was much captivated by the idea of
long chimneys and clouds of smoke--"but," I argued, "Thornfield will,
probably, be a good way from the town."

Here the socket of the candle dropped, and the wick went out.

Next day new steps were to be taken; my plans could no longer be confined
to my own breast; I must impart them in order to achieve their success.
Having sought and obtained an audience of the superintendent during the
noontide recreation, I told her I had a prospect of getting a new
situation where the salary would be double what I now received (for at
Lowood I only got 15 pounds per annum); and requested she would break the
matter for me to Mr. Brocklehurst, or some of the committee, and
ascertain whether they would permit me to mention them as references.  She
obligingly consented to act as mediatrix in the matter.  The next day she
laid the affair before Mr. Brocklehurst, who said that Mrs. Reed must be
written to, as she was my natural guardian.  A note was accordingly
addressed to that lady, who returned for answer, that "I might do as I
pleased: she had long relinquished all interference in my affairs."  This
note went the round of the committee, and at last, after what appeared to
me most tedious delay, formal leave was given me to better my condition
if I could; and an assurance added, that as I had always conducted myself
well, both as teacher and pupil, at Lowood, a testimonial of character
and capacity, signed by the inspectors of that institution, should
forthwith be furnished me.

This testimonial I accordingly received in about a month, forwarded a
copy of it to Mrs. Fairfax, and got that lady's reply, stating that she
was satisfied, and fixing that day fortnight as the period for my
assuming the post of governess in her house.

I now busied myself in preparations: the fortnight passed rapidly.  I had
not a very large wardrobe, though it was adequate to my wants; and the
last day sufficed to pack my trunk,--the same I had brought with me eight
years ago from Gateshead.

The box was corded, the card nailed on.  In half-an-hour the carrier was
to call for it to take it to Lowton, whither I myself was to repair at an
early hour the next morning to meet the coach.  I had brushed my black
stuff travelling-dress, prepared my bonnet, gloves, and muff; sought in
all my drawers to see that no article was left behind; and now having
nothing more to do, I sat down and tried to rest.  I could not; though I
had been on foot all day, I could not now repose an instant; I was too
much excited.  A phase of my life was closing to-night, a new one opening
to-morrow: impossible to slumber in the interval; I must watch feverishly
while the change was being accomplished.

"Miss," said a servant who met me in the lobby, where I was wandering
like a troubled spirit, "a person below wishes to see you."

"The carrier, no doubt," I thought, and ran downstairs without inquiry.  I
was passing the back-parlour or teachers' sitting-room, the door of which
was half open, to go to the kitchen, when some one ran out--

"It's her, I am sure!--I could have told her anywhere!" cried the
individual who stopped my progress and took my hand.

I looked: I saw a woman attired like a well-dressed servant, matronly,
yet still young; very good-looking, with black hair and eyes, and lively
complexion.

"Well, who is it?" she asked, in a voice and with a smile I half
recognised; "you've not quite forgotten me, I think, Miss Jane?"

In another second I was embracing and kissing her rapturously: "Bessie!
Bessie!  Bessie!" that was all I said; whereat she half laughed, half
cried, and we both went into the parlour.  By the fire stood a little
fellow of three years old, in plaid frock and trousers.

"That is my little boy," said Bessie directly.

"Then you are married, Bessie?"

"Yes; nearly five years since to Robert Leaven, the coachman; and I've a
little girl besides Bobby there, that I've christened Jane."

"And you don't live at Gateshead?"

"I live at the lodge: the old porter has left."

"Well, and how do they all get on?  Tell me everything about them,
Bessie: but sit down first; and, Bobby, come and sit on my knee, will
you?" but Bobby preferred sidling over to his mother.

"You're not grown so very tall, Miss Jane, nor so very stout," continued
Mrs. Leaven.  "I dare say they've not kept you too well at school: Miss
Reed is the head and shoulders taller than you are; and Miss Georgiana
would make two of you in breadth."

"Georgiana is handsome, I suppose, Bessie?"

"Very.  She went up to London last winter with her mama, and there
everybody admired her, and a young lord fell in love with her: but his
relations were against the match; and--what do you think?--he and Miss
Georgiana made it up to run away; but they were found out and stopped.  It
was Miss Reed that found them out: I believe she was envious; and now she
and her sister lead a cat and dog life together; they are always
quarrelling--"

"Well, and what of John Reed?"

"Oh, he is not doing so well as his mama could wish.  He went to college,
and he got--plucked, I think they call it: and then his uncles wanted him
to be a barrister, and study the law: but he is such a dissipated young
man, they will never make much of him, I think."

"What does he look like?"

"He is very tall: some people call him a fine-looking young man; but he
has such thick lips."

"And Mrs. Reed?"

"Missis looks stout and well enough in the face, but I think she's not
quite easy in her mind: Mr. John's conduct does not please her--he spends
a deal of money."

"Did she send you here, Bessie?"

"No, indeed: but I have long wanted to see you, and when I heard that
there had been a letter from you, and that you were going to another part
of the country, I thought I'd just set off, and get a look at you before
you were quite out of my reach."

"I am afraid you are disappointed in me, Bessie."  I said this laughing:
I perceived that Bessie's glance, though it expressed regard, did in no
shape denote admiration.

"No, Miss Jane, not exactly: you are genteel enough; you look like a
lady, and it is as much as ever I expected of you: you were no beauty as
a child."

I smiled at Bessie's frank answer: I felt that it was correct, but I
confess I was not quite indifferent to its import: at eighteen most
people wish to please, and the conviction that they have not an exterior
likely to second that desire brings anything but gratification.

"I dare say you are clever, though," continued Bessie, by way of solace.
"What can you do?  Can you play on the piano?"

"A little."

There was one in the room; Bessie went and opened it, and then asked me
to sit down and give her a tune: I played a waltz or two, and she was
charmed.

"The Miss Reeds could not play as well!" said she exultingly.  "I always
said you would surpass them in learning: and can you draw?"

"That is one of my paintings over the chimney-piece."  It was a landscape
in water colours, of which I had made a present to the superintendent, in
acknowledgment of her obliging mediation with the committee on my behalf,
and which she had framed and glazed.

"Well, that is beautiful, Miss Jane!  It is as fine a picture as any Miss
Reed's drawing-master could paint, let alone the young ladies themselves,
who could not come near it: and have you learnt French?"

"Yes, Bessie, I can both read it and speak it."

"And you can work on muslin and canvas?"

"I can."

"Oh, you are quite a lady, Miss Jane!  I knew you would be: you will get
on whether your relations notice you or not.  There was something I
wanted to ask you.  Have you ever heard anything from your father's
kinsfolk, the Eyres?"

"Never in my life."

"Well, you know Missis always said they were poor and quite despicable:
and they may be poor; but I believe they are as much gentry as the Reeds
are; for one day, nearly seven years ago, a Mr. Eyre came to Gateshead
and wanted to see you; Missis said you were at school fifty miles off; he
seemed so much disappointed, for he could not stay: he was going on a
voyage to a foreign country, and the ship was to sail from London in a
day or two.  He looked quite a gentleman, and I believe he was your
father's brother."

"What foreign country was he going to, Bessie?"

"An island thousands of miles off, where they make wine--the butler did
tell me--"

"Madeira?" I suggested.

"Yes, that is it--that is the very word."

"So he went?"

"Yes; he did not stay many minutes in the house: Missis was very high
with him; she called him afterwards a 'sneaking tradesman.'  My Robert
believes he was a wine-merchant."

"Very likely," I returned; "or perhaps clerk or agent to a
wine-merchant."

Bessie and I conversed about old times an hour longer, and then she was
obliged to leave me: I saw her again for a few minutes the next morning
at Lowton, while I was waiting for the coach.  We parted finally at the
door of the Brocklehurst Arms there: each went her separate way; she set
off for the brow of Lowood Fell to meet the conveyance which was to take
her back to Gateshead, I mounted the vehicle which was to bear me to new
duties and a new life in the unknown environs of Millcote.




\chapter{}


\lettrine{A} new chapter in a novel is something like a new scene in a play; and
when I draw up the curtain this time, reader, you must fancy you see a
room in the George Inn at Millcote, with such large figured papering on
the walls as inn rooms have; such a carpet, such furniture, such
ornaments on the mantelpiece, such prints, including a portrait of George
the Third, and another of the Prince of Wales, and a representation of
the death of Wolfe.  All this is visible to you by the light of an oil
lamp hanging from the ceiling, and by that of an excellent fire, near
which I sit in my cloak and bonnet; my muff and umbrella lie on the
table, and I am warming away the numbness and chill contracted by sixteen
hours' exposure to the rawness of an October day: I left Lowton at four
o'clock a.m., and the Millcote town clock is now just striking eight.

Reader, though I look comfortably accommodated, I am not very tranquil in
my mind.  I thought when the coach stopped here there would be some one
to meet me; I looked anxiously round as I descended the wooden steps the
"boots" placed for my convenience, expecting to hear my name pronounced,
and to see some description of carriage waiting to convey me to
Thornfield.  Nothing of the sort was visible; and when I asked a waiter
if any one had been to inquire after a Miss Eyre, I was answered in the
negative: so I had no resource but to request to be shown into a private
room: and here I am waiting, while all sorts of doubts and fears are
troubling my thoughts.

It is a very strange sensation to inexperienced youth to feel itself
quite alone in the world, cut adrift from every connection, uncertain
whether the port to which it is bound can be reached, and prevented by
many impediments from returning to that it has quitted.  The charm of
adventure sweetens that sensation, the glow of pride warms it; but then
the throb of fear disturbs it; and fear with me became predominant when
half-an-hour elapsed and still I was alone.  I bethought myself to ring
the bell.

"Is there a place in this neighbourhood called Thornfield?" I asked of
the waiter who answered the summons.

"Thornfield?  I don't know, ma'am; I'll inquire at the bar."  He
vanished, but reappeared instantly--

"Is your name Eyre, Miss?"

"Yes."

"Person here waiting for you."

I jumped up, took my muff and umbrella, and hastened into the
inn-passage: a man was standing by the open door, and in the lamp-lit
street I dimly saw a one-horse conveyance.

"This will be your luggage, I suppose?" said the man rather abruptly when
he saw me, pointing to my trunk in the passage.

"Yes."  He hoisted it on to the vehicle, which was a sort of car, and
then I got in; before he shut me up, I asked him how far it was to
Thornfield.

"A matter of six miles."

"How long shall we be before we get there?"

"Happen an hour and a half."

He fastened the car door, climbed to his own seat outside, and we set
off.  Our progress was leisurely, and gave me ample time to reflect; I
was content to be at length so near the end of my journey; and as I
leaned back in the comfortable though not elegant conveyance, I meditated
much at my ease.

"I suppose," thought I, "judging from the plainness of the servant and
carriage, Mrs. Fairfax is not a very dashing person: so much the better;
I never lived amongst fine people but once, and I was very miserable with
them.  I wonder if she lives alone except this little girl; if so, and if
she is in any degree amiable, I shall surely be able to get on with her;
I will do my best; it is a pity that doing one's best does not always
answer.  At Lowood, indeed, I took that resolution, kept it, and
succeeded in pleasing; but with Mrs. Reed, I remember my best was always
spurned with scorn.  I pray God Mrs. Fairfax may not turn out a second
Mrs. Reed; but if she does, I am not bound to stay with her! let the
worst come to the worst, I can advertise again.  How far are we on our
road now, I wonder?"

I let down the window and looked out; Millcote was behind us; judging by
the number of its lights, it seemed a place of considerable magnitude,
much larger than Lowton.  We were now, as far as I could see, on a sort
of common; but there were houses scattered all over the district; I felt
we were in a different region to Lowood, more populous, less picturesque;
more stirring, less romantic.

The roads were heavy, the night misty; my conductor let his horse walk
all the way, and the hour and a half extended, I verily believe, to two
hours; at last he turned in his seat and said--

"You're noan so far fro' Thornfield now."

Again I looked out: we were passing a church; I saw its low broad tower
against the sky, and its bell was tolling a quarter; I saw a narrow
galaxy of lights too, on a hillside, marking a village or hamlet.  About
ten minutes after, the driver got down and opened a pair of gates: we
passed through, and they clashed to behind us.  We now slowly ascended a
drive, and came upon the long front of a house: candlelight gleamed from
one curtained bow-window; all the rest were dark.  The car stopped at the
front door; it was opened by a maid-servant; I alighted and went in.

"Will you walk this way, ma'am?" said the girl; and I followed her across
a square hall with high doors all round: she ushered me into a room whose
double illumination of fire and candle at first dazzled me, contrasting
as it did with the darkness to which my eyes had been for two hours
inured; when I could see, however, a cosy and agreeable picture presented
itself to my view.

A snug small room; a round table by a cheerful fire; an arm-chair high-
backed and old-fashioned, wherein sat the neatest imaginable little
elderly lady, in widow's cap, black silk gown, and snowy muslin apron;
exactly like what I had fancied Mrs. Fairfax, only less stately and
milder looking.  She was occupied in knitting; a large cat sat demurely
at her feet; nothing in short was wanting to complete the beau-ideal of
domestic comfort.  A more reassuring introduction for a new governess
could scarcely be conceived; there was no grandeur to overwhelm, no
stateliness to embarrass; and then, as I entered, the old lady got up and
promptly and kindly came forward to meet me.

"How do you do, my dear?  I am afraid you have had a tedious ride; John
drives so slowly; you must be cold, come to the fire."

"Mrs. Fairfax, I suppose?" said I.

"Yes, you are right: do sit down."

She conducted me to her own chair, and then began to remove my shawl and
untie my bonnet-strings; I begged she would not give herself so much
trouble.

"Oh, it is no trouble; I dare say your own hands are almost numbed with
cold.  Leah, make a little hot negus and cut a sandwich or two: here are
the keys of the storeroom."

And she produced from her pocket a most housewifely bunch of keys, and
delivered them to the servant.

"Now, then, draw nearer to the fire," she continued.  "You've brought
your luggage with you, haven't you, my dear?"

"Yes, ma'am."

"I'll see it carried into your room," she said, and bustled out.

"She treats me like a visitor," thought I.  "I little expected such a
reception; I anticipated only coldness and stiffness: this is not like
what I have heard of the treatment of governesses; but I must not exult
too soon."

She returned; with her own hands cleared her knitting apparatus and a
book or two from the table, to make room for the tray which Leah now
brought, and then herself handed me the refreshments.  I felt rather
confused at being the object of more attention than I had ever before
received, and, that too, shown by my employer and superior; but as she
did not herself seem to consider she was doing anything out of her place,
I thought it better to take her civilities quietly.

"Shall I have the pleasure of seeing Miss Fairfax to-night?" I asked,
when I had partaken of what she offered me.

"What did you say, my dear?  I am a little deaf," returned the good lady,
approaching her ear to my mouth.

I repeated the question more distinctly.

"Miss Fairfax?  Oh, you mean Miss Varens!  Varens is the name of your
future pupil."

"Indeed!  Then she is not your daughter?"

"No,--I have no family."

I should have followed up my first inquiry, by asking in what way Miss
Varens was connected with her; but I recollected it was not polite to ask
too many questions: besides, I was sure to hear in time.

"I am so glad," she continued, as she sat down opposite to me, and took
the cat on her knee; "I am so glad you are come; it will be quite
pleasant living here now with a companion.  To be sure it is pleasant at
any time; for Thornfield is a fine old hall, rather neglected of late
years perhaps, but still it is a respectable place; yet you know in
winter-time one feels dreary quite alone in the best quarters.  I say
alone--Leah is a nice girl to be sure, and John and his wife are very
decent people; but then you see they are only servants, and one can't
converse with them on terms of equality: one must keep them at due
distance, for fear of losing one's authority.  I'm sure last winter (it
was a very severe one, if you recollect, and when it did not snow, it
rained and blew), not a creature but the butcher and postman came to the
house, from November till February; and I really got quite melancholy
with sitting night after night alone; I had Leah in to read to me
sometimes; but I don't think the poor girl liked the task much: she felt
it confining.  In spring and summer one got on better: sunshine and long
days make such a difference; and then, just at the commencement of this
autumn, little Adela Varens came and her nurse: a child makes a house
alive all at once; and now you are here I shall be quite gay."

My heart really warmed to the worthy lady as I heard her talk; and I drew
my chair a little nearer to her, and expressed my sincere wish that she
might find my company as agreeable as she anticipated.

"But I'll not keep you sitting up late to-night," said she; "it is on the
stroke of twelve now, and you have been travelling all day: you must feel
tired.  If you have got your feet well warmed, I'll show you your
bedroom.  I've had the room next to mine prepared for you; it is only a
small apartment, but I thought you would like it better than one of the
large front chambers: to be sure they have finer furniture, but they are
so dreary and solitary, I never sleep in them myself."

I thanked her for her considerate choice, and as I really felt fatigued
with my long journey, expressed my readiness to retire.  She took her
candle, and I followed her from the room.  First she went to see if the
hall-door was fastened; having taken the key from the lock, she led the
way upstairs.  The steps and banisters were of oak; the staircase window
was high and latticed; both it and the long gallery into which the
bedroom doors opened looked as if they belonged to a church rather than a
house.  A very chill and vault-like air pervaded the stairs and gallery,
suggesting cheerless ideas of space and solitude; and I was glad, when
finally ushered into my chamber, to find it of small dimensions, and
furnished in ordinary, modern style.

When Mrs. Fairfax had bidden me a kind good-night, and I had fastened my
door, gazed leisurely round, and in some measure effaced the eerie
impression made by that wide hall, that dark and spacious staircase, and
that long, cold gallery, by the livelier aspect of my little room, I
remembered that, after a day of bodily fatigue and mental anxiety, I was
now at last in safe haven.  The impulse of gratitude swelled my heart,
and I knelt down at the bedside, and offered up thanks where thanks were
due; not forgetting, ere I rose, to implore aid on my further path, and
the power of meriting the kindness which seemed so frankly offered me
before it was earned.  My couch had no thorns in it that night; my
solitary room no fears.  At once weary and content, I slept soon and
soundly: when I awoke it was broad day.

The chamber looked such a bright little place to me as the sun shone in
between the gay blue chintz window curtains, showing papered walls and a
carpeted floor, so unlike the bare planks and stained plaster of Lowood,
that my spirits rose at the view.  Externals have a great effect on the
young: I thought that a fairer era of life was beginning for me, one that
was to have its flowers and pleasures, as well as its thorns and toils.
My faculties, roused by the change of scene, the new field offered to
hope, seemed all astir.  I cannot precisely define what they expected,
but it was something pleasant: not perhaps that day or that month, but at
an indefinite future period.

I rose; I dressed myself with care: obliged to be plain--for I had no
article of attire that was not made with extreme simplicity--I was still
by nature solicitous to be neat.  It was not my habit to be disregardful
of appearance or careless of the impression I made: on the contrary, I
ever wished to look as well as I could, and to please as much as my want
of beauty would permit.  I sometimes regretted that I was not handsomer;
I sometimes wished to have rosy cheeks, a straight nose, and small cherry
mouth; I desired to be tall, stately, and finely developed in figure; I
felt it a misfortune that I was so little, so pale, and had features so
irregular and so marked.  And why had I these aspirations and these
regrets?  It would be difficult to say: I could not then distinctly say
it to myself; yet I had a reason, and a logical, natural reason too.
However, when I had brushed my hair very smooth, and put on my black
frock--which, Quakerlike as it was, at least had the merit of fitting to
a nicety--and adjusted my clean white tucker, I thought I should do
respectably enough to appear before Mrs. Fairfax, and that my new pupil
would not at least recoil from me with antipathy.  Having opened my
chamber window, and seen that I left all things straight and neat on the
toilet table, I ventured forth.

Traversing the long and matted gallery, I descended the slippery steps of
oak; then I gained the hall: I halted there a minute; I looked at some
pictures on the walls (one, I remember, represented a grim man in a
cuirass, and one a lady with powdered hair and a pearl necklace), at a
bronze lamp pendent from the ceiling, at a great clock whose case was of
oak curiously carved, and ebon black with time and rubbing.  Everything
appeared very stately and imposing to me; but then I was so little
accustomed to grandeur.  The hall-door, which was half of glass, stood
open; I stepped over the threshold.  It was a fine autumn morning; the
early sun shone serenely on embrowned groves and still green fields;
advancing on to the lawn, I looked up and surveyed the front of the
mansion.  It was three storeys high, of proportions not vast, though
considerable: a gentleman's manor-house, not a nobleman's seat:
battlements round the top gave it a picturesque look.  Its grey front
stood out well from the background of a rookery, whose cawing tenants
were now on the wing: they flew over the lawn and grounds to alight in a
great meadow, from which these were separated by a sunk fence, and where
an array of mighty old thorn trees, strong, knotty, and broad as oaks, at
once explained the etymology of the mansion's designation.  Farther off
were hills: not so lofty as those round Lowood, nor so craggy, nor so
like barriers of separation from the living world; but yet quiet and
lonely hills enough, and seeming to embrace Thornfield with a seclusion I
had not expected to find existent so near the stirring locality of
Millcote.  A little hamlet, whose roofs were blent with trees, straggled
up the side of one of these hills; the church of the district stood
nearer Thornfield: its old tower-top looked over a knoll between the
house and gates.

I was yet enjoying the calm prospect and pleasant fresh air, yet
listening with delight to the cawing of the rooks, yet surveying the
wide, hoary front of the hall, and thinking what a great place it was for
one lonely little dame like Mrs. Fairfax to inhabit, when that lady
appeared at the door.

"What! out already?" said she.  "I see you are an early riser."  I went
up to her, and was received with an affable kiss and shake of the hand.

"How do you like Thornfield?" she asked.  I told her I liked it very
much.

"Yes," she said, "it is a pretty place; but I fear it will be getting out
of order, unless Mr. Rochester should take it into his head to come and
reside here permanently; or, at least, visit it rather oftener: great
houses and fine grounds require the presence of the proprietor."

"Mr. Rochester!" I exclaimed.  "Who is he?"

"The owner of Thornfield," she responded quietly.  "Did you not know he
was called Rochester?"

Of course I did not--I had never heard of him before; but the old lady
seemed to regard his existence as a universally understood fact, with
which everybody must be acquainted by instinct.

"I thought," I continued, "Thornfield belonged to you."

"To me?  Bless you, child; what an idea!  To me!  I am only the
housekeeper--the manager.  To be sure I am distantly related to the
Rochesters by the mother's side, or at least my husband was; he was a
clergyman, incumbent of Hay--that little village yonder on the hill--and
that church near the gates was his.  The present Mr. Rochester's mother
was a Fairfax, and second cousin to my husband: but I never presume on
the connection--in fact, it is nothing to me; I consider myself quite in
the light of an ordinary housekeeper: my employer is always civil, and I
expect nothing more."

"And the little girl--my pupil!"

"She is Mr. Rochester's ward; he commissioned me to find a governess for
her.  He intended to have her brought up in ---shire, I believe.  Here
she comes, with her 'bonne,' as she calls her nurse."  The enigma then
was explained: this affable and kind little widow was no great dame; but
a dependant like myself.  I did not like her the worse for that; on the
contrary, I felt better pleased than ever.  The equality between her and
me was real; not the mere result of condescension on her part: so much
the better--my position was all the freer.

As I was meditating on this discovery, a little girl, followed by her
attendant, came running up the lawn.  I looked at my pupil, who did not
at first appear to notice me: she was quite a child, perhaps seven or
eight years old, slightly built, with a pale, small-featured face, and a
redundancy of hair falling in curls to her waist.

"Good morning, Miss Adela," said Mrs. Fairfax.  "Come and speak to the
lady who is to teach you, and to make you a clever woman some day."  She
approached.

"C'est la ma gouverante!" said she, pointing to me, and addressing her
nurse; who answered--

"Mais oui, certainement."

"Are they foreigners?" I inquired, amazed at hearing the French language.

"The nurse is a foreigner, and Adela was born on the Continent; and, I
believe, never left it till within six months ago.  When she first came
here she could speak no English; now she can make shift to talk it a
little: I don't understand her, she mixes it so with French; but you will
make out her meaning very well, I dare say."

Fortunately I had had the advantage of being taught French by a French
lady; and as I had always made a point of conversing with Madame Pierrot
as often as I could, and had besides, during the last seven years, learnt
a portion of French by heart daily--applying myself to take pains with my
accent, and imitating as closely as possible the pronunciation of my
teacher, I had acquired a certain degree of readiness and correctness in
the language, and was not likely to be much at a loss with Mademoiselle
Adela.  She came and shook hand with me when she heard that I was her
governess; and as I led her in to breakfast, I addressed some phrases to
her in her own tongue: she replied briefly at first, but after we were
seated at the table, and she had examined me some ten minutes with her
large hazel eyes, she suddenly commenced chattering fluently.

"Ah!" cried she, in French, "you speak my language as well as Mr.
Rochester does: I can talk to you as I can to him, and so can Sophie.  She
will be glad: nobody here understands her: Madame Fairfax is all English.
Sophie is my nurse; she came with me over the sea in a great ship with a
chimney that smoked--how it did smoke!--and I was sick, and so was
Sophie, and so was Mr. Rochester.  Mr. Rochester lay down on a sofa in a
pretty room called the salon, and Sophie and I had little beds in another
place.  I nearly fell out of mine; it was like a shelf.  And
Mademoiselle--what is your name?"

"Eyre--Jane Eyre."

"Aire?  Bah!  I cannot say it.  Well, our ship stopped in the morning,
before it was quite daylight, at a great city--a huge city, with very
dark houses and all smoky; not at all like the pretty clean town I came
from; and Mr. Rochester carried me in his arms over a plank to the land,
and Sophie came after, and we all got into a coach, which took us to a
beautiful large house, larger than this and finer, called an hotel.  We
stayed there nearly a week: I and Sophie used to walk every day in a
great green place full of trees, called the Park; and there were many
children there besides me, and a pond with beautiful birds in it, that I
fed with crumbs."

"Can you understand her when she runs on so fast?" asked Mrs. Fairfax.

I understood her very well, for I had been accustomed to the fluent
tongue of Madame Pierrot.

"I wish," continued the good lady, "you would ask her a question or two
about her parents: I wonder if she remembers them?"

"Adele," I inquired, "with whom did you live when you were in that pretty
clean town you spoke of?"

"I lived long ago with mama; but she is gone to the Holy Virgin.  Mama
used to teach me to dance and sing, and to say verses.  A great many
gentlemen and ladies came to see mama, and I used to dance before them,
or to sit on their knees and sing to them: I liked it.  Shall I let you
hear me sing now?"

She had finished her breakfast, so I permitted her to give a specimen of
her accomplishments.  Descending from her chair, she came and placed
herself on my knee; then, folding her little hands demurely before her,
shaking back her curls and lifting her eyes to the ceiling, she commenced
singing a song from some opera.  It was the strain of a forsaken lady,
who, after bewailing the perfidy of her lover, calls pride to her aid;
desires her attendant to deck her in her brightest jewels and richest
robes, and resolves to meet the false one that night at a ball, and prove
to him, by the gaiety of her demeanour, how little his desertion has
affected her.

The subject seemed strangely chosen for an infant singer; but I suppose
the point of the exhibition lay in hearing the notes of love and jealousy
warbled with the lisp of childhood; and in very bad taste that point was:
at least I thought so.

Adele sang the canzonette tunefully enough, and with the \emph{naivete} of her
age.  This achieved, she jumped from my knee and said, "Now,
Mademoiselle, I will repeat you some poetry."

Assuming an attitude, she began, "La Ligue des Rats: fable de La
Fontaine."  She then declaimed the little piece with an attention to
punctuation and emphasis, a flexibility of voice and an appropriateness
of gesture, very unusual indeed at her age, and which proved she had been
carefully trained.

"Was it your mama who taught you that piece?" I asked.

"Yes, and she just used to say it in this way: 'Qu' avez vous donc? lui
dit un de ces rats; parlez!'  She made me lift my hand--so--to remind me
to raise my voice at the question.  Now shall I dance for you?"

"No, that will do: but after your mama went to the Holy Virgin, as you
say, with whom did you live then?"

"With Madame Frederic and her husband: she took care of me, but she is
nothing related to me.  I think she is poor, for she had not so fine a
house as mama.  I was not long there.  Mr. Rochester asked me if I would
like to go and live with him in England, and I said yes; for I knew Mr.
Rochester before I knew Madame Frederic, and he was always kind to me and
gave me pretty dresses and toys: but you see he has not kept his word,
for he has brought me to England, and now he is gone back again himself,
and I never see him."

After breakfast, Adele and I withdrew to the library, which room, it
appears, Mr. Rochester had directed should be used as the schoolroom.
Most of the books were locked up behind glass doors; but there was one
bookcase left open containing everything that could be needed in the way
of elementary works, and several volumes of light literature, poetry,
biography, travels, a few romances, \&c.  I suppose he had considered that
these were all the governess would require for her private perusal; and,
indeed, they contented me amply for the present; compared with the scanty
pickings I had now and then been able to glean at Lowood, they seemed to
offer an abundant harvest of entertainment and information.  In this
room, too, there was a cabinet piano, quite new and of superior tone;
also an easel for painting and a pair of globes.

I found my pupil sufficiently docile, though disinclined to apply: she
had not been used to regular occupation of any kind.  I felt it would be
injudicious to confine her too much at first; so, when I had talked to
her a great deal, and got her to learn a little, and when the morning had
advanced to noon, I allowed her to return to her nurse.  I then proposed
to occupy myself till dinner-time in drawing some little sketches for her
use.

As I was going upstairs to fetch my portfolio and pencils, Mrs. Fairfax
called to me: "Your morning school-hours are over now, I suppose," said
she.  She was in a room the folding-doors of which stood open: I went in
when she addressed me.  It was a large, stately apartment, with purple
chairs and curtains, a Turkey carpet, walnut-panelled walls, one vast
window rich in slanted glass, and a lofty ceiling, nobly moulded.  Mrs.
Fairfax was dusting some vases of fine purple spar, which stood on a
sideboard.

"What a beautiful room!" I exclaimed, as I looked round; for I had never
before seen any half so imposing.

"Yes; this is the dining-room.  I have just opened the window, to let in
a little air and sunshine; for everything gets so damp in apartments that
are seldom inhabited; the drawing-room yonder feels like a vault."

She pointed to a wide arch corresponding to the window, and hung like it
with a Tyrian-dyed curtain, now looped up.  Mounting to it by two broad
steps, and looking through, I thought I caught a glimpse of a fairy
place, so bright to my novice-eyes appeared the view beyond.  Yet it was
merely a very pretty drawing-room, and within it a boudoir, both spread
with white carpets, on which seemed laid brilliant garlands of flowers;
both ceiled with snowy mouldings of white grapes and vine-leaves, beneath
which glowed in rich contrast crimson couches and ottomans; while the
ornaments on the pale Parian mantelpiece were of sparkling Bohemian
glass, ruby red; and between the windows large mirrors repeated the
general blending of snow and fire.

"In what order you keep these rooms, Mrs. Fairfax!" said I.  "No dust, no
canvas coverings: except that the air feels chilly, one would think they
were inhabited daily."

"Why, Miss Eyre, though Mr. Rochester's visits here are rare, they are
always sudden and unexpected; and as I observed that it put him out to
find everything swathed up, and to have a bustle of arrangement on his
arrival, I thought it best to keep the rooms in readiness."

"Is Mr. Rochester an exacting, fastidious sort of man?"

"Not particularly so; but he has a gentleman's tastes and habits, and he
expects to have things managed in conformity to them."

"Do you like him?  Is he generally liked?"

"Oh, yes; the family have always been respected here.  Almost all the
land in this neighbourhood, as far as you can see, has belonged to the
Rochesters time out of mind."

"Well, but, leaving his land out of the question, do you like him?  Is he
liked for himself?"

"I have no cause to do otherwise than like him; and I believe he is
considered a just and liberal landlord by his tenants: but he has never
lived much amongst them."

"But has he no peculiarities?  What, in short, is his character?"

"Oh! his character is unimpeachable, I suppose.  He is rather peculiar,
perhaps: he has travelled a great deal, and seen a great deal of the
world, I should think.  I dare say he is clever, but I never had much
conversation with him."

"In what way is he peculiar?"

"I don't know--it is not easy to describe--nothing striking, but you feel
it when he speaks to you; you cannot be always sure whether he is in jest
or earnest, whether he is pleased or the contrary; you don't thoroughly
understand him, in short--at least, I don't: but it is of no consequence,
he is a very good master."

This was all the account I got from Mrs. Fairfax of her employer and
mine.  There are people who seem to have no notion of sketching a
character, or observing and describing salient points, either in persons
or things: the good lady evidently belonged to this class; my queries
puzzled, but did not draw her out.  Mr. Rochester was Mr. Rochester in
her eyes; a gentleman, a landed proprietor--nothing more: she inquired
and searched no further, and evidently wondered at my wish to gain a more
definite notion of his identity.

When we left the dining-room, she proposed to show me over the rest of
the house; and I followed her upstairs and downstairs, admiring as I
went; for all was well arranged and handsome.  The large front chambers I
thought especially grand: and some of the third-storey rooms, though dark
and low, were interesting from their air of antiquity.  The furniture
once appropriated to the lower apartments had from time to time been
removed here, as fashions changed: and the imperfect light entering by
their narrow casement showed bedsteads of a hundred years old; chests in
oak or walnut, looking, with their strange carvings of palm branches and
cherubs' heads, like types of the Hebrew ark; rows of venerable chairs,
high-backed and narrow; stools still more antiquated, on whose cushioned
tops were yet apparent traces of half-effaced embroideries, wrought by
fingers that for two generations had been coffin-dust.  All these relics
gave to the third storey of Thornfield Hall the aspect of a home of the
past: a shrine of memory.  I liked the hush, the gloom, the quaintness of
these retreats in the day; but I by no means coveted a night's repose on
one of those wide and heavy beds: shut in, some of them, with doors of
oak; shaded, others, with wrought old English hangings crusted with thick
work, portraying effigies of strange flowers, and stranger birds, and
strangest human beings,--all which would have looked strange, indeed, by
the pallid gleam of moonlight.

"Do the servants sleep in these rooms?" I asked.

"No; they occupy a range of smaller apartments to the back; no one ever
sleeps here: one would almost say that, if there were a ghost at
Thornfield Hall, this would be its haunt."

"So I think: you have no ghost, then?"

"None that I ever heard of," returned Mrs. Fairfax, smiling.

"Nor any traditions of one? no legends or ghost stories?"

"I believe not.  And yet it is said the Rochesters have been rather a
violent than a quiet race in their time: perhaps, though, that is the
reason they rest tranquilly in their graves now."

"Yes--'after life's fitful fever they sleep well,'" I muttered.  "Where
are you going now, Mrs. Fairfax?" for she was moving away.

"On to the leads; will you come and see the view from thence?"  I
followed still, up a very narrow staircase to the attics, and thence by a
ladder and through a trap-door to the roof of the hall.  I was now on a
level with the crow colony, and could see into their nests.  Leaning over
the battlements and looking far down, I surveyed the grounds laid out
like a map: the bright and velvet lawn closely girdling the grey base of
the mansion; the field, wide as a park, dotted with its ancient timber;
the wood, dun and sere, divided by a path visibly overgrown, greener with
moss than the trees were with foliage; the church at the gates, the road,
the tranquil hills, all reposing in the autumn day's sun; the horizon
bounded by a propitious sky, azure, marbled with pearly white.  No
feature in the scene was extraordinary, but all was pleasing.  When I
turned from it and repassed the trap-door, I could scarcely see my way
down the ladder; the attic seemed black as a vault compared with that
arch of blue air to which I had been looking up, and to that sunlit scene
of grove, pasture, and green hill, of which the hall was the centre, and
over which I had been gazing with delight.

Mrs. Fairfax stayed behind a moment to fasten the trap-door; I, by drift
of groping, found the outlet from the attic, and proceeded to descend the
narrow garret staircase.  I lingered in the long passage to which this
led, separating the front and back rooms of the third storey: narrow,
low, and dim, with only one little window at the far end, and looking,
with its two rows of small black doors all shut, like a corridor in some
Bluebeard's castle.

While I paced softly on, the last sound I expected to hear in so still a
region, a laugh, struck my ear.  It was a curious laugh; distinct,
formal, mirthless.  I stopped: the sound ceased, only for an instant; it
began again, louder: for at first, though distinct, it was very low.  It
passed off in a clamorous peal that seemed to wake an echo in every
lonely chamber; though it originated but in one, and I could have pointed
out the door whence the accents issued.

"Mrs. Fairfax!" I called out: for I now heard her descending the great
stairs.  "Did you hear that loud laugh?  Who is it?"

"Some of the servants, very likely," she answered: "perhaps Grace Poole."

"Did you hear it?" I again inquired.

"Yes, plainly: I often hear her: she sews in one of these rooms.
Sometimes Leah is with her; they are frequently noisy together."

The laugh was repeated in its low, syllabic tone, and terminated in an
odd murmur.

"Grace!" exclaimed Mrs. Fairfax.

I really did not expect any Grace to answer; for the laugh was as tragic,
as preternatural a laugh as any I ever heard; and, but that it was high
noon, and that no circumstance of ghostliness accompanied the curious
cachinnation; but that neither scene nor season favoured fear, I should
have been superstitiously afraid.  However, the event showed me I was a
fool for entertaining a sense even of surprise.

The door nearest me opened, and a servant came out,--a woman of between
thirty and forty; a set, square-made figure, red-haired, and with a hard,
plain face: any apparition less romantic or less ghostly could scarcely
be conceived.

"Too much noise, Grace," said Mrs. Fairfax.  "Remember directions!"  Grace
curtseyed silently and went in.

"She is a person we have to sew and assist Leah in her housemaid's work,"
continued the widow; "not altogether unobjectionable in some points, but
she does well enough.  By-the-bye, how have you got on with your new
pupil this morning?"

The conversation, thus turned on Adele, continued till we reached the
light and cheerful region below.  Adele came running to meet us in the
hall, exclaiming--

"Mesdames, vous etes servies!" adding, "J'ai bien faim, moi!"

We found dinner ready, and waiting for us in Mrs. Fairfax's room.




\chapter{}


\lettrine{T}{he} promise of a smooth career, which my first calm introduction to
Thornfield Hall seemed to pledge, was not belied on a longer acquaintance
with the place and its inmates.  Mrs. Fairfax turned out to be what she
appeared, a placid-tempered, kind-natured woman, of competent education
and average intelligence.  My pupil was a lively child, who had been
spoilt and indulged, and therefore was sometimes wayward; but as she was
committed entirely to my care, and no injudicious interference from any
quarter ever thwarted my plans for her improvement, she soon forgot her
little freaks, and became obedient and teachable.  She had no great
talents, no marked traits of character, no peculiar development of
feeling or taste which raised her one inch above the ordinary level of
childhood; but neither had she any deficiency or vice which sunk her
below it.  She made reasonable progress, entertained for me a vivacious,
though perhaps not very profound, affection; and by her simplicity, gay
prattle, and efforts to please, inspired me, in return, with a degree of
attachment sufficient to make us both content in each other's society.

This, \emph{par parenthese}, will be thought cool language by persons who
entertain solemn doctrines about the angelic nature of children, and the
duty of those charged with their education to conceive for them an
idolatrous devotion: but I am not writing to flatter parental egotism, to
echo cant, or prop up humbug; I am merely telling the truth.  I felt a
conscientious solicitude for Adele's welfare and progress, and a quiet
liking for her little self: just as I cherished towards Mrs. Fairfax a
thankfulness for her kindness, and a pleasure in her society
proportionate to the tranquil regard she had for me, and the moderation
of her mind and character.

Anybody may blame me who likes, when I add further, that, now and then,
when I took a walk by myself in the grounds; when I went down to the
gates and looked through them along the road; or when, while Adele played
with her nurse, and Mrs. Fairfax made jellies in the storeroom, I climbed
the three staircases, raised the trap-door of the attic, and having
reached the leads, looked out afar over sequestered field and hill, and
along dim sky-line--that then I longed for a power of vision which might
overpass that limit; which might reach the busy world, towns, regions
full of life I had heard of but never seen--that then I desired more of
practical experience than I possessed; more of intercourse with my kind,
of acquaintance with variety of character, than was here within my reach.
I valued what was good in Mrs. Fairfax, and what was good in Adele; but I
believed in the existence of other and more vivid kinds of goodness, and
what I believed in I wished to behold.

Who blames me?  Many, no doubt; and I shall be called discontented.  I
could not help it: the restlessness was in my nature; it agitated me to
pain sometimes.  Then my sole relief was to walk along the corridor of
the third storey, backwards and forwards, safe in the silence and
solitude of the spot, and allow my mind's eye to dwell on whatever bright
visions rose before it--and, certainly, they were many and glowing; to
let my heart be heaved by the exultant movement, which, while it swelled
it in trouble, expanded it with life; and, best of all, to open my inward
ear to a tale that was never ended--a tale my imagination created, and
narrated continuously; quickened with all of incident, life, fire,
feeling, that I desired and had not in my actual existence.

It is in vain to say human beings ought to be satisfied with
tranquillity: they must have action; and they will make it if they cannot
find it.  Millions are condemned to a stiller doom than mine, and
millions are in silent revolt against their lot.  Nobody knows how many
rebellions besides political rebellions ferment in the masses of life
which people earth.  Women are supposed to be very calm generally: but
women feel just as men feel; they need exercise for their faculties, and
a field for their efforts, as much as their brothers do; they suffer from
too rigid a restraint, too absolute a stagnation, precisely as men would
suffer; and it is narrow-minded in their more privileged fellow-creatures
to say that they ought to confine themselves to making puddings and
knitting stockings, to playing on the piano and embroidering bags.  It is
thoughtless to condemn them, or laugh at them, if they seek to do more or
learn more than custom has pronounced necessary for their sex.

When thus alone, I not unfrequently heard Grace Poole's laugh: the same
peal, the same low, slow ha! ha! which, when first heard, had thrilled
me: I heard, too, her eccentric murmurs; stranger than her laugh.  There
were days when she was quite silent; but there were others when I could
not account for the sounds she made.  Sometimes I saw her: she would come
out of her room with a basin, or a plate, or a tray in her hand, go down
to the kitchen and shortly return, generally (oh, romantic reader,
forgive me for telling the plain truth!) bearing a pot of porter.  Her
appearance always acted as a damper to the curiosity raised by her oral
oddities: hard-featured and staid, she had no point to which interest
could attach.  I made some attempts to draw her into conversation, but
she seemed a person of few words: a monosyllabic reply usually cut short
every effort of that sort.

The other members of the household, viz., John and his wife, Leah the
housemaid, and Sophie the French nurse, were decent people; but in no
respect remarkable; with Sophie I used to talk French, and sometimes I
asked her questions about her native country; but she was not of a
descriptive or narrative turn, and generally gave such vapid and confused
answers as were calculated rather to check than encourage inquiry.

October, November, December passed away.  One afternoon in January, Mrs.
Fairfax had begged a holiday for Adele, because she had a cold; and, as
Adele seconded the request with an ardour that reminded me how precious
occasional holidays had been to me in my own childhood, I accorded it,
deeming that I did well in showing pliability on the point.  It was a
fine, calm day, though very cold; I was tired of sitting still in the
library through a whole long morning: Mrs. Fairfax had just written a
letter which was waiting to be posted, so I put on my bonnet and cloak
and volunteered to carry it to Hay; the distance, two miles, would be a
pleasant winter afternoon walk.  Having seen Adele comfortably seated in
her little chair by Mrs. Fairfax's parlour fireside, and given her her
best wax doll (which I usually kept enveloped in silver paper in a
drawer) to play with, and a story-book for change of amusement; and
having replied to her "Revenez bientot, ma bonne amie, ma chere Mdlle.
Jeannette," with a kiss I set out.

The ground was hard, the air was still, my road was lonely; I walked fast
till I got warm, and then I walked slowly to enjoy and analyse the
species of pleasure brooding for me in the hour and situation.  It was
three o'clock; the church bell tolled as I passed under the belfry: the
charm of the hour lay in its approaching dimness, in the low-gliding and
pale-beaming sun.  I was a mile from Thornfield, in a lane noted for wild
roses in summer, for nuts and blackberries in autumn, and even now
possessing a few coral treasures in hips and haws, but whose best winter
delight lay in its utter solitude and leafless repose.  If a breath of
air stirred, it made no sound here; for there was not a holly, not an
evergreen to rustle, and the stripped hawthorn and hazel bushes were as
still as the white, worn stones which causewayed the middle of the path.
Far and wide, on each side, there were only fields, where no cattle now
browsed; and the little brown birds, which stirred occasionally in the
hedge, looked like single russet leaves that had forgotten to drop.

This lane inclined up-hill all the way to Hay; having reached the middle,
I sat down on a stile which led thence into a field.  Gathering my mantle
about me, and sheltering my hands in my muff, I did not feel the cold,
though it froze keenly; as was attested by a sheet of ice covering the
causeway, where a little brooklet, now congealed, had overflowed after a
rapid thaw some days since.  From my seat I could look down on
Thornfield: the grey and battlemented hall was the principal object in
the vale below me; its woods and dark rookery rose against the west.  I
lingered till the sun went down amongst the trees, and sank crimson and
clear behind them.  I then turned eastward.

On the hill-top above me sat the rising moon; pale yet as a cloud, but
brightening momentarily, she looked over Hay, which, half lost in trees,
sent up a blue smoke from its few chimneys: it was yet a mile distant,
but in the absolute hush I could hear plainly its thin murmurs of life.
My ear, too, felt the flow of currents; in what dales and depths I could
not tell: but there were many hills beyond Hay, and doubtless many becks
threading their passes.  That evening calm betrayed alike the tinkle of
the nearest streams, the sough of the most remote.

A rude noise broke on these fine ripplings and whisperings, at once so
far away and so clear: a positive tramp, tramp, a metallic clatter, which
effaced the soft wave-wanderings; as, in a picture, the solid mass of a
crag, or the rough boles of a great oak, drawn in dark and strong on the
foreground, efface the aerial distance of azure hill, sunny horizon, and
blended clouds where tint melts into tint.

The din was on the causeway: a horse was coming; the windings of the lane
yet hid it, but it approached.  I was just leaving the stile; yet, as the
path was narrow, I sat still to let it go by.  In those days I was young,
and all sorts of fancies bright and dark tenanted my mind: the memories
of nursery stories were there amongst other rubbish; and when they
recurred, maturing youth added to them a vigour and vividness beyond what
childhood could give.  As this horse approached, and as I watched for it
to appear through the dusk, I remembered certain of Bessie's tales,
wherein figured a North-of-England spirit called a "Gytrash," which, in
the form of horse, mule, or large dog, haunted solitary ways, and
sometimes came upon belated travellers, as this horse was now coming upon
me.

It was very near, but not yet in sight; when, in addition to the tramp,
tramp, I heard a rush under the hedge, and close down by the hazel stems
glided a great dog, whose black and white colour made him a distinct
object against the trees.  It was exactly one form of Bessie's Gytrash--a
lion-like creature with long hair and a huge head: it passed me, however,
quietly enough; not staying to look up, with strange pretercanine eyes,
in my face, as I half expected it would.  The horse followed,--a tall
steed, and on its back a rider.  The man, the human being, broke the
spell at once.  Nothing ever rode the Gytrash: it was always alone; and
goblins, to my notions, though they might tenant the dumb carcasses of
beasts, could scarce covet shelter in the commonplace human form.  No
Gytrash was this,--only a traveller taking the short cut to Millcote.  He
passed, and I went on; a few steps, and I turned: a sliding sound and an
exclamation of "What the deuce is to do now?" and a clattering tumble,
arrested my attention.  Man and horse were down; they had slipped on the
sheet of ice which glazed the causeway.  The dog came bounding back, and
seeing his master in a predicament, and hearing the horse groan, barked
till the evening hills echoed the sound, which was deep in proportion to
his magnitude.  He snuffed round the prostrate group, and then he ran up
to me; it was all he could do,--there was no other help at hand to
summon.  I obeyed him, and walked down to the traveller, by this time
struggling himself free of his steed.  His efforts were so vigorous, I
thought he could not be much hurt; but I asked him the question--

"Are you injured, sir?"

I think he was swearing, but am not certain; however, he was pronouncing
some formula which prevented him from replying to me directly.

"Can I do anything?" I asked again.

"You must just stand on one side," he answered as he rose, first to his
knees, and then to his feet.  I did; whereupon began a heaving, stamping,
clattering process, accompanied by a barking and baying which removed me
effectually some yards' distance; but I would not be driven quite away
till I saw the event.  This was finally fortunate; the horse was
re-established, and the dog was silenced with a "Down, Pilot!"  The
traveller now, stooping, felt his foot and leg, as if trying whether they
were sound; apparently something ailed them, for he halted to the stile
whence I had just risen, and sat down.

I was in the mood for being useful, or at least officious, I think, for I
now drew near him again.

"If you are hurt, and want help, sir, I can fetch some one either from
Thornfield Hall or from Hay."

"Thank you: I shall do: I have no broken bones,--only a sprain;" and
again he stood up and tried his foot, but the result extorted an
involuntary "Ugh!"

Something of daylight still lingered, and the moon was waxing bright: I
could see him plainly.  His figure was enveloped in a riding cloak, fur
collared and steel clasped; its details were not apparent, but I traced
the general points of middle height and considerable breadth of chest.  He
had a dark face, with stern features and a heavy brow; his eyes and
gathered eyebrows looked ireful and thwarted just now; he was past youth,
but had not reached middle-age; perhaps he might be thirty-five.  I felt
no fear of him, and but little shyness.  Had he been a handsome, heroic-
looking young gentleman, I should not have dared to stand thus
questioning him against his will, and offering my services unasked.  I
had hardly ever seen a handsome youth; never in my life spoken to one.  I
had a theoretical reverence and homage for beauty, elegance, gallantry,
fascination; but had I met those qualities incarnate in masculine shape,
I should have known instinctively that they neither had nor could have
sympathy with anything in me, and should have shunned them as one would
fire, lightning, or anything else that is bright but antipathetic.

If even this stranger had smiled and been good-humoured to me when I
addressed him; if he had put off my offer of assistance gaily and with
thanks, I should have gone on my way and not felt any vocation to renew
inquiries: but the frown, the roughness of the traveller, set me at my
ease: I retained my station when he waved to me to go, and announced--

"I cannot think of leaving you, sir, at so late an hour, in this solitary
lane, till I see you are fit to mount your horse."

He looked at me when I said this; he had hardly turned his eyes in my
direction before.

"I should think you ought to be at home yourself," said he, "if you have
a home in this neighbourhood: where do you come from?"

"From just below; and I am not at all afraid of being out late when it is
moonlight: I will run over to Hay for you with pleasure, if you wish it:
indeed, I am going there to post a letter."

"You live just below--do you mean at that house with the battlements?"
pointing to Thornfield Hall, on which the moon cast a hoary gleam,
bringing it out distinct and pale from the woods that, by contrast with
the western sky, now seemed one mass of shadow.

"Yes, sir."

"Whose house is it?"

"Mr. Rochester's."

"Do you know Mr. Rochester?"

"No, I have never seen him."

"He is not resident, then?"

"No."

"Can you tell me where he is?"

"I cannot."

"You are not a servant at the hall, of course.  You are--"  He stopped,
ran his eye over my dress, which, as usual, was quite simple: a black
merino cloak, a black beaver bonnet; neither of them half fine enough for
a lady's-maid.  He seemed puzzled to decide what I was; I helped him.

"I am the governess."

"Ah, the governess!" he repeated; "deuce take me, if I had not forgotten!
The governess!" and again my raiment underwent scrutiny.  In two minutes
he rose from the stile: his face expressed pain when he tried to move.

"I cannot commission you to fetch help," he said; "but you may help me a
little yourself, if you will be so kind."

"Yes, sir."

"You have not an umbrella that I can use as a stick?"

"No."

"Try to get hold of my horse's bridle and lead him to me: you are not
afraid?"

I should have been afraid to touch a horse when alone, but when told to
do it, I was disposed to obey.  I put down my muff on the stile, and went
up to the tall steed; I endeavoured to catch the bridle, but it was a
spirited thing, and would not let me come near its head; I made effort on
effort, though in vain: meantime, I was mortally afraid of its trampling
fore-feet.  The traveller waited and watched for some time, and at last
he laughed.

{I was mortally afraid of its trampling forefeet: p107.jpg}

"I see," he said, "the mountain will never be brought to Mahomet, so all
you can do is to aid Mahomet to go to the mountain; I must beg of you to
come here."

I came.  "Excuse me," he continued: "necessity compels me to make you
useful."  He laid a heavy hand on my shoulder, and leaning on me with
some stress, limped to his horse.  Having once caught the bridle, he
mastered it directly and sprang to his saddle; grimacing grimly as he
made the effort, for it wrenched his sprain.

"Now," said he, releasing his under lip from a hard bite, "just hand me
my whip; it lies there under the hedge."

I sought it and found it.

"Thank you; now make haste with the letter to Hay, and return as fast as
you can."

A touch of a spurred heel made his horse first start and rear, and then
bound away; the dog rushed in his traces; all three vanished,

   "Like heath that, in the wilderness,
   The wild wind whirls away."

I took up my muff and walked on.  The incident had occurred and was gone
for me: it \emph{was} an incident of no moment, no romance, no interest in a
sense; yet it marked with change one single hour of a monotonous life.  My
help had been needed and claimed; I had given it: I was pleased to have
done something; trivial, transitory though the deed was, it was yet an
active thing, and I was weary of an existence all passive.  The new face,
too, was like a new picture introduced to the gallery of memory; and it
was dissimilar to all the others hanging there: firstly, because it was
masculine; and, secondly, because it was dark, strong, and stern.  I had
it still before me when I entered Hay, and slipped the letter into the
post-office; I saw it as I walked fast down-hill all the way home.  When
I came to the stile, I stopped a minute, looked round and listened, with
an idea that a horse's hoofs might ring on the causeway again, and that a
rider in a cloak, and a Gytrash-like Newfoundland dog, might be again
apparent: I saw only the hedge and a pollard willow before me, rising up
still and straight to meet the moonbeams; I heard only the faintest waft
of wind roaming fitful among the trees round Thornfield, a mile distant;
and when I glanced down in the direction of the murmur, my eye,
traversing the hall-front, caught a light kindling in a window: it
reminded me that I was late, and I hurried on.

I did not like re-entering Thornfield.  To pass its threshold was to
return to stagnation; to cross the silent hall, to ascend the darksome
staircase, to seek my own lonely little room, and then to meet tranquil
Mrs. Fairfax, and spend the long winter evening with her, and her only,
was to quell wholly the faint excitement wakened by my walk,--to slip
again over my faculties the viewless fetters of an uniform and too still
existence; of an existence whose very privileges of security and ease I
was becoming incapable of appreciating.  What good it would have done me
at that time to have been tossed in the storms of an uncertain struggling
life, and to have been taught by rough and bitter experience to long for
the calm amidst which I now repined!  Yes, just as much good as it would
do a man tired of sitting still in a "too easy chair" to take a long
walk: and just as natural was the wish to stir, under my circumstances,
as it would be under his.

I lingered at the gates; I lingered on the lawn; I paced backwards and
forwards on the pavement; the shutters of the glass door were closed; I
could not see into the interior; and both my eyes and spirit seemed drawn
from the gloomy house--from the grey-hollow filled with rayless cells, as
it appeared to me--to that sky expanded before me,--a blue sea absolved
from taint of cloud; the moon ascending it in solemn march; her orb
seeming to look up as she left the hill-tops, from behind which she had
come, far and farther below her, and aspired to the zenith, midnight dark
in its fathomless depth and measureless distance; and for those trembling
stars that followed her course; they made my heart tremble, my veins glow
when I viewed them.  Little things recall us to earth; the clock struck
in the hall; that sufficed; I turned from moon and stars, opened a side-
door, and went in.

The hall was not dark, nor yet was it lit, only by the high-hung bronze
lamp; a warm glow suffused both it and the lower steps of the oak
staircase.  This ruddy shine issued from the great dining-room, whose two-
leaved door stood open, and showed a genial fire in the grate, glancing
on marble hearth and brass fire-irons, and revealing purple draperies and
polished furniture, in the most pleasant radiance.  It revealed, too, a
group near the mantelpiece: I had scarcely caught it, and scarcely become
aware of a cheerful mingling of voices, amongst which I seemed to
distinguish the tones of Adele, when the door closed.

I hastened to Mrs. Fairfax's room; there was a fire there too, but no
candle, and no Mrs. Fairfax.  Instead, all alone, sitting upright on the
rug, and gazing with gravity at the blaze, I beheld a great black and
white long-haired dog, just like the Gytrash of the lane.  It was so like
it that I went forward and said--"Pilot" and the thing got up and came to
me and snuffed me.  I caressed him, and he wagged his great tail; but he
looked an eerie creature to be alone with, and I could not tell whence he
had come.  I rang the bell, for I wanted a candle; and I wanted, too, to
get an account of this visitant.  Leah entered.

"What dog is this?"

"He came with master."

"With whom?"

"With master--Mr. Rochester--he is just arrived."

"Indeed! and is Mrs. Fairfax with him?"

"Yes, and Miss Adele; they are in the dining-room, and John is gone for a
surgeon; for master has had an accident; his horse fell and his ankle is
sprained."

"Did the horse fall in Hay Lane?"

"Yes, coming down-hill; it slipped on some ice."

"Ah!  Bring me a candle will you Leah?"

Leah brought it; she entered, followed by Mrs. Fairfax, who repeated the
news; adding that Mr. Carter the surgeon was come, and was now with Mr.
Rochester: then she hurried out to give orders about tea, and I went
upstairs to take off my things.




\chapter{}


\lettrine{M}{r.} Rochester, it seems, by the surgeon's orders, went to bed early that
night; nor did he rise soon next morning.  When he did come down, it was
to attend to business: his agent and some of his tenants were arrived,
and waiting to speak with him.

Adele and I had now to vacate the library: it would be in daily
requisition as a reception-room for callers.  A fire was lit in an
apartment upstairs, and there I carried our books, and arranged it for
the future schoolroom.  I discerned in the course of the morning that
Thornfield Hall was a changed place: no longer silent as a church, it
echoed every hour or two to a knock at the door, or a clang of the bell;
steps, too, often traversed the hall, and new voices spoke in different
keys below; a rill from the outer world was flowing through it; it had a
master: for my part, I liked it better.

Adele was not easy to teach that day; she could not apply: she kept
running to the door and looking over the banisters to see if she could
get a glimpse of Mr. Rochester; then she coined pretexts to go
downstairs, in order, as I shrewdly suspected, to visit the library,
where I knew she was not wanted; then, when I got a little angry, and
made her sit still, she continued to talk incessantly of her "ami,
Monsieur Edouard Fairfax \emph{de} Rochester," as she dubbed him (I had not
before heard his prenomens), and to conjecture what presents he had
brought her: for it appears he had intimated the night before, that when
his luggage came from Millcote, there would be found amongst it a little
box in whose contents she had an interest.

"Et cela doit signifier," said she, "qu'il y aura la dedans un cadeau
pour moi, et peut-etre pour vous aussi, mademoiselle.  Monsieur a parle
de vous: il m'a demande le nom de ma gouvernante, et si elle n'etait pas
une petite personne, assez mince et un peu pale.  J'ai dit qu'oui: car
c'est vrai, n'est-ce pas, mademoiselle?"

I and my pupil dined as usual in Mrs. Fairfax's parlour; the afternoon
was wild and snowy, and we passed it in the schoolroom.  At dark I
allowed Adele to put away books and work, and to run downstairs; for,
from the comparative silence below, and from the cessation of appeals to
the door-bell, I conjectured that Mr. Rochester was now at liberty.  Left
alone, I walked to the window; but nothing was to be seen thence:
twilight and snowflakes together thickened the air, and hid the very
shrubs on the lawn.  I let down the curtain and went back to the
fireside.

In the clear embers I was tracing a view, not unlike a picture I
remembered to have seen of the castle of Heidelberg, on the Rhine, when
Mrs. Fairfax came in, breaking up by her entrance the fiery mosaic I had
been piercing together, and scattering too some heavy unwelcome thoughts
that were beginning to throng on my solitude.

"Mr. Rochester would be glad if you and your pupil would take tea with
him in the drawing-room this evening," said she: "he has been so much
engaged all day that he could not ask to see you before."

"When is his tea-time?" I inquired.

"Oh, at six o'clock: he keeps early hours in the country.  You had better
change your frock now; I will go with you and fasten it.  Here is a
candle."

"Is it necessary to change my frock?"

"Yes, you had better: I always dress for the evening when Mr. Rochester
is here."

This additional ceremony seemed somewhat stately; however, I repaired to
my room, and, with Mrs. Fairfax's aid, replaced my black stuff dress by
one of black silk; the best and the only additional one I had, except one
of light grey, which, in my Lowood notions of the toilette, I thought too
fine to be worn, except on first-rate occasions.

"You want a brooch," said Mrs. Fairfax.  I had a single little pearl
ornament which Miss Temple gave me as a parting keepsake: I put it on,
and then we went downstairs.  Unused as I was to strangers, it was rather
a trial to appear thus formally summoned in Mr. Rochester's presence.  I
let Mrs. Fairfax precede me into the dining-room, and kept in her shade
as we crossed that apartment; and, passing the arch, whose curtain was
now dropped, entered the elegant recess beyond.

Two wax candles stood lighted on the table, and two on the mantelpiece;
basking in the light and heat of a superb fire, lay Pilot--Adele knelt
near him.  Half reclined on a couch appeared Mr. Rochester, his foot
supported by the cushion; he was looking at Adele and the dog: the fire
shone full on his face.  I knew my traveller with his broad and jetty
eyebrows; his square forehead, made squarer by the horizontal sweep of
his black hair.  I recognised his decisive nose, more remarkable for
character than beauty; his full nostrils, denoting, I thought, choler;
his grim mouth, chin, and jaw--yes, all three were very grim, and no
mistake.  His shape, now divested of cloak, I perceived harmonised in
squareness with his physiognomy: I suppose it was a good figure in the
athletic sense of the term--broad chested and thin flanked, though
neither tall nor graceful.

Mr. Rochester must have been aware of the entrance of Mrs. Fairfax and
myself; but it appeared he was not in the mood to notice us, for he never
lifted his head as we approached.

"Here is Miss Eyre, sir," said Mrs. Fairfax, in her quiet way.  He bowed,
still not taking his eyes from the group of the dog and child.

"Let Miss Eyre be seated," said he: and there was something in the forced
stiff bow, in the impatient yet formal tone, which seemed further to
express, "What the deuce is it to me whether Miss Eyre be there or not?
At this moment I am not disposed to accost her."

I sat down quite disembarrassed.  A reception of finished politeness
would probably have confused me: I could not have returned or repaid it
by answering grace and elegance on my part; but harsh caprice laid me
under no obligation; on the contrary, a decent quiescence, under the
freak of manner, gave me the advantage.  Besides, the eccentricity of the
proceeding was piquant: I felt interested to see how he would go on.

He went on as a statue would, that is, he neither spoke nor moved.  Mrs.
Fairfax seemed to think it necessary that some one should be amiable, and
she began to talk.  Kindly, as usual--and, as usual, rather trite--she
condoled with him on the pressure of business he had had all day; on the
annoyance it must have been to him with that painful sprain: then she
commended his patience and perseverance in going through with it.

"Madam, I should like some tea," was the sole rejoinder she got.  She
hastened to ring the bell; and when the tray came, she proceeded to
arrange the cups, spoons, \&c., with assiduous celerity.  I and Adele went
to the table; but the master did not leave his couch.

"Will you hand Mr. Rochester's cup?" said Mrs. Fairfax to me; "Adele
might perhaps spill it."

I did as requested.  As he took the cup from my hand, Adele, thinking the
moment propitious for making a request in my favour, cried out--

"N'est-ce pas, monsieur, qu'il y a un cadeau pour Mademoiselle Eyre dans
votre petit coffre?"

"Who talks of cadeaux?" said he gruffly.  "Did you expect a present, Miss
Eyre?  Are you fond of presents?" and he searched my face with eyes that
I saw were dark, irate, and piercing.

"I hardly know, sir; I have little experience of them: they are generally
thought pleasant things."

"Generally thought?  But what do \emph{you} think?"

"I should be obliged to take time, sir, before I could give you an answer
worthy of your acceptance: a present has many faces to it, has it not?
and one should consider all, before pronouncing an opinion as to its
nature."

"Miss Eyre, you are not so unsophisticated as Adele: she demands a
'cadeau,' clamorously, the moment she sees me: you beat about the bush."

"Because I have less confidence in my deserts than Adele has: she can
prefer the claim of old acquaintance, and the right too of custom; for
she says you have always been in the habit of giving her playthings; but
if I had to make out a case I should be puzzled, since I am a stranger,
and have done nothing to entitle me to an acknowledgment."

"Oh, don't fall back on over-modesty!  I have examined Adele, and find
you have taken great pains with her: she is not bright, she has no
talents; yet in a short time she has made much improvement."

"Sir, you have now given me my 'cadeau;' I am obliged to you: it is the
meed teachers most covet--praise of their pupils' progress."

"Humph!" said Mr. Rochester, and he took his tea in silence.

"Come to the fire," said the master, when the tray was taken away, and
Mrs. Fairfax had settled into a corner with her knitting; while Adele was
leading me by the hand round the room, showing me the beautiful books and
ornaments on the consoles and chiffonnieres.  We obeyed, as in duty
bound; Adele wanted to take a seat on my knee, but she was ordered to
amuse herself with Pilot.

"You have been resident in my house three months?"

"Yes, sir."

"And you came from--?"

"From Lowood school, in ---shire."

"Ah! a charitable concern.  How long were you there?"

"Eight years."

"Eight years! you must be tenacious of life.  I thought half the time in
such a place would have done up any constitution!  No wonder you have
rather the look of another world.  I marvelled where you had got that
sort of face.  When you came on me in Hay Lane last night, I thought
unaccountably of fairy tales, and had half a mind to demand whether you
had bewitched my horse: I am not sure yet.  Who are your parents?"

"I have none."

"Nor ever had, I suppose: do you remember them?"

"No."

"I thought not.  And so you were waiting for your people when you sat on
that stile?"

"For whom, sir?"

"For the men in green: it was a proper moonlight evening for them.  Did I
break through one of your rings, that you spread that damned ice on the
causeway?"

I shook my head.  "The men in green all forsook England a hundred years
ago," said I, speaking as seriously as he had done.  "And not even in Hay
Lane, or the fields about it, could you find a trace of them.  I don't
think either summer or harvest, or winter moon, will ever shine on their
revels more."

Mrs. Fairfax had dropped her knitting, and, with raised eyebrows, seemed
wondering what sort of talk this was.

"Well," resumed Mr. Rochester, "if you disown parents, you must have some
sort of kinsfolk: uncles and aunts?"

"No; none that I ever saw."

"And your home?"

"I have none."

"Where do your brothers and sisters live?"

"I have no brothers or sisters."

"Who recommended you to come here?"

"I advertised, and Mrs. Fairfax answered my advertisement."

"Yes," said the good lady, who now knew what ground we were upon, "and I
am daily thankful for the choice Providence led me to make.  Miss Eyre
has been an invaluable companion to me, and a kind and careful teacher to
Adele."

"Don't trouble yourself to give her a character," returned Mr. Rochester:
"eulogiums will not bias me; I shall judge for myself.  She began by
felling my horse."

"Sir?" said Mrs. Fairfax.

"I have to thank her for this sprain."

The widow looked bewildered.

"Miss Eyre, have you ever lived in a town?"

"No, sir."

"Have you seen much society?"

"None but the pupils and teachers of Lowood, and now the inmates of
Thornfield."

"Have you read much?"

"Only such books as came in my way; and they have not been numerous or
very learned."

"You have lived the life of a nun: no doubt you are well drilled in
religious forms;--Brocklehurst, who I understand directs Lowood, is a
parson, is he not?"

"Yes, sir."

"And you girls probably worshipped him, as a convent full of religieuses
would worship their director."

"Oh, no."

"You are very cool!  No!  What! a novice not worship her priest!  That
sounds blasphemous."

"I disliked Mr. Brocklehurst; and I was not alone in the feeling.  He is
a harsh man; at once pompous and meddling; he cut off our hair; and for
economy's sake bought us bad needles and thread, with which we could
hardly sew."

"That was very false economy," remarked Mrs. Fairfax, who now again
caught the drift of the dialogue.

"And was that the head and front of his offending?" demanded Mr.
Rochester.

"He starved us when he had the sole superintendence of the provision
department, before the committee was appointed; and he bored us with long
lectures once a week, and with evening readings from books of his own
inditing, about sudden deaths and judgments, which made us afraid to go
to bed."

"What age were you when you went to Lowood?"

"About ten."

"And you stayed there eight years: you are now, then, eighteen?"

I assented.

"Arithmetic, you see, is useful; without its aid, I should hardly have
been able to guess your age.  It is a point difficult to fix where the
features and countenance are so much at variance as in your case.  And
now what did you learn at Lowood?  Can you play?"

"A little."

"Of course: that is the established answer.  Go into the library--I mean,
if you please.--(Excuse my tone of command; I am used to say, 'Do this,'
and it is done: I cannot alter my customary habits for one new
inmate.)--Go, then, into the library; take a candle with you; leave the
door open; sit down to the piano, and play a tune."

I departed, obeying his directions.

"Enough!" he called out in a few minutes.  "You play \emph{a little}, I see;
like any other English school-girl; perhaps rather better than some, but
not well."

I closed the piano and returned.  Mr. Rochester continued--"Adele showed
me some sketches this morning, which she said were yours.  I don't know
whether they were entirely of your doing; probably a master aided you?"

"No, indeed!" I interjected.

"Ah! that pricks pride.  Well, fetch me your portfolio, if you can vouch
for its contents being original; but don't pass your word unless you are
certain: I can recognise patchwork."

"Then I will say nothing, and you shall judge for yourself, sir."

I brought the portfolio from the library.

"Approach the table," said he; and I wheeled it to his couch.  Adele and
Mrs. Fairfax drew near to see the pictures.

"No crowding," said Mr. Rochester: "take the drawings from my hand as I
finish with them; but don't push your faces up to mine."

He deliberately scrutinised each sketch and painting.  Three he laid
aside; the others, when he had examined them, he swept from him.

"Take them off to the other table, Mrs. Fairfax," said he, "and look at
them with Adele;--you" (glancing at me) "resume your seat, and answer my
questions.  I perceive those pictures were done by one hand: was that
hand yours?"

"Yes."

"And when did you find time to do them?  They have taken much time, and
some thought."

"I did them in the last two vacations I spent at Lowood, when I had no
other occupation."

"Where did you get your copies?"

"Out of my head."

"That head I see now on your shoulders?"

"Yes, sir."

"Has it other furniture of the same kind within?"

"I should think it may have: I should hope--better."

He spread the pictures before him, and again surveyed them alternately.

While he is so occupied, I will tell you, reader, what they are: and
first, I must premise that they are nothing wonderful.  The subjects had,
indeed, risen vividly on my mind.  As I saw them with the spiritual eye,
before I attempted to embody them, they were striking; but my hand would
not second my fancy, and in each case it had wrought out but a pale
portrait of the thing I had conceived.

These pictures were in water-colours.  The first represented clouds low
and livid, rolling over a swollen sea: all the distance was in eclipse;
so, too, was the foreground; or rather, the nearest billows, for there
was no land.  One gleam of light lifted into relief a half-submerged
mast, on which sat a cormorant, dark and large, with wings flecked with
foam; its beak held a gold bracelet set with gems, that I had touched
with as brilliant tints as my palette could yield, and as glittering
distinctness as my pencil could impart.  Sinking below the bird and mast,
a drowned corpse glanced through the green water; a fair arm was the only
limb clearly visible, whence the bracelet had been washed or torn.

The second picture contained for foreground only the dim peak of a hill,
with grass and some leaves slanting as if by a breeze.  Beyond and above
spread an expanse of sky, dark blue as at twilight: rising into the sky
was a woman's shape to the bust, portrayed in tints as dusk and soft as I
could combine.  The dim forehead was crowned with a star; the lineaments
below were seen as through the suffusion of vapour; the eyes shone dark
and wild; the hair streamed shadowy, like a beamless cloud torn by storm
or by electric travail.  On the neck lay a pale reflection like
moonlight; the same faint lustre touched the train of thin clouds from
which rose and bowed this vision of the Evening Star.

The third showed the pinnacle of an iceberg piercing a polar winter sky:
a muster of northern lights reared their dim lances, close serried, along
the horizon.  Throwing these into distance, rose, in the foreground, a
head,--a colossal head, inclined towards the iceberg, and resting against
it.  Two thin hands, joined under the forehead, and supporting it, drew
up before the lower features a sable veil, a brow quite bloodless, white
as bone, and an eye hollow and fixed, blank of meaning but for the
glassiness of despair, alone were visible.  Above the temples, amidst
wreathed turban folds of black drapery, vague in its character and
consistency as cloud, gleamed a ring of white flame, gemmed with sparkles
of a more lurid tinge.  This pale crescent was "the likeness of a kingly
crown;" what it diademed was "the shape which shape had none."

"Were you happy when you painted these pictures?" asked Mr. Rochester
presently.

"I was absorbed, sir: yes, and I was happy.  To paint them, in short, was
to enjoy one of the keenest pleasures I have ever known."

"That is not saying much.  Your pleasures, by your own account, have been
few; but I daresay you did exist in a kind of artist's dreamland while
you blent and arranged these strange tints.  Did you sit at them long
each day?"

"I had nothing else to do, because it was the vacation, and I sat at them
from morning till noon, and from noon till night: the length of the
midsummer days favoured my inclination to apply."

"And you felt self-satisfied with the result of your ardent labours?"

"Far from it.  I was tormented by the contrast between my idea and my
handiwork: in each case I had imagined something which I was quite
powerless to realise."

"Not quite: you have secured the shadow of your thought; but no more,
probably.  You had not enough of the artist's skill and science to give
it full being: yet the drawings are, for a school-girl, peculiar.  As to
the thoughts, they are elfish.  These eyes in the Evening Star you must
have seen in a dream.  How could you make them look so clear, and yet not
at all brilliant? for the planet above quells their rays.  And what
meaning is that in their solemn depth?  And who taught you to paint wind?
There is a high gale in that sky, and on this hill-top.  Where did you
see Latmos?  For that is Latmos.  There! put the drawings away!"

I had scarce tied the strings of the portfolio, when, looking at his
watch, he said abruptly--

"It is nine o'clock: what are you about, Miss Eyre, to let Adele sit up
so long?  Take her to bed."

Adele went to kiss him before quitting the room: he endured the caress,
but scarcely seemed to relish it more than Pilot would have done, nor so
much.

"I wish you all good-night, now," said he, making a movement of the hand
towards the door, in token that he was tired of our company, and wished
to dismiss us.  Mrs. Fairfax folded up her knitting: I took my portfolio:
we curtseyed to him, received a frigid bow in return, and so withdrew.

"You said Mr. Rochester was not strikingly peculiar, Mrs. Fairfax," I
observed, when I rejoined her in her room, after putting Adele to bed.

"Well, is he?"

"I think so: he is very changeful and abrupt."

"True: no doubt he may appear so to a stranger, but I am so accustomed to
his manner, I never think of it; and then, if he has peculiarities of
temper, allowance should be made."

"Why?"

"Partly because it is his nature--and we can none of us help our nature;
and partly because he has painful thoughts, no doubt, to harass him, and
make his spirits unequal."

"What about?"

"Family troubles, for one thing."

"But he has no family."

"Not now, but he has had--or, at least, relatives.  He lost his elder
brother a few years since."

"His \emph{elder} brother?"

"Yes.  The present Mr. Rochester has not been very long in possession of
the property; only about nine years."

"Nine years is a tolerable time.  Was he so very fond of his brother as
to be still inconsolable for his loss?"

"Why, no--perhaps not.  I believe there were some misunderstandings
between them.  Mr. Rowland Rochester was not quite just to Mr. Edward;
and perhaps he prejudiced his father against him.  The old gentleman was
fond of money, and anxious to keep the family estate together.  He did
not like to diminish the property by division, and yet he was anxious
that Mr. Edward should have wealth, too, to keep up the consequence of
the name; and, soon after he was of age, some steps were taken that were
not quite fair, and made a great deal of mischief.  Old Mr. Rochester and
Mr. Rowland combined to bring Mr. Edward into what he considered a
painful position, for the sake of making his fortune: what the precise
nature of that position was I never clearly knew, but his spirit could
not brook what he had to suffer in it.  He is not very forgiving: he
broke with his family, and now for many years he has led an unsettled
kind of life.  I don't think he has ever been resident at Thornfield for
a fortnight together, since the death of his brother without a will left
him master of the estate; and, indeed, no wonder he shuns the old place."

"Why should he shun it?"

"Perhaps he thinks it gloomy."

The answer was evasive.  I should have liked something clearer; but Mrs.
Fairfax either could not, or would not, give me more explicit information
of the origin and nature of Mr. Rochester's trials.  She averred they
were a mystery to herself, and that what she knew was chiefly from
conjecture.  It was evident, indeed, that she wished me to drop the
subject, which I did accordingly.




\chapter{}


\lettrine{F}{or} several subsequent days I saw little of Mr. Rochester.  In the
mornings he seemed much engaged with business, and, in the afternoon,
gentlemen from Millcote or the neighbourhood called, and sometimes stayed
to dine with him.  When his sprain was well enough to admit of horse
exercise, he rode out a good deal; probably to return these visits, as he
generally did not come back till late at night.

During this interval, even Adele was seldom sent for to his presence, and
all my acquaintance with him was confined to an occasional rencontre in
the hall, on the stairs, or in the gallery, when he would sometimes pass
me haughtily and coldly, just acknowledging my presence by a distant nod
or a cool glance, and sometimes bow and smile with gentlemanlike
affability.  His changes of mood did not offend me, because I saw that I
had nothing to do with their alternation; the ebb and flow depended on
causes quite disconnected with me.

One day he had had company to dinner, and had sent for my portfolio; in
order, doubtless, to exhibit its contents: the gentlemen went away early,
to attend a public meeting at Millcote, as Mrs. Fairfax informed me; but
the night being wet and inclement, Mr. Rochester did not accompany them.
Soon after they were gone he rang the bell: a message came that I and
Adele were to go downstairs.  I brushed Adele's hair and made her neat,
and having ascertained that I was myself in my usual Quaker trim, where
there was nothing to retouch--all being too close and plain, braided
locks included, to admit of disarrangement--we descended, Adele wondering
whether the \emph{petit coffre} was at length come; for, owing to some
mistake, its arrival had hitherto been delayed.  She was gratified: there
it stood, a little carton, on the table when we entered the dining-room.
She appeared to know it by instinct.

"Ma boite! ma boite!" exclaimed she, running towards it.

"Yes, there is your 'boite' at last: take it into a corner, you genuine
daughter of Paris, and amuse yourself with disembowelling it," said the
deep and rather sarcastic voice of Mr. Rochester, proceeding from the
depths of an immense easy-chair at the fireside.  "And mind," he
continued, "don't bother me with any details of the anatomical process,
or any notice of the condition of the entrails: let your operation be
conducted in silence: tiens-toi tranquille, enfant; comprends-tu?"

Adele seemed scarcely to need the warning--she had already retired to a
sofa with her treasure, and was busy untying the cord which secured the
lid.  Having removed this impediment, and lifted certain silvery
envelopes of tissue paper, she merely exclaimed--

"Oh ciel!  Que c'est beau!" and then remained absorbed in ecstatic
contemplation.

"Is Miss Eyre there?" now demanded the master, half rising from his seat
to look round to the door, near which I still stood.

"Ah! well, come forward; be seated here."  He drew a chair near his own.
"I am not fond of the prattle of children," he continued; "for, old
bachelor as I am, I have no pleasant associations connected with their
lisp.  It would be intolerable to me to pass a whole evening
\emph{tete-a-tete} with a brat.  Don't draw that chair farther off, Miss Eyre;
sit down exactly where I placed it--if you please, that is.  Confound
these civilities!  I continually forget them.  Nor do I particularly
affect simple-minded old ladies.  By-the-bye, I must have mine in mind;
it won't do to neglect her; she is a Fairfax, or wed to one; and blood is
said to be thicker than water."

He rang, and despatched an invitation to Mrs. Fairfax, who soon arrived,
knitting-basket in hand.

"Good evening, madam; I sent to you for a charitable purpose.  I have
forbidden Adele to talk to me about her presents, and she is bursting
with repletion: have the goodness to serve her as auditress and
interlocutrice; it will be one of the most benevolent acts you ever
performed."

Adele, indeed, no sooner saw Mrs. Fairfax, than she summoned her to her
sofa, and there quickly filled her lap with the porcelain, the ivory, the
waxen contents of her "boite;" pouring out, meantime, explanations and
raptures in such broken English as she was mistress of.

"Now I have performed the part of a good host," pursued Mr. Rochester,
"put my guests into the way of amusing each other, I ought to be at
liberty to attend to my own pleasure.  Miss Eyre, draw your chair still a
little farther forward: you are yet too far back; I cannot see you
without disturbing my position in this comfortable chair, which I have no
mind to do."

I did as I was bid, though I would much rather have remained somewhat in
the shade; but Mr. Rochester had such a direct way of giving orders, it
seemed a matter of course to obey him promptly.

We were, as I have said, in the dining-room: the lustre, which had been
lit for dinner, filled the room with a festal breadth of light; the large
fire was all red and clear; the purple curtains hung rich and ample
before the lofty window and loftier arch; everything was still, save the
subdued chat of Adele (she dared not speak loud), and, filling up each
pause, the beating of winter rain against the panes.

Mr. Rochester, as he sat in his damask-covered chair, looked different to
what I had seen him look before; not quite so stern--much less gloomy.
There was a smile on his lips, and his eyes sparkled, whether with wine
or not, I am not sure; but I think it very probable.  He was, in short,
in his after-dinner mood; more expanded and genial, and also more self-
indulgent than the frigid and rigid temper of the morning; still he
looked preciously grim, cushioning his massive head against the swelling
back of his chair, and receiving the light of the fire on his granite-
hewn features, and in his great, dark eyes; for he had great, dark eyes,
and very fine eyes, too--not without a certain change in their depths
sometimes, which, if it was not softness, reminded you, at least, of that
feeling.

He had been looking two minutes at the fire, and I had been looking the
same length of time at him, when, turning suddenly, he caught my gaze
fastened on his physiognomy.

"You examine me, Miss Eyre," said he: "do you think me handsome?"

I should, if I had deliberated, have replied to this question by
something conventionally vague and polite; but the answer somehow slipped
from my tongue before I was aware--"No, sir."

"Ah!  By my word! there is something singular about you," said he: "you
have the air of a little \emph{nonnette}; quaint, quiet, grave, and simple, as
you sit with your hands before you, and your eyes generally bent on the
carpet (except, by-the-bye, when they are directed piercingly to my face;
as just now, for instance); and when one asks you a question, or makes a
remark to which you are obliged to reply, you rap out a round rejoinder,
which, if not blunt, is at least brusque.  What do you mean by it?"

"Sir, I was too plain; I beg your pardon.  I ought to have replied that
it was not easy to give an impromptu answer to a question about
appearances; that tastes mostly differ; and that beauty is of little
consequence, or something of that sort."

"You ought to have replied no such thing.  Beauty of little consequence,
indeed!  And so, under pretence of softening the previous outrage, of
stroking and soothing me into placidity, you stick a sly penknife under
my ear!  Go on: what fault do you find with me, pray?  I suppose I have
all my limbs and all my features like any other man?"

"Mr. Rochester, allow me to disown my first answer: I intended no pointed
repartee: it was only a blunder."

"Just so: I think so: and you shall be answerable for it.  Criticise me:
does my forehead not please you?"

He lifted up the sable waves of hair which lay horizontally over his
brow, and showed a solid enough mass of intellectual organs, but an
abrupt deficiency where the suave sign of benevolence should have risen.

"Now, ma'am, am I a fool?"

"Far from it, sir.  You would, perhaps, think me rude if I inquired in
return whether you are a philanthropist?"

"There again!  Another stick of the penknife, when she pretended to pat
my head: and that is because I said I did not like the society of
children and old women (low be it spoken!).  No, young lady, I am not a
general philanthropist; but I bear a conscience;" and he pointed to the
prominences which are said to indicate that faculty, and which,
fortunately for him, were sufficiently conspicuous; giving, indeed, a
marked breadth to the upper part of his head: "and, besides, I once had a
kind of rude tenderness of heart.  When I was as old as you, I was a
feeling fellow enough, partial to the unfledged, unfostered, and unlucky;
but Fortune has knocked me about since: she has even kneaded me with her
knuckles, and now I flatter myself I am hard and tough as an India-rubber
ball; pervious, though, through a chink or two still, and with one
sentient point in the middle of the lump.  Yes: does that leave hope for
me?"

"Hope of what, sir?"

"Of my final re-transformation from India-rubber back to flesh?"

"Decidedly he has had too much wine," I thought; and I did not know what
answer to make to his queer question: how could I tell whether he was
capable of being re-transformed?

"You looked very much puzzled, Miss Eyre; and though you are not pretty
any more than I am handsome, yet a puzzled air becomes you; besides, it
is convenient, for it keeps those searching eyes of yours away from my
physiognomy, and busies them with the worsted flowers of the rug; so
puzzle on.  Young lady, I am disposed to be gregarious and communicative
to-night."

With this announcement he rose from his chair, and stood, leaning his arm
on the marble mantelpiece: in that attitude his shape was seen plainly as
well as his face; his unusual breadth of chest, disproportionate almost
to his length of limb.  I am sure most people would have thought him an
ugly man; yet there was so much unconscious pride in his port; so much
ease in his demeanour; such a look of complete indifference to his own
external appearance; so haughty a reliance on the power of other
qualities, intrinsic or adventitious, to atone for the lack of mere
personal attractiveness, that, in looking at him, one inevitably shared
the indifference, and, even in a blind, imperfect sense, put faith in the
confidence.

"I am disposed to be gregarious and communicative to-night," he repeated,
"and that is why I sent for you: the fire and the chandelier were not
sufficient company for me; nor would Pilot have been, for none of these
can talk.  Adele is a degree better, but still far below the mark; Mrs.
Fairfax ditto; you, I am persuaded, can suit me if you will: you puzzled
me the first evening I invited you down here.  I have almost forgotten
you since: other ideas have driven yours from my head; but to-night I am
resolved to be at ease; to dismiss what importunes, and recall what
pleases.  It would please me now to draw you out--to learn more of
you--therefore speak."

Instead of speaking, I smiled; and not a very complacent or submissive
smile either.

"Speak," he urged.

"What about, sir?"

"Whatever you like.  I leave both the choice of subject and the manner of
treating it entirely to yourself."

Accordingly I sat and said nothing: "If he expects me to talk for the
mere sake of talking and showing off, he will find he has addressed
himself to the wrong person," I thought.

"You are dumb, Miss Eyre."

I was dumb still.  He bent his head a little towards me, and with a
single hasty glance seemed to dive into my eyes.

"Stubborn?" he said, "and annoyed.  Ah! it is consistent.  I put my
request in an absurd, almost insolent form.  Miss Eyre, I beg your
pardon.  The fact is, once for all, I don't wish to treat you like an
inferior: that is" (correcting himself), "I claim only such superiority
as must result from twenty years' difference in age and a century's
advance in experience.  This is legitimate, \emph{et j'y tiens}, as Adele
would say; and it is by virtue of this superiority, and this alone, that
I desire you to have the goodness to talk to me a little now, and divert
my thoughts, which are galled with dwelling on one point--cankering as a
rusty nail."

He had deigned an explanation, almost an apology, and I did not feel
insensible to his condescension, and would not seem so.

"I am willing to amuse you, if I can, sir--quite willing; but I cannot
introduce a topic, because how do I know what will interest you?  Ask me
questions, and I will do my best to answer them."

"Then, in the first place, do you agree with me that I have a right to be
a little masterful, abrupt, perhaps exacting, sometimes, on the grounds I
stated, namely, that I am old enough to be your father, and that I have
battled through a varied experience with many men of many nations, and
roamed over half the globe, while you have lived quietly with one set of
people in one house?"

"Do as you please, sir."

"That is no answer; or rather it is a very irritating, because a very
evasive one.  Reply clearly."

"I don't think, sir, you have a right to command me, merely because you
are older than I, or because you have seen more of the world than I have;
your claim to superiority depends on the use you have made of your time
and experience."

"Humph!  Promptly spoken.  But I won't allow that, seeing that it would
never suit my case, as I have made an indifferent, not to say a bad, use
of both advantages.  Leaving superiority out of the question, then, you
must still agree to receive my orders now and then, without being piqued
or hurt by the tone of command.  Will you?"

I smiled: I thought to myself Mr. Rochester \emph{is} peculiar--he seems to
forget that he pays me 30 pounds per annum for receiving his orders.

"The smile is very well," said he, catching instantly the passing
expression; "but speak too."

"I was thinking, sir, that very few masters would trouble themselves to
inquire whether or not their paid subordinates were piqued and hurt by
their orders."

"Paid subordinates!  What! you are my paid subordinate, are you?  Oh yes,
I had forgotten the salary!  Well then, on that mercenary ground, will
you agree to let me hector a little?"

"No, sir, not on that ground; but, on the ground that you did forget it,
and that you care whether or not a dependent is comfortable in his
dependency, I agree heartily."

"And will you consent to dispense with a great many conventional forms
and phrases, without thinking that the omission arises from insolence?"

"I am sure, sir, I should never mistake informality for insolence: one I
rather like, the other nothing free-born would submit to, even for a
salary."

"Humbug!  Most things free-born will submit to anything for a salary;
therefore, keep to yourself, and don't venture on generalities of which
you are intensely ignorant.  However, I mentally shake hands with you for
your answer, despite its inaccuracy; and as much for the manner in which
it was said, as for the substance of the speech; the manner was frank and
sincere; one does not often see such a manner: no, on the contrary,
affectation, or coldness, or stupid, coarse-minded misapprehension of
one's meaning are the usual rewards of candour.  Not three in three
thousand raw school-girl-governesses would have answered me as you have
just done.  But I don't mean to flatter you: if you are cast in a
different mould to the majority, it is no merit of yours: Nature did it.
And then, after all, I go too fast in my conclusions: for what I yet
know, you may be no better than the rest; you may have intolerable
defects to counterbalance your few good points."

"And so may you," I thought.  My eye met his as the idea crossed my mind:
he seemed to read the glance, answering as if its import had been spoken
as well as imagined--

"Yes, yes, you are right," said he; "I have plenty of faults of my own: I
know it, and I don't wish to palliate them, I assure you.  God wot I need
not be too severe about others; I have a past existence, a series of
deeds, a colour of life to contemplate within my own breast, which might
well call my sneers and censures from my neighbours to myself.  I
started, or rather (for like other defaulters, I like to lay half the
blame on ill fortune and adverse circumstances) was thrust on to a wrong
tack at the age of one-and-twenty, and have never recovered the right
course since: but I might have been very different; I might have been as
good as you--wiser--almost as stainless.  I envy you your peace of mind,
your clean conscience, your unpolluted memory.  Little girl, a memory
without blot or contamination must be an exquisite treasure--an
inexhaustible source of pure refreshment: is it not?"

"How was your memory when you were eighteen, sir?"

"All right then; limpid, salubrious: no gush of bilge water had turned it
to fetid puddle.  I was your equal at eighteen--quite your equal.  Nature
meant me to be, on the whole, a good man, Miss Eyre; one of the better
kind, and you see I am not so.  You would say you don't see it; at least
I flatter myself I read as much in your eye (beware, by-the-bye, what you
express with that organ; I am quick at interpreting its language).  Then
take my word for it,--I am not a villain: you are not to suppose that--not
to attribute to me any such bad eminence; but, owing, I verily believe,
rather to circumstances than to my natural bent, I am a trite commonplace
sinner, hackneyed in all the poor petty dissipations with which the rich
and worthless try to put on life.  Do you wonder that I avow this to you?
Know, that in the course of your future life you will often find yourself
elected the involuntary confidant of your acquaintances' secrets: people
will instinctively find out, as I have done, that it is not your forte to
tell of yourself, but to listen while others talk of themselves; they
will feel, too, that you listen with no malevolent scorn of their
indiscretion, but with a kind of innate sympathy; not the less comforting
and encouraging because it is very unobtrusive in its manifestations."

"How do you know?--how can you guess all this, sir?"

"I know it well; therefore I proceed almost as freely as if I were
writing my thoughts in a diary.  You would say, I should have been
superior to circumstances; so I should--so I should; but you see I was
not.  When fate wronged me, I had not the wisdom to remain cool: I turned
desperate; then I degenerated.  Now, when any vicious simpleton excites
my disgust by his paltry ribaldry, I cannot flatter myself that I am
better than he: I am forced to confess that he and I are on a level.  I
wish I had stood firm--God knows I do!  Dread remorse when you are
tempted to err, Miss Eyre; remorse is the poison of life."

"Repentance is said to be its cure, sir."

"It is not its cure.  Reformation may be its cure; and I could reform--I
have strength yet for that--if--but where is the use of thinking of it,
hampered, burdened, cursed as I am?  Besides, since happiness is
irrevocably denied me, I have a right to get pleasure out of life: and I
\emph{will} get it, cost what it may."

"Then you will degenerate still more, sir."

"Possibly: yet why should I, if I can get sweet, fresh pleasure?  And I
may get it as sweet and fresh as the wild honey the bee gathers on the
moor."

"It will sting--it will taste bitter, sir."

"How do you know?--you never tried it.  How very serious--how very solemn
you look: and you are as ignorant of the matter as this cameo head"
(taking one from the mantelpiece).  "You have no right to preach to me,
you neophyte, that have not passed the porch of life, and are absolutely
unacquainted with its mysteries."

"I only remind you of your own words, sir: you said error brought
remorse, and you pronounced remorse the poison of existence."

"And who talks of error now?  I scarcely think the notion that flittered
across my brain was an error.  I believe it was an inspiration rather
than a temptation: it was very genial, very soothing--I know that.  Here
it comes again!  It is no devil, I assure you; or if it be, it has put on
the robes of an angel of light.  I think I must admit so fair a guest
when it asks entrance to my heart."

"Distrust it, sir; it is not a true angel."

"Once more, how do you know?  By what instinct do you pretend to
distinguish between a fallen seraph of the abyss and a messenger from the
eternal throne--between a guide and a seducer?"

"I judged by your countenance, sir, which was troubled when you said the
suggestion had returned upon you.  I feel sure it will work you more
misery if you listen to it."

"Not at all--it bears the most gracious message in the world: for the
rest, you are not my conscience-keeper, so don't make yourself uneasy.
Here, come in, bonny wanderer!"

He said this as if he spoke to a vision, viewless to any eye but his own;
then, folding his arms, which he had half extended, on his chest, he
seemed to enclose in their embrace the invisible being.

"Now," he continued, again addressing me, "I have received the pilgrim--a
disguised deity, as I verily believe.  Already it has done me good: my
heart was a sort of charnel; it will now be a shrine."

"To speak truth, sir, I don't understand you at all: I cannot keep up the
conversation, because it has got out of my depth.  Only one thing, I
know: you said you were not as good as you should like to be, and that
you regretted your own imperfection;--one thing I can comprehend: you
intimated that to have a sullied memory was a perpetual bane.  It seems
to me, that if you tried hard, you would in time find it possible to
become what you yourself would approve; and that if from this day you
began with resolution to correct your thoughts and actions, you would in
a few years have laid up a new and stainless store of recollections, to
which you might revert with pleasure."

"Justly thought; rightly said, Miss Eyre; and, at this moment, I am
paving hell with energy."

"Sir?"

"I am laying down good intentions, which I believe durable as flint.
Certainly, my associates and pursuits shall be other than they have
been."

"And better?"

"And better--so much better as pure ore is than foul dross.  You seem to
doubt me; I don't doubt myself: I know what my aim is, what my motives
are; and at this moment I pass a law, unalterable as that of the Medes
and Persians, that both are right."

"They cannot be, sir, if they require a new statute to legalise them."

"They are, Miss Eyre, though they absolutely require a new statute:
unheard-of combinations of circumstances demand unheard-of rules."

"That sounds a dangerous maxim, sir; because one can see at once that it
is liable to abuse."

"Sententious sage! so it is: but I swear by my household gods not to
abuse it."

"You are human and fallible."

"I am: so are you--what then?"

"The human and fallible should not arrogate a power with which the divine
and perfect alone can be safely intrusted."

"What power?"

"That of saying of any strange, unsanctioned line of action,--'Let it be
right.'"

"'Let it be right'--the very words: you have pronounced them."

"\emph{May} it be right then," I said, as I rose, deeming it useless to
continue a discourse which was all darkness to me; and, besides, sensible
that the character of my interlocutor was beyond my penetration; at
least, beyond its present reach; and feeling the uncertainty, the vague
sense of insecurity, which accompanies a conviction of ignorance.

"Where are you going?"

"To put Adele to bed: it is past her bedtime."

"You are afraid of me, because I talk like a Sphynx."

"Your language is enigmatical, sir: but though I am bewildered, I am
certainly not afraid."

"You \emph{are} afraid--your self-love dreads a blunder."

"In that sense I do feel apprehensive--I have no wish to talk nonsense."

"If you did, it would be in such a grave, quiet manner, I should mistake
it for sense.  Do you never laugh, Miss Eyre?  Don't trouble yourself to
answer--I see you laugh rarely; but you can laugh very merrily: believe
me, you are not naturally austere, any more than I am naturally vicious.
The Lowood constraint still clings to you somewhat; controlling your
features, muffling your voice, and restricting your limbs; and you fear
in the presence of a man and a brother--or father, or master, or what you
will--to smile too gaily, speak too freely, or move too quickly: but, in
time, I think you will learn to be natural with me, as I find it
impossible to be conventional with you; and then your looks and movements
will have more vivacity and variety than they dare offer now.  I see at
intervals the glance of a curious sort of bird through the close-set bars
of a cage: a vivid, restless, resolute captive is there; were it but
free, it would soar cloud-high.  You are still bent on going?"

"It has struck nine, sir."

"Never mind,--wait a minute: Adele is not ready to go to bed yet.  My
position, Miss Eyre, with my back to the fire, and my face to the room,
favours observation.  While talking to you, I have also occasionally
watched Adele (I have my own reasons for thinking her a curious
study,--reasons that I may, nay, that I shall, impart to you some day).
She pulled out of her box, about ten minutes ago, a little pink silk
frock; rapture lit her face as she unfolded it; coquetry runs in her
blood, blends with her brains, and seasons the marrow of her bones.  'Il
faut que je l'essaie!' cried she, 'et a l'instant meme!' and she rushed
out of the room.  She is now with Sophie, undergoing a robing process: in
a few minutes she will re-enter; and I know what I shall see,--a
miniature of Celine Varens, as she used to appear on the boards at the
rising of--But never mind that.  However, my tenderest feelings are about
to receive a shock: such is my presentiment; stay now, to see whether it
will be realised."

Ere long, Adele's little foot was heard tripping across the hall.  She
entered, transformed as her guardian had predicted.  A dress of
rose-coloured satin, very short, and as full in the skirt as it could be
gathered, replaced the brown frock she had previously worn; a wreath of
rosebuds circled her forehead; her feet were dressed in silk stockings
and small white satin sandals.

"Est-ce que ma robe va bien?" cried she, bounding forwards; "et mes
souliers? et mes bas?  Tenez, je crois que je vais danser!"

And spreading out her dress, she chasseed across the room till, having
reached Mr. Rochester, she wheeled lightly round before him on tip-toe,
then dropped on one knee at his feet, exclaiming--

"Monsieur, je vous remercie mille fois de votre bonte;" then rising, she
added, "C'est comme cela que maman faisait, n'est-ce pas, monsieur?"

"Pre-cise-ly!" was the answer; "and, 'comme cela,' she charmed my English
gold out of my British breeches' pocket.  I have been green, too, Miss
Eyre,--ay, grass green: not a more vernal tint freshens you now than once
freshened me.  My Spring is gone, however, but it has left me that French
floweret on my hands, which, in some moods, I would fain be rid of.  Not
valuing now the root whence it sprang; having found that it was of a sort
which nothing but gold dust could manure, I have but half a liking to the
blossom, especially when it looks so artificial as just now.  I keep it
and rear it rather on the Roman Catholic principle of expiating numerous
sins, great or small, by one good work.  I'll explain all this some day.
Good-night."




\chapter{}


\lettrine{M}{r.} Rochester did, on a future occasion, explain it.  It was one
afternoon, when he chanced to meet me and Adele in the grounds: and while
she played with Pilot and her shuttlecock, he asked me to walk up and
down a long beech avenue within sight of her.

He then said that she was the daughter of a French opera-dancer, Celine
Varens, towards whom he had once cherished what he called a "\emph{grande
passion}."  This passion Celine had professed to return with even
superior ardour.  He thought himself her idol, ugly as he was: he
believed, as he said, that she preferred his "\emph{taille d'athlete}" to the
elegance of the Apollo Belvidere.

"And, Miss Eyre, so much was I flattered by this preference of the Gallic
sylph for her British gnome, that I installed her in an hotel; gave her a
complete establishment of servants, a carriage, cashmeres, diamonds,
dentelles, \&c.  In short, I began the process of ruining myself in the
received style, like any other spoony.  I had not, it seems, the
originality to chalk out a new road to shame and destruction, but trode
the old track with stupid exactness not to deviate an inch from the
beaten centre.  I had--as I deserved to have--the fate of all other
spoonies.  Happening to call one evening when Celine did not expect me, I
found her out; but it was a warm night, and I was tired with strolling
through Paris, so I sat down in her boudoir; happy to breathe the air
consecrated so lately by her presence.  No,--I exaggerate; I never
thought there was any consecrating virtue about her: it was rather a sort
of pastille perfume she had left; a scent of musk and amber, than an
odour of sanctity.  I was just beginning to stifle with the fumes of
conservatory flowers and sprinkled essences, when I bethought myself to
open the window and step out on to the balcony.  It was moonlight and
gaslight besides, and very still and serene.  The balcony was furnished
with a chair or two; I sat down, and took out a cigar,--I will take one
now, if you will excuse me."

Here ensued a pause, filled up by the producing and lighting of a cigar;
having placed it to his lips and breathed a trail of Havannah incense on
the freezing and sunless air, he went on--

"I liked bonbons too in those days, Miss Eyre, and I was
\emph{croquant}--(overlook the barbarism)--\emph{croquant} chocolate comfits, and
smoking alternately, watching meantime the equipages that rolled along
the fashionable streets towards the neighbouring opera-house, when in an
elegant close carriage drawn by a beautiful pair of English horses, and
distinctly seen in the brilliant city-night, I recognised the 'voiture' I
had given Celine.  She was returning: of course my heart thumped with
impatience against the iron rails I leant upon.  The carriage stopped, as
I had expected, at the hotel door; my flame (that is the very word for an
opera inamorata) alighted: though muffed in a cloak--an unnecessary
encumbrance, by-the-bye, on so warm a June evening--I knew her instantly
by her little foot, seen peeping from the skirt of her dress, as she
skipped from the carriage-step.  Bending over the balcony, I was about to
murmur 'Mon ange'--in a tone, of course, which should be audible to the
ear of love alone--when a figure jumped from the carriage after her;
cloaked also; but that was a spurred heel which had rung on the pavement,
and that was a hatted head which now passed under the arched \emph{porte
cochere} of the hotel.

"You never felt jealousy, did you, Miss Eyre?  Of course not: I need not
ask you; because you never felt love.  You have both sentiments yet to
experience: your soul sleeps; the shock is yet to be given which shall
waken it.  You think all existence lapses in as quiet a flow as that in
which your youth has hitherto slid away.  Floating on with closed eyes
and muffled ears, you neither see the rocks bristling not far off in the
bed of the flood, nor hear the breakers boil at their base.  But I tell
you--and you may mark my words--you will come some day to a craggy pass
in the channel, where the whole of life's stream will be broken up into
whirl and tumult, foam and noise: either you will be dashed to atoms on
crag points, or lifted up and borne on by some master-wave into a calmer
current--as I am now.

"I like this day; I like that sky of steel; I like the sternness and
stillness of the world under this frost.  I like Thornfield, its
antiquity, its retirement, its old crow-trees and thorn-trees, its grey
facade, and lines of dark windows reflecting that metal welkin: and yet
how long have I abhorred the very thought of it, shunned it like a great
plague-house?  How I do still abhor--"

He ground his teeth and was silent: he arrested his step and struck his
boot against the hard ground.  Some hated thought seemed to have him in
its grip, and to hold him so tightly that he could not advance.

We were ascending the avenue when he thus paused; the hall was before us.
Lifting his eye to its battlements, he cast over them a glare such as I
never saw before or since.  Pain, shame, ire, impatience, disgust,
detestation, seemed momentarily to hold a quivering conflict in the large
pupil dilating under his ebon eyebrow.  Wild was the wrestle which should
be paramount; but another feeling rose and triumphed: something hard and
cynical: self-willed and resolute: it settled his passion and petrified
his countenance: he went on--

"During the moment I was silent, Miss Eyre, I was arranging a point with
my destiny.  She stood there, by that beech-trunk--a hag like one of
those who appeared to Macbeth on the heath of Forres.  'You like
Thornfield?' she said, lifting her finger; and then she wrote in the air
a memento, which ran in lurid hieroglyphics all along the house-front,
between the upper and lower row of windows, 'Like it if you can!  Like it
if you dare!'

"'I will like it,' said I; 'I dare like it;' and" (he subjoined moodily)
"I will keep my word; I will break obstacles to happiness, to
goodness--yes, goodness.  I wish to be a better man than I have been,
than I am; as Job's leviathan broke the spear, the dart, and the
habergeon, hindrances which others count as iron and brass, I will esteem
but straw and rotten wood."

Adele here ran before him with her shuttlecock.  "Away!" he cried
harshly; "keep at a distance, child; or go in to Sophie!"  Continuing
then to pursue his walk in silence, I ventured to recall him to the point
whence he had abruptly diverged--

"Did you leave the balcony, sir," I asked, "when Mdlle. Varens entered?"

I almost expected a rebuff for this hardly well-timed question, but, on
the contrary, waking out of his scowling abstraction, he turned his eyes
towards me, and the shade seemed to clear off his brow.  "Oh, I had
forgotten Celine!  Well, to resume.  When I saw my charmer thus come in
accompanied by a cavalier, I seemed to hear a hiss, and the green snake
of jealousy, rising on undulating coils from the moonlit balcony, glided
within my waistcoat, and ate its way in two minutes to my heart's core.
Strange!" he exclaimed, suddenly starting again from the point.  "Strange
that I should choose you for the confidant of all this, young lady;
passing strange that you should listen to me quietly, as if it were the
most usual thing in the world for a man like me to tell stories of his
opera-mistresses to a quaint, inexperienced girl like you!  But the last
singularity explains the first, as I intimated once before: you, with
your gravity, considerateness, and caution were made to be the recipient
of secrets.  Besides, I know what sort of a mind I have placed in
communication with my own: I know it is one not liable to take infection:
it is a peculiar mind: it is a unique one.  Happily I do not mean to harm
it: but, if I did, it would not take harm from me.  The more you and I
converse, the better; for while I cannot blight you, you may refresh me."
After this digression he proceeded--

"I remained in the balcony.  'They will come to her boudoir, no doubt,'
thought I: 'let me prepare an ambush.'  So putting my hand in through the
open window, I drew the curtain over it, leaving only an opening through
which I could take observations; then I closed the casement, all but a
chink just wide enough to furnish an outlet to lovers' whispered vows:
then I stole back to my chair; and as I resumed it the pair came in.  My
eye was quickly at the aperture.  Celine's chamber-maid entered, lit a
lamp, left it on the table, and withdrew.  The couple were thus revealed
to me clearly: both removed their cloaks, and there was 'the Varens,'
shining in satin and jewels,--my gifts of course,--and there was her
companion in an officer's uniform; and I knew him for a young roue of a
vicomte--a brainless and vicious youth whom I had sometimes met in
society, and had never thought of hating because I despised him so
absolutely.  On recognising him, the fang of the snake Jealousy was
instantly broken; because at the same moment my love for Celine sank
under an extinguisher.  A woman who could betray me for such a rival was
not worth contending for; she deserved only scorn; less, however, than I,
who had been her dupe.

"They began to talk; their conversation eased me completely: frivolous,
mercenary, heartless, and senseless, it was rather calculated to weary
than enrage a listener.  A card of mine lay on the table; this being
perceived, brought my name under discussion.  Neither of them possessed
energy or wit to belabour me soundly, but they insulted me as coarsely as
they could in their little way: especially Celine, who even waxed rather
brilliant on my personal defects--deformities she termed them.  Now it
had been her custom to launch out into fervent admiration of what she
called my '\emph{beaute male}:' wherein she differed diametrically from you,
who told me point-blank, at the second interview, that you did not think
me handsome.  The contrast struck me at the time and--"

Adele here came running up again.

"Monsieur, John has just been to say that your agent has called and
wishes to see you."

"Ah! in that case I must abridge.  Opening the window, I walked in upon
them; liberated Celine from my protection; gave her notice to vacate her
hotel; offered her a purse for immediate exigencies; disregarded screams,
hysterics, prayers, protestations, convulsions; made an appointment with
the vicomte for a meeting at the Bois de Boulogne.  Next morning I had
the pleasure of encountering him; left a bullet in one of his poor
etiolated arms, feeble as the wing of a chicken in the pip, and then
thought I had done with the whole crew.  But unluckily the Varens, six
months before, had given me this filette Adele, who, she affirmed, was my
daughter; and perhaps she may be, though I see no proofs of such grim
paternity written in her countenance: Pilot is more like me than she.
Some years after I had broken with the mother, she abandoned her child,
and ran away to Italy with a musician or singer.  I acknowledged no
natural claim on Adele's part to be supported by me, nor do I now
acknowledge any, for I am not her father; but hearing that she was quite
destitute, I e'en took the poor thing out of the slime and mud of Paris,
and transplanted it here, to grow up clean in the wholesome soil of an
English country garden.  Mrs. Fairfax found you to train it; but now you
know that it is the illegitimate offspring of a French opera-girl, you
will perhaps think differently of your post and protegee: you will be
coming to me some day with notice that you have found another place--that
you beg me to look out for a new governess, \&c.--Eh?"

"No: Adele is not answerable for either her mother's faults or yours: I
have a regard for her; and now that I know she is, in a sense,
parentless--forsaken by her mother and disowned by you, sir--I shall
cling closer to her than before.  How could I possibly prefer the spoilt
pet of a wealthy family, who would hate her governess as a nuisance, to a
lonely little orphan, who leans towards her as a friend?"

"Oh, that is the light in which you view it!  Well, I must go in now; and
you too: it darkens."

But I stayed out a few minutes longer with Adele and Pilot--ran a race
with her, and played a game of battledore and shuttlecock.  When we went
in, and I had removed her bonnet and coat, I took her on my knee; kept
her there an hour, allowing her to prattle as she liked: not rebuking
even some little freedoms and trivialities into which she was apt to
stray when much noticed, and which betrayed in her a superficiality of
character, inherited probably from her mother, hardly congenial to an
English mind.  Still she had her merits; and I was disposed to appreciate
all that was good in her to the utmost.  I sought in her countenance and
features a likeness to Mr. Rochester, but found none: no trait, no turn
of expression announced relationship.  It was a pity: if she could but
have been proved to resemble him, he would have thought more of her.

It was not till after I had withdrawn to my own chamber for the night,
that I steadily reviewed the tale Mr. Rochester had told me.  As he had
said, there was probably nothing at all extraordinary in the substance of
the narrative itself: a wealthy Englishman's passion for a French dancer,
and her treachery to him, were every-day matters enough, no doubt, in
society; but there was something decidedly strange in the paroxysm of
emotion which had suddenly seized him when he was in the act of
expressing the present contentment of his mood, and his newly revived
pleasure in the old hall and its environs.  I meditated wonderingly on
this incident; but gradually quitting it, as I found it for the present
inexplicable, I turned to the consideration of my master's manner to
myself.  The confidence he had thought fit to repose in me seemed a
tribute to my discretion: I regarded and accepted it as such.  His
deportment had now for some weeks been more uniform towards me than at
the first.  I never seemed in his way; he did not take fits of chilling
hauteur: when he met me unexpectedly, the encounter seemed welcome; he
had always a word and sometimes a smile for me: when summoned by formal
invitation to his presence, I was honoured by a cordiality of reception
that made me feel I really possessed the power to amuse him, and that
these evening conferences were sought as much for his pleasure as for my
benefit.

I, indeed, talked comparatively little, but I heard him talk with relish.
It was his nature to be communicative; he liked to open to a mind
unacquainted with the world glimpses of its scenes and ways (I do not
mean its corrupt scenes and wicked ways, but such as derived their
interest from the great scale on which they were acted, the strange
novelty by which they were characterised); and I had a keen delight in
receiving the new ideas he offered, in imagining the new pictures he
portrayed, and following him in thought through the new regions he
disclosed, never startled or troubled by one noxious allusion.

The ease of his manner freed me from painful restraint: the friendly
frankness, as correct as cordial, with which he treated me, drew me to
him.  I felt at times as if he were my relation rather than my master:
yet he was imperious sometimes still; but I did not mind that; I saw it
was his way.  So happy, so gratified did I become with this new interest
added to life, that I ceased to pine after kindred: my thin
crescent-destiny seemed to enlarge; the blanks of existence were filled
up; my bodily health improved; I gathered flesh and strength.

And was Mr. Rochester now ugly in my eyes?  No, reader: gratitude, and
many associations, all pleasurable and genial, made his face the object I
best liked to see; his presence in a room was more cheering than the
brightest fire.  Yet I had not forgotten his faults; indeed, I could not,
for he brought them frequently before me.  He was proud, sardonic, harsh
to inferiority of every description: in my secret soul I knew that his
great kindness to me was balanced by unjust severity to many others.  He
was moody, too; unaccountably so; I more than once, when sent for to read
to him, found him sitting in his library alone, with his head bent on his
folded arms; and, when he looked up, a morose, almost a malignant, scowl
blackened his features.  But I believed that his moodiness, his
harshness, and his former faults of morality (I say \emph{former}, for now he
seemed corrected of them) had their source in some cruel cross of fate.  I
believed he was naturally a man of better tendencies, higher principles,
and purer tastes than such as circumstances had developed, education
instilled, or destiny encouraged.  I thought there were excellent
materials in him; though for the present they hung together somewhat
spoiled and tangled.  I cannot deny that I grieved for his grief,
whatever that was, and would have given much to assuage it.

Though I had now extinguished my candle and was laid down in bed, I could
not sleep for thinking of his look when he paused in the avenue, and told
how his destiny had risen up before him, and dared him to be happy at
Thornfield.

"Why not?" I asked myself.  "What alienates him from the house?  Will he
leave it again soon?  Mrs. Fairfax said he seldom stayed here longer than
a fortnight at a time; and he has now been resident eight weeks.  If he
does go, the change will be doleful.  Suppose he should be absent spring,
summer, and autumn: how joyless sunshine and fine days will seem!"

I hardly know whether I had slept or not after this musing; at any rate,
I started wide awake on hearing a vague murmur, peculiar and lugubrious,
which sounded, I thought, just above me.  I wished I had kept my candle
burning: the night was drearily dark; my spirits were depressed.  I rose
and sat up in bed, listening.  The sound was hushed.

I tried again to sleep; but my heart beat anxiously: my inward
tranquillity was broken.  The clock, far down in the hall, struck two.
Just then it seemed my chamber-door was touched; as if fingers had swept
the panels in groping a way along the dark gallery outside.  I said, "Who
is there?"  Nothing answered.  I was chilled with fear.

All at once I remembered that it might be Pilot, who, when the kitchen-
door chanced to be left open, not unfrequently found his way up to the
threshold of Mr. Rochester's chamber: I had seen him lying there myself
in the mornings.  The idea calmed me somewhat: I lay down.  Silence
composes the nerves; and as an unbroken hush now reigned again through
the whole house, I began to feel the return of slumber.  But it was not
fated that I should sleep that night.  A dream had scarcely approached my
ear, when it fled affrighted, scared by a marrow-freezing incident
enough.

This was a demoniac laugh--low, suppressed, and deep--uttered, as it
seemed, at the very keyhole of my chamber door.  The head of my bed was
near the door, and I thought at first the goblin-laugher stood at my
bedside--or rather, crouched by my pillow: but I rose, looked round, and
could see nothing; while, as I still gazed, the unnatural sound was
reiterated: and I knew it came from behind the panels.  My first impulse
was to rise and fasten the bolt; my next, again to cry out, "Who is
there?"

Something gurgled and moaned.  Ere long, steps retreated up the gallery
towards the third-storey staircase: a door had lately been made to shut
in that staircase; I heard it open and close, and all was still.

"Was that Grace Poole? and is she possessed with a devil?" thought I.
Impossible now to remain longer by myself: I must go to Mrs. Fairfax.  I
hurried on my frock and a shawl; I withdrew the bolt and opened the door
with a trembling hand.  There was a candle burning just outside, and on
the matting in the gallery.  I was surprised at this circumstance: but
still more was I amazed to perceive the air quite dim, as if filled with
smoke; and, while looking to the right hand and left, to find whence
these blue wreaths issued, I became further aware of a strong smell of
burning.

Something creaked: it was a door ajar; and that door was Mr. Rochester's,
and the smoke rushed in a cloud from thence.  I thought no more of Mrs.
Fairfax; I thought no more of Grace Poole, or the laugh: in an instant, I
was within the chamber.  Tongues of flame darted round the bed: the
curtains were on fire.  In the midst of blaze and vapour, Mr. Rochester
lay stretched motionless, in deep sleep.

"Wake! wake!" I cried.  I shook him, but he only murmured and turned: the
smoke had stupefied him.  Not a moment could be lost: the very sheets
were kindling, I rushed to his basin and ewer; fortunately, one was wide
and the other deep, and both were filled with water.  I heaved them up,
deluged the bed and its occupant, flew back to my own room, brought my
own water-jug, baptized the couch afresh, and, by God's aid, succeeded in
extinguishing the flames which were devouring it.

The hiss of the quenched element, the breakage of a pitcher which I flung
from my hand when I had emptied it, and, above all, the splash of the
shower-bath I had liberally bestowed, roused Mr. Rochester at last.
Though it was now dark, I knew he was awake; because I heard him
fulminating strange anathemas at finding himself lying in a pool of
water.

"Is there a flood?" he cried.

"No, sir," I answered; "but there has been a fire: get up, do; you are
quenched now; I will fetch you a candle."

"In the name of all the elves in Christendom, is that Jane Eyre?" he
demanded.  "What have you done with me, witch, sorceress?  Who is in the
room besides you?  Have you plotted to drown me?"

"I will fetch you a candle, sir; and, in Heaven's name, get up.  Somebody
has plotted something: you cannot too soon find out who and what it is."

"There!  I am up now; but at your peril you fetch a candle yet: wait two
minutes till I get into some dry garments, if any dry there be--yes, here
is my dressing-gown.  Now run!"

I did run; I brought the candle which still remained in the gallery.  He
took it from my hand, held it up, and surveyed the bed, all blackened and
scorched, the sheets drenched, the carpet round swimming in water.

"What is it? and who did it?" he asked.  I briefly related to him what
had transpired: the strange laugh I had heard in the gallery: the step
ascending to the third storey; the smoke,--the smell of fire which had
conducted me to his room; in what state I had found matters there, and
how I had deluged him with all the water I could lay hands on.

{"What is it and who did it?" he asked: p140.jpg}

He listened very gravely; his face, as I went on, expressed more concern
than astonishment; he did not immediately speak when I had concluded.

"Shall I call Mrs. Fairfax?" I asked.

"Mrs. Fairfax?  No; what the deuce would you call her for?  What can she
do?  Let her sleep unmolested."

"Then I will fetch Leah, and wake John and his wife."

"Not at all: just be still.  You have a shawl on.  If you are not warm
enough, you may take my cloak yonder; wrap it about you, and sit down in
the arm-chair: there,--I will put it on.  Now place your feet on the
stool, to keep them out of the wet.  I am going to leave you a few
minutes.  I shall take the candle.  Remain where you are till I return;
be as still as a mouse.  I must pay a visit to the second storey.  Don't
move, remember, or call any one."

He went: I watched the light withdraw.  He passed up the gallery very
softly, unclosed the staircase door with as little noise as possible,
shut it after him, and the last ray vanished.  I was left in total
darkness.  I listened for some noise, but heard nothing.  A very long
time elapsed.  I grew weary: it was cold, in spite of the cloak; and then
I did not see the use of staying, as I was not to rouse the house.  I was
on the point of risking Mr. Rochester's displeasure by disobeying his
orders, when the light once more gleamed dimly on the gallery wall, and I
heard his unshod feet tread the matting.  "I hope it is he," thought I,
"and not something worse."

He re-entered, pale and very gloomy.  "I have found it all out," said he,
setting his candle down on the washstand; "it is as I thought."

"How, sir?"

He made no reply, but stood with his arms folded, looking on the ground.
At the end of a few minutes he inquired in rather a peculiar tone--

"I forget whether you said you saw anything when you opened your chamber
door."

"No, sir, only the candlestick on the ground."

"But you heard an odd laugh?  You have heard that laugh before, I should
think, or something like it?"

"Yes, sir: there is a woman who sews here, called Grace Poole,--she
laughs in that way.  She is a singular person."

"Just so.  Grace Poole--you have guessed it.  She is, as you say,
singular--very.  Well, I shall reflect on the subject.  Meantime, I am
glad that you are the only person, besides myself, acquainted with the
precise details of to-night's incident.  You are no talking fool: say
nothing about it.  I will account for this state of affairs" (pointing to
the bed): "and now return to your own room.  I shall do very well on the
sofa in the library for the rest of the night.  It is near four:--in two
hours the servants will be up."

"Good-night, then, sir," said I, departing.

He seemed surprised--very inconsistently so, as he had just told me to
go.

"What!" he exclaimed, "are you quitting me already, and in that way?"

"You said I might go, sir."

"But not without taking leave; not without a word or two of
acknowledgment and good-will: not, in short, in that brief, dry fashion.
Why, you have saved my life!--snatched me from a horrible and
excruciating death! and you walk past me as if we were mutual strangers!
At least shake hands."

He held out his hand; I gave him mine: he took it first in one, them in
both his own.

"You have saved my life: I have a pleasure in owing you so immense a
debt.  I cannot say more.  Nothing else that has being would have been
tolerable to me in the character of creditor for such an obligation: but
you: it is different;--I feel your benefits no burden, Jane."

He paused; gazed at me: words almost visible trembled on his lips,--but
his voice was checked.

"Good-night again, sir.  There is no debt, benefit, burden, obligation,
in the case."

"I knew," he continued, "you would do me good in some way, at some
time;--I saw it in your eyes when I first beheld you: their expression
and smile did not"--(again he stopped)--"did not" (he proceeded hastily)
"strike delight to my very inmost heart so for nothing.  People talk of
natural sympathies; I have heard of good genii: there are grains of truth
in the wildest fable.  My cherished preserver, goodnight!"

Strange energy was in his voice, strange fire in his look.

"I am glad I happened to be awake," I said: and then I was going.

"What! you \emph{will} go?"

"I am cold, sir."

"Cold?  Yes,--and standing in a pool!  Go, then, Jane; go!"  But he still
retained my hand, and I could not free it.  I bethought myself of an
expedient.

"I think I hear Mrs. Fairfax move, sir," said I.

"Well, leave me:" he relaxed his fingers, and I was gone.

I regained my couch, but never thought of sleep.  Till morning dawned I
was tossed on a buoyant but unquiet sea, where billows of trouble rolled
under surges of joy.  I thought sometimes I saw beyond its wild waters a
shore, sweet as the hills of Beulah; and now and then a freshening gale,
wakened by hope, bore my spirit triumphantly towards the bourne: but I
could not reach it, even in fancy--a counteracting breeze blew off land,
and continually drove me back.  Sense would resist delirium: judgment
would warn passion.  Too feverish to rest, I rose as soon as day dawned.




\chapter{}


\lettrine{I} both wished and feared to see Mr. Rochester on the day which followed
this sleepless night: I wanted to hear his voice again, yet feared to
meet his eye.  During the early part of the morning, I momentarily
expected his coming; he was not in the frequent habit of entering the
schoolroom, but he did step in for a few minutes sometimes, and I had the
impression that he was sure to visit it that day.

But the morning passed just as usual: nothing happened to interrupt the
quiet course of Adele's studies; only soon after breakfast, I heard some
bustle in the neighbourhood of Mr. Rochester's chamber, Mrs. Fairfax's
voice, and Leah's, and the cook's--that is, John's wife--and even John's
own gruff tones.  There were exclamations of "What a mercy master was not
burnt in his bed!"  "It is always dangerous to keep a candle lit at
night."  "How providential that he had presence of mind to think of the
water-jug!"  "I wonder he waked nobody!"  "It is to be hoped he will not
take cold with sleeping on the library sofa," \&c.

To much confabulation succeeded a sound of scrubbing and setting to
rights; and when I passed the room, in going downstairs to dinner, I saw
through the open door that all was again restored to complete order; only
the bed was stripped of its hangings.  Leah stood up in the window-seat,
rubbing the panes of glass dimmed with smoke.  I was about to address
her, for I wished to know what account had been given of the affair: but,
on advancing, I saw a second person in the chamber--a woman sitting on a
chair by the bedside, and sewing rings to new curtains.  That woman was
no other than Grace Poole.

There she sat, staid and taciturn-looking, as usual, in her brown stuff
gown, her check apron, white handkerchief, and cap.  She was intent on
her work, in which her whole thoughts seemed absorbed: on her hard
forehead, and in her commonplace features, was nothing either of the
paleness or desperation one would have expected to see marking the
countenance of a woman who had attempted murder, and whose intended
victim had followed her last night to her lair, and (as I believed),
charged her with the crime she wished to perpetrate.  I was
amazed--confounded.  She looked up, while I still gazed at her: no start,
no increase or failure of colour betrayed emotion, consciousness of
guilt, or fear of detection.  She said "Good morning, Miss," in her usual
phlegmatic and brief manner; and taking up another ring and more tape,
went on with her sewing.

"I will put her to some test," thought I: "such absolute impenetrability
is past comprehension."

"Good morning, Grace," I said.  "Has anything happened here?  I thought I
heard the servants all talking together a while ago."

"Only master had been reading in his bed last night; he fell asleep with
his candle lit, and the curtains got on fire; but, fortunately, he awoke
before the bed-clothes or the wood-work caught, and contrived to quench
the flames with the water in the ewer."

"A strange affair!" I said, in a low voice: then, looking at her
fixedly--"Did Mr. Rochester wake nobody?  Did no one hear him move?"

She again raised her eyes to me, and this time there was something of
consciousness in their expression.  She seemed to examine me warily; then
she answered--

"The servants sleep so far off, you know, Miss, they would not be likely
to hear.  Mrs. Fairfax's room and yours are the nearest to master's; but
Mrs. Fairfax said she heard nothing: when people get elderly, they often
sleep heavy."  She paused, and then added, with a sort of assumed
indifference, but still in a marked and significant tone--"But you are
young, Miss; and I should say a light sleeper: perhaps you may have heard
a noise?"

"I did," said I, dropping my voice, so that Leah, who was still polishing
the panes, could not hear me, "and at first I thought it was Pilot: but
Pilot cannot laugh; and I am certain I heard a laugh, and a strange one."

She took a new needleful of thread, waxed it carefully, threaded her
needle with a steady hand, and then observed, with perfect composure--

"It is hardly likely master would laugh, I should think, Miss, when he
was in such danger: You must have been dreaming."

"I was not dreaming," I said, with some warmth, for her brazen coolness
provoked me.  Again she looked at me; and with the same scrutinising and
conscious eye.

"Have you told master that you heard a laugh?" she inquired.

"I have not had the opportunity of speaking to him this morning."

"You did not think of opening your door and looking out into the
gallery?" she further asked.

She appeared to be cross-questioning me, attempting to draw from me
information unawares.  The idea struck me that if she discovered I knew
or suspected her guilt, she would be playing of some of her malignant
pranks on me; I thought it advisable to be on my guard.

"On the contrary," said I, "I bolted my door."

"Then you are not in the habit of bolting your door every night before
you get into bed?"

"Fiend! she wants to know my habits, that she may lay her plans
accordingly!"  Indignation again prevailed over prudence: I replied
sharply, "Hitherto I have often omitted to fasten the bolt: I did not
think it necessary.  I was not aware any danger or annoyance was to be
dreaded at Thornfield Hall: but in future" (and I laid marked stress on
the words) "I shall take good care to make all secure before I venture to
lie down."

"It will be wise so to do," was her answer: "this neighbourhood is as
quiet as any I know, and I never heard of the hall being attempted by
robbers since it was a house; though there are hundreds of pounds' worth
of plate in the plate-closet, as is well known.  And you see, for such a
large house, there are very few servants, because master has never lived
here much; and when he does come, being a bachelor, he needs little
waiting on: but I always think it best to err on the safe side; a door is
soon fastened, and it is as well to have a drawn bolt between one and any
mischief that may be about.  A deal of people, Miss, are for trusting all
to Providence; but I say Providence will not dispense with the means,
though He often blesses them when they are used discreetly."  And here
she closed her harangue: a long one for her, and uttered with the
demureness of a Quakeress.

I still stood absolutely dumfoundered at what appeared to me her
miraculous self-possession and most inscrutable hypocrisy, when the cook
entered.

"Mrs. Poole," said she, addressing Grace, "the servants' dinner will soon
be ready: will you come down?"

"No; just put my pint of porter and bit of pudding on a tray, and I'll
carry it upstairs."

"You'll have some meat?"

"Just a morsel, and a taste of cheese, that's all."

"And the sago?"

"Never mind it at present: I shall be coming down before teatime: I'll
make it myself."

The cook here turned to me, saying that Mrs. Fairfax was waiting for me:
so I departed.

I hardly heard Mrs. Fairfax's account of the curtain conflagration during
dinner, so much was I occupied in puzzling my brains over the enigmatical
character of Grace Poole, and still more in pondering the problem of her
position at Thornfield and questioning why she had not been given into
custody that morning, or, at the very least, dismissed from her master's
service.  He had almost as much as declared his conviction of her
criminality last night: what mysterious cause withheld him from accusing
her?  Why had he enjoined me, too, to secrecy?  It was strange: a bold,
vindictive, and haughty gentleman seemed somehow in the power of one of
the meanest of his dependants; so much in her power, that even when she
lifted her hand against his life, he dared not openly charge her with the
attempt, much less punish her for it.

Had Grace been young and handsome, I should have been tempted to think
that tenderer feelings than prudence or fear influenced Mr. Rochester in
her behalf; but, hard-favoured and matronly as she was, the idea could
not be admitted.  "Yet," I reflected, "she has been young once; her youth
would be contemporary with her master's: Mrs. Fairfax told me once, she
had lived here many years.  I don't think she can ever have been pretty;
but, for aught I know, she may possess originality and strength of
character to compensate for the want of personal advantages.  Mr.
Rochester is an amateur of the decided and eccentric: Grace is eccentric
at least.  What if a former caprice (a freak very possible to a nature so
sudden and headstrong as his) has delivered him into her power, and she
now exercises over his actions a secret influence, the result of his own
indiscretion, which he cannot shake off, and dare not disregard?"  But,
having reached this point of conjecture, Mrs. Poole's square, flat
figure, and uncomely, dry, even coarse face, recurred so distinctly to my
mind's eye, that I thought, "No; impossible! my supposition cannot be
correct.  Yet," suggested the secret voice which talks to us in our own
hearts, "you are not beautiful either, and perhaps Mr. Rochester approves
you: at any rate, you have often felt as if he did; and last
night--remember his words; remember his look; remember his voice!"

I well remembered all; language, glance, and tone seemed at the moment
vividly renewed.  I was now in the schoolroom; Adele was drawing; I bent
over her and directed her pencil.  She looked up with a sort of start.

"Qu' avez-vous, mademoiselle?" said she.  "Vos doigts tremblent comme la
feuille, et vos joues sont rouges: mais, rouges comme des cerises!"

"I am hot, Adele, with stooping!"  She went on sketching; I went on
thinking.

I hastened to drive from my mind the hateful notion I had been conceiving
respecting Grace Poole; it disgusted me.  I compared myself with her, and
found we were different.  Bessie Leaven had said I was quite a lady; and
she spoke truth--I was a lady.  And now I looked much better than I did
when Bessie saw me; I had more colour and more flesh, more life, more
vivacity, because I had brighter hopes and keener enjoyments.

"Evening approaches," said I, as I looked towards the window.  "I have
never heard Mr. Rochester's voice or step in the house to-day; but surely
I shall see him before night: I feared the meeting in the morning; now I
desire it, because expectation has been so long baffled that it is grown
impatient."

When dusk actually closed, and when Adele left me to go and play in the
nursery with Sophie, I did most keenly desire it.  I listened for the
bell to ring below; I listened for Leah coming up with a message; I
fancied sometimes I heard Mr. Rochester's own tread, and I turned to the
door, expecting it to open and admit him.  The door remained shut;
darkness only came in through the window.  Still it was not late; he
often sent for me at seven and eight o'clock, and it was yet but six.
Surely I should not be wholly disappointed to-night, when I had so many
things to say to him!  I wanted again to introduce the subject of Grace
Poole, and to hear what he would answer; I wanted to ask him plainly if
he really believed it was she who had made last night's hideous attempt;
and if so, why he kept her wickedness a secret.  It little mattered
whether my curiosity irritated him; I knew the pleasure of vexing and
soothing him by turns; it was one I chiefly delighted in, and a sure
instinct always prevented me from going too far; beyond the verge of
provocation I never ventured; on the extreme brink I liked well to try my
skill.  Retaining every minute form of respect, every propriety of my
station, I could still meet him in argument without fear or uneasy
restraint; this suited both him and me.

A tread creaked on the stairs at last.  Leah made her appearance; but it
was only to intimate that tea was ready in Mrs. Fairfax's room.  Thither
I repaired, glad at least to go downstairs; for that brought me, I
imagined, nearer to Mr. Rochester's presence.

"You must want your tea," said the good lady, as I joined her; "you ate
so little at dinner.  I am afraid," she continued, "you are not well to-
day: you look flushed and feverish."

"Oh, quite well!  I never felt better."

"Then you must prove it by evincing a good appetite; will you fill the
teapot while I knit off this needle?"  Having completed her task, she
rose to draw down the blind, which she had hitherto kept up, by way, I
suppose, of making the most of daylight, though dusk was now fast
deepening into total obscurity.

"It is fair to-night," said she, as she looked through the panes, "though
not starlight; Mr. Rochester has, on the whole, had a favourable day for
his journey."

"Journey!--Is Mr. Rochester gone anywhere?  I did not know he was out."

"Oh, he set off the moment he had breakfasted!  He is gone to the Leas,
Mr. Eshton's place, ten miles on the other side Millcote.  I believe
there is quite a party assembled there; Lord Ingram, Sir George Lynn,
Colonel Dent, and others."

"Do you expect him back to-night?"

"No--nor to-morrow either; I should think he is very likely to stay a
week or more: when these fine, fashionable people get together, they are
so surrounded by elegance and gaiety, so well provided with all that can
please and entertain, they are in no hurry to separate.  Gentlemen
especially are often in request on such occasions; and Mr. Rochester is
so talented and so lively in society, that I believe he is a general
favourite: the ladies are very fond of him; though you would not think
his appearance calculated to recommend him particularly in their eyes:
but I suppose his acquirements and abilities, perhaps his wealth and good
blood, make amends for any little fault of look."

"Are there ladies at the Leas?"

"There are Mrs. Eshton and her three daughters--very elegant young ladies
indeed; and there are the Honourable Blanche and Mary Ingram, most
beautiful women, I suppose: indeed I have seen Blanche, six or seven
years since, when she was a girl of eighteen.  She came here to a
Christmas ball and party Mr. Rochester gave.  You should have seen the
dining-room that day--how richly it was decorated, how brilliantly lit
up!  I should think there were fifty ladies and gentlemen present--all of
the first county families; and Miss Ingram was considered the belle of
the evening."

"You saw her, you say, Mrs. Fairfax: what was she like?"

"Yes, I saw her.  The dining-room doors were thrown open; and, as it was
Christmas-time, the servants were allowed to assemble in the hall, to
hear some of the ladies sing and play.  Mr. Rochester would have me to
come in, and I sat down in a quiet corner and watched them.  I never saw
a more splendid scene: the ladies were magnificently dressed; most of
them--at least most of the younger ones--looked handsome; but Miss Ingram
was certainly the queen."

"And what was she like?"

"Tall, fine bust, sloping shoulders; long, graceful neck: olive
complexion, dark and clear; noble features; eyes rather like Mr.
Rochester's: large and black, and as brilliant as her jewels.  And then
she had such a fine head of hair; raven-black and so becomingly arranged:
a crown of thick plaits behind, and in front the longest, the glossiest
curls I ever saw.  She was dressed in pure white; an amber-coloured scarf
was passed over her shoulder and across her breast, tied at the side, and
descending in long, fringed ends below her knee.  She wore an
amber-coloured flower, too, in her hair: it contrasted well with the
jetty mass of her curls."

"She was greatly admired, of course?"

"Yes, indeed: and not only for her beauty, but for her accomplishments.
She was one of the ladies who sang: a gentleman accompanied her on the
piano.  She and Mr. Rochester sang a duet."

"Mr. Rochester?  I was not aware he could sing."

"Oh! he has a fine bass voice, and an excellent taste for music."

"And Miss Ingram: what sort of a voice had she?"

"A very rich and powerful one: she sang delightfully; it was a treat to
listen to her;--and she played afterwards.  I am no judge of music, but
Mr. Rochester is; and I heard him say her execution was remarkably good."

"And this beautiful and accomplished lady, she is not yet married?"

"It appears not: I fancy neither she nor her sister have very large
fortunes.  Old Lord Ingram's estates were chiefly entailed, and the
eldest son came in for everything almost."

"But I wonder no wealthy nobleman or gentleman has taken a fancy to her:
Mr. Rochester, for instance.  He is rich, is he not?"

"Oh! yes.  But you see there is a considerable difference in age: Mr.
Rochester is nearly forty; she is but twenty-five."

"What of that?  More unequal matches are made every day."

"True: yet I should scarcely fancy Mr. Rochester would entertain an idea
of the sort.  But you eat nothing: you have scarcely tasted since you
began tea."

"No: I am too thirsty to eat.  Will you let me have another cup?"

I was about again to revert to the probability of a union between Mr.
Rochester and the beautiful Blanche; but Adele came in, and the
conversation was turned into another channel.

When once more alone, I reviewed the information I had got; looked into
my heart, examined its thoughts and feelings, and endeavoured to bring
back with a strict hand such as had been straying through imagination's
boundless and trackless waste, into the safe fold of common sense.

Arraigned at my own bar, Memory having given her evidence of the hopes,
wishes, sentiments I had been cherishing since last night--of the general
state of mind in which I had indulged for nearly a fortnight past; Reason
having come forward and told, in her own quiet way a plain, unvarnished
tale, showing how I had rejected the real, and rabidly devoured the
ideal;--I pronounced judgment to this effect:--

That a greater fool than Jane Eyre had never breathed the breath of life;
that a more fantastic idiot had never surfeited herself on sweet lies,
and swallowed poison as if it were nectar.

"\emph{You}," I said, "a favourite with Mr. Rochester?  \emph{You} gifted with the
power of pleasing him?  \emph{You} of importance to him in any way?  Go! your
folly sickens me.  And you have derived pleasure from occasional tokens
of preference--equivocal tokens shown by a gentleman of family and a man
of the world to a dependent and a novice.  How dared you?  Poor stupid
dupe!--Could not even self-interest make you wiser? You repeated to
yourself this morning the brief scene of last night?--Cover your face and
be ashamed!  He said something in praise of your eyes, did he?  Blind
puppy!  Open their bleared lids and look on your own accursed
senselessness!  It does good to no woman to be flattered by her superior,
who cannot possibly intend to marry her; and it is madness in all women
to let a secret love kindle within them, which, if unreturned and
unknown, must devour the life that feeds it; and, if discovered and
responded to, must lead, \emph{ignis-fatus}-like, into miry wilds whence there
is no extrication.

"Listen, then, Jane Eyre, to your sentence: to-morrow, place the glass
before you, and draw in chalk your own picture, faithfully, without
softening one defect; omit no harsh line, smooth away no displeasing
irregularity; write under it, 'Portrait of a Governess, disconnected,
poor, and plain.'

"Afterwards, take a piece of smooth ivory--you have one prepared in your
drawing-box: take your palette, mix your freshest, finest, clearest
tints; choose your most delicate camel-hair pencils; delineate carefully
the loveliest face you can imagine; paint it in your softest shades and
sweetest lines, according to the description given by Mrs. Fairfax of
Blanche Ingram; remember the raven ringlets, the oriental eye;--What! you
revert to Mr. Rochester as a model!  Order!  No snivel!--no sentiment!--no
regret!  I will endure only sense and resolution.  Recall the august yet
harmonious lineaments, the Grecian neck and bust; let the round and
dazzling arm be visible, and the delicate hand; omit neither diamond ring
nor gold bracelet; portray faithfully the attire, aerial lace and
glistening satin, graceful scarf and golden rose; call it 'Blanche, an
accomplished lady of rank.'

"Whenever, in future, you should chance to fancy Mr. Rochester thinks
well of you, take out these two pictures and compare them: say, 'Mr.
Rochester might probably win that noble lady's love, if he chose to
strive for it; is it likely he would waste a serious thought on this
indigent and insignificant plebeian?'"

"I'll do it," I resolved: and having framed this determination, I grew
calm, and fell asleep.

I kept my word.  An hour or two sufficed to sketch my own portrait in
crayons; and in less than a fortnight I had completed an ivory miniature
of an imaginary Blanche Ingram.  It looked a lovely face enough, and when
compared with the real head in chalk, the contrast was as great as self-
control could desire.  I derived benefit from the task: it had kept my
head and hands employed, and had given force and fixedness to the new
impressions I wished to stamp indelibly on my heart.

Ere long, I had reason to congratulate myself on the course of wholesome
discipline to which I had thus forced my feelings to submit.  Thanks to
it, I was able to meet subsequent occurrences with a decent calm, which,
had they found me unprepared, I should probably have been unequal to
maintain, even externally.




\chapter{}


\lettrine{A} week passed, and no news arrived of Mr. Rochester: ten days, and still
he did not come.  Mrs. Fairfax said she should not be surprised if he
were to go straight from the Leas to London, and thence to the Continent,
and not show his face again at Thornfield for a year to come; he had not
unfrequently quitted it in a manner quite as abrupt and unexpected.  When
I heard this, I was beginning to feel a strange chill and failing at the
heart.  I was actually permitting myself to experience a sickening sense
of disappointment; but rallying my wits, and recollecting my principles,
I at once called my sensations to order; and it was wonderful how I got
over the temporary blunder--how I cleared up the mistake of supposing Mr.
Rochester's movements a matter in which I had any cause to take a vital
interest.  Not that I humbled myself by a slavish notion of inferiority:
on the contrary, I just said--

"You have nothing to do with the master of Thornfield, further than to
receive the salary he gives you for teaching his protegee, and to be
grateful for such respectful and kind treatment as, if you do your duty,
you have a right to expect at his hands.  Be sure that is the only tie he
seriously acknowledges between you and him; so don't make him the object
of your fine feelings, your raptures, agonies, and so forth.  He is not
of your order: keep to your caste, and be too self-respecting to lavish
the love of the whole heart, soul, and strength, where such a gift is not
wanted and would be despised."

I went on with my day's business tranquilly; but ever and anon vague
suggestions kept wandering across my brain of reasons why I should quit
Thornfield; and I kept involuntarily framing advertisements and pondering
conjectures about new situations: these thoughts I did not think to
check; they might germinate and bear fruit if they could.

Mr. Rochester had been absent upwards of a fortnight, when the post
brought Mrs. Fairfax a letter.

"It is from the master," said she, as she looked at the direction.  "Now
I suppose we shall know whether we are to expect his return or not."

And while she broke the seal and perused the document, I went on taking
my coffee (we were at breakfast): it was hot, and I attributed to that
circumstance a fiery glow which suddenly rose to my face.  Why my hand
shook, and why I involuntarily spilt half the contents of my cup into my
saucer, I did not choose to consider.

"Well, I sometimes think we are too quiet; but we run a chance of being
busy enough now: for a little while at least," said Mrs. Fairfax, still
holding the note before her spectacles.

Ere I permitted myself to request an explanation, I tied the string of
Adele's pinafore, which happened to be loose: having helped her also to
another bun and refilled her mug with milk, I said, nonchalantly--

"Mr. Rochester is not likely to return soon, I suppose?"

"Indeed he is--in three days, he says: that will be next Thursday; and
not alone either.  I don't know how many of the fine people at the Leas
are coming with him: he sends directions for all the best bedrooms to be
prepared; and the library and drawing-rooms are to be cleaned out; I am
to get more kitchen hands from the George Inn, at Millcote, and from
wherever else I can; and the ladies will bring their maids and the
gentlemen their valets: so we shall have a full house of it."  And Mrs.
Fairfax swallowed her breakfast and hastened away to commence operations.

The three days were, as she had foretold, busy enough.  I had thought all
the rooms at Thornfield beautifully clean and well arranged; but it
appears I was mistaken.  Three women were got to help; and such
scrubbing, such brushing, such washing of paint and beating of carpets,
such taking down and putting up of pictures, such polishing of mirrors
and lustres, such lighting of fires in bedrooms, such airing of sheets
and feather-beds on hearths, I never beheld, either before or since.
Adele ran quite wild in the midst of it: the preparations for company and
the prospect of their arrival, seemed to throw her into ecstasies.  She
would have Sophie to look over all her "toilettes," as she called frocks;
to furbish up any that were "\emph{passees}," and to air and arrange the new.
For herself, she did nothing but caper about in the front chambers, jump
on and off the bedsteads, and lie on the mattresses and piled-up bolsters
and pillows before the enormous fires roaring in the chimneys.  From
school duties she was exonerated: Mrs. Fairfax had pressed me into her
service, and I was all day in the storeroom, helping (or hindering) her
and the cook; learning to make custards and cheese-cakes and French
pastry, to truss game and garnish desert-dishes.

The party were expected to arrive on Thursday afternoon, in time for
dinner at six.  During the intervening period I had no time to nurse
chimeras; and I believe I was as active and gay as anybody--Adele
excepted.  Still, now and then, I received a damping check to my
cheerfulness; and was, in spite of myself, thrown back on the region of
doubts and portents, and dark conjectures.  This was when I chanced to
see the third-storey staircase door (which of late had always been kept
locked) open slowly, and give passage to the form of Grace Poole, in prim
cap, white apron, and handkerchief; when I watched her glide along the
gallery, her quiet tread muffled in a list slipper; when I saw her look
into the bustling, topsy-turvy bedrooms,--just say a word, perhaps, to
the charwoman about the proper way to polish a grate, or clean a marble
mantelpiece, or take stains from papered walls, and then pass on.  She
would thus descend to the kitchen once a day, eat her dinner, smoke a
moderate pipe on the hearth, and go back, carrying her pot of porter with
her, for her private solace, in her own gloomy, upper haunt.  Only one
hour in the twenty-four did she pass with her fellow-servants below; all
the rest of her time was spent in some low-ceiled, oaken chamber of the
second storey: there she sat and sewed--and probably laughed drearily to
herself,--as companionless as a prisoner in his dungeon.

The strangest thing of all was, that not a soul in the house, except me,
noticed her habits, or seemed to marvel at them: no one discussed her
position or employment; no one pitied her solitude or isolation.  I once,
indeed, overheard part of a dialogue between Leah and one of the
charwomen, of which Grace formed the subject.  Leah had been saying
something I had not caught, and the charwoman remarked--

"She gets good wages, I guess?"

"Yes," said Leah; "I wish I had as good; not that mine are to complain
of,--there's no stinginess at Thornfield; but they're not one fifth of
the sum Mrs. Poole receives.  And she is laying by: she goes every
quarter to the bank at Millcote.  I should not wonder but she has saved
enough to keep her independent if she liked to leave; but I suppose she's
got used to the place; and then she's not forty yet, and strong and able
for anything.  It is too soon for her to give up business."

"She is a good hand, I daresay," said the charwoman.

"Ah!--she understands what she has to do,--nobody better," rejoined Leah
significantly; "and it is not every one could fill her shoes--not for all
the money she gets."

"That it is not!" was the reply.  "I wonder whether the master--"

The charwoman was going on; but here Leah turned and perceived me, and
she instantly gave her companion a nudge.

"Doesn't she know?" I heard the woman whisper.

Leah shook her head, and the conversation was of course dropped.  All I
had gathered from it amounted to this,--that there was a mystery at
Thornfield; and that from participation in that mystery I was purposely
excluded.

Thursday came: all work had been completed the previous evening; carpets
were laid down, bed-hangings festooned, radiant white counterpanes
spread, toilet tables arranged, furniture rubbed, flowers piled in vases:
both chambers and saloons looked as fresh and bright as hands could make
them.  The hall, too, was scoured; and the great carved clock, as well as
the steps and banisters of the staircase, were polished to the brightness
of glass; in the dining-room, the sideboard flashed resplendent with
plate; in the drawing-room and boudoir, vases of exotics bloomed on all
sides.

Afternoon arrived: Mrs. Fairfax assumed her best black satin gown, her
gloves, and her gold watch; for it was her part to receive the
company,--to conduct the ladies to their rooms, \&c.  Adele, too, would be
dressed: though I thought she had little chance of being introduced to
the party that day at least.  However, to please her, I allowed Sophie to
apparel her in one of her short, full muslin frocks.  For myself, I had
no need to make any change; I should not be called upon to quit my
sanctum of the schoolroom; for a sanctum it was now become to me,--"a
very pleasant refuge in time of trouble."

It had been a mild, serene spring day--one of those days which, towards
the end of March or the beginning of April, rise shining over the earth
as heralds of summer.  It was drawing to an end now; but the evening was
even warm, and I sat at work in the schoolroom with the window open.

"It gets late," said Mrs. Fairfax, entering in rustling state.  "I am
glad I ordered dinner an hour after the time Mr. Rochester mentioned; for
it is past six now.  I have sent John down to the gates to see if there
is anything on the road: one can see a long way from thence in the
direction of Millcote."  She went to the window.  "Here he is!" said she.
"Well, John" (leaning out), "any news?"

"They're coming, ma'am," was the answer.  "They'll be here in ten
minutes."

Adele flew to the window.  I followed, taking care to stand on one side,
so that, screened by the curtain, I could see without being seen.

The ten minutes John had given seemed very long, but at last wheels were
heard; four equestrians galloped up the drive, and after them came two
open carriages.  Fluttering veils and waving plumes filled the vehicles;
two of the cavaliers were young, dashing-looking gentlemen; the third was
Mr. Rochester, on his black horse, Mesrour, Pilot bounding before him; at
his side rode a lady, and he and she were the first of the party.  Her
purple riding-habit almost swept the ground, her veil streamed long on
the breeze; mingling with its transparent folds, and gleaming through
them, shone rich raven ringlets.

"Miss Ingram!" exclaimed Mrs. Fairfax, and away she hurried to her post
below.

The cavalcade, following the sweep of the drive, quickly turned the angle
of the house, and I lost sight of it.  Adele now petitioned to go down;
but I took her on my knee, and gave her to understand that she must not
on any account think of venturing in sight of the ladies, either now or
at any other time, unless expressly sent for: that Mr. Rochester would be
very angry, \&c.  "Some natural tears she shed" on being told this; but as
I began to look very grave, she consented at last to wipe them.

A joyous stir was now audible in the hall: gentlemen's deep tones and
ladies' silvery accents blent harmoniously together, and distinguishable
above all, though not loud, was the sonorous voice of the master of
Thornfield Hall, welcoming his fair and gallant guests under its roof.
Then light steps ascended the stairs; and there was a tripping through
the gallery, and soft cheerful laughs, and opening and closing doors,
and, for a time, a hush.

"Elles changent de toilettes," said Adele; who, listening attentively,
had followed every movement; and she sighed.

"Chez maman," said she, "quand il y avait du monde, je le suivais
partout, au salon et a leurs chambres; souvent je regardais les femmes de
chambre coiffer et habiller les dames, et c'etait si amusant: comme cela
on apprend."

"Don't you feel hungry, Adele?"

"Mais oui, mademoiselle: voila cinq ou six heures que nous n'avons pas
mange."

"Well now, while the ladies are in their rooms, I will venture down and
get you something to eat."

And issuing from my asylum with precaution, I sought a back-stairs which
conducted directly to the kitchen.  All in that region was fire and
commotion; the soup and fish were in the last stage of projection, and
the cook hung over her crucibles in a frame of mind and body threatening
spontaneous combustion.  In the servants' hall two coachmen and three
gentlemen's gentlemen stood or sat round the fire; the abigails, I
suppose, were upstairs with their mistresses; the new servants, that had
been hired from Millcote, were bustling about everywhere.  Threading this
chaos, I at last reached the larder; there I took possession of a cold
chicken, a roll of bread, some tarts, a plate or two and a knife and
fork: with this booty I made a hasty retreat.  I had regained the
gallery, and was just shutting the back-door behind me, when an
accelerated hum warned me that the ladies were about to issue from their
chambers.  I could not proceed to the schoolroom without passing some of
their doors, and running the risk of being surprised with my cargo of
victualage; so I stood still at this end, which, being windowless, was
dark: quite dark now, for the sun was set and twilight gathering.

Presently the chambers gave up their fair tenants one after another: each
came out gaily and airily, with dress that gleamed lustrous through the
dusk.  For a moment they stood grouped together at the other extremity of
the gallery, conversing in a key of sweet subdued vivacity: they then
descended the staircase almost as noiselessly as a bright mist rolls down
a hill.  Their collective appearance had left on me an impression of high-
born elegance, such as I had never before received.

I found Adele peeping through the schoolroom door, which she held ajar.
"What beautiful ladies!" cried she in English.  "Oh, I wish I might go to
them!  Do you think Mr. Rochester will send for us by-and-bye, after
dinner?"

"No, indeed, I don't; Mr. Rochester has something else to think about.
Never mind the ladies to-night; perhaps you will see them to-morrow: here
is your dinner."

She was really hungry, so the chicken and tarts served to divert her
attention for a time.  It was well I secured this forage, or both she, I,
and Sophie, to whom I conveyed a share of our repast, would have run a
chance of getting no dinner at all: every one downstairs was too much
engaged to think of us.  The dessert was not carried out till after nine
and at ten footmen were still running to and fro with trays and coffee-
cups.  I allowed Adele to sit up much later than usual; for she declared
she could not possibly go to sleep while the doors kept opening and
shutting below, and people bustling about.  Besides, she added, a message
might possibly come from Mr. Rochester when she was undressed; "et alors
quel dommage!"

I told her stories as long as she would listen to them; and then for a
change I took her out into the gallery.  The hall lamp was now lit, and
it amused her to look over the balustrade and watch the servants passing
backwards and forwards.  When the evening was far advanced, a sound of
music issued from the drawing-room, whither the piano had been removed;
Adele and I sat down on the top step of the stairs to listen.  Presently
a voice blent with the rich tones of the instrument; it was a lady who
sang, and very sweet her notes were.  The solo over, a duet followed, and
then a glee: a joyous conversational murmur filled up the intervals.  I
listened long: suddenly I discovered that my ear was wholly intent on
analysing the mingled sounds, and trying to discriminate amidst the
confusion of accents those of Mr. Rochester; and when it caught them,
which it soon did, it found a further task in framing the tones, rendered
by distance inarticulate, into words.

The clock struck eleven.  I looked at Adele, whose head leant against my
shoulder; her eyes were waxing heavy, so I took her up in my arms and
carried her off to bed.  It was near one before the gentlemen and ladies
sought their chambers.

The next day was as fine as its predecessor: it was devoted by the party
to an excursion to some site in the neighbourhood.  They set out early in
the forenoon, some on horseback, the rest in carriages; I witnessed both
the departure and the return.  Miss Ingram, as before, was the only lady
equestrian; and, as before, Mr. Rochester galloped at her side; the two
rode a little apart from the rest.  I pointed out this circumstance to
Mrs. Fairfax, who was standing at the window with me--

"You said it was not likely they should think of being married," said I,
"but you see Mr. Rochester evidently prefers her to any of the other
ladies."

"Yes, I daresay: no doubt he admires her."

"And she him," I added; "look how she leans her head towards him as if
she were conversing confidentially; I wish I could see her face; I have
never had a glimpse of it yet."

"You will see her this evening," answered Mrs. Fairfax.  "I happened to
remark to Mr. Rochester how much Adele wished to be introduced to the
ladies, and he said: 'Oh! let her come into the drawing-room after
dinner; and request Miss Eyre to accompany her.'"

"Yes; he said that from mere politeness: I need not go, I am sure," I
answered.

"Well, I observed to him that as you were unused to company, I did not
think you would like appearing before so gay a party--all strangers; and
he replied, in his quick way--'Nonsense!  If she objects, tell her it is
my particular wish; and if she resists, say I shall come and fetch her in
case of contumacy.'"

"I will not give him that trouble," I answered.  "I will go, if no better
may be; but I don't like it.  Shall you be there, Mrs. Fairfax?"

"No; I pleaded off, and he admitted my plea.  I'll tell you how to manage
so as to avoid the embarrassment of making a formal entrance, which is
the most disagreeable part of the business.  You must go into the drawing-
room while it is empty, before the ladies leave the dinner-table; choose
your seat in any quiet nook you like; you need not stay long after the
gentlemen come in, unless you please: just let Mr. Rochester see you are
there and then slip away--nobody will notice you."

"Will these people remain long, do you think?"

"Perhaps two or three weeks, certainly not more.  After the Easter
recess, Sir George Lynn, who was lately elected member for Millcote, will
have to go up to town and take his seat; I daresay Mr. Rochester will
accompany him: it surprises me that he has already made so protracted a
stay at Thornfield."

It was with some trepidation that I perceived the hour approach when I
was to repair with my charge to the drawing-room.  Adele had been in a
state of ecstasy all day, after hearing she was to be presented to the
ladies in the evening; and it was not till Sophie commenced the operation
of dressing her that she sobered down.  Then the importance of the
process quickly steadied her, and by the time she had her curls arranged
in well-smoothed, drooping clusters, her pink satin frock put on, her
long sash tied, and her lace mittens adjusted, she looked as grave as any
judge.  No need to warn her not to disarrange her attire: when she was
dressed, she sat demurely down in her little chair, taking care
previously to lift up the satin skirt for fear she should crease it, and
assured me she would not stir thence till I was ready.  This I quickly
was: my best dress (the silver-grey one, purchased for Miss Temple's
wedding, and never worn since) was soon put on; my hair was soon
smoothed; my sole ornament, the pearl brooch, soon assumed.  We
descended.

Fortunately there was another entrance to the drawing-room than that
through the saloon where they were all seated at dinner.  We found the
apartment vacant; a large fire burning silently on the marble hearth, and
wax candles shining in bright solitude, amid the exquisite flowers with
which the tables were adorned.  The crimson curtain hung before the arch:
slight as was the separation this drapery formed from the party in the
adjoining saloon, they spoke in so low a key that nothing of their
conversation could be distinguished beyond a soothing murmur.

Adele, who appeared to be still under the influence of a most solemnising
impression, sat down, without a word, on the footstool I pointed out to
her.  I retired to a window-seat, and taking a book from a table near,
endeavoured to read.  Adele brought her stool to my feet; ere long she
touched my knee.

"What is it, Adele?"

"Est-ce que je ne puis pas prendrie une seule de ces fleurs magnifiques,
mademoiselle?  Seulement pour completer ma toilette."

"You think too much of your 'toilette,' Adele: but you may have a
flower."  And I took a rose from a vase and fastened it in her sash.  She
sighed a sigh of ineffable satisfaction, as if her cup of happiness were
now full.  I turned my face away to conceal a smile I could not suppress:
there was something ludicrous as well as painful in the little
Parisienne's earnest and innate devotion to matters of dress.

A soft sound of rising now became audible; the curtain was swept back
from the arch; through it appeared the dining-room, with its lit lustre
pouring down light on the silver and glass of a magnificent
dessert-service covering a long table; a band of ladies stood in the
opening; they entered, and the curtain fell behind them.

There were but eight; yet, somehow, as they flocked in, they gave the
impression of a much larger number.  Some of them were very tall; many
were dressed in white; and all had a sweeping amplitude of array that
seemed to magnify their persons as a mist magnifies the moon.  I rose and
curtseyed to them: one or two bent their heads in return, the others only
stared at me.

They dispersed about the room, reminding me, by the lightness and
buoyancy of their movements, of a flock of white plumy birds.  Some of
them threw themselves in half-reclining positions on the sofas and
ottomans: some bent over the tables and examined the flowers and books:
the rest gathered in a group round the fire: all talked in a low but
clear tone which seemed habitual to them.  I knew their names afterwards,
and may as well mention them now.

First, there was Mrs. Eshton and two of her daughters.  She had evidently
been a handsome woman, and was well preserved still.  Of her daughters,
the eldest, Amy, was rather little: naive, and child-like in face and
manner, and piquant in form; her white muslin dress and blue sash became
her well.  The second, Louisa, was taller and more elegant in figure;
with a very pretty face, of that order the French term \emph{minois chiffone}:
both sisters were fair as lilies.

Lady Lynn was a large and stout personage of about forty, very erect,
very haughty-looking, richly dressed in a satin robe of changeful sheen:
her dark hair shone glossily under the shade of an azure plume, and
within the circlet of a band of gems.

Mrs. Colonel Dent was less showy; but, I thought, more lady-like.  She
had a slight figure, a pale, gentle face, and fair hair.  Her black satin
dress, her scarf of rich foreign lace, and her pearl ornaments, pleased
me better than the rainbow radiance of the titled dame.

But the three most distinguished--partly, perhaps, because the tallest
figures of the band--were the Dowager Lady Ingram and her daughters,
Blanche and Mary.  They were all three of the loftiest stature of women.
The Dowager might be between forty and fifty: her shape was still fine;
her hair (by candle-light at least) still black; her teeth, too, were
still apparently perfect.  Most people would have termed her a splendid
woman of her age: and so she was, no doubt, physically speaking; but then
there was an expression of almost insupportable haughtiness in her
bearing and countenance.  She had Roman features and a double chin,
disappearing into a throat like a pillar: these features appeared to me
not only inflated and darkened, but even furrowed with pride; and the
chin was sustained by the same principle, in a position of almost
preternatural erectness.  She had, likewise, a fierce and a hard eye: it
reminded me of Mrs. Reed's; she mouthed her words in speaking; her voice
was deep, its inflections very pompous, very dogmatical,--very
intolerable, in short.  A crimson velvet robe, and a shawl turban of some
gold-wrought Indian fabric, invested her (I suppose she thought) with a
truly imperial dignity.

Blanche and Mary were of equal stature,--straight and tall as poplars.
Mary was too slim for her height, but Blanche was moulded like a Dian.  I
regarded her, of course, with special interest.  First, I wished to see
whether her appearance accorded with Mrs. Fairfax's description;
secondly, whether it at all resembled the fancy miniature I had painted
of her; and thirdly--it will out!--whether it were such as I should fancy
likely to suit Mr. Rochester's taste.

As far as person went, she answered point for point, both to my picture
and Mrs. Fairfax's description.  The noble bust, the sloping shoulders,
the graceful neck, the dark eyes and black ringlets were all there;--but
her face?  Her face was like her mother's; a youthful unfurrowed
likeness: the same low brow, the same high features, the same pride.  It
was not, however, so saturnine a pride! she laughed continually; her
laugh was satirical, and so was the habitual expression of her arched and
haughty lip.

Genius is said to be self-conscious.  I cannot tell whether Miss Ingram
was a genius, but she was self-conscious--remarkably self-conscious
indeed.  She entered into a discourse on botany with the gentle Mrs.
Dent.  It seemed Mrs. Dent had not studied that science: though, as she
said, she liked flowers, "especially wild ones;" Miss Ingram had, and she
ran over its vocabulary with an air.  I presently perceived she was (what
is vernacularly termed) \emph{trailing} Mrs. Dent; that is, playing on her
ignorance--her \emph{trail} might be clever, but it was decidedly not good-
natured.  She played: her execution was brilliant; she sang: her voice
was fine; she talked French apart to her mamma; and she talked it well,
with fluency and with a good accent.

Mary had a milder and more open countenance than Blanche; softer features
too, and a skin some shades fairer (Miss Ingram was dark as a
Spaniard)--but Mary was deficient in life: her face lacked expression,
her eye lustre; she had nothing to say, and having once taken her seat,
remained fixed like a statue in its niche.  The sisters were both attired
in spotless white.

And did I now think Miss Ingram such a choice as Mr. Rochester would be
likely to make?  I could not tell--I did not know his taste in female
beauty.  If he liked the majestic, she was the very type of majesty: then
she was accomplished, sprightly.  Most gentlemen would admire her, I
thought; and that he \emph{did} admire her, I already seemed to have obtained
proof: to remove the last shade of doubt, it remained but to see them
together.

You are not to suppose, reader, that Adele has all this time been sitting
motionless on the stool at my feet: no; when the ladies entered, she
rose, advanced to meet them, made a stately reverence, and said with
gravity--

"Bon jour, mesdames."

And Miss Ingram had looked down at her with a mocking air, and exclaimed,
"Oh, what a little puppet!"

Lady Lynn had remarked, "It is Mr. Rochester's ward, I suppose--the
little French girl he was speaking of."

Mrs. Dent had kindly taken her hand, and given her a kiss.

Amy and Louisa Eshton had cried out simultaneously--"What a love of a
child!"

And then they had called her to a sofa, where she now sat, ensconced
between them, chattering alternately in French and broken English;
absorbing not only the young ladies' attention, but that of Mrs. Eshton
and Lady Lynn, and getting spoilt to her heart's content.

At last coffee is brought in, and the gentlemen are summoned.  I sit in
the shade--if any shade there be in this brilliantly-lit apartment; the
window-curtain half hides me.  Again the arch yawns; they come.  The
collective appearance of the gentlemen, like that of the ladies, is very
imposing: they are all costumed in black; most of them are tall, some
young.  Henry and Frederick Lynn are very dashing sparks indeed; and
Colonel Dent is a fine soldierly man.  Mr. Eshton, the magistrate of the
district, is gentleman-like: his hair is quite white, his eyebrows and
whiskers still dark, which gives him something of the appearance of a
"pere noble de theatre."  Lord Ingram, like his sisters, is very tall;
like them, also, he is handsome; but he shares Mary's apathetic and
listless look: he seems to have more length of limb than vivacity of
blood or vigour of brain.

And where is Mr. Rochester?

He comes in last: I am not looking at the arch, yet I see him enter.  I
try to concentrate my attention on those netting-needles, on the meshes
of the purse I am forming--I wish to think only of the work I have in my
hands, to see only the silver beads and silk threads that lie in my lap;
whereas, I distinctly behold his figure, and I inevitably recall the
moment when I last saw it; just after I had rendered him, what he deemed,
an essential service, and he, holding my hand, and looking down on my
face, surveyed me with eyes that revealed a heart full and eager to
overflow; in whose emotions I had a part.  How near had I approached him
at that moment!  What had occurred since, calculated to change his and my
relative positions?  Yet now, how distant, how far estranged we were!  So
far estranged, that I did not expect him to come and speak to me.  I did
not wonder, when, without looking at me, he took a seat at the other side
of the room, and began conversing with some of the ladies.

No sooner did I see that his attention was riveted on them, and that I
might gaze without being observed, than my eyes were drawn involuntarily
to his face; I could not keep their lids under control: they would rise,
and the irids would fix on him.  I looked, and had an acute pleasure in
looking,--a precious yet poignant pleasure; pure gold, with a steely
point of agony: a pleasure like what the thirst-perishing man might feel
who knows the well to which he has crept is poisoned, yet stoops and
drinks divine draughts nevertheless.

Most true is it that "beauty is in the eye of the gazer."  My master's
colourless, olive face, square, massive brow, broad and jetty eyebrows,
deep eyes, strong features, firm, grim mouth,--all energy, decision,
will,--were not beautiful, according to rule; but they were more than
beautiful to me; they were full of an interest, an influence that quite
mastered me,--that took my feelings from my own power and fettered them
in his.  I had not intended to love him; the reader knows I had wrought
hard to extirpate from my soul the germs of love there detected; and now,
at the first renewed view of him, they spontaneously arrived, green and
strong!  He made me love him without looking at me.

I compared him with his guests.  What was the gallant grace of the Lynns,
the languid elegance of Lord Ingram,--even the military distinction of
Colonel Dent, contrasted with his look of native pith and genuine power?
I had no sympathy in their appearance, their expression: yet I could
imagine that most observers would call them attractive, handsome,
imposing; while they would pronounce Mr. Rochester at once harsh-featured
and melancholy-looking.  I saw them smile, laugh--it was nothing; the
light of the candles had as much soul in it as their smile; the tinkle of
the bell as much significance as their laugh.  I saw Mr. Rochester
smile:--his stern features softened; his eye grew both brilliant and
gentle, its ray both searching and sweet.  He was talking, at the moment,
to Louisa and Amy Eshton.  I wondered to see them receive with calm that
look which seemed to me so penetrating: I expected their eyes to fall,
their colour to rise under it; yet I was glad when I found they were in
no sense moved.  "He is not to them what he is to me," I thought: "he is
not of their kind.  I believe he is of mine;--I am sure he is--I feel
akin to him--I understand the language of his countenance and movements:
though rank and wealth sever us widely, I have something in my brain and
heart, in my blood and nerves, that assimilates me mentally to him.  Did
I say, a few days since, that I had nothing to do with him but to receive
my salary at his hands?  Did I forbid myself to think of him in any other
light than as a paymaster?  Blasphemy against nature!  Every good, true,
vigorous feeling I have gathers impulsively round him.  I know I must
conceal my sentiments: I must smother hope; I must remember that he
cannot care much for me.  For when I say that I am of his kind, I do not
mean that I have his force to influence, and his spell to attract; I mean
only that I have certain tastes and feelings in common with him.  I must,
then, repeat continually that we are for ever sundered:--and yet, while I
breathe and think, I must love him."

Coffee is handed.  The ladies, since the gentlemen entered, have become
lively as larks; conversation waxes brisk and merry.  Colonel Dent and
Mr. Eshton argue on politics; their wives listen.  The two proud
dowagers, Lady Lynn and Lady Ingram, confabulate together.  Sir
George--whom, by-the-bye, I have forgotten to describe,--a very big, and
very fresh-looking country gentleman, stands before their sofa, coffee-
cup in hand, and occasionally puts in a word.  Mr. Frederick Lynn has
taken a seat beside Mary Ingram, and is showing her the engravings of a
splendid volume: she looks, smiles now and then, but apparently says
little.  The tall and phlegmatic Lord Ingram leans with folded arms on
the chair-back of the little and lively Amy Eshton; she glances up at
him, and chatters like a wren: she likes him better than she does Mr.
Rochester.  Henry Lynn has taken possession of an ottoman at the feet of
Louisa: Adele shares it with him: he is trying to talk French with her,
and Louisa laughs at his blunders.  With whom will Blanche Ingram pair?
She is standing alone at the table, bending gracefully over an album.  She
seems waiting to be sought; but she will not wait too long: she herself
selects a mate.

Mr. Rochester, having quitted the Eshtons, stands on the hearth as
solitary as she stands by the table: she confronts him, taking her
station on the opposite side of the mantelpiece.

"Mr. Rochester, I thought you were not fond of children?"

"Nor am I."

"Then, what induced you to take charge of such a little doll as that?"
(pointing to Adele).  "Where did you pick her up?"

"I did not pick her up; she was left on my hands."

"You should have sent her to school."

"I could not afford it: schools are so dear."

"Why, I suppose you have a governess for her: I saw a person with her
just now--is she gone?  Oh, no! there she is still, behind the window-
curtain.  You pay her, of course; I should think it quite as
expensive,--more so; for you have them both to keep in addition."

I feared--or should I say, hoped?--the allusion to me would make Mr.
Rochester glance my way; and I involuntarily shrank farther into the
shade: but he never turned his eyes.

"I have not considered the subject," said he indifferently, looking
straight before him.

"No, you men never do consider economy and common sense.  You should hear
mama on the chapter of governesses: Mary and I have had, I should think,
a dozen at least in our day; half of them detestable and the rest
ridiculous, and all incubi--were they not, mama?"

"Did you speak, my own?"

The young lady thus claimed as the dowager's special property, reiterated
her question with an explanation.

"My dearest, don't mention governesses; the word makes me nervous.  I
have suffered a martyrdom from their incompetency and caprice.  I thank
Heaven I have now done with them!"

Mrs. Dent here bent over to the pious lady and whispered something in her
ear; I suppose, from the answer elicited, it was a reminder that one of
the anathematised race was present.

"Tant pis!" said her Ladyship, "I hope it may do her good!"  Then, in a
lower tone, but still loud enough for me to hear, "I noticed her; I am a
judge of physiognomy, and in hers I see all the faults of her class."

"What are they, madam?" inquired Mr. Rochester aloud.

"I will tell you in your private ear," replied she, wagging her turban
three times with portentous significancy.

"But my curiosity will be past its appetite; it craves food now."

"Ask Blanche; she is nearer you than I."

"Oh, don't refer him to me, mama!  I have just one word to say of the
whole tribe; they are a nuisance.  Not that I ever suffered much from
them; I took care to turn the tables.  What tricks Theodore and I used to
play on our Miss Wilsons, and Mrs. Greys, and Madame Jouberts!  Mary was
always too sleepy to join in a plot with spirit.  The best fun was with
Madame Joubert: Miss Wilson was a poor sickly thing, lachrymose and low-
spirited, not worth the trouble of vanquishing, in short; and Mrs. Grey
was coarse and insensible; no blow took effect on her.  But poor Madame
Joubert!  I see her yet in her raging passions, when we had driven her to
extremities--spilt our tea, crumbled our bread and butter, tossed our
books up to the ceiling, and played a charivari with the ruler and desk,
the fender and fire-irons.  Theodore, do you remember those merry days?"

"Yaas, to be sure I do," drawled Lord Ingram; "and the poor old stick
used to cry out 'Oh you villains childs!'--and then we sermonised her on
the presumption of attempting to teach such clever blades as we were,
when she was herself so ignorant."

"We did; and, Tedo, you know, I helped you in prosecuting (or
persecuting) your tutor, whey-faced Mr. Vining--the parson in the pip, as
we used to call him.  He and Miss Wilson took the liberty of falling in
love with each other--at least Tedo and I thought so; we surprised sundry
tender glances and sighs which we interpreted as tokens of 'la belle
passion,' and I promise you the public soon had the benefit of our
discovery; we employed it as a sort of lever to hoist our dead-weights
from the house.  Dear mama, there, as soon as she got an inkling of the
business, found out that it was of an immoral tendency.  Did you not, my
lady-mother?"

"Certainly, my best.  And I was quite right: depend on that: there are a
thousand reasons why liaisons between governesses and tutors should never
be tolerated a moment in any well-regulated house; firstly--"

"Oh, gracious, mama!  Spare us the enumeration!  \emph{Au reste}, we all know
them: danger of bad example to innocence of childhood; distractions and
consequent neglect of duty on the part of the attached--mutual alliance
and reliance; confidence thence resulting--insolence accompanying--mutiny
and general blow-up.  Am I right, Baroness Ingram, of Ingram Park?"

"My lily-flower, you are right now, as always."

"Then no more need be said: change the subject."

Amy Eshton, not hearing or not heeding this dictum, joined in with her
soft, infantine tone: "Louisa and I used to quiz our governess too; but
she was such a good creature, she would bear anything: nothing put her
out.  She was never cross with us; was she, Louisa?"

"No, never: we might do what we pleased; ransack her desk and her
workbox, and turn her drawers inside out; and she was so good-natured,
she would give us anything we asked for."

"I suppose, now," said Miss Ingram, curling her lip sarcastically, "we
shall have an abstract of the memoirs of all the governesses extant: in
order to avert such a visitation, I again move the introduction of a new
topic.  Mr. Rochester, do you second my motion?"

"Madam, I support you on this point, as on every other."

"Then on me be the onus of bringing it forward.  Signior Eduardo, are you
in voice to-night?"

"Donna Bianca, if you command it, I will be."

"Then, signior, I lay on you my sovereign behest to furbish up your lungs
and other vocal organs, as they will be wanted on my royal service."

"Who would not be the Rizzio of so divine a Mary?"

"A fig for Rizzio!" cried she, tossing her head with all its curls, as
she moved to the piano.  "It is my opinion the fiddler David must have
been an insipid sort of fellow; I like black Bothwell better: to my mind
a man is nothing without a spice of the devil in him; and history may say
what it will of James Hepburn, but I have a notion, he was just the sort
of wild, fierce, bandit hero whom I could have consented to gift with my
hand."

"Gentlemen, you hear!  Now which of you most resembles Bothwell?" cried
Mr. Rochester.

"I should say the preference lies with you," responded Colonel Dent.

"On my honour, I am much obliged to you," was the reply.

Miss Ingram, who had now seated herself with proud grace at the piano,
spreading out her snowy robes in queenly amplitude, commenced a brilliant
prelude; talking meantime.  She appeared to be on her high horse
to-night; both her words and her air seemed intended to excite not only
the admiration, but the amazement of her auditors: she was evidently bent
on striking them as something very dashing and daring indeed.

"Oh, I am so sick of the young men of the present day!" exclaimed she,
rattling away at the instrument.  "Poor, puny things, not fit to stir a
step beyond papa's park gates: nor to go even so far without mama's
permission and guardianship!  Creatures so absorbed in care about their
pretty faces, and their white hands, and their small feet; as if a man
had anything to do with beauty!  As if loveliness were not the special
prerogative of woman--her legitimate appanage and heritage!  I grant an
ugly \emph{woman} is a blot on the fair face of creation; but as to the
\emph{gentlemen}, let them be solicitous to possess only strength and valour:
let their motto be:--Hunt, shoot, and fight: the rest is not worth a
fillip.  Such should be my device, were I a man."

"Whenever I marry," she continued after a pause which none interrupted,
"I am resolved my husband shall not be a rival, but a foil to me.  I will
suffer no competitor near the throne; I shall exact an undivided homage:
his devotions shall not be shared between me and the shape he sees in his
mirror.  Mr. Rochester, now sing, and I will play for you."

"I am all obedience," was the response.

"Here then is a Corsair-song.  Know that I doat on Corsairs; and for that
reason, sing it \emph{con spirito}."

"Commands from Miss Ingram's lips would put spirit into a mug of milk and
water."

"Take care, then: if you don't please me, I will shame you by showing how
such things \emph{should} be done."

"That is offering a premium on incapacity: I shall now endeavour to
fail."

"Gardez-vous en bien!  If you err wilfully, I shall devise a
proportionate punishment."

"Miss Ingram ought to be clement, for she has it in her power to inflict
a chastisement beyond mortal endurance."

"Ha! explain!" commanded the lady.

"Pardon me, madam: no need of explanation; your own fine sense must
inform you that one of your frowns would be a sufficient substitute for
capital punishment."

"Sing!" said she, and again touching the piano, she commenced an
accompaniment in spirited style.

"Now is my time to slip away," thought I: but the tones that then severed
the air arrested me.  Mrs. Fairfax had said Mr. Rochester possessed a
fine voice: he did--a mellow, powerful bass, into which he threw his own
feeling, his own force; finding a way through the ear to the heart, and
there waking sensation strangely.  I waited till the last deep and full
vibration had expired--till the tide of talk, checked an instant, had
resumed its flow; I then quitted my sheltered corner and made my exit by
the side-door, which was fortunately near.  Thence a narrow passage led
into the hall: in crossing it, I perceived my sandal was loose; I stopped
to tie it, kneeling down for that purpose on the mat at the foot of the
staircase.  I heard the dining-room door unclose; a gentleman came out;
rising hastily, I stood face to face with him: it was Mr. Rochester.

"How do you do?" he asked.

"I am very well, sir."

"Why did you not come and speak to me in the room?"

I thought I might have retorted the question on him who put it: but I
would not take that freedom.  I answered--

"I did not wish to disturb you, as you seemed engaged, sir."

"What have you been doing during my absence?"

"Nothing particular; teaching Adele as usual."

"And getting a good deal paler than you were--as I saw at first sight.
What is the matter?"

"Nothing at all, sir."

"Did you take any cold that night you half drowned me?"

"Not the least."

"Return to the drawing-room: you are deserting too early."

"I am tired, sir."

He looked at me for a minute.

"And a little depressed," he said.  "What about?  Tell me."

"Nothing--nothing, sir.  I am not depressed."

"But I affirm that you are: so much depressed that a few more words would
bring tears to your eyes--indeed, they are there now, shining and
swimming; and a bead has slipped from the lash and fallen on to the flag.
If I had time, and was not in mortal dread of some prating prig of a
servant passing, I would know what all this means.  Well, to-night I
excuse you; but understand that so long as my visitors stay, I expect you
to appear in the drawing-room every evening; it is my wish; don't neglect
it.  Now go, and send Sophie for Adele.  Good-night, my--"  He stopped,
bit his lip, and abruptly left me.




\chapter{}


\lettrine{M}{erry} days were these at Thornfield Hall; and busy days too: how
different from the first three months of stillness, monotony, and
solitude I had passed beneath its roof!  All sad feelings seemed now
driven from the house, all gloomy associations forgotten: there was life
everywhere, movement all day long.  You could not now traverse the
gallery, once so hushed, nor enter the front chambers, once so
tenantless, without encountering a smart lady's-maid or a dandy valet.

The kitchen, the butler's pantry, the servants' hall, the entrance hall,
were equally alive; and the saloons were only left void and still when
the blue sky and halcyon sunshine of the genial spring weather called
their occupants out into the grounds.  Even when that weather was broken,
and continuous rain set in for some days, no damp seemed cast over
enjoyment: indoor amusements only became more lively and varied, in
consequence of the stop put to outdoor gaiety.

I wondered what they were going to do the first evening a change of
entertainment was proposed: they spoke of "playing charades," but in my
ignorance I did not understand the term.  The servants were called in,
the dining-room tables wheeled away, the lights otherwise disposed, the
chairs placed in a semicircle opposite the arch.  While Mr. Rochester and
the other gentlemen directed these alterations, the ladies were running
up and down stairs ringing for their maids.  Mrs. Fairfax was summoned to
give information respecting the resources of the house in shawls,
dresses, draperies of any kind; and certain wardrobes of the third storey
were ransacked, and their contents, in the shape of brocaded and hooped
petticoats, satin sacques, black modes, lace lappets, \&c., were brought
down in armfuls by the abigails; then a selection was made, and such
things as were chosen were carried to the boudoir within the
drawing-room.

Meantime, Mr. Rochester had again summoned the ladies round him, and was
selecting certain of their number to be of his party.  "Miss Ingram is
mine, of course," said he: afterwards he named the two Misses Eshton, and
Mrs. Dent.  He looked at me: I happened to be near him, as I had been
fastening the clasp of Mrs. Dent's bracelet, which had got loose.

"Will you play?" he asked.  I shook my head.  He did not insist, which I
rather feared he would have done; he allowed me to return quietly to my
usual seat.

He and his aids now withdrew behind the curtain: the other party, which
was headed by Colonel Dent, sat down on the crescent of chairs.  One of
the gentlemen, Mr. Eshton, observing me, seemed to propose that I should
be asked to join them; but Lady Ingram instantly negatived the notion.

"No," I heard her say: "she looks too stupid for any game of the sort."

Ere long a bell tinkled, and the curtain drew up.  Within the arch, the
bulky figure of Sir George Lynn, whom Mr. Rochester had likewise chosen,
was seen enveloped in a white sheet: before him, on a table, lay open a
large book; and at his side stood Amy Eshton, draped in Mr. Rochester's
cloak, and holding a book in her hand.  Somebody, unseen, rang the bell
merrily; then Adele (who had insisted on being one of her guardian's
party), bounded forward, scattering round her the contents of a basket of
flowers she carried on her arm.  Then appeared the magnificent figure of
Miss Ingram, clad in white, a long veil on her head, and a wreath of
roses round her brow; by her side walked Mr. Rochester, and together they
drew near the table.  They knelt; while Mrs. Dent and Louisa Eshton,
dressed also in white, took up their stations behind them.  A ceremony
followed, in dumb show, in which it was easy to recognise the pantomime
of a marriage.  At its termination, Colonel Dent and his party consulted
in whispers for two minutes, then the Colonel called out--

"Bride!" Mr. Rochester bowed, and the curtain fell.

A considerable interval elapsed before it again rose.  Its second rising
displayed a more elaborately prepared scene than the last.  The drawing-
room, as I have before observed, was raised two steps above the dining-
room, and on the top of the upper step, placed a yard or two back within
the room, appeared a large marble basin--which I recognised as an
ornament of the conservatory--where it usually stood, surrounded by
exotics, and tenanted by gold fish--and whence it must have been
transported with some trouble, on account of its size and weight.

Seated on the carpet, by the side of this basin, was seen Mr. Rochester,
costumed in shawls, with a turban on his head.  His dark eyes and swarthy
skin and Paynim features suited the costume exactly: he looked the very
model of an Eastern emir, an agent or a victim of the bowstring.
Presently advanced into view Miss Ingram.  She, too, was attired in
oriental fashion: a crimson scarf tied sash-like round the waist: an
embroidered handkerchief knotted about her temples; her
beautifully-moulded arms bare, one of them upraised in the act of
supporting a pitcher, poised gracefully on her head.  Both her cast of
form and feature, her complexion and her general air, suggested the idea
of some Israelitish princess of the patriarchal days; and such was
doubtless the character she intended to represent.

She approached the basin, and bent over it as if to fill her pitcher; she
again lifted it to her head.  The personage on the well-brink now seemed
to accost her; to make some request:--"She hasted, let down her pitcher
on her hand, and gave him to drink."  From the bosom of his robe he then
produced a casket, opened it and showed magnificent bracelets and
earrings; she acted astonishment and admiration; kneeling, he laid the
treasure at her feet; incredulity and delight were expressed by her looks
and gestures; the stranger fastened the bracelets on her arms and the
rings in her ears.  It was Eliezer and Rebecca: the camels only were
wanting.

The divining party again laid their heads together: apparently they could
not agree about the word or syllable the scene illustrated.  Colonel
Dent, their spokesman, demanded "the tableau of the whole;" whereupon the
curtain again descended.

On its third rising only a portion of the drawing-room was disclosed; the
rest being concealed by a screen, hung with some sort of dark and coarse
drapery.  The marble basin was removed; in its place, stood a deal table
and a kitchen chair: these objects were visible by a very dim light
proceeding from a horn lantern, the wax candles being all extinguished.

Amidst this sordid scene, sat a man with his clenched hands resting on
his knees, and his eyes bent on the ground.  I knew Mr. Rochester; though
the begrimed face, the disordered dress (his coat hanging loose from one
arm, as if it had been almost torn from his back in a scuffle), the
desperate and scowling countenance, the rough, bristling hair might well
have disguised him.  As he moved, a chain clanked; to his wrists were
attached fetters.

"Bridewell!" exclaimed Colonel Dent, and the charade was solved.

A sufficient interval having elapsed for the performers to resume their
ordinary costume, they re-entered the dining-room.  Mr. Rochester led in
Miss Ingram; she was complimenting him on his acting.

"Do you know," said she, "that, of the three characters, I liked you in
the last best?  Oh, had you but lived a few years earlier, what a gallant
gentleman-highwayman you would have made!"

"Is all the soot washed from my face?" he asked, turning it towards her.

"Alas! yes: the more's the pity!  Nothing could be more becoming to your
complexion than that ruffian's rouge."

"You would like a hero of the road then?"

"An English hero of the road would be the next best thing to an Italian
bandit; and that could only be surpassed by a Levantine pirate."

"Well, whatever I am, remember you are my wife; we were married an hour
since, in the presence of all these witnesses."  She giggled, and her
colour rose.

"Now, Dent," continued Mr. Rochester, "it is your turn."  And as the
other party withdrew, he and his band took the vacated seats.  Miss
Ingram placed herself at her leader's right hand; the other diviners
filled the chairs on each side of him and her.  I did not now watch the
actors; I no longer waited with interest for the curtain to rise; my
attention was absorbed by the spectators; my eyes, erewhile fixed on the
arch, were now irresistibly attracted to the semicircle of chairs.  What
charade Colonel Dent and his party played, what word they chose, how they
acquitted themselves, I no longer remember; but I still see the
consultation which followed each scene: I see Mr. Rochester turn to Miss
Ingram, and Miss Ingram to him; I see her incline her head towards him,
till the jetty curls almost touch his shoulder and wave against his
cheek; I hear their mutual whisperings; I recall their interchanged
glances; and something even of the feeling roused by the spectacle
returns in memory at this moment.

I have told you, reader, that I had learnt to love Mr. Rochester: I could
not unlove him now, merely because I found that he had ceased to notice
me--because I might pass hours in his presence, and he would never once
turn his eyes in my direction--because I saw all his attentions
appropriated by a great lady, who scorned to touch me with the hem of her
robes as she passed; who, if ever her dark and imperious eye fell on me
by chance, would withdraw it instantly as from an object too mean to
merit observation.  I could not unlove him, because I felt sure he would
soon marry this very lady--because I read daily in her a proud security
in his intentions respecting her--because I witnessed hourly in him a
style of courtship which, if careless and choosing rather to be sought
than to seek, was yet, in its very carelessness, captivating, and in its
very pride, irresistible.

There was nothing to cool or banish love in these circumstances, though
much to create despair.  Much too, you will think, reader, to engender
jealousy: if a woman, in my position, could presume to be jealous of a
woman in Miss Ingram's.  But I was not jealous: or very rarely;--the
nature of the pain I suffered could not be explained by that word.  Miss
Ingram was a mark beneath jealousy: she was too inferior to excite the
feeling.  Pardon the seeming paradox; I mean what I say.  She was very
showy, but she was not genuine: she had a fine person, many brilliant
attainments; but her mind was poor, her heart barren by nature: nothing
bloomed spontaneously on that soil; no unforced natural fruit delighted
by its freshness.  She was not good; she was not original: she used to
repeat sounding phrases from books: she never offered, nor had, an
opinion of her own.  She advocated a high tone of sentiment; but she did
not know the sensations of sympathy and pity; tenderness and truth were
not in her.  Too often she betrayed this, by the undue vent she gave to a
spiteful antipathy she had conceived against little Adele: pushing her
away with some contumelious epithet if she happened to approach her;
sometimes ordering her from the room, and always treating her with
coldness and acrimony.  Other eyes besides mine watched these
manifestations of character--watched them closely, keenly, shrewdly.  Yes;
the future bridegroom, Mr. Rochester himself, exercised over his intended
a ceaseless surveillance; and it was from this sagacity--this guardedness
of his--this perfect, clear consciousness of his fair one's defects--this
obvious absence of passion in his sentiments towards her, that my ever-
torturing pain arose.

I saw he was going to marry her, for family, perhaps political reasons,
because her rank and connections suited him; I felt he had not given her
his love, and that her qualifications were ill adapted to win from him
that treasure.  This was the point--this was where the nerve was touched
and teased--this was where the fever was sustained and fed: \emph{she could
not charm him}.

If she had managed the victory at once, and he had yielded and sincerely
laid his heart at her feet, I should have covered my face, turned to the
wall, and (figuratively) have died to them.  If Miss Ingram had been a
good and noble woman, endowed with force, fervour, kindness, sense, I
should have had one vital struggle with two tigers--jealousy and despair:
then, my heart torn out and devoured, I should have admired
her--acknowledged her excellence, and been quiet for the rest of my days:
and the more absolute her superiority, the deeper would have been my
admiration--the more truly tranquil my quiescence.  But as matters really
stood, to watch Miss Ingram's efforts at fascinating Mr. Rochester, to
witness their repeated failure--herself unconscious that they did fail;
vainly fancying that each shaft launched hit the mark, and infatuatedly
pluming herself on success, when her pride and self-complacency repelled
further and further what she wished to allure--to witness \emph{this}, was to
be at once under ceaseless excitation and ruthless restraint.

Because, when she failed, I saw how she might have succeeded.  Arrows
that continually glanced off from Mr. Rochester's breast and fell
harmless at his feet, might, I knew, if shot by a surer hand, have
quivered keen in his proud heart--have called love into his stern eye,
and softness into his sardonic face; or, better still, without weapons a
silent conquest might have been won.

"Why can she not influence him more, when she is privileged to draw so
near to him?" I asked myself.  "Surely she cannot truly like him, or not
like him with true affection!  If she did, she need not coin her smiles
so lavishly, flash her glances so unremittingly, manufacture airs so
elaborate, graces so multitudinous.  It seems to me that she might, by
merely sitting quietly at his side, saying little and looking less, get
nigher his heart.  I have seen in his face a far different expression
from that which hardens it now while she is so vivaciously accosting him;
but then it came of itself: it was not elicited by meretricious arts and
calculated manoeuvres; and one had but to accept it--to answer what he
asked without pretension, to address him when needful without grimace--and
it increased and grew kinder and more genial, and warmed one like a
fostering sunbeam.  How will she manage to please him when they are
married?  I do not think she will manage it; and yet it might be managed;
and his wife might, I verily believe, be the very happiest woman the sun
shines on."

I have not yet said anything condemnatory of Mr. Rochester's project of
marrying for interest and connections.  It surprised me when I first
discovered that such was his intention: I had thought him a man unlikely
to be influenced by motives so commonplace in his choice of a wife; but
the longer I considered the position, education, \&c., of the parties, the
less I felt justified in judging and blaming either him or Miss Ingram
for acting in conformity to ideas and principles instilled into them,
doubtless, from their childhood.  All their class held these principles:
I supposed, then, they had reasons for holding them such as I could not
fathom.  It seemed to me that, were I a gentleman like him, I would take
to my bosom only such a wife as I could love; but the very obviousness of
the advantages to the husband's own happiness offered by this plan
convinced me that there must be arguments against its general adoption of
which I was quite ignorant: otherwise I felt sure all the world would act
as I wished to act.

But in other points, as well as this, I was growing very lenient to my
master: I was forgetting all his faults, for which I had once kept a
sharp look-out.  It had formerly been my endeavour to study all sides of
his character: to take the bad with the good; and from the just weighing
of both, to form an equitable judgment.  Now I saw no bad.  The sarcasm
that had repelled, the harshness that had startled me once, were only
like keen condiments in a choice dish: their presence was pungent, but
their absence would be felt as comparatively insipid.  And as for the
vague something--was it a sinister or a sorrowful, a designing or a
desponding expression?--that opened upon a careful observer, now and
then, in his eye, and closed again before one could fathom the strange
depth partially disclosed; that something which used to make me fear and
shrink, as if I had been wandering amongst volcanic-looking hills, and
had suddenly felt the ground quiver and seen it gape: that something, I,
at intervals, beheld still; and with throbbing heart, but not with
palsied nerves.  Instead of wishing to shun, I longed only to dare--to
divine it; and I thought Miss Ingram happy, because one day she might
look into the abyss at her leisure, explore its secrets and analyse their
nature.

Meantime, while I thought only of my master and his future bride--saw
only them, heard only their discourse, and considered only their
movements of importance--the rest of the party were occupied with their
own separate interests and pleasures.  The Ladies Lynn and Ingram
continued to consort in solemn conferences, where they nodded their two
turbans at each other, and held up their four hands in confronting
gestures of surprise, or mystery, or horror, according to the theme on
which their gossip ran, like a pair of magnified puppets.  Mild Mrs. Dent
talked with good-natured Mrs. Eshton; and the two sometimes bestowed a
courteous word or smile on me.  Sir George Lynn, Colonel Dent, and Mr.
Eshton discussed politics, or county affairs, or justice business.  Lord
Ingram flirted with Amy Eshton; Louisa played and sang to and with one of
the Messrs. Lynn; and Mary Ingram listened languidly to the gallant
speeches of the other.  Sometimes all, as with one consent, suspended
their by-play to observe and listen to the principal actors: for, after
all, Mr. Rochester and--because closely connected with him--Miss Ingram
were the life and soul of the party.  If he was absent from the room an
hour, a perceptible dulness seemed to steal over the spirits of his
guests; and his re-entrance was sure to give a fresh impulse to the
vivacity of conversation.

The want of his animating influence appeared to be peculiarly felt one
day that he had been summoned to Millcote on business, and was not likely
to return till late.  The afternoon was wet: a walk the party had
proposed to take to see a gipsy camp, lately pitched on a common beyond
Hay, was consequently deferred.  Some of the gentlemen were gone to the
stables: the younger ones, together with the younger ladies, were playing
billiards in the billiard-room.  The dowagers Ingram and Lynn sought
solace in a quiet game at cards.  Blanche Ingram, after having repelled,
by supercilious taciturnity, some efforts of Mrs. Dent and Mrs. Eshton to
draw her into conversation, had first murmured over some sentimental
tunes and airs on the piano, and then, having fetched a novel from the
library, had flung herself in haughty listlessness on a sofa, and
prepared to beguile, by the spell of fiction, the tedious hours of
absence.  The room and the house were silent: only now and then the
merriment of the billiard-players was heard from above.

It was verging on dusk, and the clock had already given warning of the
hour to dress for dinner, when little Adele, who knelt by me in the
drawing-room window-seat, suddenly exclaimed--

"Voila, Monsieur Rochester, qui revient!"

I turned, and Miss Ingram darted forwards from her sofa: the others, too,
looked up from their several occupations; for at the same time a
crunching of wheels and a splashing tramp of horse-hoofs became audible
on the wet gravel.  A post-chaise was approaching.

"What can possess him to come home in that style?" said Miss Ingram.  "He
rode Mesrour (the black horse), did he not, when he went out? and Pilot
was with him:--what has he done with the animals?"

As she said this, she approached her tall person and ample garments so
near the window, that I was obliged to bend back almost to the breaking
of my spine: in her eagerness she did not observe me at first, but when
she did, she curled her lip and moved to another casement.  The
post-chaise stopped; the driver rang the door-bell, and a gentleman
alighted attired in travelling garb; but it was not Mr. Rochester; it was
a tall, fashionable-looking man, a stranger.

"How provoking!" exclaimed Miss Ingram: "you tiresome monkey!"
(apostrophising Adele), "who perched you up in the window to give false
intelligence?" and she cast on me an angry glance, as if I were in fault.

Some parleying was audible in the hall, and soon the new-comer entered.
He bowed to Lady Ingram, as deeming her the eldest lady present.

"It appears I come at an inopportune time, madam," said he, "when my
friend, Mr. Rochester, is from home; but I arrive from a very long
journey, and I think I may presume so far on old and intimate
acquaintance as to instal myself here till he returns."

His manner was polite; his accent, in speaking, struck me as being
somewhat unusual,--not precisely foreign, but still not altogether
English: his age might be about Mr. Rochester's,--between thirty and
forty; his complexion was singularly sallow: otherwise he was a
fine-looking man, at first sight especially.  On closer examination, you
detected something in his face that displeased, or rather that failed to
please.  His features were regular, but too relaxed: his eye was large
and well cut, but the life looking out of it was a tame, vacant life--at
least so I thought.

The sound of the dressing-bell dispersed the party.  It was not till
after dinner that I saw him again: he then seemed quite at his ease.  But
I liked his physiognomy even less than before: it struck me as being at
the same time unsettled and inanimate.  His eye wandered, and had no
meaning in its wandering: this gave him an odd look, such as I never
remembered to have seen.  For a handsome and not an unamiable-looking
man, he repelled me exceedingly: there was no power in that
smooth-skinned face of a full oval shape: no firmness in that aquiline
nose and small cherry mouth; there was no thought on the low, even
forehead; no command in that blank, brown eye.

As I sat in my usual nook, and looked at him with the light of the
girandoles on the mantelpiece beaming full over him--for he occupied an
arm-chair drawn close to the fire, and kept shrinking still nearer, as if
he were cold, I compared him with Mr. Rochester.  I think (with deference
be it spoken) the contrast could not be much greater between a sleek
gander and a fierce falcon: between a meek sheep and the rough-coated
keen-eyed dog, its guardian.

He had spoken of Mr. Rochester as an old friend.  A curious friendship
theirs must have been: a pointed illustration, indeed, of the old adage
that "extremes meet."

Two or three of the gentlemen sat near him, and I caught at times scraps
of their conversation across the room.  At first I could not make much
sense of what I heard; for the discourse of Louisa Eshton and Mary
Ingram, who sat nearer to me, confused the fragmentary sentences that
reached me at intervals.  These last were discussing the stranger; they
both called him "a beautiful man."  Louisa said he was "a love of a
creature," and she "adored him;" and Mary instanced his "pretty little
mouth, and nice nose," as her ideal of the charming.

"And what a sweet-tempered forehead he has!" cried Louisa,--"so
smooth--none of those frowning irregularities I dislike so much; and such
a placid eye and smile!"

And then, to my great relief, Mr. Henry Lynn summoned them to the other
side of the room, to settle some point about the deferred excursion to
Hay Common.

I was now able to concentrate my attention on the group by the fire, and
I presently gathered that the new-comer was called Mr. Mason; then I
learned that he was but just arrived in England, and that he came from
some hot country: which was the reason, doubtless, his face was so
sallow, and that he sat so near the hearth, and wore a surtout in the
house.  Presently the words Jamaica, Kingston, Spanish Town, indicated
the West Indies as his residence; and it was with no little surprise I
gathered, ere long, that he had there first seen and become acquainted
with Mr. Rochester.  He spoke of his friend's dislike of the burning
heats, the hurricanes, and rainy seasons of that region.  I knew Mr.
Rochester had been a traveller: Mrs. Fairfax had said so; but I thought
the continent of Europe had bounded his wanderings; till now I had never
heard a hint given of visits to more distant shores.

I was pondering these things, when an incident, and a somewhat unexpected
one, broke the thread of my musings.  Mr. Mason, shivering as some one
chanced to open the door, asked for more coal to be put on the fire,
which had burnt out its flame, though its mass of cinder still shone hot
and red.  The footman who brought the coal, in going out, stopped near
Mr. Eshton's chair, and said something to him in a low voice, of which I
heard only the words, "old woman,"--"quite troublesome."

"Tell her she shall be put in the stocks if she does not take herself
off," replied the magistrate.

"No--stop!" interrupted Colonel Dent.  "Don't send her away, Eshton; we
might turn the thing to account; better consult the ladies."  And
speaking aloud, he continued--"Ladies, you talked of going to Hay Common
to visit the gipsy camp; Sam here says that one of the old Mother Bunches
is in the servants' hall at this moment, and insists upon being brought
in before 'the quality,' to tell them their fortunes.  Would you like to
see her?"

"Surely, colonel," cried Lady Ingram, "you would not encourage such a low
impostor?  Dismiss her, by all means, at once!"

"But I cannot persuade her to go away, my lady," said the footman; "nor
can any of the servants: Mrs. Fairfax is with her just now, entreating
her to be gone; but she has taken a chair in the chimney-corner, and says
nothing shall stir her from it till she gets leave to come in here."

"What does she want?" asked Mrs. Eshton.

"'To tell the gentry their fortunes,' she says, ma'am; and she swears she
must and will do it."

"What is she like?" inquired the Misses Eshton, in a breath.

"A shockingly ugly old creature, miss; almost as black as a crock."

"Why, she's a real sorceress!" cried Frederick Lynn.  "Let us have her
in, of course."

"To be sure," rejoined his brother; "it would be a thousand pities to
throw away such a chance of fun."

"My dear boys, what are you thinking about?" exclaimed Mrs. Lynn.

"I cannot possibly countenance any such inconsistent proceeding," chimed
in the Dowager Ingram.

"Indeed, mama, but you can--and will," pronounced the haughty voice of
Blanche, as she turned round on the piano-stool; where till now she had
sat silent, apparently examining sundry sheets of music.  "I have a
curiosity to hear my fortune told: therefore, Sam, order the beldame
forward."

"My darling Blanche! recollect--"

"I do--I recollect all you can suggest; and I must have my will--quick,
Sam!"

"Yes--yes--yes!" cried all the juveniles, both ladies and gentlemen.  "Let
her come--it will be excellent sport!"

The footman still lingered.  "She looks such a rough one," said he.

"Go!" ejaculated Miss Ingram, and the man went.

Excitement instantly seized the whole party: a running fire of raillery
and jests was proceeding when Sam returned.

"She won't come now," said he.  "She says it's not her mission to appear
before the 'vulgar herd' (them's her words).  I must show her into a room
by herself, and then those who wish to consult her must go to her one by
one."

"You see now, my queenly Blanche," began Lady Ingram, "she encroaches.  Be
advised, my angel girl--and--"

"Show her into the library, of course," cut in the "angel girl."  "It is
not my mission to listen to her before the vulgar herd either: I mean to
have her all to myself.  Is there a fire in the library?"

"Yes, ma'am--but she looks such a tinkler."

"Cease that chatter, blockhead! and do my bidding."

Again Sam vanished; and mystery, animation, expectation rose to full flow
once more.

"She's ready now," said the footman, as he reappeared.  "She wishes to
know who will be her first visitor."

"I think I had better just look in upon her before any of the ladies go,"
said Colonel Dent.

"Tell her, Sam, a gentleman is coming."

Sam went and returned.

"She says, sir, that she'll have no gentlemen; they need not trouble
themselves to come near her; nor," he added, with difficulty suppressing
a titter, "any ladies either, except the young, and single."

"By Jove, she has taste!" exclaimed Henry Lynn.

Miss Ingram rose solemnly: "I go first," she said, in a tone which might
have befitted the leader of a forlorn hope, mounting a breach in the van
of his men.

"Oh, my best! oh, my dearest! pause--reflect!" was her mama's cry; but
she swept past her in stately silence, passed through the door which
Colonel Dent held open, and we heard her enter the library.

A comparative silence ensued.  Lady Ingram thought it "le cas" to wring
her hands: which she did accordingly.  Miss Mary declared she felt, for
her part, she never dared venture.  Amy and Louisa Eshton tittered under
their breath, and looked a little frightened.

The minutes passed very slowly: fifteen were counted before the library-
door again opened.  Miss Ingram returned to us through the arch.

Would she laugh?  Would she take it as a joke?  All eyes met her with a
glance of eager curiosity, and she met all eyes with one of rebuff and
coldness; she looked neither flurried nor merry: she walked stiffly to
her seat, and took it in silence.

"Well, Blanche?" said Lord Ingram.

"What did she say, sister?" asked Mary.

"What did you think?  How do you feel?--Is she a real fortune-teller?"
demanded the Misses Eshton.

"Now, now, good people," returned Miss Ingram, "don't press upon me.
Really your organs of wonder and credulity are easily excited: you seem,
by the importance of you all--my good mama included--ascribe to this
matter, absolutely to believe we have a genuine witch in the house, who
is in close alliance with the old gentleman.  I have seen a gipsy
vagabond; she has practised in hackneyed fashion the science of palmistry
and told me what such people usually tell.  My whim is gratified; and now
I think Mr. Eshton will do well to put the hag in the stocks to-morrow
morning, as he threatened."

Miss Ingram took a book, leant back in her chair, and so declined further
conversation.  I watched her for nearly half-an-hour: during all that
time she never turned a page, and her face grew momently darker, more
dissatisfied, and more sourly expressive of disappointment.  She had
obviously not heard anything to her advantage: and it seemed to me, from
her prolonged fit of gloom and taciturnity, that she herself,
notwithstanding her professed indifference, attached undue importance to
whatever revelations had been made her.

{During all that time she never turned a page: p184.jpg}

Meantime, Mary Ingram, Amy and Louisa Eshton, declared they dared not go
alone; and yet they all wished to go.  A negotiation was opened through
the medium of the ambassador, Sam; and after much pacing to and fro,
till, I think, the said Sam's calves must have ached with the exercise,
permission was at last, with great difficulty, extorted from the rigorous
Sibyl, for the three to wait upon her in a body.

Their visit was not so still as Miss Ingram's had been: we heard
hysterical giggling and little shrieks proceeding from the library; and
at the end of about twenty minutes they burst the door open, and came
running across the hall, as if they were half-scared out of their wits.

"I am sure she is something not right!" they cried, one and all.  "She
told us such things!  She knows all about us!" and they sank breathless
into the various seats the gentlemen hastened to bring them.

Pressed for further explanation, they declared she had told them of
things they had said and done when they were mere children; described
books and ornaments they had in their boudoirs at home: keepsakes that
different relations had presented to them.  They affirmed that she had
even divined their thoughts, and had whispered in the ear of each the
name of the person she liked best in the world, and informed them of what
they most wished for.

Here the gentlemen interposed with earnest petitions to be further
enlightened on these two last-named points; but they got only blushes,
ejaculations, tremors, and titters, in return for their importunity.  The
matrons, meantime, offered vinaigrettes and wielded fans; and again and
again reiterated the expression of their concern that their warning had
not been taken in time; and the elder gentlemen laughed, and the younger
urged their services on the agitated fair ones.

In the midst of the tumult, and while my eyes and ears were fully engaged
in the scene before me, I heard a hem close at my elbow: I turned, and
saw Sam.

"If you please, miss, the gipsy declares that there is another young
single lady in the room who has not been to her yet, and she swears she
will not go till she has seen all.  I thought it must be you: there is no
one else for it.  What shall I tell her?"

"Oh, I will go by all means," I answered: and I was glad of the
unexpected opportunity to gratify my much-excited curiosity.  I slipped
out of the room, unobserved by any eye--for the company were gathered in
one mass about the trembling trio just returned--and I closed the door
quietly behind me.

"If you like, miss," said Sam, "I'll wait in the hall for you; and if she
frightens you, just call and I'll come in."

"No, Sam, return to the kitchen: I am not in the least afraid."  Nor was
I; but I was a good deal interested and excited.




\chapter{}


\lettrine{T}{he} library looked tranquil enough as I entered it, and the Sibyl--if
Sibyl she were--was seated snugly enough in an easy-chair at the chimney-
corner.  She had on a red cloak and a black bonnet: or rather, a broad-
brimmed gipsy hat, tied down with a striped handkerchief under her chin.
An extinguished candle stood on the table; she was bending over the fire,
and seemed reading in a little black book, like a prayer-book, by the
light of the blaze: she muttered the words to herself, as most old women
do, while she read; she did not desist immediately on my entrance: it
appeared she wished to finish a paragraph.

I stood on the rug and warmed my hands, which were rather cold with
sitting at a distance from the drawing-room fire.  I felt now as composed
as ever I did in my life: there was nothing indeed in the gipsy's
appearance to trouble one's calm.  She shut her book and slowly looked
up; her hat-brim partially shaded her face, yet I could see, as she
raised it, that it was a strange one.  It looked all brown and black: elf-
locks bristled out from beneath a white band which passed under her chin,
and came half over her cheeks, or rather jaws: her eye confronted me at
once, with a bold and direct gaze.

"Well, and you want your fortune told?" she said, in a voice as decided
as her glance, as harsh as her features.

"I don't care about it, mother; you may please yourself: but I ought to
warn you, I have no faith."

"It's like your impudence to say so: I expected it of you; I heard it in
your step as you crossed the threshold."

"Did you?  You've a quick ear."

"I have; and a quick eye and a quick brain."

"You need them all in your trade."

"I do; especially when I've customers like you to deal with.  Why don't
you tremble?"

"I'm not cold."

"Why don't you turn pale?"

"I am not sick."

"Why don't you consult my art?"

"I'm not silly."

The old crone "nichered" a laugh under her bonnet and bandage; she then
drew out a short black pipe, and lighting it began to smoke.  Having
indulged a while in this sedative, she raised her bent body, took the
pipe from her lips, and while gazing steadily at the fire, said very
deliberately--"You are cold; you are sick; and you are silly."

"Prove it," I rejoined.

"I will, in few words.  You are cold, because you are alone: no contact
strikes the fire from you that is in you.  You are sick; because the best
of feelings, the highest and the sweetest given to man, keeps far away
from you.  You are silly, because, suffer as you may, you will not beckon
it to approach, nor will you stir one step to meet it where it waits
you."

She again put her short black pipe to her lips, and renewed her smoking
with vigour.

"You might say all that to almost any one who you knew lived as a
solitary dependent in a great house."

"I might say it to almost any one: but would it be true of almost any
one?"

"In my circumstances."

"Yes; just so, in \emph{your} circumstances: but find me another precisely
placed as you are."

"It would be easy to find you thousands."

"You could scarcely find me one.  If you knew it, you are peculiarly
situated: very near happiness; yes, within reach of it.  The materials
are all prepared; there only wants a movement to combine them.  Chance
laid them somewhat apart; let them be once approached and bliss results."

"I don't understand enigmas.  I never could guess a riddle in my life."

"If you wish me to speak more plainly, show me your palm."

"And I must cross it with silver, I suppose?"

"To be sure."

I gave her a shilling: she put it into an old stocking-foot which she
took out of her pocket, and having tied it round and returned it, she
told me to hold out my hand.  I did.  She approached her face to the
palm, and pored over it without touching it.

"It is too fine," said she.  "I can make nothing of such a hand as that;
almost without lines: besides, what is in a palm?  Destiny is not written
there."

"I believe you," said I.

"No," she continued, "it is in the face: on the forehead, about the eyes,
in the lines of the mouth.  Kneel, and lift up your head."

"Ah! now you are coming to reality," I said, as I obeyed her.  "I shall
begin to put some faith in you presently."

I knelt within half a yard of her.  She stirred the fire, so that a
ripple of light broke from the disturbed coal: the glare, however, as she
sat, only threw her face into deeper shadow: mine, it illumined.

"I wonder with what feelings you came to me to-night," she said, when she
had examined me a while.  "I wonder what thoughts are busy in your heart
during all the hours you sit in yonder room with the fine people flitting
before you like shapes in a magic-lantern: just as little sympathetic
communion passing between you and them as if they were really mere
shadows of human forms, and not the actual substance."

"I feel tired often, sleepy sometimes, but seldom sad."

"Then you have some secret hope to buoy you up and please you with
whispers of the future?"

"Not I.  The utmost I hope is, to save money enough out of my earnings to
set up a school some day in a little house rented by myself."

"A mean nutriment for the spirit to exist on: and sitting in that window-
seat (you see I know your habits )--"

"You have learned them from the servants."

"Ah! you think yourself sharp.  Well, perhaps I have: to speak truth, I
have an acquaintance with one of them, Mrs. Poole--"

I started to my feet when I heard the name.

"You have--have you?" thought I; "there is diablerie in the business
after all, then!"

"Don't be alarmed," continued the strange being; "she's a safe hand is
Mrs. Poole: close and quiet; any one may repose confidence in her.  But,
as I was saying: sitting in that window-seat, do you think of nothing but
your future school?  Have you no present interest in any of the company
who occupy the sofas and chairs before you?  Is there not one face you
study? one figure whose movements you follow with at least curiosity?"

"I like to observe all the faces and all the figures."

"But do you never single one from the rest--or it may be, two?"

"I do frequently; when the gestures or looks of a pair seem telling a
tale: it amuses me to watch them."

"What tale do you like best to hear?"

"Oh, I have not much choice!  They generally run on the same
theme--courtship; and promise to end in the same catastrophe--marriage."

"And do you like that monotonous theme?"

"Positively, I don't care about it: it is nothing to me."

"Nothing to you?  When a lady, young and full of life and health,
charming with beauty and endowed with the gifts of rank and fortune, sits
and smiles in the eyes of a gentleman you--"

"I what?"

"You know--and perhaps think well of."

"I don't know the gentlemen here.  I have scarcely interchanged a
syllable with one of them; and as to thinking well of them, I consider
some respectable, and stately, and middle-aged, and others young,
dashing, handsome, and lively: but certainly they are all at liberty to
be the recipients of whose smiles they please, without my feeling
disposed to consider the transaction of any moment to me."

"You don't know the gentlemen here?  You have not exchanged a syllable
with one of them?  Will you say that of the master of the house!"

"He is not at home."

"A profound remark!  A most ingenious quibble!  He went to Millcote this
morning, and will be back here to-night or to-morrow: does that
circumstance exclude him from the list of your acquaintance--blot him, as
it were, out of existence?"

"No; but I can scarcely see what Mr. Rochester has to do with the theme
you had introduced."

"I was talking of ladies smiling in the eyes of gentlemen; and of late so
many smiles have been shed into Mr. Rochester's eyes that they overflow
like two cups filled above the brim: have you never remarked that?"

"Mr. Rochester has a right to enjoy the society of his guests."

"No question about his right: but have you never observed that, of all
the tales told here about matrimony, Mr. Rochester has been favoured with
the most lively and the most continuous?"

"The eagerness of a listener quickens the tongue of a narrator."  I said
this rather to myself than to the gipsy, whose strange talk, voice,
manner, had by this time wrapped me in a kind of dream.  One unexpected
sentence came from her lips after another, till I got involved in a web
of mystification; and wondered what unseen spirit had been sitting for
weeks by my heart watching its workings and taking record of every pulse.

"Eagerness of a listener!" repeated she: "yes; Mr. Rochester has sat by
the hour, his ear inclined to the fascinating lips that took such delight
in their task of communicating; and Mr. Rochester was so willing to
receive and looked so grateful for the pastime given him; you have
noticed this?"

"Grateful!  I cannot remember detecting gratitude in his face."

"Detecting!  You have analysed, then.  And what did you detect, if not
gratitude?"

I said nothing.

"You have seen love: have you not?--and, looking forward, you have seen
him married, and beheld his bride happy?"

"Humph!  Not exactly.  Your witch's skill is rather at fault sometimes."

"What the devil have you seen, then?"

"Never mind: I came here to inquire, not to confess.  Is it known that
Mr. Rochester is to be married?"

"Yes; and to the beautiful Miss Ingram."

"Shortly?"

"Appearances would warrant that conclusion: and, no doubt (though, with
an audacity that wants chastising out of you, you seem to question it),
they will be a superlatively happy pair.  He must love such a handsome,
noble, witty, accomplished lady; and probably she loves him, or, if not
his person, at least his purse.  I know she considers the Rochester
estate eligible to the last degree; though (God pardon me!) I told her
something on that point about an hour ago which made her look wondrous
grave: the corners of her mouth fell half an inch.  I would advise her
blackaviced suitor to look out: if another comes, with a longer or
clearer rent-roll,--he's dished--"

"But, mother, I did not come to hear Mr. Rochester's fortune: I came to
hear my own; and you have told me nothing of it."

"Your fortune is yet doubtful: when I examined your face, one trait
contradicted another.  Chance has meted you a measure of happiness: that
I know.  I knew it before I came here this evening.  She has laid it
carefully on one side for you.  I saw her do it.  It depends on yourself
to stretch out your hand, and take it up: but whether you will do so, is
the problem I study.  Kneel again on the rug."

"Don't keep me long; the fire scorches me."

{She did not stoop towards me, but only gazed, leaning back in her chair:
p190.jpg}

I knelt.  She did not stoop towards me, but only gazed, leaning back in
her chair.  She began muttering,--

"The flame flickers in the eye; the eye shines like dew; it looks soft
and full of feeling; it smiles at my jargon: it is susceptible;
impression follows impression through its clear sphere; where it ceases
to smile, it is sad; an unconscious lassitude weighs on the lid: that
signifies melancholy resulting from loneliness.  It turns from me; it
will not suffer further scrutiny; it seems to deny, by a mocking glance,
the truth of the discoveries I have already made,--to disown the charge
both of sensibility and chagrin: its pride and reserve only confirm me in
my opinion.  The eye is favourable.

"As to the mouth, it delights at times in laughter; it is disposed to
impart all that the brain conceives; though I daresay it would be silent
on much the heart experiences.  Mobile and flexible, it was never
intended to be compressed in the eternal silence of solitude: it is a
mouth which should speak much and smile often, and have human affection
for its interlocutor.  That feature too is propitious.

"I see no enemy to a fortunate issue but in the brow; and that brow
professes to say,--'I can live alone, if self-respect, and circumstances
require me so to do.  I need not sell my soul to buy bliss.  I have an
inward treasure born with me, which can keep me alive if all extraneous
delights should be withheld, or offered only at a price I cannot afford
to give.'  The forehead declares, 'Reason sits firm and holds the reins,
and she will not let the feelings burst away and hurry her to wild
chasms.  The passions may rage furiously, like true heathens, as they
are; and the desires may imagine all sorts of vain things: but judgment
shall still have the last word in every argument, and the casting vote in
every decision.  Strong wind, earthquake-shock, and fire may pass by: but
I shall follow the guiding of that still small voice which interprets the
dictates of conscience.'

"Well said, forehead; your declaration shall be respected.  I have formed
my plans--right plans I deem them--and in them I have attended to the
claims of conscience, the counsels of reason.  I know how soon youth
would fade and bloom perish, if, in the cup of bliss offered, but one
dreg of shame, or one flavour of remorse were detected; and I do not want
sacrifice, sorrow, dissolution--such is not my taste.  I wish to foster,
not to blight--to earn gratitude, not to wring tears of blood--no, nor of
brine: my harvest must be in smiles, in endearments, in sweet--That will
do.  I think I rave in a kind of exquisite delirium.  I should wish now
to protract this moment \emph{ad infinitum}; but I dare not.  So far I have
governed myself thoroughly.  I have acted as I inwardly swore I would
act; but further might try me beyond my strength.  Rise, Miss Eyre: leave
me; the play is played out'."

Where was I?  Did I wake or sleep?  Had I been dreaming?  Did I dream
still?  The old woman's voice had changed: her accent, her gesture, and
all were familiar to me as my own face in a glass--as the speech of my
own tongue.  I got up, but did not go.  I looked; I stirred the fire, and
I looked again: but she drew her bonnet and her bandage closer about her
face, and again beckoned me to depart.  The flame illuminated her hand
stretched out: roused now, and on the alert for discoveries, I at once
noticed that hand.  It was no more the withered limb of eld than my own;
it was a rounded supple member, with smooth fingers, symmetrically
turned; a broad ring flashed on the little finger, and stooping forward,
I looked at it, and saw a gem I had seen a hundred times before.  Again I
looked at the face; which was no longer turned from me--on the contrary,
the bonnet was doffed, the bandage displaced, the head advanced.

"Well, Jane, do you know me?" asked the familiar voice.

"Only take off the red cloak, sir, and then--"

"But the string is in a knot--help me."

"Break it, sir."

"There, then--'Off, ye lendings!'"  And Mr. Rochester stepped out of his
disguise.

"Now, sir, what a strange idea!"

"But well carried out, eh?  Don't you think so?"

"With the ladies you must have managed well."

"But not with you?"

"You did not act the character of a gipsy with me."

"What character did I act?  My own?"

"No; some unaccountable one.  In short, I believe you have been trying to
draw me out--or in; you have been talking nonsense to make me talk
nonsense.  It is scarcely fair, sir."

"Do you forgive me, Jane?"

"I cannot tell till I have thought it all over.  If, on reflection, I
find I have fallen into no great absurdity, I shall try to forgive you;
but it was not right."

"Oh, you have been very correct--very careful, very sensible."

I reflected, and thought, on the whole, I had.  It was a comfort; but,
indeed, I had been on my guard almost from the beginning of the
interview.  Something of masquerade I suspected.  I knew gipsies and
fortune-tellers did not express themselves as this seeming old woman had
expressed herself; besides I had noted her feigned voice, her anxiety to
conceal her features.  But my mind had been running on Grace Poole--that
living enigma, that mystery of mysteries, as I considered her.  I had
never thought of Mr. Rochester.

"Well," said he, "what are you musing about?  What does that grave smile
signify?"

"Wonder and self-congratulation, sir.  I have your permission to retire
now, I suppose?"

"No; stay a moment; and tell me what the people in the drawing-room
yonder are doing."

"Discussing the gipsy, I daresay."

"Sit down!--Let me hear what they said about me."

"I had better not stay long, sir; it must be near eleven o'clock.  Oh,
are you aware, Mr. Rochester, that a stranger has arrived here since you
left this morning?"

"A stranger!--no; who can it be?  I expected no one; is he gone?"

"No; he said he had known you long, and that he could take the liberty of
installing himself here till you returned."

"The devil he did!  Did he give his name?"

"His name is Mason, sir; and he comes from the West Indies; from Spanish
Town, in Jamaica, I think."

Mr. Rochester was standing near me; he had taken my hand, as if to lead
me to a chair.  As I spoke he gave my wrist a convulsive grip; the smile
on his lips froze: apparently a spasm caught his breath.

"Mason!--the West Indies!" he said, in the tone one might fancy a
speaking automaton to enounce its single words; "Mason!--the West
Indies!" he reiterated; and he went over the syllables three times,
growing, in the intervals of speaking, whiter than ashes: he hardly
seemed to know what he was doing.

"Do you feel ill, sir?" I inquired.

"Jane, I've got a blow; I've got a blow, Jane!"  He staggered.

"Oh, lean on me, sir."

"Jane, you offered me your shoulder once before; let me have it now."

"Yes, sir, yes; and my arm."

He sat down, and made me sit beside him.  Holding my hand in both his
own, he chafed it; gazing on me, at the same time, with the most troubled
and dreary look.

"My little friend!" said he, "I wish I were in a quiet island with only
you; and trouble, and danger, and hideous recollections removed from me."

"Can I help you, sir?--I'd give my life to serve you."

"Jane, if aid is wanted, I'll seek it at your hands; I promise you that."

"Thank you, sir.  Tell me what to do,--I'll try, at least, to do it."

"Fetch me now, Jane, a glass of wine from the dining-room: they will be
at supper there; and tell me if Mason is with them, and what he is
doing."

I went.  I found all the party in the dining-room at supper, as Mr.
Rochester had said; they were not seated at table,--the supper was
arranged on the sideboard; each had taken what he chose, and they stood
about here and there in groups, their plates and glasses in their hands.
Every one seemed in high glee; laughter and conversation were general and
animated.  Mr. Mason stood near the fire, talking to Colonel and Mrs.
Dent, and appeared as merry as any of them.  I filled a wine-glass (I saw
Miss Ingram watch me frowningly as I did so: she thought I was taking a
liberty, I daresay), and I returned to the library.

Mr. Rochester's extreme pallor had disappeared, and he looked once more
firm and stern.  He took the glass from my hand.

"Here is to your health, ministrant spirit!" he said.  He swallowed the
contents and returned it to me.  "What are they doing, Jane?"

"Laughing and talking, sir."

"They don't look grave and mysterious, as if they had heard something
strange?"

"Not at all: they are full of jests and gaiety."

"And Mason?"

"He was laughing too."

"If all these people came in a body and spat at me, what would you do,
Jane?"

"Turn them out of the room, sir, if I could."

He half smiled.  "But if I were to go to them, and they only looked at me
coldly, and whispered sneeringly amongst each other, and then dropped off
and left me one by one, what then?  Would you go with them?"

"I rather think not, sir: I should have more pleasure in staying with
you."

"To comfort me?"

"Yes, sir, to comfort you, as well as I could."

"And if they laid you under a ban for adhering to me?"

"I, probably, should know nothing about their ban; and if I did, I should
care nothing about it."

"Then, you could dare censure for my sake?"

"I could dare it for the sake of any friend who deserved my adherence; as
you, I am sure, do."

"Go back now into the room; step quietly up to Mason, and whisper in his
ear that Mr. Rochester is come and wishes to see him: show him in here
and then leave me."

"Yes, sir."

I did his behest.  The company all stared at me as I passed straight
among them.  I sought Mr. Mason, delivered the message, and preceded him
from the room: I ushered him into the library, and then I went upstairs.

At a late hour, after I had been in bed some time, I heard the visitors
repair to their chambers: I distinguished Mr. Rochester's voice, and
heard him say, "This way, Mason; this is your room."

He spoke cheerfully: the gay tones set my heart at ease.  I was soon
asleep.




\chapter{}


\lettrine{I} had forgotten to draw my curtain, which I usually did, and also to let
down my window-blind.  The consequence was, that when the moon, which was
full and bright (for the night was fine), came in her course to that
space in the sky opposite my casement, and looked in at me through the
unveiled panes, her glorious gaze roused me.  Awaking in the dead of
night, I opened my eyes on her disk--silver-white and crystal clear.  It
was beautiful, but too solemn; I half rose, and stretched my arm to draw
the curtain.

Good God!  What a cry!

The night--its silence--its rest, was rent in twain by a savage, a sharp,
a shrilly sound that ran from end to end of Thornfield Hall.

My pulse stopped: my heart stood still; my stretched arm was paralysed.
The cry died, and was not renewed.  Indeed, whatever being uttered that
fearful shriek could not soon repeat it: not the widest-winged condor on
the Andes could, twice in succession, send out such a yell from the cloud
shrouding his eyrie.  The thing delivering such utterance must rest ere
it could repeat the effort.

It came out of the third storey; for it passed overhead.  And
overhead--yes, in the room just above my chamber-ceiling--I now heard a
struggle: a deadly one it seemed from the noise; and a half-smothered
voice shouted--

"Help! help! help!" three times rapidly.

"Will no one come?" it cried; and then, while the staggering and stamping
went on wildly, I distinguished through plank and plaster:--

"Rochester!  Rochester! for God's sake, come!"

A chamber-door opened: some one ran, or rushed, along the gallery.
Another step stamped on the flooring above and something fell; and there
was silence.

I had put on some clothes, though horror shook all my limbs; I issued
from my apartment.  The sleepers were all aroused: ejaculations,
terrified murmurs sounded in every room; door after door unclosed; one
looked out and another looked out; the gallery filled.  Gentlemen and
ladies alike had quitted their beds; and "Oh! what is it?"--"Who is
hurt?"--"What has happened?"--"Fetch a light!"--"Is it fire?"--"Are there
robbers?"--"Where shall we run?" was demanded confusedly on all hands.
But for the moonlight they would have been in complete darkness.  They
ran to and fro; they crowded together: some sobbed, some stumbled: the
confusion was inextricable.

"Where the devil is Rochester?" cried Colonel Dent.  "I cannot find him
in his bed."

"Here! here!" was shouted in return.  "Be composed, all of you: I'm
coming."

And the door at the end of the gallery opened, and Mr. Rochester advanced
with a candle: he had just descended from the upper storey.  One of the
ladies ran to him directly; she seized his arm: it was Miss Ingram.

"What awful event has taken place?" said she.  "Speak! let us know the
worst at once!"

"But don't pull me down or strangle me," he replied: for the Misses
Eshton were clinging about him now; and the two dowagers, in vast white
wrappers, were bearing down on him like ships in full sail.

"All's right!--all's right!" he cried.  "It's a mere rehearsal of Much
Ado about Nothing.  Ladies, keep off, or I shall wax dangerous."

And dangerous he looked: his black eyes darted sparks.  Calming himself
by an effort, he added--

"A servant has had the nightmare; that is all.  She's an excitable,
nervous person: she construed her dream into an apparition, or something
of that sort, no doubt; and has taken a fit with fright.  Now, then, I
must see you all back into your rooms; for, till the house is settled,
she cannot be looked after.  Gentlemen, have the goodness to set the
ladies the example.  Miss Ingram, I am sure you will not fail in evincing
superiority to idle terrors.  Amy and Louisa, return to your nests like a
pair of doves, as you are.   Mesdames" (to the dowagers), "you will take
cold to a dead certainty, if you stay in this chill gallery any longer."

And so, by dint of alternate coaxing and commanding, he contrived to get
them all once more enclosed in their separate dormitories.  I did not
wait to be ordered back to mine, but retreated unnoticed, as unnoticed I
had left it.

Not, however, to go to bed: on the contrary, I began and dressed myself
carefully.  The sounds I had heard after the scream, and the words that
had been uttered, had probably been heard only by me; for they had
proceeded from the room above mine: but they assured me that it was not a
servant's dream which had thus struck horror through the house; and that
the explanation Mr. Rochester had given was merely an invention framed to
pacify his guests.  I dressed, then, to be ready for emergencies.  When
dressed, I sat a long time by the window looking out over the silent
grounds and silvered fields and waiting for I knew not what.  It seemed
to me that some event must follow the strange cry, struggle, and call.

No: stillness returned: each murmur and movement ceased gradually, and in
about an hour Thornfield Hall was again as hushed as a desert.  It seemed
that sleep and night had resumed their empire.  Meantime the moon
declined: she was about to set.  Not liking to sit in the cold and
darkness, I thought I would lie down on my bed, dressed as I was.  I left
the window, and moved with little noise across the carpet; as I stooped
to take off my shoes, a cautious hand tapped low at the door.

"Am I wanted?" I asked.

"Are you up?" asked the voice I expected to hear, viz., my master's.

"Yes, sir."

"And dressed?"

"Yes."

"Come out, then, quietly."

I obeyed.  Mr. Rochester stood in the gallery holding a light.

"I want you," he said: "come this way: take your time, and make no
noise."

My slippers were thin: I could walk the matted floor as softly as a cat.
He glided up the gallery and up the stairs, and stopped in the dark, low
corridor of the fateful third storey: I had followed and stood at his
side.

"Have you a sponge in your room?" he asked in a whisper.

"Yes, sir."

"Have you any salts--volatile salts?"

"Yes."

"Go back and fetch both."

I returned, sought the sponge on the washstand, the salts in my drawer,
and once more retraced my steps.  He still waited; he held a key in his
hand: approaching one of the small, black doors, he put it in the lock;
he paused, and addressed me again.

"You don't turn sick at the sight of blood?"

"I think I shall not: I have never been tried yet."

I felt a thrill while I answered him; but no coldness, and no faintness.

"Just give me your hand," he said: "it will not do to risk a fainting
fit."

I put my fingers into his.  "Warm and steady," was his remark: he turned
the key and opened the door.

I saw a room I remembered to have seen before, the day Mrs. Fairfax
showed me over the house: it was hung with tapestry; but the tapestry was
now looped up in one part, and there was a door apparent, which had then
been concealed.  This door was open; a light shone out of the room
within: I heard thence a snarling, snatching sound, almost like a dog
quarrelling.  Mr. Rochester, putting down his candle, said to me, "Wait a
minute," and he went forward to the inner apartment.  A shout of laughter
greeted his entrance; noisy at first, and terminating in Grace Poole's
own goblin ha! ha!  \emph{She} then was there.  He made some sort of
arrangement without speaking, though I heard a low voice address him: he
came out and closed the door behind him.

"Here, Jane!" he said; and I walked round to the other side of a large
bed, which with its drawn curtains concealed a considerable portion of
the chamber.  An easy-chair was near the bed-head: a man sat in it,
dressed with the exception of his coat; he was still; his head leant
back; his eyes were closed.  Mr. Rochester held the candle over him; I
recognised in his pale and seemingly lifeless face--the stranger, Mason:
I saw too that his linen on one side, and one arm, was almost soaked in
blood.

"Hold the candle," said Mr. Rochester, and I took it: he fetched a basin
of water from the washstand: "Hold that," said he.  I obeyed.  He took
the sponge, dipped it in, and moistened the corpse-like face; he asked
for my smelling-bottle, and applied it to the nostrils.  Mr. Mason
shortly unclosed his eyes; he groaned.  Mr. Rochester opened the shirt of
the wounded man, whose arm and shoulder were bandaged: he sponged away
blood, trickling fast down.

"Is there immediate danger?" murmured Mr. Mason.

"Pooh!  No--a mere scratch.  Don't be so overcome, man: bear up!  I'll
fetch a surgeon for you now, myself: you'll be able to be removed by
morning, I hope.  Jane," he continued.

"Sir?"

"I shall have to leave you in this room with this gentleman, for an hour,
or perhaps two hours: you will sponge the blood as I do when it returns:
if he feels faint, you will put the glass of water on that stand to his
lips, and your salts to his nose.  You will not speak to him on any
pretext--and--Richard, it will be at the peril of your life if you speak
to her: open your lips--agitate yourself--and I'll not answer for the
consequences."

Again the poor man groaned; he looked as if he dared not move; fear,
either of death or of something else, appeared almost to paralyse him.
Mr. Rochester put the now bloody sponge into my hand, and I proceeded to
use it as he had done.  He watched me a second, then saying,
"Remember!--No conversation," he left the room.  I experienced a strange
feeling as the key grated in the lock, and the sound of his retreating
step ceased to be heard.

Here then I was in the third storey, fastened into one of its mystic
cells; night around me; a pale and bloody spectacle under my eyes and
hands; a murderess hardly separated from me by a single door: yes--that
was appalling--the rest I could bear; but I shuddered at the thought of
Grace Poole bursting out upon me.

I must keep to my post, however.  I must watch this ghastly
countenance--these blue, still lips forbidden to unclose--these eyes now
shut, now opening, now wandering through the room, now fixing on me, and
ever glazed with the dulness of horror.  I must dip my hand again and
again in the basin of blood and water, and wipe away the trickling gore.
I must see the light of the unsnuffed candle wane on my employment; the
shadows darken on the wrought, antique tapestry round me, and grow black
under the hangings of the vast old bed, and quiver strangely over the
doors of a great cabinet opposite--whose front, divided into twelve
panels, bore, in grim design, the heads of the twelve apostles, each
enclosed in its separate panel as in a frame; while above them at the top
rose an ebon crucifix and a dying Christ.

According as the shifting obscurity and flickering gleam hovered here or
glanced there, it was now the bearded physician, Luke, that bent his
brow; now St. John's long hair that waved; and anon the devilish face of
Judas, that grew out of the panel, and seemed gathering life and
threatening a revelation of the arch-traitor--of Satan himself--in his
subordinate's form.

Amidst all this, I had to listen as well as watch: to listen for the
movements of the wild beast or the fiend in yonder side den.  But since
Mr. Rochester's visit it seemed spellbound: all the night I heard but
three sounds at three long intervals,--a step creak, a momentary renewal
of the snarling, canine noise, and a deep human groan.

Then my own thoughts worried me.  What crime was this that lived
incarnate in this sequestered mansion, and could neither be expelled nor
subdued by the owner?--what mystery, that broke out now in fire and now
in blood, at the deadest hours of night?  What creature was it, that,
masked in an ordinary woman's face and shape, uttered the voice, now of a
mocking demon, and anon of a carrion-seeking bird of prey?

And this man I bent over--this commonplace, quiet stranger--how had he
become involved in the web of horror? and why had the Fury flown at him?
What made him seek this quarter of the house at an untimely season, when
he should have been asleep in bed?  I had heard Mr. Rochester assign him
an apartment below--what brought him here!  And why, now, was he so tame
under the violence or treachery done him?  Why did he so quietly submit
to the concealment Mr. Rochester enforced?  Why \emph{did} Mr. Rochester
enforce this concealment?  His guest had been outraged, his own life on a
former occasion had been hideously plotted against; and both attempts he
smothered in secrecy and sank in oblivion!  Lastly, I saw Mr. Mason was
submissive to Mr. Rochester; that the impetuous will of the latter held
complete sway over the inertness of the former: the few words which had
passed between them assured me of this.  It was evident that in their
former intercourse, the passive disposition of the one had been
habitually influenced by the active energy of the other: whence then had
arisen Mr. Rochester's dismay when he heard of Mr. Mason's arrival?  Why
had the mere name of this unresisting individual--whom his word now
sufficed to control like a child--fallen on him, a few hours since, as a
thunderbolt might fall on an oak?

Oh!  I could not forget his look and his paleness when he whispered:
"Jane, I have got a blow--I have got a blow, Jane."  I could not forget
how the arm had trembled which he rested on my shoulder: and it was no
light matter which could thus bow the resolute spirit and thrill the
vigorous frame of Fairfax Rochester.

"When will he come?  When will he come?" I cried inwardly, as the night
lingered and lingered--as my bleeding patient drooped, moaned, sickened:
and neither day nor aid arrived.  I had, again and again, held the water
to Mason's white lips; again and again offered him the stimulating salts:
my efforts seemed ineffectual: either bodily or mental suffering, or loss
of blood, or all three combined, were fast prostrating his strength.  He
moaned so, and looked so weak, wild, and lost, I feared he was dying; and
I might not even speak to him.

The candle, wasted at last, went out; as it expired, I perceived streaks
of grey light edging the window curtains: dawn was then approaching.
Presently I heard Pilot bark far below, out of his distant kennel in the
courtyard: hope revived.  Nor was it unwarranted: in five minutes more
the grating key, the yielding lock, warned me my watch was relieved.  It
could not have lasted more than two hours: many a week has seemed
shorter.

Mr. Rochester entered, and with him the surgeon he had been to fetch.

"Now, Carter, be on the alert," he said to this last: "I give you but
half-an-hour for dressing the wound, fastening the bandages, getting the
patient downstairs and all."

"But is he fit to move, sir?"

"No doubt of it; it is nothing serious; he is nervous, his spirits must
be kept up.  Come, set to work."

Mr. Rochester drew back the thick curtain, drew up the holland blind, let
in all the daylight he could; and I was surprised and cheered to see how
far dawn was advanced: what rosy streaks were beginning to brighten the
east.  Then he approached Mason, whom the surgeon was already handling.

"Now, my good fellow, how are you?" he asked.

"She's done for me, I fear," was the faint reply.

"Not a whit!--courage!  This day fortnight you'll hardly be a pin the
worse of it: you've lost a little blood; that's all.  Carter, assure him
there's no danger."

"I can do that conscientiously," said Carter, who had now undone the
bandages; "only I wish I could have got here sooner: he would not have
bled so much--but how is this?  The flesh on the shoulder is torn as well
as cut.  This wound was not done with a knife: there have been teeth
here!"

"She bit me," he murmured.  "She worried me like a tigress, when
Rochester got the knife from her."

"You should not have yielded: you should have grappled with her at once,"
said Mr. Rochester.

"But under such circumstances, what could one do?" returned Mason.  "Oh,
it was frightful!" he added, shuddering.  "And I did not expect it: she
looked so quiet at first."

"I warned you," was his friend's answer; "I said--be on your guard when
you go near her.  Besides, you might have waited till to-morrow, and had
me with you: it was mere folly to attempt the interview to-night, and
alone."

"I thought I could have done some good."

"You thought! you thought!  Yes, it makes me impatient to hear you: but,
however, you have suffered, and are likely to suffer enough for not
taking my advice; so I'll say no more.  Carter--hurry!--hurry!  The sun
will soon rise, and I must have him off."

"Directly, sir; the shoulder is just bandaged.  I must look to this other
wound in the arm: she has had her teeth here too, I think."

"She sucked the blood: she said she'd drain my heart," said Mason.

I saw Mr. Rochester shudder: a singularly marked expression of disgust,
horror, hatred, warped his countenance almost to distortion; but he only
said--

"Come, be silent, Richard, and never mind her gibberish: don't repeat
it."

"I wish I could forget it," was the answer.

"You will when you are out of the country: when you get back to Spanish
Town, you may think of her as dead and buried--or rather, you need not
think of her at all."

"Impossible to forget this night!"

"It is not impossible: have some energy, man.  You thought you were as
dead as a herring two hours since, and you are all alive and talking now.
There!--Carter has done with you or nearly so; I'll make you decent in a
trice.  Jane" (he turned to me for the first time since his re-entrance),
"take this key: go down into my bedroom, and walk straight forward into
my dressing-room: open the top drawer of the wardrobe and take out a
clean shirt and neck-handkerchief: bring them here; and be nimble."

I went; sought the repository he had mentioned, found the articles named,
and returned with them.

"Now," said he, "go to the other side of the bed while I order his
toilet; but don't leave the room: you may be wanted again."

I retired as directed.

"Was anybody stirring below when you went down, Jane?" inquired Mr.
Rochester presently.

"No, sir; all was very still."

"We shall get you off cannily, Dick: and it will be better, both for your
sake, and for that of the poor creature in yonder.  I have striven long
to avoid exposure, and I should not like it to come at last.  Here,
Carter, help him on with his waist-coat.  Where did you leave your furred
cloak?  You can't travel a mile without that, I know, in this damned cold
climate.  In your room?--Jane, run down to Mr. Mason's room,--the one
next mine,--and fetch a cloak you will see there."

Again I ran, and again returned, bearing an immense mantle lined and
edged with fur.

"Now, I've another errand for you," said my untiring master; "you must
away to my room again.  What a mercy you are shod with velvet, Jane!--a
clod-hopping messenger would never do at this juncture.  You must open
the middle drawer of my toilet-table and take out a little phial and a
little glass you will find there,--quick!"

I flew thither and back, bringing the desired vessels.

"That's well!  Now, doctor, I shall take the liberty of administering a
dose myself, on my own responsibility.  I got this cordial at Rome, of an
Italian charlatan--a fellow you would have kicked, Carter.  It is not a
thing to be used indiscriminately, but it is good upon occasion: as now,
for instance.  Jane, a little water."

He held out the tiny glass, and I half filled it from the water-bottle on
the washstand.

"That will do;--now wet the lip of the phial."

I did so; he measured twelve drops of a crimson liquid, and presented it
to Mason.

"Drink, Richard: it will give you the heart you lack, for an hour or so."

"But will it hurt me?--is it inflammatory?"

"Drink! drink! drink!"

Mr. Mason obeyed, because it was evidently useless to resist.  He was
dressed now: he still looked pale, but he was no longer gory and sullied.
Mr. Rochester let him sit three minutes after he had swallowed the
liquid; he then took his arm--

"Now I am sure you can get on your feet," he said--"try."

The patient rose.

"Carter, take him under the other shoulder.  Be of good cheer, Richard;
step out--that's it!"

"I do feel better," remarked Mr. Mason.

"I am sure you do.  Now, Jane, trip on before us away to the backstairs;
unbolt the side-passage door, and tell the driver of the post-chaise you
will see in the yard--or just outside, for I told him not to drive his
rattling wheels over the pavement--to be ready; we are coming: and, Jane,
if any one is about, come to the foot of the stairs and hem."

It was by this time half-past five, and the sun was on the point of
rising; but I found the kitchen still dark and silent.  The side-passage
door was fastened; I opened it with as little noise as possible: all the
yard was quiet; but the gates stood wide open, and there was a
post-chaise, with horses ready harnessed, and driver seated on the box,
stationed outside.  I approached him, and said the gentlemen were coming;
he nodded: then I looked carefully round and listened.  The stillness of
early morning slumbered everywhere; the curtains were yet drawn over the
servants' chamber windows; little birds were just twittering in the
blossom-blanched orchard trees, whose boughs drooped like white garlands
over the wall enclosing one side of the yard; the carriage horses stamped
from time to time in their closed stables: all else was still.

The gentlemen now appeared.  Mason, supported by Mr. Rochester and the
surgeon, seemed to walk with tolerable ease: they assisted him into the
chaise; Carter followed.

"Take care of him," said Mr. Rochester to the latter, "and keep him at
your house till he is quite well: I shall ride over in a day or two to
see how he gets on.  Richard, how is it with you?"

"The fresh air revives me, Fairfax."

"Leave the window open on his side, Carter; there is no wind--good-bye,
Dick."

"Fairfax--"

"Well what is it?"

"Let her be taken care of; let her be treated as tenderly as may be: let
her--" he stopped and burst into tears.

"I do my best; and have done it, and will do it," was the answer: he shut
up the chaise door, and the vehicle drove away.

"Yet would to God there was an end of all this!" added Mr. Rochester, as
he closed and barred the heavy yard-gates.

This done, he moved with slow step and abstracted air towards a door in
the wall bordering the orchard.  I, supposing he had done with me,
prepared to return to the house; again, however, I heard him call "Jane!"
He had opened the portal and stood at it, waiting for me.

"Come where there is some freshness, for a few moments," he said; "that
house is a mere dungeon: don't you feel it so?"

"It seems to me a splendid mansion, sir."

"The glamour of inexperience is over your eyes," he answered; "and you
see it through a charmed medium: you cannot discern that the gilding is
slime and the silk draperies cobwebs; that the marble is sordid slate,
and the polished woods mere refuse chips and scaly bark.  Now \emph{here}" (he
pointed to the leafy enclosure we had entered) "all is real, sweet, and
pure."

He strayed down a walk edged with box, with apple trees, pear trees, and
cherry trees on one side, and a border on the other full of all sorts of
old-fashioned flowers, stocks, sweet-williams, primroses, pansies,
mingled with southernwood, sweet-briar, and various fragrant herbs.  They
were fresh now as a succession of April showers and gleams, followed by a
lovely spring morning, could make them: the sun was just entering the
dappled east, and his light illumined the wreathed and dewy orchard trees
and shone down the quiet walks under them.

"Jane, will you have a flower?"

He gathered a half-blown rose, the first on the bush, and offered it to
me.

"Thank you, sir."

"Do you like this sunrise, Jane?  That sky with its high and light clouds
which are sure to melt away as the day waxes warm--this placid and balmly
atmosphere?"

"I do, very much."

"You have passed a strange night, Jane."

"Yes, sir."

"And it has made you look pale--were you afraid when I left you alone
with Mason?"

"I was afraid of some one coming out of the inner room."

"But I had fastened the door--I had the key in my pocket: I should have
been a careless shepherd if I had left a lamb--my pet lamb--so near a
wolf's den, unguarded: you were safe."

"Will Grace Poole live here still, sir?"

"Oh yes! don't trouble your head about her--put the thing out of your
thoughts."

"Yet it seems to me your life is hardly secure while she stays."

"Never fear--I will take care of myself."

"Is the danger you apprehended last night gone by now, sir?"

"I cannot vouch for that till Mason is out of England: nor even then.  To
live, for me, Jane, is to stand on a crater-crust which may crack and
spue fire any day."

"But Mr. Mason seems a man easily led.  Your influence, sir, is evidently
potent with him: he will never set you at defiance or wilfully injure
you."

"Oh, no!  Mason will not defy me; nor, knowing it, will he hurt me--but,
unintentionally, he might in a moment, by one careless word, deprive me,
if not of life, yet for ever of happiness."

"Tell him to be cautious, sir: let him know what you fear, and show him
how to avert the danger."

He laughed sardonically, hastily took my hand, and as hastily threw it
from him.

"If I could do that, simpleton, where would the danger be?  Annihilated
in a moment.  Ever since I have known Mason, I have only had to say to
him 'Do that,' and the thing has been done.  But I cannot give him orders
in this case: I cannot say 'Beware of harming me, Richard;' for it is
imperative that I should keep him ignorant that harm to me is possible.
Now you look puzzled; and I will puzzle you further.  You are my little
friend, are you not?"

"I like to serve you, sir, and to obey you in all that is right."

"Precisely: I see you do.  I see genuine contentment in your gait and
mien, your eye and face, when you are helping me and pleasing me--working
for me, and with me, in, as you characteristically say, '\emph{all that is
right}:' for if I bid you do what you thought wrong, there would be no
light-footed running, no neat-handed alacrity, no lively glance and
animated complexion.  My friend would then turn to me, quiet and pale,
and would say, 'No, sir; that is impossible: I cannot do it, because it
is wrong;' and would become immutable as a fixed star.  Well, you too
have power over me, and may injure me: yet I dare not show you where I am
vulnerable, lest, faithful and friendly as you are, you should transfix
me at once."

"If you have no more to fear from Mr. Mason than you have from me, sir,
you are very safe."

"God grant it may be so!  Here, Jane, is an arbour; sit down."

The arbour was an arch in the wall, lined with ivy; it contained a rustic
seat.  Mr. Rochester took it, leaving room, however, for me: but I stood
before him.

"Sit," he said; "the bench is long enough for two.  You don't hesitate to
take a place at my side, do you?  Is that wrong, Jane?"

I answered him by assuming it: to refuse would, I felt, have been unwise.

"Now, my little friend, while the sun drinks the dew--while all the
flowers in this old garden awake and expand, and the birds fetch their
young ones' breakfast out of the Thornfield, and the early bees do their
first spell of work--I'll put a case to you, which you must endeavour to
suppose your own: but first, look at me, and tell me you are at ease, and
not fearing that I err in detaining you, or that you err in staying."

"No, sir; I am content."

"Well then, Jane, call to aid your fancy:--suppose you were no longer a
girl well reared and disciplined, but a wild boy indulged from childhood
upwards; imagine yourself in a remote foreign land; conceive that you
there commit a capital error, no matter of what nature or from what
motives, but one whose consequences must follow you through life and
taint all your existence.  Mind, I don't say a \emph{crime}; I am not speaking
of shedding of blood or any other guilty act, which might make the
perpetrator amenable to the law: my word is \emph{error}.  The results of what
you have done become in time to you utterly insupportable; you take
measures to obtain relief: unusual measures, but neither unlawful nor
culpable.  Still you are miserable; for hope has quitted you on the very
confines of life: your sun at noon darkens in an eclipse, which you feel
will not leave it till the time of setting.  Bitter and base associations
have become the sole food of your memory: you wander here and there,
seeking rest in exile: happiness in pleasure--I mean in heartless,
sensual pleasure--such as dulls intellect and blights feeling.
Heart-weary and soul-withered, you come home after years of voluntary
banishment: you make a new acquaintance--how or where no matter: you find
in this stranger much of the good and bright qualities which you have
sought for twenty years, and never before encountered; and they are all
fresh, healthy, without soil and without taint.  Such society revives,
regenerates: you feel better days come back--higher wishes, purer
feelings; you desire to recommence your life, and to spend what remains
to you of days in a way more worthy of an immortal being.  To attain this
end, are you justified in overleaping an obstacle of custom--a mere
conventional impediment which neither your conscience sanctifies nor your
judgment approves?"

He paused for an answer: and what was I to say?  Oh, for some good spirit
to suggest a judicious and satisfactory response!  Vain aspiration!  The
west wind whispered in the ivy round me; but no gentle Ariel borrowed its
breath as a medium of speech: the birds sang in the tree-tops; but their
song, however sweet, was inarticulate.

Again Mr. Rochester propounded his query:

"Is the wandering and sinful, but now rest-seeking and repentant, man
justified in daring the world's opinion, in order to attach to him for
ever this gentle, gracious, genial stranger, thereby securing his own
peace of mind and regeneration of life?"

"Sir," I answered, "a wanderer's repose or a sinner's reformation should
never depend on a fellow-creature.  Men and women die; philosophers
falter in wisdom, and Christians in goodness: if any one you know has
suffered and erred, let him look higher than his equals for strength to
amend and solace to heal."

"But the instrument--the instrument!  God, who does the work, ordains the
instrument.  I have myself--I tell it you without parable--been a
worldly, dissipated, restless man; and I believe I have found the
instrument for my cure in--"

He paused: the birds went on carolling, the leaves lightly rustling.  I
almost wondered they did not check their songs and whispers to catch the
suspended revelation; but they would have had to wait many minutes--so
long was the silence protracted.  At last I looked up at the tardy
speaker: he was looking eagerly at me.

"Little friend," said he, in quite a changed tone--while his face changed
too, losing all its softness and gravity, and becoming harsh and
sarcastic--"you have noticed my tender penchant for Miss Ingram: don't
you think if I married her she would regenerate me with a vengeance?"

He got up instantly, went quite to the other end of the walk, and when he
came back he was humming a tune.

"Jane, Jane," said he, stopping before me, "you are quite pale with your
vigils: don't you curse me for disturbing your rest?"

"Curse you?  No, sir."

"Shake hands in confirmation of the word.  What cold fingers!  They were
warmer last night when I touched them at the door of the mysterious
chamber.  Jane, when will you watch with me again?"

"Whenever I can be useful, sir."

"For instance, the night before I am married!  I am sure I shall not be
able to sleep.  Will you promise to sit up with me to bear me company?  To
you I can talk of my lovely one: for now you have seen her and know her."

"Yes, sir."

"She's a rare one, is she not, Jane?"

"Yes, sir."

"A strapper--a real strapper, Jane: big, brown, and buxom; with hair just
such as the ladies of Carthage must have had.  Bless me! there's Dent and
Lynn in the stables!  Go in by the shrubbery, through that wicket."

As I went one way, he went another, and I heard him in the yard, saying
cheerfully--

"Mason got the start of you all this morning; he was gone before sunrise:
I rose at four to see him off."




\chapter{}


\lettrine{P}{resentiments} are strange things! and so are sympathies; and so are
signs; and the three combined make one mystery to which humanity has not
yet found the key.  I never laughed at presentiments in my life, because
I have had strange ones of my own.  Sympathies, I believe, exist (for
instance, between far-distant, long-absent, wholly estranged relatives
asserting, notwithstanding their alienation, the unity of the source to
which each traces his origin) whose workings baffle mortal comprehension.
And signs, for aught we know, may be but the sympathies of Nature with
man.

When I was a little girl, only six years old, I one night heard Bessie
Leaven say to Martha Abbot that she had been dreaming about a little
child; and that to dream of children was a sure sign of trouble, either
to one's self or one's kin.  The saying might have worn out of my memory,
had not a circumstance immediately followed which served indelibly to fix
it there.  The next day Bessie was sent for home to the deathbed of her
little sister.

Of late I had often recalled this saying and this incident; for during
the past week scarcely a night had gone over my couch that had not
brought with it a dream of an infant, which I sometimes hushed in my
arms, sometimes dandled on my knee, sometimes watched playing with
daisies on a lawn, or again, dabbling its hands in running water.  It was
a wailing child this night, and a laughing one the next: now it nestled
close to me, and now it ran from me; but whatever mood the apparition
evinced, whatever aspect it wore, it failed not for seven successive
nights to meet me the moment I entered the land of slumber.

I did not like this iteration of one idea--this strange recurrence of one
image, and I grew nervous as bedtime approached and the hour of the
vision drew near.  It was from companionship with this baby-phantom I had
been roused on that moonlight night when I heard the cry; and it was on
the afternoon of the day following I was summoned downstairs by a message
that some one wanted me in Mrs. Fairfax's room.  On repairing thither, I
found a man waiting for me, having the appearance of a gentleman's
servant: he was dressed in deep mourning, and the hat he held in his hand
was surrounded with a crape band.

"I daresay you hardly remember me, Miss," he said, rising as I entered;
"but my name is Leaven: I lived coachman with Mrs. Reed when you were at
Gateshead, eight or nine years since, and I live there still."

"Oh, Robert! how do you do?  I remember you very well: you used to give
me a ride sometimes on Miss Georgiana's bay pony.  And how is Bessie?  You
are married to Bessie?"

"Yes, Miss: my wife is very hearty, thank you; she brought me another
little one about two months since--we have three now--and both mother and
child are thriving."

"And are the family well at the house, Robert?"

"I am sorry I can't give you better news of them, Miss: they are very
badly at present--in great trouble."

"I hope no one is dead," I said, glancing at his black dress.  He too
looked down at the crape round his hat and replied--

"Mr. John died yesterday was a week, at his chambers in London."

"Mr. John?"

"Yes."

"And how does his mother bear it?"

"Why, you see, Miss Eyre, it is not a common mishap: his life has been
very wild: these last three years he gave himself up to strange ways, and
his death was shocking."

"I heard from Bessie he was not doing well."

"Doing well!  He could not do worse: he ruined his health and his estate
amongst the worst men and the worst women.  He got into debt and into
jail: his mother helped him out twice, but as soon as he was free he
returned to his old companions and habits.  His head was not strong: the
knaves he lived amongst fooled him beyond anything I ever heard.  He came
down to Gateshead about three weeks ago and wanted missis to give up all
to him.  Missis refused: her means have long been much reduced by his
extravagance; so he went back again, and the next news was that he was
dead.  How he died, God knows!--they say he killed himself."

I was silent: the things were frightful.  Robert Leaven resumed--

"Missis had been out of health herself for some time: she had got very
stout, but was not strong with it; and the loss of money and fear of
poverty were quite breaking her down.  The information about Mr. John's
death and the manner of it came too suddenly: it brought on a stroke.  She
was three days without speaking; but last Tuesday she seemed rather
better: she appeared as if she wanted to say something, and kept making
signs to my wife and mumbling.  It was only yesterday morning, however,
that Bessie understood she was pronouncing your name; and at last she
made out the words, 'Bring Jane--fetch Jane Eyre: I want to speak to
her.'  Bessie is not sure whether she is in her right mind, or means
anything by the words; but she told Miss Reed and Miss Georgiana, and
advised them to send for you.  The young ladies put it off at first; but
their mother grew so restless, and said, 'Jane, Jane,' so many times,
that at last they consented.  I left Gateshead yesterday: and if you can
get ready, Miss, I should like to take you back with me early to-morrow
morning."

"Yes, Robert, I shall be ready: it seems to me that I ought to go."

"I think so too, Miss.  Bessie said she was sure you would not refuse:
but I suppose you will have to ask leave before you can get off?"

"Yes; and I will do it now;" and having directed him to the servants'
hall, and recommended him to the care of John's wife, and the attentions
of John himself, I went in search of Mr. Rochester.

He was not in any of the lower rooms; he was not in the yard, the
stables, or the grounds.  I asked Mrs. Fairfax if she had seen him;--yes:
she believed he was playing billiards with Miss Ingram.  To the billiard-
room I hastened: the click of balls and the hum of voices resounded
thence; Mr. Rochester, Miss Ingram, the two Misses Eshton, and their
admirers, were all busied in the game.  It required some courage to
disturb so interesting a party; my errand, however, was one I could not
defer, so I approached the master where he stood at Miss Ingram's side.
She turned as I drew near, and looked at me haughtily: her eyes seemed to
demand, "What can the creeping creature want now?" and when I said, in a
low voice, "Mr. Rochester," she made a movement as if tempted to order me
away.  I remember her appearance at the moment--it was very graceful and
very striking: she wore a morning robe of sky-blue crape; a gauzy azure
scarf was twisted in her hair.  She had been all animation with the game,
and irritated pride did not lower the expression of her haughty
lineaments.

"Does that person want you?" she inquired of Mr. Rochester; and Mr.
Rochester turned to see who the "person" was.  He made a curious
grimace--one of his strange and equivocal demonstrations--threw down his
cue and followed me from the room.

"Well, Jane?" he said, as he rested his back against the schoolroom door,
which he had shut.

"If you please, sir, I want leave of absence for a week or two."

"What to do?--where to go?"

"To see a sick lady who has sent for me."

"What sick lady?--where does she live?"

"At Gateshead; in ---shire."

"-shire?  That is a hundred miles off!  Who may she be that sends for
people to see her that distance?"

"Her name is Reed, sir--Mrs. Reed."

"Reed of Gateshead?  There was a Reed of Gateshead, a magistrate."

"It is his widow, sir."

"And what have you to do with her?  How do you know her?"

"Mr. Reed was my uncle--my mother's brother."

"The deuce he was!  You never told me that before: you always said you
had no relations."

"None that would own me, sir.  Mr. Reed is dead, and his wife cast me
off."

"Why?"

"Because I was poor, and burdensome, and she disliked me."

"But Reed left children?--you must have cousins?  Sir George Lynn was
talking of a Reed of Gateshead yesterday, who, he said, was one of the
veriest rascals on town; and Ingram was mentioning a Georgiana Reed of
the same place, who was much admired for her beauty a season or two ago
in London."

"John Reed is dead, too, sir: he ruined himself and half-ruined his
family, and is supposed to have committed suicide.  The news so shocked
his mother that it brought on an apoplectic attack."

"And what good can you do her?  Nonsense, Jane!  I would never think of
running a hundred miles to see an old lady who will, perhaps, be dead
before you reach her: besides, you say she cast you off."

"Yes, sir, but that is long ago; and when her circumstances were very
different: I could not be easy to neglect her wishes now."

"How long will you stay?"

"As short a time as possible, sir."

"Promise me only to stay a week--"

"I had better not pass my word: I might be obliged to break it."

"At all events you \emph{will} come back: you will not be induced under any
pretext to take up a permanent residence with her?"

"Oh, no!  I shall certainly return if all be well."

"And who goes with you?  You don't travel a hundred miles alone."

"No, sir, she has sent her coachman."

"A person to be trusted?"

"Yes, sir, he has lived ten years in the family."

Mr. Rochester meditated.  "When do you wish to go?"

"Early to-morrow morning, sir."

"Well, you must have some money; you can't travel without money, and I
daresay you have not much: I have given you no salary yet.  How much have
you in the world, Jane?" he asked, smiling.

I drew out my purse; a meagre thing it was.  "Five shillings, sir."  He
took the purse, poured the hoard into his palm, and chuckled over it as
if its scantiness amused him.  Soon he produced his pocket-book: "Here,"
said he, offering me a note; it was fifty pounds, and he owed me but
fifteen.  I told him I had no change.

"I don't want change; you know that.  Take your wages."

I declined accepting more than was my due.  He scowled at first; then, as
if recollecting something, he said--

"Right, right!  Better not give you all now: you would, perhaps, stay
away three months if you had fifty pounds.  There are ten; is it not
plenty?"

"Yes, sir, but now you owe me five."

"Come back for it, then; I am your banker for forty pounds."

"Mr. Rochester, I may as well mention another matter of business to you
while I have the opportunity."

"Matter of business?  I am curious to hear it."

"You have as good as informed me, sir, that you are going shortly to be
married?"

"Yes; what then?"

"In that case, sir, Adele ought to go to school: I am sure you will
perceive the necessity of it."

"To get her out of my bride's way, who might otherwise walk over her
rather too emphatically?  There's sense in the suggestion; not a doubt of
it.  Adele, as you say, must go to school; and you, of course, must march
straight to--the devil?"

"I hope not, sir; but I must seek another situation somewhere."

"In course!" he exclaimed, with a twang of voice and a distortion of
features equally fantastic and ludicrous.  He looked at me some minutes.

"And old Madam Reed, or the Misses, her daughters, will be solicited by
you to seek a place, I suppose?"

"No, sir; I am not on such terms with my relatives as would justify me in
asking favours of them--but I shall advertise."

"You shall walk up the pyramids of Egypt!" he growled.  "At your peril
you advertise!  I wish I had only offered you a sovereign instead of ten
pounds.  Give me back nine pounds, Jane; I've a use for it."

"And so have I, sir," I returned, putting my hands and my purse behind
me.  "I could not spare the money on any account."

"Little niggard!" said he, "refusing me a pecuniary request!  Give me
five pounds, Jane."

"Not five shillings, sir; nor five pence."

"Just let me look at the cash."

"No, sir; you are not to be trusted."

"Jane!"

"Sir?"

"Promise me one thing."

"I'll promise you anything, sir, that I think I am likely to perform."

"Not to advertise: and to trust this quest of a situation to me.  I'll
find you one in time."

"I shall be glad so to do, sir, if you, in your turn, will promise that I
and Adele shall be both safe out of the house before your bride enters
it."

"Very well! very well!  I'll pledge my word on it.  You go to-morrow,
then?"

"Yes, sir; early."

"Shall you come down to the drawing-room after dinner?"

"No, sir, I must prepare for the journey."

"Then you and I must bid good-bye for a little while?"

"I suppose so, sir."

"And how do people perform that ceremony of parting, Jane?  Teach me; I'm
not quite up to it."

"They say, Farewell, or any other form they prefer."

"Then say it."

"Farewell, Mr. Rochester, for the present."

"What must I say?"

"The same, if you like, sir."

"Farewell, Miss Eyre, for the present; is that all?"

"Yes?"

"It seems stingy, to my notions, and dry, and unfriendly.  I should like
something else: a little addition to the rite.  If one shook hands, for
instance; but no--that would not content me either.  So you'll do no more
than say Farewell, Jane?"

"It is enough, sir: as much good-will may be conveyed in one hearty word
as in many."

"Very likely; but it is blank and cool--'Farewell.'"

"How long is he going to stand with his back against that door?" I asked
myself; "I want to commence my packing."  The dinner-bell rang, and
suddenly away he bolted, without another syllable: I saw him no more
during the day, and was off before he had risen in the morning.

I reached the lodge at Gateshead about five o'clock in the afternoon of
the first of May: I stepped in there before going up to the hall.  It was
very clean and neat: the ornamental windows were hung with little white
curtains; the floor was spotless; the grate and fire-irons were burnished
bright, and the fire burnt clear.  Bessie sat on the hearth, nursing her
last-born, and Robert and his sister played quietly in a corner.

"Bless you!--I knew you would come!" exclaimed Mrs. Leaven, as I entered.

"Yes, Bessie," said I, after I had kissed her; "and I trust I am not too
late.  How is Mrs. Reed?--Alive still, I hope."

"Yes, she is alive; and more sensible and collected than she was.  The
doctor says she may linger a week or two yet; but he hardly thinks she
will finally recover."

"Has she mentioned me lately?"

"She was talking of you only this morning, and wishing you would come,
but she is sleeping now, or was ten minutes ago, when I was up at the
house.  She generally lies in a kind of lethargy all the afternoon, and
wakes up about six or seven.  Will you rest yourself here an hour, Miss,
and then I will go up with you?"

Robert here entered, and Bessie laid her sleeping child in the cradle and
went to welcome him: afterwards she insisted on my taking off my bonnet
and having some tea; for she said I looked pale and tired.  I was glad to
accept her hospitality; and I submitted to be relieved of my travelling
garb just as passively as I used to let her undress me when a child.

Old times crowded fast back on me as I watched her bustling about--setting
out the tea-tray with her best china, cutting bread and butter, toasting
a tea-cake, and, between whiles, giving little Robert or Jane an
occasional tap or push, just as she used to give me in former days.
Bessie had retained her quick temper as well as her light foot and good
looks.

Tea ready, I was going to approach the table; but she desired me to sit
still, quite in her old peremptory tones.  I must be served at the
fireside, she said; and she placed before me a little round stand with my
cup and a plate of toast, absolutely as she used to accommodate me with
some privately purloined dainty on a nursery chair: and I smiled and
obeyed her as in bygone days.

She wanted to know if I was happy at Thornfield Hall, and what sort of a
person the mistress was; and when I told her there was only a master,
whether he was a nice gentleman, and if I liked him.  I told her he was
rather an ugly man, but quite a gentleman; and that he treated me kindly,
and I was content.  Then I went on to describe to her the gay company
that had lately been staying at the house; and to these details Bessie
listened with interest: they were precisely of the kind she relished.

In such conversation an hour was soon gone: Bessie restored to me my
bonnet, \&c., and, accompanied by her, I quitted the lodge for the hall.
It was also accompanied by her that I had, nearly nine years ago, walked
down the path I was now ascending.  On a dark, misty, raw morning in
January, I had left a hostile roof with a desperate and embittered
heart--a sense of outlawry and almost of reprobation--to seek the chilly
harbourage of Lowood: that bourne so far away and unexplored.  The same
hostile roof now again rose before me: my prospects were doubtful yet;
and I had yet an aching heart.  I still felt as a wanderer on the face of
the earth; but I experienced firmer trust in myself and my own powers,
and less withering dread of oppression.  The gaping wound of my wrongs,
too, was now quite healed; and the flame of resentment extinguished.

"You shall go into the breakfast-room first," said Bessie, as she
preceded me through the hall; "the young ladies will be there."

In another moment I was within that apartment.  There was every article
of furniture looking just as it did on the morning I was first introduced
to Mr. Brocklehurst: the very rug he had stood upon still covered the
hearth.  Glancing at the bookcases, I thought I could distinguish the two
volumes of Bewick's British Birds occupying their old place on the third
shelf, and Gulliver's Travels and the Arabian Nights ranged just above.
The inanimate objects were not changed; but the living things had altered
past recognition.

Two young ladies appeared before me; one very tall, almost as tall as
Miss Ingram--very thin too, with a sallow face and severe mien.  There
was something ascetic in her look, which was augmented by the extreme
plainness of a straight-skirted, black, stuff dress, a starched linen
collar, hair combed away from the temples, and the nun-like ornament of a
string of ebony beads and a crucifix.  This I felt sure was Eliza, though
I could trace little resemblance to her former self in that elongated and
colourless visage.

The other was as certainly Georgiana: but not the Georgiana I
remembered--the slim and fairy-like girl of eleven.  This was a
full-blown, very plump damsel, fair as waxwork, with handsome and regular
features, languishing blue eyes, and ringleted yellow hair.  The hue of
her dress was black too; but its fashion was so different from her
sister's--so much more flowing and becoming--it looked as stylish as the
other's looked puritanical.

In each of the sisters there was one trait of the mother--and only one;
the thin and pallid elder daughter had her parent's Cairngorm eye: the
blooming and luxuriant younger girl had her contour of jaw and
chin--perhaps a little softened, but still imparting an indescribable
hardness to the countenance otherwise so voluptuous and buxom.

Both ladies, as I advanced, rose to welcome me, and both addressed me by
the name of "Miss Eyre."  Eliza's greeting was delivered in a short,
abrupt voice, without a smile; and then she sat down again, fixed her
eyes on the fire, and seemed to forget me.  Georgiana added to her "How
d'ye do?" several commonplaces about my journey, the weather, and so on,
uttered in rather a drawling tone: and accompanied by sundry side-glances
that measured me from head to foot--now traversing the folds of my drab
merino pelisse, and now lingering on the plain trimming of my cottage
bonnet.  Young ladies have a remarkable way of letting you know that they
think you a "quiz" without actually saying the words.  A certain
superciliousness of look, coolness of manner, nonchalance of tone,
express fully their sentiments on the point, without committing them by
any positive rudeness in word or deed.

A sneer, however, whether covert or open, had now no longer that power
over me it once possessed: as I sat between my cousins, I was surprised
to find how easy I felt under the total neglect of the one and the semi-
sarcastic attentions of the other--Eliza did not mortify, nor Georgiana
ruffle me.  The fact was, I had other things to think about; within the
last few months feelings had been stirred in me so much more potent than
any they could raise--pains and pleasures so much more acute and
exquisite had been excited than any it was in their power to inflict or
bestow--that their airs gave me no concern either for good or bad.

"How is Mrs. Reed?" I asked soon, looking calmly at Georgiana, who
thought fit to bridle at the direct address, as if it were an unexpected
liberty.

"Mrs. Reed?  Ah! mama, you mean; she is extremely poorly: I doubt if you
can see her to-night."

"If," said I, "you would just step upstairs and tell her I am come, I
should be much obliged to you."

Georgiana almost started, and she opened her blue eyes wild and wide.  "I
know she had a particular wish to see me," I added, "and I would not
defer attending to her desire longer than is absolutely necessary."

"Mama dislikes being disturbed in an evening," remarked Eliza.  I soon
rose, quietly took off my bonnet and gloves, uninvited, and said I would
just step out to Bessie--who was, I dared say, in the kitchen--and ask
her to ascertain whether Mrs. Reed was disposed to receive me or not to-
night.  I went, and having found Bessie and despatched her on my errand,
I proceeded to take further measures.  It had heretofore been my habit
always to shrink from arrogance: received as I had been to-day, I should,
a year ago, have resolved to quit Gateshead the very next morning; now,
it was disclosed to me all at once that that would be a foolish plan.  I
had taken a journey of a hundred miles to see my aunt, and I must stay
with her till she was better--or dead: as to her daughters' pride or
folly, I must put it on one side, make myself independent of it.  So I
addressed the housekeeper; asked her to show me a room, told her I should
probably be a visitor here for a week or two, had my trunk conveyed to my
chamber, and followed it thither myself: I met Bessie on the landing.

"Missis is awake," said she; "I have told her you are here: come and let
us see if she will know you."

I did not need to be guided to the well-known room, to which I had so
often been summoned for chastisement or reprimand in former days.  I
hastened before Bessie; I softly opened the door: a shaded light stood on
the table, for it was now getting dark.  There was the great four-post
bed with amber hangings as of old; there the toilet-table, the armchair,
and the footstool, at which I had a hundred times been sentenced to
kneel, to ask pardon for offences by me uncommitted.  I looked into a
certain corner near, half-expecting to see the slim outline of a once
dreaded switch which used to lurk there, waiting to leap out imp-like and
lace my quivering palm or shrinking neck.  I approached the bed; I opened
the curtains and leant over the high-piled pillows.

Well did I remember Mrs. Reed's face, and I eagerly sought the familiar
image.  It is a happy thing that time quells the longings of vengeance
and hushes the promptings of rage and aversion.  I had left this woman in
bitterness and hate, and I came back to her now with no other emotion
than a sort of ruth for her great sufferings, and a strong yearning to
forget and forgive all injuries--to be reconciled and clasp hands in
amity.

The well-known face was there: stern, relentless as ever--there was that
peculiar eye which nothing could melt, and the somewhat raised,
imperious, despotic eyebrow.  How often had it lowered on me menace and
hate! and how the recollection of childhood's terrors and sorrows revived
as I traced its harsh line now!  And yet I stooped down and kissed her:
she looked at me.

"Is this Jane Eyre?" she said.

"Yes, Aunt Reed.  How are you, dear aunt?"

I had once vowed that I would never call her aunt again: I thought it no
sin to forget and break that vow now.  My fingers had fastened on her
hand which lay outside the sheet: had she pressed mine kindly, I should
at that moment have experienced true pleasure.  But unimpressionable
natures are not so soon softened, nor are natural antipathies so readily
eradicated.  Mrs. Reed took her hand away, and, turning her face rather
from me, she remarked that the night was warm.  Again she regarded me so
icily, I felt at once that her opinion of me--her feeling towards me--was
unchanged and unchangeable.  I knew by her stony eye--opaque to
tenderness, indissoluble to tears--that she was resolved to consider me
bad to the last; because to believe me good would give her no generous
pleasure: only a sense of mortification.

I felt pain, and then I felt ire; and then I felt a determination to
subdue her--to be her mistress in spite both of her nature and her will.
My tears had risen, just as in childhood: I ordered them back to their
source.  I brought a chair to the bed-head: I sat down and leaned over
the pillow.

"You sent for me," I said, "and I am here; and it is my intention to stay
till I see how you get on."

"Oh, of course!  You have seen my daughters?"

"Yes."

"Well, you may tell them I wish you to stay till I can talk some things
over with you I have on my mind: to-night it is too late, and I have a
difficulty in recalling them.  But there was something I wished to
say--let me see--"

The wandering look and changed utterance told what wreck had taken place
in her once vigorous frame.  Turning restlessly, she drew the bedclothes
round her; my elbow, resting on a corner of the quilt, fixed it down: she
was at once irritated.

"Sit up!" said she; "don't annoy me with holding the clothes fast.  Are
you Jane Eyre?"

"I am Jane Eyre."

"I have had more trouble with that child than any one would believe.  Such
a burden to be left on my hands--and so much annoyance as she caused me,
daily and hourly, with her incomprehensible disposition, and her sudden
starts of temper, and her continual, unnatural watchings of one's
movements!  I declare she talked to me once like something mad, or like a
fiend--no child ever spoke or looked as she did; I was glad to get her
away from the house.  What did they do with her at Lowood?  The fever
broke out there, and many of the pupils died.  She, however, did not die:
but I said she did--I wish she had died!"

"A strange wish, Mrs. Reed; why do you hate her so?"

"I had a dislike to her mother always; for she was my husband's only
sister, and a great favourite with him: he opposed the family's disowning
her when she made her low marriage; and when news came of her death, he
wept like a simpleton.  He would send for the baby; though I entreated
him rather to put it out to nurse and pay for its maintenance.  I hated
it the first time I set my eyes on it--a sickly, whining, pining thing!
It would wail in its cradle all night long--not screaming heartily like
any other child, but whimpering and moaning.  Reed pitied it; and he used
to nurse it and notice it as if it had been his own: more, indeed, than
he ever noticed his own at that age.  He would try to make my children
friendly to the little beggar: the darlings could not bear it, and he was
angry with them when they showed their dislike.  In his last illness, he
had it brought continually to his bedside; and but an hour before he
died, he bound me by vow to keep the creature.  I would as soon have been
charged with a pauper brat out of a workhouse: but he was weak, naturally
weak.  John does not at all resemble his father, and I am glad of it:
John is like me and like my brothers--he is quite a Gibson.  Oh, I wish
he would cease tormenting me with letters for money?  I have no more
money to give him: we are getting poor.  I must send away half the
servants and shut up part of the house; or let it off.  I can never
submit to do that--yet how are we to get on?  Two-thirds of my income
goes in paying the interest of mortgages.  John gambles dreadfully, and
always loses--poor boy!  He is beset by sharpers: John is sunk and
degraded--his look is frightful--I feel ashamed for him when I see him."

She was getting much excited.  "I think I had better leave her now," said
I to Bessie, who stood on the other side of the bed.

"Perhaps you had, Miss: but she often talks in this way towards night--in
the morning she is calmer."

I rose.  "Stop!" exclaimed Mrs. Reed, "there is another thing I wished to
say.  He threatens me--he continually threatens me with his own death, or
mine: and I dream sometimes that I see him laid out with a great wound in
his throat, or with a swollen and blackened face.  I am come to a strange
pass: I have heavy troubles.  What is to be done?  How is the money to be
had?"

Bessie now endeavoured to persuade her to take a sedative draught: she
succeeded with difficulty.  Soon after, Mrs. Reed grew more composed, and
sank into a dozing state.  I then left her.

More than ten days elapsed before I had again any conversation with her.
She continued either delirious or lethargic; and the doctor forbade
everything which could painfully excite her.  Meantime, I got on as well
as I could with Georgiana and Eliza.  They were very cold, indeed, at
first.  Eliza would sit half the day sewing, reading, or writing, and
scarcely utter a word either to me or her sister.  Georgiana would
chatter nonsense to her canary bird by the hour, and take no notice of
me.  But I was determined not to seem at a loss for occupation or
amusement: I had brought my drawing materials with me, and they served me
for both.

Provided with a case of pencils, and some sheets of paper, I used to take
a seat apart from them, near the window, and busy myself in sketching
fancy vignettes, representing any scene that happened momentarily to
shape itself in the ever-shifting kaleidoscope of imagination: a glimpse
of sea between two rocks; the rising moon, and a ship crossing its disk;
a group of reeds and water-flags, and a naiad's head, crowned with lotus-
flowers, rising out of them; an elf sitting in a hedge-sparrow's nest,
under a wreath of hawthorn-bloom.

One morning I fell to sketching a face: what sort of a face it was to be,
I did not care or know.  I took a soft black pencil, gave it a broad
point, and worked away.  Soon I had traced on the paper a broad and
prominent forehead and a square lower outline of visage: that contour
gave me pleasure; my fingers proceeded actively to fill it with features.
Strongly-marked horizontal eyebrows must be traced under that brow; then
followed, naturally, a well-defined nose, with a straight ridge and full
nostrils; then a flexible-looking mouth, by no means narrow; then a firm
chin, with a decided cleft down the middle of it: of course, some black
whiskers were wanted, and some jetty hair, tufted on the temples, and
waved above the forehead.  Now for the eyes: I had left them to the last,
because they required the most careful working.  I drew them large; I
shaped them well: the eyelashes I traced long and sombre; the irids
lustrous and large.  "Good! but not quite the thing," I thought, as I
surveyed the effect: "they want more force and spirit;" and I wrought the
shades blacker, that the lights might flash more brilliantly--a happy
touch or two secured success.  There, I had a friend's face under my
gaze; and what did it signify that those young ladies turned their backs
on me?  I looked at it; I smiled at the speaking likeness: I was absorbed
and content.

"Is that a portrait of some one you know?" asked Eliza, who had
approached me unnoticed.  I responded that it was merely a fancy head,
and hurried it beneath the other sheets.  Of course, I lied: it was, in
fact, a very faithful representation of Mr. Rochester.  But what was that
to her, or to any one but myself?  Georgiana also advanced to look.  The
other drawings pleased her much, but she called that "an ugly man." They
both seemed surprised at my skill.  I offered to sketch their portraits;
and each, in turn, sat for a pencil outline.  Then Georgiana produced her
album.  I promised to contribute a water-colour drawing: this put her at
once into good humour.  She proposed a walk in the grounds.  Before we
had been out two hours, we were deep in a confidential conversation: she
had favoured me with a description of the brilliant winter she had spent
in London two seasons ago--of the admiration she had there excited--the
attention she had received; and I even got hints of the titled conquest
she had made.  In the course of the afternoon and evening these hints
were enlarged on: various soft conversations were reported, and
sentimental scenes represented; and, in short, a volume of a novel of
fashionable life was that day improvised by her for my benefit.  The
communications were renewed from day to day: they always ran on the same
theme--herself, her loves, and woes.  It was strange she never once
adverted either to her mother's illness, or her brother's death, or the
present gloomy state of the family prospects.  Her mind seemed wholly
taken up with reminiscences of past gaiety, and aspirations after
dissipations to come.  She passed about five minutes each day in her
mother's sick-room, and no more.

Eliza still spoke little: she had evidently no time to talk.  I never saw
a busier person than she seemed to be; yet it was difficult to say what
she did: or rather, to discover any result of her diligence.  She had an
alarm to call her up early.  I know not how she occupied herself before
breakfast, but after that meal she divided her time into regular
portions, and each hour had its allotted task.  Three times a day she
studied a little book, which I found, on inspection, was a Common Prayer
Book.  I asked her once what was the great attraction of that volume, and
she said, "the Rubric."  Three hours she gave to stitching, with gold
thread, the border of a square crimson cloth, almost large enough for a
carpet.  In answer to my inquiries after the use of this article, she
informed me it was a covering for the altar of a new church lately
erected near Gateshead.  Two hours she devoted to her diary; two to
working by herself in the kitchen-garden; and one to the regulation of
her accounts.  She seemed to want no company; no conversation.  I believe
she was happy in her way: this routine sufficed for her; and nothing
annoyed her so much as the occurrence of any incident which forced her to
vary its clockwork regularity.

She told me one evening, when more disposed to be communicative than
usual, that John's conduct, and the threatened ruin of the family, had
been a source of profound affliction to her: but she had now, she said,
settled her mind, and formed her resolution.  Her own fortune she had
taken care to secure; and when her mother died--and it was wholly
improbable, she tranquilly remarked, that she should either recover or
linger long--she would execute a long-cherished project: seek a
retirement where punctual habits would be permanently secured from
disturbance, and place safe barriers between herself and a frivolous
world.  I asked if Georgiana would accompany her.

"Of course not.  Georgiana and she had nothing in common: they never had
had.  She would not be burdened with her society for any consideration.
Georgiana should take her own course; and she, Eliza, would take hers."

Georgiana, when not unburdening her heart to me, spent most of her time
in lying on the sofa, fretting about the dulness of the house, and
wishing over and over again that her aunt Gibson would send her an
invitation up to town.  "It would be so much better," she said, "if she
could only get out of the way for a month or two, till all was over."  I
did not ask what she meant by "all being over," but I suppose she
referred to the expected decease of her mother and the gloomy sequel of
funeral rites.  Eliza generally took no more notice of her sister's
indolence and complaints than if no such murmuring, lounging object had
been before her.  One day, however, as she put away her account-book and
unfolded her embroidery, she suddenly took her up thus--

"Georgiana, a more vain and absurd animal than you was certainly never
allowed to cumber the earth.  You had no right to be born, for you make
no use of life.  Instead of living for, in, and with yourself, as a
reasonable being ought, you seek only to fasten your feebleness on some
other person's strength: if no one can be found willing to burden her or
himself with such a fat, weak, puffy, useless thing, you cry out that you
are ill-treated, neglected, miserable.  Then, too, existence for you must
be a scene of continual change and excitement, or else the world is a
dungeon: you must be admired, you must be courted, you must be
flattered--you must have music, dancing, and society--or you languish,
you die away.  Have you no sense to devise a system which will make you
independent of all efforts, and all wills, but your own?  Take one day;
share it into sections; to each section apportion its task: leave no
stray unemployed quarters of an hour, ten minutes, five minutes--include
all; do each piece of business in its turn with method, with rigid
regularity.  The day will close almost before you are aware it has begun;
and you are indebted to no one for helping you to get rid of one vacant
moment: you have had to seek no one's company, conversation, sympathy,
forbearance; you have lived, in short, as an independent being ought to
do.  Take this advice: the first and last I shall offer you; then you
will not want me or any one else, happen what may.  Neglect it--go on as
heretofore, craving, whining, and idling--and suffer the results of your
idiocy, however bad and insuperable they may be.  I tell you this
plainly; and listen: for though I shall no more repeat what I am now
about to say, I shall steadily act on it.  After my mother's death, I
wash my hands of you: from the day her coffin is carried to the vault in
Gateshead Church, you and I will be as separate as if we had never known
each other.  You need not think that because we chanced to be born of the
same parents, I shall suffer you to fasten me down by even the feeblest
claim: I can tell you this--if the whole human race, ourselves excepted,
were swept away, and we two stood alone on the earth, I would leave you
in the old world, and betake myself to the new."

She closed her lips.

"You might have spared yourself the trouble of delivering that tirade,"
answered Georgiana.  "Everybody knows you are the most selfish, heartless
creature in existence: and \emph{I} know your spiteful hatred towards me: I
have had a specimen of it before in the trick you played me about Lord
Edwin Vere: you could not bear me to be raised above you, to have a
title, to be received into circles where you dare not show your face, and
so you acted the spy and informer, and ruined my prospects for ever."
Georgiana took out her handkerchief and blew her nose for an hour
afterwards; Eliza sat cold, impassable, and assiduously industrious.

True, generous feeling is made small account of by some, but here were
two natures rendered, the one intolerably acrid, the other despicably
savourless for the want of it.  Feeling without judgment is a washy
draught indeed; but judgment untempered by feeling is too bitter and
husky a morsel for human deglutition.

It was a wet and windy afternoon: Georgiana had fallen asleep on the sofa
over the perusal of a novel; Eliza was gone to attend a saint's-day
service at the new church--for in matters of religion she was a rigid
formalist: no weather ever prevented the punctual discharge of what she
considered her devotional duties; fair or foul, she went to church thrice
every Sunday, and as often on week-days as there were prayers.

I bethought myself to go upstairs and see how the dying woman sped, who
lay there almost unheeded: the very servants paid her but a remittent
attention: the hired nurse, being little looked after, would slip out of
the room whenever she could.  Bessie was faithful; but she had her own
family to mind, and could only come occasionally to the hall.  I found
the sick-room unwatched, as I had expected: no nurse was there; the
patient lay still, and seemingly lethargic; her livid face sunk in the
pillows: the fire was dying in the grate.  I renewed the fuel,
re-arranged the bedclothes, gazed awhile on her who could not now gaze on
me, and then I moved away to the window.

The rain beat strongly against the panes, the wind blew tempestuously:
"One lies there," I thought, "who will soon be beyond the war of earthly
elements.  Whither will that spirit--now struggling to quit its material
tenement--flit when at length released?"

In pondering the great mystery, I thought of Helen Burns, recalled her
dying words--her faith--her doctrine of the equality of disembodied
souls.  I was still listening in thought to her well-remembered
tones--still picturing her pale and spiritual aspect, her wasted face and
sublime gaze, as she lay on her placid deathbed, and whispered her
longing to be restored to her divine Father's bosom--when a feeble voice
murmured from the couch behind: "Who is that?"

I knew Mrs. Reed had not spoken for days: was she reviving?  I went up to
her.

"It is I, Aunt Reed."

"Who--I?" was her answer.  "Who are you?" looking at me with surprise and
a sort of alarm, but still not wildly.  "You are quite a stranger to
me--where is Bessie?"

"She is at the lodge, aunt."

"Aunt," she repeated.  "Who calls me aunt?  You are not one of the
Gibsons; and yet I know you--that face, and the eyes and forehead, are
quiet familiar to me: you are like--why, you are like Jane Eyre!"

I said nothing: I was afraid of occasioning some shock by declaring my
identity.

"Yet," said she, "I am afraid it is a mistake: my thoughts deceive me.  I
wished to see Jane Eyre, and I fancy a likeness where none exists:
besides, in eight years she must be so changed."  I now gently assured
her that I was the person she supposed and desired me to be: and seeing
that I was understood, and that her senses were quite collected, I
explained how Bessie had sent her husband to fetch me from Thornfield.

"I am very ill, I know," she said ere long.  "I was trying to turn myself
a few minutes since, and find I cannot move a limb.  It is as well I
should ease my mind before I die: what we think little of in health,
burdens us at such an hour as the present is to me.  Is the nurse here?
or is there no one in the room but you?"

I assured her we were alone.

"Well, I have twice done you a wrong which I regret now.  One was in
breaking the promise which I gave my husband to bring you up as my own
child; the other--" she stopped.  "After all, it is of no great
importance, perhaps," she murmured to herself: "and then I may get
better; and to humble myself so to her is painful."

She made an effort to alter her position, but failed: her face changed;
she seemed to experience some inward sensation--the precursor, perhaps,
of the last pang.

"Well, I must get it over.  Eternity is before me: I had better tell
her.--Go to my dressing-case, open it, and take out a letter you will see
there."

I obeyed her directions.  "Read the letter," she said.

It was short, and thus conceived:--

   "Madam,--Will you have the goodness to send me the address of my
   niece, Jane Eyre, and to tell me how she is?  It is my intention to
   write shortly and desire her to come to me at Madeira.  Providence has
   blessed my endeavours to secure a competency; and as I am unmarried
   and childless, I wish to adopt her during my life, and bequeath her at
   my death whatever I may have to leave.--I am, Madam, \&c., \&c.,

   "JOHN EYRE, Madeira."

It was dated three years back.

"Why did I never hear of this?" I asked.

"Because I disliked you too fixedly and thoroughly ever to lend a hand in
lifting you to prosperity.  I could not forget your conduct to me,
Jane--the fury with which you once turned on me; the tone in which you
declared you abhorred me the worst of anybody in the world; the
unchildlike look and voice with which you affirmed that the very thought
of me made you sick, and asserted that I had treated you with miserable
cruelty.  I could not forget my own sensations when you thus started up
and poured out the venom of your mind: I felt fear as if an animal that I
had struck or pushed had looked up at me with human eyes and cursed me in
a man's voice.--Bring me some water!  Oh, make haste!"

"Dear Mrs. Reed," said I, as I offered her the draught she required,
"think no more of all this, let it pass away from your mind.  Forgive me
for my passionate language: I was a child then; eight, nine years have
passed since that day."

She heeded nothing of what I said; but when she had tasted the water and
drawn breath, she went on thus--

"I tell you I could not forget it; and I took my revenge: for you to be
adopted by your uncle, and placed in a state of ease and comfort, was
what I could not endure.  I wrote to him; I said I was sorry for his
disappointment, but Jane Eyre was dead: she had died of typhus fever at
Lowood.  Now act as you please: write and contradict my assertion--expose
my falsehood as soon as you like.  You were born, I think, to be my
torment: my last hour is racked by the recollection of a deed which, but
for you, I should never have been tempted to commit."

"If you could but be persuaded to think no more of it, aunt, and to
regard me with kindness and forgiveness"

"You have a very bad disposition," said she, "and one to this day I feel
it impossible to understand: how for nine years you could be patient and
quiescent under any treatment, and in the tenth break out all fire and
violence, I can never comprehend."

"My disposition is not so bad as you think: I am passionate, but not
vindictive.  Many a time, as a little child, I should have been glad to
love you if you would have let me; and I long earnestly to be reconciled
to you now: kiss me, aunt."

I approached my cheek to her lips: she would not touch it.  She said I
oppressed her by leaning over the bed, and again demanded water.  As I
laid her down--for I raised her and supported her on my arm while she
drank--I covered her ice-cold and clammy hand with mine: the feeble
fingers shrank from my touch--the glazing eyes shunned my gaze.

"Love me, then, or hate me, as you will," I said at last, "you have my
full and free forgiveness: ask now for God's, and be at peace."

Poor, suffering woman! it was too late for her to make now the effort to
change her habitual frame of mind: living, she had ever hated me--dying,
she must hate me still.

The nurse now entered, and Bessie followed.  I yet lingered half-an-hour
longer, hoping to see some sign of amity: but she gave none.  She was
fast relapsing into stupor; nor did her mind again rally: at twelve
o'clock that night she died.  I was not present to close her eyes, nor
were either of her daughters.  They came to tell us the next morning that
all was over.  She was by that time laid out.  Eliza and I went to look
at her: Georgiana, who had burst out into loud weeping, said she dared
not go.  There was stretched Sarah Reed's once robust and active frame,
rigid and still: her eye of flint was covered with its cold lid; her brow
and strong traits wore yet the impress of her inexorable soul.  A strange
and solemn object was that corpse to me.  I gazed on it with gloom and
pain: nothing soft, nothing sweet, nothing pitying, or hopeful, or
subduing did it inspire; only a grating anguish for \emph{her} woes--not \emph{my}
loss--and a sombre tearless dismay at the fearfulness of death in such a
form.

Eliza surveyed her parent calmly.  After a silence of some minutes she
observed--

"With her constitution she should have lived to a good old age: her life
was shortened by trouble."  And then a spasm constricted her mouth for an
instant: as it passed away she turned and left the room, and so did I.
Neither of us had dropt a tear.




\chapter{}


\lettrine{M}{r.} Rochester had given me but one week's leave of absence: yet a month
elapsed before I quitted Gateshead.  I wished to leave immediately after
the funeral, but Georgiana entreated me to stay till she could get off to
London, whither she was now at last invited by her uncle, Mr. Gibson, who
had come down to direct his sister's interment and settle the family
affairs.  Georgiana said she dreaded being left alone with Eliza; from
her she got neither sympathy in her dejection, support in her fears, nor
aid in her preparations; so I bore with her feeble-minded wailings and
selfish lamentations as well as I could, and did my best in sewing for
her and packing her dresses.  It is true, that while I worked, she would
idle; and I thought to myself, "If you and I were destined to live always
together, cousin, we would commence matters on a different footing.  I
should not settle tamely down into being the forbearing party; I should
assign you your share of labour, and compel you to accomplish it, or else
it should be left undone: I should insist, also, on your keeping some of
those drawling, half-insincere complaints hushed in your own breast.  It
is only because our connection happens to be very transitory, and comes
at a peculiarly mournful season, that I consent thus to render it so
patient and compliant on my part."

At last I saw Georgiana off; but now it was Eliza's turn to request me to
stay another week.  Her plans required all her time and attention, she
said; she was about to depart for some unknown bourne; and all day long
she stayed in her own room, her door bolted within, filling trunks,
emptying drawers, burning papers, and holding no communication with any
one.  She wished me to look after the house, to see callers, and answer
notes of condolence.

One morning she told me I was at liberty.  "And," she added, "I am
obliged to you for your valuable services and discreet conduct!  There is
some difference between living with such an one as you and with
Georgiana: you perform your own part in life and burden no one.
To-morrow," she continued, "I set out for the Continent.  I shall take up
my abode in a religious house near Lisle--a nunnery you would call it;
there I shall be quiet and unmolested.  I shall devote myself for a time
to the examination of the Roman Catholic dogmas, and to a careful study
of the workings of their system: if I find it to be, as I half suspect it
is, the one best calculated to ensure the doing of all things decently
and in order, I shall embrace the tenets of Rome and probably take the
veil."

I neither expressed surprise at this resolution nor attempted to dissuade
her from it.  "The vocation will fit you to a hair," I thought: "much
good may it do you!"

When we parted, she said: "Good-bye, cousin Jane Eyre; I wish you well:
you have some sense."

I then returned: "You are not without sense, cousin Eliza; but what you
have, I suppose, in another year will be walled up alive in a French
convent.  However, it is not my business, and so it suits you, I don't
much care."

"You are in the right," said she; and with these words we each went our
separate way.  As I shall not have occasion to refer either to her or her
sister again, I may as well mention here, that Georgiana made an
advantageous match with a wealthy worn-out man of fashion, and that Eliza
actually took the veil, and is at this day superior of the convent where
she passed the period of her novitiate, and which she endowed with her
fortune.

How people feel when they are returning home from an absence, long or
short, I did not know: I had never experienced the sensation.  I had
known what it was to come back to Gateshead when a child after a long
walk, to be scolded for looking cold or gloomy; and later, what it was to
come back from church to Lowood, to long for a plenteous meal and a good
fire, and to be unable to get either.  Neither of these returnings was
very pleasant or desirable: no magnet drew me to a given point,
increasing in its strength of attraction the nearer I came.  The return
to Thornfield was yet to be tried.

My journey seemed tedious--very tedious: fifty miles one day, a night
spent at an inn; fifty miles the next day.  During the first twelve hours
I thought of Mrs. Reed in her last moments; I saw her disfigured and
discoloured face, and heard her strangely altered voice.  I mused on the
funeral day, the coffin, the hearse, the black train of tenants and
servants--few was the number of relatives--the gaping vault, the silent
church, the solemn service.  Then I thought of Eliza and Georgiana; I
beheld one the cynosure of a ball-room, the other the inmate of a convent
cell; and I dwelt on and analysed their separate peculiarities of person
and character.  The evening arrival at the great town of--scattered these
thoughts; night gave them quite another turn: laid down on my traveller's
bed, I left reminiscence for anticipation.

I was going back to Thornfield: but how long was I to stay there?  Not
long; of that I was sure.  I had heard from Mrs. Fairfax in the interim
of my absence: the party at the hall was dispersed; Mr. Rochester had
left for London three weeks ago, but he was then expected to return in a
fortnight.  Mrs. Fairfax surmised that he was gone to make arrangements
for his wedding, as he had talked of purchasing a new carriage: she said
the idea of his marrying Miss Ingram still seemed strange to her; but
from what everybody said, and from what she had herself seen, she could
no longer doubt that the event would shortly take place.  "You would be
strangely incredulous if you did doubt it," was my mental comment.  "I
don't doubt it."

The question followed, "Where was I to go?"  I dreamt of Miss Ingram all
the night: in a vivid morning dream I saw her closing the gates of
Thornfield against me and pointing me out another road; and Mr. Rochester
looked on with his arms folded--smiling sardonically, as it seemed, at
both her and me.

I had not notified to Mrs. Fairfax the exact day of my return; for I did
not wish either car or carriage to meet me at Millcote.  I proposed to
walk the distance quietly by myself; and very quietly, after leaving my
box in the ostler's care, did I slip away from the George Inn, about six
o'clock of a June evening, and take the old road to Thornfield: a road
which lay chiefly through fields, and was now little frequented.

It was not a bright or splendid summer evening, though fair and soft: the
haymakers were at work all along the road; and the sky, though far from
cloudless, was such as promised well for the future: its blue--where blue
was visible--was mild and settled, and its cloud strata high and thin.
The west, too, was warm: no watery gleam chilled it--it seemed as if
there was a fire lit, an altar burning behind its screen of marbled
vapour, and out of apertures shone a golden redness.

I felt glad as the road shortened before me: so glad that I stopped once
to ask myself what that joy meant: and to remind reason that it was not
to my home I was going, or to a permanent resting-place, or to a place
where fond friends looked out for me and waited my arrival.  "Mrs.
Fairfax will smile you a calm welcome, to be sure," said I; "and little
Adele will clap her hands and jump to see you: but you know very well you
are thinking of another than they, and that he is not thinking of you."

But what is so headstrong as youth?  What so blind as inexperience?  These
affirmed that it was pleasure enough to have the privilege of again
looking on Mr. Rochester, whether he looked on me or not; and they
added--"Hasten! hasten! be with him while you may: but a few more days or
weeks, at most, and you are parted from him for ever!"  And then I
strangled a new-born agony--a deformed thing which I could not persuade
myself to own and rear--and ran on.

They are making hay, too, in Thornfield meadows: or rather, the labourers
are just quitting their work, and returning home with their rakes on
their shoulders, now, at the hour I arrive.  I have but a field or two to
traverse, and then I shall cross the road and reach the gates.  How full
the hedges are of roses!  But I have no time to gather any; I want to be
at the house.  I passed a tall briar, shooting leafy and flowery branches
across the path; I see the narrow stile with stone steps; and I see--Mr.
Rochester sitting there, a book and a pencil in his hand; he is writing.

Well, he is not a ghost; yet every nerve I have is unstrung: for a moment
I am beyond my own mastery.  What does it mean?  I did not think I should
tremble in this way when I saw him, or lose my voice or the power of
motion in his presence.  I will go back as soon as I can stir: I need not
make an absolute fool of myself.  I know another way to the house.  It
does not signify if I knew twenty ways; for he has seen me.

"Hillo!" he cries; and he puts up his book and his pencil.  "There you
are!  Come on, if you please."

I suppose I do come on; though in what fashion I know not; being scarcely
cognisant of my movements, and solicitous only to appear calm; and, above
all, to control the working muscles of my face--which I feel rebel
insolently against my will, and struggle to express what I had resolved
to conceal.  But I have a veil--it is down: I may make shift yet to
behave with decent composure.

"And this is Jane Eyre?  Are you coming from Millcote, and on foot?
Yes--just one of your tricks: not to send for a carriage, and come
clattering over street and road like a common mortal, but to steal into
the vicinage of your home along with twilight, just as if you were a
dream or a shade.  What the deuce have you done with yourself this last
month?"

"I have been with my aunt, sir, who is dead."

"A true Janian reply!  Good angels be my guard!  She comes from the other
world--from the abode of people who are dead; and tells me so when she
meets me alone here in the gloaming!  If I dared, I'd touch you, to see
if you are substance or shadow, you elf!--but I'd as soon offer to take
hold of a blue \emph{ignis fatuus} light in a marsh.  Truant! truant!" he
added, when he had paused an instant.  "Absent from me a whole month, and
forgetting me quite, I'll be sworn!"

I knew there would be pleasure in meeting my master again, even though
broken by the fear that he was so soon to cease to be my master, and by
the knowledge that I was nothing to him: but there was ever in Mr.
Rochester (so at least I thought) such a wealth of the power of
communicating happiness, that to taste but of the crumbs he scattered to
stray and stranger birds like me, was to feast genially.  His last words
were balm: they seemed to imply that it imported something to him whether
I forgot him or not.  And he had spoken of Thornfield as my home--would
that it were my home!

He did not leave the stile, and I hardly liked to ask to go by.  I
inquired soon if he had not been to London.

"Yes; I suppose you found that out by second-sight."

"Mrs. Fairfax told me in a letter."

"And did she inform you what I went to do?"

"Oh, yes, sir!  Everybody knew your errand."

"You must see the carriage, Jane, and tell me if you don't think it will
suit Mrs. Rochester exactly; and whether she won't look like Queen
Boadicea, leaning back against those purple cushions.  I wish, Jane, I
were a trifle better adapted to match with her externally.  Tell me now,
fairy as you are--can't you give me a charm, or a philter, or something
of that sort, to make me a handsome man?"

"It would be past the power of magic, sir;" and, in thought, I added, "A
loving eye is all the charm needed: to such you are handsome enough; or
rather your sternness has a power beyond beauty."

Mr. Rochester had sometimes read my unspoken thoughts with an acumen to
me incomprehensible: in the present instance he took no notice of my
abrupt vocal response; but he smiled at me with a certain smile he had of
his own, and which he used but on rare occasions.  He seemed to think it
too good for common purposes: it was the real sunshine of feeling--he
shed it over me now.

"Pass, Janet," said he, making room for me to cross the stile: "go up
home, and stay your weary little wandering feet at a friend's threshold."

All I had now to do was to obey him in silence: no need for me to
colloquise further.  I got over the stile without a word, and meant to
leave him calmly.  An impulse held me fast--a force turned me round.  I
said--or something in me said for me, and in spite of me--

"Thank you, Mr. Rochester, for your great kindness.  I am strangely glad
to get back again to you: and wherever you are is my home--my only home."

I walked on so fast that even he could hardly have overtaken me had he
tried.  Little Adele was half wild with delight when she saw me.  Mrs.
Fairfax received me with her usual plain friendliness.  Leah smiled, and
even Sophie bid me "bon soir" with glee.  This was very pleasant; there
is no happiness like that of being loved by your fellow-creatures, and
feeling that your presence is an addition to their comfort.

I that evening shut my eyes resolutely against the future: I stopped my
ears against the voice that kept warning me of near separation and coming
grief.  When tea was over and Mrs. Fairfax had taken her knitting, and I
had assumed a low seat near her, and Adele, kneeling on the carpet, had
nestled close up to me, and a sense of mutual affection seemed to
surround us with a ring of golden peace, I uttered a silent prayer that
we might not be parted far or soon; but when, as we thus sat, Mr.
Rochester entered, unannounced, and looking at us, seemed to take
pleasure in the spectacle of a group so amicable--when he said he
supposed the old lady was all right now that she had got her adopted
daughter back again, and added that he saw Adele was "prete a croquer sa
petite maman Anglaise"--I half ventured to hope that he would, even after
his marriage, keep us together somewhere under the shelter of his
protection, and not quite exiled from the sunshine of his presence.

A fortnight of dubious calm succeeded my return to Thornfield Hall.
Nothing was said of the master's marriage, and I saw no preparation going
on for such an event.  Almost every day I asked Mrs. Fairfax if she had
yet heard anything decided: her answer was always in the negative.  Once
she said she had actually put the question to Mr. Rochester as to when he
was going to bring his bride home; but he had answered her only by a joke
and one of his queer looks, and she could not tell what to make of him.

One thing specially surprised me, and that was, there were no journeyings
backward and forward, no visits to Ingram Park: to be sure it was twenty
miles off, on the borders of another county; but what was that distance
to an ardent lover?  To so practised and indefatigable a horseman as Mr.
Rochester, it would be but a morning's ride.  I began to cherish hopes I
had no right to conceive: that the match was broken off; that rumour had
been mistaken; that one or both parties had changed their minds.  I used
to look at my master's face to see if it were sad or fierce; but I could
not remember the time when it had been so uniformly clear of clouds or
evil feelings.  If, in the moments I and my pupil spent with him, I
lacked spirits and sank into inevitable dejection, he became even gay.
Never had he called me more frequently to his presence; never been kinder
to me when there--and, alas! never had I loved him so well.




\chapter{}


\lettrine{A} splendid Midsummer shone over England: skies so pure, suns so radiant
as were then seen in long succession, seldom favour even singly, our wave-
girt land.  It was as if a band of Italian days had come from the South,
like a flock of glorious passenger birds, and lighted to rest them on the
cliffs of Albion.  The hay was all got in; the fields round Thornfield
were green and shorn; the roads white and baked; the trees were in their
dark prime; hedge and wood, full-leaved and deeply tinted, contrasted
well with the sunny hue of the cleared meadows between.

On Midsummer-eve, Adele, weary with gathering wild strawberries in Hay
Lane half the day, had gone to bed with the sun.  I watched her drop
asleep, and when I left her, I sought the garden.

It was now the sweetest hour of the twenty-four:--"Day its fervid fires
had wasted," and dew fell cool on panting plain and scorched summit.
Where the sun had gone down in simple state--pure of the pomp of
clouds--spread a solemn purple, burning with the light of red jewel and
furnace flame at one point, on one hill-peak, and extending high and
wide, soft and still softer, over half heaven.  The east had its own
charm or fine deep blue, and its own modest gem, a casino and solitary
star: soon it would boast the moon; but she was yet beneath the horizon.

I walked a while on the pavement; but a subtle, well-known scent--that of
a cigar--stole from some window; I saw the library casement open a
handbreadth; I knew I might be watched thence; so I went apart into the
orchard.  No nook in the grounds more sheltered and more Eden-like; it
was full of trees, it bloomed with flowers: a very high wall shut it out
from the court, on one side; on the other, a beech avenue screened it
from the lawn.  At the bottom was a sunk fence; its sole separation from
lonely fields: a winding walk, bordered with laurels and terminating in a
giant horse-chestnut, circled at the base by a seat, led down to the
fence.  Here one could wander unseen.  While such honey-dew fell, such
silence reigned, such gloaming gathered, I felt as if I could haunt such
shade for ever; but in threading the flower and fruit parterres at the
upper part of the enclosure, enticed there by the light the now rising
moon cast on this more open quarter, my step is stayed--not by sound, not
by sight, but once more by a warning fragrance.

Sweet-briar and southernwood, jasmine, pink, and rose have long been
yielding their evening sacrifice of incense: this new scent is neither of
shrub nor flower; it is--I know it well--it is Mr. Rochester's cigar.  I
look round and I listen.  I see trees laden with ripening fruit.  I hear
a nightingale warbling in a wood half a mile off; no moving form is
visible, no coming step audible; but that perfume increases: I must flee.
I make for the wicket leading to the shrubbery, and I see Mr. Rochester
entering.  I step aside into the ivy recess; he will not stay long: he
will soon return whence he came, and if I sit still he will never see me.

But no--eventide is as pleasant to him as to me, and this antique garden
as attractive; and he strolls on, now lifting the gooseberry-tree
branches to look at the fruit, large as plums, with which they are laden;
now taking a ripe cherry from the wall; now stooping towards a knot of
flowers, either to inhale their fragrance or to admire the dew-beads on
their petals.  A great moth goes humming by me; it alights on a plant at
Mr. Rochester's foot: he sees it, and bends to examine it.

"Now, he has his back towards me," thought I, "and he is occupied too;
perhaps, if I walk softly, I can slip away unnoticed."

I trode on an edging of turf that the crackle of the pebbly gravel might
not betray me: he was standing among the beds at a yard or two distant
from where I had to pass; the moth apparently engaged him.  "I shall get
by very well," I meditated.  As I crossed his shadow, thrown long over
the garden by the moon, not yet risen high, he said quietly, without
turning--

"Jane, come and look at this fellow."

I had made no noise: he had not eyes behind--could his shadow feel?  I
started at first, and then I approached him.

"Look at his wings," said he, "he reminds me rather of a West Indian
insect; one does not often see so large and gay a night-rover in England;
there! he is flown."

The moth roamed away.  I was sheepishly retreating also; but Mr.
Rochester followed me, and when we reached the wicket, he said--

"Turn back: on so lovely a night it is a shame to sit in the house; and
surely no one can wish to go to bed while sunset is thus at meeting with
moonrise."

It is one of my faults, that though my tongue is sometimes prompt enough
at an answer, there are times when it sadly fails me in framing an
excuse; and always the lapse occurs at some crisis, when a facile word or
plausible pretext is specially wanted to get me out of painful
embarrassment.  I did not like to walk at this hour alone with Mr.
Rochester in the shadowy orchard; but I could not find a reason to allege
for leaving him.  I followed with lagging step, and thoughts busily bent
on discovering a means of extrication; but he himself looked so composed
and so grave also, I became ashamed of feeling any confusion: the evil--if
evil existent or prospective there was--seemed to lie with me only; his
mind was unconscious and quiet.

"Jane," he recommenced, as we entered the laurel walk, and slowly strayed
down in the direction of the sunk fence and the horse-chestnut,
"Thornfield is a pleasant place in summer, is it not?"

"Yes, sir."

"You must have become in some degree attached to the house,--you, who
have an eye for natural beauties, and a good deal of the organ of
Adhesiveness?"

"I am attached to it, indeed."

"And though I don't comprehend how it is, I perceive you have acquired a
degree of regard for that foolish little child Adele, too; and even for
simple dame Fairfax?"

"Yes, sir; in different ways, I have an affection for both."

"And would be sorry to part with them?"

"Yes."

"Pity!" he said, and sighed and paused.  "It is always the way of events
in this life," he continued presently: "no sooner have you got settled in
a pleasant resting-place, than a voice calls out to you to rise and move
on, for the hour of repose is expired."

"Must I move on, sir?" I asked.  "Must I leave Thornfield?"

"I believe you must, Jane.  I am sorry, Janet, but I believe indeed you
must."

This was a blow: but I did not let it prostrate me.

"Well, sir, I shall be ready when the order to march comes."

"It is come now--I must give it to-night."

"Then you \emph{are} going to be married, sir?"

"Ex-act-ly--pre-cise-ly: with your usual acuteness, you have hit the nail
straight on the head."

"Soon, sir?"

"Very soon, my--that is, Miss Eyre: and you'll remember, Jane, the first
time I, or Rumour, plainly intimated to you that it was my intention to
put my old bachelor's neck into the sacred noose, to enter into the holy
estate of matrimony--to take Miss Ingram to my bosom, in short (she's an
extensive armful: but that's not to the point--one can't have too much of
such a very excellent thing as my beautiful Blanche): well, as I was
saying--listen to me, Jane!  You're not turning your head to look after
more moths, are you?  That was only a lady-clock, child, 'flying away
home.'  I wish to remind you that it was you who first said to me, with
that discretion I respect in you--with that foresight, prudence, and
humility which befit your responsible and dependent position--that in
case I married Miss Ingram, both you and little Adele had better trot
forthwith.  I pass over the sort of slur conveyed in this suggestion on
the character of my beloved; indeed, when you are far away, Janet, I'll
try to forget it: I shall notice only its wisdom; which is such that I
have made it my law of action.  Adele must go to school; and you, Miss
Eyre, must get a new situation."

"Yes, sir, I will advertise immediately: and meantime, I suppose--" I was
going to say, "I suppose I may stay here, till I find another shelter to
betake myself to:" but I stopped, feeling it would not do to risk a long
sentence, for my voice was not quite under command.

"In about a month I hope to be a bridegroom," continued Mr. Rochester;
"and in the interim, I shall myself look out for employment and an asylum
for you."

"Thank you, sir; I am sorry to give--"

"Oh, no need to apologise!  I consider that when a dependent does her
duty as well as you have done yours, she has a sort of claim upon her
employer for any little assistance he can conveniently render her; indeed
I have already, through my future mother-in-law, heard of a place that I
think will suit: it is to undertake the education of the five daughters
of Mrs. Dionysius O'Gall of Bitternutt Lodge, Connaught, Ireland.  You'll
like Ireland, I think: they're such warm-hearted people there, they say."

"It is a long way off, sir."

"No matter--a girl of your sense will not object to the voyage or the
distance."

"Not the voyage, but the distance: and then the sea is a barrier--"

"From what, Jane?"

"From England and from Thornfield: and--"

"Well?"

"From \emph{you}, sir."

I said this almost involuntarily, and, with as little sanction of free
will, my tears gushed out.  I did not cry so as to be heard, however; I
avoided sobbing.  The thought of Mrs. O'Gall and Bitternutt Lodge struck
cold to my heart; and colder the thought of all the brine and foam,
destined, as it seemed, to rush between me and the master at whose side I
now walked, and coldest the remembrance of the wider ocean--wealth,
caste, custom intervened between me and what I naturally and inevitably
loved.

"It is a long way," I again said.

"It is, to be sure; and when you get to Bitternutt Lodge, Connaught,
Ireland, I shall never see you again, Jane: that's morally certain.  I
never go over to Ireland, not having myself much of a fancy for the
country.  We have been good friends, Jane; have we not?"

"Yes, sir."

"And when friends are on the eve of separation, they like to spend the
little time that remains to them close to each other.  Come! we'll talk
over the voyage and the parting quietly half-an-hour or so, while the
stars enter into their shining life up in heaven yonder: here is the
chestnut tree: here is the bench at its old roots.  Come, we will sit
there in peace to-night, though we should never more be destined to sit
there together."  He seated me and himself.

"It is a long way to Ireland, Janet, and I am sorry to send my little
friend on such weary travels: but if I can't do better, how is it to be
helped?  Are you anything akin to me, do you think, Jane?"

I could risk no sort of answer by this time: my heart was still.

"Because," he said, "I sometimes have a queer feeling with regard to
you--especially when you are near me, as now: it is as if I had a string
somewhere under my left ribs, tightly and inextricably knotted to a
similar string situated in the corresponding quarter of your little
frame.  And if that boisterous Channel, and two hundred miles or so of
land come broad between us, I am afraid that cord of communion will be
snapt; and then I've a nervous notion I should take to bleeding inwardly.
As for you,--you'd forget me."

"That I \emph{never} should, sir: you know--"  Impossible to proceed.

"Jane, do you hear that nightingale singing in the wood?  Listen!"

In listening, I sobbed convulsively; for I could repress what I endured
no longer; I was obliged to yield, and I was shaken from head to foot
with acute distress.  When I did speak, it was only to express an
impetuous wish that I had never been born, or never come to Thornfield.

"Because you are sorry to leave it?"

The vehemence of emotion, stirred by grief and love within me, was
claiming mastery, and struggling for full sway, and asserting a right to
predominate, to overcome, to live, rise, and reign at last: yes,--and to
speak.

"I grieve to leave Thornfield: I love Thornfield:--I love it, because I
have lived in it a full and delightful life,--momentarily at least.  I
have not been trampled on.  I have not been petrified.  I have not been
buried with inferior minds, and excluded from every glimpse of communion
with what is bright and energetic and high.  I have talked, face to face,
with what I reverence, with what I delight in,--with an original, a
vigorous, an expanded mind.  I have known you, Mr. Rochester; and it
strikes me with terror and anguish to feel I absolutely must be torn from
you for ever.  I see the necessity of departure; and it is like looking
on the necessity of death."

"Where do you see the necessity?" he asked suddenly.

"Where?  You, sir, have placed it before me."

"In what shape?"

"In the shape of Miss Ingram; a noble and beautiful woman,--your bride."

"My bride!  What bride?  I have no bride!"

"But you will have."

"Yes;--I will!--I will!"  He set his teeth.

"Then I must go:--you have said it yourself."

"No: you must stay!  I swear it--and the oath shall be kept."

"I tell you I must go!" I retorted, roused to something like passion.  "Do
you think I can stay to become nothing to you?  Do you think I am an
automaton?--a machine without feelings? and can bear to have my morsel of
bread snatched from my lips, and my drop of living water dashed from my
cup?  Do you think, because I am poor, obscure, plain, and little, I am
soulless and heartless?  You think wrong!--I have as much soul as
you,--and full as much heart!  And if God had gifted me with some beauty
and much wealth, I should have made it as hard for you to leave me, as it
is now for me to leave you.  I am not talking to you now through the
medium of custom, conventionalities, nor even of mortal flesh;--it is my
spirit that addresses your spirit; just as if both had passed through the
grave, and we stood at God's feet, equal,--as we are!"

"As we are!" repeated Mr. Rochester--"so," he added, enclosing me in his
arms.  Gathering me to his breast, pressing his lips on my lips: "so,
Jane!"

"Yes, so, sir," I rejoined: "and yet not so; for you are a married man--or
as good as a married man, and wed to one inferior to you--to one with
whom you have no sympathy--whom I do not believe you truly love; for I
have seen and heard you sneer at her.  I would scorn such a union:
therefore I am better than you--let me go!"

"Where, Jane?  To Ireland?"

"Yes--to Ireland.  I have spoken my mind, and can go anywhere now."

"Jane, be still; don't struggle so, like a wild frantic bird that is
rending its own plumage in its desperation."

"I am no bird; and no net ensnares me; I am a free human being with an
independent will, which I now exert to leave you."

Another effort set me at liberty, and I stood erect before him.

"And your will shall decide your destiny," he said: "I offer you my hand,
my heart, and a share of all my possessions."

"You play a farce, which I merely laugh at."

"I ask you to pass through life at my side--to be my second self, and
best earthly companion."

"For that fate you have already made your choice, and must abide by it."

"Jane, be still a few moments: you are over-excited: I will be still
too."

A waft of wind came sweeping down the laurel-walk, and trembled through
the boughs of the chestnut: it wandered away--away--to an indefinite
distance--it died.  The nightingale's song was then the only voice of the
hour: in listening to it, I again wept.  Mr. Rochester sat quiet, looking
at me gently and seriously.  Some time passed before he spoke; he at last
said--

"Come to my side, Jane, and let us explain and understand one another."

"I will never again come to your side: I am torn away now, and cannot
return."

"But, Jane, I summon you as my wife: it is you only I intend to marry."

I was silent: I thought he mocked me.

"Come, Jane--come hither."

"Your bride stands between us."

He rose, and with a stride reached me.

"My bride is here," he said, again drawing me to him, "because my equal
is here, and my likeness.  Jane, will you marry me?"

Still I did not answer, and still I writhed myself from his grasp: for I
was still incredulous.

"Do you doubt me, Jane?"

"Entirely."

"You have no faith in me?"

"Not a whit."

"Am I a liar in your eyes?" he asked passionately.  "Little sceptic, you
\emph{shall} be convinced.  What love have I for Miss Ingram?  None: and that
you know.  What love has she for me?  None: as I have taken pains to
prove: I caused a rumour to reach her that my fortune was not a third of
what was supposed, and after that I presented myself to see the result;
it was coldness both from her and her mother.  I would not--I could
not--marry Miss Ingram.  You--you strange, you almost unearthly thing!--I
love as my own flesh.  You--poor and obscure, and small and plain as you
are--I entreat to accept me as a husband."

"What, me!" I ejaculated, beginning in his earnestness--and especially in
his incivility--to credit his sincerity: "me who have not a friend in the
world but you--if you are my friend: not a shilling but what you have
given me?"

"You, Jane, I must have you for my own--entirely my own.  Will you be
mine?  Say yes, quickly."

"Mr. Rochester, let me look at your face: turn to the moonlight."

"Why?"

"Because I want to read your countenance--turn!"

"There! you will find it scarcely more legible than a crumpled, scratched
page.  Read on: only make haste, for I suffer."

His face was very much agitated and very much flushed, and there were
strong workings in the features, and strange gleams in the eyes.

"Oh, Jane, you torture me!" he exclaimed.  "With that searching and yet
faithful and generous look, you torture me!"

"How can I do that?  If you are true, and your offer real, my only
feelings to you must be gratitude and devotion--they cannot torture."

"Gratitude!" he ejaculated; and added wildly--"Jane accept me quickly.
Say, Edward--give me my name--Edward--I will marry you."

"Are you in earnest?  Do you truly love me?  Do you sincerely wish me to
be your wife?"

"I do; and if an oath is necessary to satisfy you, I swear it."

"Then, sir, I will marry you."

"Edward--my little wife!"

"Dear Edward!"

"Come to me--come to me entirely now," said he; and added, in his deepest
tone, speaking in my ear as his cheek was laid on mine, "Make my
happiness--I will make yours."

"God pardon me!" he subjoined ere long; "and man meddle not with me: I
have her, and will hold her."

"There is no one to meddle, sir.  I have no kindred to interfere."

"No--that is the best of it," he said.  And if I had loved him less I
should have thought his accent and look of exultation savage; but,
sitting by him, roused from the nightmare of parting--called to the
paradise of union--I thought only of the bliss given me to drink in so
abundant a flow.  Again and again he said, "Are you happy, Jane?"  And
again and again I answered, "Yes."  After which he murmured, "It will
atone--it will atone.  Have I not found her friendless, and cold, and
comfortless?  Will I not guard, and cherish, and solace her?  Is there
not love in my heart, and constancy in my resolves?  It will expiate at
God's tribunal.  I know my Maker sanctions what I do.  For the world's
judgment--I wash my hands thereof.  For man's opinion--I defy it."

But what had befallen the night?  The moon was not yet set, and we were
all in shadow: I could scarcely see my master's face, near as I was.  And
what ailed the chestnut tree? it writhed and groaned; while wind roared
in the laurel walk, and came sweeping over us.

"We must go in," said Mr. Rochester: "the weather changes.  I could have
sat with thee till morning, Jane."

"And so," thought I, "could I with you."  I should have said so, perhaps,
but a livid, vivid spark leapt out of a cloud at which I was looking, and
there was a crack, a crash, and a close rattling peal; and I thought only
of hiding my dazzled eyes against Mr. Rochester's shoulder.

The rain rushed down.  He hurried me up the walk, through the grounds,
and into the house; but we were quite wet before we could pass the
threshold.  He was taking off my shawl in the hall, and shaking the water
out of my loosened hair, when Mrs. Fairfax emerged from her room.  I did
not observe her at first, nor did Mr. Rochester.  The lamp was lit.  The
clock was on the stroke of twelve.

"Hasten to take off your wet things," said he; "and before you go, good-
night--good-night, my darling!"

He kissed me repeatedly.  When I looked up, on leaving his arms, there
stood the widow, pale, grave, and amazed.  I only smiled at her, and ran
upstairs.  "Explanation will do for another time," thought I.  Still,
when I reached my chamber, I felt a pang at the idea she should even
temporarily misconstrue what she had seen.  But joy soon effaced every
other feeling; and loud as the wind blew, near and deep as the thunder
crashed, fierce and frequent as the lightning gleamed, cataract-like as
the rain fell during a storm of two hours' duration, I experienced no
fear and little awe.  Mr. Rochester came thrice to my door in the course
of it, to ask if I was safe and tranquil: and that was comfort, that was
strength for anything.

Before I left my bed in the morning, little Adele came running in to tell
me that the great horse-chestnut at the bottom of the orchard had been
struck by lightning in the night, and half of it split away.




\chapter{}


\lettrine{A}{s} I rose and dressed, I thought over what had happened, and wondered if
it were a dream.  I could not be certain of the reality till I had seen
Mr. Rochester again, and heard him renew his words of love and promise.

While arranging my hair, I looked at my face in the glass, and felt it
was no longer plain: there was hope in its aspect and life in its colour;
and my eyes seemed as if they had beheld the fount of fruition, and
borrowed beams from the lustrous ripple.  I had often been unwilling to
look at my master, because I feared he could not be pleased at my look;
but I was sure I might lift my face to his now, and not cool his
affection by its expression.  I took a plain but clean and light summer
dress from my drawer and put it on: it seemed no attire had ever so well
become me, because none had I ever worn in so blissful a mood.

I was not surprised, when I ran down into the hall, to see that a
brilliant June morning had succeeded to the tempest of the night; and to
feel, through the open glass door, the breathing of a fresh and fragrant
breeze.  Nature must be gladsome when I was so happy.  A beggar-woman and
her little boy--pale, ragged objects both--were coming up the walk, and I
ran down and gave them all the money I happened to have in my purse--some
three or four shillings: good or bad, they must partake of my jubilee.
The rooks cawed, and blither birds sang; but nothing was so merry or so
musical as my own rejoicing heart.

Mrs. Fairfax surprised me by looking out of the window with a sad
countenance, and saying gravely--"Miss Eyre, will you come to breakfast?"
During the meal she was quiet and cool: but I could not undeceive her
then.  I must wait for my master to give explanations; and so must she.  I
ate what I could, and then I hastened upstairs.  I met Adele leaving the
schoolroom.

"Where are you going?  It is time for lessons."

"Mr. Rochester has sent me away to the nursery."

"Where is he?"

"In there," pointing to the apartment she had left; and I went in, and
there he stood.

"Come and bid me good-morning," said he.  I gladly advanced; and it was
not merely a cold word now, or even a shake of the hand that I received,
but an embrace and a kiss.  It seemed natural: it seemed genial to be so
well loved, so caressed by him.

"Jane, you look blooming, and smiling, and pretty," said he: "truly
pretty this morning.  Is this my pale, little elf?  Is this my mustard-
seed?  This little sunny-faced girl with the dimpled cheek and rosy lips;
the satin-smooth hazel hair, and the radiant hazel eyes?"  (I had green
eyes, reader; but you must excuse the mistake: for him they were
new-dyed, I suppose.)

"It is Jane Eyre, sir."

"Soon to be Jane Rochester," he added: "in four weeks, Janet; not a day
more.  Do you hear that?"

I did, and I could not quite comprehend it: it made me giddy.  The
feeling, the announcement sent through me, was something stronger than
was consistent with joy--something that smote and stunned.  It was, I
think almost fear.

"You blushed, and now you are white, Jane: what is that for?"

"Because you gave me a new name--Jane Rochester; and it seems so
strange."

"Yes, Mrs. Rochester," said he; "young Mrs. Rochester--Fairfax
Rochester's girl-bride."

"It can never be, sir; it does not sound likely.  Human beings never
enjoy complete happiness in this world.  I was not born for a different
destiny to the rest of my species: to imagine such a lot befalling me is
a fairy tale--a day-dream."

"Which I can and will realise.  I shall begin to-day.  This morning I
wrote to my banker in London to send me certain jewels he has in his
keeping,--heirlooms for the ladies of Thornfield.  In a day or two I hope
to pour them into your lap: for every privilege, every attention shall be
yours that I would accord a peer's daughter, if about to marry her."

"Oh, sir!--never rain jewels!  I don't like to hear them spoken of.
Jewels for Jane Eyre sounds unnatural and strange: I would rather not
have them."

"I will myself put the diamond chain round your neck, and the circlet on
your forehead,--which it will become: for nature, at least, has stamped
her patent of nobility on this brow, Jane; and I will clasp the bracelets
on these fine wrists, and load these fairy-like fingers with rings."

"No, no, sir! think of other subjects, and speak of other things, and in
another strain.  Don't address me as if I were a beauty; I am your plain,
Quakerish governess."

"You are a beauty in my eyes, and a beauty just after the desire of my
heart,--delicate and aerial."

"Puny and insignificant, you mean.  You are dreaming, sir,--or you are
sneering.  For God's sake don't be ironical!"

"I will make the world acknowledge you a beauty, too," he went on, while
I really became uneasy at the strain he had adopted, because I felt he
was either deluding himself or trying to delude me.  "I will attire my
Jane in satin and lace, and she shall have roses in her hair; and I will
cover the head I love best with a priceless veil."

"And then you won't know me, sir; and I shall not be your Jane Eyre any
longer, but an ape in a harlequin's jacket--a jay in borrowed plumes.  I
would as soon see you, Mr. Rochester, tricked out in stage-trappings, as
myself clad in a court-lady's robe; and I don't call you handsome, sir,
though I love you most dearly: far too dearly to flatter you.  Don't
flatter me."

He pursued his theme, however, without noticing my deprecation.  "This
very day I shall take you in the carriage to Millcote, and you must
choose some dresses for yourself.  I told you we shall be married in four
weeks.  The wedding is to take place quietly, in the church down below
yonder; and then I shall waft you away at once to town.  After a brief
stay there, I shall bear my treasure to regions nearer the sun: to French
vineyards and Italian plains; and she shall see whatever is famous in old
story and in modern record: she shall taste, too, of the life of cities;
and she shall learn to value herself by just comparison with others."

"Shall I travel?--and with you, sir?"

"You shall sojourn at Paris, Rome, and Naples: at Florence, Venice, and
Vienna: all the ground I have wandered over shall be re-trodden by you:
wherever I stamped my hoof, your sylph's foot shall step also.  Ten years
since, I flew through Europe half mad; with disgust, hate, and rage as my
companions: now I shall revisit it healed and cleansed, with a very angel
as my comforter."

I laughed at him as he said this.  "I am not an angel," I asserted; "and
I will not be one till I die: I will be myself.  Mr. Rochester, you must
neither expect nor exact anything celestial of me--for you will not get
it, any more than I shall get it of you: which I do not at all
anticipate."

"What do you anticipate of me?"

"For a little while you will perhaps be as you are now,--a very little
while; and then you will turn cool; and then you will be capricious; and
then you will be stern, and I shall have much ado to please you: but when
you get well used to me, you will perhaps like me again,--\emph{like} me, I
say, not \emph{love} me.  I suppose your love will effervesce in six months,
or less.  I have observed in books written by men, that period assigned
as the farthest to which a husband's ardour extends.  Yet, after all, as
a friend and companion, I hope never to become quite distasteful to my
dear master."

"Distasteful! and like you again!  I think I shall like you again, and
yet again: and I will make you confess I do not only \emph{like}, but \emph{love}
you--with truth, fervour, constancy."

"Yet are you not capricious, sir?"

"To women who please me only by their faces, I am the very devil when I
find out they have neither souls nor hearts--when they open to me a
perspective of flatness, triviality, and perhaps imbecility, coarseness,
and ill-temper: but to the clear eye and eloquent tongue, to the soul
made of fire, and the character that bends but does not break--at once
supple and stable, tractable and consistent--I am ever tender and true."

"Had you ever experience of such a character, sir?  Did you ever love
such an one?"

"I love it now."

"But before me: if I, indeed, in any respect come up to your difficult
standard?"

"I never met your likeness.  Jane, you please me, and you master me--you
seem to submit, and I like the sense of pliancy you impart; and while I
am twining the soft, silken skein round my finger, it sends a thrill up
my arm to my heart.  I am influenced--conquered; and the influence is
sweeter than I can express; and the conquest I undergo has a witchery
beyond any triumph I can win.  Why do you smile, Jane?  What does that
inexplicable, that uncanny turn of countenance mean?"

"I was thinking, sir (you will excuse the idea; it was involuntary), I
was thinking of Hercules and Samson with their charmers--"

"You were, you little elfish--"

"Hush, sir!  You don't talk very wisely just now; any more than those
gentlemen acted very wisely.  However, had they been married, they would
no doubt by their severity as husbands have made up for their softness as
suitors; and so will you, I fear.  I wonder how you will answer me a year
hence, should I ask a favour it does not suit your convenience or
pleasure to grant."

"Ask me something now, Jane,--the least thing: I desire to be entreated--"

"Indeed I will, sir; I have my petition all ready."

"Speak!  But if you look up and smile with that countenance, I shall
swear concession before I know to what, and that will make a fool of me."

"Not at all, sir; I ask only this: don't send for the jewels, and don't
crown me with roses: you might as well put a border of gold lace round
that plain pocket handkerchief you have there."

"I might as well 'gild refined gold.'  I know it: your request is granted
then--for the time.  I will remand the order I despatched to my banker.
But you have not yet asked for anything; you have prayed a gift to be
withdrawn: try again."

"Well then, sir, have the goodness to gratify my curiosity, which is much
piqued on one point."

He looked disturbed.  "What? what?" he said hastily.  "Curiosity is a
dangerous petition: it is well I have not taken a vow to accord every
request--"

"But there can be no danger in complying with this, sir."

"Utter it, Jane: but I wish that instead of a mere inquiry into, perhaps,
a secret, it was a wish for half my estate."

"Now, King Ahasuerus!  What do I want with half your estate?  Do you
think I am a Jew-usurer, seeking good investment in land?  I would much
rather have all your confidence.  You will not exclude me from your
confidence if you admit me to your heart?"

"You are welcome to all my confidence that is worth having, Jane; but for
God's sake, don't desire a useless burden!  Don't long for poison--don't
turn out a downright Eve on my hands!"

"Why not, sir?  You have just been telling me how much you liked to be
conquered, and how pleasant over-persuasion is to you.  Don't you think I
had better take advantage of the confession, and begin and coax and
entreat--even cry and be sulky if necessary--for the sake of a mere essay
of my power?"

"I dare you to any such experiment.  Encroach, presume, and the game is
up."

"Is it, sir?  You soon give in.  How stern you look now!  Your eyebrows
have become as thick as my finger, and your forehead resembles what, in
some very astonishing poetry, I once saw styled, 'a blue-piled
thunderloft.'  That will be your married look, sir, I suppose?"

"If that will be \emph{your} married look, I, as a Christian, will soon give
up the notion of consorting with a mere sprite or salamander.  But what
had you to ask, thing,--out with it?"

"There, you are less than civil now; and I like rudeness a great deal
better than flattery.  I had rather be a \emph{thing} than an angel.  This is
what I have to ask,--Why did you take such pains to make me believe you
wished to marry Miss Ingram?"

"Is that all?  Thank God it is no worse!"  And now he unknit his black
brows; looked down, smiling at me, and stroked my hair, as if well
pleased at seeing a danger averted.  "I think I may confess," he
continued, "even although I should make you a little indignant, Jane--and
I have seen what a fire-spirit you can be when you are indignant.  You
glowed in the cool moonlight last night, when you mutinied against fate,
and claimed your rank as my equal.  Janet, by-the-bye, it was you who
made me the offer."

"Of course I did.  But to the point if you please, sir--Miss Ingram?"

"Well, I feigned courtship of Miss Ingram, because I wished to render you
as madly in love with me as I was with you; and I knew jealousy would be
the best ally I could call in for the furtherance of that end."

"Excellent!  Now you are small--not one whit bigger than the end of my
little finger.  It was a burning shame and a scandalous disgrace to act
in that way.  Did you think nothing of Miss Ingram's feelings, sir?"

"Her feelings are concentrated in one--pride; and that needs humbling.
Were you jealous, Jane?"

"Never mind, Mr. Rochester: it is in no way interesting to you to know
that.  Answer me truly once more.  Do you think Miss Ingram will not
suffer from your dishonest coquetry?  Won't she feel forsaken and
deserted?"

"Impossible!--when I told you how she, on the contrary, deserted me: the
idea of my insolvency cooled, or rather extinguished, her flame in a
moment."

"You have a curious, designing mind, Mr. Rochester.  I am afraid your
principles on some points are eccentric."

"My principles were never trained, Jane: they may have grown a little
awry for want of attention."

"Once again, seriously; may I enjoy the great good that has been
vouchsafed to me, without fearing that any one else is suffering the
bitter pain I myself felt a while ago?"

"That you may, my good little girl: there is not another being in the
world has the same pure love for me as yourself--for I lay that pleasant
unction to my soul, Jane, a belief in your affection."

I turned my lips to the hand that lay on my shoulder.  I loved him very
much--more than I could trust myself to say--more than words had power to
express.

"Ask something more," he said presently; "it is my delight to be
entreated, and to yield."

I was again ready with my request.  "Communicate your intentions to Mrs.
Fairfax, sir: she saw me with you last night in the hall, and she was
shocked.  Give her some explanation before I see her again.  It pains me
to be misjudged by so good a woman."

"Go to your room, and put on your bonnet," he replied.  "I mean you to
accompany me to Millcote this morning; and while you prepare for the
drive, I will enlighten the old lady's understanding.  Did she think,
Janet, you had given the world for love, and considered it well lost?"

"I believe she thought I had forgotten my station, and yours, sir."

"Station! station!--your station is in my heart, and on the necks of
those who would insult you, now or hereafter.--Go."

I was soon dressed; and when I heard Mr. Rochester quit Mrs. Fairfax's
parlour, I hurried down to it.  The old lady, had been reading her
morning portion of Scripture--the Lesson for the day; her Bible lay open
before her, and her spectacles were upon it.  Her occupation, suspended
by Mr. Rochester's announcement, seemed now forgotten: her eyes, fixed on
the blank wall opposite, expressed the surprise of a quiet mind stirred
by unwonted tidings.  Seeing me, she roused herself: she made a sort of
effort to smile, and framed a few words of congratulation; but the smile
expired, and the sentence was abandoned unfinished.  She put up her
spectacles, shut the Bible, and pushed her chair back from the table.

"I feel so astonished," she began, "I hardly know what to say to you,
Miss Eyre.  I have surely not been dreaming, have I?  Sometimes I half
fall asleep when I am sitting alone and fancy things that have never
happened.  It has seemed to me more than once when I have been in a doze,
that my dear husband, who died fifteen years since, has come in and sat
down beside me; and that I have even heard him call me by my name, Alice,
as he used to do.  Now, can you tell me whether it is actually true that
Mr. Rochester has asked you to marry him?  Don't laugh at me.  But I
really thought he came in here five minutes ago, and said that in a month
you would be his wife."

"He has said the same thing to me," I replied.

"He has!  Do you believe him?  Have you accepted him?"

"Yes."

She looked at me bewildered.  "I could never have thought it.  He is a
proud man: all the Rochesters were proud: and his father, at least, liked
money.  He, too, has always been called careful.  He means to marry you?"

"He tells me so."

She surveyed my whole person: in her eyes I read that they had there
found no charm powerful enough to solve the enigma.

"It passes me!" she continued; "but no doubt, it is true since you say
so.  How it will answer, I cannot tell: I really don't know.  Equality of
position and fortune is often advisable in such cases; and there are
twenty years of difference in your ages.  He might almost be your
father."

"No, indeed, Mrs. Fairfax!" exclaimed I, nettled; "he is nothing like my
father!  No one, who saw us together, would suppose it for an instant.
Mr. Rochester looks as young, and is as young, as some men at five-and-
twenty."

"Is it really for love he is going to marry you?" she asked.

I was so hurt by her coldness and scepticism, that the tears rose to my
eyes.

"I am sorry to grieve you," pursued the widow; "but you are so young, and
so little acquainted with men, I wished to put you on your guard.  It is
an old saying that 'all is not gold that glitters;' and in this case I do
fear there will be something found to be different to what either you or
I expect."

"Why?--am I a monster?" I said: "is it impossible that Mr. Rochester
should have a sincere affection for me?"

"No: you are very well; and much improved of late; and Mr. Rochester, I
daresay, is fond of you.  I have always noticed that you were a sort of
pet of his.  There are times when, for your sake, I have been a little
uneasy at his marked preference, and have wished to put you on your
guard: but I did not like to suggest even the possibility of wrong.  I
knew such an idea would shock, perhaps offend you; and you were so
discreet, and so thoroughly modest and sensible, I hoped you might be
trusted to protect yourself.  Last night I cannot tell you what I
suffered when I sought all over the house, and could find you nowhere,
nor the master either; and then, at twelve o'clock, saw you come in with
him."

"Well, never mind that now," I interrupted impatiently; "it is enough
that all was right."

"I hope all will be right in the end," she said: "but believe me, you
cannot be too careful.  Try and keep Mr. Rochester at a distance:
distrust yourself as well as him.  Gentlemen in his station are not
accustomed to marry their governesses."

I was growing truly irritated: happily, Adele ran in.

"Let me go,--let me go to Millcote too!" she cried.  "Mr. Rochester
won't: though there is so much room in the new carriage.  Beg him to let
me go mademoiselle."

"That I will, Adele;" and I hastened away with her, glad to quit my
gloomy monitress.  The carriage was ready: they were bringing it round to
the front, and my master was pacing the pavement, Pilot following him
backwards and forwards.

"Adele may accompany us, may she not, sir?"

"I told her no.  I'll have no brats!--I'll have only you."

"Do let her go, Mr. Rochester, if you please: it would be better."

"Not it: she will be a restraint."

He was quite peremptory, both in look and voice.  The chill of Mrs.
Fairfax's warnings, and the damp of her doubts were upon me: something of
unsubstantiality and uncertainty had beset my hopes.  I half lost the
sense of power over him.  I was about mechanically to obey him, without
further remonstrance; but as he helped me into the carriage, he looked at
my face.

"What is the matter?" he asked; "all the sunshine is gone.  Do you really
wish the bairn to go?  Will it annoy you if she is left behind?"

"I would far rather she went, sir."

"Then off for your bonnet, and back like a flash of lightning!" cried he
to Adele.

She obeyed him with what speed she might.

"After all, a single morning's interruption will not matter much," said
he, "when I mean shortly to claim you--your thoughts, conversation, and
company--for life."

Adele, when lifted in, commenced kissing me, by way of expressing her
gratitude for my intercession: she was instantly stowed away into a
corner on the other side of him.  She then peeped round to where I sat;
so stern a neighbour was too restrictive to him, in his present fractious
mood, she dared whisper no observations, nor ask of him any information.

"Let her come to me," I entreated: "she will, perhaps, trouble you, sir:
there is plenty of room on this side."

He handed her over as if she had been a lapdog.  "I'll send her to school
yet," he said, but now he was smiling.

Adele heard him, and asked if she was to go to school "sans
mademoiselle?"

"Yes," he replied, "absolutely sans mademoiselle; for I am to take
mademoiselle to the moon, and there I shall seek a cave in one of the
white valleys among the volcano-tops, and mademoiselle shall live with me
there, and only me."

"She will have nothing to eat: you will starve her," observed Adele.

"I shall gather manna for her morning and night: the plains and hillsides
in the moon are bleached with manna, Adele."

"She will want to warm herself: what will she do for a fire?"

"Fire rises out of the lunar mountains: when she is cold, I'll carry her
up to a peak, and lay her down on the edge of a crater."

"Oh, qu' elle y sera mal--peu comfortable!  And her clothes, they will
wear out: how can she get new ones?"

Mr. Rochester professed to be puzzled.  "Hem!" said he.  "What would you
do, Adele?  Cudgel your brains for an expedient.  How would a white or a
pink cloud answer for a gown, do you think?  And one could cut a pretty
enough scarf out of a rainbow."

"She is far better as she is," concluded Adele, after musing some time:
"besides, she would get tired of living with only you in the moon.  If I
were mademoiselle, I would never consent to go with you."

"She has consented: she has pledged her word."

"But you can't get her there; there is no road to the moon: it is all
air; and neither you nor she can fly."

"Adele, look at that field."  We were now outside Thornfield gates, and
bowling lightly along the smooth road to Millcote, where the dust was
well laid by the thunderstorm, and, where the low hedges and lofty timber
trees on each side glistened green and rain-refreshed.

"In that field, Adele, I was walking late one evening about a fortnight
since--the evening of the day you helped me to make hay in the orchard
meadows; and, as I was tired with raking swaths, I sat down to rest me on
a stile; and there I took out a little book and a pencil, and began to
write about a misfortune that befell me long ago, and a wish I had for
happy days to come: I was writing away very fast, though daylight was
fading from the leaf, when something came up the path and stopped two
yards off me.  I looked at it.  It was a little thing with a veil of
gossamer on its head.  I beckoned it to come near me; it stood soon at my
knee.  I never spoke to it, and it never spoke to me, in words; but I
read its eyes, and it read mine; and our speechless colloquy was to this
effect--

"It was a fairy, and come from Elf-land, it said; and its errand was to
make me happy: I must go with it out of the common world to a lonely
place--such as the moon, for instance--and it nodded its head towards her
horn, rising over Hay-hill: it told me of the alabaster cave and silver
vale where we might live.  I said I should like to go; but reminded it,
as you did me, that I had no wings to fly.

"'Oh,' returned the fairy, 'that does not signify!  Here is a talisman
will remove all difficulties;' and she held out a pretty gold ring.  'Put
it,' she said, 'on the fourth finger of my left hand, and I am yours, and
you are mine; and we shall leave earth, and make our own heaven yonder.'
She nodded again at the moon.  The ring, Adele, is in my breeches-pocket,
under the disguise of a sovereign: but I mean soon to change it to a ring
again."

"But what has mademoiselle to do with it?  I don't care for the fairy:
you said it was mademoiselle you would take to the moon?"

"Mademoiselle is a fairy," he said, whispering mysteriously.  Whereupon I
told her not to mind his badinage; and she, on her part, evinced a fund
of genuine French scepticism: denominating Mr. Rochester "un vrai
menteur," and assuring him that she made no account whatever of his
"contes de fee," and that "du reste, il n'y avait pas de fees, et quand
meme il y en avait:" she was sure they would never appear to him, nor
ever give him rings, or offer to live with him in the moon.

The hour spent at Millcote was a somewhat harassing one to me.  Mr.
Rochester obliged me to go to a certain silk warehouse: there I was
ordered to choose half-a-dozen dresses.  I hated the business, I begged
leave to defer it: no--it should be gone through with now.  By dint of
entreaties expressed in energetic whispers, I reduced the half-dozen to
two: these however, he vowed he would select himself.  With anxiety I
watched his eye rove over the gay stores: he fixed on a rich silk of the
most brilliant amethyst dye, and a superb pink satin.  I told him in a
new series of whispers, that he might as well buy me a gold gown and a
silver bonnet at once: I should certainly never venture to wear his
choice.  With infinite difficulty, for he was stubborn as a stone, I
persuaded him to make an exchange in favour of a sober black satin and
pearl-grey silk.  "It might pass for the present," he said; "but he would
yet see me glittering like a parterre."

Glad was I to get him out of the silk warehouse, and then out of a
jewellers shop: the more he bought me, the more my cheek burned with a
sense of annoyance and degradation.  As we re-entered the carriage, and I
sat back feverish and fagged, I remembered what, in the hurry of events,
dark and bright, I had wholly forgotten--the letter of my uncle, John
Eyre, to Mrs. Reed: his intention to adopt me and make me his legatee.
"It would, indeed, be a relief," I thought, "if I had ever so small an
independency; I never can bear being dressed like a doll by Mr.
Rochester, or sitting like a second Danae with the golden shower falling
daily round me.  I will write to Madeira the moment I get home, and tell
my uncle John I am going to be married, and to whom: if I had but a
prospect of one day bringing Mr. Rochester an accession of fortune, I
could better endure to be kept by him now."  And somewhat relieved by
this idea (which I failed not to execute that day), I ventured once more
to meet my master's and lover's eye, which most pertinaciously sought
mine, though I averted both face and gaze.  He smiled; and I thought his
smile was such as a sultan might, in a blissful and fond moment, bestow
on a slave his gold and gems had enriched: I crushed his hand, which was
ever hunting mine, vigorously, and thrust it back to him red with the
passionate pressure.

"You need not look in that way," I said; "if you do, I'll wear nothing
but my old Lowood frocks to the end of the chapter.  I'll be married in
this lilac gingham: you may make a dressing-gown for yourself out of the
pearl-grey silk, and an infinite series of waistcoats out of the black
satin."

He chuckled; he rubbed his hands.  "Oh, it is rich to see and hear her?"
he exclaimed.  "Is she original?  Is she piquant?  I would not exchange
this one little English girl for the Grand Turk's whole seraglio, gazelle-
eyes, houri forms, and all!"

The Eastern allusion bit me again.  "I'll not stand you an inch in the
stead of a seraglio," I said; "so don't consider me an equivalent for
one.  If you have a fancy for anything in that line, away with you, sir,
to the bazaars of Stamboul without delay, and lay out in extensive slave-
purchases some of that spare cash you seem at a loss to spend
satisfactorily here."

"And what will you do, Janet, while I am bargaining for so many tons of
flesh and such an assortment of black eyes?"

"I'll be preparing myself to go out as a missionary to preach liberty to
them that are enslaved--your harem inmates amongst the rest.  I'll get
admitted there, and I'll stir up mutiny; and you, three-tailed bashaw as
you are, sir, shall in a trice find yourself fettered amongst our hands:
nor will I, for one, consent to cut your bonds till you have signed a
charter, the most liberal that despot ever yet conferred."

"I would consent to be at your mercy, Jane."

"I would have no mercy, Mr. Rochester, if you supplicated for it with an
eye like that.  While you looked so, I should be certain that whatever
charter you might grant under coercion, your first act, when released,
would be to violate its conditions."

"Why, Jane, what would you have?  I fear you will compel me to go through
a private marriage ceremony, besides that performed at the altar.  You
will stipulate, I see, for peculiar terms--what will they be?"

"I only want an easy mind, sir; not crushed by crowded obligations.  Do
you remember what you said of Celine Varens?--of the diamonds, the
cashmeres you gave her?  I will not be your English Celine Varens.  I
shall continue to act as Adele's governess; by that I shall earn my board
and lodging, and thirty pounds a year besides.  I'll furnish my own
wardrobe out of that money, and you shall give me nothing but--"

"Well, but what?"

"Your regard; and if I give you mine in return, that debt will be quit."

"Well, for cool native impudence and pure innate pride, you haven't your
equal," said he.  We were now approaching Thornfield.  "Will it please
you to dine with me to-day?" he asked, as we re-entered the gates.

"No, thank you, sir."

"And what for, 'no, thank you?' if one may inquire."

"I never have dined with you, sir: and I see no reason why I should now:
till--"

"Till what?  You delight in half-phrases."

"Till I can't help it."

"Do you suppose I eat like an ogre or a ghoul, that you dread being the
companion of my repast?"

"I have formed no supposition on the subject, sir; but I want to go on as
usual for another month."

"You will give up your governessing slavery at once."

"Indeed, begging your pardon, sir, I shall not.  I shall just go on with
it as usual.  I shall keep out of your way all day, as I have been
accustomed to do: you may send for me in the evening, when you feel
disposed to see me, and I'll come then; but at no other time."

"I want a smoke, Jane, or a pinch of snuff, to comfort me under all this,
'pour me donner une contenance,' as Adele would say; and unfortunately I
have neither my cigar-case, nor my snuff-box.  But listen--whisper.  It
is your time now, little tyrant, but it will be mine presently; and when
once I have fairly seized you, to have and to hold, I'll
just--figuratively speaking--attach you to a chain like this" (touching
his watch-guard).  "Yes, bonny wee thing, I'll wear you in my bosom, lest
my jewel I should tyne."

He said this as he helped me to alight from the carriage, and while he
afterwards lifted out Adele, I entered the house, and made good my
retreat upstairs.

He duly summoned me to his presence in the evening.  I had prepared an
occupation for him; for I was determined not to spend the whole time in a
\emph{tete-a-tete} conversation.  I remembered his fine voice; I knew he liked
to sing--good singers generally do.  I was no vocalist myself, and, in
his fastidious judgment, no musician, either; but I delighted in
listening when the performance was good.  No sooner had twilight, that
hour of romance, began to lower her blue and starry banner over the
lattice, than I rose, opened the piano, and entreated him, for the love
of heaven, to give me a song.  He said I was a capricious witch, and that
he would rather sing another time; but I averred that no time was like
the present.

"Did I like his voice?" he asked.

"Very much."  I was not fond of pampering that susceptible vanity of his;
but for once, and from motives of expediency, I would e'en soothe and
stimulate it.

"Then, Jane, you must play the accompaniment."

"Very well, sir, I will try."

I did try, but was presently swept off the stool and denominated "a
little bungler."  Being pushed unceremoniously to one side--which was
precisely what I wished--he usurped my place, and proceeded to accompany
himself: for he could play as well as sing.  I hied me to the
window-recess.  And while I sat there and looked out on the still trees
and dim lawn, to a sweet air was sung in mellow tones the following
strain:--

   "The truest love that ever heart
      Felt at its kindled core,
   Did through each vein, in quickened start,
      The tide of being pour.

   Her coming was my hope each day,
      Her parting was my pain;
   The chance that did her steps delay
      Was ice in every vein.

   I dreamed it would be nameless bliss,
      As I loved, loved to be;
   And to this object did I press
      As blind as eagerly.

   But wide as pathless was the space
      That lay our lives between,
   And dangerous as the foamy race
      Of ocean-surges green.

   And haunted as a robber-path
      Through wilderness or wood;
   For Might and Right, and Woe and Wrath,
      Between our spirits stood.

   I dangers dared; I hindrance scorned;
      I omens did defy:
   Whatever menaced, harassed, warned,
      I passed impetuous by.

   On sped my rainbow, fast as light;
      I flew as in a dream;
   For glorious rose upon my sight
      That child of Shower and Gleam.

   Still bright on clouds of suffering dim
      Shines that soft, solemn joy;
   Nor care I now, how dense and grim
      Disasters gather nigh.

   I care not in this moment sweet,
      Though all I have rushed o'er
   Should come on pinion, strong and fleet,
      Proclaiming vengeance sore:

   Though haughty Hate should strike me down,
      Right, bar approach to me,
   And grinding Might, with furious frown,
      Swear endless enmity.

   My love has placed her little hand
      With noble faith in mine,
   And vowed that wedlock's sacred band
      Our nature shall entwine.

   My love has sworn, with sealing kiss,
      With me to live--to die;
   I have at last my nameless bliss.
      As I love--loved am I!"

He rose and came towards me, and I saw his face all kindled, and his full
falcon-eye flashing, and tenderness and passion in every lineament.  I
quailed momentarily--then I rallied.  Soft scene, daring demonstration, I
would not have; and I stood in peril of both: a weapon of defence must be
prepared--I whetted my tongue: as he reached me, I asked with asperity,
"whom he was going to marry now?"

"That was a strange question to be put by his darling Jane."

"Indeed!  I considered it a very natural and necessary one: he had talked
of his future wife dying with him.  What did he mean by such a pagan
idea?  \emph{I} had no intention of dying with him--he might depend on that."

"Oh, all he longed, all he prayed for, was that I might live with him!
Death was not for such as I."

"Indeed it was: I had as good a right to die when my time came as he had:
but I should bide that time, and not be hurried away in a suttee."

"Would I forgive him for the selfish idea, and prove my pardon by a
reconciling kiss?"

"No: I would rather be excused."

Here I heard myself apostrophised as a "hard little thing;" and it was
added, "any other woman would have been melted to marrow at hearing such
stanzas crooned in her praise."

I assured him I was naturally hard--very flinty, and that he would often
find me so; and that, moreover, I was determined to show him divers
rugged points in my character before the ensuing four weeks elapsed: he
should know fully what sort of a bargain he had made, while there was yet
time to rescind it.

"Would I be quiet and talk rationally?"

"I would be quiet if he liked, and as to talking rationally, I flattered
myself I was doing that now."

He fretted, pished, and pshawed.  "Very good," I thought; "you may fume
and fidget as you please: but this is the best plan to pursue with you, I
am certain.  I like you more than I can say; but I'll not sink into a
bathos of sentiment: and with this needle of repartee I'll keep you from
the edge of the gulf too; and, moreover, maintain by its pungent aid that
distance between you and myself most conducive to our real mutual
advantage."

From less to more, I worked him up to considerable irritation; then,
after he had retired, in dudgeon, quite to the other end of the room, I
got up, and saying, "I wish you good-night, sir," in my natural and
wonted respectful manner, I slipped out by the side-door and got away.

The system thus entered on, I pursued during the whole season of
probation; and with the best success.  He was kept, to be sure, rather
cross and crusty; but on the whole I could see he was excellently
entertained, and that a lamb-like submission and turtle-dove sensibility,
while fostering his despotism more, would have pleased his judgment,
satisfied his common-sense, and even suited his taste less.

In other people's presence I was, as formerly, deferential and quiet; any
other line of conduct being uncalled for: it was only in the evening
conferences I thus thwarted and afflicted him.  He continued to send for
me punctually the moment the clock struck seven; though when I appeared
before him now, he had no such honeyed terms as "love" and "darling" on
his lips: the best words at my service were "provoking puppet,"
"malicious elf," "sprite," "changeling," \&c.  For caresses, too, I now
got grimaces; for a pressure of the hand, a pinch on the arm; for a kiss
on the cheek, a severe tweak of the ear.  It was all right: at present I
decidedly preferred these fierce favours to anything more tender.  Mrs.
Fairfax, I saw, approved me: her anxiety on my account vanished;
therefore I was certain I did well.  Meantime, Mr. Rochester affirmed I
was wearing him to skin and bone, and threatened awful vengeance for my
present conduct at some period fast coming.  I laughed in my sleeve at
his menaces.  "I can keep you in reasonable check now," I reflected; "and
I don't doubt to be able to do it hereafter: if one expedient loses its
virtue, another must be devised."

Yet after all my task was not an easy one; often I would rather have
pleased than teased him.  My future husband was becoming to me my whole
world; and more than the world: almost my hope of heaven.  He stood
between me and every thought of religion, as an eclipse intervenes
between man and the broad sun.  I could not, in those days, see God for
His creature: of whom I had made an idol.




\chapter{}


\lettrine{T}{he} month of courtship had wasted: its very last hours were being
numbered.  There was no putting off the day that advanced--the bridal
day; and all preparations for its arrival were complete.  \emph{I}, at least,
had nothing more to do: there were my trunks, packed, locked, corded,
ranged in a row along the wall of my little chamber; to-morrow, at this
time, they would be far on their road to London: and so should I
(D.V.),--or rather, not I, but one Jane Rochester, a person whom as yet I
knew not.  The cards of address alone remained to nail on: they lay, four
little squares, in the drawer.  Mr. Rochester had himself written the
direction, "Mrs. Rochester, --- Hotel, London," on each: I could not
persuade myself to affix them, or to have them affixed.  Mrs. Rochester!
She did not exist: she would not be born till to-morrow, some time after
eight o'clock a.m.; and I would wait to be assured she had come into the
world alive before I assigned to her all that property.  It was enough
that in yonder closet, opposite my dressing-table, garments said to be
hers had already displaced my black stuff Lowood frock and straw bonnet:
for not to me appertained that suit of wedding raiment; the
pearl-coloured robe, the vapoury veil pendent from the usurped
portmanteau.  I shut the closet to conceal the strange, wraith-like
apparel it contained; which, at this evening hour--nine o'clock--gave out
certainly a most ghostly shimmer through the shadow of my apartment.  "I
will leave you by yourself, white dream," I said.  "I am feverish: I hear
the wind blowing: I will go out of doors and feel it."

It was not only the hurry of preparation that made me feverish; not only
the anticipation of the great change--the new life which was to commence
to-morrow: both these circumstances had their share, doubtless, in
producing that restless, excited mood which hurried me forth at this late
hour into the darkening grounds: but a third cause influenced my mind
more than they.

I had at heart a strange and anxious thought.  Something had happened
which I could not comprehend; no one knew of or had seen the event but
myself: it had taken place the preceding night.  Mr. Rochester that night
was absent from home; nor was he yet returned: business had called him to
a small estate of two or three farms he possessed thirty miles
off--business it was requisite he should settle in person, previous to
his meditated departure from England.  I waited now his return; eager to
disburthen my mind, and to seek of him the solution of the enigma that
perplexed me.  Stay till he comes, reader; and, when I disclose my secret
to him, you shall share the confidence.

I sought the orchard, driven to its shelter by the wind, which all day
had blown strong and full from the south, without, however, bringing a
speck of rain.  Instead of subsiding as night drew on, it seemed to
augment its rush and deepen its roar: the trees blew steadfastly one way,
never writhing round, and scarcely tossing back their boughs once in an
hour; so continuous was the strain bending their branchy heads
northward--the clouds drifted from pole to pole, fast following, mass on
mass: no glimpse of blue sky had been visible that July day.

It was not without a certain wild pleasure I ran before the wind,
delivering my trouble of mind to the measureless air-torrent thundering
through space.  Descending the laurel walk, I faced the wreck of the
chestnut-tree; it stood up black and riven: the trunk, split down the
centre, gasped ghastly.  The cloven halves were not broken from each
other, for the firm base and strong roots kept them unsundered below;
though community of vitality was destroyed--the sap could flow no more:
their great boughs on each side were dead, and next winter's tempests
would be sure to fell one or both to earth: as yet, however, they might
be said to form one tree--a ruin, but an entire ruin.

"You did right to hold fast to each other," I said: as if the monster-
splinters were living things, and could hear me.  "I think, scathed as
you look, and charred and scorched, there must be a little sense of life
in you yet, rising out of that adhesion at the faithful, honest roots:
you will never have green leaves more--never more see birds making nests
and singing idyls in your boughs; the time of pleasure and love is over
with you: but you are not desolate: each of you has a comrade to
sympathise with him in his decay."  As I looked up at them, the moon
appeared momentarily in that part of the sky which filled their fissure;
her disk was blood-red and half overcast; she seemed to throw on me one
bewildered, dreary glance, and buried herself again instantly in the deep
drift of cloud.  The wind fell, for a second, round Thornfield; but far
away over wood and water, poured a wild, melancholy wail: it was sad to
listen to, and I ran off again.

Here and there I strayed through the orchard, gathered up the apples with
which the grass round the tree roots was thickly strewn; then I employed
myself in dividing the ripe from the unripe; I carried them into the
house and put them away in the store-room.  Then I repaired to the
library to ascertain whether the fire was lit, for, though summer, I knew
on such a gloomy evening Mr. Rochester would like to see a cheerful
hearth when he came in: yes, the fire had been kindled some time, and
burnt well.  I placed his arm-chair by the chimney-corner: I wheeled the
table near it: I let down the curtain, and had the candles brought in
ready for lighting.  More restless than ever, when I had completed these
arrangements I could not sit still, nor even remain in the house: a
little time-piece in the room and the old clock in the hall
simultaneously struck ten.

"How late it grows!" I said.  "I will run down to the gates: it is
moonlight at intervals; I can see a good way on the road.  He may be
coming now, and to meet him will save some minutes of suspense."

The wind roared high in the great trees which embowered the gates; but
the road as far as I could see, to the right hand and the left, was all
still and solitary: save for the shadows of clouds crossing it at
intervals as the moon looked out, it was but a long pale line, unvaried
by one moving speck.

A puerile tear dimmed my eye while I looked--a tear of disappointment and
impatience; ashamed of it, I wiped it away.  I lingered; the moon shut
herself wholly within her chamber, and drew close her curtain of dense
cloud: the night grew dark; rain came driving fast on the gale.

"I wish he would come!  I wish he would come!" I exclaimed, seized with
hypochondriac foreboding.  I had expected his arrival before tea; now it
was dark: what could keep him?  Had an accident happened?  The event of
last night again recurred to me.  I interpreted it as a warning of
disaster.  I feared my hopes were too bright to be realised; and I had
enjoyed so much bliss lately that I imagined my fortune had passed its
meridian, and must now decline.

"Well, I cannot return to the house," I thought; "I cannot sit by the
fireside, while he is abroad in inclement weather: better tire my limbs
than strain my heart; I will go forward and meet him."

I set out; I walked fast, but not far: ere I had measured a quarter of a
mile, I heard the tramp of hoofs; a horseman came on, full gallop; a dog
ran by his side.  Away with evil presentiment!  It was he: here he was,
mounted on Mesrour, followed by Pilot.  He saw me; for the moon had
opened a blue field in the sky, and rode in it watery bright: he took his
hat off, and waved it round his head.  I now ran to meet him.

"There!" he exclaimed, as he stretched out his hand and bent from the
saddle: "You can't do without me, that is evident.  Step on my boot-toe;
give me both hands: mount!"

I obeyed: joy made me agile: I sprang up before him.  A hearty kissing I
got for a welcome, and some boastful triumph, which I swallowed as well
as I could.  He checked himself in his exultation to demand, "But is
there anything the matter, Janet, that you come to meet me at such an
hour?  Is there anything wrong?"

"No, but I thought you would never come.  I could not bear to wait in the
house for you, especially with this rain and wind."

"Rain and wind, indeed!  Yes, you are dripping like a mermaid; pull my
cloak round you: but I think you are feverish, Jane: both your cheek and
hand are burning hot.  I ask again, is there anything the matter?"

"Nothing now; I am neither afraid nor unhappy."

"Then you have been both?"

"Rather: but I'll tell you all about it by-and-bye, sir; and I daresay
you will only laugh at me for my pains."

"I'll laugh at you heartily when to-morrow is past; till then I dare not:
my prize is not certain.  This is you, who have been as slippery as an
eel this last month, and as thorny as a briar-rose?  I could not lay a
finger anywhere but I was pricked; and now I seem to have gathered up a
stray lamb in my arms.  You wandered out of the fold to seek your
shepherd, did you, Jane?"

"I wanted you: but don't boast.  Here we are at Thornfield: now let me
get down."

He landed me on the pavement.  As John took his horse, and he followed me
into the hall, he told me to make haste and put something dry on, and
then return to him in the library; and he stopped me, as I made for the
staircase, to extort a promise that I would not be long: nor was I long;
in five minutes I rejoined him.  I found him at supper.

"Take a seat and bear me company, Jane: please God, it is the last meal
but one you will eat at Thornfield Hall for a long time."

I sat down near him, but told him I could not eat.  "Is it because you
have the prospect of a journey before you, Jane?  Is it the thoughts of
going to London that takes away your appetite?"

"I cannot see my prospects clearly to-night, sir; and I hardly know what
thoughts I have in my head.  Everything in life seems unreal."

"Except me: I am substantial enough--touch me."

"You, sir, are the most phantom-like of all: you are a mere dream."

He held out his hand, laughing.  "Is that a dream?" said he, placing it
close to my eyes.  He had a rounded, muscular, and vigorous hand, as well
as a long, strong arm.

"Yes; though I touch it, it is a dream," said I, as I put it down from
before my face.  "Sir, have you finished supper?"

"Yes, Jane."

I rang the bell and ordered away the tray.  When we were again alone, I
stirred the fire, and then took a low seat at my master's knee.

"It is near midnight," I said.

"Yes: but remember, Jane, you promised to wake with me the night before
my wedding."

"I did; and I will keep my promise, for an hour or two at least: I have
no wish to go to bed."

"Are all your arrangements complete?"

"All, sir."

"And on my part likewise," he returned, "I have settled everything; and
we shall leave Thornfield to-morrow, within half-an-hour after our return
from church."

"Very well, sir."

"With what an extraordinary smile you uttered that word--'very well,'
Jane!  What a bright spot of colour you have on each cheek! and how
strangely your eyes glitter!  Are you well?"

"I believe I am."

"Believe!  What is the matter?  Tell me what you feel."

"I could not, sir: no words could tell you what I feel.  I wish this
present hour would never end: who knows with what fate the next may come
charged?"

"This is hypochondria, Jane.  You have been over-excited, or
over-fatigued."

"Do you, sir, feel calm and happy?"

"Calm?--no: but happy--to the heart's core."

I looked up at him to read the signs of bliss in his face: it was ardent
and flushed.

"Give me your confidence, Jane," he said: "relieve your mind of any
weight that oppresses it, by imparting it to me.  What do you fear?--that
I shall not prove a good husband?"

"It is the idea farthest from my thoughts."

"Are you apprehensive of the new sphere you are about to enter?--of the
new life into which you are passing?"

"No."

"You puzzle me, Jane: your look and tone of sorrowful audacity perplex
and pain me.  I want an explanation."

"Then, sir, listen.  You were from home last night?"

"I was: I know that; and you hinted a while ago at something which had
happened in my absence:--nothing, probably, of consequence; but, in
short, it has disturbed you.  Let me hear it.  Mrs. Fairfax has said
something, perhaps? or you have overheard the servants talk?--your
sensitive self-respect has been wounded?"

"No, sir."  It struck twelve--I waited till the time-piece had concluded
its silver chime, and the clock its hoarse, vibrating stroke, and then I
proceeded.

"All day yesterday I was very busy, and very happy in my ceaseless
bustle; for I am not, as you seem to think, troubled by any haunting
fears about the new sphere, et cetera: I think it a glorious thing to
have the hope of living with you, because I love you.  No, sir, don't
caress me now--let me talk undisturbed.  Yesterday I trusted well in
Providence, and believed that events were working together for your good
and mine: it was a fine day, if you recollect--the calmness of the air
and sky forbade apprehensions respecting your safety or comfort on your
journey.  I walked a little while on the pavement after tea, thinking of
you; and I beheld you in imagination so near me, I scarcely missed your
actual presence.  I thought of the life that lay before me--\emph{your} life,
sir--an existence more expansive and stirring than my own: as much more
so as the depths of the sea to which the brook runs are than the shallows
of its own strait channel.  I wondered why moralists call this world a
dreary wilderness: for me it blossomed like a rose.  Just at sunset, the
air turned cold and the sky cloudy: I went in, Sophie called me upstairs
to look at my wedding-dress, which they had just brought; and under it in
the box I found your present--the veil which, in your princely
extravagance, you sent for from London: resolved, I suppose, since I
would not have jewels, to cheat me into accepting something as costly.  I
smiled as I unfolded it, and devised how I would tease you about your
aristocratic tastes, and your efforts to masque your plebeian bride in
the attributes of a peeress.  I thought how I would carry down to you the
square of unembroidered blond I had myself prepared as a covering for my
low-born head, and ask if that was not good enough for a woman who could
bring her husband neither fortune, beauty, nor connections.  I saw
plainly how you would look; and heard your impetuous republican answers,
and your haughty disavowal of any necessity on your part to augment your
wealth, or elevate your standing, by marrying either a purse or a
coronet."

"How well you read me, you witch!" interposed Mr. Rochester: "but what
did you find in the veil besides its embroidery?  Did you find poison, or
a dagger, that you look so mournful now?"

"No, no, sir; besides the delicacy and richness of the fabric, I found
nothing save Fairfax Rochester's pride; and that did not scare me,
because I am used to the sight of the demon.  But, sir, as it grew dark,
the wind rose: it blew yesterday evening, not as it blows now--wild and
high--but 'with a sullen, moaning sound' far more eerie.  I wished you
were at home.  I came into this room, and the sight of the empty chair
and fireless hearth chilled me.  For some time after I went to bed, I
could not sleep--a sense of anxious excitement distressed me.  The gale
still rising, seemed to my ear to muffle a mournful under-sound; whether
in the house or abroad I could not at first tell, but it recurred,
doubtful yet doleful at every lull; at last I made out it must be some
dog howling at a distance.  I was glad when it ceased.  On sleeping, I
continued in dreams the idea of a dark and gusty night.  I continued also
the wish to be with you, and experienced a strange, regretful
consciousness of some barrier dividing us.  During all my first sleep, I
was following the windings of an unknown road; total obscurity environed
me; rain pelted me; I was burdened with the charge of a little child: a
very small creature, too young and feeble to walk, and which shivered in
my cold arms, and wailed piteously in my ear.  I thought, sir, that you
were on the road a long way before me; and I strained every nerve to
overtake you, and made effort on effort to utter your name and entreat
you to stop--but my movements were fettered, and my voice still died away
inarticulate; while you, I felt, withdrew farther and farther every
moment."

"And these dreams weigh on your spirits now, Jane, when I am close to
you?  Little nervous subject!  Forget visionary woe, and think only of
real happiness!  You say you love me, Janet: yes--I will not forget that;
and you cannot deny it.  \emph{Those} words did not die inarticulate on your
lips.  I heard them clear and soft: a thought too solemn perhaps, but
sweet as music--'I think it is a glorious thing to have the hope of
living with you, Edward, because I love you.'  Do you love me,
Jane?--repeat it."

"I do, sir--I do, with my whole heart."

"Well," he said, after some minutes' silence, "it is strange; but that
sentence has penetrated my breast painfully.  Why?  I think because you
said it with such an earnest, religious energy, and because your upward
gaze at me now is the very sublime of faith, truth, and devotion: it is
too much as if some spirit were near me.  Look wicked, Jane: as you know
well how to look: coin one of your wild, shy, provoking smiles; tell me
you hate me--tease me, vex me; do anything but move me: I would rather be
incensed than saddened."

"I will tease you and vex you to your heart's content, when I have
finished my tale: but hear me to the end."

"I thought, Jane, you had told me all.  I thought I had found the source
of your melancholy in a dream."

I shook my head.  "What! is there more?  But I will not believe it to be
anything important.  I warn you of incredulity beforehand.  Go on."

The disquietude of his air, the somewhat apprehensive impatience of his
manner, surprised me: but I proceeded.

"I dreamt another dream, sir: that Thornfield Hall was a dreary ruin, the
retreat of bats and owls.  I thought that of all the stately front
nothing remained but a shell-like wall, very high and very
fragile-looking.  I wandered, on a moonlight night, through the grass-
grown enclosure within: here I stumbled over a marble hearth, and there
over a fallen fragment of cornice.  Wrapped up in a shawl, I still
carried the unknown little child: I might not lay it down anywhere,
however tired were my arms--however much its weight impeded my progress,
I must retain it.  I heard the gallop of a horse at a distance on the
road; I was sure it was you; and you were departing for many years and
for a distant country.  I climbed the thin wall with frantic perilous
haste, eager to catch one glimpse of you from the top: the stones rolled
from under my feet, the ivy branches I grasped gave way, the child clung
round my neck in terror, and almost strangled me; at last I gained the
summit.  I saw you like a speck on a white track, lessening every moment.
The blast blew so strong I could not stand.  I sat down on the narrow
ledge; I hushed the scared infant in my lap: you turned an angle of the
road: I bent forward to take a last look; the wall crumbled; I was
shaken; the child rolled from my knee, I lost my balance, fell, and
woke."

"Now, Jane, that is all."

"All the preface, sir; the tale is yet to come.  On waking, a gleam
dazzled my eyes; I thought--Oh, it is daylight!  But I was mistaken; it
was only candlelight.  Sophie, I supposed, had come in.  There was a
light in the dressing-table, and the door of the closet, where, before
going to bed, I had hung my wedding-dress and veil, stood open; I heard a
rustling there.  I asked, 'Sophie, what are you doing?'  No one answered;
but a form emerged from the closet; it took the light, held it aloft, and
surveyed the garments pendent from the portmanteau.  'Sophie!  Sophie!'  I
again cried: and still it was silent.  I had risen up in bed, I bent
forward: first surprise, then bewilderment, came over me; and then my
blood crept cold through my veins.  Mr. Rochester, this was not Sophie,
it was not Leah, it was not Mrs. Fairfax: it was not--no, I was sure of
it, and am still--it was not even that strange woman, Grace Poole."

"It must have been one of them," interrupted my master.

"No, sir, I solemnly assure you to the contrary.  The shape standing
before me had never crossed my eyes within the precincts of Thornfield
Hall before; the height, the contour were new to me."

"Describe it, Jane."

"It seemed, sir, a woman, tall and large, with thick and dark hair
hanging long down her back.  I know not what dress she had on: it was
white and straight; but whether gown, sheet, or shroud, I cannot tell."

"Did you see her face?"

"Not at first.  But presently she took my veil from its place; she held
it up, gazed at it long, and then she threw it over her own head, and
turned to the mirror.  At that moment I saw the reflection of the visage
and features quite distinctly in the dark oblong glass."

"And how were they?"

"Fearful and ghastly to me--oh, sir, I never saw a face like it!  It was
a discoloured face--it was a savage face.  I wish I could forget the roll
of the red eyes and the fearful blackened inflation of the lineaments!"

"Ghosts are usually pale, Jane."

"This, sir, was purple: the lips were swelled and dark; the brow
furrowed: the black eyebrows widely raised over the bloodshot eyes.  Shall
I tell you of what it reminded me?"

"You may."

"Of the foul German spectre--the Vampyre."

"Ah!--what did it do?"

"Sir, it removed my veil from its gaunt head, rent it in two parts, and
flinging both on the floor, trampled on them."

{It removed my veil from its gaunt head, rent it in two parts, and
flinging both on the floor, trampled on them: p272.jpg}

"Afterwards?"

"It drew aside the window-curtain and looked out; perhaps it saw dawn
approaching, for, taking the candle, it retreated to the door.  Just at
my bedside, the figure stopped: the fiery eyes glared upon me--she thrust
up her candle close to my face, and extinguished it under my eyes.  I was
aware her lurid visage flamed over mine, and I lost consciousness: for
the second time in my life--only the second time--I became insensible
from terror."

"Who was with you when you revived?"

"No one, sir, but the broad day.  I rose, bathed my head and face in
water, drank a long draught; felt that though enfeebled I was not ill,
and determined that to none but you would I impart this vision.  Now,
sir, tell me who and what that woman was?"

"The creature of an over-stimulated brain; that is certain.  I must be
careful of you, my treasure: nerves like yours were not made for rough
handling."

"Sir, depend on it, my nerves were not in fault; the thing was real: the
transaction actually took place."

"And your previous dreams, were they real too?  Is Thornfield Hall a
ruin?  Am I severed from you by insuperable obstacles?  Am I leaving you
without a tear--without a kiss--without a word?"

"Not yet."

"Am I about to do it?  Why, the day is already commenced which is to bind
us indissolubly; and when we are once united, there shall be no
recurrence of these mental terrors: I guarantee that."

"Mental terrors, sir!  I wish I could believe them to be only such: I
wish it more now than ever; since even you cannot explain to me the
mystery of that awful visitant."

"And since I cannot do it, Jane, it must have been unreal."

"But, sir, when I said so to myself on rising this morning, and when I
looked round the room to gather courage and comfort from the cheerful
aspect of each familiar object in full daylight, there--on the carpet--I
saw what gave the distinct lie to my hypothesis,--the veil, torn from top
to bottom in two halves!"

I felt Mr. Rochester start and shudder; he hastily flung his arms round
me.  "Thank God!" he exclaimed, "that if anything malignant did come near
you last night, it was only the veil that was harmed.  Oh, to think what
might have happened!"

He drew his breath short, and strained me so close to him, I could
scarcely pant.  After some minutes' silence, he continued, cheerily--

"Now, Janet, I'll explain to you all about it.  It was half dream, half
reality.  A woman did, I doubt not, enter your room: and that woman
was--must have been--Grace Poole.  You call her a strange being yourself:
from all you know, you have reason so to call her--what did she do to me?
what to Mason?  In a state between sleeping and waking, you noticed her
entrance and her actions; but feverish, almost delirious as you were, you
ascribed to her a goblin appearance different from her own: the long
dishevelled hair, the swelled black face, the exaggerated stature, were
figments of imagination; results of nightmare: the spiteful tearing of
the veil was real: and it is like her.  I see you would ask why I keep
such a woman in my house: when we have been married a year and a day, I
will tell you; but not now.  Are you satisfied, Jane?  Do you accept my
solution of the mystery?"

I reflected, and in truth it appeared to me the only possible one:
satisfied I was not, but to please him I endeavoured to appear
so--relieved, I certainly did feel; so I answered him with a contented
smile.  And now, as it was long past one, I prepared to leave him.

"Does not Sophie sleep with Adele in the nursery?" he asked, as I lit my
candle.

"Yes, sir."

"And there is room enough in Adele's little bed for you.  You must share
it with her to-night, Jane: it is no wonder that the incident you have
related should make you nervous, and I would rather you did not sleep
alone: promise me to go to the nursery."

"I shall be very glad to do so, sir."

"And fasten the door securely on the inside.  Wake Sophie when you go
upstairs, under pretence of requesting her to rouse you in good time to-
morrow; for you must be dressed and have finished breakfast before eight.
And now, no more sombre thoughts: chase dull care away, Janet.  Don't you
hear to what soft whispers the wind has fallen? and there is no more
beating of rain against the window-panes: look here" (he lifted up the
curtain)--"it is a lovely night!"

It was.  Half heaven was pure and stainless: the clouds, now trooping
before the wind, which had shifted to the west, were filing off eastward
in long, silvered columns.  The moon shone peacefully.

"Well," said Mr. Rochester, gazing inquiringly into my eyes, "how is my
Janet now?"

"The night is serene, sir; and so am I."

"And you will not dream of separation and sorrow to-night; but of happy
love and blissful union."

This prediction was but half fulfilled: I did not indeed dream of sorrow,
but as little did I dream of joy; for I never slept at all.  With little
Adele in my arms, I watched the slumber of childhood--so tranquil, so
passionless, so innocent--and waited for the coming day: all my life was
awake and astir in my frame: and as soon as the sun rose I rose too.  I
remember Adele clung to me as I left her: I remember I kissed her as I
loosened her little hands from my neck; and I cried over her with strange
emotion, and quitted her because I feared my sobs would break her still
sound repose.  She seemed the emblem of my past life; and here I was now
to array myself to meet, the dread, but adored, type of my unknown future
day.




\chapter{}


\lettrine{S}{ophie} came at seven to dress me: she was very long indeed in
accomplishing her task; so long that Mr. Rochester, grown, I suppose,
impatient of my delay, sent up to ask why I did not come.  She was just
fastening my veil (the plain square of blond after all) to my hair with a
brooch; I hurried from under her hands as soon as I could.

"Stop!" she cried in French.  "Look at yourself in the mirror: you have
not taken one peep."

So I turned at the door: I saw a robed and veiled figure, so unlike my
usual self that it seemed almost the image of a stranger.  "Jane!" called
a voice, and I hastened down.  I was received at the foot of the stairs
by Mr. Rochester.

"Lingerer!" he said, "my brain is on fire with impatience, and you tarry
so long!"

He took me into the dining-room, surveyed me keenly all over, pronounced
me "fair as a lily, and not only the pride of his life, but the desire of
his eyes," and then telling me he would give me but ten minutes to eat
some breakfast, he rang the bell.  One of his lately hired servants, a
footman, answered it.

"Is John getting the carriage ready?"

"Yes, sir."

"Is the luggage brought down?"

"They are bringing it down, sir."

"Go you to the church: see if Mr. Wood (the clergyman) and the clerk are
there: return and tell me."

The church, as the reader knows, was but just beyond the gates; the
footman soon returned.

"Mr. Wood is in the vestry, sir, putting on his surplice."

"And the carriage?"

"The horses are harnessing."

"We shall not want it to go to church; but it must be ready the moment we
return: all the boxes and luggage arranged and strapped on, and the
coachman in his seat."

"Yes, sir."

"Jane, are you ready?"

I rose.  There were no groomsmen, no bridesmaids, no relatives to wait
for or marshal: none but Mr. Rochester and I.  Mrs. Fairfax stood in the
hall as we passed.   I would fain have spoken to her, but my hand was
held by a grasp of iron: I was hurried along by a stride I could hardly
follow; and to look at Mr. Rochester's face was to feel that not a second
of delay would be tolerated for any purpose.  I wonder what other
bridegroom ever looked as he did--so bent up to a purpose, so grimly
resolute: or who, under such steadfast brows, ever revealed such flaming
and flashing eyes.

I know not whether the day was fair or foul; in descending the drive, I
gazed neither on sky nor earth: my heart was with my eyes; and both
seemed migrated into Mr. Rochester's frame.  I wanted to see the
invisible thing on which, as we went along, he appeared to fasten a
glance fierce and fell.  I wanted to feel the thoughts whose force he
seemed breasting and resisting.

At the churchyard wicket he stopped: he discovered I was quite out of
breath.  "Am I cruel in my love?" he said.  "Delay an instant: lean on
me, Jane."

And now I can recall the picture of the grey old house of God rising calm
before me, of a rook wheeling round the steeple, of a ruddy morning sky
beyond.  I remember something, too, of the green grave-mounds; and I have
not forgotten, either, two figures of strangers straying amongst the low
hillocks and reading the mementoes graven on the few mossy head-stones.  I
noticed them, because, as they saw us, they passed round to the back of
the church; and I doubted not they were going to enter by the side-aisle
door and witness the ceremony.  By Mr. Rochester they were not observed;
he was earnestly looking at my face from which the blood had, I daresay,
momentarily fled: for I felt my forehead dewy, and my cheeks and lips
cold.  When I rallied, which I soon did, he walked gently with me up the
path to the porch.

We entered the quiet and humble temple; the priest waited in his white
surplice at the lowly altar, the clerk beside him.  All was still: two
shadows only moved in a remote corner.  My conjecture had been correct:
the strangers had slipped in before us, and they now stood by the vault
of the Rochesters, their backs towards us, viewing through the rails the
old time-stained marble tomb, where a kneeling angel guarded the remains
of Damer de Rochester, slain at Marston Moor in the time of the civil
wars, and of Elizabeth, his wife.

Our place was taken at the communion rails.  Hearing a cautious step
behind me, I glanced over my shoulder: one of the strangers--a gentleman,
evidently--was advancing up the chancel.  The service began.  The
explanation of the intent of matrimony was gone through; and then the
clergyman came a step further forward, and, bending slightly towards Mr.
Rochester, went on.

"I require and charge you both (as ye will answer at the dreadful day of
judgment, when the secrets of all hearts shall be disclosed), that if
either of you know any impediment why ye may not lawfully be joined
together in matrimony, ye do now confess it; for be ye well assured that
so many as are coupled together otherwise than God's Word doth allow, are
not joined together by God, neither is their matrimony lawful."

He paused, as the custom is.  When is the pause after that sentence ever
broken by reply?  Not, perhaps, once in a hundred years.  And the
clergyman, who had not lifted his eyes from his book, and had held his
breath but for a moment, was proceeding: his hand was already stretched
towards Mr. Rochester, as his lips unclosed to ask, "Wilt thou have this
woman for thy wedded wife?"--when a distinct and near voice said--

"The marriage cannot go on: I declare the existence of an impediment."

The clergyman looked up at the speaker and stood mute; the clerk did the
same; Mr. Rochester moved slightly, as if an earthquake had rolled under
his feet: taking a firmer footing, and not turning his head or eyes, he
said, "Proceed."

Profound silence fell when he had uttered that word, with deep but low
intonation.  Presently Mr. Wood said--

"I cannot proceed without some investigation into what has been asserted,
and evidence of its truth or falsehood."

"The ceremony is quite broken off," subjoined the voice behind us.  "I am
in a condition to prove my allegation: an insuperable impediment to this
marriage exists."

Mr. Rochester heard, but heeded not: he stood stubborn and rigid, making
no movement but to possess himself of my hand.  What a hot and strong
grasp he had! and how like quarried marble was his pale, firm, massive
front at this moment!  How his eye shone, still watchful, and yet wild
beneath!

Mr. Wood seemed at a loss.  "What is the nature of the impediment?" he
asked.  "Perhaps it may be got over--explained away?"

"Hardly," was the answer.  "I have called it insuperable, and I speak
advisedly."

The speaker came forward and leaned on the rails.  He continued, uttering
each word distinctly, calmly, steadily, but not loudly--

"It simply consists in the existence of a previous marriage.  Mr.
Rochester has a wife now living."

My nerves vibrated to those low-spoken words as they had never vibrated
to thunder--my blood felt their subtle violence as it had never felt
frost or fire; but I was collected, and in no danger of swooning.  I
looked at Mr. Rochester: I made him look at me.  His whole face was
colourless rock: his eye was both spark and flint.  He disavowed nothing:
he seemed as if he would defy all things.  Without speaking, without
smiling, without seeming to recognise in me a human being, he only twined
my waist with his arm and riveted me to his side.

"Who are you?" he asked of the intruder.

"My name is Briggs, a solicitor of --- Street, London."

"And you would thrust on me a wife?"

"I would remind you of your lady's existence, sir, which the law
recognises, if you do not."

"Favour me with an account of her--with her name, her parentage, her
place of abode."

"Certainly."  Mr. Briggs calmly took a paper from his pocket, and read
out in a sort of official, nasal voice:--

"'I affirm and can prove that on the 20th of October A.D. --- (a date of
fifteen years back), Edward Fairfax Rochester, of Thornfield Hall, in the
county of ---, and of Ferndean Manor, in ---shire, England, was married
to my sister, Bertha Antoinetta Mason, daughter of Jonas Mason, merchant,
and of Antoinetta his wife, a Creole, at --- church, Spanish Town,
Jamaica.  The record of the marriage will be found in the register of
that church--a copy of it is now in my possession.  Signed, Richard
Mason.'"

"That--if a genuine document--may prove I have been married, but it does
not prove that the woman mentioned therein as my wife is still living."

"She was living three months ago," returned the lawyer.

"How do you know?"

"I have a witness to the fact, whose testimony even you, sir, will
scarcely controvert."

"Produce him--or go to hell."

"I will produce him first--he is on the spot.  Mr. Mason, have the
goodness to step forward."

Mr. Rochester, on hearing the name, set his teeth; he experienced, too, a
sort of strong convulsive quiver; near to him as I was, I felt the
spasmodic movement of fury or despair run through his frame.  The second
stranger, who had hitherto lingered in the background, now drew near; a
pale face looked over the solicitor's shoulder--yes, it was Mason
himself.  Mr. Rochester turned and glared at him.  His eye, as I have
often said, was a black eye: it had now a tawny, nay, a bloody light in
its gloom; and his face flushed--olive cheek and hueless forehead
received a glow as from spreading, ascending heart-fire: and he stirred,
lifted his strong arm--he could have struck Mason, dashed him on the
church-floor, shocked by ruthless blow the breath from his body--but
Mason shrank away, and cried faintly, "Good God!"  Contempt fell cool on
Mr. Rochester--his passion died as if a blight had shrivelled it up: he
only asked--"What have \emph{you} to say?"

An inaudible reply escaped Mason's white lips.

"The devil is in it if you cannot answer distinctly.  I again demand,
what have you to say?"

"Sir--sir," interrupted the clergyman, "do not forget you are in a sacred
place."  Then addressing Mason, he inquired gently, "Are you aware, sir,
whether or not this gentleman's wife is still living?"

"Courage," urged the lawyer,--"speak out."

"She is now living at Thornfield Hall," said Mason, in more articulate
tones: "I saw her there last April.  I am her brother."

"At Thornfield Hall!" ejaculated the clergyman.  "Impossible!  I am an
old resident in this neighbourhood, sir, and I never heard of a Mrs.
Rochester at Thornfield Hall."

I saw a grim smile contort Mr. Rochester's lips, and he muttered--

"No, by God!  I took care that none should hear of it--or of her under
that name."  He mused--for ten minutes he held counsel with himself: he
formed his resolve, and announced it--

"Enough! all shall bolt out at once, like the bullet from the barrel.
Wood, close your book and take off your surplice; John Green (to the
clerk), leave the church: there will be no wedding to-day."  The man
obeyed.

Mr. Rochester continued, hardily and recklessly: "Bigamy is an ugly
word!--I meant, however, to be a bigamist; but fate has out-manoeuvred
me, or Providence has checked me,--perhaps the last.  I am little better
than a devil at this moment; and, as my pastor there would tell me,
deserve no doubt the sternest judgments of God, even to the quenchless
fire and deathless worm.  Gentlemen, my plan is broken up:--what this
lawyer and his client say is true: I have been married, and the woman to
whom I was married lives!  You say you never heard of a Mrs. Rochester at
the house up yonder, Wood; but I daresay you have many a time inclined
your ear to gossip about the mysterious lunatic kept there under watch
and ward.  Some have whispered to you that she is my bastard half-sister:
some, my cast-off mistress.  I now inform you that she is my wife, whom I
married fifteen years ago,--Bertha Mason by name; sister of this resolute
personage, who is now, with his quivering limbs and white cheeks, showing
you what a stout heart men may bear.  Cheer up, Dick!--never fear me!--I'd
almost as soon strike a woman as you.  Bertha Mason is mad; and she came
of a mad family; idiots and maniacs through three generations!  Her
mother, the Creole, was both a madwoman and a drunkard!--as I found out
after I had wed the daughter: for they were silent on family secrets
before.  Bertha, like a dutiful child, copied her parent in both points.
I had a charming partner--pure, wise, modest: you can fancy I was a happy
man.  I went through rich scenes!  Oh! my experience has been heavenly,
if you only knew it!  But I owe you no further explanation.  Briggs,
Wood, Mason, I invite you all to come up to the house and visit Mrs.
Poole's patient, and \emph{my wife}!  You shall see what sort of a being I was
cheated into espousing, and judge whether or not I had a right to break
the compact, and seek sympathy with something at least human.  This
girl," he continued, looking at me, "knew no more than you, Wood, of the
disgusting secret: she thought all was fair and legal and never dreamt
she was going to be entrapped into a feigned union with a defrauded
wretch, already bound to a bad, mad, and embruted partner!  Come all of
you--follow!"

Still holding me fast, he left the church: the three gentlemen came
after.  At the front door of the hall we found the carriage.

"Take it back to the coach-house, John," said Mr. Rochester coolly; "it
will not be wanted to-day."

At our entrance, Mrs. Fairfax, Adele, Sophie, Leah, advanced to meet and
greet us.

"To the right-about--every soul!" cried the master; "away with your
congratulations!  Who wants them?  Not I!--they are fifteen years too
late!"

He passed on and ascended the stairs, still holding my hand, and still
beckoning the gentlemen to follow him, which they did.  We mounted the
first staircase, passed up the gallery, proceeded to the third storey:
the low, black door, opened by Mr. Rochester's master-key, admitted us to
the tapestried room, with its great bed and its pictorial cabinet.

"You know this place, Mason," said our guide; "she bit and stabbed you
here."

He lifted the hangings from the wall, uncovering the second door: this,
too, he opened.  In a room without a window, there burnt a fire guarded
by a high and strong fender, and a lamp suspended from the ceiling by a
chain.  Grace Poole bent over the fire, apparently cooking something in a
saucepan.  In the deep shade, at the farther end of the room, a figure
ran backwards and forwards.  What it was, whether beast or human being,
one could not, at first sight, tell: it grovelled, seemingly, on all
fours; it snatched and growled like some strange wild animal: but it was
covered with clothing, and a quantity of dark, grizzled hair, wild as a
mane, hid its head and face.

"Good-morrow, Mrs. Poole!" said Mr. Rochester.  "How are you? and how is
your charge to-day?"

"We're tolerable, sir, I thank you," replied Grace, lifting the boiling
mess carefully on to the hob: "rather snappish, but not 'rageous."

A fierce cry seemed to give the lie to her favourable report: the clothed
hyena rose up, and stood tall on its hind-feet.

"Ah! sir, she sees you!" exclaimed Grace: "you'd better not stay."

"Only a few moments, Grace: you must allow me a few moments."

"Take care then, sir!--for God's sake, take care!"

The maniac bellowed: she parted her shaggy locks from her visage, and
gazed wildly at her visitors.  I recognised well that purple face,--those
bloated features.  Mrs. Poole advanced.

"Keep out of the way," said Mr. Rochester, thrusting her aside: "she has
no knife now, I suppose, and I'm on my guard."

"One never knows what she has, sir: she is so cunning: it is not in
mortal discretion to fathom her craft."

"We had better leave her," whispered Mason.

"Go to the devil!" was his brother-in-law's recommendation.

"'Ware!" cried Grace.  The three gentlemen retreated simultaneously.  Mr.
Rochester flung me behind him: the lunatic sprang and grappled his throat
viciously, and laid her teeth to his cheek: they struggled.  She was a
big woman, in stature almost equalling her husband, and corpulent
besides: she showed virile force in the contest--more than once she
almost throttled him, athletic as he was.  He could have settled her with
a well-planted blow; but he would not strike: he would only wrestle.  At
last he mastered her arms; Grace Poole gave him a cord, and he pinioned
them behind her: with more rope, which was at hand, he bound her to a
chair.  The operation was performed amidst the fiercest yells and the
most convulsive plunges.  Mr. Rochester then turned to the spectators: he
looked at them with a smile both acrid and desolate.

"That is \emph{my wife}," said he.  "Such is the sole conjugal embrace I am
ever to know--such are the endearments which are to solace my leisure
hours!  And \emph{this} is what I wished to have" (laying his hand on my
shoulder): "this young girl, who stands so grave and quiet at the mouth
of hell, looking collectedly at the gambols of a demon, I wanted her just
as a change after that fierce ragout.  Wood and Briggs, look at the
difference!  Compare these clear eyes with the red balls yonder--this
face with that mask--this form with that bulk; then judge me, priest of
the gospel and man of the law, and remember with what judgment ye judge
ye shall be judged!  Off with you now.  I must shut up my prize."

We all withdrew.  Mr. Rochester stayed a moment behind us, to give some
further order to Grace Poole.  The solicitor addressed me as he descended
the stair.

"You, madam," said he, "are cleared from all blame: your uncle will be
glad to hear it--if, indeed, he should be still living--when Mr. Mason
returns to Madeira."

"My uncle!  What of him?  Do you know him?"

"Mr. Mason does.  Mr. Eyre has been the Funchal correspondent of his
house for some years.  When your uncle received your letter intimating
the contemplated union between yourself and Mr. Rochester, Mr. Mason, who
was staying at Madeira to recruit his health, on his way back to Jamaica,
happened to be with him.  Mr. Eyre mentioned the intelligence; for he
knew that my client here was acquainted with a gentleman of the name of
Rochester.  Mr. Mason, astonished and distressed as you may suppose,
revealed the real state of matters.  Your uncle, I am sorry to say, is
now on a sick bed; from which, considering the nature of his
disease--decline--and the stage it has reached, it is unlikely he will
ever rise.  He could not then hasten to England himself, to extricate you
from the snare into which you had fallen, but he implored Mr. Mason to
lose no time in taking steps to prevent the false marriage.  He referred
him to me for assistance.  I used all despatch, and am thankful I was not
too late: as you, doubtless, must be also.  Were I not morally certain
that your uncle will be dead ere you reach Madeira, I would advise you to
accompany Mr. Mason back; but as it is, I think you had better remain in
England till you can hear further, either from or of Mr. Eyre.  Have we
anything else to stay for?" he inquired of Mr. Mason.

"No, no--let us be gone," was the anxious reply; and without waiting to
take leave of Mr. Rochester, they made their exit at the hall door.  The
clergyman stayed to exchange a few sentences, either of admonition or
reproof, with his haughty parishioner; this duty done, he too departed.

I heard him go as I stood at the half-open door of my own room, to which
I had now withdrawn.  The house cleared, I shut myself in, fastened the
bolt that none might intrude, and proceeded--not to weep, not to mourn, I
was yet too calm for that, but--mechanically to take off the wedding
dress, and replace it by the stuff gown I had worn yesterday, as I
thought, for the last time.  I then sat down: I felt weak and tired.  I
leaned my arms on a table, and my head dropped on them.  And now I
thought: till now I had only heard, seen, moved--followed up and down
where I was led or dragged--watched event rush on event, disclosure open
beyond disclosure: but \emph{now}, \emph{I thought}.

The morning had been a quiet morning enough--all except the brief scene
with the lunatic: the transaction in the church had not been noisy; there
was no explosion of passion, no loud altercation, no dispute, no defiance
or challenge, no tears, no sobs: a few words had been spoken, a calmly
pronounced objection to the marriage made; some stern, short questions
put by Mr. Rochester; answers, explanations given, evidence adduced; an
open admission of the truth had been uttered by my master; then the
living proof had been seen; the intruders were gone, and all was over.

I was in my own room as usual--just myself, without obvious change:
nothing had smitten me, or scathed me, or maimed me.  And yet where was
the Jane Eyre of yesterday?--where was her life?--where were her
prospects?

Jane Eyre, who had been an ardent, expectant woman--almost a bride, was a
cold, solitary girl again: her life was pale; her prospects were
desolate.  A Christmas frost had come at midsummer; a white December
storm had whirled over June; ice glazed the ripe apples, drifts crushed
the blowing roses; on hayfield and cornfield lay a frozen shroud: lanes
which last night blushed full of flowers, to-day were pathless with
untrodden snow; and the woods, which twelve hours since waved leafy and
flagrant as groves between the tropics, now spread, waste, wild, and
white as pine-forests in wintry Norway.  My hopes were all dead--struck
with a subtle doom, such as, in one night, fell on all the first-born in
the land of Egypt.  I looked on my cherished wishes, yesterday so
blooming and glowing; they lay stark, chill, livid corpses that could
never revive.  I looked at my love: that feeling which was my
master's--which he had created; it shivered in my heart, like a suffering
child in a cold cradle; sickness and anguish had seized it; it could not
seek Mr. Rochester's arms--it could not derive warmth from his breast.
Oh, never more could it turn to him; for faith was blighted--confidence
destroyed!  Mr. Rochester was not to me what he had been; for he was not
what I had thought him.  I would not ascribe vice to him; I would not say
he had betrayed me; but the attribute of stainless truth was gone from
his idea, and from his presence I must go: \emph{that} I perceived well.
When--how--whither, I could not yet discern; but he himself, I doubted
not, would hurry me from Thornfield.  Real affection, it seemed, he could
not have for me; it had been only fitful passion: that was balked; he
would want me no more.  I should fear even to cross his path now: my view
must be hateful to him.  Oh, how blind had been my eyes!  How weak my
conduct!

My eyes were covered and closed: eddying darkness seemed to swim round
me, and reflection came in as black and confused a flow.  Self-abandoned,
relaxed, and effortless, I seemed to have laid me down in the dried-up
bed of a great river; I heard a flood loosened in remote mountains, and
felt the torrent come: to rise I had no will, to flee I had no strength.
I lay faint, longing to be dead.  One idea only still throbbed life-like
within me--a remembrance of God: it begot an unuttered prayer: these
words went wandering up and down in my rayless mind, as something that
should be whispered, but no energy was found to express them--

"Be not far from me, for trouble is near: there is none to help."

It was near: and as I had lifted no petition to Heaven to avert it--as I
had neither joined my hands, nor bent my knees, nor moved my lips--it
came: in full heavy swing the torrent poured over me.  The whole
consciousness of my life lorn, my love lost, my hope quenched, my faith
death-struck, swayed full and mighty above me in one sullen mass.  That
bitter hour cannot be described: in truth, "the waters came into my soul;
I sank in deep mire: I felt no standing; I came into deep waters; the
floods overflowed me."




\chapter{}


\lettrine{S}{ome} time in the afternoon I raised my head, and looking round and seeing
the western sun gilding the sign of its decline on the wall, I asked,
"What am I to do?"

But the answer my mind gave--"Leave Thornfield at once"--was so prompt,
so dread, that I stopped my ears.  I said I could not bear such words
now.  "That I am not Edward Rochester's bride is the least part of my
woe," I alleged: "that I have wakened out of most glorious dreams, and
found them all void and vain, is a horror I could bear and master; but
that I must leave him decidedly, instantly, entirely, is intolerable.  I
cannot do it."

But, then, a voice within me averred that I could do it and foretold that
I should do it.  I wrestled with my own resolution: I wanted to be weak
that I might avoid the awful passage of further suffering I saw laid out
for me; and Conscience, turned tyrant, held Passion by the throat, told
her tauntingly, she had yet but dipped her dainty foot in the slough, and
swore that with that arm of iron he would thrust her down to unsounded
depths of agony.

"Let me be torn away," then I cried.  "Let another help me!"

"No; you shall tear yourself away, none shall help you: you shall
yourself pluck out your right eye; yourself cut off your right hand: your
heart shall be the victim, and you the priest to transfix it."

I rose up suddenly, terror-struck at the solitude which so ruthless a
judge haunted,--at the silence which so awful a voice filled.  My head
swam as I stood erect.  I perceived that I was sickening from excitement
and inanition; neither meat nor drink had passed my lips that day, for I
had taken no breakfast.  And, with a strange pang, I now reflected that,
long as I had been shut up here, no message had been sent to ask how I
was, or to invite me to come down: not even little Adele had tapped at
the door; not even Mrs. Fairfax had sought me.  "Friends always forget
those whom fortune forsakes," I murmured, as I undrew the bolt and passed
out.  I stumbled over an obstacle: my head was still dizzy, my sight was
dim, and my limbs were feeble.  I could not soon recover myself.  I fell,
but not on to the ground: an outstretched arm caught me.  I looked up--I
was supported by Mr. Rochester, who sat in a chair across my chamber
threshold.

"You come out at last," he said.  "Well, I have been waiting for you
long, and listening: yet not one movement have I heard, nor one sob: five
minutes more of that death-like hush, and I should have forced the lock
like a burglar.  So you shun me?--you shut yourself up and grieve alone!
I would rather you had come and upbraided me with vehemence.  You are
passionate.  I expected a scene of some kind.  I was prepared for the hot
rain of tears; only I wanted them to be shed on my breast: now a
senseless floor has received them, or your drenched handkerchief.  But I
err: you have not wept at all!  I see a white cheek and a faded eye, but
no trace of tears.  I suppose, then, your heart has been weeping blood?"

"Well, Jane! not a word of reproach?  Nothing bitter--nothing poignant?
Nothing to cut a feeling or sting a passion?  You sit quietly where I
have placed you, and regard me with a weary, passive look."

"Jane, I never meant to wound you thus.  If the man who had but one
little ewe lamb that was dear to him as a daughter, that ate of his bread
and drank of his cup, and lay in his bosom, had by some mistake
slaughtered it at the shambles, he would not have rued his bloody blunder
more than I now rue mine.  Will you ever forgive me?"

Reader, I forgave him at the moment and on the spot.  There was such deep
remorse in his eye, such true pity in his tone, such manly energy in his
manner; and besides, there was such unchanged love in his whole look and
mien--I forgave him all: yet not in words, not outwardly; only at my
heart's core.

"You know I am a scoundrel, Jane?" ere long he inquired
wistfully--wondering, I suppose, at my continued silence and tameness,
the result rather of weakness than of will.

"Yes, sir."

"Then tell me so roundly and sharply--don't spare me."

"I cannot: I am tired and sick.  I want some water."  He heaved a sort of
shuddering sigh, and taking me in his arms, carried me downstairs.  At
first I did not know to what room he had borne me; all was cloudy to my
glazed sight: presently I felt the reviving warmth of a fire; for, summer
as it was, I had become icy cold in my chamber.  He put wine to my lips;
I tasted it and revived; then I ate something he offered me, and was soon
myself.  I was in the library--sitting in his chair--he was quite near.
"If I could go out of life now, without too sharp a pang, it would be
well for me," I thought; "then I should not have to make the effort of
cracking my heart-strings in rending them from among Mr. Rochester's.  I
must leave him, it appears.  I do not want to leave him--I cannot leave
him."

"How are you now, Jane?"

"Much better, sir; I shall be well soon."

"Taste the wine again, Jane."

I obeyed him; then he put the glass on the table, stood before me, and
looked at me attentively.  Suddenly he turned away, with an inarticulate
exclamation, full of passionate emotion of some kind; he walked fast
through the room and came back; he stooped towards me as if to kiss me;
but I remembered caresses were now forbidden.  I turned my face away and
put his aside.

"What!--How is this?" he exclaimed hastily.  "Oh, I know! you won't kiss
the husband of Bertha Mason?  You consider my arms filled and my embraces
appropriated?"

"At any rate, there is neither room nor claim for me, sir."

"Why, Jane?  I will spare you the trouble of much talking; I will answer
for you--Because I have a wife already, you would reply.--I guess
rightly?"

"Yes."

"If you think so, you must have a strange opinion of me; you must regard
me as a plotting profligate--a base and low rake who has been simulating
disinterested love in order to draw you into a snare deliberately laid,
and strip you of honour and rob you of self-respect.  What do you say to
that?  I see you can say nothing in the first place, you are faint still,
and have enough to do to draw your breath; in the second place, you
cannot yet accustom yourself to accuse and revile me, and besides, the
flood-gates of tears are opened, and they would rush out if you spoke
much; and you have no desire to expostulate, to upbraid, to make a scene:
you are thinking how \emph{to act}--\emph{talking} you consider is of no use.  I
know you--I am on my guard."

"Sir, I do not wish to act against you," I said; and my unsteady voice
warned me to curtail my sentence.

"Not in your sense of the word, but in mine you are scheming to destroy
me.  You have as good as said that I am a married man--as a married man
you will shun me, keep out of my way: just now you have refused to kiss
me.  You intend to make yourself a complete stranger to me: to live under
this roof only as Adele's governess; if ever I say a friendly word to
you, if ever a friendly feeling inclines you again to me, you will
say,--'That man had nearly made me his mistress: I must be ice and rock
to him;' and ice and rock you will accordingly become."

I cleared and steadied my voice to reply: "All is changed about me, sir;
I must change too--there is no doubt of that; and to avoid fluctuations
of feeling, and continual combats with recollections and associations,
there is only one way--Adele must have a new governess, sir."

"Oh, Adele will go to school--I have settled that already; nor do I mean
to torment you with the hideous associations and recollections of
Thornfield Hall--this accursed place--this tent of Achan--this insolent
vault, offering the ghastliness of living death to the light of the open
sky--this narrow stone hell, with its one real fiend, worse than a legion
of such as we imagine.  Jane, you shall not stay here, nor will I.  I was
wrong ever to bring you to Thornfield Hall, knowing as I did how it was
haunted.  I charged them to conceal from you, before I ever saw you, all
knowledge of the curse of the place; merely because I feared Adele never
would have a governess to stay if she knew with what inmate she was
housed, and my plans would not permit me to remove the maniac
elsewhere--though I possess an old house, Ferndean Manor, even more
retired and hidden than this, where I could have lodged her safely
enough, had not a scruple about the unhealthiness of the situation, in
the heart of a wood, made my conscience recoil from the arrangement.
Probably those damp walls would soon have eased me of her charge: but to
each villain his own vice; and mine is not a tendency to indirect
assassination, even of what I most hate.

"Concealing the mad-woman's neighbourhood from you, however, was
something like covering a child with a cloak and laying it down near a
upas-tree: that demon's vicinage is poisoned, and always was.  But I'll
shut up Thornfield Hall: I'll nail up the front door and board the lower
windows: I'll give Mrs. Poole two hundred a year to live here with \emph{my
wife}, as you term that fearful hag: Grace will do much for money, and
she shall have her son, the keeper at Grimsby Retreat, to bear her
company and be at hand to give her aid in the paroxysms, when \emph{my wife}
is prompted by her familiar to burn people in their beds at night, to
stab them, to bite their flesh from their bones, and so on--"

"Sir," I interrupted him, "you are inexorable for that unfortunate lady:
you speak of her with hate--with vindictive antipathy.  It is cruel--she
cannot help being mad."

"Jane, my little darling (so I will call you, for so you are), you don't
know what you are talking about; you misjudge me again: it is not because
she is mad I hate her.  If you were mad, do you think I should hate you?"

"I do indeed, sir."

"Then you are mistaken, and you know nothing about me, and nothing about
the sort of love of which I am capable.  Every atom of your flesh is as
dear to me as my own: in pain and sickness it would still be dear.  Your
mind is my treasure, and if it were broken, it would be my treasure
still: if you raved, my arms should confine you, and not a strait
waistcoat--your grasp, even in fury, would have a charm for me: if you
flew at me as wildly as that woman did this morning, I should receive you
in an embrace, at least as fond as it would be restrictive.  I should not
shrink from you with disgust as I did from her: in your quiet moments you
should have no watcher and no nurse but me; and I could hang over you
with untiring tenderness, though you gave me no smile in return; and
never weary of gazing into your eyes, though they had no longer a ray of
recognition for me.--But why do I follow that train of ideas?  I was
talking of removing you from Thornfield.  All, you know, is prepared for
prompt departure: to-morrow you shall go.  I only ask you to endure one
more night under this roof, Jane; and then, farewell to its miseries and
terrors for ever!  I have a place to repair to, which will be a secure
sanctuary from hateful reminiscences, from unwelcome intrusion--even from
falsehood and slander."

"And take Adele with you, sir," I interrupted; "she will be a companion
for you."

"What do you mean, Jane?  I told you I would send Adele to school; and
what do I want with a child for a companion, and not my own child,--a
French dancer's bastard?  Why do you importune me about her!  I say, why
do you assign Adele to me for a companion?"

"You spoke of a retirement, sir; and retirement and solitude are dull:
too dull for you."

"Solitude! solitude!" he reiterated with irritation.  "I see I must come
to an explanation.  I don't know what sphynx-like expression is forming
in your countenance.  You are to share my solitude.  Do you understand?"

I shook my head: it required a degree of courage, excited as he was
becoming, even to risk that mute sign of dissent.  He had been walking
fast about the room, and he stopped, as if suddenly rooted to one spot.
He looked at me long and hard: I turned my eyes from him, fixed them on
the fire, and tried to assume and maintain a quiet, collected aspect.

"Now for the hitch in Jane's character," he said at last, speaking more
calmly than from his look I had expected him to speak.  "The reel of silk
has run smoothly enough so far; but I always knew there would come a knot
and a puzzle: here it is.  Now for vexation, and exasperation, and
endless trouble!  By God!  I long to exert a fraction of Samson's
strength, and break the entanglement like tow!"

He recommenced his walk, but soon again stopped, and this time just
before me.

"Jane! will you hear reason?" (he stooped and approached his lips to my
ear); "because, if you won't, I'll try violence."  His voice was hoarse;
his look that of a man who is just about to burst an insufferable bond
and plunge headlong into wild license.  I saw that in another moment, and
with one impetus of frenzy more, I should be able to do nothing with him.
The present--the passing second of time--was all I had in which to
control and restrain him--a movement of repulsion, flight, fear would
have sealed my doom,--and his.  But I was not afraid: not in the least.  I
felt an inward power; a sense of influence, which supported me.  The
crisis was perilous; but not without its charm: such as the Indian,
perhaps, feels when he slips over the rapid in his canoe.  I took hold of
his clenched hand, loosened the contorted fingers, and said to him,
soothingly--

"Sit down; I'll talk to you as long as you like, and hear all you have to
say, whether reasonable or unreasonable."

He sat down: but he did not get leave to speak directly.  I had been
struggling with tears for some time: I had taken great pains to repress
them, because I knew he would not like to see me weep.  Now, however, I
considered it well to let them flow as freely and as long as they liked.
If the flood annoyed him, so much the better.  So I gave way and cried
heartily.

Soon I heard him earnestly entreating me to be composed.  I said I could
not while he was in such a passion.

"But I am not angry, Jane: I only love you too well; and you had steeled
your little pale face with such a resolute, frozen look, I could not
endure it.  Hush, now, and wipe your eyes."

His softened voice announced that he was subdued; so I, in my turn,
became calm.  Now he made an effort to rest his head on my shoulder, but
I would not permit it.  Then he would draw me to him: no.

"Jane! Jane!" he said, in such an accent of bitter sadness it thrilled
along every nerve I had; "you don't love me, then?  It was only my
station, and the rank of my wife, that you valued?  Now that you think me
disqualified to become your husband, you recoil from my touch as if I
were some toad or ape."

These words cut me: yet what could I do or I say?  I ought probably to
have done or said nothing; but I was so tortured by a sense of remorse at
thus hurting his feelings, I could not control the wish to drop balm
where I had wounded.

"I \emph{do} love you," I said, "more than ever: but I must not show or
indulge the feeling: and this is the last time I must express it."

"The last time, Jane!  What! do you think you can live with me, and see
me daily, and yet, if you still love me, be always cold and distant?"

"No, sir; that I am certain I could not; and therefore I see there is but
one way: but you will be furious if I mention it."

"Oh, mention it!  If I storm, you have the art of weeping."

"Mr. Rochester, I must leave you."

"For how long, Jane?  For a few minutes, while you smooth your hair--which
is somewhat dishevelled; and bathe your face--which looks feverish?"

"I must leave Adele and Thornfield.  I must part with you for my whole
life: I must begin a new existence among strange faces and strange
scenes."

"Of course: I told you you should.  I pass over the madness about parting
from me.  You mean you must become a part of me.  As to the new
existence, it is all right: you shall yet be my wife: I am not married.
You shall be Mrs. Rochester--both virtually and nominally.  I shall keep
only to you so long as you and I live.  You shall go to a place I have in
the south of France: a whitewashed villa on the shores of the
Mediterranean.  There you shall live a happy, and guarded, and most
innocent life.  Never fear that I wish to lure you into error--to make
you my mistress.  Why did you shake your head?  Jane, you must be
reasonable, or in truth I shall again become frantic."

His voice and hand quivered: his large nostrils dilated; his eye blazed:
still I dared to speak.

"Sir, your wife is living: that is a fact acknowledged this morning by
yourself.  If I lived with you as you desire, I should then be your
mistress: to say otherwise is sophistical--is false."

"Jane, I am not a gentle-tempered man--you forget that: I am not long-
enduring; I am not cool and dispassionate.  Out of pity to me and
yourself, put your finger on my pulse, feel how it throbs, and--beware!"

He bared his wrist, and offered it to me: the blood was forsaking his
cheek and lips, they were growing livid; I was distressed on all hands.
To agitate him thus deeply, by a resistance he so abhorred, was cruel: to
yield was out of the question.  I did what human beings do instinctively
when they are driven to utter extremity--looked for aid to one higher
than man: the words "God help me!" burst involuntarily from my lips.

"I am a fool!" cried Mr. Rochester suddenly.  "I keep telling her I am
not married, and do not explain to her why.  I forget she knows nothing
of the character of that woman, or of the circumstances attending my
infernal union with her.  Oh, I am certain Jane will agree with me in
opinion, when she knows all that I know!  Just put your hand in mine,
Janet--that I may have the evidence of touch as well as sight, to prove
you are near me--and I will in a few words show you the real state of the
case.  Can you listen to me?"

"Yes, sir; for hours if you will."

"I ask only minutes.  Jane, did you ever hear or know that I was not the
eldest son of my house: that I had once a brother older than I?"

"I remember Mrs. Fairfax told me so once."

"And did you ever hear that my father was an avaricious, grasping man?"

"I have understood something to that effect."

"Well, Jane, being so, it was his resolution to keep the property
together; he could not bear the idea of dividing his estate and leaving
me a fair portion: all, he resolved, should go to my brother, Rowland.
Yet as little could he endure that a son of his should be a poor man.  I
must be provided for by a wealthy marriage.  He sought me a partner
betimes.  Mr. Mason, a West India planter and merchant, was his old
acquaintance.  He was certain his possessions were real and vast: he made
inquiries.  Mr. Mason, he found, had a son and daughter; and he learned
from him that he could and would give the latter a fortune of thirty
thousand pounds: that sufficed.  When I left college, I was sent out to
Jamaica, to espouse a bride already courted for me.  My father said
nothing about her money; but he told me Miss Mason was the boast of
Spanish Town for her beauty: and this was no lie.  I found her a fine
woman, in the style of Blanche Ingram: tall, dark, and majestic.  Her
family wished to secure me because I was of a good race; and so did she.
They showed her to me in parties, splendidly dressed.  I seldom saw her
alone, and had very little private conversation with her.  She flattered
me, and lavishly displayed for my pleasure her charms and
accomplishments.  All the men in her circle seemed to admire her and envy
me.  I was dazzled, stimulated: my senses were excited; and being
ignorant, raw, and inexperienced, I thought I loved her.  There is no
folly so besotted that the idiotic rivalries of society, the prurience,
the rashness, the blindness of youth, will not hurry a man to its
commission.  Her relatives encouraged me; competitors piqued me; she
allured me: a marriage was achieved almost before I knew where I was.  Oh,
I have no respect for myself when I think of that act!--an agony of
inward contempt masters me.  I never loved, I never esteemed, I did not
even know her.  I was not sure of the existence of one virtue in her
nature: I had marked neither modesty, nor benevolence, nor candour, nor
refinement in her mind or manners--and, I married her:--gross,
grovelling, mole-eyed blockhead that I was!  With less sin I might
have--But let me remember to whom I am speaking."

"My bride's mother I had never seen: I understood she was dead.  The
honeymoon over, I learned my mistake; she was only mad, and shut up in a
lunatic asylum.  There was a younger brother, too--a complete dumb idiot.
The elder one, whom you have seen (and whom I cannot hate, whilst I abhor
all his kindred, because he has some grains of affection in his feeble
mind, shown in the continued interest he takes in his wretched sister,
and also in a dog-like attachment he once bore me), will probably be in
the same state one day.  My father and my brother Rowland knew all this;
but they thought only of the thirty thousand pounds, and joined in the
plot against me."

"These were vile discoveries; but except for the treachery of
concealment, I should have made them no subject of reproach to my wife,
even when I found her nature wholly alien to mine, her tastes obnoxious
to me, her cast of mind common, low, narrow, and singularly incapable of
being led to anything higher, expanded to anything larger--when I found
that I could not pass a single evening, nor even a single hour of the day
with her in comfort; that kindly conversation could not be sustained
between us, because whatever topic I started, immediately received from
her a turn at once coarse and trite, perverse and imbecile--when I
perceived that I should never have a quiet or settled household, because
no servant would bear the continued outbreaks of her violent and
unreasonable temper, or the vexations of her absurd, contradictory,
exacting orders--even then I restrained myself: I eschewed upbraiding, I
curtailed remonstrance; I tried to devour my repentance and disgust in
secret; I repressed the deep antipathy I felt.

"Jane, I will not trouble you with abominable details: some strong words
shall express what I have to say.  I lived with that woman upstairs four
years, and before that time she had tried me indeed: her character
ripened and developed with frightful rapidity; her vices sprang up fast
and rank: they were so strong, only cruelty could check them, and I would
not use cruelty.  What a pigmy intellect she had, and what giant
propensities!  How fearful were the curses those propensities entailed on
me!  Bertha Mason, the true daughter of an infamous mother, dragged me
through all the hideous and degrading agonies which must attend a man
bound to a wife at once intemperate and unchaste.

"My brother in the interval was dead, and at the end of the four years my
father died too.  I was rich enough now--yet poor to hideous indigence: a
nature the most gross, impure, depraved I ever saw, was associated with
mine, and called by the law and by society a part of me.  And I could not
rid myself of it by any legal proceedings: for the doctors now discovered
that \emph{my wife} was mad--her excesses had prematurely developed the germs
of insanity.  Jane, you don't like my narrative; you look almost
sick--shall I defer the rest to another day?"

"No, sir, finish it now; I pity you--I do earnestly pity you."

"Pity, Jane, from some people is a noxious and insulting sort of tribute,
which one is justified in hurling back in the teeth of those who offer
it; but that is the sort of pity native to callous, selfish hearts; it is
a hybrid, egotistical pain at hearing of woes, crossed with ignorant
contempt for those who have endured them.  But that is not your pity,
Jane; it is not the feeling of which your whole face is full at this
moment--with which your eyes are now almost overflowing--with which your
heart is heaving--with which your hand is trembling in mine.  Your pity,
my darling, is the suffering mother of love: its anguish is the very
natal pang of the divine passion.  I accept it, Jane; let the daughter
have free advent--my arms wait to receive her."

"Now, sir, proceed; what did you do when you found she was mad?"

"Jane, I approached the verge of despair; a remnant of self-respect was
all that intervened between me and the gulf.  In the eyes of the world, I
was doubtless covered with grimy dishonour; but I resolved to be clean in
my own sight--and to the last I repudiated the contamination of her
crimes, and wrenched myself from connection with her mental defects.
Still, society associated my name and person with hers; I yet saw her and
heard her daily: something of her breath (faugh!) mixed with the air I
breathed; and besides, I remembered I had once been her husband--that
recollection was then, and is now, inexpressibly odious to me; moreover,
I knew that while she lived I could never be the husband of another and
better wife; and, though five years my senior (her family and her father
had lied to me even in the particular of her age), she was likely to live
as long as I, being as robust in frame as she was infirm in mind.  Thus,
at the age of twenty-six, I was hopeless.

"One night I had been awakened by her yells--(since the medical men had
pronounced her mad, she had, of course, been shut up)--it was a fiery
West Indian night; one of the description that frequently precede the
hurricanes of those climates.  Being unable to sleep in bed, I got up and
opened the window.  The air was like sulphur-steams--I could find no
refreshment anywhere.  Mosquitoes came buzzing in and hummed sullenly
round the room; the sea, which I could hear from thence, rumbled dull
like an earthquake--black clouds were casting up over it; the moon was
setting in the waves, broad and red, like a hot cannon-ball--she threw
her last bloody glance over a world quivering with the ferment of
tempest.  I was physically influenced by the atmosphere and scene, and my
ears were filled with the curses the maniac still shrieked out; wherein
she momentarily mingled my name with such a tone of demon-hate, with such
language!--no professed harlot ever had a fouler vocabulary than she:
though two rooms off, I heard every word--the thin partitions of the West
India house opposing but slight obstruction to her wolfish cries.

"'This life,' said I at last, 'is hell: this is the air--those are the
sounds of the bottomless pit!  I have a right to deliver myself from it
if I can.  The sufferings of this mortal state will leave me with the
heavy flesh that now cumbers my soul.  Of the fanatic's burning eternity
I have no fear: there is not a future state worse than this present
one--let me break away, and go home to God!'

"I said this whilst I knelt down at, and unlocked a trunk which contained
a brace of loaded pistols: I mean to shoot myself.  I only entertained
the intention for a moment; for, not being insane, the crisis of
exquisite and unalloyed despair, which had originated the wish and design
of self-destruction, was past in a second.

"A wind fresh from Europe blew over the ocean and rushed through the open
casement: the storm broke, streamed, thundered, blazed, and the air grew
pure.  I then framed and fixed a resolution.  While I walked under the
dripping orange-trees of my wet garden, and amongst its drenched
pomegranates and pine-apples, and while the refulgent dawn of the tropics
kindled round me--I reasoned thus, Jane--and now listen; for it was true
Wisdom that consoled me in that hour, and showed me the right path to
follow.

"The sweet wind from Europe was still whispering in the refreshed leaves,
and the Atlantic was thundering in glorious liberty; my heart, dried up
and scorched for a long time, swelled to the tone, and filled with living
blood--my being longed for renewal--my soul thirsted for a pure draught.
I saw hope revive--and felt regeneration possible.  From a flowery arch
at the bottom of my garden I gazed over the sea--bluer than the sky: the
old world was beyond; clear prospects opened thus:--

"'Go,' said Hope, 'and live again in Europe: there it is not known what a
sullied name you bear, nor what a filthy burden is bound to you.  You may
take the maniac with you to England; confine her with due attendance and
precautions at Thornfield: then travel yourself to what clime you will,
and form what new tie you like.  That woman, who has so abused your long-
suffering, so sullied your name, so outraged your honour, so blighted
your youth, is not your wife, nor are you her husband.  See that she is
cared for as her condition demands, and you have done all that God and
humanity require of you.  Let her identity, her connection with yourself,
be buried in oblivion: you are bound to impart them to no living being.
Place her in safety and comfort: shelter her degradation with secrecy,
and leave her.'

"I acted precisely on this suggestion.  My father and brother had not
made my marriage known to their acquaintance; because, in the very first
letter I wrote to apprise them of the union--having already begun to
experience extreme disgust of its consequences, and, from the family
character and constitution, seeing a hideous future opening to me--I
added an urgent charge to keep it secret: and very soon the infamous
conduct of the wife my father had selected for me was such as to make him
blush to own her as his daughter-in-law.  Far from desiring to publish
the connection, he became as anxious to conceal it as myself.

"To England, then, I conveyed her; a fearful voyage I had with such a
monster in the vessel.  Glad was I when I at last got her to Thornfield,
and saw her safely lodged in that third-storey room, of whose secret
inner cabinet she has now for ten years made a wild beast's den--a
goblin's cell.  I had some trouble in finding an attendant for her, as it
was necessary to select one on whose fidelity dependence could be placed;
for her ravings would inevitably betray my secret: besides, she had lucid
intervals of days--sometimes weeks--which she filled up with abuse of me.
At last I hired Grace Poole from the Grimbsy Retreat.  She and the
surgeon, Carter (who dressed Mason's wounds that night he was stabbed and
worried), are the only two I have ever admitted to my confidence.  Mrs.
Fairfax may indeed have suspected something, but she could have gained no
precise knowledge as to facts.  Grace has, on the whole, proved a good
keeper; though, owing partly to a fault of her own, of which it appears
nothing can cure her, and which is incident to her harassing profession,
her vigilance has been more than once lulled and baffled.  The lunatic is
both cunning and malignant; she has never failed to take advantage of her
guardian's temporary lapses; once to secrete the knife with which she
stabbed her brother, and twice to possess herself of the key of her cell,
and issue therefrom in the night-time.  On the first of these occasions,
she perpetrated the attempt to burn me in my bed; on the second, she paid
that ghastly visit to you.  I thank Providence, who watched over you,
that she then spent her fury on your wedding apparel, which perhaps
brought back vague reminiscences of her own bridal days: but on what
might have happened, I cannot endure to reflect.  When I think of the
thing which flew at my throat this morning, hanging its black and scarlet
visage over the nest of my dove, my blood curdles--"

"And what, sir," I asked, while he paused, "did you do when you had
settled her here?  Where did you go?"

"What did I do, Jane?  I transformed myself into a will-o'-the-wisp.
Where did I go?  I pursued wanderings as wild as those of the
March-spirit.  I sought the Continent, and went devious through all its
lands.  My fixed desire was to seek and find a good and intelligent
woman, whom I could love: a contrast to the fury I left at Thornfield--"

"But you could not marry, sir."

"I had determined and was convinced that I could and ought.  It was not
my original intention to deceive, as I have deceived you.  I meant to
tell my tale plainly, and make my proposals openly: and it appeared to me
so absolutely rational that I should be considered free to love and be
loved, I never doubted some woman might be found willing and able to
understand my case and accept me, in spite of the curse with which I was
burdened."

"Well, sir?"

"When you are inquisitive, Jane, you always make me smile.  You open your
eyes like an eager bird, and make every now and then a restless movement,
as if answers in speech did not flow fast enough for you, and you wanted
to read the tablet of one's heart.  But before I go on, tell me what you
mean by your 'Well, sir?'  It is a small phrase very frequent with you;
and which many a time has drawn me on and on through interminable talk: I
don't very well know why."

"I mean,--What next?  How did you proceed?  What came of such an event?"

"Precisely! and what do you wish to know now?"

"Whether you found any one you liked: whether you asked her to marry you;
and what she said."

"I can tell you whether I found any one I liked, and whether I asked her
to marry me: but what she said is yet to be recorded in the book of Fate.
For ten long years I roved about, living first in one capital, then
another: sometimes in St. Petersburg; oftener in Paris; occasionally in
Rome, Naples, and Florence.  Provided with plenty of money and the
passport of an old name, I could choose my own society: no circles were
closed against me.  I sought my ideal of a woman amongst English ladies,
French countesses, Italian signoras, and German grafinnen.  I could not
find her.  Sometimes, for a fleeting moment, I thought I caught a glance,
heard a tone, beheld a form, which announced the realisation of my dream:
but I was presently undeserved.  You are not to suppose that I desired
perfection, either of mind or person.  I longed only for what suited
me--for the antipodes of the Creole: and I longed vainly.  Amongst them
all I found not one whom, had I been ever so free, I--warned as I was of
the risks, the horrors, the loathings of incongruous unions--would have
asked to marry me.  Disappointment made me reckless.  I tried
dissipation--never debauchery: that I hated, and hate.  That was my
Indian Messalina's attribute: rooted disgust at it and her restrained me
much, even in pleasure.  Any enjoyment that bordered on riot seemed to
approach me to her and her vices, and I eschewed it.

"Yet I could not live alone; so I tried the companionship of mistresses.
The first I chose was Celine Varens--another of those steps which make a
man spurn himself when he recalls them.  You already know what she was,
and how my liaison with her terminated.  She had two successors: an
Italian, Giacinta, and a German, Clara; both considered singularly
handsome.  What was their beauty to me in a few weeks?  Giacinta was
unprincipled and violent: I tired of her in three months.  Clara was
honest and quiet; but heavy, mindless, and unimpressible: not one whit to
my taste.  I was glad to give her a sufficient sum to set her up in a
good line of business, and so get decently rid of her.  But, Jane, I see
by your face you are not forming a very favourable opinion of me just
now.  You think me an unfeeling, loose-principled rake: don't you?"

"I don't like you so well as I have done sometimes, indeed, sir.  Did it
not seem to you in the least wrong to live in that way, first with one
mistress and then another?  You talk of it as a mere matter of course."

"It was with me; and I did not like it.  It was a grovelling fashion of
existence: I should never like to return to it.  Hiring a mistress is the
next worse thing to buying a slave: both are often by nature, and always
by position, inferior: and to live familiarly with inferiors is
degrading.  I now hate the recollection of the time I passed with Celine,
Giacinta, and Clara."

I felt the truth of these words; and I drew from them the certain
inference, that if I were so far to forget myself and all the teaching
that had ever been instilled into me, as--under any pretext--with any
justification--through any temptation--to become the successor of these
poor girls, he would one day regard me with the same feeling which now in
his mind desecrated their memory.  I did not give utterance to this
conviction: it was enough to feel it.  I impressed it on my heart, that
it might remain there to serve me as aid in the time of trial.

"Now, Jane, why don't you say 'Well, sir?'  I have not done.  You are
looking grave.  You disapprove of me still, I see.  But let me come to
the point.  Last January, rid of all mistresses--in a harsh, bitter frame
of mind, the result of a useless, roving, lonely life--corroded with
disappointment, sourly disposed against all men, and especially against
all womankind (for I began to regard the notion of an intellectual,
faithful, loving woman as a mere dream), recalled by business, I came
back to England.

"On a frosty winter afternoon, I rode in sight of Thornfield Hall.
Abhorred spot!  I expected no peace--no pleasure there.  On a stile in
Hay Lane I saw a quiet little figure sitting by itself.  I passed it as
negligently as I did the pollard willow opposite to it: I had no
presentiment of what it would be to me; no inward warning that the
arbitress of my life--my genius for good or evil--waited there in humble
guise.  I did not know it, even when, on the occasion of Mesrour's
accident, it came up and gravely offered me help.  Childish and slender
creature!  It seemed as if a linnet had hopped to my foot and proposed to
bear me on its tiny wing.  I was surly; but the thing would not go: it
stood by me with strange perseverance, and looked and spoke with a sort
of authority.  I must be aided, and by that hand: and aided I was.

"When once I had pressed the frail shoulder, something new--a fresh sap
and sense--stole into my frame.  It was well I had learnt that this elf
must return to me--that it belonged to my house down below--or I could
not have felt it pass away from under my hand, and seen it vanish behind
the dim hedge, without singular regret.  I heard you come home that
night, Jane, though probably you were not aware that I thought of you or
watched for you.  The next day I observed you--myself unseen--for half-an-
hour, while you played with Adele in the gallery.  It was a snowy day, I
recollect, and you could not go out of doors.  I was in my room; the door
was ajar: I could both listen and watch.  Adele claimed your outward
attention for a while; yet I fancied your thoughts were elsewhere: but
you were very patient with her, my little Jane; you talked to her and
amused her a long time.  When at last she left you, you lapsed at once
into deep reverie: you betook yourself slowly to pace the gallery.  Now
and then, in passing a casement, you glanced out at the thick-falling
snow; you listened to the sobbing wind, and again you paced gently on and
dreamed.  I think those day visions were not dark: there was a
pleasurable illumination in your eye occasionally, a soft excitement in
your aspect, which told of no bitter, bilious, hypochondriac brooding:
your look revealed rather the sweet musings of youth when its spirit
follows on willing wings the flight of Hope up and on to an ideal heaven.
The voice of Mrs. Fairfax, speaking to a servant in the hall, wakened
you: and how curiously you smiled to and at yourself, Janet!  There was
much sense in your smile: it was very shrewd, and seemed to make light of
your own abstraction.  It seemed to say--'My fine visions are all very
well, but I must not forget they are absolutely unreal.  I have a rosy
sky and a green flowery Eden in my brain; but without, I am perfectly
aware, lies at my feet a rough tract to travel, and around me gather
black tempests to encounter.'  You ran downstairs and demanded of Mrs.
Fairfax some occupation: the weekly house accounts to make up, or
something of that sort, I think it was.  I was vexed with you for getting
out of my sight.

"Impatiently I waited for evening, when I might summon you to my
presence.  An unusual--to me--a perfectly new character I suspected was
yours: I desired to search it deeper and know it better.  You entered the
room with a look and air at once shy and independent: you were quaintly
dressed--much as you are now.  I made you talk: ere long I found you full
of strange contrasts.  Your garb and manner were restricted by rule; your
air was often diffident, and altogether that of one refined by nature,
but absolutely unused to society, and a good deal afraid of making
herself disadvantageously conspicuous by some solecism or blunder; yet
when addressed, you lifted a keen, a daring, and a glowing eye to your
interlocutor's face: there was penetration and power in each glance you
gave; when plied by close questions, you found ready and round answers.
Very soon you seemed to get used to me: I believe you felt the existence
of sympathy between you and your grim and cross master, Jane; for it was
astonishing to see how quickly a certain pleasant ease tranquillised your
manner: snarl as I would, you showed no surprise, fear, annoyance, or
displeasure at my moroseness; you watched me, and now and then smiled at
me with a simple yet sagacious grace I cannot describe.  I was at once
content and stimulated with what I saw: I liked what I had seen, and
wished to see more.  Yet, for a long time, I treated you distantly, and
sought your company rarely.  I was an intellectual epicure, and wished to
prolong the gratification of making this novel and piquant acquaintance:
besides, I was for a while troubled with a haunting fear that if I
handled the flower freely its bloom would fade--the sweet charm of
freshness would leave it.  I did not then know that it was no transitory
blossom, but rather the radiant resemblance of one, cut in an
indestructible gem.  Moreover, I wished to see whether you would seek me
if I shunned you--but you did not; you kept in the schoolroom as still as
your own desk and easel; if by chance I met you, you passed me as soon,
and with as little token of recognition, as was consistent with respect.
Your habitual expression in those days, Jane, was a thoughtful look; not
despondent, for you were not sickly; but not buoyant, for you had little
hope, and no actual pleasure.  I wondered what you thought of me, or if
you ever thought of me, and resolved to find this out.

"I resumed my notice of you.  There was something glad in your glance,
and genial in your manner, when you conversed: I saw you had a social
heart; it was the silent schoolroom--it was the tedium of your life--that
made you mournful.  I permitted myself the delight of being kind to you;
kindness stirred emotion soon: your face became soft in expression, your
tones gentle; I liked my name pronounced by your lips in a grateful happy
accent.  I used to enjoy a chance meeting with you, Jane, at this time:
there was a curious hesitation in your manner: you glanced at me with a
slight trouble--a hovering doubt: you did not know what my caprice might
be--whether I was going to play the master and be stern, or the friend
and be benignant.  I was now too fond of you often to simulate the first
whim; and, when I stretched my hand out cordially, such bloom and light
and bliss rose to your young, wistful features, I had much ado often to
avoid straining you then and there to my heart."

"Don't talk any more of those days, sir," I interrupted, furtively
dashing away some tears from my eyes; his language was torture to me; for
I knew what I must do--and do soon--and all these reminiscences, and
these revelations of his feelings only made my work more difficult.

"No, Jane," he returned: "what necessity is there to dwell on the Past,
when the Present is so much surer--the Future so much brighter?"

I shuddered to hear the infatuated assertion.

"You see now how the case stands--do you not?" he continued.  "After a
youth and manhood passed half in unutterable misery and half in dreary
solitude, I have for the first time found what I can truly love--I have
found you.  You are my sympathy--my better self--my good angel.  I am
bound to you with a strong attachment.  I think you good, gifted, lovely:
a fervent, a solemn passion is conceived in my heart; it leans to you,
draws you to my centre and spring of life, wraps my existence about you,
and, kindling in pure, powerful flame, fuses you and me in one.

"It was because I felt and knew this, that I resolved to marry you.  To
tell me that I had already a wife is empty mockery: you know now that I
had but a hideous demon.  I was wrong to attempt to deceive you; but I
feared a stubbornness that exists in your character.  I feared early
instilled prejudice: I wanted to have you safe before hazarding
confidences.  This was cowardly: I should have appealed to your nobleness
and magnanimity at first, as I do now--opened to you plainly my life of
agony--described to you my hunger and thirst after a higher and worthier
existence--shown to you, not my \emph{resolution} (that word is weak), but my
resistless \emph{bent} to love faithfully and well, where I am faithfully and
well loved in return.  Then I should have asked you to accept my pledge
of fidelity and to give me yours.  Jane--give it me now."

A pause.

"Why are you silent, Jane?"

I was experiencing an ordeal: a hand of fiery iron grasped my vitals.
Terrible moment: full of struggle, blackness, burning!  Not a human being
that ever lived could wish to be loved better than I was loved; and him
who thus loved me I absolutely worshipped: and I must renounce love and
idol.  One drear word comprised my intolerable duty--"Depart!"

"Jane, you understand what I want of you?  Just this promise--'I will be
yours, Mr. Rochester.'"

"Mr. Rochester, I will \emph{not} be yours."

Another long silence.

"Jane!" recommenced he, with a gentleness that broke me down with grief,
and turned me stone-cold with ominous terror--for this still voice was
the pant of a lion rising--"Jane, do you mean to go one way in the world,
and to let me go another?"

"I do."

"Jane" (bending towards and embracing me), "do you mean it now?"

"I do."

"And now?" softly kissing my forehead and cheek.

"I do," extricating myself from restraint rapidly and completely.

"Oh, Jane, this is bitter!  This--this is wicked.  It would not be wicked
to love me."

"It would to obey you."

A wild look raised his brows--crossed his features: he rose; but he
forebore yet.  I laid my hand on the back of a chair for support: I
shook, I feared--but I resolved.

"One instant, Jane.  Give one glance to my horrible life when you are
gone.  All happiness will be torn away with you.  What then is left?  For
a wife I have but the maniac upstairs: as well might you refer me to some
corpse in yonder churchyard.  What shall I do, Jane?  Where turn for a
companion and for some hope?"

"Do as I do: trust in God and yourself.  Believe in heaven.  Hope to meet
again there."

"Then you will not yield?"

"No."

"Then you condemn me to live wretched and to die accursed?"  His voice
rose.

"I advise you to live sinless, and I wish you to die tranquil."

"Then you snatch love and innocence from me?  You fling me back on lust
for a passion--vice for an occupation?"

"Mr. Rochester, I no more assign this fate to you than I grasp at it for
myself.  We were born to strive and endure--you as well as I: do so.  You
will forget me before I forget you."

"You make me a liar by such language: you sully my honour.  I declared I
could not change: you tell me to my face I shall change soon.  And what a
distortion in your judgment, what a perversity in your ideas, is proved
by your conduct!  Is it better to drive a fellow-creature to despair than
to transgress a mere human law, no man being injured by the breach? for
you have neither relatives nor acquaintances whom you need fear to offend
by living with me?"

This was true: and while he spoke my very conscience and reason turned
traitors against me, and charged me with crime in resisting him.  They
spoke almost as loud as Feeling: and that clamoured wildly.  "Oh,
comply!" it said.  "Think of his misery; think of his danger--look at his
state when left alone; remember his headlong nature; consider the
recklessness following on despair--soothe him; save him; love him; tell
him you love him and will be his.  Who in the world cares for \emph{you}? or
who will be injured by what you do?"

Still indomitable was the reply--"\emph{I} care for myself.  The more
solitary, the more friendless, the more unsustained I am, the more I will
respect myself.  I will keep the law given by God; sanctioned by man.  I
will hold to the principles received by me when I was sane, and not
mad--as I am now.  Laws and principles are not for the times when there
is no temptation: they are for such moments as this, when body and soul
rise in mutiny against their rigour; stringent are they; inviolate they
shall be.  If at my individual convenience I might break them, what would
be their worth?  They have a worth--so I have always believed; and if I
cannot believe it now, it is because I am insane--quite insane: with my
veins running fire, and my heart beating faster than I can count its
throbs.  Preconceived opinions, foregone determinations, are all I have
at this hour to stand by: there I plant my foot."

I did.  Mr. Rochester, reading my countenance, saw I had done so.  His
fury was wrought to the highest: he must yield to it for a moment,
whatever followed; he crossed the floor and seized my arm and grasped my
waist.  He seemed to devour me with his flaming glance: physically, I
felt, at the moment, powerless as stubble exposed to the draught and glow
of a furnace: mentally, I still possessed my soul, and with it the
certainty of ultimate safety.  The soul, fortunately, has an
interpreter--often an unconscious, but still a truthful interpreter--in
the eye.  My eye rose to his; and while I looked in his fierce face I
gave an involuntary sigh; his gripe was painful, and my over-taxed
strength almost exhausted.

"Never," said he, as he ground his teeth, "never was anything at once so
frail and so indomitable.  A mere reed she feels in my hand!"  (And he
shook me with the force of his hold.)  "I could bend her with my finger
and thumb: and what good would it do if I bent, if I uptore, if I crushed
her?  Consider that eye: consider the resolute, wild, free thing looking
out of it, defying me, with more than courage--with a stern triumph.
Whatever I do with its cage, I cannot get at it--the savage, beautiful
creature!  If I tear, if I rend the slight prison, my outrage will only
let the captive loose.  Conqueror I might be of the house; but the inmate
would escape to heaven before I could call myself possessor of its clay
dwelling-place.  And it is you, spirit--with will and energy, and virtue
and purity--that I want: not alone your brittle frame.  Of yourself you
could come with soft flight and nestle against my heart, if you would:
seized against your will, you will elude the grasp like an essence--you
will vanish ere I inhale your fragrance.  Oh! come, Jane, come!"

As he said this, he released me from his clutch, and only looked at me.
The look was far worse to resist than the frantic strain: only an idiot,
however, would have succumbed now.  I had dared and baffled his fury; I
must elude his sorrow: I retired to the door.

"You are going, Jane?"

"I am going, sir."

"You are leaving me?"

"Yes."

"You will not come?  You will not be my comforter, my rescuer?  My deep
love, my wild woe, my frantic prayer, are all nothing to you?"

What unutterable pathos was in his voice!  How hard it was to reiterate
firmly, "I am going."

"Jane!"

"Mr. Rochester!"

"Withdraw, then,--I consent; but remember, you leave me here in anguish.
Go up to your own room; think over all I have said, and, Jane, cast a
glance on my sufferings--think of me."

He turned away; he threw himself on his face on the sofa.  "Oh, Jane! my
hope--my love--my life!" broke in anguish from his lips.  Then came a
deep, strong sob.

I had already gained the door; but, reader, I walked back--walked back as
determinedly as I had retreated.  I knelt down by him; I turned his face
from the cushion to me; I kissed his cheek; I smoothed his hair with my
hand.

"God bless you, my dear master!" I said.  "God keep you from harm and
wrong--direct you, solace you--reward you well for your past kindness to
me."

"Little Jane's love would have been my best reward," he answered;
"without it, my heart is broken.  But Jane will give me her love:
yes--nobly, generously."

Up the blood rushed to his face; forth flashed the fire from his eyes;
erect he sprang; he held his arms out; but I evaded the embrace, and at
once quitted the room.

"Farewell!" was the cry of my heart as I left him.  Despair added,
"Farewell for ever!"

* * * * *

That night I never thought to sleep; but a slumber fell on me as soon as
I lay down in bed.  I was transported in thought to the scenes of
childhood: I dreamt I lay in the red-room at Gateshead; that the night
was dark, and my mind impressed with strange fears.  The light that long
ago had struck me into syncope, recalled in this vision, seemed glidingly
to mount the wall, and tremblingly to pause in the centre of the obscured
ceiling.  I lifted up my head to look: the roof resolved to clouds, high
and dim; the gleam was such as the moon imparts to vapours she is about
to sever.  I watched her come--watched with the strangest anticipation;
as though some word of doom were to be written on her disk.  She broke
forth as never moon yet burst from cloud: a hand first penetrated the
sable folds and waved them away; then, not a moon, but a white human form
shone in the azure, inclining a glorious brow earthward.  It gazed and
gazed on me.  It spoke to my spirit: immeasurably distant was the tone,
yet so near, it whispered in my heart--

"My daughter, flee temptation."

"Mother, I will."

So I answered after I had waked from the trance-like dream.  It was yet
night, but July nights are short: soon after midnight, dawn comes.  "It
cannot be too early to commence the task I have to fulfil," thought I.  I
rose: I was dressed; for I had taken off nothing but my shoes.  I knew
where to find in my drawers some linen, a locket, a ring.  In seeking
these articles, I encountered the beads of a pearl necklace Mr. Rochester
had forced me to accept a few days ago.  I left that; it was not mine: it
was the visionary bride's who had melted in air.  The other articles I
made up in a parcel; my purse, containing twenty shillings (it was all I
had), I put in my pocket: I tied on my straw bonnet, pinned my shawl,
took the parcel and my slippers, which I would not put on yet, and stole
from my room.

"Farewell, kind Mrs. Fairfax!" I whispered, as I glided past her door.
"Farewell, my darling Adele!" I said, as I glanced towards the nursery.
No thought could be admitted of entering to embrace her.  I had to
deceive a fine ear: for aught I knew it might now be listening.

I would have got past Mr. Rochester's chamber without a pause; but my
heart momentarily stopping its beat at that threshold, my foot was forced
to stop also.  No sleep was there: the inmate was walking restlessly from
wall to wall; and again and again he sighed while I listened.  There was
a heaven--a temporary heaven--in this room for me, if I chose: I had but
to go in and to say--

"Mr. Rochester, I will love you and live with you through life till
death," and a fount of rapture would spring to my lips.  I thought of
this.

That kind master, who could not sleep now, was waiting with impatience
for day.  He would send for me in the morning; I should be gone.  He
would have me sought for: vainly.  He would feel himself forsaken; his
love rejected: he would suffer; perhaps grow desperate.  I thought of
this too.  My hand moved towards the lock: I caught it back, and glided
on.

Drearily I wound my way downstairs: I knew what I had to do, and I did it
mechanically.  I sought the key of the side-door in the kitchen; I
sought, too, a phial of oil and a feather; I oiled the key and the lock.
I got some water, I got some bread: for perhaps I should have to walk
far; and my strength, sorely shaken of late, must not break down.  All
this I did without one sound.  I opened the door, passed out, shut it
softly.  Dim dawn glimmered in the yard.  The great gates were closed and
locked; but a wicket in one of them was only latched.  Through that I
departed: it, too, I shut; and now I was out of Thornfield.

A mile off, beyond the fields, lay a road which stretched in the contrary
direction to Millcote; a road I had never travelled, but often noticed,
and wondered where it led: thither I bent my steps.  No reflection was to
be allowed now: not one glance was to be cast back; not even one forward.
Not one thought was to be given either to the past or the future.  The
first was a page so heavenly sweet--so deadly sad--that to read one line
of it would dissolve my courage and break down my energy.  The last was
an awful blank: something like the world when the deluge was gone by.

I skirted fields, and hedges, and lanes till after sunrise.  I believe it
was a lovely summer morning: I know my shoes, which I had put on when I
left the house, were soon wet with dew.  But I looked neither to rising
sun, nor smiling sky, nor wakening nature.  He who is taken out to pass
through a fair scene to the scaffold, thinks not of the flowers that
smile on his road, but of the block and axe-edge; of the disseverment of
bone and vein; of the grave gaping at the end: and I thought of drear
flight and homeless wandering--and oh! with agony I thought of what I
left.  I could not help it.  I thought of him now--in his room--watching
the sunrise; hoping I should soon come to say I would stay with him and
be his.  I longed to be his; I panted to return: it was not too late; I
could yet spare him the bitter pang of bereavement.  As yet my flight, I
was sure, was undiscovered.  I could go back and be his comforter--his
pride; his redeemer from misery, perhaps from ruin.  Oh, that fear of his
self-abandonment--far worse than my abandonment--how it goaded me!  It
was a barbed arrow-head in my breast; it tore me when I tried to extract
it; it sickened me when remembrance thrust it farther in.  Birds began
singing in brake and copse: birds were faithful to their mates; birds
were emblems of love.  What was I?  In the midst of my pain of heart and
frantic effort of principle, I abhorred myself.  I had no solace from
self-approbation: none even from self-respect.  I had
injured--wounded--left my master.  I was hateful in my own eyes.  Still I
could not turn, nor retrace one step.  God must have led me on.  As to my
own will or conscience, impassioned grief had trampled one and stifled
the other.  I was weeping wildly as I walked along my solitary way: fast,
fast I went like one delirious.  A weakness, beginning inwardly,
extending to the limbs, seized me, and I fell: I lay on the ground some
minutes, pressing my face to the wet turf.  I had some fear--or hope--that
here I should die: but I was soon up; crawling forwards on my hands and
knees, and then again raised to my feet--as eager and as determined as
ever to reach the road.

When I got there, I was forced to sit to rest me under the hedge; and
while I sat, I heard wheels, and saw a coach come on.  I stood up and
lifted my hand; it stopped.  I asked where it was going: the driver named
a place a long way off, and where I was sure Mr. Rochester had no
connections.  I asked for what sum he would take me there; he said thirty
shillings; I answered I had but twenty; well, he would try to make it do.
He further gave me leave to get into the inside, as the vehicle was
empty: I entered, was shut in, and it rolled on its way.

Gentle reader, may you never feel what I then felt!  May your eyes never
shed such stormy, scalding, heart-wrung tears as poured from mine.  May
you never appeal to Heaven in prayers so hopeless and so agonised as in
that hour left my lips; for never may you, like me, dread to be the
instrument of evil to what you wholly love.




\chapter{}


\lettrine{T}{wo} days are passed.  It is a summer evening; the coachman has set me
down at a place called Whitcross; he could take me no farther for the sum
I had given, and I was not possessed of another shilling in the world.
The coach is a mile off by this time; I am alone.  At this moment I
discover that I forgot to take my parcel out of the pocket of the coach,
where I had placed it for safety; there it remains, there it must remain;
and now, I am absolutely destitute.

Whitcross is no town, nor even a hamlet; it is but a stone pillar set up
where four roads meet: whitewashed, I suppose, to be more obvious at a
distance and in darkness.  Four arms spring from its summit: the nearest
town to which these point is, according to the inscription, distant ten
miles; the farthest, above twenty.  From the well-known names of these
towns I learn in what county I have lighted; a north-midland shire, dusk
with moorland, ridged with mountain: this I see.  There are great moors
behind and on each hand of me; there are waves of mountains far beyond
that deep valley at my feet.  The population here must be thin, and I see
no passengers on these roads: they stretch out east, west, north, and
south--white, broad, lonely; they are all cut in the moor, and the
heather grows deep and wild to their very verge.  Yet a chance traveller
might pass by; and I wish no eye to see me now: strangers would wonder
what I am doing, lingering here at the sign-post, evidently objectless
and lost.  I might be questioned: I could give no answer but what would
sound incredible and excite suspicion.  Not a tie holds me to human
society at this moment--not a charm or hope calls me where my
fellow-creatures are--none that saw me would have a kind thought or a
good wish for me.  I have no relative but the universal mother, Nature: I
will seek her breast and ask repose.

I struck straight into the heath; I held on to a hollow I saw deeply
furrowing the brown moorside; I waded knee-deep in its dark growth; I
turned with its turnings, and finding a moss-blackened granite crag in a
hidden angle, I sat down under it.  High banks of moor were about me; the
crag protected my head: the sky was over that.

Some time passed before I felt tranquil even here: I had a vague dread
that wild cattle might be near, or that some sportsman or poacher might
discover me.  If a gust of wind swept the waste, I looked up, fearing it
was the rush of a bull; if a plover whistled, I imagined it a man.
Finding my apprehensions unfounded, however, and calmed by the deep
silence that reigned as evening declined at nightfall, I took confidence.
As yet I had not thought; I had only listened, watched, dreaded; now I
regained the faculty of reflection.

What was I to do?  Where to go?  Oh, intolerable questions, when I could
do nothing and go nowhere!--when a long way must yet be measured by my
weary, trembling limbs before I could reach human habitation--when cold
charity must be entreated before I could get a lodging: reluctant
sympathy importuned, almost certain repulse incurred, before my tale
could be listened to, or one of my wants relieved!

I touched the heath: it was dry, and yet warm with the heat of the summer
day.  I looked at the sky; it was pure: a kindly star twinkled just above
the chasm ridge.  The dew fell, but with propitious softness; no breeze
whispered.  Nature seemed to me benign and good; I thought she loved me,
outcast as I was; and I, who from man could anticipate only mistrust,
rejection, insult, clung to her with filial fondness.  To-night, at
least, I would be her guest, as I was her child: my mother would lodge me
without money and without price.  I had one morsel of bread yet: the
remnant of a roll I had bought in a town we passed through at noon with a
stray penny--my last coin.  I saw ripe bilberries gleaming here and
there, like jet beads in the heath: I gathered a handful and ate them
with the bread.  My hunger, sharp before, was, if not satisfied, appeased
by this hermit's meal.  I said my evening prayers at its conclusion, and
then chose my couch.

{I said my evening prayers: p311.jpg}

Beside the crag the heath was very deep: when I lay down my feet were
buried in it; rising high on each side, it left only a narrow space for
the night-air to invade.  I folded my shawl double, and spread it over me
for a coverlet; a low, mossy swell was my pillow.  Thus lodged, I was
not, at least--at the commencement of the night, cold.

My rest might have been blissful enough, only a sad heart broke it.  It
plained of its gaping wounds, its inward bleeding, its riven chords.  It
trembled for Mr. Rochester and his doom; it bemoaned him with bitter
pity; it demanded him with ceaseless longing; and, impotent as a bird
with both wings broken, it still quivered its shattered pinions in vain
attempts to seek him.

Worn out with this torture of thought, I rose to my knees.  Night was
come, and her planets were risen: a safe, still night: too serene for the
companionship of fear.  We know that God is everywhere; but certainly we
feel His presence most when His works are on the grandest scale spread
before us; and it is in the unclouded night-sky, where His worlds wheel
their silent course, that we read clearest His infinitude, His
omnipotence, His omnipresence.  I had risen to my knees to pray for Mr.
Rochester.  Looking up, I, with tear-dimmed eyes, saw the mighty Milky-
way.  Remembering what it was--what countless systems there swept space
like a soft trace of light--I felt the might and strength of God.  Sure
was I of His efficiency to save what He had made: convinced I grew that
neither earth should perish, nor one of the souls it treasured.  I turned
my prayer to thanksgiving: the Source of Life was also the Saviour of
spirits.  Mr. Rochester was safe; he was God's, and by God would he be
guarded.  I again nestled to the breast of the hill; and ere long in
sleep forgot sorrow.

But next day, Want came to me pale and bare.  Long after the little birds
had left their nests; long after bees had come in the sweet prime of day
to gather the heath honey before the dew was dried--when the long morning
shadows were curtailed, and the sun filled earth and sky--I got up, and I
looked round me.

What a still, hot, perfect day!  What a golden desert this spreading
moor!  Everywhere sunshine.  I wished I could live in it and on it.  I
saw a lizard run over the crag; I saw a bee busy among the sweet
bilberries.  I would fain at the moment have become bee or lizard, that I
might have found fitting nutriment, permanent shelter here.  But I was a
human being, and had a human being's wants: I must not linger where there
was nothing to supply them.  I rose; I looked back at the bed I had left.
Hopeless of the future, I wished but this--that my Maker had that night
thought good to require my soul of me while I slept; and that this weary
frame, absolved by death from further conflict with fate, had now but to
decay quietly, and mingle in peace with the soil of this wilderness.
Life, however, was yet in my possession, with all its requirements, and
pains, and responsibilities.  The burden must be carried; the want
provided for; the suffering endured; the responsibility fulfilled.  I set
out.

Whitcross regained, I followed a road which led from the sun, now fervent
and high.  By no other circumstance had I will to decide my choice.  I
walked a long time, and when I thought I had nearly done enough, and
might conscientiously yield to the fatigue that almost overpowered
me--might relax this forced action, and, sitting down on a stone I saw
near, submit resistlessly to the apathy that clogged heart and limb--I
heard a bell chime--a church bell.

I turned in the direction of the sound, and there, amongst the romantic
hills, whose changes and aspect I had ceased to note an hour ago, I saw a
hamlet and a spire.  All the valley at my right hand was full of pasture-
fields, and cornfields, and wood; and a glittering stream ran zig-zag
through the varied shades of green, the mellowing grain, the sombre
woodland, the clear and sunny lea.  Recalled by the rumbling of wheels to
the road before me, I saw a heavily-laden waggon labouring up the hill,
and not far beyond were two cows and their drover.  Human life and human
labour were near.  I must struggle on: strive to live and bend to toil
like the rest.

About two o'clock p.m. I entered the village.  At the bottom of its one
street there was a little shop with some cakes of bread in the window.  I
coveted a cake of bread.  With that refreshment I could perhaps regain a
degree of energy: without it, it would be difficult to proceed.  The wish
to have some strength and some vigour returned to me as soon as I was
amongst my fellow-beings.  I felt it would be degrading to faint with
hunger on the causeway of a hamlet.  Had I nothing about me I could offer
in exchange for one of these rolls?  I considered.  I had a small silk
handkerchief tied round my throat; I had my gloves.  I could hardly tell
how men and women in extremities of destitution proceeded.  I did not
know whether either of these articles would be accepted: probably they
would not; but I must try.

I entered the shop: a woman was there.  Seeing a respectably-dressed
person, a lady as she supposed, she came forward with civility.  How
could she serve me?  I was seized with shame: my tongue would not utter
the request I had prepared.  I dared not offer her the half-worn gloves,
the creased handkerchief: besides, I felt it would be absurd.  I only
begged permission to sit down a moment, as I was tired.  Disappointed in
the expectation of a customer, she coolly acceded to my request.  She
pointed to a seat; I sank into it.  I felt sorely urged to weep; but
conscious how unseasonable such a manifestation would be, I restrained
it.  Soon I asked her "if there were any dressmaker or plain-workwoman in
the village?"

"Yes; two or three.  Quite as many as there was employment for."

I reflected.  I was driven to the point now.  I was brought face to face
with Necessity.  I stood in the position of one without a resource,
without a friend, without a coin.  I must do something.  What?  I must
apply somewhere.  Where?

"Did she know of any place in the neighbourhood where a servant was
wanted?"

"Nay; she couldn't say."

"What was the chief trade in this place?  What did most of the people
do?"

"Some were farm labourers; a good deal worked at Mr. Oliver's
needle-factory, and at the foundry."

"Did Mr. Oliver employ women?"

"Nay; it was men's work."

"And what do the women do?"

"I knawn't," was the answer.  "Some does one thing, and some another.
Poor folk mun get on as they can."

She seemed to be tired of my questions: and, indeed, what claim had I to
importune her?  A neighbour or two came in; my chair was evidently
wanted.  I took leave.

I passed up the street, looking as I went at all the houses to the right
hand and to the left; but I could discover no pretext, nor see an
inducement to enter any.  I rambled round the hamlet, going sometimes to
a little distance and returning again, for an hour or more.  Much
exhausted, and suffering greatly now for want of food, I turned aside
into a lane and sat down under the hedge.  Ere many minutes had elapsed,
I was again on my feet, however, and again searching something--a
resource, or at least an informant.  A pretty little house stood at the
top of the lane, with a garden before it, exquisitely neat and
brilliantly blooming.  I stopped at it.  What business had I to approach
the white door or touch the glittering knocker?  In what way could it
possibly be the interest of the inhabitants of that dwelling to serve me?
Yet I drew near and knocked.  A mild-looking, cleanly-attired young woman
opened the door.  In such a voice as might be expected from a hopeless
heart and fainting frame--a voice wretchedly low and faltering--I asked
if a servant was wanted here?

"No," said she; "we do not keep a servant."

"Can you tell me where I could get employment of any kind?" I continued.
"I am a stranger, without acquaintance in this place.  I want some work:
no matter what."

But it was not her business to think for me, or to seek a place for me:
besides, in her eyes, how doubtful must have appeared my character,
position, tale.  She shook her head, she "was sorry she could give me no
information," and the white door closed, quite gently and civilly: but it
shut me out.  If she had held it open a little longer, I believe I should
have begged a piece of bread; for I was now brought low.

I could not bear to return to the sordid village, where, besides, no
prospect of aid was visible.  I should have longed rather to deviate to a
wood I saw not far off, which appeared in its thick shade to offer
inviting shelter; but I was so sick, so weak, so gnawed with nature's
cravings, instinct kept me roaming round abodes where there was a chance
of food.  Solitude would be no solitude--rest no rest--while the vulture,
hunger, thus sank beak and talons in my side.

I drew near houses; I left them, and came back again, and again I
wandered away: always repelled by the consciousness of having no claim to
ask--no right to expect interest in my isolated lot.  Meantime, the
afternoon advanced, while I thus wandered about like a lost and starving
dog.  In crossing a field, I saw the church spire before me: I hastened
towards it.  Near the churchyard, and in the middle of a garden, stood a
well-built though small house, which I had no doubt was the parsonage.  I
remembered that strangers who arrive at a place where they have no
friends, and who want employment, sometimes apply to the clergyman for
introduction and aid.  It is the clergyman's function to help--at least
with advice--those who wished to help themselves.  I seemed to have
something like a right to seek counsel here.  Renewing then my courage,
and gathering my feeble remains of strength, I pushed on.  I reached the
house, and knocked at the kitchen-door.  An old woman opened: I asked was
this the parsonage?

"Yes."

"Was the clergyman in?"

"No."

"Would he be in soon?"

"No, he was gone from home."

"To a distance?"

"Not so far--happen three mile.  He had been called away by the sudden
death of his father: he was at Marsh End now, and would very likely stay
there a fortnight longer."

"Was there any lady of the house?"

"Nay, there was naught but her, and she was housekeeper;" and of her,
reader, I could not bear to ask the relief for want of which I was
sinking; I could not yet beg; and again I crawled away.

Once more I took off my handkerchief--once more I thought of the cakes of
bread in the little shop.  Oh, for but a crust! for but one mouthful to
allay the pang of famine!  Instinctively I turned my face again to the
village; I found the shop again, and I went in; and though others were
there besides the woman I ventured the request--"Would she give me a roll
for this handkerchief?"

She looked at me with evident suspicion: "Nay, she never sold stuff i'
that way."

Almost desperate, I asked for half a cake; she again refused.  "How could
she tell where I had got the handkerchief?" she said.

"Would she take my gloves?"

"No! what could she do with them?"

Reader, it is not pleasant to dwell on these details.  Some say there is
enjoyment in looking back to painful experience past; but at this day I
can scarcely bear to review the times to which I allude: the moral
degradation, blent with the physical suffering, form too distressing a
recollection ever to be willingly dwelt on.  I blamed none of those who
repulsed me.  I felt it was what was to be expected, and what could not
be helped: an ordinary beggar is frequently an object of suspicion; a
well-dressed beggar inevitably so.  To be sure, what I begged was
employment; but whose business was it to provide me with employment?  Not,
certainly, that of persons who saw me then for the first time, and who
knew nothing about my character.  And as to the woman who would not take
my handkerchief in exchange for her bread, why, she was right, if the
offer appeared to her sinister or the exchange unprofitable.  Let me
condense now.  I am sick of the subject.

A little before dark I passed a farm-house, at the open door of which the
farmer was sitting, eating his supper of bread and cheese.  I stopped and
said--

"Will you give me a piece of bread? for I am very hungry."  He cast on me
a glance of surprise; but without answering, he cut a thick slice from
his loaf, and gave it to me.  I imagine he did not think I was a beggar,
but only an eccentric sort of lady, who had taken a fancy to his brown
loaf.  As soon as I was out of sight of his house, I sat down and ate it.

I could not hope to get a lodging under a roof, and sought it in the wood
I have before alluded to.  But my night was wretched, my rest broken: the
ground was damp, the air cold: besides, intruders passed near me more
than once, and I had again and again to change my quarters; no sense of
safety or tranquillity befriended me.  Towards morning it rained; the
whole of the following day was wet.  Do not ask me, reader, to give a
minute account of that day; as before, I sought work; as before, I was
repulsed; as before, I starved; but once did food pass my lips.  At the
door of a cottage I saw a little girl about to throw a mess of cold
porridge into a pig trough.  "Will you give me that?" I asked.

{"Will you give me that?"  I asked: p316.jpg}

She stared at me.  "Mother!" she exclaimed, "there is a woman wants me to
give her these porridge."

"Well lass," replied a voice within, "give it her if she's a beggar.  T'
pig doesn't want it."

The girl emptied the stiffened mould into my hand, and I devoured it
ravenously.

As the wet twilight deepened, I stopped in a solitary bridle-path, which
I had been pursuing an hour or more.

"My strength is quite failing me," I said in a soliloquy.  "I feel I
cannot go much farther.  Shall I be an outcast again this night?  While
the rain descends so, must I lay my head on the cold, drenched ground?  I
fear I cannot do otherwise: for who will receive me?  But it will be very
dreadful, with this feeling of hunger, faintness, chill, and this sense
of desolation--this total prostration of hope.  In all likelihood,
though, I should die before morning.  And why cannot I reconcile myself
to the prospect of death?  Why do I struggle to retain a valueless life?
Because I know, or believe, Mr. Rochester is living: and then, to die of
want and cold is a fate to which nature cannot submit passively.  Oh,
Providence! sustain me a little longer!  Aid!--direct me!"

My glazed eye wandered over the dim and misty landscape.  I saw I had
strayed far from the village: it was quite out of sight.  The very
cultivation surrounding it had disappeared.  I had, by cross-ways and by-
paths, once more drawn near the tract of moorland; and now, only a few
fields, almost as wild and unproductive as the heath from which they were
scarcely reclaimed, lay between me and the dusky hill.

"Well, I would rather die yonder than in a street or on a frequented
road," I reflected.  "And far better that crows and ravens--if any ravens
there be in these regions--should pick my flesh from my bones, than that
they should be prisoned in a workhouse coffin and moulder in a pauper's
grave."

To the hill, then, I turned.  I reached it.  It remained now only to find
a hollow where I could lie down, and feel at least hidden, if not secure.
But all the surface of the waste looked level.  It showed no variation
but of tint: green, where rush and moss overgrew the marshes; black,
where the dry soil bore only heath.  Dark as it was getting, I could
still see these changes, though but as mere alternations of light and
shade; for colour had faded with the daylight.

My eye still roved over the sullen swell and along the moor-edge,
vanishing amidst the wildest scenery, when at one dim point, far in among
the marshes and the ridges, a light sprang up.  "That is an \emph{ignis
fatuus}," was my first thought; and I expected it would soon vanish.  It
burnt on, however, quite steadily, neither receding nor advancing.  "Is
it, then, a bonfire just kindled?" I questioned.  I watched to see
whether it would spread: but no; as it did not diminish, so it did not
enlarge.  "It may be a candle in a house," I then conjectured; "but if
so, I can never reach it.  It is much too far away: and were it within a
yard of me, what would it avail?  I should but knock at the door to have
it shut in my face."

And I sank down where I stood, and hid my face against the ground.  I lay
still a while: the night-wind swept over the hill and over me, and died
moaning in the distance; the rain fell fast, wetting me afresh to the
skin.  Could I but have stiffened to the still frost--the friendly
numbness of death--it might have pelted on; I should not have felt it;
but my yet living flesh shuddered at its chilling influence.  I rose ere
long.

The light was yet there, shining dim but constant through the rain.  I
tried to walk again: I dragged my exhausted limbs slowly towards it.  It
led me aslant over the hill, through a wide bog, which would have been
impassable in winter, and was splashy and shaking even now, in the height
of summer.  Here I fell twice; but as often I rose and rallied my
faculties.  This light was my forlorn hope: I must gain it.

Having crossed the marsh, I saw a trace of white over the moor.  I
approached it; it was a road or a track: it led straight up to the light,
which now beamed from a sort of knoll, amidst a clump of trees--firs,
apparently, from what I could distinguish of the character of their forms
and foliage through the gloom.  My star vanished as I drew near: some
obstacle had intervened between me and it.  I put out my hand to feel the
dark mass before me: I discriminated the rough stones of a low wall--above
it, something like palisades, and within, a high and prickly hedge.  I
groped on.  Again a whitish object gleamed before me: it was a gate--a
wicket; it moved on its hinges as I touched it.  On each side stood a
sable bush-holly or yew.

Entering the gate and passing the shrubs, the silhouette of a house rose
to view, black, low, and rather long; but the guiding light shone
nowhere.  All was obscurity.  Were the inmates retired to rest?  I feared
it must be so.  In seeking the door, I turned an angle: there shot out
the friendly gleam again, from the lozenged panes of a very small
latticed window, within a foot of the ground, made still smaller by the
growth of ivy or some other creeping plant, whose leaves clustered thick
over the portion of the house wall in which it was set.  The aperture was
so screened and narrow, that curtain or shutter had been deemed
unnecessary; and when I stooped down and put aside the spray of foliage
shooting over it, I could see all within.  I could see clearly a room
with a sanded floor, clean scoured; a dresser of walnut, with pewter
plates ranged in rows, reflecting the redness and radiance of a glowing
peat-fire.  I could see a clock, a white deal table, some chairs.  The
candle, whose ray had been my beacon, burnt on the table; and by its
light an elderly woman, somewhat rough-looking, but scrupulously clean,
like all about her, was knitting a stocking.

I noticed these objects cursorily only--in them there was nothing
extraordinary.  A group of more interest appeared near the hearth,
sitting still amidst the rosy peace and warmth suffusing it.  Two young,
graceful women--ladies in every point--sat, one in a low rocking-chair,
the other on a lower stool; both wore deep mourning of crape and
bombazeen, which sombre garb singularly set off very fair necks and
faces: a large old pointer dog rested its massive head on the knee of one
girl--in the lap of the other was cushioned a black cat.

A strange place was this humble kitchen for such occupants!  Who were
they?  They could not be the daughters of the elderly person at the
table; for she looked like a rustic, and they were all delicacy and
cultivation.  I had nowhere seen such faces as theirs: and yet, as I
gazed on them, I seemed intimate with every lineament.  I cannot call
them handsome--they were too pale and grave for the word: as they each
bent over a book, they looked thoughtful almost to severity.  A stand
between them supported a second candle and two great volumes, to which
they frequently referred, comparing them, seemingly, with the smaller
books they held in their hands, like people consulting a dictionary to
aid them in the task of translation.  This scene was as silent as if all
the figures had been shadows and the firelit apartment a picture: so
hushed was it, I could hear the cinders fall from the grate, the clock
tick in its obscure corner; and I even fancied I could distinguish the
click-click of the woman's knitting-needles.  When, therefore, a voice
broke the strange stillness at last, it was audible enough to me.

"Listen, Diana," said one of the absorbed students; "Franz and old Daniel
are together in the night-time, and Franz is telling a dream from which
he has awakened in terror--listen!"  And in a low voice she read
something, of which not one word was intelligible to me; for it was in an
unknown tongue--neither French nor Latin.  Whether it were Greek or
German I could not tell.

"That is strong," she said, when she had finished: "I relish it."  The
other girl, who had lifted her head to listen to her sister, repeated,
while she gazed at the fire, a line of what had been read.  At a later
day, I knew the language and the book; therefore, I will here quote the
line: though, when I first heard it, it was only like a stroke on
sounding brass to me--conveying no meaning:--

"'Da trat hervor Einer, anzusehen wie die Sternen Nacht.'  Good! good!"
she exclaimed, while her dark and deep eye sparkled.  "There you have a
dim and mighty archangel fitly set before you!  The line is worth a
hundred pages of fustian.  'Ich wage die Gedanken in der Schale meines
Zornes und die Werke mit dem Gewichte meines Grimms.'  I like it!"

Both were again silent.

"Is there ony country where they talk i' that way?" asked the old woman,
looking up from her knitting.

"Yes, Hannah--a far larger country than England, where they talk in no
other way."

"Well, for sure case, I knawn't how they can understand t' one t'other:
and if either o' ye went there, ye could tell what they said, I guess?"

"We could probably tell something of what they said, but not all--for we
are not as clever as you think us, Hannah.  We don't speak German, and we
cannot read it without a dictionary to help us."

"And what good does it do you?"

"We mean to teach it some time--or at least the elements, as they say;
and then we shall get more money than we do now."

"Varry like: but give ower studying; ye've done enough for to-night."

"I think we have: at least I'm tired.  Mary, are you?"

"Mortally: after all, it's tough work fagging away at a language with no
master but a lexicon."

"It is, especially such a language as this crabbed but glorious Deutsch.
I wonder when St. John will come home."

"Surely he will not be long now: it is just ten (looking at a little gold
watch she drew from her girdle).  It rains fast, Hannah: will you have
the goodness to look at the fire in the parlour?"

The woman rose: she opened a door, through which I dimly saw a passage:
soon I heard her stir a fire in an inner room; she presently came back.

"Ah, childer!" said she, "it fair troubles me to go into yond' room now:
it looks so lonesome wi' the chair empty and set back in a corner."

She wiped her eyes with her apron: the two girls, grave before, looked
sad now.

"But he is in a better place," continued Hannah: "we shouldn't wish him
here again.  And then, nobody need to have a quieter death nor he had."

"You say he never mentioned us?" inquired one of the ladies.

"He hadn't time, bairn: he was gone in a minute, was your father.  He had
been a bit ailing like the day before, but naught to signify; and when
Mr. St. John asked if he would like either o' ye to be sent for, he fair
laughed at him.  He began again with a bit of a heaviness in his head the
next day--that is, a fortnight sin'--and he went to sleep and niver
wakened: he wor a'most stark when your brother went into t' chamber and
fand him.  Ah, childer! that's t' last o' t' old stock--for ye and Mr.
St. John is like of different soart to them 'at's gone; for all your
mother wor mich i' your way, and a'most as book-learned.  She wor the
pictur' o' ye, Mary: Diana is more like your father."

I thought them so similar I could not tell where the old servant (for
such I now concluded her to be) saw the difference.  Both were fair
complexioned and slenderly made; both possessed faces full of distinction
and intelligence.  One, to be sure, had hair a shade darker than the
other, and there was a difference in their style of wearing it; Mary's
pale brown locks were parted and braided smooth: Diana's duskier tresses
covered her neck with thick curls.  The clock struck ten.

"Ye'll want your supper, I am sure," observed Hannah; "and so will Mr.
St. John when he comes in."

And she proceeded to prepare the meal.  The ladies rose; they seemed
about to withdraw to the parlour.  Till this moment, I had been so intent
on watching them, their appearance and conversation had excited in me so
keen an interest, I had half-forgotten my own wretched position: now it
recurred to me.  More desolate, more desperate than ever, it seemed from
contrast.  And how impossible did it appear to touch the inmates of this
house with concern on my behalf; to make them believe in the truth of my
wants and woes--to induce them to vouchsafe a rest for my wanderings!  As
I groped out the door, and knocked at it hesitatingly, I felt that last
idea to be a mere chimera.  Hannah opened.

"What do you want?" she inquired, in a voice of surprise, as she surveyed
me by the light of the candle she held.

"May I speak to your mistresses?" I said.

"You had better tell me what you have to say to them.  Where do you come
from?"

"I am a stranger."

"What is your business here at this hour?"

"I want a night's shelter in an out-house or anywhere, and a morsel of
bread to eat."

Distrust, the very feeling I dreaded, appeared in Hannah's face.  "I'll
give you a piece of bread," she said, after a pause; "but we can't take
in a vagrant to lodge.  It isn't likely."

"Do let me speak to your mistresses."

"No, not I.  What can they do for you?  You should not be roving about
now; it looks very ill."

"But where shall I go if you drive me away?  What shall I do?"

"Oh, I'll warrant you know where to go and what to do.  Mind you don't do
wrong, that's all.  Here is a penny; now go--"

"A penny cannot feed me, and I have no strength to go farther.  Don't
shut the door:--oh, don't, for God's sake!"

"I must; the rain is driving in--"

"Tell the young ladies.  Let me see them--"

"Indeed, I will not.  You are not what you ought to be, or you wouldn't
make such a noise.  Move off."

"But I must die if I am turned away."

"Not you.  I'm fear'd you have some ill plans agate, that bring you about
folk's houses at this time o' night.  If you've any
followers--housebreakers or such like--anywhere near, you may tell them
we are not by ourselves in the house; we have a gentleman, and dogs, and
guns."  Here the honest but inflexible servant clapped the door to and
bolted it within.

This was the climax.  A pang of exquisite suffering--a throe of true
despair--rent and heaved my heart.  Worn out, indeed, I was; not another
step could I stir.  I sank on the wet doorstep: I groaned--I wrung my
hands--I wept in utter anguish.  Oh, this spectre of death!  Oh, this
last hour, approaching in such horror!  Alas, this isolation--this
banishment from my kind!  Not only the anchor of hope, but the footing of
fortitude was gone--at least for a moment; but the last I soon
endeavoured to regain.

"I can but die," I said, "and I believe in God.  Let me try to wait His
will in silence."

These words I not only thought, but uttered; and thrusting back all my
misery into my heart, I made an effort to compel it to remain there--dumb
and still.

"All men must die," said a voice quite close at hand; "but all are not
condemned to meet a lingering and premature doom, such as yours would be
if you perished here of want."

"Who or what speaks?" I asked, terrified at the unexpected sound, and
incapable now of deriving from any occurrence a hope of aid.  A form was
near--what form, the pitch-dark night and my enfeebled vision prevented
me from distinguishing.  With a loud long knock, the new-comer appealed
to the door.

"Is it you, Mr. St. John?" cried Hannah.

"Yes--yes; open quickly."

"Well, how wet and cold you must be, such a wild night as it is!  Come
in--your sisters are quite uneasy about you, and I believe there are bad
folks about.  There has been a beggar-woman--I declare she is not gone
yet!--laid down there.  Get up! for shame!  Move off, I say!"

"Hush, Hannah!  I have a word to say to the woman.  You have done your
duty in excluding, now let me do mine in admitting her.  I was near, and
listened to both you and her.  I think this is a peculiar case--I must at
least examine into it.  Young woman, rise, and pass before me into the
house."

{Hush, Hannah; I have a word to say to the woman: p323.jpg}

With difficulty I obeyed him.  Presently I stood within that clean,
bright kitchen--on the very hearth--trembling, sickening; conscious of an
aspect in the last degree ghastly, wild, and weather-beaten.  The two
ladies, their brother, Mr. St. John, the old servant, were all gazing at
me.

"St. John, who is it?" I heard one ask.

"I cannot tell: I found her at the door," was the reply.

"She does look white," said Hannah.

"As white as clay or death," was responded.  "She will fall: let her
sit."

And indeed my head swam: I dropped, but a chair received me.  I still
possessed my senses, though just now I could not speak.

"Perhaps a little water would restore her.  Hannah, fetch some.  But she
is worn to nothing.  How very thin, and how very bloodless!"

"A mere spectre!"

"Is she ill, or only famished?"

"Famished, I think.  Hannah, is that milk?  Give it me, and a piece of
bread."

Diana (I knew her by the long curls which I saw drooping between me and
the fire as she bent over me) broke some bread, dipped it in milk, and
put it to my lips.  Her face was near mine: I saw there was pity in it,
and I felt sympathy in her hurried breathing.  In her simple words, too,
the same balm-like emotion spoke: "Try to eat."

"Yes--try," repeated Mary gently; and Mary's hand removed my sodden
bonnet and lifted my head.  I tasted what they offered me: feebly at
first, eagerly soon.

"Not too much at first--restrain her," said the brother; "she has had
enough."  And he withdrew the cup of milk and the plate of bread.

"A little more, St. John--look at the avidity in her eyes."

"No more at present, sister.  Try if she can speak now--ask her her
name."

I felt I could speak, and I answered--"My name is Jane Elliott."  Anxious
as ever to avoid discovery, I had before resolved to assume an \emph{alias}.

"And where do you live?  Where are your friends?"

I was silent.

"Can we send for any one you know?"

I shook my head.

"What account can you give of yourself?"

Somehow, now that I had once crossed the threshold of this house, and
once was brought face to face with its owners, I felt no longer outcast,
vagrant, and disowned by the wide world.  I dared to put off the
mendicant--to resume my natural manner and character.  I began once more
to know myself; and when Mr. St. John demanded an account--which at
present I was far too weak to render--I said after a brief pause--

"Sir, I can give you no details to-night."

"But what, then," said he, "do you expect me to do for you?"

"Nothing," I replied.  My strength sufficed for but short answers.  Diana
took the word--

"Do you mean," she asked, "that we have now given you what aid you
require? and that we may dismiss you to the moor and the rainy night?"

I looked at her.  She had, I thought, a remarkable countenance, instinct
both with power and goodness.  I took sudden courage.  Answering her
compassionate gaze with a smile, I said--"I will trust you.  If I were a
masterless and stray dog, I know that you would not turn me from your
hearth to-night: as it is, I really have no fear.  Do with me and for me
as you like; but excuse me from much discourse--my breath is short--I
feel a spasm when I speak."  All three surveyed me, and all three were
silent.

"Hannah," said Mr. St. John, at last, "let her sit there at present, and
ask her no questions; in ten minutes more, give her the remainder of that
milk and bread.  Mary and Diana, let us go into the parlour and talk the
matter over."

They withdrew.  Very soon one of the ladies returned--I could not tell
which.  A kind of pleasant stupor was stealing over me as I sat by the
genial fire.  In an undertone she gave some directions to Hannah.  Ere
long, with the servant's aid, I contrived to mount a staircase; my
dripping clothes were removed; soon a warm, dry bed received me.  I
thanked God--experienced amidst unutterable exhaustion a glow of grateful
joy--and slept.




\chapter{}


\lettrine{T}{he} recollection of about three days and nights succeeding this is very
dim in my mind.  I can recall some sensations felt in that interval; but
few thoughts framed, and no actions performed.  I knew I was in a small
room and in a narrow bed.  To that bed I seemed to have grown; I lay on
it motionless as a stone; and to have torn me from it would have been
almost to kill me.  I took no note of the lapse of time--of the change
from morning to noon, from noon to evening.  I observed when any one
entered or left the apartment: I could even tell who they were; I could
understand what was said when the speaker stood near to me; but I could
not answer; to open my lips or move my limbs was equally impossible.
Hannah, the servant, was my most frequent visitor.  Her coming disturbed
me.  I had a feeling that she wished me away: that she did not understand
me or my circumstances; that she was prejudiced against me.  Diana and
Mary appeared in the chamber once or twice a day.  They would whisper
sentences of this sort at my bedside--

"It is very well we took her in."

"Yes; she would certainly have been found dead at the door in the morning
had she been left out all night.  I wonder what she has gone through?"

"Strange hardships, I imagine--poor, emaciated, pallid wanderer?"

"She is not an uneducated person, I should think, by her manner of
speaking; her accent was quite pure; and the clothes she took off, though
splashed and wet, were little worn and fine."

"She has a peculiar face; fleshless and haggard as it is, I rather like
it; and when in good health and animated, I can fancy her physiognomy
would be agreeable."

Never once in their dialogues did I hear a syllable of regret at the
hospitality they had extended to me, or of suspicion of, or aversion to,
myself.  I was comforted.

Mr. St. John came but once: he looked at me, and said my state of
lethargy was the result of reaction from excessive and protracted
fatigue.  He pronounced it needless to send for a doctor: nature, he was
sure, would manage best, left to herself.  He said every nerve had been
overstrained in some way, and the whole system must sleep torpid a while.
There was no disease.  He imagined my recovery would be rapid enough when
once commenced.  These opinions he delivered in a few words, in a quiet,
low voice; and added, after a pause, in the tone of a man little
accustomed to expansive comment, "Rather an unusual physiognomy;
certainly, not indicative of vulgarity or degradation."

"Far otherwise," responded Diana.  "To speak truth, St. John, my heart
rather warms to the poor little soul.  I wish we may be able to benefit
her permanently."

"That is hardly likely," was the reply.  "You will find she is some young
lady who has had a misunderstanding with her friends, and has probably
injudiciously left them.  We may, perhaps, succeed in restoring her to
them, if she is not obstinate: but I trace lines of force in her face
which make me sceptical of her tractability."  He stood considering me
some minutes; then added, "She looks sensible, but not at all handsome."

"She is so ill, St. John."

"Ill or well, she would always be plain.  The grace and harmony of beauty
are quite wanting in those features."

On the third day I was better; on the fourth, I could speak, move, rise
in bed, and turn.  Hannah had brought me some gruel and dry toast, about,
as I supposed, the dinner-hour.  I had eaten with relish: the food was
good--void of the feverish flavour which had hitherto poisoned what I had
swallowed.  When she left me, I felt comparatively strong and revived:
ere long satiety of repose and desire for action stirred me.  I wished to
rise; but what could I put on?  Only my damp and bemired apparel; in
which I had slept on the ground and fallen in the marsh.  I felt ashamed
to appear before my benefactors so clad.  I was spared the humiliation.

On a chair by the bedside were all my own things, clean and dry.  My
black silk frock hung against the wall.  The traces of the bog were
removed from it; the creases left by the wet smoothed out: it was quite
decent.  My very shoes and stockings were purified and rendered
presentable.  There were the means of washing in the room, and a comb and
brush to smooth my hair.  After a weary process, and resting every five
minutes, I succeeded in dressing myself.  My clothes hung loose on me;
for I was much wasted, but I covered deficiencies with a shawl, and once
more, clean and respectable looking--no speck of the dirt, no trace of
the disorder I so hated, and which seemed so to degrade me, left--I crept
down a stone staircase with the aid of the banisters, to a narrow low
passage, and found my way presently to the kitchen.

It was full of the fragrance of new bread and the warmth of a generous
fire.  Hannah was baking.  Prejudices, it is well known, are most
difficult to eradicate from the heart whose soil has never been loosened
or fertilised by education: they grow there, firm as weeds among stones.
Hannah had been cold and stiff, indeed, at the first: latterly she had
begun to relent a little; and when she saw me come in tidy and
well-dressed, she even smiled.

"What, you have got up!" she said.  "You are better, then.  You may sit
you down in my chair on the hearthstone, if you will."

She pointed to the rocking-chair: I took it.  She bustled about,
examining me every now and then with the corner of her eye.  Turning to
me, as she took some loaves from the oven, she asked bluntly--

"Did you ever go a-begging afore you came here?"

I was indignant for a moment; but remembering that anger was out of the
question, and that I had indeed appeared as a beggar to her, I answered
quietly, but still not without a certain marked firmness--

"You are mistaken in supposing me a beggar.  I am no beggar; any more
than yourself or your young ladies."

After a pause she said, "I dunnut understand that: you've like no house,
nor no brass, I guess?"

"The want of house or brass (by which I suppose you mean money) does not
make a beggar in your sense of the word."

"Are you book-learned?" she inquired presently.

"Yes, very."

"But you've never been to a boarding-school?"

"I was at a boarding-school eight years."

She opened her eyes wide.  "Whatever cannot ye keep yourself for, then?"

"I have kept myself; and, I trust, shall keep myself again.  What are you
going to do with these gooseberries?" I inquired, as she brought out a
basket of the fruit.

"Mak' 'em into pies."

"Give them to me and I'll pick them."

"Nay; I dunnut want ye to do nought."

"But I must do something.  Let me have them."

She consented; and she even brought me a clean towel to spread over my
dress, "lest," as she said, "I should mucky it."

"Ye've not been used to sarvant's wark, I see by your hands," she
remarked.  "Happen ye've been a dressmaker?"

"No, you are wrong.  And now, never mind what I have been: don't trouble
your head further about me; but tell me the name of the house where we
are."

"Some calls it Marsh End, and some calls it Moor House."

"And the gentleman who lives here is called Mr. St. John?"

"Nay; he doesn't live here: he is only staying a while.  When he is at
home, he is in his own parish at Morton."

"That village a few miles off?

"Aye."

"And what is he?"

"He is a parson."

I remembered the answer of the old housekeeper at the parsonage, when I
had asked to see the clergyman.  "This, then, was his father's
residence?"

"Aye; old Mr. Rivers lived here, and his father, and grandfather, and
gurt (great) grandfather afore him."

"The name, then, of that gentleman, is Mr. St. John Rivers?"

"Aye; St. John is like his kirstened name."

"And his sisters are called Diana and Mary Rivers?"

"Yes."

"Their father is dead?"

"Dead three weeks sin' of a stroke."

"They have no mother?"

"The mistress has been dead this mony a year."

"Have you lived with the family long?"

"I've lived here thirty year.  I nursed them all three."

"That proves you must have been an honest and faithful servant.  I will
say so much for you, though you have had the incivility to call me a
beggar."

She again regarded me with a surprised stare.  "I believe," she said, "I
was quite mista'en in my thoughts of you: but there is so mony cheats
goes about, you mun forgie me."

"And though," I continued, rather severely, "you wished to turn me from
the door, on a night when you should not have shut out a dog."

"Well, it was hard: but what can a body do?  I thought more o' th'
childer nor of mysel: poor things!  They've like nobody to tak' care on
'em but me.  I'm like to look sharpish."

I maintained a grave silence for some minutes.

"You munnut think too hardly of me," she again remarked.

"But I do think hardly of you," I said; "and I'll tell you why--not so
much because you refused to give me shelter, or regarded me as an
impostor, as because you just now made it a species of reproach that I
had no 'brass' and no house.  Some of the best people that ever lived
have been as destitute as I am; and if you are a Christian, you ought not
to consider poverty a crime."

"No more I ought," said she: "Mr. St. John tells me so too; and I see I
wor wrang--but I've clear a different notion on you now to what I had.
You look a raight down dacent little crater."

"That will do--I forgive you now.  Shake hands."

She put her floury and horny hand into mine; another and heartier smile
illumined her rough face, and from that moment we were friends.

Hannah was evidently fond of talking.  While I picked the fruit, and she
made the paste for the pies, she proceeded to give me sundry details
about her deceased master and mistress, and "the childer," as she called
the young people.

Old Mr. Rivers, she said, was a plain man enough, but a gentleman, and of
as ancient a family as could be found.  Marsh End had belonged to the
Rivers ever since it was a house: and it was, she affirmed, "aboon two
hundred year old--for all it looked but a small, humble place, naught to
compare wi' Mr. Oliver's grand hall down i' Morton Vale.  But she could
remember Bill Oliver's father a journeyman needlemaker; and th' Rivers
wor gentry i' th' owd days o' th' Henrys, as onybody might see by looking
into th' registers i' Morton Church vestry."  Still, she allowed, "the
owd maister was like other folk--naught mich out o' t' common way: stark
mad o' shooting, and farming, and sich like."  The mistress was
different.  She was a great reader, and studied a deal; and the "bairns"
had taken after her.  There was nothing like them in these parts, nor
ever had been; they had liked learning, all three, almost from the time
they could speak; and they had always been "of a mak' of their own."  Mr.
St. John, when he grew up, would go to college and be a parson; and the
girls, as soon as they left school, would seek places as governesses: for
they had told her their father had some years ago lost a great deal of
money by a man he had trusted turning bankrupt; and as he was now not
rich enough to give them fortunes, they must provide for themselves.  They
had lived very little at home for a long while, and were only come now to
stay a few weeks on account of their father's death; but they did so like
Marsh End and Morton, and all these moors and hills about.  They had been
in London, and many other grand towns; but they always said there was no
place like home; and then they were so agreeable with each other--never
fell out nor "threaped."  She did not know where there was such a family
for being united.

Having finished my task of gooseberry picking, I asked where the two
ladies and their brother were now.

"Gone over to Morton for a walk; but they would be back in half-an-hour
to tea."

They returned within the time Hannah had allotted them: they entered by
the kitchen door.  Mr. St. John, when he saw me, merely bowed and passed
through; the two ladies stopped: Mary, in a few words, kindly and calmly
expressed the pleasure she felt in seeing me well enough to be able to
come down; Diana took my hand: she shook her head at me.

"You should have waited for my leave to descend," she said.  "You still
look very pale--and so thin!  Poor child!--poor girl!"

Diana had a voice toned, to my ear, like the cooing of a dove.  She
possessed eyes whose gaze I delighted to encounter.  Her whole face
seemed to me full of charm.  Mary's countenance was equally
intelligent--her features equally pretty; but her expression was more
reserved, and her manners, though gentle, more distant.  Diana looked and
spoke with a certain authority: she had a will, evidently.  It was my
nature to feel pleasure in yielding to an authority supported like hers,
and to bend, where my conscience and self-respect permitted, to an active
will.

"And what business have you here?" she continued.  "It is not your place.
Mary and I sit in the kitchen sometimes, because at home we like to be
free, even to license--but you are a visitor, and must go into the
parlour."

"I am very well here."

"Not at all, with Hannah bustling about and covering you with flour."

"Besides, the fire is too hot for you," interposed Mary.

"To be sure," added her sister.  "Come, you must be obedient."  And still
holding my hand she made me rise, and led me into the inner room.

"Sit there," she said, placing me on the sofa, "while we take our things
off and get the tea ready; it is another privilege we exercise in our
little moorland home--to prepare our own meals when we are so inclined,
or when Hannah is baking, brewing, washing, or ironing."

She closed the door, leaving me solus with Mr. St. John, who sat
opposite, a book or newspaper in his hand.  I examined first, the
parlour, and then its occupant.

The parlour was rather a small room, very plainly furnished, yet
comfortable, because clean and neat.  The old-fashioned chairs were very
bright, and the walnut-wood table was like a looking-glass.  A few
strange, antique portraits of the men and women of other days decorated
the stained walls; a cupboard with glass doors contained some books and
an ancient set of china.  There was no superfluous ornament in the
room--not one modern piece of furniture, save a brace of workboxes and a
lady's desk in rosewood, which stood on a side-table:
everything--including the carpet and curtains--looked at once well worn
and well saved.

Mr. St. John--sitting as still as one of the dusty pictures on the walls,
keeping his eyes fixed on the page he perused, and his lips mutely
sealed--was easy enough to examine.  Had he been a statue instead of a
man, he could not have been easier.  He was young--perhaps from twenty-
eight to thirty--tall, slender; his face riveted the eye; it was like a
Greek face, very pure in outline: quite a straight, classic nose; quite
an Athenian mouth and chin.  It is seldom, indeed, an English face comes
so near the antique models as did his.  He might well be a little shocked
at the irregularity of my lineaments, his own being so harmonious.  His
eyes were large and blue, with brown lashes; his high forehead,
colourless as ivory, was partially streaked over by careless locks of
fair hair.

This is a gentle delineation, is it not, reader?  Yet he whom it
describes scarcely impressed one with the idea of a gentle, a yielding,
an impressible, or even of a placid nature.  Quiescent as he now sat,
there was something about his nostril, his mouth, his brow, which, to my
perceptions, indicated elements within either restless, or hard, or
eager.  He did not speak to me one word, nor even direct to me one
glance, till his sisters returned.  Diana, as she passed in and out, in
the course of preparing tea, brought me a little cake, baked on the top
of the oven.

"Eat that now," she said: "you must be hungry.  Hannah says you have had
nothing but some gruel since breakfast."

I did not refuse it, for my appetite was awakened and keen.  Mr. Rivers
now closed his book, approached the table, and, as he took a seat, fixed
his blue pictorial-looking eyes full on me.  There was an unceremonious
directness, a searching, decided steadfastness in his gaze now, which
told that intention, and not diffidence, had hitherto kept it averted
from the stranger.

"You are very hungry," he said.

"I am, sir."  It is my way--it always was my way, by instinct--ever to
meet the brief with brevity, the direct with plainness.

"It is well for you that a low fever has forced you to abstain for the
last three days: there would have been danger in yielding to the cravings
of your appetite at first.  Now you may eat, though still not
immoderately."

"I trust I shall not eat long at your expense, sir," was my very clumsily-
contrived, unpolished answer.

"No," he said coolly: "when you have indicated to us the residence of
your friends, we can write to them, and you may be restored to home."

"That, I must plainly tell you, is out of my power to do; being
absolutely without home and friends."

The three looked at me, but not distrustfully; I felt there was no
suspicion in their glances: there was more of curiosity.  I speak
particularly of the young ladies.  St. John's eyes, though clear enough
in a literal sense, in a figurative one were difficult to fathom.  He
seemed to use them rather as instruments to search other people's
thoughts, than as agents to reveal his own: the which combination of
keenness and reserve was considerably more calculated to embarrass than
to encourage.

"Do you mean to say," he asked, "that you are completely isolated from
every connection?"

"I do.  Not a tie links me to any living thing: not a claim do I possess
to admittance under any roof in England."

"A most singular position at your age!"

Here I saw his glance directed to my hands, which were folded on the
table before me.  I wondered what he sought there: his words soon
explained the quest.

"You have never been married?  You are a spinster?"

Diana laughed.  "Why, she can't be above seventeen or eighteen years old,
St. John," said she.

"I am near nineteen: but I am not married.  No."

I felt a burning glow mount to my face; for bitter and agitating
recollections were awakened by the allusion to marriage.  They all saw
the embarrassment and the emotion.  Diana and Mary relieved me by turning
their eyes elsewhere than to my crimsoned visage; but the colder and
sterner brother continued to gaze, till the trouble he had excited forced
out tears as well as colour.

"Where did you last reside?" he now asked.

"You are too inquisitive, St. John," murmured Mary in a low voice; but he
leaned over the table and required an answer by a second firm and
piercing look.

"The name of the place where, and of the person with whom I lived, is my
secret," I replied concisely.

"Which, if you like, you have, in my opinion, a right to keep, both from
St. John and every other questioner," remarked Diana.

"Yet if I know nothing about you or your history, I cannot help you," he
said.  "And you need help, do you not?"

"I need it, and I seek it so far, sir, that some true philanthropist will
put me in the way of getting work which I can do, and the remuneration
for which will keep me, if but in the barest necessaries of life."

"I know not whether I am a true philanthropist; yet I am willing to aid
you to the utmost of my power in a purpose so honest.  First, then, tell
me what you have been accustomed to do, and what you \emph{can} do."

I had now swallowed my tea.  I was mightily refreshed by the beverage; as
much so as a giant with wine: it gave new tone to my unstrung nerves, and
enabled me to address this penetrating young judge steadily.

"Mr. Rivers," I said, turning to him, and looking at him, as he looked at
me, openly and without diffidence, "you and your sisters have done me a
great service--the greatest man can do his fellow-being; you have rescued
me, by your noble hospitality, from death.  This benefit conferred gives
you an unlimited claim on my gratitude, and a claim, to a certain extent,
on my confidence.  I will tell you as much of the history of the wanderer
you have harboured, as I can tell without compromising my own peace of
mind--my own security, moral and physical, and that of others.

"I am an orphan, the daughter of a clergyman.  My parents died before I
could know them.  I was brought up a dependant; educated in a charitable
institution.  I will even tell you the name of the establishment, where I
passed six years as a pupil, and two as a teacher--Lowood Orphan Asylum,
---shire: you will have heard of it, Mr. Rivers?--the Rev. Robert
Brocklehurst is the treasurer."

"I have heard of Mr. Brocklehurst, and I have seen the school."

"I left Lowood nearly a year since to become a private governess.  I
obtained a good situation, and was happy.  This place I was obliged to
leave four days before I came here.  The reason of my departure I cannot
and ought not to explain: it would be useless, dangerous, and would sound
incredible.  No blame attached to me: I am as free from culpability as
any one of you three.  Miserable I am, and must be for a time; for the
catastrophe which drove me from a house I had found a paradise was of a
strange and direful nature.  I observed but two points in planning my
departure--speed, secrecy: to secure these, I had to leave behind me
everything I possessed except a small parcel; which, in my hurry and
trouble of mind, I forgot to take out of the coach that brought me to
Whitcross.  To this neighbourhood, then, I came, quite destitute.  I
slept two nights in the open air, and wandered about two days without
crossing a threshold: but twice in that space of time did I taste food;
and it was when brought by hunger, exhaustion, and despair almost to the
last gasp, that you, Mr. Rivers, forbade me to perish of want at your
door, and took me under the shelter of your roof.  I know all your
sisters have done for me since--for I have not been insensible during my
seeming torpor--and I owe to their spontaneous, genuine, genial
compassion as large a debt as to your evangelical charity."

"Don't make her talk any more now, St. John," said Diana, as I paused;
"she is evidently not yet fit for excitement.  Come to the sofa and sit
down now, Miss Elliott."

I gave an involuntary half start at hearing the \emph{alias}: I had forgotten
my new name.  Mr. Rivers, whom nothing seemed to escape, noticed it at
once.

"You said your name was Jane Elliott?" he observed.

"I did say so; and it is the name by which I think it expedient to be
called at present, but it is not my real name, and when I hear it, it
sounds strange to me."

"Your real name you will not give?"

"No: I fear discovery above all things; and whatever disclosure would
lead to it, I avoid."

"You are quite right, I am sure," said Diana.  "Now do, brother, let her
be at peace a while."

But when St. John had mused a few moments he recommenced as imperturbably
and with as much acumen as ever.

"You would not like to be long dependent on our hospitality--you would
wish, I see, to dispense as soon as may be with my sisters' compassion,
and, above all, with my \emph{charity} (I am quite sensible of the distinction
drawn, nor do I resent it--it is just): you desire to be independent of
us?"

"I do: I have already said so.  Show me how to work, or how to seek work:
that is all I now ask; then let me go, if it be but to the meanest
cottage; but till then, allow me to stay here: I dread another essay of
the horrors of homeless destitution."

"Indeed you \emph{shall} stay here," said Diana, putting her white hand on my
head.  "You \emph{shall}," repeated Mary, in the tone of undemonstrative
sincerity which seemed natural to her.

"My sisters, you see, have a pleasure in keeping you," said Mr. St. John,
"as they would have a pleasure in keeping and cherishing a half-frozen
bird, some wintry wind might have driven through their casement.  I feel
more inclination to put you in the way of keeping yourself, and shall
endeavour to do so; but observe, my sphere is narrow.  I am but the
incumbent of a poor country parish: my aid must be of the humblest sort.
And if you are inclined to despise the day of small things, seek some
more efficient succour than such as I can offer."

"She has already said that she is willing to do anything honest she can
do," answered Diana for me; "and you know, St. John, she has no choice of
helpers: she is forced to put up with such crusty people as you."

"I will be a dressmaker; I will be a plain-workwoman; I will be a
servant, a nurse-girl, if I can be no better," I answered.

"Right," said Mr. St. John, quite coolly.  "If such is your spirit, I
promise to aid you, in my own time and way."

He now resumed the book with which he had been occupied before tea.  I
soon withdrew, for I had talked as much, and sat up as long, as my
present strength would permit.




\chapter{}


\lettrine{T}{he} more I knew of the inmates of Moor House, the better I liked them.  In
a few days I had so far recovered my health that I could sit up all day,
and walk out sometimes.  I could join with Diana and Mary in all their
occupations; converse with them as much as they wished, and aid them when
and where they would allow me.  There was a reviving pleasure in this
intercourse, of a kind now tasted by me for the first time--the pleasure
arising from perfect congeniality of tastes, sentiments, and principles.

I liked to read what they liked to read: what they enjoyed, delighted me;
what they approved, I reverenced.  They loved their sequestered home.  I,
too, in the grey, small, antique structure, with its low roof, its
latticed casements, its mouldering walls, its avenue of aged firs--all
grown aslant under the stress of mountain winds; its garden, dark with
yew and holly--and where no flowers but of the hardiest species would
bloom--found a charm both potent and permanent.  They clung to the purple
moors behind and around their dwelling--to the hollow vale into which the
pebbly bridle-path leading from their gate descended, and which wound
between fern-banks first, and then amongst a few of the wildest little
pasture-fields that ever bordered a wilderness of heath, or gave
sustenance to a flock of grey moorland sheep, with their little mossy-
faced lambs:--they clung to this scene, I say, with a perfect enthusiasm
of attachment.  I could comprehend the feeling, and share both its
strength and truth.  I saw the fascination of the locality.  I felt the
consecration of its loneliness: my eye feasted on the outline of swell
and sweep--on the wild colouring communicated to ridge and dell by moss,
by heath-bell, by flower-sprinkled turf, by brilliant bracken, and mellow
granite crag.  These details were just to me what they were to them--so
many pure and sweet sources of pleasure.  The strong blast and the soft
breeze; the rough and the halcyon day; the hours of sunrise and sunset;
the moonlight and the clouded night, developed for me, in these regions,
the same attraction as for them--wound round my faculties the same spell
that entranced theirs.

Indoors we agreed equally well.  They were both more accomplished and
better read than I was; but with eagerness I followed in the path of
knowledge they had trodden before me.  I devoured the books they lent me:
then it was full satisfaction to discuss with them in the evening what I
had perused during the day.  Thought fitted thought; opinion met opinion:
we coincided, in short, perfectly.

If in our trio there was a superior and a leader, it was Diana.
Physically, she far excelled me: she was handsome; she was vigorous.  In
her animal spirits there was an affluence of life and certainty of flow,
such as excited my wonder, while it baffled my comprehension.  I could
talk a while when the evening commenced, but the first gush of vivacity
and fluency gone, I was fain to sit on a stool at Diana's feet, to rest
my head on her knee, and listen alternately to her and Mary, while they
sounded thoroughly the topic on which I had but touched.  Diana offered
to teach me German.  I liked to learn of her: I saw the part of
instructress pleased and suited her; that of scholar pleased and suited
me no less.  Our natures dovetailed: mutual affection--of the strongest
kind--was the result.  They discovered I could draw: their pencils and
colour-boxes were immediately at my service.  My skill, greater in this
one point than theirs, surprised and charmed them.  Mary would sit and
watch me by the hour together: then she would take lessons; and a docile,
intelligent, assiduous pupil she made.  Thus occupied, and mutually
entertained, days passed like hours, and weeks like days.

As to Mr. St John, the intimacy which had arisen so naturally and rapidly
between me and his sisters did not extend to him.  One reason of the
distance yet observed between us was, that he was comparatively seldom at
home: a large proportion of his time appeared devoted to visiting the
sick and poor among the scattered population of his parish.

No weather seemed to hinder him in these pastoral excursions: rain or
fair, he would, when his hours of morning study were over, take his hat,
and, followed by his father's old pointer, Carlo, go out on his mission
of love or duty--I scarcely know in which light he regarded it.
Sometimes, when the day was very unfavourable, his sisters would
expostulate.  He would then say, with a peculiar smile, more solemn than
cheerful--

"And if I let a gust of wind or a sprinkling of rain turn me aside from
these easy tasks, what preparation would such sloth be for the future I
propose to myself?"

Diana and Mary's general answer to this question was a sigh, and some
minutes of apparently mournful meditation.

But besides his frequent absences, there was another barrier to
friendship with him: he seemed of a reserved, an abstracted, and even of
a brooding nature.  Zealous in his ministerial labours, blameless in his
life and habits, he yet did not appear to enjoy that mental serenity,
that inward content, which should be the reward of every sincere
Christian and practical philanthropist.  Often, of an evening, when he
sat at the window, his desk and papers before him, he would cease reading
or writing, rest his chin on his hand, and deliver himself up to I know
not what course of thought; but that it was perturbed and exciting might
be seen in the frequent flash and changeful dilation of his eye.

I think, moreover, that Nature was not to him that treasury of delight it
was to his sisters.  He expressed once, and but once in my hearing, a
strong sense of the rugged charm of the hills, and an inborn affection
for the dark roof and hoary walls he called his home; but there was more
of gloom than pleasure in the tone and words in which the sentiment was
manifested; and never did he seem to roam the moors for the sake of their
soothing silence--never seek out or dwell upon the thousand peaceful
delights they could yield.

Incommunicative as he was, some time elapsed before I had an opportunity
of gauging his mind.  I first got an idea of its calibre when I heard him
preach in his own church at Morton.  I wish I could describe that sermon:
but it is past my power.  I cannot even render faithfully the effect it
produced on me.

It began calm--and indeed, as far as delivery and pitch of voice went, it
was calm to the end: an earnestly felt, yet strictly restrained zeal
breathed soon in the distinct accents, and prompted the nervous language.
This grew to force--compressed, condensed, controlled.  The heart was
thrilled, the mind astonished, by the power of the preacher: neither were
softened.  Throughout there was a strange bitterness; an absence of
consolatory gentleness; stern allusions to Calvinistic
doctrines--election, predestination, reprobation--were frequent; and each
reference to these points sounded like a sentence pronounced for doom.
When he had done, instead of feeling better, calmer, more enlightened by
his discourse, I experienced an inexpressible sadness; for it seemed to
me--I know not whether equally so to others--that the eloquence to which
I had been listening had sprung from a depth where lay turbid dregs of
disappointment--where moved troubling impulses of insatiate yearnings and
disquieting aspirations.  I was sure St. John Rivers--pure-lived,
conscientious, zealous as he was--had not yet found that peace of God
which passeth all understanding: he had no more found it, I thought, than
had I with my concealed and racking regrets for my broken idol and lost
elysium--regrets to which I have latterly avoided referring, but which
possessed me and tyrannised over me ruthlessly.

Meantime a month was gone.  Diana and Mary were soon to leave Moor House,
and return to the far different life and scene which awaited them, as
governesses in a large, fashionable, south-of-England city, where each
held a situation in families by whose wealthy and haughty members they
were regarded only as humble dependants, and who neither knew nor sought
out their innate excellences, and appreciated only their acquired
accomplishments as they appreciated the skill of their cook or the taste
of their waiting-woman.  Mr. St. John had said nothing to me yet about
the employment he had promised to obtain for me; yet it became urgent
that I should have a vocation of some kind.  One morning, being left
alone with him a few minutes in the parlour, I ventured to approach the
window-recess--which his table, chair, and desk consecrated as a kind of
study--and I was going to speak, though not very well knowing in what
words to frame my inquiry--for it is at all times difficult to break the
ice of reserve glassing over such natures as his--when he saved me the
trouble by being the first to commence a dialogue.

Looking up as I drew near--"You have a question to ask of me?" he said.

"Yes; I wish to know whether you have heard of any service I can offer
myself to undertake?"

"I found or devised something for you three weeks ago; but as you seemed
both useful and happy here--as my sisters had evidently become attached
to you, and your society gave them unusual pleasure--I deemed it
inexpedient to break in on your mutual comfort till their approaching
departure from Marsh End should render yours necessary."

"And they will go in three days now?" I said.

"Yes; and when they go, I shall return to the parsonage at Morton: Hannah
will accompany me; and this old house will be shut up."

I waited a few moments, expecting he would go on with the subject first
broached: but he seemed to have entered another train of reflection: his
look denoted abstraction from me and my business.  I was obliged to
recall him to a theme which was of necessity one of close and anxious
interest to me.

"What is the employment you had in view, Mr. Rivers?  I hope this delay
will not have increased the difficulty of securing it."

"Oh, no; since it is an employment which depends only on me to give, and
you to accept."

He again paused: there seemed a reluctance to continue.  I grew
impatient: a restless movement or two, and an eager and exacting glance
fastened on his face, conveyed the feeling to him as effectually as words
could have done, and with less trouble.

"You need be in no hurry to hear," he said: "let me frankly tell you, I
have nothing eligible or profitable to suggest.  Before I explain,
recall, if you please, my notice, clearly given, that if I helped you, it
must be as the blind man would help the lame.  I am poor; for I find
that, when I have paid my father's debts, all the patrimony remaining to
me will be this crumbling grange, the row of scathed firs behind, and the
patch of moorish soil, with the yew-trees and holly-bushes in front.  I
am obscure: Rivers is an old name; but of the three sole descendants of
the race, two earn the dependant's crust among strangers, and the third
considers himself an alien from his native country--not only for life,
but in death.  Yes, and deems, and is bound to deem, himself honoured by
the lot, and aspires but after the day when the cross of separation from
fleshly ties shall be laid on his shoulders, and when the Head of that
church-militant of whose humblest members he is one, shall give the word,
'Rise, follow Me!'"

St. John said these words as he pronounced his sermons, with a quiet,
deep voice; with an unflushed cheek, and a coruscating radiance of
glance.  He resumed--

"And since I am myself poor and obscure, I can offer you but a service of
poverty and obscurity.  \emph{You} may even think it degrading--for I see now
your habits have been what the world calls refined: your tastes lean to
the ideal, and your society has at least been amongst the educated; but
\emph{I} consider that no service degrades which can better our race.  I hold
that the more arid and unreclaimed the soil where the Christian
labourer's task of tillage is appointed him--the scantier the meed his
toil brings--the higher the honour.  His, under such circumstances, is
the destiny of the pioneer; and the first pioneers of the Gospel were the
Apostles--their captain was Jesus, the Redeemer, Himself."

"Well?" I said, as he again paused--"proceed."

He looked at me before he proceeded: indeed, he seemed leisurely to read
my face, as if its features and lines were characters on a page.  The
conclusions drawn from this scrutiny he partially expressed in his
succeeding observations.

"I believe you will accept the post I offer you," said he, "and hold it
for a while: not permanently, though: any more than I could permanently
keep the narrow and narrowing--the tranquil, hidden office of English
country incumbent; for in your nature is an alloy as detrimental to
repose as that in mine, though of a different kind."

"Do explain," I urged, when he halted once more.

"I will; and you shall hear how poor the proposal is,--how trivial--how
cramping.  I shall not stay long at Morton, now that my father is dead,
and that I am my own master.  I shall leave the place probably in the
course of a twelve-month; but while I do stay, I will exert myself to the
utmost for its improvement.  Morton, when I came to it two years ago, had
no school: the children of the poor were excluded from every hope of
progress.  I established one for boys: I mean now to open a second school
for girls.  I have hired a building for the purpose, with a cottage of
two rooms attached to it for the mistress's house.  Her salary will be
thirty pounds a year: her house is already furnished, very simply, but
sufficiently, by the kindness of a lady, Miss Oliver; the only daughter
of the sole rich man in my parish--Mr. Oliver, the proprietor of a needle-
factory and iron-foundry in the valley.  The same lady pays for the
education and clothing of an orphan from the workhouse, on condition that
she shall aid the mistress in such menial offices connected with her own
house and the school as her occupation of teaching will prevent her
having time to discharge in person.  Will you be this mistress?"

He put the question rather hurriedly; he seemed half to expect an
indignant, or at least a disdainful rejection of the offer: not knowing
all my thoughts and feelings, though guessing some, he could not tell in
what light the lot would appear to me.  In truth it was humble--but then
it was sheltered, and I wanted a safe asylum: it was plodding--but then,
compared with that of a governess in a rich house, it was independent;
and the fear of servitude with strangers entered my soul like iron: it
was not ignoble--not unworthy--not mentally degrading, I made my
decision.

"I thank you for the proposal, Mr. Rivers, and I accept it with all my
heart."

"But you comprehend me?" he said.  "It is a village school: your scholars
will be only poor girls--cottagers' children--at the best, farmers'
daughters.  Knitting, sewing, reading, writing, ciphering, will be all
you will have to teach.  What will you do with your accomplishments?
What, with the largest portion of your mind--sentiments--tastes?"

"Save them till they are wanted.  They will keep."

"You know what you undertake, then?"

"I do."

He now smiled: and not a bitter or a sad smile, but one well pleased and
deeply gratified.

"And when will you commence the exercise of your function?"

"I will go to my house to-morrow, and open the school, if you like, next
week."

"Very well: so be it."

He rose and walked through the room.  Standing still, he again looked at
me.  He shook his head.

"What do you disapprove of, Mr. Rivers?" I asked.

"You will not stay at Morton long: no, no!"

"Why?  What is your reason for saying so?"

"I read it in your eye; it is not of that description which promises the
maintenance of an even tenor in life."

"I am not ambitious."

He started at the word "ambitious."  He repeated, "No.  What made you
think of ambition?  Who is ambitious?  I know I am: but how did you find
it out?"

"I was speaking of myself."

"Well, if you are not ambitious, you are--"  He paused.

"What?"

"I was going to say, impassioned: but perhaps you would have
misunderstood the word, and been displeased.  I mean, that human
affections and sympathies have a most powerful hold on you.  I am sure
you cannot long be content to pass your leisure in solitude, and to
devote your working hours to a monotonous labour wholly void of stimulus:
any more than I can be content," he added, with emphasis, "to live here
buried in morass, pent in with mountains--my nature, that God gave me,
contravened; my faculties, heaven-bestowed, paralysed--made useless.  You
hear now how I contradict myself.  I, who preached contentment with a
humble lot, and justified the vocation even of hewers of wood and drawers
of water in God's service--I, His ordained minister, almost rave in my
restlessness.  Well, propensities and principles must be reconciled by
some means."

He left the room.  In this brief hour I had learnt more of him than in
the whole previous month: yet still he puzzled me.

Diana and Mary Rivers became more sad and silent as the day approached
for leaving their brother and their home.  They both tried to appear as
usual; but the sorrow they had to struggle against was one that could not
be entirely conquered or concealed.  Diana intimated that this would be a
different parting from any they had ever yet known.  It would probably,
as far as St. John was concerned, be a parting for years: it might be a
parting for life.

"He will sacrifice all to his long-framed resolves," she said: "natural
affection and feelings more potent still.  St. John looks quiet, Jane;
but he hides a fever in his vitals.  You would think him gentle, yet in
some things he is inexorable as death; and the worst of it is, my
conscience will hardly permit me to dissuade him from his severe
decision: certainly, I cannot for a moment blame him for it.  It is
right, noble, Christian: yet it breaks my heart!"  And the tears gushed
to her fine eyes.  Mary bent her head low over her work.

"We are now without father: we shall soon be without home and brother,"
she murmured.

At that moment a little accident supervened, which seemed decreed by fate
purposely to prove the truth of the adage, that "misfortunes never come
singly," and to add to their distresses the vexing one of the slip
between the cup and the lip.  St. John passed the window reading a
letter.  He entered.

"Our uncle John is dead," said he.

Both the sisters seemed struck: not shocked or appalled; the tidings
appeared in their eyes rather momentous than afflicting.

"Dead?" repeated Diana.

"Yes."

She riveted a searching gaze on her brother's face.  "And what then?" she
demanded, in a low voice.

"What then, Die?" he replied, maintaining a marble immobility of feature.
"What then?  Why--nothing.  Read."

He threw the letter into her lap.  She glanced over it, and handed it to
Mary.  Mary perused it in silence, and returned it to her brother.  All
three looked at each other, and all three smiled--a dreary, pensive smile
enough.

"Amen!  We can yet live," said Diana at last.

"At any rate, it makes us no worse off than we were before," remarked
Mary.

"Only it forces rather strongly on the mind the picture of what \emph{might
have been}," said Mr. Rivers, "and contrasts it somewhat too vividly with
what \emph{is}."

He folded the letter, locked it in his desk, and again went out.

For some minutes no one spoke.  Diana then turned to me.

"Jane, you will wonder at us and our mysteries," she said, "and think us
hard-hearted beings not to be more moved at the death of so near a
relation as an uncle; but we have never seen him or known him.  He was my
mother's brother.  My father and he quarrelled long ago.  It was by his
advice that my father risked most of his property in the speculation that
ruined him.  Mutual recrimination passed between them: they parted in
anger, and were never reconciled.  My uncle engaged afterwards in more
prosperous undertakings: it appears he realised a fortune of twenty
thousand pounds.  He was never married, and had no near kindred but
ourselves and one other person, not more closely related than we.  My
father always cherished the idea that he would atone for his error by
leaving his possessions to us; that letter informs us that he has
bequeathed every penny to the other relation, with the exception of
thirty guineas, to be divided between St. John, Diana, and Mary Rivers,
for the purchase of three mourning rings.  He had a right, of course, to
do as he pleased: and yet a momentary damp is cast on the spirits by the
receipt of such news.  Mary and I would have esteemed ourselves rich with
a thousand pounds each; and to St. John such a sum would have been
valuable, for the good it would have enabled him to do."

This explanation given, the subject was dropped, and no further reference
made to it by either Mr. Rivers or his sisters.  The next day I left
Marsh End for Morton.  The day after, Diana and Mary quitted it for
distant B-.  In a week, Mr. Rivers and Hannah repaired to the parsonage:
and so the old grange was abandoned.




\chapter{}


\lettrine{M}{y} home, then, when I at last find a home,--is a cottage; a little room
with whitewashed walls and a sanded floor, containing four painted chairs
and a table, a clock, a cupboard, with two or three plates and dishes,
and a set of tea-things in delf.  Above, a chamber of the same dimensions
as the kitchen, with a deal bedstead and chest of drawers; small, yet too
large to be filled with my scanty wardrobe: though the kindness of my
gentle and generous friends has increased that, by a modest stock of such
things as are necessary.

It is evening.  I have dismissed, with the fee of an orange, the little
orphan who serves me as a handmaid.  I am sitting alone on the hearth.
This morning, the village school opened.  I had twenty scholars.  But
three of the number can read: none write or cipher.  Several knit, and a
few sew a little.  They speak with the broadest accent of the district.
At present, they and I have a difficulty in understanding each other's
language.  Some of them are unmannered, rough, intractable, as well as
ignorant; but others are docile, have a wish to learn, and evince a
disposition that pleases me.  I must not forget that these coarsely-clad
little peasants are of flesh and blood as good as the scions of gentlest
genealogy; and that the germs of native excellence, refinement,
intelligence, kind feeling, are as likely to exist in their hearts as in
those of the best-born.  My duty will be to develop these germs: surely I
shall find some happiness in discharging that office.  Much enjoyment I
do not expect in the life opening before me: yet it will, doubtless, if I
regulate my mind, and exert my powers as I ought, yield me enough to live
on from day to day.

Was I very gleeful, settled, content, during the hours I passed in yonder
bare, humble schoolroom this morning and afternoon?  Not to deceive
myself, I must reply--No: I felt desolate to a degree.  I felt--yes,
idiot that I am--I felt degraded.  I doubted I had taken a step which
sank instead of raising me in the scale of social existence.  I was
weakly dismayed at the ignorance, the poverty, the coarseness of all I
heard and saw round me.  But let me not hate and despise myself too much
for these feelings; I know them to be wrong--that is a great step gained;
I shall strive to overcome them.  To-morrow, I trust, I shall get the
better of them partially; and in a few weeks, perhaps, they will be quite
subdued.  In a few months, it is possible, the happiness of seeing
progress, and a change for the better in my scholars may substitute
gratification for disgust.

Meantime, let me ask myself one question--Which is better?--To have
surrendered to temptation; listened to passion; made no painful effort--no
struggle;--but to have sunk down in the silken snare; fallen asleep on
the flowers covering it; wakened in a southern clime, amongst the
luxuries of a pleasure villa: to have been now living in France, Mr.
Rochester's mistress; delirious with his love half my time--for he
would--oh, yes, he would have loved me well for a while.  He \emph{did} love
me--no one will ever love me so again.  I shall never more know the sweet
homage given to beauty, youth, and grace--for never to any one else shall
I seem to possess these charms.  He was fond and proud of me--it is what
no man besides will ever be.--But where am I wandering, and what am I
saying, and above all, feeling?  Whether is it better, I ask, to be a
slave in a fool's paradise at Marseilles--fevered with delusive bliss one
hour--suffocating with the bitterest tears of remorse and shame the
next--or to be a village-schoolmistress, free and honest, in a breezy
mountain nook in the healthy heart of England?

Yes; I feel now that I was right when I adhered to principle and law, and
scorned and crushed the insane promptings of a frenzied moment.  God
directed me to a correct choice: I thank His providence for the guidance!

Having brought my eventide musings to this point, I rose, went to my
door, and looked at the sunset of the harvest-day, and at the quiet
fields before my cottage, which, with the school, was distant half a mile
from the village.  The birds were singing their last strains--

   "The air was mild, the dew was balm."

While I looked, I thought myself happy, and was surprised to find myself
ere long weeping--and why?  For the doom which had reft me from adhesion
to my master: for him I was no more to see; for the desperate grief and
fatal fury--consequences of my departure--which might now, perhaps, be
dragging him from the path of right, too far to leave hope of ultimate
restoration thither.  At this thought, I turned my face aside from the
lovely sky of eve and lonely vale of Morton--I say \emph{lonely}, for in that
bend of it visible to me there was no building apparent save the church
and the parsonage, half-hid in trees, and, quite at the extremity, the
roof of Vale Hall, where the rich Mr. Oliver and his daughter lived.  I
hid my eyes, and leant my head against the stone frame of my door; but
soon a slight noise near the wicket which shut in my tiny garden from the
meadow beyond it made me look up.  A dog--old Carlo, Mr. Rivers' pointer,
as I saw in a moment--was pushing the gate with his nose, and St. John
himself leant upon it with folded arms; his brow knit, his gaze, grave
almost to displeasure, fixed on me.  I asked him to come in.

"No, I cannot stay; I have only brought you a little parcel my sisters
left for you.  I think it contains a colour-box, pencils, and paper."

I approached to take it: a welcome gift it was.  He examined my face, I
thought, with austerity, as I came near: the traces of tears were
doubtless very visible upon it.

"Have you found your first day's work harder than you expected?" he
asked.

"Oh, no!  On the contrary, I think in time I shall get on with my
scholars very well."

"But perhaps your accommodations--your cottage--your furniture--have
disappointed your expectations?  They are, in truth, scanty enough; but--"
I interrupted--

"My cottage is clean and weather-proof; my furniture sufficient and
commodious.  All I see has made me thankful, not despondent.  I am not
absolutely such a fool and sensualist as to regret the absence of a
carpet, a sofa, and silver plate; besides, five weeks ago I had nothing--I
was an outcast, a beggar, a vagrant; now I have acquaintance, a home, a
business.  I wonder at the goodness of God; the generosity of my friends;
the bounty of my lot.  I do not repine."

"But you feel solitude an oppression?  The little house there behind you
is dark and empty."

"I have hardly had time yet to enjoy a sense of tranquillity, much less
to grow impatient under one of loneliness."

"Very well; I hope you feel the content you express: at any rate, your
good sense will tell you that it is too soon yet to yield to the
vacillating fears of Lot's wife.  What you had left before I saw you, of
course I do not know; but I counsel you to resist firmly every temptation
which would incline you to look back: pursue your present career
steadily, for some months at least."

"It is what I mean to do," I answered.  St. John continued--

"It is hard work to control the workings of inclination and turn the bent
of nature; but that it may be done, I know from experience.  God has
given us, in a measure, the power to make our own fate; and when our
energies seem to demand a sustenance they cannot get--when our will
strains after a path we may not follow--we need neither starve from
inanition, nor stand still in despair: we have but to seek another
nourishment for the mind, as strong as the forbidden food it longed to
taste--and perhaps purer; and to hew out for the adventurous foot a road
as direct and broad as the one Fortune has blocked up against us, if
rougher than it.

"A year ago I was myself intensely miserable, because I thought I had
made a mistake in entering the ministry: its uniform duties wearied me to
death.  I burnt for the more active life of the world--for the more
exciting toils of a literary career--for the destiny of an artist,
author, orator; anything rather than that of a priest: yes, the heart of
a politician, of a soldier, of a votary of glory, a lover of renown, a
luster after power, beat under my curate's surplice.  I considered; my
life was so wretched, it must be changed, or I must die.  After a season
of darkness and struggling, light broke and relief fell: my cramped
existence all at once spread out to a plain without bounds--my powers
heard a call from heaven to rise, gather their full strength, spread
their wings, and mount beyond ken.  God had an errand for me; to bear
which afar, to deliver it well, skill and strength, courage and
eloquence, the best qualifications of soldier, statesman, and orator,
were all needed: for these all centre in the good missionary.

"A missionary I resolved to be.  From that moment my state of mind
changed; the fetters dissolved and dropped from every faculty, leaving
nothing of bondage but its galling soreness--which time only can heal.  My
father, indeed, imposed the determination, but since his death, I have
not a legitimate obstacle to contend with; some affairs settled, a
successor for Morton provided, an entanglement or two of the feelings
broken through or cut asunder--a last conflict with human weakness, in
which I know I shall overcome, because I have vowed that I \emph{will}
overcome--and I leave Europe for the East."

He said this, in his peculiar, subdued, yet emphatic voice; looking, when
he had ceased speaking, not at me, but at the setting sun, at which I
looked too.  Both he and I had our backs towards the path leading up the
field to the wicket.  We had heard no step on that grass-grown track; the
water running in the vale was the one lulling sound of the hour and
scene; we might well then start when a gay voice, sweet as a silver bell,
exclaimed--

"Good evening, Mr. Rivers.  And good evening, old Carlo.  Your dog is
quicker to recognise his friends than you are, sir; he pricked his ears
and wagged his tail when I was at the bottom of the field, and you have
your back towards me now."

It was true.  Though Mr. Rivers had started at the first of those musical
accents, as if a thunderbolt had split a cloud over his head, he stood
yet, at the close of the sentence, in the same attitude in which the
speaker had surprised him--his arm resting on the gate, his face directed
towards the west.  He turned at last, with measured deliberation.  A
vision, as it seemed to me, had risen at his side.  There appeared,
within three feet of him, a form clad in pure white--a youthful, graceful
form: full, yet fine in contour; and when, after bending to caress Carlo,
it lifted up its head, and threw back a long veil, there bloomed under
his glance a face of perfect beauty.  Perfect beauty is a strong
expression; but I do not retrace or qualify it: as sweet features as ever
the temperate clime of Albion moulded; as pure hues of rose and lily as
ever her humid gales and vapoury skies generated and screened, justified,
in this instance, the term.  No charm was wanting, no defect was
perceptible; the young girl had regular and delicate lineaments; eyes
shaped and coloured as we see them in lovely pictures, large, and dark,
and full; the long and shadowy eyelash which encircles a fine eye with so
soft a fascination; the pencilled brow which gives such clearness; the
white smooth forehead, which adds such repose to the livelier beauties of
tint and ray; the cheek oval, fresh, and smooth; the lips, fresh too,
ruddy, healthy, sweetly formed; the even and gleaming teeth without flaw;
the small dimpled chin; the ornament of rich, plenteous tresses--all
advantages, in short, which, combined, realise the ideal of beauty, were
fully hers.  I wondered, as I looked at this fair creature: I admired her
with my whole heart.  Nature had surely formed her in a partial mood;
and, forgetting her usual stinted step-mother dole of gifts, had endowed
this, her darling, with a grand-dame's bounty.

What did St. John Rivers think of this earthly angel?  I naturally asked
myself that question as I saw him turn to her and look at her; and, as
naturally, I sought the answer to the inquiry in his countenance.  He had
already withdrawn his eye from the Peri, and was looking at a humble tuft
of daisies which grew by the wicket.

"A lovely evening, but late for you to be out alone," he said, as he
crushed the snowy heads of the closed flowers with his foot.

"Oh, I only came home from S-" (she mentioned the name of a large town
some twenty miles distant) "this afternoon.  Papa told me you had opened
your school, and that the new mistress was come; and so I put on my
bonnet after tea, and ran up the valley to see her: this is she?"
pointing to me.

"It is," said St. John.

"Do you think you shall like Morton?" she asked of me, with a direct and
naive simplicity of tone and manner, pleasing, if child-like.

"I hope I shall.  I have many inducements to do so."

"Did you find your scholars as attentive as you expected?"

"Quite."

"Do you like your house?"

"Very much."

"Have I furnished it nicely?"

"Very nicely, indeed."

"And made a good choice of an attendant for you in Alice Wood?"

"You have indeed.  She is teachable and handy."  (This then, I thought,
is Miss Oliver, the heiress; favoured, it seems, in the gifts of fortune,
as well as in those of nature!  What happy combination of the planets
presided over her birth, I wonder?)

"I shall come up and help you to teach sometimes," she added.  "It will
be a change for me to visit you now and then; and I like a change.  Mr.
Rivers, I have been \emph{so} gay during my stay at S-.  Last night, or rather
this morning, I was dancing till two o'clock.  The ---th regiment are
stationed there since the riots; and the officers are the most agreeable
men in the world: they put all our young knife-grinders and scissor
merchants to shame."

It seemed to me that Mr. St. John's under lip protruded, and his upper
lip curled a moment.  His mouth certainly looked a good deal compressed,
and the lower part of his face unusually stern and square, as the
laughing girl gave him this information.  He lifted his gaze, too, from
the daisies, and turned it on her.  An unsmiling, a searching, a meaning
gaze it was.  She answered it with a second laugh, and laughter well
became her youth, her roses, her dimples, her bright eyes.

As he stood, mute and grave, she again fell to caressing Carlo.  "Poor
Carlo loves me," said she.  "\emph{He} is not stern and distant to his
friends; and if he could speak, he would not be silent."

As she patted the dog's head, bending with native grace before his young
and austere master, I saw a glow rise to that master's face.  I saw his
solemn eye melt with sudden fire, and flicker with resistless emotion.
Flushed and kindled thus, he looked nearly as beautiful for a man as she
for a woman.  His chest heaved once, as if his large heart, weary of
despotic constriction, had expanded, despite the will, and made a
vigorous bound for the attainment of liberty.  But he curbed it, I think,
as a resolute rider would curb a rearing steed.  He responded neither by
word nor movement to the gentle advances made him.

"Papa says you never come to see us now," continued Miss Oliver, looking
up.  "You are quite a stranger at Vale Hall.  He is alone this evening,
and not very well: will you return with me and visit him?"

"It is not a seasonable hour to intrude on Mr. Oliver," answered St.
John.

"Not a seasonable hour!  But I declare it is.  It is just the hour when
papa most wants company: when the works are closed and he has no business
to occupy him.  Now, Mr. Rivers, \emph{do} come.  Why are you so very shy, and
so very sombre?"  She filled up the hiatus his silence left by a reply of
her own.

"I forgot!" she exclaimed, shaking her beautiful curled head, as if
shocked at herself.  "I am so giddy and thoughtless!  \emph{Do} excuse me.  It
had slipped my memory that you have good reasons to be indisposed for
joining in my chatter.  Diana and Mary have left you, and Moor House is
shut up, and you are so lonely.  I am sure I pity you.  Do come and see
papa."

"Not to-night, Miss Rosamond, not to-night."

Mr. St. John spoke almost like an automaton: himself only knew the effort
it cost him thus to refuse.

"Well, if you are so obstinate, I will leave you; for I dare not stay any
longer: the dew begins to fall.  Good evening!"

She held out her hand.  He just touched it.  "Good evening!" he repeated,
in a voice low and hollow as an echo.  She turned, but in a moment
returned.

"Are you well?" she asked.  Well might she put the question: his face was
blanched as her gown.

"Quite well," he enunciated; and, with a bow, he left the gate.  She went
one way; he another.  She turned twice to gaze after him as she tripped
fairy-like down the field; he, as he strode firmly across, never turned
at all.

This spectacle of another's suffering and sacrifice rapt my thoughts from
exclusive meditation on my own.  Diana Rivers had designated her brother
"inexorable as death."  She had not exaggerated.




\chapter{}


\lettrine{I} continued the labours of the village-school as actively and faithfully
as I could.  It was truly hard work at first.  Some time elapsed before,
with all my efforts, I could comprehend my scholars and their nature.
Wholly untaught, with faculties quite torpid, they seemed to me
hopelessly dull; and, at first sight, all dull alike: but I soon found I
was mistaken.  There was a difference amongst them as amongst the
educated; and when I got to know them, and they me, this difference
rapidly developed itself.  Their amazement at me, my language, my rules,
and ways, once subsided, I found some of these heavy-looking, gaping
rustics wake up into sharp-witted girls enough.  Many showed themselves
obliging, and amiable too; and I discovered amongst them not a few
examples of natural politeness, and innate self-respect, as well as of
excellent capacity, that won both my goodwill and my admiration.  These
soon took a pleasure in doing their work well, in keeping their persons
neat, in learning their tasks regularly, in acquiring quiet and orderly
manners.  The rapidity of their progress, in some instances, was even
surprising; and an honest and happy pride I took in it: besides, I began
personally to like some of the best girls; and they liked me.  I had
amongst my scholars several farmers' daughters: young women grown,
almost.  These could already read, write, and sew; and to them I taught
the elements of grammar, geography, history, and the finer kinds of
needlework.  I found estimable characters amongst them--characters
desirous of information and disposed for improvement--with whom I passed
many a pleasant evening hour in their own homes.  Their parents then (the
farmer and his wife) loaded me with attentions.  There was an enjoyment
in accepting their simple kindness, and in repaying it by a
consideration--a scrupulous regard to their feelings--to which they were
not, perhaps, at all times accustomed, and which both charmed and
benefited them; because, while it elevated them in their own eyes, it
made them emulous to merit the deferential treatment they received.

I felt I became a favourite in the neighbourhood.  Whenever I went out, I
heard on all sides cordial salutations, and was welcomed with friendly
smiles.  To live amidst general regard, though it be but the regard of
working people, is like "sitting in sunshine, calm and sweet;" serene
inward feelings bud and bloom under the ray.  At this period of my life,
my heart far oftener swelled with thankfulness than sank with dejection:
and yet, reader, to tell you all, in the midst of this calm, this useful
existence--after a day passed in honourable exertion amongst my scholars,
an evening spent in drawing or reading contentedly alone--I used to rush
into strange dreams at night: dreams many-coloured, agitated, full of the
ideal, the stirring, the stormy--dreams where, amidst unusual scenes,
charged with adventure, with agitating risk and romantic chance, I still
again and again met Mr. Rochester, always at some exciting crisis; and
then the sense of being in his arms, hearing his voice, meeting his eye,
touching his hand and cheek, loving him, being loved by him--the hope of
passing a lifetime at his side, would be renewed, with all its first
force and fire.  Then I awoke.  Then I recalled where I was, and how
situated.  Then I rose up on my curtainless bed, trembling and quivering;
and then the still, dark night witnessed the convulsion of despair, and
heard the burst of passion.  By nine o'clock the next morning I was
punctually opening the school; tranquil, settled, prepared for the steady
duties of the day.

Rosamond Oliver kept her word in coming to visit me.  Her call at the
school was generally made in the course of her morning ride.  She would
canter up to the door on her pony, followed by a mounted livery servant.
Anything more exquisite than her appearance, in her purple habit, with
her Amazon's cap of black velvet placed gracefully above the long curls
that kissed her cheek and floated to her shoulders, can scarcely be
imagined: and it was thus she would enter the rustic building, and glide
through the dazzled ranks of the village children.  She generally came at
the hour when Mr. Rivers was engaged in giving his daily catechising
lesson.  Keenly, I fear, did the eye of the visitress pierce the young
pastor's heart.  A sort of instinct seemed to warn him of her entrance,
even when he did not see it; and when he was looking quite away from the
door, if she appeared at it, his cheek would glow, and his marble-seeming
features, though they refused to relax, changed indescribably, and in
their very quiescence became expressive of a repressed fervour, stronger
than working muscle or darting glance could indicate.

Of course, she knew her power: indeed, he did not, because he could not,
conceal it from her.  In spite of his Christian stoicism, when she went
up and addressed him, and smiled gaily, encouragingly, even fondly in his
face, his hand would tremble and his eye burn.  He seemed to say, with
his sad and resolute look, if he did not say it with his lips, "I love
you, and I know you prefer me.  It is not despair of success that keeps
me dumb.  If I offered my heart, I believe you would accept it.  But that
heart is already laid on a sacred altar: the fire is arranged round it.
It will soon be no more than a sacrifice consumed."

And then she would pout like a disappointed child; a pensive cloud would
soften her radiant vivacity; she would withdraw her hand hastily from
his, and turn in transient petulance from his aspect, at once so heroic
and so martyr-like.  St. John, no doubt, would have given the world to
follow, recall, retain her, when she thus left him; but he would not give
one chance of heaven, nor relinquish, for the elysium of her love, one
hope of the true, eternal Paradise.  Besides, he could not bind all that
he had in his nature--the rover, the aspirant, the poet, the priest--in
the limits of a single passion.  He could not--he would not--renounce his
wild field of mission warfare for the parlours and the peace of Vale
Hall.  I learnt so much from himself in an inroad I once, despite his
reserve, had the daring to make on his confidence.

Miss Oliver already honoured me with frequent visits to my cottage.  I
had learnt her whole character, which was without mystery or disguise:
she was coquettish but not heartless; exacting, but not worthlessly
selfish.  She had been indulged from her birth, but was not absolutely
spoilt.  She was hasty, but good-humoured; vain (she could not help it,
when every glance in the glass showed her such a flush of loveliness),
but not affected; liberal-handed; innocent of the pride of wealth;
ingenuous; sufficiently intelligent; gay, lively, and unthinking: she was
very charming, in short, even to a cool observer of her own sex like me;
but she was not profoundly interesting or thoroughly impressive.  A very
different sort of mind was hers from that, for instance, of the sisters
of St. John.  Still, I liked her almost as I liked my pupil Adele; except
that, for a child whom we have watched over and taught, a closer
affection is engendered than we can give an equally attractive adult
acquaintance.

She had taken an amiable caprice to me.  She said I was like Mr. Rivers,
only, certainly, she allowed, "not one-tenth so handsome, though I was a
nice neat little soul enough, but he was an angel."  I was, however,
good, clever, composed, and firm, like him.  I was a \emph{lusus naturae}, she
affirmed, as a village schoolmistress: she was sure my previous history,
if known, would make a delightful romance.

One evening, while, with her usual child-like activity, and thoughtless
yet not offensive inquisitiveness, she was rummaging the cupboard and the
table-drawer of my little kitchen, she discovered first two French books,
a volume of Schiller, a German grammar and dictionary, and then my
drawing-materials and some sketches, including a pencil-head of a pretty
little cherub-like girl, one of my scholars, and sundry views from
nature, taken in the Vale of Morton and on the surrounding moors.  She
was first transfixed with surprise, and then electrified with delight.

"Had I done these pictures?  Did I know French and German?  What a
love--what a miracle I was!  I drew better than her master in the first
school in S-.  Would I sketch a portrait of her, to show to papa?"

"With pleasure," I replied; and I felt a thrill of artist-delight at the
idea of copying from so perfect and radiant a model.  She had then on a
dark-blue silk dress; her arms and her neck were bare; her only ornament
was her chestnut tresses, which waved over her shoulders with all the
wild grace of natural curls.  I took a sheet of fine card-board, and drew
a careful outline.  I promised myself the pleasure of colouring it; and,
as it was getting late then, I told her she must come and sit another
day.

She made such a report of me to her father, that Mr. Oliver himself
accompanied her next evening--a tall, massive-featured, middle-aged, and
grey-headed man, at whose side his lovely daughter looked like a bright
flower near a hoary turret.  He appeared a taciturn, and perhaps a proud
personage; but he was very kind to me.  The sketch of Rosamond's portrait
pleased him highly: he said I must make a finished picture of it.  He
insisted, too, on my coming the next day to spend the evening at Vale
Hall.

I went.  I found it a large, handsome residence, showing abundant
evidences of wealth in the proprietor.  Rosamond was full of glee and
pleasure all the time I stayed.  Her father was affable; and when he
entered into conversation with me after tea, he expressed in strong terms
his approbation of what I had done in Morton school, and said he only
feared, from what he saw and heard, I was too good for the place, and
would soon quit it for one more suitable.

"Indeed," cried Rosamond, "she is clever enough to be a governess in a
high family, papa."

I thought I would far rather be where I am than in any high family in the
land.  Mr. Oliver spoke of Mr. Rivers--of the Rivers family--with great
respect.  He said it was a very old name in that neighbourhood; that the
ancestors of the house were wealthy; that all Morton had once belonged to
them; that even now he considered the representative of that house might,
if he liked, make an alliance with the best.  He accounted it a pity that
so fine and talented a young man should have formed the design of going
out as a missionary; it was quite throwing a valuable life away.  It
appeared, then, that her father would throw no obstacle in the way of
Rosamond's union with St. John.  Mr. Oliver evidently regarded the young
clergyman's good birth, old name, and sacred profession as sufficient
compensation for the want of fortune.

It was the 5th of November, and a holiday.  My little servant, after
helping me to clean my house, was gone, well satisfied with the fee of a
penny for her aid.  All about me was spotless and bright--scoured floor,
polished grate, and well-rubbed chairs.  I had also made myself neat, and
had now the afternoon before me to spend as I would.

The translation of a few pages of German occupied an hour; then I got my
palette and pencils, and fell to the more soothing, because easier
occupation, of completing Rosamond Oliver's miniature.  The head was
finished already: there was but the background to tint and the drapery to
shade off; a touch of carmine, too, to add to the ripe lips--a soft curl
here and there to the tresses--a deeper tinge to the shadow of the lash
under the azured eyelid.  I was absorbed in the execution of these nice
details, when, after one rapid tap, my door unclosed, admitting St. John
Rivers.

"I am come to see how you are spending your holiday," he said.  "Not, I
hope, in thought?  No, that is well: while you draw you will not feel
lonely.  You see, I mistrust you still, though you have borne up
wonderfully so far.  I have brought you a book for evening solace," and
he laid on the table a new publication--a poem: one of those genuine
productions so often vouchsafed to the fortunate public of those days--the
golden age of modern literature.  Alas! the readers of our era are less
favoured.  But courage!  I will not pause either to accuse or repine.  I
know poetry is not dead, nor genius lost; nor has Mammon gained power
over either, to bind or slay: they will both assert their existence,
their presence, their liberty and strength again one day.  Powerful
angels, safe in heaven! they smile when sordid souls triumph, and feeble
ones weep over their destruction.  Poetry destroyed?  Genius banished?
No!  Mediocrity, no: do not let envy prompt you to the thought.  No; they
not only live, but reign and redeem: and without their divine influence
spread everywhere, you would be in hell--the hell of your own meanness.

While I was eagerly glancing at the bright pages of "Marmion" (for
"Marmion" it was), St. John stooped to examine my drawing.  His tall
figure sprang erect again with a start: he said nothing.  I looked up at
him: he shunned my eye.  I knew his thoughts well, and could read his
heart plainly; at the moment I felt calmer and cooler than he: I had then
temporarily the advantage of him, and I conceived an inclination to do
him some good, if I could.

"With all his firmness and self-control," thought I, "he tasks himself
too far: locks every feeling and pang within--expresses, confesses,
imparts nothing.  I am sure it would benefit him to talk a little about
this sweet Rosamond, whom he thinks he ought not to marry: I will make
him talk."

I said first, "Take a chair, Mr. Rivers."  But he answered, as he always
did, that he could not stay.  "Very well," I responded, mentally, "stand
if you like; but you shall not go just yet, I am determined: solitude is
at least as bad for you as it is for me.  I'll try if I cannot discover
the secret spring of your confidence, and find an aperture in that marble
breast through which I can shed one drop of the balm of sympathy."

"Is this portrait like?" I asked bluntly.

"Like!  Like whom?  I did not observe it closely."

"You did, Mr. Rivers."

He almost started at my sudden and strange abruptness: he looked at me
astonished.  "Oh, that is nothing yet," I muttered within.  "I don't mean
to be baffled by a little stiffness on your part; I'm prepared to go to
considerable lengths."  I continued, "You observed it closely and
distinctly; but I have no objection to your looking at it again," and I
rose and placed it in his hand.

"A well-executed picture," he said; "very soft, clear colouring; very
graceful and correct drawing."

"Yes, yes; I know all that.  But what of the resemblance?  Who is it
like?"

Mastering some hesitation, he answered, "Miss Oliver, I presume."

"Of course.  And now, sir, to reward you for the accurate guess, I will
promise to paint you a careful and faithful duplicate of this very
picture, provided you admit that the gift would be acceptable to you.  I
don't wish to throw away my time and trouble on an offering you would
deem worthless."

He continued to gaze at the picture: the longer he looked, the firmer he
held it, the more he seemed to covet it.  "It is like!" he murmured; "the
eye is well managed: the colour, light, expression, are perfect.  It
smiles!"

"Would it comfort, or would it wound you to have a similar painting?  Tell
me that.  When you are at Madagascar, or at the Cape, or in India, would
it be a consolation to have that memento in your possession? or would the
sight of it bring recollections calculated to enervate and distress?"

He now furtively raised his eyes: he glanced at me, irresolute,
disturbed: he again surveyed the picture.

"That I should like to have it is certain: whether it would be judicious
or wise is another question."

Since I had ascertained that Rosamond really preferred him, and that her
father was not likely to oppose the match, I--less exalted in my views
than St. John--had been strongly disposed in my own heart to advocate
their union.  It seemed to me that, should he become the possessor of Mr.
Oliver's large fortune, he might do as much good with it as if he went
and laid his genius out to wither, and his strength to waste, under a
tropical sun.  With this persuasion I now answered--

"As far as I can see, it would be wiser and more judicious if you were to
take to yourself the original at once."

By this time he had sat down: he had laid the picture on the table before
him, and with his brow supported on both hands, hung fondly over it.  I
discerned he was now neither angry nor shocked at my audacity.  I saw
even that to be thus frankly addressed on a subject he had deemed
unapproachable--to hear it thus freely handled--was beginning to be felt
by him as a new pleasure--an unhoped-for relief.  Reserved people often
really need the frank discussion of their sentiments and griefs more than
the expansive.  The sternest-seeming stoic is human after all; and to
"burst" with boldness and good-will into "the silent sea" of their souls
is often to confer on them the first of obligations.

"She likes you, I am sure," said I, as I stood behind his chair, "and her
father respects you.  Moreover, she is a sweet girl--rather thoughtless;
but you would have sufficient thought for both yourself and her.  You
ought to marry her."

"\emph{Does} she like me?" he asked.

"Certainly; better than she likes any one else.  She talks of you
continually: there is no subject she enjoys so much or touches upon so
often."

"It is very pleasant to hear this," he said--"very: go on for another
quarter of an hour."  And he actually took out his watch and laid it upon
the table to measure the time.

"But where is the use of going on," I asked, "when you are probably
preparing some iron blow of contradiction, or forging a fresh chain to
fetter your heart?"

"Don't imagine such hard things.  Fancy me yielding and melting, as I am
doing: human love rising like a freshly opened fountain in my mind and
overflowing with sweet inundation all the field I have so carefully and
with such labour prepared--so assiduously sown with the seeds of good
intentions, of self-denying plans.  And now it is deluged with a
nectarous flood--the young germs swamped--delicious poison cankering
them: now I see myself stretched on an ottoman in the drawing-room at
Vale Hall at my bride Rosamond Oliver's feet: she is talking to me with
her sweet voice--gazing down on me with those eyes your skilful hand has
copied so well--smiling at me with these coral lips.  She is mine--I am
hers--this present life and passing world suffice to me.  Hush! say
nothing--my heart is full of delight--my senses are entranced--let the
time I marked pass in peace."

I humoured him: the watch ticked on: he breathed fast and low: I stood
silent.  Amidst this hush the quartet sped; he replaced the watch, laid
the picture down, rose, and stood on the hearth.

"Now," said he, "that little space was given to delirium and delusion.  I
rested my temples on the breast of temptation, and put my neck
voluntarily under her yoke of flowers.  I tasted her cup.  The pillow was
burning: there is an asp in the garland: the wine has a bitter taste: her
promises are hollow--her offers false: I see and know all this."

I gazed at him in wonder.

"It is strange," pursued he, "that while I love Rosamond Oliver so
wildly--with all the intensity, indeed, of a first passion, the object of
which is exquisitely beautiful, graceful, fascinating--I experience at
the same time a calm, unwarped consciousness that she would not make me a
good wife; that she is not the partner suited to me; that I should
discover this within a year after marriage; and that to twelve months'
rapture would succeed a lifetime of regret.  This I know."

"Strange indeed!" I could not help ejaculating.

"While something in me," he went on, "is acutely sensible to her charms,
something else is as deeply impressed with her defects: they are such
that she could sympathise in nothing I aspired to--co-operate in nothing
I undertook.  Rosamond a sufferer, a labourer, a female apostle?  Rosamond
a missionary's wife?  No!"

"But you need not be a missionary.  You might relinquish that scheme."

"Relinquish!  What! my vocation?  My great work?  My foundation laid on
earth for a mansion in heaven?  My hopes of being numbered in the band
who have merged all ambitions in the glorious one of bettering their
race--of carrying knowledge into the realms of ignorance--of substituting
peace for war--freedom for bondage--religion for superstition--the hope
of heaven for the fear of hell?  Must I relinquish that?  It is dearer
than the blood in my veins.  It is what I have to look forward to, and to
live for."

After a considerable pause, I said--"And Miss Oliver?  Are her
disappointment and sorrow of no interest to you?"

"Miss Oliver is ever surrounded by suitors and flatterers: in less than a
month, my image will be effaced from her heart.  She will forget me; and
will marry, probably, some one who will make her far happier than I
should do."

"You speak coolly enough; but you suffer in the conflict.  You are
wasting away."

"No.  If I get a little thin, it is with anxiety about my prospects, yet
unsettled--my departure, continually procrastinated.  Only this morning,
I received intelligence that the successor, whose arrival I have been so
long expecting, cannot be ready to replace me for three months to come
yet; and perhaps the three months may extend to six."

"You tremble and become flushed whenever Miss Oliver enters the
schoolroom."

Again the surprised expression crossed his face.  He had not imagined
that a woman would dare to speak so to a man.  For me, I felt at home in
this sort of discourse.  I could never rest in communication with strong,
discreet, and refined minds, whether male or female, till I had passed
the outworks of conventional reserve, and crossed the threshold of
confidence, and won a place by their heart's very hearthstone.

"You are original," said he, "and not timid.  There is something brave in
your spirit, as well as penetrating in your eye; but allow me to assure
you that you partially misinterpret my emotions.  You think them more
profound and potent than they are.  You give me a larger allowance of
sympathy than I have a just claim to.  When I colour, and when I shade
before Miss Oliver, I do not pity myself.  I scorn the weakness.  I know
it is ignoble: a mere fever of the flesh: not, I declare, the convulsion
of the soul.  \emph{That} is just as fixed as a rock, firm set in the depths
of a restless sea.  Know me to be what I am--a cold hard man."

I smiled incredulously.

"You have taken my confidence by storm," he continued, "and now it is
much at your service.  I am simply, in my original state--stripped of
that blood-bleached robe with which Christianity covers human deformity--a
cold, hard, ambitious man.  Natural affection only, of all the
sentiments, has permanent power over me.  Reason, and not feeling, is my
guide; my ambition is unlimited: my desire to rise higher, to do more
than others, insatiable.  I honour endurance, perseverance, industry,
talent; because these are the means by which men achieve great ends and
mount to lofty eminence.  I watch your career with interest, because I
consider you a specimen of a diligent, orderly, energetic woman: not
because I deeply compassionate what you have gone through, or what you
still suffer."

"You would describe yourself as a mere pagan philosopher," I said.

"No.  There is this difference between me and deistic philosophers: I
believe; and I believe the Gospel.  You missed your epithet.  I am not a
pagan, but a Christian philosopher--a follower of the sect of Jesus.  As
His disciple I adopt His pure, His merciful, His benignant doctrines.  I
advocate them: I am sworn to spread them.  Won in youth to religion, she
has cultivated my original qualities thus:--From the minute germ, natural
affection, she has developed the overshadowing tree, philanthropy.  From
the wild stringy root of human uprightness, she has reared a due sense of
the Divine justice.  Of the ambition to win power and renown for my
wretched self, she has formed the ambition to spread my Master's kingdom;
to achieve victories for the standard of the cross.  So much has religion
done for me; turning the original materials to the best account; pruning
and training nature.  But she could not eradicate nature: nor will it be
eradicated 'till this mortal shall put on immortality.'"

Having said this, he took his hat, which lay on the table beside my
palette.  Once more he looked at the portrait.

"She \emph{is} lovely," he murmured.  "She is well named the Rose of the
World, indeed!"

"And may I not paint one like it for you?"

"\emph{Cui bono}?  No."

He drew over the picture the sheet of thin paper on which I was
accustomed to rest my hand in painting, to prevent the cardboard from
being sullied.  What he suddenly saw on this blank paper, it was
impossible for me to tell; but something had caught his eye.  He took it
up with a snatch; he looked at the edge; then shot a glance at me,
inexpressibly peculiar, and quite incomprehensible: a glance that seemed
to take and make note of every point in my shape, face, and dress; for it
traversed all, quick, keen as lightning.  His lips parted, as if to
speak: but he checked the coming sentence, whatever it was.

"What is the matter?" I asked.

"Nothing in the world," was the reply; and, replacing the paper, I saw
him dexterously tear a narrow slip from the margin.  It disappeared in
his glove; and, with one hasty nod and "good-afternoon," he vanished.

"Well!" I exclaimed, using an expression of the district, "that caps the
globe, however!"

I, in my turn, scrutinised the paper; but saw nothing on it save a few
dingy stains of paint where I had tried the tint in my pencil.  I
pondered the mystery a minute or two; but finding it insolvable, and
being certain it could not be of much moment, I dismissed, and soon
forgot it.




\chapter{}


\lettrine{W}{hen} Mr. St. John went, it was beginning to snow; the whirling storm
continued all night.  The next day a keen wind brought fresh and blinding
falls; by twilight the valley was drifted up and almost impassable.  I
had closed my shutter, laid a mat to the door to prevent the snow from
blowing in under it, trimmed my fire, and after sitting nearly an hour on
the hearth listening to the muffled fury of the tempest, I lit a candle,
took down "Marmion," and beginning--

   "Day set on Norham's castled steep,
   And Tweed's fair river broad and deep,
      And Cheviot's mountains lone;
   The massive towers, the donjon keep,
   The flanking walls that round them sweep,
      In yellow lustre shone"--

I soon forgot storm in music.

I heard a noise: the wind, I thought, shook the door.  No; it was St.
John Rivers, who, lifting the latch, came in out of the frozen
hurricane--the howling darkness--and stood before me: the cloak that
covered his tall figure all white as a glacier.  I was almost in
consternation, so little had I expected any guest from the blocked-up
vale that night.

"Any ill news?" I demanded.  "Has anything happened?"

"No.  How very easily alarmed you are!" he answered, removing his cloak
and hanging it up against the door, towards which he again coolly pushed
the mat which his entrance had deranged.  He stamped the snow from his
boots.

"I shall sully the purity of your floor," said he, "but you must excuse
me for once."  Then he approached the fire.  "I have had hard work to get
here, I assure you," he observed, as he warmed his hands over the flame.
"One drift took me up to the waist; happily the snow is quite soft yet."

"But why are you come?" I could not forbear saying.

"Rather an inhospitable question to put to a visitor; but since you ask
it, I answer simply to have a little talk with you; I got tired of my
mute books and empty rooms.  Besides, since yesterday I have experienced
the excitement of a person to whom a tale has been half-told, and who is
impatient to hear the sequel."

He sat down.  I recalled his singular conduct of yesterday, and really I
began to fear his wits were touched.  If he were insane, however, his was
a very cool and collected insanity: I had never seen that
handsome-featured face of his look more like chiselled marble than it did
just now, as he put aside his snow-wet hair from his forehead and let the
firelight shine free on his pale brow and cheek as pale, where it grieved
me to discover the hollow trace of care or sorrow now so plainly graved.
I waited, expecting he would say something I could at least comprehend;
but his hand was now at his chin, his finger on his lip: he was thinking.
It struck me that his hand looked wasted like his face.  A perhaps
uncalled-for gush of pity came over my heart: I was moved to say--

"I wish Diana or Mary would come and live with you: it is too bad that
you should be quite alone; and you are recklessly rash about your own
health."

"Not at all," said he: "I care for myself when necessary.  I am well now.
What do you see amiss in me?"

This was said with a careless, abstracted indifference, which showed that
my solicitude was, at least in his opinion, wholly superfluous.  I was
silenced.

He still slowly moved his finger over his upper lip, and still his eye
dwelt dreamily on the glowing grate; thinking it urgent to say something,
I asked him presently if he felt any cold draught from the door, which
was behind him.

"No, no!" he responded shortly and somewhat testily.

"Well," I reflected, "if you won't talk, you may be still; I'll let you
alone now, and return to my book."

So I snuffed the candle and resumed the perusal of "Marmion."  He soon
stirred; my eye was instantly drawn to his movements; he only took out a
morocco pocket-book, thence produced a letter, which he read in silence,
folded it, put it back, relapsed into meditation.  It was vain to try to
read with such an inscrutable fixture before me; nor could I, in
impatience, consent to be dumb; he might rebuff me if he liked, but talk
I would.

"Have you heard from Diana and Mary lately?"

"Not since the letter I showed you a week ago."

"There has not been any change made about your own arrangements?  You
will not be summoned to leave England sooner than you expected?"

"I fear not, indeed: such chance is too good to befall me."  Baffled so
far, I changed my ground.  I bethought myself to talk about the school
and my scholars.

"Mary Garrett's mother is better, and Mary came back to the school this
morning, and I shall have four new girls next week from the Foundry
Close--they would have come to-day but for the snow."

"Indeed!"

"Mr. Oliver pays for two."

"Does he?"

"He means to give the whole school a treat at Christmas."

"I know."

"Was it your suggestion?"

"No."

"Whose, then?"

"His daughter's, I think."

"It is like her: she is so good-natured."

"Yes."

Again came the blank of a pause: the clock struck eight strokes.  It
aroused him; he uncrossed his legs, sat erect, turned to me.

"Leave your book a moment, and come a little nearer the fire," he said.

Wondering, and of my wonder finding no end, I complied.

"Half-an-hour ago," he pursued, "I spoke of my impatience to hear the
sequel of a tale: on reflection, I find the matter will be better managed
by my assuming the narrator's part, and converting you into a listener.
Before commencing, it is but fair to warn you that the story will sound
somewhat hackneyed in your ears; but stale details often regain a degree
of freshness when they pass through new lips.  For the rest, whether
trite or novel, it is short.

"Twenty years ago, a poor curate--never mind his name at this moment--fell
in love with a rich man's daughter; she fell in love with him, and
married him, against the advice of all her friends, who consequently
disowned her immediately after the wedding.  Before two years passed, the
rash pair were both dead, and laid quietly side by side under one slab.
(I have seen their grave; it formed part of the pavement of a huge
churchyard surrounding the grim, soot-black old cathedral of an overgrown
manufacturing town in ---shire.)  They left a daughter, which, at its
very birth, Charity received in her lap--cold as that of the snow-drift I
almost stuck fast in to-night.  Charity carried the friendless thing to
the house of its rich maternal relations; it was reared by an aunt-in-
law, called (I come to names now) Mrs. Reed of Gateshead.  You start--did
you hear a noise?  I daresay it is only a rat scrambling along the
rafters of the adjoining schoolroom: it was a barn before I had it
repaired and altered, and barns are generally haunted by rats.--To
proceed.  Mrs. Reed kept the orphan ten years: whether it was happy or
not with her, I cannot say, never having been told; but at the end of
that time she transferred it to a place you know--being no other than
Lowood School, where you so long resided yourself.  It seems her career
there was very honourable: from a pupil, she became a teacher, like
yourself--really it strikes me there are parallel points in her history
and yours--she left it to be a governess: there, again, your fates were
analogous; she undertook the education of the ward of a certain Mr.
Rochester."

"Mr. Rivers!" I interrupted.

"I can guess your feelings," he said, "but restrain them for a while: I
have nearly finished; hear me to the end.  Of Mr. Rochester's character I
know nothing, but the one fact that he professed to offer honourable
marriage to this young girl, and that at the very altar she discovered he
had a wife yet alive, though a lunatic.  What his subsequent conduct and
proposals were is a matter of pure conjecture; but when an event
transpired which rendered inquiry after the governess necessary, it was
discovered she was gone--no one could tell when, where, or how.  She had
left Thornfield Hall in the night; every research after her course had
been vain: the country had been scoured far and wide; no vestige of
information could be gathered respecting her.  Yet that she should be
found is become a matter of serious urgency: advertisements have been put
in all the papers; I myself have received a letter from one Mr. Briggs, a
solicitor, communicating the details I have just imparted.  Is it not an
odd tale?"

"Just tell me this," said I, "and since you know so much, you surely can
tell it me--what of Mr. Rochester?  How and where is he?  What is he
doing?  Is he well?"

"I am ignorant of all concerning Mr. Rochester: the letter never mentions
him but to narrate the fraudulent and illegal attempt I have adverted to.
You should rather ask the name of the governess--the nature of the event
which requires her appearance."

"Did no one go to Thornfield Hall, then?  Did no one see Mr. Rochester?"

"I suppose not."

"But they wrote to him?"

"Of course."

"And what did he say?  Who has his letters?"

"Mr. Briggs intimates that the answer to his application was not from Mr.
Rochester, but from a lady: it is signed 'Alice Fairfax.'"

I felt cold and dismayed: my worst fears then were probably true: he had
in all probability left England and rushed in reckless desperation to
some former haunt on the Continent.  And what opiate for his severe
sufferings--what object for his strong passions--had he sought there?  I
dared not answer the question.  Oh, my poor master--once almost my
husband--whom I had often called "my dear Edward!"

"He must have been a bad man," observed Mr. Rivers.

"You don't know him--don't pronounce an opinion upon him," I said, with
warmth.

"Very well," he answered quietly: "and indeed my head is otherwise
occupied than with him: I have my tale to finish.  Since you won't ask
the governess's name, I must tell it of my own accord.  Stay!  I have it
here--it is always more satisfactory to see important points written
down, fairly committed to black and white."

And the pocket-book was again deliberately produced, opened, sought
through; from one of its compartments was extracted a shabby slip of
paper, hastily torn off: I recognised in its texture and its stains of
ultra-marine, and lake, and vermillion, the ravished margin of the
portrait-cover.  He got up, held it close to my eyes: and I read, traced
in Indian ink, in my own handwriting, the words "JANE EYRE"--the work
doubtless of some moment of abstraction.

"Briggs wrote to me of a Jane Eyre:" he said, "the advertisements
demanded a Jane Eyre: I knew a Jane Elliott.--I confess I had my
suspicions, but it was only yesterday afternoon they were at once
resolved into certainty.  You own the name and renounce the \emph{alias}?"

"Yes--yes; but where is Mr. Briggs?  He perhaps knows more of Mr.
Rochester than you do."

"Briggs is in London.  I should doubt his knowing anything at all about
Mr. Rochester; it is not in Mr. Rochester he is interested.  Meantime,
you forget essential points in pursuing trifles: you do not inquire why
Mr. Briggs sought after you--what he wanted with you."

"Well, what did he want?"

"Merely to tell you that your uncle, Mr. Eyre of Madeira, is dead; that
he has left you all his property, and that you are now rich--merely
that--nothing more."

"I!--rich?"

"Yes, you, rich--quite an heiress."

Silence succeeded.

"You must prove your identity of course," resumed St. John presently: "a
step which will offer no difficulties; you can then enter on immediate
possession.  Your fortune is vested in the English funds; Briggs has the
will and the necessary documents."

Here was a new card turned up!  It is a fine thing, reader, to be lifted
in a moment from indigence to wealth--a very fine thing; but not a matter
one can comprehend, or consequently enjoy, all at once.  And then there
are other chances in life far more thrilling and rapture-giving: \emph{this}
is solid, an affair of the actual world, nothing ideal about it: all its
associations are solid and sober, and its manifestations are the same.
One does not jump, and spring, and shout hurrah! at hearing one has got a
fortune; one begins to consider responsibilities, and to ponder business;
on a base of steady satisfaction rise certain grave cares, and we contain
ourselves, and brood over our bliss with a solemn brow.

Besides, the words Legacy, Bequest, go side by side with the words,
Death, Funeral.  My uncle I had heard was dead--my only relative; ever
since being made aware of his existence, I had cherished the hope of one
day seeing him: now, I never should.  And then this money came only to
me: not to me and a rejoicing family, but to my isolated self.  It was a
grand boon doubtless; and independence would be glorious--yes, I felt
that--that thought swelled my heart.

"You unbend your forehead at last," said Mr. Rivers.  "I thought Medusa
had looked at you, and that you were turning to stone.  Perhaps now you
will ask how much you are worth?"

"How much am I worth?"

"Oh, a trifle!  Nothing of course to speak of--twenty thousand pounds, I
think they say--but what is that?"

"Twenty thousand pounds?"

Here was a new stunner--I had been calculating on four or five thousand.
This news actually took my breath for a moment: Mr. St. John, whom I had
never heard laugh before, laughed now.

"Well," said he, "if you had committed a murder, and I had told you your
crime was discovered, you could scarcely look more aghast."

"It is a large sum--don't you think there is a mistake?"

"No mistake at all."

"Perhaps you have read the figures wrong--it may be two thousand!"

"It is written in letters, not figures,--twenty thousand."

I again felt rather like an individual of but average gastronomical
powers sitting down to feast alone at a table spread with provisions for
a hundred.  Mr. Rivers rose now and put his cloak on.

"If it were not such a very wild night," he said, "I would send Hannah
down to keep you company: you look too desperately miserable to be left
alone.  But Hannah, poor woman! could not stride the drifts so well as I:
her legs are not quite so long: so I must e'en leave you to your sorrows.
Good-night."

He was lifting the latch: a sudden thought occurred to me.  "Stop one
minute!" I cried.

"Well?"

"It puzzles me to know why Mr. Briggs wrote to you about me; or how he
knew you, or could fancy that you, living in such an out-of-the-way
place, had the power to aid in my discovery."

"Oh!  I am a clergyman," he said; "and the clergy are often appealed to
about odd matters."  Again the latch rattled.

"No; that does not satisfy me!" I exclaimed: and indeed there was
something in the hasty and unexplanatory reply which, instead of
allaying, piqued my curiosity more than ever.

"It is a very strange piece of business," I added; "I must know more
about it."

"Another time."

"No; to-night!--to-night!" and as he turned from the door, I placed
myself between it and him.  He looked rather embarrassed.

"You certainly shall not go till you have told me all," I said.

"I would rather not just now."

"You shall!--you must!"

"I would rather Diana or Mary informed you."

Of course these objections wrought my eagerness to a climax: gratified it
must be, and that without delay; and I told him so.

"But I apprised you that I was a hard man," said he, "difficult to
persuade."

"And I am a hard woman,--impossible to put off."

{And I am a hard woman,--impossible to put off: p369.jpg}

"And then," he pursued, "I am cold: no fervour infects me."

"Whereas I am hot, and fire dissolves ice.  The blaze there has thawed
all the snow from your cloak; by the same token, it has streamed on to my
floor, and made it like a trampled street.  As you hope ever to be
forgiven, Mr. Rivers, the high crime and misdemeanour of spoiling a
sanded kitchen, tell me what I wish to know."

"Well, then," he said, "I yield; if not to your earnestness, to your
perseverance: as stone is worn by continual dropping.  Besides, you must
know some day,--as well now as later.  Your name is Jane Eyre?"

"Of course: that was all settled before."

"You are not, perhaps, aware that I am your namesake?--that I was
christened St. John Eyre Rivers?"

"No, indeed!  I remember now seeing the letter E. comprised in your
initials written in books you have at different times lent me; but I
never asked for what name it stood.  But what then?  Surely--"

I stopped: I could not trust myself to entertain, much less to express,
the thought that rushed upon me--that embodied itself,--that, in a
second, stood out a strong, solid probability.  Circumstances knit
themselves, fitted themselves, shot into order: the chain that had been
lying hitherto a formless lump of links was drawn out straight,--every
ring was perfect, the connection complete.  I knew, by instinct, how the
matter stood, before St. John had said another word; but I cannot expect
the reader to have the same intuitive perception, so I must repeat his
explanation.

"My mother's name was Eyre; she had two brothers; one a clergyman, who
married Miss Jane Reed, of Gateshead; the other, John Eyre, Esq.,
merchant, late of Funchal, Madeira.  Mr. Briggs, being Mr. Eyre's
solicitor, wrote to us last August to inform us of our uncle's death, and
to say that he had left his property to his brother the clergyman's
orphan daughter, overlooking us, in consequence of a quarrel, never
forgiven, between him and my father.  He wrote again a few weeks since,
to intimate that the heiress was lost, and asking if we knew anything of
her.  A name casually written on a slip of paper has enabled me to find
her out.  You know the rest."  Again he was going, but I set my back
against the door.

"Do let me speak," I said; "let me have one moment to draw breath and
reflect."  I paused--he stood before me, hat in hand, looking composed
enough.  I resumed--

"Your mother was my father's sister?"

"Yes."

"My aunt, consequently?"

He bowed.

"My uncle John was your uncle John?  You, Diana, and Mary are his
sister's children, as I am his brother's child?"

"Undeniably."

"You three, then, are my cousins; half our blood on each side flows from
the same source?"

"We are cousins; yes."

I surveyed him.  It seemed I had found a brother: one I could be proud
of,--one I could love; and two sisters, whose qualities were such, that,
when I knew them but as mere strangers, they had inspired me with genuine
affection and admiration.  The two girls, on whom, kneeling down on the
wet ground, and looking through the low, latticed window of Moor House
kitchen, I had gazed with so bitter a mixture of interest and despair,
were my near kinswomen; and the young and stately gentleman who had found
me almost dying at his threshold was my blood relation.  Glorious
discovery to a lonely wretch!  This was wealth indeed!--wealth to the
heart!--a mine of pure, genial affections.  This was a blessing, bright,
vivid, and exhilarating;--not like the ponderous gift of gold: rich and
welcome enough in its way, but sobering from its weight.  I now clapped
my hands in sudden joy--my pulse bounded, my veins thrilled.

"Oh, I am glad!--I am glad!" I exclaimed.

St. John smiled.  "Did I not say you neglected essential points to pursue
trifles?" he asked.  "You were serious when I told you you had got a
fortune; and now, for a matter of no moment, you are excited."

"What can you mean?  It may be of no moment to you; you have sisters and
don't care for a cousin; but I had nobody; and now three relations,--or
two, if you don't choose to be counted,--are born into my world
full-grown.  I say again, I am glad!"

I walked fast through the room: I stopped, half suffocated with the
thoughts that rose faster than I could receive, comprehend, settle
them:--thoughts of what might, could, would, and should be, and that ere
long.  I looked at the blank wall: it seemed a sky thick with ascending
stars,--every one lit me to a purpose or delight.  Those who had saved my
life, whom, till this hour, I had loved barrenly, I could now benefit.
They were under a yoke,--I could free them: they were scattered,--I could
reunite them: the independence, the affluence which was mine, might be
theirs too.  Were we not four?  Twenty thousand pounds shared equally
would be five thousand each, justice--enough and to spare: justice would
be done,--mutual happiness secured.  Now the wealth did not weigh on me:
now it was not a mere bequest of coin,--it was a legacy of life, hope,
enjoyment.

How I looked while these ideas were taking my spirit by storm, I cannot
tell; but I perceived soon that Mr. Rivers had placed a chair behind me,
and was gently attempting to make me sit down on it.  He also advised me
to be composed; I scorned the insinuation of helplessness and
distraction, shook off his hand, and began to walk about again.

"Write to Diana and Mary to-morrow," I said, "and tell them to come home
directly.  Diana said they would both consider themselves rich with a
thousand pounds, so with five thousand they will do very well."

"Tell me where I can get you a glass of water," said St. John; "you must
really make an effort to tranquillise your feelings."

"Nonsense! and what sort of an effect will the bequest have on you?  Will
it keep you in England, induce you to marry Miss Oliver, and settle down
like an ordinary mortal?"

"You wander: your head becomes confused.  I have been too abrupt in
communicating the news; it has excited you beyond your strength."

"Mr. Rivers! you quite put me out of patience: I am rational enough; it
is you who misunderstand, or rather who affect to misunderstand."

"Perhaps, if you explained yourself a little more fully, I should
comprehend better."

"Explain!  What is there to explain?  You cannot fail to see that twenty
thousand pounds, the sum in question, divided equally between the nephew
and three nieces of our uncle, will give five thousand to each?  What I
want is, that you should write to your sisters and tell them of the
fortune that has accrued to them."

"To you, you mean."

"I have intimated my view of the case: I am incapable of taking any
other.  I am not brutally selfish, blindly unjust, or fiendishly
ungrateful.  Besides, I am resolved I will have a home and connections.  I
like Moor House, and I will live at Moor House; I like Diana and Mary,
and I will attach myself for life to Diana and Mary.  It would please and
benefit me to have five thousand pounds; it would torment and oppress me
to have twenty thousand; which, moreover, could never be mine in justice,
though it might in law.  I abandon to you, then, what is absolutely
superfluous to me.  Let there be no opposition, and no discussion about
it; let us agree amongst each other, and decide the point at once."

"This is acting on first impulses; you must take days to consider such a
matter, ere your word can be regarded as valid."

"Oh! if all you doubt is my sincerity, I am easy: you see the justice of
the case?"

"I \emph{do} see a certain justice; but it is contrary to all custom.  Besides,
the entire fortune is your right: my uncle gained it by his own efforts;
he was free to leave it to whom he would: he left it to you.  After all,
justice permits you to keep it: you may, with a clear conscience,
consider it absolutely your own."

"With me," said I, "it is fully as much a matter of feeling as of
conscience: I must indulge my feelings; I so seldom have had an
opportunity of doing so.  Were you to argue, object, and annoy me for a
year, I could not forego the delicious pleasure of which I have caught a
glimpse--that of repaying, in part, a mighty obligation, and winning to
myself lifelong friends."

"You think so now," rejoined St. John, "because you do not know what it
is to possess, nor consequently to enjoy wealth: you cannot form a notion
of the importance twenty thousand pounds would give you; of the place it
would enable you to take in society; of the prospects it would open to
you: you cannot--"

"And you," I interrupted, "cannot at all imagine the craving I have for
fraternal and sisterly love.  I never had a home, I never had brothers or
sisters; I must and will have them now: you are not reluctant to admit me
and own me, are you?"

"Jane, I will be your brother--my sisters will be your sisters--without
stipulating for this sacrifice of your just rights."

"Brother?  Yes; at the distance of a thousand leagues!  Sisters?  Yes;
slaving amongst strangers!  I, wealthy--gorged with gold I never earned
and do not merit!  You, penniless!  Famous equality and fraternisation!
Close union!  Intimate attachment!"

"But, Jane, your aspirations after family ties and domestic happiness may
be realised otherwise than by the means you contemplate: you may marry."

"Nonsense, again!  Marry!  I don't want to marry, and never shall marry."

"That is saying too much: such hazardous affirmations are a proof of the
excitement under which you labour."

"It is not saying too much: I know what I feel, and how averse are my
inclinations to the bare thought of marriage.  No one would take me for
love; and I will not be regarded in the light of a mere money
speculation.  And I do not want a stranger--unsympathising, alien,
different from me; I want my kindred: those with whom I have full fellow-
feeling.  Say again you will be my brother: when you uttered the words I
was satisfied, happy; repeat them, if you can, repeat them sincerely."

"I think I can.  I know I have always loved my own sisters; and I know on
what my affection for them is grounded,--respect for their worth and
admiration of their talents.  You too have principle and mind: your
tastes and habits resemble Diana's and Mary's; your presence is always
agreeable to me; in your conversation I have already for some time found
a salutary solace.  I feel I can easily and naturally make room in my
heart for you, as my third and youngest sister."

"Thank you: that contents me for to-night.  Now you had better go; for if
you stay longer, you will perhaps irritate me afresh by some mistrustful
scruple."

"And the school, Miss Eyre?  It must now be shut up, I suppose?"

"No.  I will retain my post of mistress till you get a substitute."

He smiled approbation: we shook hands, and he took leave.

I need not narrate in detail the further struggles I had, and arguments I
used, to get matters regarding the legacy settled as I wished.  My task
was a very hard one; but, as I was absolutely resolved--as my cousins saw
at length that my mind was really and immutably fixed on making a just
division of the property--as they must in their own hearts have felt the
equity of the intention; and must, besides, have been innately conscious
that in my place they would have done precisely what I wished to do--they
yielded at length so far as to consent to put the affair to arbitration.
The judges chosen were Mr. Oliver and an able lawyer: both coincided in
my opinion: I carried my point.  The instruments of transfer were drawn
out: St. John, Diana, Mary, and I, each became possessed of a competency.




\chapter{}


\lettrine{I}{t} was near Christmas by the time all was settled: the season of general
holiday approached.  I now closed Morton school, taking care that the
parting should not be barren on my side.  Good fortune opens the hand as
well as the heart wonderfully; and to give somewhat when we have largely
received, is but to afford a vent to the unusual ebullition of the
sensations.  I had long felt with pleasure that many of my rustic
scholars liked me, and when we parted, that consciousness was confirmed:
they manifested their affection plainly and strongly.  Deep was my
gratification to find I had really a place in their unsophisticated
hearts: I promised them that never a week should pass in future that I
did not visit them, and give them an hour's teaching in their school.

Mr. Rivers came up as, having seen the classes, now numbering sixty
girls, file out before me, and locked the door, I stood with the key in
my hand, exchanging a few words of special farewell with some half-dozen
of my best scholars: as decent, respectable, modest, and well-informed
young women as could be found in the ranks of the British peasantry.  And
that is saying a great deal; for after all, the British peasantry are the
best taught, best mannered, most self-respecting of any in Europe: since
those days I have seen paysannes and Bauerinnen; and the best of them
seemed to me ignorant, coarse, and besotted, compared with my Morton
girls.

"Do you consider you have got your reward for a season of exertion?"
asked Mr. Rivers, when they were gone.  "Does not the consciousness of
having done some real good in your day and generation give pleasure?"

"Doubtless."

"And you have only toiled a few months!  Would not a life devoted to the
task of regenerating your race be well spent?"

"Yes," I said; "but I could not go on for ever so: I want to enjoy my own
faculties as well as to cultivate those of other people.  I must enjoy
them now; don't recall either my mind or body to the school; I am out of
it and disposed for full holiday."

He looked grave.  "What now?  What sudden eagerness is this you evince?
What are you going to do?"

"To be active: as active as I can.  And first I must beg you to set
Hannah at liberty, and get somebody else to wait on you."

"Do you want her?"

"Yes, to go with me to Moor House.  Diana and Mary will be at home in a
week, and I want to have everything in order against their arrival."

"I understand.  I thought you were for flying off on some excursion.  It
is better so: Hannah shall go with you."

"Tell her to be ready by to-morrow then; and here is the schoolroom key:
I will give you the key of my cottage in the morning."

He took it.  "You give it up very gleefully," said he; "I don't quite
understand your light-heartedness, because I cannot tell what employment
you propose to yourself as a substitute for the one you are
relinquishing.  What aim, what purpose, what ambition in life have you
now?"

"My first aim will be to \emph{clean down} (do you comprehend the full force
of the expression?)--to \emph{clean down} Moor House from chamber to cellar;
my next to rub it up with bees-wax, oil, and an indefinite number of
cloths, till it glitters again; my third, to arrange every chair, table,
bed, carpet, with mathematical precision; afterwards I shall go near to
ruin you in coals and peat to keep up good fires in every room; and
lastly, the two days preceding that on which your sisters are expected
will be devoted by Hannah and me to such a beating of eggs, sorting of
currants, grating of spices, compounding of Christmas cakes, chopping up
of materials for mince-pies, and solemnising of other culinary rites, as
words can convey but an inadequate notion of to the uninitiated like you.
My purpose, in short, is to have all things in an absolutely perfect
state of readiness for Diana and Mary before next Thursday; and my
ambition is to give them a beau-ideal of a welcome when they come."

St. John smiled slightly: still he was dissatisfied.

"It is all very well for the present," said he; "but seriously, I trust
that when the first flush of vivacity is over, you will look a little
higher than domestic endearments and household joys."

"The best things the world has!" I interrupted.

"No, Jane, no: this world is not the scene of fruition; do not attempt to
make it so: nor of rest; do not turn slothful."

"I mean, on the contrary, to be busy."

"Jane, I excuse you for the present: two months' grace I allow you for
the full enjoyment of your new position, and for pleasing yourself with
this late-found charm of relationship; but \emph{then}, I hope you will begin
to look beyond Moor House and Morton, and sisterly society, and the
selfish calm and sensual comfort of civilised affluence.  I hope your
energies will then once more trouble you with their strength."

I looked at him with surprise.  "St. John," I said, "I think you are
almost wicked to talk so.  I am disposed to be as content as a queen, and
you try to stir me up to restlessness!  To what end?"

"To the end of turning to profit the talents which God has committed to
your keeping; and of which He will surely one day demand a strict
account.  Jane, I shall watch you closely and anxiously--I warn you of
that.  And try to restrain the disproportionate fervour with which you
throw yourself into commonplace home pleasures.  Don't cling so
tenaciously to ties of the flesh; save your constancy and ardour for an
adequate cause; forbear to waste them on trite transient objects.  Do you
hear, Jane?"

"Yes; just as if you were speaking Greek.  I feel I have adequate cause
to be happy, and I \emph{will} be happy.  Goodbye!"

Happy at Moor House I was, and hard I worked; and so did Hannah: she was
charmed to see how jovial I could be amidst the bustle of a house turned
topsy-turvy--how I could brush, and dust, and clean, and cook.  And
really, after a day or two of confusion worse confounded, it was
delightful by degrees to invoke order from the chaos ourselves had made.
I had previously taken a journey to S--- to purchase some new furniture:
my cousins having given me \emph{carte blanche} to effect what alterations I
pleased, and a sum having been set aside for that purpose.  The ordinary
sitting-room and bedrooms I left much as they were: for I knew Diana and
Mary would derive more pleasure from seeing again the old homely tables,
and chairs, and beds, than from the spectacle of the smartest
innovations.  Still some novelty was necessary, to give to their return
the piquancy with which I wished it to be invested.  Dark handsome new
carpets and curtains, an arrangement of some carefully selected antique
ornaments in porcelain and bronze, new coverings, and mirrors, and
dressing-cases, for the toilet tables, answered the end: they looked
fresh without being glaring.  A spare parlour and bedroom I refurnished
entirely, with old mahogany and crimson upholstery: I laid canvas on the
passage, and carpets on the stairs.  When all was finished, I thought
Moor House as complete a model of bright modest snugness within, as it
was, at this season, a specimen of wintry waste and desert dreariness
without.

The eventful Thursday at length came.  They were expected about dark, and
ere dusk fires were lit upstairs and below; the kitchen was in perfect
trim; Hannah and I were dressed, and all was in readiness.

St. John arrived first.  I had entreated him to keep quite clear of the
house till everything was arranged: and, indeed, the bare idea of the
commotion, at once sordid and trivial, going on within its walls sufficed
to scare him to estrangement.  He found me in the kitchen, watching the
progress of certain cakes for tea, then baking.  Approaching the hearth,
he asked, "If I was at last satisfied with housemaid's work?"  I answered
by inviting him to accompany me on a general inspection of the result of
my labours.  With some difficulty, I got him to make the tour of the
house.  He just looked in at the doors I opened; and when he had wandered
upstairs and downstairs, he said I must have gone through a great deal of
fatigue and trouble to have effected such considerable changes in so
short a time: but not a syllable did he utter indicating pleasure in the
improved aspect of his abode.

This silence damped me.  I thought perhaps the alterations had disturbed
some old associations he valued.  I inquired whether this was the case:
no doubt in a somewhat crest-fallen tone.

"Not at all; he had, on the contrary, remarked that I had scrupulously
respected every association: he feared, indeed, I must have bestowed more
thought on the matter than it was worth.  How many minutes, for instance,
had I devoted to studying the arrangement of this very room?--By-the-bye,
could I tell him where such a book was?"

I showed him the volume on the shelf: he took it down, and withdrawing to
his accustomed window recess, he began to read it.

Now, I did not like this, reader.  St. John was a good man; but I began
to feel he had spoken truth of himself when he said he was hard and cold.
The humanities and amenities of life had no attraction for him--its
peaceful enjoyments no charm.  Literally, he lived only to aspire--after
what was good and great, certainly; but still he would never rest, nor
approve of others resting round him.  As I looked at his lofty forehead,
still and pale as a white stone--at his fine lineaments fixed in study--I
comprehended all at once that he would hardly make a good husband: that
it would be a trying thing to be his wife.  I understood, as by
inspiration, the nature of his love for Miss Oliver; I agreed with him
that it was but a love of the senses.  I comprehended how he should
despise himself for the feverish influence it exercised over him; how he
should wish to stifle and destroy it; how he should mistrust its ever
conducting permanently to his happiness or hers.  I saw he was of the
material from which nature hews her heroes--Christian and Pagan--her
lawgivers, her statesmen, her conquerors: a steadfast bulwark for great
interests to rest upon; but, at the fireside, too often a cold cumbrous
column, gloomy and out of place.

"This parlour is not his sphere," I reflected: "the Himalayan ridge or
Caffre bush, even the plague-cursed Guinea Coast swamp would suit him
better.  Well may he eschew the calm of domestic life; it is not his
element: there his faculties stagnate--they cannot develop or appear to
advantage.  It is in scenes of strife and danger--where courage is
proved, and energy exercised, and fortitude tasked--that he will speak
and move, the leader and superior.  A merry child would have the
advantage of him on this hearth.  He is right to choose a missionary's
career--I see it now."

"They are coming! they are coming!" cried Hannah, throwing open the
parlour door.  At the same moment old Carlo barked joyfully.  Out I ran.
It was now dark; but a rumbling of wheels was audible.  Hannah soon had a
lantern lit.  The vehicle had stopped at the wicket; the driver opened
the door: first one well-known form, then another, stepped out.  In a
minute I had my face under their bonnets, in contact first with Mary's
soft cheek, then with Diana's flowing curls.  They laughed--kissed
me--then Hannah: patted Carlo, who was half wild with delight; asked
eagerly if all was well; and being assured in the affirmative, hastened
into the house.

They were stiff with their long and jolting drive from Whitcross, and
chilled with the frosty night air; but their pleasant countenances
expanded to the cheerful firelight.  While the driver and Hannah brought
in the boxes, they demanded St. John.  At this moment he advanced from
the parlour.  They both threw their arms round his neck at once.  He gave
each one quiet kiss, said in a low tone a few words of welcome, stood a
while to be talked to, and then, intimating that he supposed they would
soon rejoin him in the parlour, withdrew there as to a place of refuge.

I had lit their candles to go upstairs, but Diana had first to give
hospitable orders respecting the driver; this done, both followed me.
They were delighted with the renovation and decorations of their rooms;
with the new drapery, and fresh carpets, and rich tinted china vases:
they expressed their gratification ungrudgingly.  I had the pleasure of
feeling that my arrangements met their wishes exactly, and that what I
had done added a vivid charm to their joyous return home.

Sweet was that evening.  My cousins, full of exhilaration, were so
eloquent in narrative and comment, that their fluency covered St. John's
taciturnity: he was sincerely glad to see his sisters; but in their glow
of fervour and flow of joy he could not sympathise.  The event of the
day--that is, the return of Diana and Mary--pleased him; but the
accompaniments of that event, the glad tumult, the garrulous glee of
reception irked him: I saw he wished the calmer morrow was come.  In the
very meridian of the night's enjoyment, about an hour after tea, a rap
was heard at the door.  Hannah entered with the intimation that "a poor
lad was come, at that unlikely time, to fetch Mr. Rivers to see his
mother, who was drawing away."

"Where does she live, Hannah?"

"Clear up at Whitcross Brow, almost four miles off, and moor and moss all
the way."

"Tell him I will go."

"I'm sure, sir, you had better not.  It's the worst road to travel after
dark that can be: there's no track at all over the bog.  And then it is
such a bitter night--the keenest wind you ever felt.  You had better send
word, sir, that you will be there in the morning."

But he was already in the passage, putting on his cloak; and without one
objection, one murmur, he departed.  It was then nine o'clock: he did not
return till midnight.  Starved and tired enough he was: but he looked
happier than when he set out.  He had performed an act of duty; made an
exertion; felt his own strength to do and deny, and was on better terms
with himself.

I am afraid the whole of the ensuing week tried his patience.  It was
Christmas week: we took to no settled employment, but spent it in a sort
of merry domestic dissipation.  The air of the moors, the freedom of
home, the dawn of prosperity, acted on Diana and Mary's spirits like some
life-giving elixir: they were gay from morning till noon, and from noon
till night.  They could always talk; and their discourse, witty, pithy,
original, had such charms for me, that I preferred listening to, and
sharing in it, to doing anything else.  St. John did not rebuke our
vivacity; but he escaped from it: he was seldom in the house; his parish
was large, the population scattered, and he found daily business in
visiting the sick and poor in its different districts.

One morning at breakfast, Diana, after looking a little pensive for some
minutes, asked him, "If his plans were yet unchanged."

"Unchanged and unchangeable," was the reply.  And he proceeded to inform
us that his departure from England was now definitively fixed for the
ensuing year.

"And Rosamond Oliver?" suggested Mary, the words seeming to escape her
lips involuntarily: for no sooner had she uttered them, than she made a
gesture as if wishing to recall them.  St. John had a book in his hand--it
was his unsocial custom to read at meals--he closed it, and looked up.

"Rosamond Oliver," said he, "is about to be married to Mr. Granby, one of
the best connected and most estimable residents in S-, grandson and heir
to Sir Frederic Granby: I had the intelligence from her father
yesterday."

His sisters looked at each other and at me; we all three looked at him:
he was serene as glass.

"The match must have been got up hastily," said Diana: "they cannot have
known each other long."

"But two months: they met in October at the county ball at S-.  But where
there are no obstacles to a union, as in the present case, where the
connection is in every point desirable, delays are unnecessary: they will
be married as soon as S--- Place, which Sir Frederic gives up to them,
can he refitted for their reception."

The first time I found St. John alone after this communication, I felt
tempted to inquire if the event distressed him: but he seemed so little
to need sympathy, that, so far from venturing to offer him more, I
experienced some shame at the recollection of what I had already
hazarded.  Besides, I was out of practice in talking to him: his reserve
was again frozen over, and my frankness was congealed beneath it.  He had
not kept his promise of treating me like his sisters; he continually made
little chilling differences between us, which did not at all tend to the
development of cordiality: in short, now that I was acknowledged his
kinswoman, and lived under the same roof with him, I felt the distance
between us to be far greater than when he had known me only as the
village schoolmistress.  When I remembered how far I had once been
admitted to his confidence, I could hardly comprehend his present
frigidity.

Such being the case, I felt not a little surprised when he raised his
head suddenly from the desk over which he was stooping, and said--

"You see, Jane, the battle is fought and the victory won."

Startled at being thus addressed, I did not immediately reply: after a
moment's hesitation I answered--

"But are you sure you are not in the position of those conquerors whose
triumphs have cost them too dear?  Would not such another ruin you?"

"I think not; and if I were, it does not much signify; I shall never be
called upon to contend for such another.  The event of the conflict is
decisive: my way is now clear; I thank God for it!"  So saying, he
returned to his papers and his silence.

As our mutual happiness (\emph{i.e.}, Diana's, Mary's, and mine) settled into
a quieter character, and we resumed our usual habits and regular studies,
St. John stayed more at home: he sat with us in the same room, sometimes
for hours together.  While Mary drew, Diana pursued a course of
encyclopaedic reading she had (to my awe and amazement) undertaken, and I
fagged away at German, he pondered a mystic lore of his own: that of some
Eastern tongue, the acquisition of which he thought necessary to his
plans.

Thus engaged, he appeared, sitting in his own recess, quiet and absorbed
enough; but that blue eye of his had a habit of leaving the outlandish-
looking grammar, and wandering over, and sometimes fixing upon us, his
fellow-students, with a curious intensity of observation: if caught, it
would be instantly withdrawn; yet ever and anon, it returned searchingly
to our table.  I wondered what it meant: I wondered, too, at the punctual
satisfaction he never failed to exhibit on an occasion that seemed to me
of small moment, namely, my weekly visit to Morton school; and still more
was I puzzled when, if the day was unfavourable, if there was snow, or
rain, or high wind, and his sisters urged me not to go, he would
invariably make light of their solicitude, and encourage me to accomplish
the task without regard to the elements.

"Jane is not such a weakling as you would make her," he would say: "she
can bear a mountain blast, or a shower, or a few flakes of snow, as well
as any of us.  Her constitution is both sound and elastic;--better
calculated to endure variations of climate than many more robust."

And when I returned, sometimes a good deal tired, and not a little
weather-beaten, I never dared complain, because I saw that to murmur
would be to vex him: on all occasions fortitude pleased him; the reverse
was a special annoyance.

One afternoon, however, I got leave to stay at home, because I really had
a cold.  His sisters were gone to Morton in my stead: I sat reading
Schiller; he, deciphering his crabbed Oriental scrolls.  As I exchanged a
translation for an exercise, I happened to look his way: there I found
myself under the influence of the ever-watchful blue eye.  How long it
had been searching me through and through, and over and over, I cannot
tell: so keen was it, and yet so cold, I felt for the moment
superstitious--as if I were sitting in the room with something uncanny.

"Jane, what are you doing?"

"Learning German."

"I want you to give up German and learn Hindostanee."

"You are not in earnest?"

"In such earnest that I must have it so: and I will tell you why."

He then went on to explain that Hindostanee was the language he was
himself at present studying; that, as he advanced, he was apt to forget
the commencement; that it would assist him greatly to have a pupil with
whom he might again and again go over the elements, and so fix them
thoroughly in his mind; that his choice had hovered for some time between
me and his sisters; but that he had fixed on me because he saw I could
sit at a task the longest of the three.  Would I do him this favour?  I
should not, perhaps, have to make the sacrifice long, as it wanted now
barely three months to his departure.

St. John was not a man to be lightly refused: you felt that every
impression made on him, either for pain or pleasure, was deep-graved and
permanent.  I consented.  When Diana and Mary returned, the former found
her scholar transferred from her to her brother: she laughed, and both
she and Mary agreed that St. John should never have persuaded them to
such a step.  He answered quietly--

"I know it."

I found him a very patient, very forbearing, and yet an exacting master:
he expected me to do a great deal; and when I fulfilled his expectations,
he, in his own way, fully testified his approbation.  By degrees, he
acquired a certain influence over me that took away my liberty of mind:
his praise and notice were more restraining than his indifference.  I
could no longer talk or laugh freely when he was by, because a tiresomely
importunate instinct reminded me that vivacity (at least in me) was
distasteful to him.  I was so fully aware that only serious moods and
occupations were acceptable, that in his presence every effort to sustain
or follow any other became vain: I fell under a freezing spell.  When he
said "go," I went; "come," I came; "do this," I did it.  But I did not
love my servitude: I wished, many a time, he had continued to neglect me.

One evening when, at bedtime, his sisters and I stood round him, bidding
him good-night, he kissed each of them, as was his custom; and, as was
equally his custom, he gave me his hand.  Diana, who chanced to be in a
frolicsome humour (\emph{she} was not painfully controlled by his will; for
hers, in another way, was as strong), exclaimed--

"St. John! you used to call Jane your third sister, but you don't treat
her as such: you should kiss her too."

She pushed me towards him.  I thought Diana very provoking, and felt
uncomfortably confused; and while I was thus thinking and feeling, St.
John bent his head; his Greek face was brought to a level with mine, his
eyes questioned my eyes piercingly--he kissed me.  There are no such
things as marble kisses or ice kisses, or I should say my ecclesiastical
cousin's salute belonged to one of these classes; but there may be
experiment kisses, and his was an experiment kiss.  When given, he viewed
me to learn the result; it was not striking: I am sure I did not blush;
perhaps I might have turned a little pale, for I felt as if this kiss
were a seal affixed to my fetters.  He never omitted the ceremony
afterwards, and the gravity and quiescence with which I underwent it,
seemed to invest it for him with a certain charm.

As for me, I daily wished more to please him; but to do so, I felt daily
more and more that I must disown half my nature, stifle half my
faculties, wrest my tastes from their original bent, force myself to the
adoption of pursuits for which I had no natural vocation.  He wanted to
train me to an elevation I could never reach; it racked me hourly to
aspire to the standard he uplifted.  The thing was as impossible as to
mould my irregular features to his correct and classic pattern, to give
to my changeable green eyes the sea-blue tint and solemn lustre of his
own.

Not his ascendancy alone, however, held me in thrall at present.  Of late
it had been easy enough for me to look sad: a cankering evil sat at my
heart and drained my happiness at its source--the evil of suspense.

Perhaps you think I had forgotten Mr. Rochester, reader, amidst these
changes of place and fortune.  Not for a moment.  His idea was still with
me, because it was not a vapour sunshine could disperse, nor a
sand-traced effigy storms could wash away; it was a name graven on a
tablet, fated to last as long as the marble it inscribed.  The craving to
know what had become of him followed me everywhere; when I was at Morton,
I re-entered my cottage every evening to think of that; and now at Moor
House, I sought my bedroom each night to brood over it.

In the course of my necessary correspondence with Mr. Briggs about the
will, I had inquired if he knew anything of Mr. Rochester's present
residence and state of health; but, as St. John had conjectured, he was
quite ignorant of all concerning him.  I then wrote to Mrs. Fairfax,
entreating information on the subject.  I had calculated with certainty
on this step answering my end: I felt sure it would elicit an early
answer.  I was astonished when a fortnight passed without reply; but when
two months wore away, and day after day the post arrived and brought
nothing for me, I fell a prey to the keenest anxiety.

I wrote again: there was a chance of my first letter having missed.
Renewed hope followed renewed effort: it shone like the former for some
weeks, then, like it, it faded, flickered: not a line, not a word reached
me.  When half a year wasted in vain expectancy, my hope died out, and
then I felt dark indeed.

A fine spring shone round me, which I could not enjoy.  Summer
approached; Diana tried to cheer me: she said I looked ill, and wished to
accompany me to the sea-side.  This St. John opposed; he said I did not
want dissipation, I wanted employment; my present life was too
purposeless, I required an aim; and, I suppose, by way of supplying
deficiencies, he prolonged still further my lessons in Hindostanee, and
grew more urgent in requiring their accomplishment: and I, like a fool,
never thought of resisting him--I could not resist him.

One day I had come to my studies in lower spirits than usual; the ebb was
occasioned by a poignantly felt disappointment.  Hannah had told me in
the morning there was a letter for me, and when I went down to take it,
almost certain that the long-looked for tidings were vouchsafed me at
last, I found only an unimportant note from Mr. Briggs on business.  The
bitter check had wrung from me some tears; and now, as I sat poring over
the crabbed characters and flourishing tropes of an Indian scribe, my
eyes filled again.

St. John called me to his side to read; in attempting to do this my voice
failed me: words were lost in sobs.  He and I were the only occupants of
the parlour: Diana was practising her music in the drawing-room, Mary was
gardening--it was a very fine May day, clear, sunny, and breezy.  My
companion expressed no surprise at this emotion, nor did he question me
as to its cause; he only said--

"We will wait a few minutes, Jane, till you are more composed."  And
while I smothered the paroxysm with all haste, he sat calm and patient,
leaning on his desk, and looking like a physician watching with the eye
of science an expected and fully understood crisis in a patient's malady.
Having stifled my sobs, wiped my eyes, and muttered something about not
being very well that morning, I resumed my task, and succeeded in
completing it.  St. John put away my books and his, locked his desk, and
said--

"Now, Jane, you shall take a walk; and with me."

"I will call Diana and Mary."

"No; I want only one companion this morning, and that must be you.  Put
on your things; go out by the kitchen-door: take the road towards the
head of Marsh Glen: I will join you in a moment."

I know no medium: I never in my life have known any medium in my dealings
with positive, hard characters, antagonistic to my own, between absolute
submission and determined revolt.  I have always faithfully observed the
one, up to the very moment of bursting, sometimes with volcanic
vehemence, into the other; and as neither present circumstances
warranted, nor my present mood inclined me to mutiny, I observed careful
obedience to St. John's directions; and in ten minutes I was treading the
wild track of the glen, side by side with him.

The breeze was from the west: it came over the hills, sweet with scents
of heath and rush; the sky was of stainless blue; the stream descending
the ravine, swelled with past spring rains, poured along plentiful and
clear, catching golden gleams from the sun, and sapphire tints from the
firmament.  As we advanced and left the track, we trod a soft turf, mossy
fine and emerald green, minutely enamelled with a tiny white flower, and
spangled with a star-like yellow blossom: the hills, meantime, shut us
quite in; for the glen, towards its head, wound to their very core.

"Let us rest here," said St. John, as we reached the first stragglers of
a battalion of rocks, guarding a sort of pass, beyond which the beck
rushed down a waterfall; and where, still a little farther, the mountain
shook off turf and flower, had only heath for raiment and crag for
gem--where it exaggerated the wild to the savage, and exchanged the fresh
for the frowning--where it guarded the forlorn hope of solitude, and a
last refuge for silence.

I took a seat: St. John stood near me.  He looked up the pass and down
the hollow; his glance wandered away with the stream, and returned to
traverse the unclouded heaven which coloured it: he removed his hat, let
the breeze stir his hair and kiss his brow.  He seemed in communion with
the genius of the haunt: with his eye he bade farewell to something.

"And I shall see it again," he said aloud, "in dreams when I sleep by the
Ganges: and again in a more remote hour--when another slumber overcomes
me--on the shore of a darker stream!"

Strange words of a strange love!  An austere patriot's passion for his
fatherland!  He sat down; for half-an-hour we never spoke; neither he to
me nor I to him: that interval past, he recommenced--

"Jane, I go in six weeks; I have taken my berth in an East Indiaman which
sails on the 20th of June."

"God will protect you; for you have undertaken His work," I answered.

"Yes," said he, "there is my glory and joy.  I am the servant of an
infallible Master.  I am not going out under human guidance, subject to
the defective laws and erring control of my feeble fellow-worms: my king,
my lawgiver, my captain, is the All-perfect.  It seems strange to me that
all round me do not burn to enlist under the same banner,--to join in the
same enterprise."

"All have not your powers, and it would be folly for the feeble to wish
to march with the strong."

"I do not speak to the feeble, or think of them: I address only such as
are worthy of the work, and competent to accomplish it."

"Those are few in number, and difficult to discover."

"You say truly; but when found, it is right to stir them up--to urge and
exhort them to the effort--to show them what their gifts are, and why
they were given--to speak Heaven's message in their ear,--to offer them,
direct from God, a place in the ranks of His chosen."

"If they are really qualified for the task, will not their own hearts be
the first to inform them of it?"

I felt as if an awful charm was framing round and gathering over me: I
trembled to hear some fatal word spoken which would at once declare and
rivet the spell.

"And what does \emph{your} heart say?" demanded St. John.

"My heart is mute,--my heart is mute," I answered, struck and thrilled.

"Then I must speak for it," continued the deep, relentless voice.  "Jane,
come with me to India: come as my helpmeet and fellow-labourer."

The glen and sky spun round: the hills heaved!  It was as if I had heard
a summons from Heaven--as if a visionary messenger, like him of
Macedonia, had enounced, "Come over and help us!"  But I was no
apostle,--I could not behold the herald,--I could not receive his call.

"Oh, St. John!" I cried, "have some mercy!"

I appealed to one who, in the discharge of what he believed his duty,
knew neither mercy nor remorse.  He continued--

"God and nature intended you for a missionary's wife.  It is not
personal, but mental endowments they have given you: you are formed for
labour, not for love.  A missionary's wife you must--shall be.  You shall
be mine: I claim you--not for my pleasure, but for my Sovereign's
service."

"I am not fit for it: I have no vocation," I said.

He had calculated on these first objections: he was not irritated by
them.  Indeed, as he leaned back against the crag behind him, folded his
arms on his chest, and fixed his countenance, I saw he was prepared for a
long and trying opposition, and had taken in a stock of patience to last
him to its close--resolved, however, that that close should be conquest
for him.

"Humility, Jane," said he, "is the groundwork of Christian virtues: you
say right that you are not fit for the work.  Who is fit for it?  Or who,
that ever was truly called, believed himself worthy of the summons?  I,
for instance, am but dust and ashes.  With St. Paul, I acknowledge myself
the chiefest of sinners; but I do not suffer this sense of my personal
vileness to daunt me.  I know my Leader: that He is just as well as
mighty; and while He has chosen a feeble instrument to perform a great
task, He will, from the boundless stores of His providence, supply the
inadequacy of the means to the end.  Think like me, Jane--trust like me.
It is the Rock of Ages I ask you to lean on: do not doubt but it will
bear the weight of your human weakness."

"I do not understand a missionary life: I have never studied missionary
labours."

"There I, humble as I am, can give you the aid you want: I can set you
your task from hour to hour; stand by you always; help you from moment to
moment.  This I could do in the beginning: soon (for I know your powers)
you would be as strong and apt as myself, and would not require my help."

"But my powers--where are they for this undertaking?  I do not feel them.
Nothing speaks or stirs in me while you talk.  I am sensible of no light
kindling--no life quickening--no voice counselling or cheering.  Oh, I
wish I could make you see how much my mind is at this moment like a
rayless dungeon, with one shrinking fear fettered in its depths--the fear
of being persuaded by you to attempt what I cannot accomplish!"

"I have an answer for you--hear it.  I have watched you ever since we
first met: I have made you my study for ten months.  I have proved you in
that time by sundry tests: and what have I seen and elicited?  In the
village school I found you could perform well, punctually, uprightly,
labour uncongenial to your habits and inclinations; I saw you could
perform it with capacity and tact: you could win while you controlled.  In
the calm with which you learnt you had become suddenly rich, I read a
mind clear of the vice of Demas:--lucre had no undue power over you.  In
the resolute readiness with which you cut your wealth into four shares,
keeping but one to yourself, and relinquishing the three others to the
claim of abstract justice, I recognised a soul that revelled in the flame
and excitement of sacrifice.  In the tractability with which, at my wish,
you forsook a study in which you were interested, and adopted another
because it interested me; in the untiring assiduity with which you have
since persevered in it--in the unflagging energy and unshaken temper with
which you have met its difficulties--I acknowledge the complement of the
qualities I seek.  Jane, you are docile, diligent, disinterested,
faithful, constant, and courageous; very gentle, and very heroic: cease
to mistrust yourself--I can trust you unreservedly.  As a conductress of
Indian schools, and a helper amongst Indian women, your assistance will
be to me invaluable."

My iron shroud contracted round me; persuasion advanced with slow sure
step.  Shut my eyes as I would, these last words of his succeeded in
making the way, which had seemed blocked up, comparatively clear.  My
work, which had appeared so vague, so hopelessly diffuse, condensed
itself as he proceeded, and assumed a definite form under his shaping
hand.  He waited for an answer.  I demanded a quarter of an hour to
think, before I again hazarded a reply.

"Very willingly," he rejoined; and rising, he strode a little distance up
the pass, threw himself down on a swell of heath, and there lay still.

{He threw himself down on a swell of heath, and there lay still:
p389.jpg}

"I \emph{can} do what he wants me to do: I am forced to see and acknowledge
that," I meditated,--"that is, if life be spared me.  But I feel mine is
not the existence to be long protracted under an Indian sun.  What then?
He does not care for that: when my time came to die, he would resign me,
in all serenity and sanctity, to the God who gave me.  The case is very
plain before me.  In leaving England, I should leave a loved but empty
land--Mr. Rochester is not there; and if he were, what is, what can that
ever be to me?  My business is to live without him now: nothing so
absurd, so weak as to drag on from day to day, as if I were waiting some
impossible change in circumstances, which might reunite me to him.  Of
course (as St. John once said) I must seek another interest in life to
replace the one lost: is not the occupation he now offers me truly the
most glorious man can adopt or God assign?  Is it not, by its noble cares
and sublime results, the one best calculated to fill the void left by
uptorn affections and demolished hopes?  I believe I must say, Yes--and
yet I shudder.  Alas!  If I join St. John, I abandon half myself: if I go
to India, I go to premature death.  And how will the interval between
leaving England for India, and India for the grave, be filled?  Oh, I
know well!  That, too, is very clear to my vision.  By straining to
satisfy St. John till my sinews ache, I \emph{shall} satisfy him--to the
finest central point and farthest outward circle of his expectations.  If
I \emph{do} go with him--if I \emph{do} make the sacrifice he urges, I will make it
absolutely: I will throw all on the altar--heart, vitals, the entire
victim.  He will never love me; but he shall approve me; I will show him
energies he has not yet seen, resources he has never suspected.  Yes, I
can work as hard as he can, and with as little grudging.

"Consent, then, to his demand is possible: but for one item--one dreadful
item.  It is--that he asks me to be his wife, and has no more of a
husband's heart for me than that frowning giant of a rock, down which the
stream is foaming in yonder gorge.  He prizes me as a soldier would a
good weapon; and that is all.  Unmarried to him, this would never grieve
me; but can I let him complete his calculations--coolly put into practice
his plans--go through the wedding ceremony?  Can I receive from him the
bridal ring, endure all the forms of love (which I doubt not he would
scrupulously observe) and know that the spirit was quite absent?  Can I
bear the consciousness that every endearment he bestows is a sacrifice
made on principle?  No: such a martyrdom would be monstrous.  I will
never undergo it.  As his sister, I might accompany him--not as his wife:
I will tell him so."

I looked towards the knoll: there he lay, still as a prostrate column;
his face turned to me: his eye beaming watchful and keen.  He started to
his feet and approached me.

"I am ready to go to India, if I may go free."

"Your answer requires a commentary," he said; "it is not clear."

"You have hitherto been my adopted brother--I, your adopted sister: let
us continue as such: you and I had better not marry."

He shook his head.  "Adopted fraternity will not do in this case.  If you
were my real sister it would be different: I should take you, and seek no
wife.  But as it is, either our union must be consecrated and sealed by
marriage, or it cannot exist: practical obstacles oppose themselves to
any other plan.  Do you not see it, Jane?  Consider a moment--your strong
sense will guide you."

I did consider; and still my sense, such as it was, directed me only to
the fact that we did not love each other as man and wife should: and
therefore it inferred we ought not to marry.  I said so.  "St. John," I
returned, "I regard you as a brother--you, me as a sister: so let us
continue."

"We cannot--we cannot," he answered, with short, sharp determination: "it
would not do.  You have said you will go with me to India: remember--you
have said that."

"Conditionally."

"Well--well.  To the main point--the departure with me from England, the
co-operation with me in my future labours--you do not object.  You have
already as good as put your hand to the plough: you are too consistent to
withdraw it.  You have but one end to keep in view--how the work you have
undertaken can best be done.  Simplify your complicated interests,
feelings, thoughts, wishes, aims; merge all considerations in one
purpose: that of fulfilling with effect--with power--the mission of your
great Master.  To do so, you must have a coadjutor: not a brother--that
is a loose tie--but a husband.  I, too, do not want a sister: a sister
might any day be taken from me.  I want a wife: the sole helpmeet I can
influence efficiently in life, and retain absolutely till death."

I shuddered as he spoke: I felt his influence in my marrow--his hold on
my limbs.

"Seek one elsewhere than in me, St. John: seek one fitted to you."

"One fitted to my purpose, you mean--fitted to my vocation.  Again I tell
you it is not the insignificant private individual--the mere man, with
the man's selfish senses--I wish to mate: it is the missionary."

"And I will give the missionary my energies--it is all he wants--but not
myself: that would be only adding the husk and shell to the kernel.  For
them he has no use: I retain them."

"You cannot--you ought not.  Do you think God will be satisfied with half
an oblation?  Will He accept a mutilated sacrifice?  It is the cause of
God I advocate: it is under His standard I enlist you.  I cannot accept
on His behalf a divided allegiance: it must be entire."

"Oh!  I will give my heart to God," I said.  "\emph{You} do not want it."

I will not swear, reader, that there was not something of repressed
sarcasm both in the tone in which I uttered this sentence, and in the
feeling that accompanied it.  I had silently feared St. John till now,
because I had not understood him.  He had held me in awe, because he had
held me in doubt.  How much of him was saint, how much mortal, I could
not heretofore tell: but revelations were being made in this conference:
the analysis of his nature was proceeding before my eyes.  I saw his
fallibilities: I comprehended them.  I understood that, sitting there
where I did, on the bank of heath, and with that handsome form before me,
I sat at the feet of a man, caring as I.  The veil fell from his hardness
and despotism.  Having felt in him the presence of these qualities, I
felt his imperfection and took courage.  I was with an equal--one with
whom I might argue--one whom, if I saw good, I might resist.

He was silent after I had uttered the last sentence, and I presently
risked an upward glance at his countenance.

His eye, bent on me, expressed at once stern surprise and keen inquiry.
"Is she sarcastic, and sarcastic to \emph{me}!" it seemed to say.  "What does
this signify?"

"Do not let us forget that this is a solemn matter," he said ere long;
"one of which we may neither think nor talk lightly without sin.  I
trust, Jane, you are in earnest when you say you will serve your heart to
God: it is all I want.  Once wrench your heart from man, and fix it on
your Maker, the advancement of that Maker's spiritual kingdom on earth
will be your chief delight and endeavour; you will be ready to do at once
whatever furthers that end.  You will see what impetus would be given to
your efforts and mine by our physical and mental union in marriage: the
only union that gives a character of permanent conformity to the
destinies and designs of human beings; and, passing over all minor
caprices--all trivial difficulties and delicacies of feeling--all scruple
about the degree, kind, strength or tenderness of mere personal
inclination--you will hasten to enter into that union at once."

"Shall I?" I said briefly; and I looked at his features, beautiful in
their harmony, but strangely formidable in their still severity; at his
brow, commanding but not open; at his eyes, bright and deep and
searching, but never soft; at his tall imposing figure; and fancied
myself in idea \emph{his wife}.  Oh! it would never do!  As his curate, his
comrade, all would be right: I would cross oceans with him in that
capacity; toil under Eastern suns, in Asian deserts with him in that
office; admire and emulate his courage and devotion and vigour;
accommodate quietly to his masterhood; smile undisturbed at his
ineradicable ambition; discriminate the Christian from the man:
profoundly esteem the one, and freely forgive the other.  I should suffer
often, no doubt, attached to him only in this capacity: my body would be
under rather a stringent yoke, but my heart and mind would be free.  I
should still have my unblighted self to turn to: my natural unenslaved
feelings with which to communicate in moments of loneliness.  There would
be recesses in my mind which would be only mine, to which he never came,
and sentiments growing there fresh and sheltered which his austerity
could never blight, nor his measured warrior-march trample down: but as
his wife--at his side always, and always restrained, and always
checked--forced to keep the fire of my nature continually low, to compel
it to burn inwardly and never utter a cry, though the imprisoned flame
consumed vital after vital--\emph{this} would be unendurable.

"St. John!" I exclaimed, when I had got so far in my meditation.

"Well?" he answered icily.

"I repeat I freely consent to go with you as your fellow-missionary, but
not as your wife; I cannot marry you and become part of you."

"A part of me you must become," he answered steadily; "otherwise the
whole bargain is void.  How can I, a man not yet thirty, take out with me
to India a girl of nineteen, unless she be married to me?  How can we be
for ever together--sometimes in solitudes, sometimes amidst savage
tribes--and unwed?"

"Very well," I said shortly; "under the circumstances, quite as well as
if I were either your real sister, or a man and a clergyman like
yourself."

"It is known that you are not my sister; I cannot introduce you as such:
to attempt it would be to fasten injurious suspicions on us both.  And
for the rest, though you have a man's vigorous brain, you have a woman's
heart and--it would not do."

"It would do," I affirmed with some disdain, "perfectly well.  I have a
woman's heart, but not where you are concerned; for you I have only a
comrade's constancy; a fellow-soldier's frankness, fidelity, fraternity,
if you like; a neophyte's respect and submission to his hierophant:
nothing more--don't fear."

"It is what I want," he said, speaking to himself; "it is just what I
want.  And there are obstacles in the way: they must be hewn down.  Jane,
you would not repent marrying me--be certain of that; we \emph{must} be
married.  I repeat it: there is no other way; and undoubtedly enough of
love would follow upon marriage to render the union right even in your
eyes."

"I scorn your idea of love," I could not help saying, as I rose up and
stood before him, leaning my back against the rock.  "I scorn the
counterfeit sentiment you offer: yes, St. John, and I scorn you when you
offer it."

He looked at me fixedly, compressing his well-cut lips while he did so.
Whether he was incensed or surprised, or what, it was not easy to tell:
he could command his countenance thoroughly.

"I scarcely expected to hear that expression from you," he said: "I think
I have done and uttered nothing to deserve scorn."

I was touched by his gentle tone, and overawed by his high, calm mien.

"Forgive me the words, St. John; but it is your own fault that I have
been roused to speak so unguardedly.  You have introduced a topic on
which our natures are at variance--a topic we should never discuss: the
very name of love is an apple of discord between us.  If the reality were
required, what should we do?  How should we feel?  My dear cousin,
abandon your scheme of marriage--forget it."

"No," said he; "it is a long-cherished scheme, and the only one which can
secure my great end: but I shall urge you no further at present.
To-morrow, I leave home for Cambridge: I have many friends there to whom
I should wish to say farewell.  I shall be absent a fortnight--take that
space of time to consider my offer: and do not forget that if you reject
it, it is not me you deny, but God.  Through my means, He opens to you a
noble career; as my wife only can you enter upon it.  Refuse to be my
wife, and you limit yourself for ever to a track of selfish ease and
barren obscurity.  Tremble lest in that case you should be numbered with
those who have denied the faith, and are worse than infidels!"

He had done.  Turning from me, he once more

   "Looked to river, looked to hill."

But this time his feelings were all pent in his heart: I was not worthy
to hear them uttered.  As I walked by his side homeward, I read well in
his iron silence all he felt towards me: the disappointment of an austere
and despotic nature, which has met resistance where it expected
submission--the disapprobation of a cool, inflexible judgment, which has
detected in another feelings and views in which it has no power to
sympathise: in short, as a man, he would have wished to coerce me into
obedience: it was only as a sincere Christian he bore so patiently with
my perversity, and allowed so long a space for reflection and repentance.

That night, after he had kissed his sisters, he thought proper to forget
even to shake hands with me, but left the room in silence.  I--who,
though I had no love, had much friendship for him--was hurt by the marked
omission: so much hurt that tears started to my eyes.

"I see you and St. John have been quarrelling, Jane," said Diana, "during
your walk on the moor.  But go after him; he is now lingering in the
passage expecting you--he will make it up."

I have not much pride under such circumstances: I would always rather be
happy than dignified; and I ran after him--he stood at the foot of the
stairs.

"Good-night, St. John," said I.

"Good-night, Jane," he replied calmly.

"Then shake hands," I added.

What a cold, loose touch, he impressed on my fingers!  He was deeply
displeased by what had occurred that day; cordiality would not warm, nor
tears move him.  No happy reconciliation was to be had with him--no
cheering smile or generous word: but still the Christian was patient and
placid; and when I asked him if he forgave me, he answered that he was
not in the habit of cherishing the remembrance of vexation; that he had
nothing to forgive, not having been offended.

And with that answer he left me.  I would much rather he had knocked me
down.




\chapter{}


\lettrine{H}{e} did not leave for Cambridge the next day, as he had said he would.  He
deferred his departure a whole week, and during that time he made me feel
what severe punishment a good yet stern, a conscientious yet implacable
man can inflict on one who has offended him.  Without one overt act of
hostility, one upbraiding word, he contrived to impress me momently with
the conviction that I was put beyond the pale of his favour.

Not that St. John harboured a spirit of unchristian vindictiveness--not
that he would have injured a hair of my head, if it had been fully in his
power to do so.  Both by nature and principle, he was superior to the
mean gratification of vengeance: he had forgiven me for saying I scorned
him and his love, but he had not forgotten the words; and as long as he
and I lived he never would forget them.  I saw by his look, when he
turned to me, that they were always written on the air between me and
him; whenever I spoke, they sounded in my voice to his ear, and their
echo toned every answer he gave me.

He did not abstain from conversing with me: he even called me as usual
each morning to join him at his desk; and I fear the corrupt man within
him had a pleasure unimparted to, and unshared by, the pure Christian, in
evincing with what skill he could, while acting and speaking apparently
just as usual, extract from every deed and every phrase the spirit of
interest and approval which had formerly communicated a certain austere
charm to his language and manner.  To me, he was in reality become no
longer flesh, but marble; his eye was a cold, bright, blue gem; his
tongue a speaking instrument--nothing more.

All this was torture to me--refined, lingering torture.  It kept up a
slow fire of indignation and a trembling trouble of grief, which harassed
and crushed me altogether.  I felt how--if I were his wife, this good
man, pure as the deep sunless source, could soon kill me, without drawing
from my veins a single drop of blood, or receiving on his own crystal
conscience the faintest stain of crime.  Especially I felt this when I
made any attempt to propitiate him.  No ruth met my ruth.  \emph{He}
experienced no suffering from estrangement--no yearning after
reconciliation; and though, more than once, my fast falling tears
blistered the page over which we both bent, they produced no more effect
on him than if his heart had been really a matter of stone or metal.  To
his sisters, meantime, he was somewhat kinder than usual: as if afraid
that mere coldness would not sufficiently convince me how completely I
was banished and banned, he added the force of contrast; and this I am
sure he did not by force, but on principle.

The night before he left home, happening to see him walking in the garden
about sunset, and remembering, as I looked at him, that this man,
alienated as he now was, had once saved my life, and that we were near
relations, I was moved to make a last attempt to regain his friendship.  I
went out and approached him as he stood leaning over the little gate; I
spoke to the point at once.

"St. John, I am unhappy because you are still angry with me.  Let us be
friends."

"I hope we are friends," was the unmoved reply; while he still watched
the rising of the moon, which he had been contemplating as I approached.

"No, St. John, we are not friends as we were.  You know that."

"Are we not?  That is wrong.  For my part, I wish you no ill and all
good."

"I believe you, St. John; for I am sure you are incapable of wishing any
one ill; but, as I am your kinswoman, I should desire somewhat more of
affection than that sort of general philanthropy you extend to mere
strangers."

"Of course," he said.  "Your wish is reasonable, and I am far from
regarding you as a stranger."

This, spoken in a cool, tranquil tone, was mortifying and baffling
enough.  Had I attended to the suggestions of pride and ire, I should
immediately have left him; but something worked within me more strongly
than those feelings could.  I deeply venerated my cousin's talent and
principle.  His friendship was of value to me: to lose it tried me
severely.  I would not so soon relinquish the attempt to reconquer it.

"Must we part in this way, St. John?  And when you go to India, will you
leave me so, without a kinder word than you have yet spoken?"

He now turned quite from the moon and faced me.

"When I go to India, Jane, will I leave you!  What! do you not go to
India?"

"You said I could not unless I married you."

"And you will not marry me!  You adhere to that resolution?"

Reader, do you know, as I do, what terror those cold people can put into
the ice of their questions?  How much of the fall of the avalanche is in
their anger? of the breaking up of the frozen sea in their displeasure?

"No.  St. John, I will not marry you.  I adhere to my resolution."

The avalanche had shaken and slid a little forward, but it did not yet
crash down.

"Once more, why this refusal?" he asked.

"Formerly," I answered, "because you did not love me; now, I reply,
because you almost hate me.  If I were to marry you, you would kill me.
You are killing me now."

His lips and cheeks turned white--quite white.

"\emph{I should kill you}--\emph{I am killing you}?  Your words are such as ought
not to be used: violent, unfeminine, and untrue.  They betray an
unfortunate state of mind: they merit severe reproof: they would seem
inexcusable, but that it is the duty of man to forgive his fellow even
until seventy-and-seven times."

I had finished the business now.  While earnestly wishing to erase from
his mind the trace of my former offence, I had stamped on that tenacious
surface another and far deeper impression, I had burnt it in.

"Now you will indeed hate me," I said.  "It is useless to attempt to
conciliate you: I see I have made an eternal enemy of you."

A fresh wrong did these words inflict: the worse, because they touched on
the truth.  That bloodless lip quivered to a temporary spasm.  I knew the
steely ire I had whetted.  I was heart-wrung.

"You utterly misinterpret my words," I said, at once seizing his hand: "I
have no intention to grieve or pain you--indeed, I have not."

Most bitterly he smiled--most decidedly he withdrew his hand from mine.
"And now you recall your promise, and will not go to India at all, I
presume?" said he, after a considerable pause.

"Yes, I will, as your assistant," I answered.

A very long silence succeeded.  What struggle there was in him between
Nature and Grace in this interval, I cannot tell: only singular gleams
scintillated in his eyes, and strange shadows passed over his face.  He
spoke at last.

"I before proved to you the absurdity of a single woman of your age
proposing to accompany abroad a single man of mine.  I proved it to you
in such terms as, I should have thought, would have prevented your ever
again alluding to the plan.  That you have done so, I regret--for your
sake."

I interrupted him.  Anything like a tangible reproach gave me courage at
once.  "Keep to common sense, St. John: you are verging on nonsense.  You
pretend to be shocked by what I have said.  You are not really shocked:
for, with your superior mind, you cannot be either so dull or so
conceited as to misunderstand my meaning.  I say again, I will be your
curate, if you like, but never your wife."

Again he turned lividly pale; but, as before, controlled his passion
perfectly.  He answered emphatically but calmly--

"A female curate, who is not my wife, would never suit me.  With me,
then, it seems, you cannot go: but if you are sincere in your offer, I
will, while in town, speak to a married missionary, whose wife needs a
coadjutor.  Your own fortune will make you independent of the Society's
aid; and thus you may still be spared the dishonour of breaking your
promise and deserting the band you engaged to join."

Now I never had, as the reader knows, either given any formal promise or
entered into any engagement; and this language was all much too hard and
much too despotic for the occasion.  I replied--

"There is no dishonour, no breach of promise, no desertion in the case.  I
am not under the slightest obligation to go to India, especially with
strangers.  With you I would have ventured much, because I admire,
confide in, and, as a sister, I love you; but I am convinced that, go
when and with whom I would, I should not live long in that climate."

"Ah! you are afraid of yourself," he said, curling his lip.

"I am.  God did not give me my life to throw away; and to do as you wish
me would, I begin to think, be almost equivalent to committing suicide.
Moreover, before I definitively resolve on quitting England, I will know
for certain whether I cannot be of greater use by remaining in it than by
leaving it."

"What do you mean?"

"It would be fruitless to attempt to explain; but there is a point on
which I have long endured painful doubt, and I can go nowhere till by
some means that doubt is removed."

"I know where your heart turns and to what it clings.  The interest you
cherish is lawless and unconsecrated.  Long since you ought to have
crushed it: now you should blush to allude to it.  You think of Mr.
Rochester?"

It was true.  I confessed it by silence.

"Are you going to seek Mr. Rochester?"

"I must find out what is become of him."

"It remains for me, then," he said, "to remember you in my prayers, and
to entreat God for you, in all earnestness, that you may not indeed
become a castaway.  I had thought I recognised in you one of the chosen.
But God sees not as man sees: \emph{His} will be done--"

He opened the gate, passed through it, and strayed away down the glen.  He
was soon out of sight.

On re-entering the parlour, I found Diana standing at the window, looking
very thoughtful.  Diana was a great deal taller than I: she put her hand
on my shoulder, and, stooping, examined my face.

"Jane," she said, "you are always agitated and pale now.  I am sure there
is something the matter.  Tell me what business St. John and you have on
hands.  I have watched you this half hour from the window; you must
forgive my being such a spy, but for a long time I have fancied I hardly
know what.  St. John is a strange being--"

She paused--I did not speak: soon she resumed--

"That brother of mine cherishes peculiar views of some sort respecting
you, I am sure: he has long distinguished you by a notice and interest he
never showed to any one else--to what end?  I wish he loved you--does he,
Jane?"

I put her cool hand to my hot forehead; "No, Die, not one whit."

"Then why does he follow you so with his eyes, and get you so frequently
alone with him, and keep you so continually at his side?  Mary and I had
both concluded he wished you to marry him."

"He does--he has asked me to be his wife."

Diana clapped her hands.  "That is just what we hoped and thought!  And
you will marry him, Jane, won't you?  And then he will stay in England."

"Far from that, Diana; his sole idea in proposing to me is to procure a
fitting fellow-labourer in his Indian toils."

"What!  He wishes you to go to India?"

"Yes."

"Madness!" she exclaimed.  "You would not live three months there, I am
certain.  You never shall go: you have not consented, have you, Jane?"

"I have refused to marry him--"

"And have consequently displeased him?" she suggested.

"Deeply: he will never forgive me, I fear: yet I offered to accompany him
as his sister."

"It was frantic folly to do so, Jane.  Think of the task you
undertook--one of incessant fatigue, where fatigue kills even the strong,
and you are weak.  St. John--you know him--would urge you to
impossibilities: with him there would be no permission to rest during the
hot hours; and unfortunately, I have noticed, whatever he exacts, you
force yourself to perform.  I am astonished you found courage to refuse
his hand.  You do not love him then, Jane?"

"Not as a husband."

"Yet he is a handsome fellow."

"And I am so plain, you see, Die.  We should never suit."

"Plain!  You?  Not at all.  You are much too pretty, as well as too good,
to be grilled alive in Calcutta."  And again she earnestly conjured me to
give up all thoughts of going out with her brother.

"I must indeed," I said; "for when just now I repeated the offer of
serving him for a deacon, he expressed himself shocked at my want of
decency.  He seemed to think I had committed an impropriety in proposing
to accompany him unmarried: as if I had not from the first hoped to find
in him a brother, and habitually regarded him as such."

"What makes you say he does not love you, Jane?"

"You should hear himself on the subject.  He has again and again
explained that it is not himself, but his office he wishes to mate.  He
has told me I am formed for labour--not for love: which is true, no
doubt.  But, in my opinion, if I am not formed for love, it follows that
I am not formed for marriage.  Would it not be strange, Die, to be
chained for life to a man who regarded one but as a useful tool?"

"Insupportable--unnatural--out of the question!"

"And then," I continued, "though I have only sisterly affection for him
now, yet, if forced to be his wife, I can imagine the possibility of
conceiving an inevitable, strange, torturing kind of love for him,
because he is so talented; and there is often a certain heroic grandeur
in his look, manner, and conversation.  In that case, my lot would become
unspeakably wretched.  He would not want me to love him; and if I showed
the feeling, he would make me sensible that it was a superfluity,
unrequired by him, unbecoming in me.  I know he would."

"And yet St. John is a good man," said Diana.

"He is a good and a great man; but he forgets, pitilessly, the feelings
and claims of little people, in pursuing his own large views.  It is
better, therefore, for the insignificant to keep out of his way, lest, in
his progress, he should trample them down.  Here he comes!  I will leave
you, Diana."  And I hastened upstairs as I saw him entering the garden.

But I was forced to meet him again at supper.  During that meal he
appeared just as composed as usual.  I had thought he would hardly speak
to me, and I was certain he had given up the pursuit of his matrimonial
scheme: the sequel showed I was mistaken on both points.  He addressed me
precisely in his ordinary manner, or what had, of late, been his ordinary
manner--one scrupulously polite.  No doubt he had invoked the help of the
Holy Spirit to subdue the anger I had roused in him, and now believed he
had forgiven me once more.

For the evening reading before prayers, he selected the twenty-first
chapter of Revelation.  It was at all times pleasant to listen while from
his lips fell the words of the Bible: never did his fine voice sound at
once so sweet and full--never did his manner become so impressive in its
noble simplicity, as when he delivered the oracles of God: and to-night
that voice took a more solemn tone--that manner a more thrilling
meaning--as he sat in the midst of his household circle (the May moon
shining in through the uncurtained window, and rendering almost
unnecessary the light of the candle on the table): as he sat there,
bending over the great old Bible, and described from its page the vision
of the new heaven and the new earth--told how God would come to dwell
with men, how He would wipe away all tears from their eyes, and promised
that there should be no more death, neither sorrow nor crying, nor any
more pain, because the former things were passed away.

The succeeding words thrilled me strangely as he spoke them: especially
as I felt, by the slight, indescribable alteration in sound, that in
uttering them, his eye had turned on me.

"He that overcometh shall inherit all things; and I will be his God, and
he shall be my son.  But," was slowly, distinctly read, "the fearful, the
unbelieving, \&c., shall have their part in the lake which burneth with
fire and brimstone, which is the second death."

Henceforward, I knew what fate St. John feared for me.

A calm, subdued triumph, blent with a longing earnestness, marked his
enunciation of the last glorious verses of that chapter.  The reader
believed his name was already written in the Lamb's book of life, and he
yearned after the hour which should admit him to the city to which the
kings of the earth bring their glory and honour; which has no need of sun
or moon to shine in it, because the glory of God lightens it, and the
Lamb is the light thereof.

In the prayer following the chapter, all his energy gathered--all his
stern zeal woke: he was in deep earnest, wrestling with God, and resolved
on a conquest.  He supplicated strength for the weak-hearted; guidance
for wanderers from the fold: a return, even at the eleventh hour, for
those whom the temptations of the world and the flesh were luring from
the narrow path.  He asked, he urged, he claimed the boon of a brand
snatched from the burning.  Earnestness is ever deeply solemn: first, as
I listened to that prayer, I wondered at his; then, when it continued and
rose, I was touched by it, and at last awed.  He felt the greatness and
goodness of his purpose so sincerely: others who heard him plead for it,
could not but feel it too.

The prayer over, we took leave of him: he was to go at a very early hour
in the morning.  Diana and Mary having kissed him, left the room--in
compliance, I think, with a whispered hint from him: I tendered my hand,
and wished him a pleasant journey.

"Thank you, Jane.  As I said, I shall return from Cambridge in a
fortnight: that space, then, is yet left you for reflection.  If I
listened to human pride, I should say no more to you of marriage with me;
but I listen to my duty, and keep steadily in view my first aim--to do
all things to the glory of God.  My Master was long-suffering: so will I
be.  I cannot give you up to perdition as a vessel of wrath:
repent--resolve, while there is yet time.  Remember, we are bid to work
while it is day--warned that 'the night cometh when no man shall work.'
Remember the fate of Dives, who had his good things in this life.  God
give you strength to choose that better part which shall not be taken
from you!"

He laid his hand on my head as he uttered the last words.  He had spoken
earnestly, mildly: his look was not, indeed, that of a lover beholding
his mistress, but it was that of a pastor recalling his wandering
sheep--or better, of a guardian angel watching the soul for which he is
responsible.  All men of talent, whether they be men of feeling or not;
whether they be zealots, or aspirants, or despots--provided only they be
sincere--have their sublime moments, when they subdue and rule.  I felt
veneration for St. John--veneration so strong that its impetus thrust me
at once to the point I had so long shunned.  I was tempted to cease
struggling with him--to rush down the torrent of his will into the gulf
of his existence, and there lose my own.  I was almost as hard beset by
him now as I had been once before, in a different way, by another.  I was
a fool both times.  To have yielded then would have been an error of
principle; to have yielded now would have been an error of judgment.  So
I think at this hour, when I look back to the crisis through the quiet
medium of time: I was unconscious of folly at the instant.

I stood motionless under my hierophant's touch.  My refusals were
forgotten--my fears overcome--my wrestlings paralysed.  The
Impossible--\emph{i.e.}, my marriage with St. John--was fast becoming the
Possible.  All was changing utterly with a sudden sweep.  Religion
called--Angels beckoned--God commanded--life rolled together like a
scroll--death's gates opening, showed eternity beyond: it seemed, that
for safety and bliss there, all here might be sacrificed in a second.  The
dim room was full of visions.

"Could you decide now?" asked the missionary.  The inquiry was put in
gentle tones: he drew me to him as gently.  Oh, that gentleness! how far
more potent is it than force!  I could resist St. John's wrath: I grew
pliant as a reed under his kindness.  Yet I knew all the time, if I
yielded now, I should not the less be made to repent, some day, of my
former rebellion.  His nature was not changed by one hour of solemn
prayer: it was only elevated.

"I could decide if I were but certain," I answered: "were I but convinced
that it is God's will I should marry you, I could vow to marry you here
and now--come afterwards what would!"

"My prayers are heard!" ejaculated St. John.  He pressed his hand firmer
on my head, as if he claimed me: he surrounded me with his arm, \emph{almost}
as if he loved me (I say \emph{almost}--I knew the difference--for I had felt
what it was to be loved; but, like him, I had now put love out of the
question, and thought only of duty).  I contended with my inward dimness
of vision, before which clouds yet rolled.  I sincerely, deeply,
fervently longed to do what was right; and only that.  "Show me, show me
the path!" I entreated of Heaven.  I was excited more than I had ever
been; and whether what followed was the effect of excitement the reader
shall judge.

All the house was still; for I believe all, except St. John and myself,
were now retired to rest.  The one candle was dying out: the room was
full of moonlight.  My heart beat fast and thick: I heard its throb.
Suddenly it stood still to an inexpressible feeling that thrilled it
through, and passed at once to my head and extremities.  The feeling was
not like an electric shock, but it was quite as sharp, as strange, as
startling: it acted on my senses as if their utmost activity hitherto had
been but torpor, from which they were now summoned and forced to wake.
They rose expectant: eye and ear waited while the flesh quivered on my
bones.

"What have you heard?  What do you see?" asked St. John.  I saw nothing,
but I heard a voice somewhere cry--

"Jane!  Jane!  Jane!"--nothing more.

"O God! what is it?" I gasped.

I might have said, "Where is it?" for it did not seem in the room--nor in
the house--nor in the garden; it did not come out of the air--nor from
under the earth--nor from overhead.  I had heard it--where, or whence,
for ever impossible to know!  And it was the voice of a human being--a
known, loved, well-remembered voice--that of Edward Fairfax Rochester;
and it spoke in pain and woe, wildly, eerily, urgently.

"I am coming!" I cried.  "Wait for me!  Oh, I will come!"  I flew to the
door and looked into the passage: it was dark.  I ran out into the
garden: it was void.

"Where are you?" I exclaimed.

The hills beyond Marsh Glen sent the answer faintly back--"Where are
you?"  I listened.  The wind sighed low in the firs: all was moorland
loneliness and midnight hush.

"Down superstition!" I commented, as that spectre rose up black by the
black yew at the gate.  "This is not thy deception, nor thy witchcraft:
it is the work of nature.  She was roused, and did--no miracle--but her
best."

I broke from St. John, who had followed, and would have detained me.  It
was \emph{my} time to assume ascendency.  \emph{My} powers were in play and in
force.  I told him to forbear question or remark; I desired him to leave
me: I must and would be alone.  He obeyed at once.  Where there is energy
to command well enough, obedience never fails.  I mounted to my chamber;
locked myself in; fell on my knees; and prayed in my way--a different way
to St. John's, but effective in its own fashion.  I seemed to penetrate
very near a Mighty Spirit; and my soul rushed out in gratitude at His
feet.  I rose from the thanksgiving--took a resolve--and lay down,
unscared, enlightened--eager but for the daylight.




\chapter{}


\lettrine{T}{he} daylight came.  I rose at dawn.  I busied myself for an hour or two
with arranging my things in my chamber, drawers, and wardrobe, in the
order wherein I should wish to leave them during a brief absence.
Meantime, I heard St. John quit his room.  He stopped at my door: I
feared he would knock--no, but a slip of paper was passed under the door.
I took it up.  It bore these words--

"You left me too suddenly last night.  Had you stayed but a little
longer, you would have laid your hand on the Christian's cross and the
angel's crown.  I shall expect your clear decision when I return this day
fortnight.  Meantime, watch and pray that you enter not into temptation:
the spirit, I trust, is willing, but the flesh, I see, is weak.  I shall
pray for you hourly.--Yours, ST. JOHN."

"My spirit," I answered mentally, "is willing to do what is right; and my
flesh, I hope, is strong enough to accomplish the will of Heaven, when
once that will is distinctly known to me.  At any rate, it shall be
strong enough to search--inquire--to grope an outlet from this cloud of
doubt, and find the open day of certainty."

It was the first of June; yet the morning was overcast and chilly: rain
beat fast on my casement.  I heard the front-door open, and St. John pass
out.  Looking through the window, I saw him traverse the garden.  He took
the way over the misty moors in the direction of Whitcross--there he
would meet the coach.

"In a few more hours I shall succeed you in that track, cousin," thought
I: "I too have a coach to meet at Whitcross.  I too have some to see and
ask after in England, before I depart for ever."

It wanted yet two hours of breakfast-time.  I filled the interval in
walking softly about my room, and pondering the visitation which had
given my plans their present bent.  I recalled that inward sensation I
had experienced: for I could recall it, with all its unspeakable
strangeness.  I recalled the voice I had heard; again I questioned whence
it came, as vainly as before: it seemed in \emph{me}--not in the external
world.  I asked was it a mere nervous impression--a delusion?  I could
not conceive or believe: it was more like an inspiration.  The wondrous
shock of feeling had come like the earthquake which shook the foundations
of Paul and Silas's prison; it had opened the doors of the soul's cell
and loosed its bands--it had wakened it out of its sleep, whence it
sprang trembling, listening, aghast; then vibrated thrice a cry on my
startled ear, and in my quaking heart and through my spirit, which
neither feared nor shook, but exulted as if in joy over the success of
one effort it had been privileged to make, independent of the cumbrous
body.

"Ere many days," I said, as I terminated my musings, "I will know
something of him whose voice seemed last night to summon me.  Letters
have proved of no avail--personal inquiry shall replace them."

At breakfast I announced to Diana and Mary that I was going a journey,
and should be absent at least four days.

"Alone, Jane?" they asked.

"Yes; it was to see or hear news of a friend about whom I had for some
time been uneasy."

They might have said, as I have no doubt they thought, that they had
believed me to be without any friends save them: for, indeed, I had often
said so; but, with their true natural delicacy, they abstained from
comment, except that Diana asked me if I was sure I was well enough to
travel.  I looked very pale, she observed.  I replied, that nothing ailed
me save anxiety of mind, which I hoped soon to alleviate.

It was easy to make my further arrangements; for I was troubled with no
inquiries--no surmises.  Having once explained to them that I could not
now be explicit about my plans, they kindly and wisely acquiesced in the
silence with which I pursued them, according to me the privilege of free
action I should under similar circumstances have accorded them.

I left Moor House at three o'clock p.m., and soon after four I stood at
the foot of the sign-post of Whitcross, waiting the arrival of the coach
which was to take me to distant Thornfield.  Amidst the silence of those
solitary roads and desert hills, I heard it approach from a great
distance.  It was the same vehicle whence, a year ago, I had alighted one
summer evening on this very spot--how desolate, and hopeless, and
objectless!  It stopped as I beckoned.  I entered--not now obliged to
part with my whole fortune as the price of its accommodation.  Once more
on the road to Thornfield, I felt like the messenger-pigeon flying home.

It was a journey of six-and-thirty hours.  I had set out from Whitcross
on a Tuesday afternoon, and early on the succeeding Thursday morning the
coach stopped to water the horses at a wayside inn, situated in the midst
of scenery whose green hedges and large fields and low pastoral hills
(how mild of feature and verdant of hue compared with the stern North-
Midland moors of Morton!) met my eye like the lineaments of a once
familiar face.  Yes, I knew the character of this landscape: I was sure
we were near my bourne.

"How far is Thornfield Hall from here?" I asked of the ostler.

"Just two miles, ma'am, across the fields."

"My journey is closed," I thought to myself.  I got out of the coach,
gave a box I had into the ostler's charge, to be kept till I called for
it; paid my fare; satisfied the coachman, and was going: the brightening
day gleamed on the sign of the inn, and I read in gilt letters, "The
Rochester Arms."  My heart leapt up: I was already on my master's very
lands.  It fell again: the thought struck it:--

"Your master himself may be beyond the British Channel, for aught you
know: and then, if he is at Thornfield Hall, towards which you hasten,
who besides him is there?  His lunatic wife: and you have nothing to do
with him: you dare not speak to him or seek his presence.  You have lost
your labour--you had better go no farther," urged the monitor.  "Ask
information of the people at the inn; they can give you all you seek:
they can solve your doubts at once.  Go up to that man, and inquire if
Mr. Rochester be at home."

The suggestion was sensible, and yet I could not force myself to act on
it.  I so dreaded a reply that would crush me with despair.  To prolong
doubt was to prolong hope.  I might yet once more see the Hall under the
ray of her star.  There was the stile before me--the very fields through
which I had hurried, blind, deaf, distracted with a revengeful fury
tracking and scourging me, on the morning I fled from Thornfield: ere I
well knew what course I had resolved to take, I was in the midst of them.
How fast I walked!  How I ran sometimes!  How I looked forward to catch
the first view of the well-known woods!  With what feelings I welcomed
single trees I knew, and familiar glimpses of meadow and hill between
them!

At last the woods rose; the rookery clustered dark; a loud cawing broke
the morning stillness.  Strange delight inspired me: on I hastened.
Another field crossed--a lane threaded--and there were the courtyard
walls--the back offices: the house itself, the rookery still hid.  "My
first view of it shall be in front," I determined, "where its bold
battlements will strike the eye nobly at once, and where I can single out
my master's very window: perhaps he will be standing at it--he rises
early: perhaps he is now walking in the orchard, or on the pavement in
front.  Could I but see him!--but a moment!  Surely, in that case, I
should not be so mad as to run to him?  I cannot tell--I am not certain.
And if I did--what then?  God bless him!  What then?  Who would be hurt
by my once more tasting the life his glance can give me?  I rave: perhaps
at this moment he is watching the sun rise over the Pyrenees, or on the
tideless sea of the south."

I had coasted along the lower wall of the orchard--turned its angle:
there was a gate just there, opening into the meadow, between two stone
pillars crowned by stone balls.  From behind one pillar I could peep
round quietly at the full front of the mansion.  I advanced my head with
precaution, desirous to ascertain if any bedroom window-blinds were yet
drawn up: battlements, windows, long front--all from this sheltered
station were at my command.

The crows sailing overhead perhaps watched me while I took this survey.  I
wonder what they thought.  They must have considered I was very careful
and timid at first, and that gradually I grew very bold and reckless.  A
peep, and then a long stare; and then a departure from my niche and a
straying out into the meadow; and a sudden stop full in front of the
great mansion, and a protracted, hardy gaze towards it.  "What
affectation of diffidence was this at first?" they might have demanded;
"what stupid regardlessness now?"

Hear an illustration, reader.

A lover finds his mistress asleep on a mossy bank; he wishes to catch a
glimpse of her fair face without waking her.  He steals softly over the
grass, careful to make no sound; he pauses--fancying she has stirred: he
withdraws: not for worlds would he be seen.  All is still: he again
advances: he bends above her; a light veil rests on her features: he
lifts it, bends lower; now his eyes anticipate the vision of beauty--warm,
and blooming, and lovely, in rest.  How hurried was their first glance!
But how they fix!  How he starts!  How he suddenly and vehemently clasps
in both arms the form he dared not, a moment since, touch with his
finger!  How he calls aloud a name, and drops his burden, and gazes on it
wildly!  He thus grasps and cries, and gazes, because he no longer fears
to waken by any sound he can utter--by any movement he can make.  He
thought his love slept sweetly: he finds she is stone dead.

I looked with timorous joy towards a stately house: I saw a blackened
ruin.

No need to cower behind a gate-post, indeed!--to peep up at chamber
lattices, fearing life was astir behind them!  No need to listen for
doors opening--to fancy steps on the pavement or the gravel-walk!  The
lawn, the grounds were trodden and waste: the portal yawned void.  The
front was, as I had once seen it in a dream, but a well-like wall, very
high and very fragile-looking, perforated with paneless windows: no roof,
no battlements, no chimneys--all had crashed in.

And there was the silence of death about it: the solitude of a lonesome
wild.  No wonder that letters addressed to people here had never received
an answer: as well despatch epistles to a vault in a church aisle.  The
grim blackness of the stones told by what fate the Hall had fallen--by
conflagration: but how kindled?  What story belonged to this disaster?
What loss, besides mortar and marble and wood-work had followed upon it?
Had life been wrecked as well as property?  If so, whose?  Dreadful
question: there was no one here to answer it--not even dumb sign, mute
token.

In wandering round the shattered walls and through the devastated
interior, I gathered evidence that the calamity was not of late
occurrence.  Winter snows, I thought, had drifted through that void arch,
winter rains beaten in at those hollow casements; for, amidst the
drenched piles of rubbish, spring had cherished vegetation: grass and
weed grew here and there between the stones and fallen rafters.  And oh!
where meantime was the hapless owner of this wreck?  In what land?  Under
what auspices?  My eye involuntarily wandered to the grey church tower
near the gates, and I asked, "Is he with Damer de Rochester, sharing the
shelter of his narrow marble house?"

Some answer must be had to these questions.  I could find it nowhere but
at the inn, and thither, ere long, I returned.  The host himself brought
my breakfast into the parlour.  I requested him to shut the door and sit
down: I had some questions to ask him.  But when he complied, I scarcely
knew how to begin; such horror had I of the possible answers.  And yet
the spectacle of desolation I had just left prepared me in a measure for
a tale of misery.  The host was a respectable-looking, middle-aged man.

"You know Thornfield Hall, of course?" I managed to say at last.

"Yes, ma'am; I lived there once."

"Did you?"  Not in my time, I thought: you are a stranger to me.

"I was the late Mr. Rochester's butler," he added.

The late!  I seem to have received, with full force, the blow I had been
trying to evade.

"The late!" I gasped.  "Is he dead?"

"I mean the present gentleman, Mr. Edward's father," he explained.  I
breathed again: my blood resumed its flow.  Fully assured by these words
that Mr. Edward--\emph{my} Mr. Rochester (God bless him, wherever he was!)--was
at least alive: was, in short, "the present gentleman."  Gladdening
words!  It seemed I could hear all that was to come--whatever the
disclosures might be--with comparative tranquillity.  Since he was not in
the grave, I could bear, I thought, to learn that he was at the
Antipodes.

"Is Mr. Rochester living at Thornfield Hall now?" I asked, knowing, of
course, what the answer would be, but yet desirous of deferring the
direct question as to where he really was.

"No, ma'am--oh, no!  No one is living there.  I suppose you are a
stranger in these parts, or you would have heard what happened last
autumn,--Thornfield Hall is quite a ruin: it was burnt down just about
harvest-time.  A dreadful calamity! such an immense quantity of valuable
property destroyed: hardly any of the furniture could be saved.  The fire
broke out at dead of night, and before the engines arrived from Millcote,
the building was one mass of flame.  It was a terrible spectacle: I
witnessed it myself."

"At dead of night!" I muttered.  Yes, that was ever the hour of fatality
at Thornfield.  "Was it known how it originated?" I demanded.

"They guessed, ma'am: they guessed.  Indeed, I should say it was
ascertained beyond a doubt.  You are not perhaps aware," he continued,
edging his chair a little nearer the table, and speaking low, "that there
was a lady--a--a lunatic, kept in the house?"

"I have heard something of it."

"She was kept in very close confinement, ma'am: people even for some
years was not absolutely certain of her existence.  No one saw her: they
only knew by rumour that such a person was at the Hall; and who or what
she was it was difficult to conjecture.  They said Mr. Edward had brought
her from abroad, and some believed she had been his mistress.  But a
queer thing happened a year since--a very queer thing."

I feared now to hear my own story.  I endeavoured to recall him to the
main fact.

"And this lady?"

"This lady, ma'am," he answered, "turned out to be Mr. Rochester's wife!
The discovery was brought about in the strangest way.  There was a young
lady, a governess at the Hall, that Mr. Rochester fell in--"

"But the fire," I suggested.

"I'm coming to that, ma'am--that Mr. Edward fell in love with.  The
servants say they never saw anybody so much in love as he was: he was
after her continually.  They used to watch him--servants will, you know,
ma'am--and he set store on her past everything: for all, nobody but him
thought her so very handsome.  She was a little small thing, they say,
almost like a child.  I never saw her myself; but I've heard Leah, the
house-maid, tell of her.  Leah liked her well enough.  Mr. Rochester was
about forty, and this governess not twenty; and you see, when gentlemen
of his age fall in love with girls, they are often like as if they were
bewitched.  Well, he would marry her."

"You shall tell me this part of the story another time," I said; "but now
I have a particular reason for wishing to hear all about the fire.  Was
it suspected that this lunatic, Mrs. Rochester, had any hand in it?"

"You've hit it, ma'am: it's quite certain that it was her, and nobody but
her, that set it going.  She had a woman to take care of her called Mrs.
Poole--an able woman in her line, and very trustworthy, but for one
fault--a fault common to a deal of them nurses and matrons--she \emph{kept a
private bottle of gin by her}, and now and then took a drop over-much.  It
is excusable, for she had a hard life of it: but still it was dangerous;
for when Mrs. Poole was fast asleep after the gin and water, the mad
lady, who was as cunning as a witch, would take the keys out of her
pocket, let herself out of her chamber, and go roaming about the house,
doing any wild mischief that came into her head.  They say she had nearly
burnt her husband in his bed once: but I don't know about that.  However,
on this night, she set fire first to the hangings of the room next her
own, and then she got down to a lower storey, and made her way to the
chamber that had been the governess's--(she was like as if she knew
somehow how matters had gone on, and had a spite at her)--and she kindled
the bed there; but there was nobody sleeping in it, fortunately.  The
governess had run away two months before; and for all Mr. Rochester
sought her as if she had been the most precious thing he had in the
world, he never could hear a word of her; and he grew savage--quite
savage on his disappointment: he never was a wild man, but he got
dangerous after he lost her.  He would be alone, too.  He sent Mrs.
Fairfax, the housekeeper, away to her friends at a distance; but he did
it handsomely, for he settled an annuity on her for life: and she
deserved it--she was a very good woman.  Miss Adele, a ward he had, was
put to school.  He broke off acquaintance with all the gentry, and shut
himself up like a hermit at the Hall."

"What! did he not leave England?"

"Leave England?  Bless you, no!  He would not cross the door-stones of
the house, except at night, when he walked just like a ghost about the
grounds and in the orchard as if he had lost his senses--which it is my
opinion he had; for a more spirited, bolder, keener gentleman than he was
before that midge of a governess crossed him, you never saw, ma'am.  He
was not a man given to wine, or cards, or racing, as some are, and he was
not so very handsome; but he had a courage and a will of his own, if ever
man had.  I knew him from a boy, you see: and for my part, I have often
wished that Miss Eyre had been sunk in the sea before she came to
Thornfield Hall."

"Then Mr. Rochester was at home when the fire broke out?"

"Yes, indeed was he; and he went up to the attics when all was burning
above and below, and got the servants out of their beds and helped them
down himself, and went back to get his mad wife out of her cell.  And
then they called out to him that she was on the roof, where she was
standing, waving her arms, above the battlements, and shouting out till
they could hear her a mile off: I saw her and heard her with my own eyes.
She was a big woman, and had long black hair: we could see it streaming
against the flames as she stood.  I witnessed, and several more
witnessed, Mr. Rochester ascend through the sky-light on to the roof; we
heard him call 'Bertha!'  We saw him approach her; and then, ma'am, she
yelled and gave a spring, and the next minute she lay smashed on the
pavement."

{The next minute she lay smashed on the pavement: p413.jpg}

"Dead?"

"Dead!  Ay, dead as the stones on which her brains and blood were
scattered."

"Good God!"

"You may well say so, ma'am: it was frightful!"

He shuddered.

"And afterwards?" I urged.

"Well, ma'am, afterwards the house was burnt to the ground: there are
only some bits of walls standing now."

"Were any other lives lost?"

"No--perhaps it would have been better if there had."

"What do you mean?"

"Poor Mr. Edward!" he ejaculated, "I little thought ever to have seen it!
Some say it was a just judgment on him for keeping his first marriage
secret, and wanting to take another wife while he had one living: but I
pity him, for my part."

"You said he was alive?" I exclaimed.

"Yes, yes: he is alive; but many think he had better be dead."

"Why?  How?"  My blood was again running cold.  "Where is he?" I
demanded.  "Is he in England?"

"Ay--ay--he's in England; he can't get out of England, I fancy--he's a
fixture now."

What agony was this!  And the man seemed resolved to protract it.

"He is stone-blind," he said at last.  "Yes, he is stone-blind, is Mr.
Edward."

I had dreaded worse.  I had dreaded he was mad.  I summoned strength to
ask what had caused this calamity.

"It was all his own courage, and a body may say, his kindness, in a way,
ma'am: he wouldn't leave the house till every one else was out before
him.  As he came down the great staircase at last, after Mrs. Rochester
had flung herself from the battlements, there was a great crash--all
fell.  He was taken out from under the ruins, alive, but sadly hurt: a
beam had fallen in such a way as to protect him partly; but one eye was
knocked out, and one hand so crushed that Mr. Carter, the surgeon, had to
amputate it directly.  The other eye inflamed: he lost the sight of that
also.  He is now helpless, indeed--blind and a cripple."

"Where is he?  Where does he now live?"

"At Ferndean, a manor-house on a farm he has, about thirty miles off:
quite a desolate spot."

"Who is with him?"

"Old John and his wife: he would have none else.  He is quite broken
down, they say."

"Have you any sort of conveyance?"

"We have a chaise, ma'am, a very handsome chaise."

"Let it be got ready instantly; and if your post-boy can drive me to
Ferndean before dark this day, I'll pay both you and him twice the hire
you usually demand."




\chapter{}


\lettrine{T}{he} manor-house of Ferndean was a building of considerable antiquity,
moderate size, and no architectural pretensions, deep buried in a wood.  I
had heard of it before.  Mr. Rochester often spoke of it, and sometimes
went there.  His father had purchased the estate for the sake of the game
covers.  He would have let the house, but could find no tenant, in
consequence of its ineligible and insalubrious site.  Ferndean then
remained uninhabited and unfurnished, with the exception of some two or
three rooms fitted up for the accommodation of the squire when he went
there in the season to shoot.

To this house I came just ere dark on an evening marked by the
characteristics of sad sky, cold gale, and continued small penetrating
rain.  The last mile I performed on foot, having dismissed the chaise and
driver with the double remuneration I had promised.  Even when within a
very short distance of the manor-house, you could see nothing of it, so
thick and dark grew the timber of the gloomy wood about it.  Iron gates
between granite pillars showed me where to enter, and passing through
them, I found myself at once in the twilight of close-ranked trees.  There
was a grass-grown track descending the forest aisle between hoar and
knotty shafts and under branched arches.  I followed it, expecting soon
to reach the dwelling; but it stretched on and on, it wound far and
farther: no sign of habitation or grounds was visible.

I thought I had taken a wrong direction and lost my way.  The darkness of
natural as well as of sylvan dusk gathered over me.  I looked round in
search of another road.  There was none: all was interwoven stem,
columnar trunk, dense summer foliage--no opening anywhere.

I proceeded: at last my way opened, the trees thinned a little; presently
I beheld a railing, then the house--scarce, by this dim light,
distinguishable from the trees; so dank and green were its decaying
walls.  Entering a portal, fastened only by a latch, I stood amidst a
space of enclosed ground, from which the wood swept away in a semicircle.
There were no flowers, no garden-beds; only a broad gravel-walk girdling
a grass-plat, and this set in the heavy frame of the forest.  The house
presented two pointed gables in its front; the windows were latticed and
narrow: the front door was narrow too, one step led up to it.  The whole
looked, as the host of the Rochester Arms had said, "quite a desolate
spot."  It was as still as a church on a week-day: the pattering rain on
the forest leaves was the only sound audible in its vicinage.

"Can there be life here?" I asked.

Yes, life of some kind there was; for I heard a movement--that narrow
front-door was unclosing, and some shape was about to issue from the
grange.

It opened slowly: a figure came out into the twilight and stood on the
step; a man without a hat: he stretched forth his hand as if to feel
whether it rained.  Dusk as it was, I had recognised him--it was my
master, Edward Fairfax Rochester, and no other.

I stayed my step, almost my breath, and stood to watch him--to examine
him, myself unseen, and alas! to him invisible.  It was a sudden meeting,
and one in which rapture was kept well in check by pain.  I had no
difficulty in restraining my voice from exclamation, my step from hasty
advance.

His form was of the same strong and stalwart contour as ever: his port
was still erect, his hair was still raven black; nor were his features
altered or sunk: not in one year's space, by any sorrow, could his
athletic strength be quelled or his vigorous prime blighted.  But in his
countenance I saw a change: that looked desperate and brooding--that
reminded me of some wronged and fettered wild beast or bird, dangerous to
approach in his sullen woe.  The caged eagle, whose gold-ringed eyes
cruelty has extinguished, might look as looked that sightless Samson.

And, reader, do you think I feared him in his blind ferocity?--if you do,
you little know me.  A soft hope blest with my sorrow that soon I should
dare to drop a kiss on that brow of rock, and on those lips so sternly
sealed beneath it: but not yet.  I would not accost him yet.

He descended the one step, and advanced slowly and gropingly towards the
grass-plat.  Where was his daring stride now?  Then he paused, as if he
knew not which way to turn.  He lifted his hand and opened his eyelids;
gazed blank, and with a straining effort, on the sky, and toward the
amphitheatre of trees: one saw that all to him was void darkness.  He
stretched his right hand (the left arm, the mutilated one, he kept hidden
in his bosom); he seemed to wish by touch to gain an idea of what lay
around him: he met but vacancy still; for the trees were some yards off
where he stood.  He relinquished the endeavour, folded his arms, and
stood quiet and mute in the rain, now falling fast on his uncovered head.
At this moment John approached him from some quarter.

"Will you take my arm, sir?" he said; "there is a heavy shower coming on:
had you not better go in?"

"Let me alone," was the answer.

John withdrew without having observed me.  Mr. Rochester now tried to
walk about: vainly,--all was too uncertain.  He groped his way back to
the house, and, re-entering it, closed the door.

I now drew near and knocked: John's wife opened for me.  "Mary," I said,
"how are you?"

She started as if she had seen a ghost: I calmed her.  To her hurried "Is
it really you, miss, come at this late hour to this lonely place?"  I
answered by taking her hand; and then I followed her into the kitchen,
where John now sat by a good fire.  I explained to them, in few words,
that I had heard all which had happened since I left Thornfield, and that
I was come to see Mr. Rochester.  I asked John to go down to the turn-
pike-house, where I had dismissed the chaise, and bring my trunk, which I
had left there: and then, while I removed my bonnet and shawl, I
questioned Mary as to whether I could be accommodated at the Manor House
for the night; and finding that arrangements to that effect, though
difficult, would not be impossible, I informed her I should stay.  Just
at this moment the parlour-bell rang.

"When you go in," said I, "tell your master that a person wishes to speak
to him, but do not give my name."

"I don't think he will see you," she answered; "he refuses everybody."

When she returned, I inquired what he had said.  "You are to send in your
name and your business," she replied.  She then proceeded to fill a glass
with water, and place it on a tray, together with candles.

"Is that what he rang for?" I asked.

"Yes: he always has candles brought in at dark, though he is blind."

"Give the tray to me; I will carry it in."

I took it from her hand: she pointed me out the parlour door.  The tray
shook as I held it; the water spilt from the glass; my heart struck my
ribs loud and fast.  Mary opened the door for me, and shut it behind me.

This parlour looked gloomy: a neglected handful of fire burnt low in the
grate; and, leaning over it, with his head supported against the high,
old-fashioned mantelpiece, appeared the blind tenant of the room.  His
old dog, Pilot, lay on one side, removed out of the way, and coiled up as
if afraid of being inadvertently trodden upon.  Pilot pricked up his ears
when I came in: then he jumped up with a yelp and a whine, and bounded
towards me: he almost knocked the tray from my hands.  I set it on the
table; then patted him, and said softly, "Lie down!"  Mr. Rochester
turned mechanically to \emph{see} what the commotion was: but as he \emph{saw}
nothing, he returned and sighed.

"Give me the water, Mary," he said.

I approached him with the now only half-filled glass; Pilot followed me,
still excited.

"What is the matter?" he inquired.

"Down, Pilot!" I again said.  He checked the water on its way to his
lips, and seemed to listen: he drank, and put the glass down.  "This is
you, Mary, is it not?"

"Mary is in the kitchen," I answered.

He put out his hand with a quick gesture, but not seeing where I stood,
he did not touch me.  "Who is this?  Who is this?" he demanded, trying,
as it seemed, to \emph{see} with those sightless eyes--unavailing and
distressing attempt!  "Answer me--speak again!" he ordered, imperiously
and aloud.

"Will you have a little more water, sir?  I spilt half of what was in the
glass," I said.

"\emph{Who} is it?  \emph{What} is it?  Who speaks?"

"Pilot knows me, and John and Mary know I am here.  I came only this
evening," I answered.

"Great God!--what delusion has come over me?  What sweet madness has
seized me?"

"No delusion--no madness: your mind, sir, is too strong for delusion,
your health too sound for frenzy."

"And where is the speaker?  Is it only a voice?  Oh!  I \emph{cannot} see, but
I must feel, or my heart will stop and my brain burst.  Whatever--whoever
you are--be perceptible to the touch or I cannot live!"

He groped; I arrested his wandering hand, and prisoned it in both mine.

"Her very fingers!" he cried; "her small, slight fingers!  If so there
must be more of her."

The muscular hand broke from my custody; my arm was seized, my
shoulder--neck--waist--I was entwined and gathered to him.

"Is it Jane?  \emph{What} is it?  This is her shape--this is her size--"

"And this her voice," I added.  "She is all here: her heart, too.  God
bless you, sir!  I am glad to be so near you again."

"Jane Eyre!--Jane Eyre," was all he said.

"My dear master," I answered, "I am Jane Eyre: I have found you out--I am
come back to you."

"In truth?--in the flesh?  My living Jane?"

"You touch me, sir,--you hold me, and fast enough: I am not cold like a
corpse, nor vacant like air, am I?"

"My living darling!  These are certainly her limbs, and these her
features; but I cannot be so blest, after all my misery.  It is a dream;
such dreams as I have had at night when I have clasped her once more to
my heart, as I do now; and kissed her, as thus--and felt that she loved
me, and trusted that she would not leave me."

"Which I never will, sir, from this day."

"Never will, says the vision?  But I always woke and found it an empty
mockery; and I was desolate and abandoned--my life dark, lonely,
hopeless--my soul athirst and forbidden to drink--my heart famished and
never to be fed.  Gentle, soft dream, nestling in my arms now, you will
fly, too, as your sisters have all fled before you: but kiss me before
you go--embrace me, Jane."

"There, sir--and there!"'

I pressed my lips to his once brilliant and now rayless eyes--I swept his
hair from his brow, and kissed that too.  He suddenly seemed to arouse
himself: the conviction of the reality of all this seized him.

"It is you--is it, Jane?  You are come back to me then?"

"I am."

"And you do not lie dead in some ditch under some stream?  And you are
not a pining outcast amongst strangers?"

"No, sir!  I am an independent woman now."

"Independent!  What do you mean, Jane?"

"My uncle in Madeira is dead, and he left me five thousand pounds."

"Ah! this is practical--this is real!" he cried: "I should never dream
that.  Besides, there is that peculiar voice of hers, so animating and
piquant, as well as soft: it cheers my withered heart; it puts life into
it.--What, Janet!  Are you an independent woman?  A rich woman?"

"If you won't let me live with you, I can build a house of my own close
up to your door, and you may come and sit in my parlour when you want
company of an evening."

"But as you are rich, Jane, you have now, no doubt, friends who will look
after you, and not suffer you to devote yourself to a blind lameter like
me?"

"I told you I am independent, sir, as well as rich: I am my own
mistress."

"And you will stay with me?"

"Certainly--unless you object.  I will be your neighbour, your nurse,
your housekeeper.  I find you lonely: I will be your companion--to read
to you, to walk with you, to sit with you, to wait on you, to be eyes and
hands to you.  Cease to look so melancholy, my dear master; you shall not
be left desolate, so long as I live."

He replied not: he seemed serious--abstracted; he sighed; he half-opened
his lips as if to speak: he closed them again.  I felt a little
embarrassed.  Perhaps I had too rashly over-leaped conventionalities; and
he, like St. John, saw impropriety in my inconsiderateness.  I had indeed
made my proposal from the idea that he wished and would ask me to be his
wife: an expectation, not the less certain because unexpressed, had
buoyed me up, that he would claim me at once as his own.  But no hint to
that effect escaping him and his countenance becoming more overcast, I
suddenly remembered that I might have been all wrong, and was perhaps
playing the fool unwittingly; and I began gently to withdraw myself from
his arms--but he eagerly snatched me closer.

"No--no--Jane; you must not go.  No--I have touched you, heard you, felt
the comfort of your presence--the sweetness of your consolation: I cannot
give up these joys.  I have little left in myself--I must have you.  The
world may laugh--may call me absurd, selfish--but it does not signify.  My
very soul demands you: it will be satisfied, or it will take deadly
vengeance on its frame."

"Well, sir, I will stay with you: I have said so."

"Yes--but you understand one thing by staying with me; and I understand
another.  You, perhaps, could make up your mind to be about my hand and
chair--to wait on me as a kind little nurse (for you have an affectionate
heart and a generous spirit, which prompt you to make sacrifices for
those you pity), and that ought to suffice for me no doubt.  I suppose I
should now entertain none but fatherly feelings for you: do you think so?
Come--tell me."

"I will think what you like, sir: I am content to be only your nurse, if
you think it better."

"But you cannot always be my nurse, Janet: you are young--you must marry
one day."

"I don't care about being married."

"You should care, Janet: if I were what I once was, I would try to make
you care--but--a sightless block!"

He relapsed again into gloom.  I, on the contrary, became more cheerful,
and took fresh courage: these last words gave me an insight as to where
the difficulty lay; and as it was no difficulty with me, I felt quite
relieved from my previous embarrassment.  I resumed a livelier vein of
conversation.

"It is time some one undertook to rehumanise you," said I, parting his
thick and long uncut locks; "for I see you are being metamorphosed into a
lion, or something of that sort.  You have a 'faux air' of Nebuchadnezzar
in the fields about you, that is certain: your hair reminds me of eagles'
feathers; whether your nails are grown like birds' claws or not, I have
not yet noticed."

"On this arm, I have neither hand nor nails," he said, drawing the
mutilated limb from his breast, and showing it to me.  "It is a mere
stump--a ghastly sight!  Don't you think so, Jane?"

"It is a pity to see it; and a pity to see your eyes--and the scar of
fire on your forehead: and the worst of it is, one is in danger of loving
you too well for all this; and making too much of you."

"I thought you would be revolted, Jane, when you saw my arm, and my
cicatrised visage."

"Did you?  Don't tell me so--lest I should say something disparaging to
your judgment.  Now, let me leave you an instant, to make a better fire,
and have the hearth swept up.  Can you tell when there is a good fire?"

"Yes; with the right eye I see a glow--a ruddy haze."

"And you see the candles?"

"Very dimly--each is a luminous cloud."

"Can you see me?"

"No, my fairy: but I am only too thankful to hear and feel you."

"When do you take supper?"

"I never take supper."

"But you shall have some to-night.  I am hungry: so are you, I daresay,
only you forget."

Summoning Mary, I soon had the room in more cheerful order: I prepared
him, likewise, a comfortable repast.  My spirits were excited, and with
pleasure and ease I talked to him during supper, and for a long time
after.  There was no harassing restraint, no repressing of glee and
vivacity with him; for with him I was at perfect ease, because I knew I
suited him; all I said or did seemed either to console or revive him.
Delightful consciousness!  It brought to life and light my whole nature:
in his presence I thoroughly lived; and he lived in mine.  Blind as he
was, smiles played over his face, joy dawned on his forehead: his
lineaments softened and warmed.

After supper, he began to ask me many questions, of where I had been,
what I had been doing, how I had found him out; but I gave him only very
partial replies: it was too late to enter into particulars that night.
Besides, I wished to touch no deep-thrilling chord--to open no fresh well
of emotion in his heart: my sole present aim was to cheer him.  Cheered,
as I have said, he was: and yet but by fits.  If a moment's silence broke
the conversation, he would turn restless, touch me, then say, "Jane."

"You are altogether a human being, Jane?  You are certain of that?"

{You are altogether a human being, Jane?  You are certain of that?:
p422.jpg}

"I conscientiously believe so, Mr. Rochester."

"Yet how, on this dark and doleful evening, could you so suddenly rise on
my lone hearth?  I stretched my hand to take a glass of water from a
hireling, and it was given me by you: I asked a question, expecting
John's wife to answer me, and your voice spoke at my ear."

"Because I had come in, in Mary's stead, with the tray."

"And there is enchantment in the very hour I am now spending with you.
Who can tell what a dark, dreary, hopeless life I have dragged on for
months past?  Doing nothing, expecting nothing; merging night in day;
feeling but the sensation of cold when I let the fire go out, of hunger
when I forgot to eat: and then a ceaseless sorrow, and, at times, a very
delirium of desire to behold my Jane again.  Yes: for her restoration I
longed, far more than for that of my lost sight.  How can it be that Jane
is with me, and says she loves me?  Will she not depart as suddenly as
she came?  To-morrow, I fear I shall find her no more."

A commonplace, practical reply, out of the train of his own disturbed
ideas, was, I was sure, the best and most reassuring for him in this
frame of mind.  I passed my finger over his eyebrows, and remarked that
they were scorched, and that I would apply something which would make
them grow as broad and black as ever.

"Where is the use of doing me good in any way, beneficent spirit, when,
at some fatal moment, you will again desert me--passing like a shadow,
whither and how to me unknown, and for me remaining afterwards
undiscoverable?

"Have you a pocket-comb about you, sir?"

"What for, Jane?"

"Just to comb out this shaggy black mane.  I find you rather alarming,
when I examine you close at hand: you talk of my being a fairy, but I am
sure, you are more like a brownie."

"Am I hideous, Jane?"

"Very, sir: you always were, you know."

"Humph!  The wickedness has not been taken out of you, wherever you have
sojourned."

"Yet I have been with good people; far better than you: a hundred times
better people; possessed of ideas and views you never entertained in your
life: quite more refined and exalted."

"Who the deuce have you been with?"

"If you twist in that way you will make me pull the hair out of your
head; and then I think you will cease to entertain doubts of my
substantiality."

"Who have you been with, Jane?"

"You shall not get it out of me to-night, sir; you must wait till
to-morrow; to leave my tale half told, will, you know, be a sort of
security that I shall appear at your breakfast table to finish it.  By
the bye, I must mind not to rise on your hearth with only a glass of
water then: I must bring an egg at the least, to say nothing of fried
ham."

"You mocking changeling--fairy-born and human-bred!  You make me feel as
I have not felt these twelve months.  If Saul could have had you for his
David, the evil spirit would have been exorcised without the aid of the
harp."

"There, sir, you are redd up and made decent.  Now I'll leave you: I have
been travelling these last three days, and I believe I am tired.  Good
night."

"Just one word, Jane: were there only ladies in the house where you have
been?"

I laughed and made my escape, still laughing as I ran upstairs.  "A good
idea!"  I thought with glee.  "I see I have the means of fretting him out
of his melancholy for some time to come."

Very early the next morning I heard him up and astir, wandering from one
room to another.  As soon as Mary came down I heard the question: "Is
Miss Eyre here?"  Then: "Which room did you put her into?  Was it dry?  Is
she up?  Go and ask if she wants anything; and when she will come down."

I came down as soon as I thought there was a prospect of breakfast.
Entering the room very softly, I had a view of him before he discovered
my presence.  It was mournful, indeed, to witness the subjugation of that
vigorous spirit to a corporeal infirmity.  He sat in his chair--still,
but not at rest: expectant evidently; the lines of now habitual sadness
marking his strong features.  His countenance reminded one of a lamp
quenched, waiting to be re-lit--and alas! it was not himself that could
now kindle the lustre of animated expression: he was dependent on another
for that office!  I had meant to be gay and careless, but the
powerlessness of the strong man touched my heart to the quick: still I
accosted him with what vivacity I could.

"It is a bright, sunny morning, sir," I said.  "The rain is over and
gone, and there is a tender shining after it: you shall have a walk
soon."

I had wakened the glow: his features beamed.

"Oh, you are indeed there, my skylark!  Come to me.  You are not gone:
not vanished?  I heard one of your kind an hour ago, singing high over
the wood: but its song had no music for me, any more than the rising sun
had rays.  All the melody on earth is concentrated in my Jane's tongue to
my ear (I am glad it is not naturally a silent one): all the sunshine I
can feel is in her presence."

The water stood in my eyes to hear this avowal of his dependence; just as
if a royal eagle, chained to a perch, should be forced to entreat a
sparrow to become its purveyor.  But I would not be lachrymose: I dashed
off the salt drops, and busied myself with preparing breakfast.

Most of the morning was spent in the open air.  I led him out of the wet
and wild wood into some cheerful fields: I described to him how
brilliantly green they were; how the flowers and hedges looked refreshed;
how sparklingly blue was the sky.  I sought a seat for him in a hidden
and lovely spot, a dry stump of a tree; nor did I refuse to let him, when
seated, place me on his knee.  Why should I, when both he and I were
happier near than apart?  Pilot lay beside us: all was quiet.  He broke
out suddenly while clasping me in his arms--

"Cruel, cruel deserter!  Oh, Jane, what did I feel when I discovered you
had fled from Thornfield, and when I could nowhere find you; and, after
examining your apartment, ascertained that you had taken no money, nor
anything which could serve as an equivalent!  A pearl necklace I had
given you lay untouched in its little casket; your trunks were left
corded and locked as they had been prepared for the bridal tour.  What
could my darling do, I asked, left destitute and penniless?  And what did
she do?  Let me hear now."

Thus urged, I began the narrative of my experience for the last year.  I
softened considerably what related to the three days of wandering and
starvation, because to have told him all would have been to inflict
unnecessary pain: the little I did say lacerated his faithful heart
deeper than I wished.

I should not have left him thus, he said, without any means of making my
way: I should have told him my intention.  I should have confided in him:
he would never have forced me to be his mistress.  Violent as he had
seemed in his despair, he, in truth, loved me far too well and too
tenderly to constitute himself my tyrant: he would have given me half his
fortune, without demanding so much as a kiss in return, rather than I
should have flung myself friendless on the wide world.  I had endured, he
was certain, more than I had confessed to him.

"Well, whatever my sufferings had been, they were very short," I
answered: and then I proceeded to tell him how I had been received at
Moor House; how I had obtained the office of schoolmistress, \&c.  The
accession of fortune, the discovery of my relations, followed in due
order.  Of course, St. John Rivers' name came in frequently in the
progress of my tale.  When I had done, that name was immediately taken
up.

"This St. John, then, is your cousin?"

"Yes."

"You have spoken of him often: do you like him?"

"He was a very good man, sir; I could not help liking him."

"A good man.  Does that mean a respectable well-conducted man of fifty?
Or what does it mean?"

"St John was only twenty-nine, sir."

"'\emph{Jeune encore},' as the French say.  Is he a person of low stature,
phlegmatic, and plain.  A person whose goodness consists rather in his
guiltlessness of vice, than in his prowess in virtue."

"He is untiringly active.  Great and exalted deeds are what he lives to
perform."

"But his brain?  That is probably rather soft?  He means well: but you
shrug your shoulders to hear him talk?"

"He talks little, sir: what he does say is ever to the point.  His brain
is first-rate, I should think not impressible, but vigorous."

"Is he an able man, then?"

"Truly able."

"A thoroughly educated man?"

"St. John is an accomplished and profound scholar."

"His manners, I think, you said are not to your taste?--priggish and
parsonic?"

"I never mentioned his manners; but, unless I had a very bad taste, they
must suit it; they are polished, calm, and gentlemanlike."

"His appearance,--I forget what description you gave of his appearance;--a
sort of raw curate, half strangled with his white neckcloth, and stilted
up on his thick-soled high-lows, eh?"

"St. John dresses well.  He is a handsome man: tall, fair, with blue
eyes, and a Grecian profile."

(Aside.)  "Damn him!"--(To me.)  "Did you like him, Jane?"

"Yes, Mr. Rochester, I liked him: but you asked me that before."

I perceived, of course, the drift of my interlocutor.  Jealousy had got
hold of him: she stung him; but the sting was salutary: it gave him
respite from the gnawing fang of melancholy.  I would not, therefore,
immediately charm the snake.

"Perhaps you would rather not sit any longer on my knee, Miss Eyre?" was
the next somewhat unexpected observation.

"Why not, Mr. Rochester?"

"The picture you have just drawn is suggestive of a rather too
overwhelming contrast.  Your words have delineated very prettily a
graceful Apollo: he is present to your imagination,--tall, fair, blue-
eyed, and with a Grecian profile.  Your eyes dwell on a Vulcan,--a real
blacksmith, brown, broad-shouldered: and blind and lame into the
bargain."

"I never thought of it, before; but you certainly are rather like Vulcan,
sir."

"Well, you can leave me, ma'am: but before you go" (and he retained me by
a firmer grasp than ever), "you will be pleased just to answer me a
question or two."  He paused.

"What questions, Mr. Rochester?"

Then followed this cross-examination.

"St. John made you schoolmistress of Morton before he knew you were his
cousin?"

"Yes."

"You would often see him?  He would visit the school sometimes?"

"Daily."

"He would approve of your plans, Jane?  I know they would be clever, for
you are a talented creature!"

"He approved of them--yes."

"He would discover many things in you he could not have expected to find?
Some of your accomplishments are not ordinary."

"I don't know about that."

"You had a little cottage near the school, you say: did he ever come
there to see you?"

"Now and then?"

"Of an evening?"

"Once or twice."

A pause.

"How long did you reside with him and his sisters after the cousinship
was discovered?"

"Five months."

"Did Rivers spend much time with the ladies of his family?"

"Yes; the back parlour was both his study and ours: he sat near the
window, and we by the table."

"Did he study much?"

"A good deal."

"What?"

"Hindostanee."

"And what did you do meantime?"

"I learnt German, at first."

"Did he teach you?"

"He did not understand German."

"Did he teach you nothing?"

"A little Hindostanee."

"Rivers taught you Hindostanee?"

"Yes, sir."

"And his sisters also?"

"No."

"Only you?"

"Only me."

"Did you ask to learn?"

"No."

"He wished to teach you?"

"Yes."

A second pause.

"Why did he wish it?  Of what use could Hindostanee be to you?"

"He intended me to go with him to India."

"Ah! here I reach the root of the matter.  He wanted you to marry him?"

"He asked me to marry him."

"That is a fiction--an impudent invention to vex me."

"I beg your pardon, it is the literal truth: he asked me more than once,
and was as stiff about urging his point as ever you could be."

"Miss Eyre, I repeat it, you can leave me.  How often am I to say the
same thing?  Why do you remain pertinaciously perched on my knee, when I
have given you notice to quit?"

"Because I am comfortable there."

"No, Jane, you are not comfortable there, because your heart is not with
me: it is with this cousin--this St. John.  Oh, till this moment, I
thought my little Jane was all mine!  I had a belief she loved me even
when she left me: that was an atom of sweet in much bitter.  Long as we
have been parted, hot tears as I have wept over our separation, I never
thought that while I was mourning her, she was loving another!  But it is
useless grieving.  Jane, leave me: go and marry Rivers."

"Shake me off, then, sir,--push me away, for I'll not leave you of my own
accord."

"Jane, I ever like your tone of voice: it still renews hope, it sounds so
truthful.  When I hear it, it carries me back a year.  I forget that you
have formed a new tie.  But I am not a fool--go--"

"Where must I go, sir?"

"Your own way--with the husband you have chosen."

"Who is that?"

"You know--this St. John Rivers."

"He is not my husband, nor ever will be.  He does not love me: I do not
love him.  He loves (as he \emph{can} love, and that is not as you love) a
beautiful young lady called Rosamond.  He wanted to marry me only because
he thought I should make a suitable missionary's wife, which she would
not have done.  He is good and great, but severe; and, for me, cold as an
iceberg.  He is not like you, sir: I am not happy at his side, nor near
him, nor with him.  He has no indulgence for me--no fondness.  He sees
nothing attractive in me; not even youth--only a few useful mental
points.--Then I must leave you, sir, to go to him?"

I shuddered involuntarily, and clung instinctively closer to my blind but
beloved master.  He smiled.

"What, Jane!  Is this true?  Is such really the state of matters between
you and Rivers?"

"Absolutely, sir!  Oh, you need not be jealous!  I wanted to tease you a
little to make you less sad: I thought anger would be better than grief.
But if you wish me to love you, could you but see how much I \emph{do} love
you, you would be proud and content.  All my heart is yours, sir: it
belongs to you; and with you it would remain, were fate to exile the rest
of me from your presence for ever."

Again, as he kissed me, painful thoughts darkened his aspect.

"My seared vision!  My crippled strength!" he murmured regretfully.

I caressed, in order to soothe him.  I knew of what he was thinking, and
wanted to speak for him, but dared not.  As he turned aside his face a
minute, I saw a tear slide from under the sealed eyelid, and trickle down
the manly cheek.  My heart swelled.

"I am no better than the old lightning-struck chestnut-tree in Thornfield
orchard," he remarked ere long.  "And what right would that ruin have to
bid a budding woodbine cover its decay with freshness?"

"You are no ruin, sir--no lightning-struck tree: you are green and
vigorous.  Plants will grow about your roots, whether you ask them or
not, because they take delight in your bountiful shadow; and as they grow
they will lean towards you, and wind round you, because your strength
offers them so safe a prop."

Again he smiled: I gave him comfort.

"You speak of friends, Jane?" he asked.

"Yes, of friends," I answered rather hesitatingly: for I knew I meant
more than friends, but could not tell what other word to employ.  He
helped me.

"Ah!  Jane.  But I want a wife."

"Do you, sir?"

"Yes: is it news to you?"

"Of course: you said nothing about it before."

"Is it unwelcome news?"

"That depends on circumstances, sir--on your choice."

"Which you shall make for me, Jane.  I will abide by your decision."

"Choose then, sir--\emph{her who loves you best}."

"I will at least choose--\emph{her I love best}.  Jane, will you marry me?"

"Yes, sir."

"A poor blind man, whom you will have to lead about by the hand?"

"Yes, sir."

"A crippled man, twenty years older than you, whom you will have to wait
on?"

"Yes, sir."

"Truly, Jane?"

"Most truly, sir."

"Oh! my darling!  God bless you and reward you!"

"Mr. Rochester, if ever I did a good deed in my life--if ever I thought a
good thought--if ever I prayed a sincere and blameless prayer--if ever I
wished a righteous wish,--I am rewarded now.  To be your wife is, for me,
to be as happy as I can be on earth."

"Because you delight in sacrifice."

"Sacrifice!  What do I sacrifice?  Famine for food, expectation for
content.  To be privileged to put my arms round what I value--to press my
lips to what I love--to repose on what I trust: is that to make a
sacrifice?  If so, then certainly I delight in sacrifice."

"And to bear with my infirmities, Jane: to overlook my deficiencies."

"Which are none, sir, to me.  I love you better now, when I can really be
useful to you, than I did in your state of proud independence, when you
disdained every part but that of the giver and protector."

"Hitherto I have hated to be helped--to be led: henceforth, I feel I
shall hate it no more.  I did not like to put my hand into a hireling's,
but it is pleasant to feel it circled by Jane's little fingers.  I
preferred utter loneliness to the constant attendance of servants; but
Jane's soft ministry will be a perpetual joy.  Jane suits me: do I suit
her?"

"To the finest fibre of my nature, sir."

"The case being so, we have nothing in the world to wait for: we must be
married instantly."

He looked and spoke with eagerness: his old impetuosity was rising.

"We must become one flesh without any delay, Jane: there is but the
licence to get--then we marry."

"Mr. Rochester, I have just discovered the sun is far declined from its
meridian, and Pilot is actually gone home to his dinner.  Let me look at
your watch."

"Fasten it into your girdle, Janet, and keep it henceforward: I have no
use for it."

"It is nearly four o'clock in the afternoon, sir.  Don't you feel
hungry?"

"The third day from this must be our wedding-day, Jane.  Never mind fine
clothes and jewels, now: all that is not worth a fillip."

"The sun has dried up all the rain-drops, sir.  The breeze is still: it
is quite hot."

"Do you know, Jane, I have your little pearl necklace at this moment
fastened round my bronze scrag under my cravat?  I have worn it since the
day I lost my only treasure, as a memento of her."

"We will go home through the wood: that will be the shadiest way."

He pursued his own thoughts without heeding me.

"Jane! you think me, I daresay, an irreligious dog: but my heart swells
with gratitude to the beneficent God of this earth just now.  He sees not
as man sees, but far clearer: judges not as man judges, but far more
wisely.  I did wrong: I would have sullied my innocent flower--breathed
guilt on its purity: the Omnipotent snatched it from me.  I, in my stiff-
necked rebellion, almost cursed the dispensation: instead of bending to
the decree, I defied it.  Divine justice pursued its course; disasters
came thick on me: I was forced to pass through the valley of the shadow
of death.  \emph{His} chastisements are mighty; and one smote me which has
humbled me for ever.  You know I was proud of my strength: but what is it
now, when I must give it over to foreign guidance, as a child does its
weakness?  Of late, Jane--only--only of late--I began to see and
acknowledge the hand of God in my doom.  I began to experience remorse,
repentance; the wish for reconcilement to my Maker.  I began sometimes to
pray: very brief prayers they were, but very sincere.

"Some days since: nay, I can number them--four; it was last Monday night,
a singular mood came over me: one in which grief replaced frenzy--sorrow,
sullenness.  I had long had the impression that since I could nowhere
find you, you must be dead.  Late that night--perhaps it might be between
eleven and twelve o'clock--ere I retired to my dreary rest, I supplicated
God, that, if it seemed good to Him, I might soon be taken from this
life, and admitted to that world to come, where there was still hope of
rejoining Jane.

"I was in my own room, and sitting by the window, which was open: it
soothed me to feel the balmy night-air; though I could see no stars and
only by a vague, luminous haze, knew the presence of a moon.  I longed
for thee, Janet!  Oh, I longed for thee both with soul and flesh!  I
asked of God, at once in anguish and humility, if I had not been long
enough desolate, afflicted, tormented; and might not soon taste bliss and
peace once more.  That I merited all I endured, I acknowledged--that I
could scarcely endure more, I pleaded; and the alpha and omega of my
heart's wishes broke involuntarily from my lips in the words--'Jane!
Jane!  Jane!'"

"Did you speak these words aloud?"

"I did, Jane.  If any listener had heard me, he would have thought me
mad: I pronounced them with such frantic energy."

"And it was last Monday night, somewhere near midnight?"

"Yes; but the time is of no consequence: what followed is the strange
point.  You will think me superstitious,--some superstition I have in my
blood, and always had: nevertheless, this is true--true at least it is
that I heard what I now relate.

"As I exclaimed 'Jane!  Jane!  Jane!' a voice--I cannot tell whence the
voice came, but I know whose voice it was--replied, 'I am coming: wait
for me;' and a moment after, went whispering on the wind the words--'Where
are you?'

"I'll tell you, if I can, the idea, the picture these words opened to my
mind: yet it is difficult to express what I want to express.  Ferndean is
buried, as you see, in a heavy wood, where sound falls dull, and dies
unreverberating.  'Where are you?' seemed spoken amongst mountains; for I
heard a hill-sent echo repeat the words.  Cooler and fresher at the
moment the gale seemed to visit my brow: I could have deemed that in some
wild, lone scene, I and Jane were meeting.  In spirit, I believe we must
have met.  You no doubt were, at that hour, in unconscious sleep, Jane:
perhaps your soul wandered from its cell to comfort mine; for those were
your accents--as certain as I live--they were yours!"

Reader, it was on Monday night--near midnight--that I too had received
the mysterious summons: those were the very words by which I replied to
it.  I listened to Mr. Rochester's narrative, but made no disclosure in
return.  The coincidence struck me as too awful and inexplicable to be
communicated or discussed.  If I told anything, my tale would be such as
must necessarily make a profound impression on the mind of my hearer: and
that mind, yet from its sufferings too prone to gloom, needed not the
deeper shade of the supernatural.  I kept these things then, and pondered
them in my heart.

"You cannot now wonder," continued my master, "that when you rose upon me
so unexpectedly last night, I had difficulty in believing you any other
than a mere voice and vision, something that would melt to silence and
annihilation, as the midnight whisper and mountain echo had melted
before.  Now, I thank God!  I know it to be otherwise.  Yes, I thank
God!"

He put me off his knee, rose, and reverently lifting his hat from his
brow, and bending his sightless eyes to the earth, he stood in mute
devotion.  Only the last words of the worship were audible.

"I thank my Maker, that, in the midst of judgment, he has remembered
mercy.  I humbly entreat my Redeemer to give me strength to lead
henceforth a purer life than I have done hitherto!"

Then he stretched his hand out to be led.  I took that dear hand, held it
a moment to my lips, then let it pass round my shoulder: being so much
lower of stature than he, I served both for his prop and guide.  We
entered the wood, and wended homeward.




\chapter{CONCLUSION}


\lettrine{R}{eader}, I married him.  A quiet wedding we had: he and I, the parson and
clerk, were alone present.  When we got back from church, I went into the
kitchen of the manor-house, where Mary was cooking the dinner and John
cleaning the knives, and I said--

"Mary, I have been married to Mr. Rochester this morning."  The
housekeeper and her husband were both of that decent phlegmatic order of
people, to whom one may at any time safely communicate a remarkable piece
of news without incurring the danger of having one's ears pierced by some
shrill ejaculation, and subsequently stunned by a torrent of wordy
wonderment.  Mary did look up, and she did stare at me: the ladle with
which she was basting a pair of chickens roasting at the fire, did for
some three minutes hang suspended in air; and for the same space of time
John's knives also had rest from the polishing process: but Mary, bending
again over the roast, said only--

"Have you, Miss?  Well, for sure!"

A short time after she pursued--"I seed you go out with the master, but I
didn't know you were gone to church to be wed;" and she basted away.
John, when I turned to him, was grinning from ear to ear.

"I telled Mary how it would be," he said: "I knew what Mr. Edward" (John
was an old servant, and had known his master when he was the cadet of the
house, therefore, he often gave him his Christian name)--"I knew what Mr.
Edward would do; and I was certain he would not wait long neither: and
he's done right, for aught I know.  I wish you joy, Miss!" and he
politely pulled his forelock.

"Thank you, John.  Mr. Rochester told me to give you and Mary this."  I
put into his hand a five-pound note.  Without waiting to hear more, I
left the kitchen.  In passing the door of that sanctum some time after, I
caught the words--

"She'll happen do better for him nor ony o't' grand ladies."  And again,
"If she ben't one o' th' handsomest, she's noan faal and varry
good-natured; and i' his een she's fair beautiful, onybody may see that."

I wrote to Moor House and to Cambridge immediately, to say what I had
done: fully explaining also why I had thus acted.  Diana and Mary
approved the step unreservedly.  Diana announced that she would just give
me time to get over the honeymoon, and then she would come and see me.

"She had better not wait till then, Jane," said Mr. Rochester, when I
read her letter to him; "if she does, she will be too late, for our
honeymoon will shine our life long: its beams will only fade over your
grave or mine."

How St. John received the news, I don't know: he never answered the
letter in which I communicated it: yet six months after he wrote to me,
without, however, mentioning Mr. Rochester's name or alluding to my
marriage.  His letter was then calm, and, though very serious, kind.  He
has maintained a regular, though not frequent, correspondence ever since:
he hopes I am happy, and trusts I am not of those who live without God in
the world, and only mind earthly things.

You have not quite forgotten little Adele, have you, reader?  I had not;
I soon asked and obtained leave of Mr. Rochester, to go and see her at
the school where he had placed her.  Her frantic joy at beholding me
again moved me much.  She looked pale and thin: she said she was not
happy.  I found the rules of the establishment were too strict, its
course of study too severe for a child of her age: I took her home with
me.  I meant to become her governess once more, but I soon found this
impracticable; my time and cares were now required by another--my husband
needed them all.  So I sought out a school conducted on a more indulgent
system, and near enough to permit of my visiting her often, and bringing
her home sometimes.  I took care she should never want for anything that
could contribute to her comfort: she soon settled in her new abode,
became very happy there, and made fair progress in her studies.  As she
grew up, a sound English education corrected in a great measure her
French defects; and when she left school, I found in her a pleasing and
obliging companion: docile, good-tempered, and well-principled.  By her
grateful attention to me and mine, she has long since well repaid any
little kindness I ever had it in my power to offer her.

My tale draws to its close: one word respecting my experience of married
life, and one brief glance at the fortunes of those whose names have most
frequently recurred in this narrative, and I have done.

I have now been married ten years.  I know what it is to live entirely
for and with what I love best on earth.  I hold myself supremely
blest--blest beyond what language can express; because I am my husband's
life as fully as he is mine.  No woman was ever nearer to her mate than I
am: ever more absolutely bone of his bone and flesh of his flesh.  I know
no weariness of my Edward's society: he knows none of mine, any more than
we each do of the pulsation of the heart that beats in our separate
bosoms; consequently, we are ever together.  To be together is for us to
be at once as free as in solitude, as gay as in company.  We talk, I
believe, all day long: to talk to each other is but a more animated and
an audible thinking.  All my confidence is bestowed on him, all his
confidence is devoted to me; we are precisely suited in character--perfect
concord is the result.

Mr. Rochester continued blind the first two years of our union; perhaps
it was that circumstance that drew us so very near--that knit us so very
close: for I was then his vision, as I am still his right hand.
Literally, I was (what he often called me) the apple of his eye.  He saw
nature--he saw books through me; and never did I weary of gazing for his
behalf, and of putting into words the effect of field, tree, town, river,
cloud, sunbeam--of the landscape before us; of the weather round us--and
impressing by sound on his ear what light could no longer stamp on his
eye.  Never did I weary of reading to him; never did I weary of
conducting him where he wished to go: of doing for him what he wished to
be done.  And there was a pleasure in my services, most full, most
exquisite, even though sad--because he claimed these services without
painful shame or damping humiliation.  He loved me so truly, that he knew
no reluctance in profiting by my attendance: he felt I loved him so
fondly, that to yield that attendance was to indulge my sweetest wishes.

One morning at the end of the two years, as I was writing a letter to his
dictation, he came and bent over me, and said--"Jane, have you a
glittering ornament round your neck?"

I had a gold watch-chain: I answered "Yes."

"And have you a pale blue dress on?"

{And have you a pale blue dress on?: p435.jpg}

I had.  He informed me then, that for some time he had fancied the
obscurity clouding one eye was becoming less dense; and that now he was
sure of it.

He and I went up to London.  He had the advice of an eminent oculist; and
he eventually recovered the sight of that one eye.  He cannot now see
very distinctly: he cannot read or write much; but he can find his way
without being led by the hand: the sky is no longer a blank to him--the
earth no longer a void.  When his first-born was put into his arms, he
could see that the boy had inherited his own eyes, as they once
were--large, brilliant, and black.  On that occasion, he again, with a
full heart, acknowledged that God had tempered judgment with mercy.

My Edward and I, then, are happy: and the more so, because those we most
love are happy likewise.  Diana and Mary Rivers are both married:
alternately, once every year, they come to see us, and we go to see them.
Diana's husband is a captain in the navy, a gallant officer and a good
man.  Mary's is a clergyman, a college friend of her brother's, and, from
his attainments and principles, worthy of the connection.  Both Captain
Fitzjames and Mr. Wharton love their wives, and are loved by them.

As to St. John Rivers, he left England: he went to India.  He entered on
the path he had marked for himself; he pursues it still.  A more
resolute, indefatigable pioneer never wrought amidst rocks and dangers.
Firm, faithful, and devoted, full of energy, and zeal, and truth, he
labours for his race; he clears their painful way to improvement; he hews
down like a giant the prejudices of creed and caste that encumber it.  He
may be stern; he may be exacting; he may be ambitious yet; but his is the
sternness of the warrior Greatheart, who guards his pilgrim convoy from
the onslaught of Apollyon.  His is the exaction of the apostle, who
speaks but for Christ, when he says--"Whosoever will come after me, let
him deny himself, and take up his cross and follow me."  His is the
ambition of the high master-spirit, which aims to fill a place in the
first rank of those who are redeemed from the earth--who stand without
fault before the throne of God, who share the last mighty victories of
the Lamb, who are called, and chosen, and faithful.

St. John is unmarried: he never will marry now.  Himself has hitherto
sufficed to the toil, and the toil draws near its close: his glorious sun
hastens to its setting.  The last letter I received from him drew from my
eyes human tears, and yet filled my heart with divine joy: he anticipated
his sure reward, his incorruptible crown.  I know that a stranger's hand
will write to me next, to say that the good and faithful servant has been
called at length into the joy of his Lord.  And why weep for this?  No
fear of death will darken St. John's last hour: his mind will be
unclouded, his heart will be undaunted, his hope will be sure, his faith
steadfast.  His own words are a pledge of this--

"My Master," he says, "has forewarned me.  Daily He announces more
distinctly,--'Surely I come quickly!' and hourly I more eagerly
respond,--'Amen; even so come, Lord Jesus!'"


\end{document}

